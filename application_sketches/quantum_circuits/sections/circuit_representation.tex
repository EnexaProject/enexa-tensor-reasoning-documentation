\section{Quantum Computation Basics}

\subsection{State Encoding Schemes}

Basis Encoding in Quantum Computation refers to the representation of classical $n$ bit strings by $n$ qubit basis states, and is called one-hot encoding in \tnreason{}.
The Basis Encoding scheme in \tnreason{} goes beyond this scheme and also encodes subsets by sums of one-hot encodings to the members of the set.
In this way, relations and functions are represented by boolean tensors and contraction of them is refered as \BasisCalculus{}.

Amplitude Encoding in Quantum Computation refers to the storage of complex numbers in the amplitudes of quantum states.
The pendant in \tnreason{} is the Coordinate Encoding scheme, where real numbers are stored in the coordinates of real-valued tensors.
Compared to Amplitude Encoding, Coordinate Encoding does not have the normalization constraint of quantum states.
The Amplitude Encoding of the square root of a probability distribution is sometimes called q-sample.

\subsection{Controlled Single Qubit Gates}

We define the rotation gate around the Y-axis by an angle $\alpha$ as
\begin{align*}
    \yrotationofat{\alpha}{\avariableof{\insymbol},\avariableof{\outsymbol}} \coloneqq
    \begin{bmatrix}
        \cosof{\frac{\alpha}{2}} & -\sinof{\frac{\alpha}{2}} \\
        \sinof{\frac{\alpha}{2}} & \cosof{\frac{\alpha}{2}}
    \end{bmatrix}
\end{align*}

Further we define the Pauli-X:
\begin{align*}
    \paulixat{\avariableof{\insymbol},\avariableof{\outsymbol}} \coloneqq
    \begin{bmatrix}
        0 & 1 \\
        1 & 0
    \end{bmatrix}
\end{align*}

%% Control notation
Controlled single qubit gates are defined using control qubits, where the gate is applied to the target qubit if the control qubits are in a specific state and the identity is applied otherwise.
In the tensor network diagrams, we do not distinguish between incoming and outgoing control qubit variables, since the control acts as a Dirac tensor.
Thus, controlled unitary with target qubit $\catvariableof{t}$ and control qubits $\catvariableof{c}$ are represented by tensors
\begin{align*}
    \contunitaryat{\catvariableof{t,\insymbol},\catvariableof{t,\outsymbol},\catvariableof{c}}
\end{align*}
where for each state $\catindexof{c}$ to the control variables we have that
\begin{align*}
    \contunitaryat{\catvariableof{t,\insymbol},\catvariableof{t,\outsymbol},\indexedcatvariableof{c}}
\end{align*}
is a unitary matrix acting on the leg space of the target variable.

\subsection{Measurement and Phases}

The computational basis measurement of the qubits $\catvariableof{\variableset}$ of a Quantum State $\qstatewith$ is equal to drawing samples from a distribution
\begin{align*}
    \probat{\catvariableof{\variableset}} = \contractionof{
        \qstatewith, \comconqstatewith
    }{\catvariableof{\variableset}} \, .
\end{align*}
Here $\comconqstatewith$ is the complex conjugate of $\qstatewith$.
When $\qstate$ is prepared by a quantum circuit acting on a initial state, the complex conjugate is the hermitean conjugate of the circuit acting on the complex conjgate of the initial state.

%% Add drawing!
We abbreviate these contractions by extending the contraction diagrams with measurement symbols (see \figref{fig:measurementSketch}).
\begin{figure}
    \begin{center}
        \begin{tikzpicture}[scale=0.35,thick]

    \draw (0,0) rectangle (2,10);
    \node[anchor=center] (text) at (1,5) {$\qstate$};

    \draw (2,1) -- (4,1);

    \draw (2,9) -- (4,9) node[midway,above] {\colorlabelsize $\catvariableof{0}$};
    \drawqcmeasuresymbol{5}{9}
    \node[anchor=center] (text) at (3,8.25) {$\vdots$};
    \draw (2,6) -- (4,6) node[midway,above] {\colorlabelsize $\catvariableof{\atomenumerator}$};
    \drawqcmeasuresymbol{5}{6}

    \draw (2,4) -- (4,4) node[midway,above] {\colorlabelsize $\catvariableof{\atomenumerator+1}$};
    \node[anchor=center] (text) at (3,3.25) {$\vdots$};
    \draw (2,1) -- (4,1) node[midway,above] {\colorlabelsize $\catvariableof{\atomorder-1}$};

    \node[anchor=center] (text) at (8,5) {${=}$};

    \begin{scope}
        [shift={(10,0)}]
        \draw (0,0) rectangle (2,10);
        \node[anchor=center] (text) at (1,5) {$\qstate$};

        \draw (2,1) -- (4,1);

        \draw (2,9) -- (4,9) node[midway,above] {\colorlabelsize $\catvariableof{0}$};
        \node[anchor=center] (text) at (3,8.25) {$\vdots$};
        \draw (2,6) -- (4,6) node[midway,above] {\colorlabelsize $\catvariableof{\atomenumerator}$};

        \drawvariabledot{4}{9}
        \draw (4,9) -- (4,11);

        \drawvariabledot{6}{6}
        \draw (6,6) -- (6,11);

        \node[anchor=center] (text) at (5,10.25) {$\cdots$};

        \draw (4,9) -- (6,9);
        \draw (4,6) -- (6,6);
        \draw (4,4) -- (6,4);
        \draw (4,1) -- (6,1);

        \draw (2,4) -- (4,4) node[midway,above] {\colorlabelsize $\catvariableof{\atomenumerator+1}$};
        \node[anchor=center] (text) at (3,3.25) {$\vdots$};
        \draw (2,1) -- (4,1) node[midway,above] {\colorlabelsize $\catvariableof{\atomorder-1}$};


        \begin{scope}
            [xscale = -1, shift={(-10,0)}]
            \draw (0,0) rectangle (2,10);
            \node[anchor=center] (text) at (1,5) {$\comconqstate$};

            \draw (2,1) -- (4,1);

            \draw (2,9) -- (4,9) node[midway,above] {\colorlabelsize $\catvariableof{0}$};

            \node[anchor=center] (text) at (3,8.25) {$\vdots$};
            \draw (2,6) -- (4,6) node[midway,above] {\colorlabelsize $\catvariableof{\atomenumerator}$};


            \draw (2,4) -- (4,4) node[midway,above] {\colorlabelsize $\catvariableof{\atomenumerator+1}$};
            \node[anchor=center] (text) at (3,3.25) {$\vdots$};
            \draw (2,1) -- (4,1) node[midway,above] {\colorlabelsize $\catvariableof{\atomorder-1}$};
        \end{scope}

    \end{scope}

    \node[anchor=center] (text) at (23,5) {${=}$};

    \begin{scope}
        [shift={(25,0)}]
        \draw (0,0) rectangle (2,10);
        \node[anchor=center] (text) at (1,5) {$\absof{\qstate}^2$};

        \draw (2,1) -- (4,1);

        \draw (2,9) -- (4,9) node[midway,above] {\colorlabelsize $\catvariableof{0}$};
        \node[anchor=center] (text) at (3,8.25) {$\vdots$};
        \draw (2,6) -- (4,6) node[midway,above] {\colorlabelsize $\catvariableof{\atomenumerator}$};

        \draw (2,4) -- (4,4) node[midway,above] {\colorlabelsize $\catvariableof{\atomenumerator+1}$};
        \draw (4,3) rectangle (6,5);
        \node[anchor=center] (text) at (5,4) {$\ones$};
        \node[anchor=center] (text) at (3,3.25) {$\vdots$};
        \draw (2,1) -- (4,1) node[midway,above] {\colorlabelsize $\catvariableof{\atomorder-1}$};
        \draw (4,0) rectangle (6,2);
        \node[anchor=center] (text) at (5,1) {$\ones$};
    \end{scope}

\end{tikzpicture}
    \end{center}
    \caption{Computational Basis Measurement of a quantum state $\qstate$.
    The measurement symbols on the left side indicate the measured qubits and the first equation is understood as a definition.
    In the second equation we sketch, that the measurement distribution is equal to the contraction of the square absolute transform of $\qstate$ to the measured variables.
    }\label{fig:measurementSketch}
\end{figure}

%% Phase-Absolut decomposition
Each complex-valued tensor $\qstatewith$ has a decomposition into a phase tensor $\phasecorewith$ and an absolute tensor $\absof{\qstate}[\shortcatvariables]$ defined by
\begin{align*}
    \qstatewith = \contractionof{\expof{i\cdot\phasecorewith}, \absof{\qstate}[\shortcatvariables]}{\shortcatvariables}\, .
\end{align*}

The measurement distribution is depends only on $\absof{\phi}$, that is
\begin{align*}
    \probwith = \absof{\qstate}^2[\shortcatvariables] \, .
\end{align*}

Note, that when only a subset of variables is measured, the distribution is the contraction of the absolute square transform (these operations do not commute)
\begin{align*}
    \probat{\catvariableof{\variableset}} = \contractionof{\absof{\qstate}^2[\shortcatvariables]}{\catvariableof{\variableset}} \, .
\end{align*}

%% Phasecore vanishing as gauging
When we are interested in the preparation of quantum states with a specific computational basis measurement distribution, we can restrict to states with vanishing phase cores, that is
\begin{align*}
    \qstatewith
    = \contractionof{\expof{i\cdot\zerosat{\shortcatvariables}}, \absof{\qstate}[\shortcatvariables]}{\shortcatvariables}
    = \absof{\qstate}[\shortcatvariables]  \, .
\end{align*}