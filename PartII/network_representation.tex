\chapter{\chatextnetworkRepresentation}\label{cha:networkRepresentation}

\HybridLogicNetworks{} are graphical models with an interpretation by propositional logics.
Since they unify probabilistic and logical reasoning, we call them hybrid.
In this chapter we investigate their representation based on tensor networks, while the reasoning mechanisms are discussed in \charef{cha:networkReasoning}.
We first distinguish between \MarkovLogicNetworks{}, which are an approach to soft logics in the framework of exponential families, and \HardLogicNetworks{}, which correspond with propositional knowledge bases.
Then we exploit non-trivial boolean base measures to unify both approaches by \HybridLogicNetworks{}, which are itself in exponential families.
We then investigate the corresponding mean parameter polytopes of \HybridLogicNetworks{} and show expressivity results.

\sect{\MarkovLogicNetworks{}}

\MarkovLogicNetworks{} exploit the efficiency and interpretability of logical calculus as well as the expressivity of graphical models.

\subsect{\MarkovLogicNetworks{} as Exponential Families}

We introduce \MarkovLogicNetworks{} in the formalism of exponential families (see \secref{sec:exponentialFamilies}).

\begin{definition}[\MarkovLogicNetwork]
    \label{def:mln}
    \MarkovLogicNetworks{} are exponential families $\mlnexpfamily$ with sufficient statistics by functions
    \begin{align*}
        \hlnstat \defcols\atomstates \rightarrow \bigtimes_{\exformulain}[2] \subset \rr^{\cardof{\formulaset}}
    \end{align*}
    defined coordinatewise by propositional formulas $\exformulain$.
\end{definition}

% Characterization of MLNs among exponential families\defcolsWhen choosing binary features
Since the image of each coordinate $\sstatcoordinateof{\selindex}$ is contained in $\ozset$, each is a propositional formulas (see \defref{def:formulas}).
%Conversely, any boolean feature $\sstatcoordinateof{\selindex}$ of an exponential family defines a propositional formula (see \defref{def:formulas}).
Thus, any exponential family of distributions of $\atomstates$, such that $\imageof{\sstatcoordinateof{\selindex}}\subset\ozset$ for all $\selindexin$ is a set of \MarkovLogicNetworks{} with fixed formulas.

\begin{remark}[Alternative Definitions]
    We here defined \MarkovLogicNetworks{} on propositional logic, while originally they are defined in the more expressive first-order logics \cite{richardson_markov_2006}.
    The relation of both frameworks will be discussed further in \charef{cha:folModels}.
\end{remark}



\subsect{Tensor Network Representation}

Based on the previous discussion on the representation of exponential families by tensor networks in \secref{sec:exponentialFamilies} we now derive a representation for \MarkovLogicNetworks{}.

\subsubsect{Basis encodings for distributions}

\begin{theorem}[Basis encodings for \MarkovLogicNetworks{}]
    \label{the:mlnTensorRep}
    A \MarkovLogicNetwork{} to a set of formulas $\formulaset = \{\enumformula \wcols \selindexin\}$ is represented as
    \begin{align*}
        \mlnprobat{\shortcatvariables} =
        \normalizationof{\{\enumformulaccwith \wcols \selindexin \} \cup \{\softactlegwith \wcols \selindexin \}
        }{\shortcatvariables}
    \end{align*}
    where we denote for each $\selindexin$ an activation core
    \begin{align*}
        \softactleg\left[\indexedheadvariableof{\selindex}\right]
        = \begin{cases}
              1 & \ifspace \headindexof{\selindex} = 0 \\
              \expof{\canparamat{\indexedselvariable}} & \ifspace \headindexof{\selindex}  = 1
        \end{cases}  \, .
    \end{align*}
\end{theorem}
\begin{proof}
    \MarkovLogicNetworks{} are exponential families, which base measure is trivial and which statistic consist of boolean features.
    We apply the tensor network decomposition of more generic exponential families \theref{the:expFamilyTensorRep} to this case and get
    \begin{align*}
        \mlnprobat{\shortcatvariables} =
        \normalizationof{\{\onesat{\shortcatvariables}\}
        \cup \{\bencodingofat{\sstatcoordinateof{\selindex}}{\sstatcatof{\selindex},\shortcatvariables} \wcols \selindexin\}
        \cup\{\softactlegwith \wcols \selindexin\}}{\shortcatvariables} \, .
    \end{align*}
    While the base measure tensor is trivial, it can be ignored in the contraction.
    Since the image of each feature $\enumformula$ is in $[2]$, we choose the index interpretation function by the identity $\indexinterpretation \defcols[2] \rightarrow \ozset$ and get
%	We further choose the standard index interpretation for booleans (see \secref{sec:booleanEncoding}) and get
    \begin{align*}
        \softactleg\left[\indexedheadvariableof{\selindex}\right]
        &= \expof{\canparamat{\indexedselvariable} \cdot \indexinterpretationofat{\selindex}{\headindexof{\selindex}} }
        = \expof{\canparamat{\indexedselvariable} \cdot \headindexof{\selindex}} \\
        &= \begin{cases}
               1 & \ifspace \headindexof{\selindex} = 0 \\
               \expof{\canparamat{\indexedselvariable}} & \ifspace \headindexof{\selindex}  = 1
        \end{cases} \, . \qedhere
    \end{align*}
\end{proof}

\begin{figure}[t]
    \begin{center}
        \begin{tikzpicture}[thick, scale=0.35] % , baseline = -3.5pt

\drawundiratomindices{0}{-4}
\draw (-2,-1) rectangle (6, -3);
\node[anchor=center] (text) at (2,-2) {$\expof{\canparamat{\indexedselvariable}\cdot\enumformula}$};

		
\node[anchor=center] (text) at (10,-2) {${=}$};


\begin{scope}[shift={(15,-2)}]

		\draw (-0.5,3) rectangle (4.5, 5);
		\node[anchor=center] (text) at (2,4) {$\enumformulaac$};

		\drawvariabledot{2}{2.25}
		\draw[] (2,2.25) -- (2,3);
		\draw[->] (2,1) -- (2,2.5) node[midway, left] {\tiny $\headvariableof{\selindex}$};
		
		\draw (-1,1) rectangle (5, -1);
		\node[anchor=center] (text) at (2,0) {$\bencodingof{\enumformula}$};

		\drawatomindices{0}{-2}

\end{scope}

\end{tikzpicture}
    \end{center}
    \caption{Factor of a \MarkovLogicNetwork{} to a formula $\enumformula$, represented as the contraction of a computation core $\enumformulacc$ and an activation core $\softactleg$.
    While the computation core $\enumformulacc$ prepares based on basis calculus a categorical variable representing the value of the statistic formula $\enumformula$ dependent on assignments to the distributed variables, the activation core multiplies an exponential weight to coordinates satisfying the formula.
    }
    \label{fig:mlnFactor}
\end{figure}

% Graphical model representation
\theref{the:mlnTensorRep} provides a decomposition of \MarkovLogicNetworks{} by a tensor network of computation cores $\bencodingof{\enumformula}$ and accompanying activation cores $\softactleg$.
Since the head variable $\headvariableof{\selindex}$ appears exclusively in these pairs, we can contract each computation core with the corresponding activation core to get a factor, see \figref{fig:mlnFactor}.
With this we get the decomposition
\begin{align*}
    \mlnprobat{\shortcatvariables}
    = \normalizationof{\{\expof{\canparamat{\indexedselvariable}\cdot\enumformulaat{\shortcatvariables}} \wcols \selindexin\}}{\shortcatvariables} \, .
\end{align*}
More precisely, this transformation of the decomposition holds by \theref{the:splittingContractions} to be shown in \charef{cha:messagePassing}, stating that the contraction of computation and activation cores can be performed before the global contraction of the result.

% Sparsification by trivial variables
While in the decomposition of \theref{the:mlnTensorRep} the basis encodings of the features carry all distributed variables $\shortcatvariables$, we now seek towards sparser decompositions.
To each $\selindexin$ we denote by $\nodesof{\selindex}$ the maximal subset of $[\catorder]$ such that there is a reduced function
$\tilde{\formula}_{\selindex} \defcols\bigtimes_{\catenumerator\in\nodesof{\selindex}}[\catdimof{\catenumerator}] \rightarrow [2]$
with
\begin{align*}
    \enumformulaat{\shortcatvariables}
    = \contractionof{\tilde{\formula}_{\selindex}[\catvariableof{\nodesof{\selindex}}]}{\shortcatvariables} \, .
\end{align*}
We often account for such situations of sparse formulas, when $\enumformula$ has a syntactical decomposition involving only the atomic variables $\nodesof{\selindex}$.
As a consequence we have
\begin{align*}
    \bencodingofat{\enumformula}{\shortcatvariables}
    = \bencodingofat{\tilde{\formula}_{\selindex}}{\catvariableof{\nodesof{\selindex}}}
\end{align*}
and
\begin{align*}
    \mlnprobat{\shortcatvariables}
    = \normalizationof{\{\expof{\canparamat{\indexedselvariable}\cdot\tilde{\formula}_{\selindex}[\catvariableof{\nodesof{\selindex}}]} \wcols \selindexin\}}{\shortcatvariables} \, .
\end{align*}
Thus, any \MarkovLogicNetwork{} has a sparse representation by a markov network on the graph
\begin{align*}
    \graphof{\formulaset} = ([\catorder],\{\nodesof{\selindex} \wcols \selindexin\}) \, .
\end{align*}
This sparsity inducing mechanism is analogous to the decomposition of probability distributions based on conditional independence assumptions, when understanding each formula in the \MarkovLogicNetwork{} as an introduced dependency among the affected variables $\nodesof{\selindex}$.


\begin{figure}[t]
    \begin{center}
        \begin{tikzpicture}[scale=0.35, thick, yscale=-1] % , baseline = -3.5pt

\draw (-5,-3) rectangle (1, -5);
\node[anchor=center] (text) at (-2,-4) {$\expof{\canparamat{0}\cdot\formulaof{0}}$};

\draw (3,-3) rectangle (9, -5);
\node[anchor=center] (text) at (6,-4) {$\expof{\canparamat{1}\cdot\formulaof{1}}$};


\draw[->] (0,1)--(0,-1)  node[midway,left] {\tiny $\catvariableof{a}$}; 
\drawvariabledot{0}{-1}
\draw[->] (0,-1) to[bend right=25] (-4,-3);
\draw[->] (0,-1) to[bend left=25] (4,-3);

\draw[->]  (2,1)--(2,-1) node[midway,right] {\tiny $\catvariableof{b}$}; 
\drawvariabledot{2}{-1}
\draw[->] (2,-1) to[bend right=25] (-2,-3);
\draw[->] (2,-1) to[bend left=25] (6,-3);

\draw[->] (4,1)--(4,-1) node[midway,right] {\tiny $\catvariableof{c}$}; 
\drawvariabledot{4}{-1}
\draw[->] (4,-1) to[bend right=25] (0,-3);
\draw[->] (4,-1) to[bend left=25] (8,-3);


\node[anchor=center] (text) at (12.5,-2) {${=}$};


\begin{scope}[shift={(20,0)}]

\draw[->] (0,1)--(0,-1)  node[midway,left] {\tiny $\catvariableof{a}$}; 
\draw[->]  (3,1)--(3,-1) node[midway,right] {\tiny $\catvariableof{b}$}; 
\draw[->] (6,1)--(6,-1) node[midway,right] {\tiny $\catvariableof{c}$}; 

	
\draw (-1,-1) rectangle (4, -3);
\node[anchor=center] (text) at (1.5,-2) {$\rencodingof{\lor}$};

\draw[->] (1.5,-3) --(1.5,-4.5) node[midway,right]{\tiny $\catvariableof{a \lor b}$};
\drawvariabledot{1.5}{-4.5}
\draw[->] (1.5,-4.5) --(1.5,-6) ;

\draw[] (1.5,-4.5) -- (0,-4.5);

\draw[]  (-7, -3.5) rectangle (0, -6.5);
\node[anchor=center,] (text) at (-3.5,-5) {$\begin{bmatrix} 
1 \\
\expof{\canparamat{0}}
\end{bmatrix}$};


\draw (5,-1) rectangle (7, -3);
\node[anchor=center] (text) at (6,-2) {$\rencodingof{\lnot}$};

\draw[->] (6,-3) --(6,-4.5) node[midway,right]{\tiny $\catvariableof{\lnot c}$};
\drawvariabledot{6}{-4.5}
\draw[->] (6,-4.5) --(6,-6);	
	
	
\draw (0.5,-6) rectangle (6.5,-8);
\node[anchor=center] (text) at (3.5,-7) {$\rencodingof{\lor}$};
	
\draw[<-] (4,-9.5) -- (4,-8) node[midway,right] {\tiny $\catvariableof{a \lor b \lor \lnot c}$};

\drawvariabledot{4}{-9.5}
\draw[] (4,-9.5) -- (2.75,-9.5);

\draw[]  (-4.25, -8.5) rectangle (2.75, -11.5);
\node[anchor=center,] (text) at (-0.75,-10) {$\begin{bmatrix} 
1 \\
\expof{\canparamat{1}}
\end{bmatrix}$};



\end{scope}

\end{tikzpicture}
    \end{center}
    \caption{Example of a decomposed Markov Network representation of a \MarkovLogicNetwork{} with formulas $\{\formulaof{0} = a\lor b, \formulaof{1} = a \lor b \lor \lnot c\}$.
    Since both formulas share the subformula $a\lor b$, their contracted factors have a representation by a connected tensor network.}
    \label{fig:mlnDecRep}
\end{figure}


% Sparsity by decomposition
A further sparsity introducing mechanism is through exploiting redundancy in the computation of $\enumformula$, when a decomposition of the feature is known.
For the propositional formulas $\enumformula$ this amounts to a syntactic representation of the formula as a composition of logical connectives is available (see \figref{fig:mlnDecRep}). % Make a precise definition in logicRepresentation and link it!
In this case, we exploit the representation by tensor networks of the basis encodings (shown as \theref{the:compositionByContraction} in \charef{cha:basisCalculus})
Note, that this decomposition scheme introduces further auxiliary variables $\headvariable$ with deterministic dependence on the distributed variables $\shortcatvariables$.
Such variables are often referred to as hidden.

% Shared formulas
We can further exploit common syntactical structure in the formulas $\enumformula\in\formulaset$ to reduce the number of basis encodings of connectives.
This is the case, when the syntax graph of two or more formulas share a subgraph.
In that case, the respective syntax graph needs to be represented only once an can be incorporated into the decomposition of all formulas, which share this subgraph.
For an example see \figref{fig:mlnDecRep}, where the syntactical representation of the formula $\formulaof{0}$ is a subgraph of the syntactical representation of $\formulaof{1}$.


%Since any member of an exponential family is a Markov Network with tensors to each coordinate of the statistic, also \MarkovLogicNetworks{} are Markov Networks.

%\begin{corollary}\label{cor:MLNasMN}
%	Given a set $\formulaset$ of formulas on atomic variables $\catvariableof{\nodes}$, we construct a $\graph=(\nodes,\edges)$, where $\nodes$ are decorated by the atoms and
%		\[ \edges = \{ \nodesof{\formula}: \formula\in\formulaset \} \, , \]
%	where by $\nodesof{\formula}$ we denote the minimal set such that there exists a tensor $\hypercoreat{\catvariableof{\nodesof{\formula}}}$ with
%		\[ \formulaat{\catvariableof{\nodes}} = \hypercoreat{\catvariableof{\nodesof{\formula}}} \otimes \onesat{\catvariableof{\nodes/\nodesof{\formula}}} \, . \]
%	Any \MarkovLogicNetwork{} $\mlnprobat{\shortcatvariables}$ is then a Markov Network given the graph $\graphof{\formulaset}$
%	$\{\expof{\canparamat{\indexedselvariable}\cdot\enumformula}
%\, :\,\selindexin\}$.
%\end{corollary}


% MLN as graphical models
%\MarkovLogicNetworks{} are Markov Networks with the factors given in a restricted form from the weighted truth of a formula.
%Each formula is seen as a factor of the graphical model.

To summarize, our sparse representations of \MarkovLogicNetworks{} implement the two sparsity mechanisms described in \parref{par:one}, originating from graphical models and propositional syntax:
\begin{itemize}
    \item \textbf{\IndependenceMechanism{}} (see \secref{sec:independenceMechanism}):
    Formulas often depend only on subsets of atoms, and therefore the variable $\formulavar$ is independent on some atomic variables $\catvariableof{\atomenumerator}$.
    This exploits the main sparsity mechanism in graphical models, where factors in sparse representations depend only on a subset of variables.
    \item \textbf{\ComputationMechanism{}} (see \secref{sec:computationMechanism}):
    When the features of an exponential family are compositions of smaller formulas, the computation core is decomposed into a tensor network of their basis encodings.
    This can be regarded as the main sparsity mechanism of propositional logics, where syntactical decompositions of formulas are exploited.
    Further, when the structure of the smaller formulas is shared among different features, the respective basis encodings need to be instantiated only once.
\end{itemize}


\subsubsect{Selection encodings for energy tensors}

%% Tensor Representation of MLN

As for generic exponential families, we can represent \MarkovLogicNetworks{} in terms of their energy tensors
%The energy tensor of an \MarkovLogicNetwork{} is the contraction
\begin{align}
    \mlnenergy\left[\shortcatvariables\right]
    = \sum_{\selindexin} \canparamat{\indexedselvariable} \cdot \enumformulaat{\shortcatvariables}
    = \contractionof{\sencodingofat{\formulaset}{\shortcatvariables,\selvariable},\canparamat{\selvariable}}{\shortcatvariables} \, .
\end{align}
The energy tensor provides an exponential representation of the distribution by
\begin{align}
    \mlnprobat{\shortcatvariables} = \normalizationof{\expof{\mlnenergy}}{\shortcatvariables} \, .
\end{align}

In case of a common structure of the formulas in a \MarkovLogicNetwork{}, formula selecting networks (see \charef{cha:formulaSelection}) can be applied to represent their energies.
% Energy representation
%The weighted sum of formulas is then the energy of the \MarkovLogicNetwork{}.
We represent the superposition of formulas as a contraction with a parameter tensor.
Given a factored parametrization of formulas $\exformula_{\selindices}$ with indices $\selindexof{\selenumerator}$ we have the superposition by the network representation:
\begin{center}
    \begin{tikzpicture}[thick, scale=0.35] % , baseline = -3.5pt

\node[anchor=east] (text) at (-3,0) {$\sum_{\parindexof{[\parorder]}\in\parstates} \canparamat{\selvariableof{[\parorder]}=\parindexof{[\parorder]}} {\exformula_{\parindexof{[\parorder]}}} \quad {=}$};

%\node[anchor=center] (text) at (0.5,-8) {$\mathrm{log}$};

%\drawatomcore{3.5}{-8}{$\rencodingof{\fselectionmap}$}
%\drawatomindices{3.5}{-12}	
%
%
%\drawatomcore{3.5}{-4}{$\canparam$}
%\drawparindices{3.5}{-8}	



\drawatomindices{0}{-4}
\draw (-1,3) rectangle (5, -3);
\node[anchor=center] (text) at (2,0) {$\rencodingof{\fselectionmap}$};

\draw[->] (2,3)--(2,5) node[midway,right] {\tiny $\catvariableof{\fselectionmap}$}; 
\draw (1,5) rectangle (3,7);
\node[anchor=center] (text) at (2,6) {$\tbasis$};

\draw[<-] (5,-2)--(7,-2) node[midway,below] {\tiny $\selvariableof{0}$}; 
\draw[<-] (5,-0.5)--(7,-0.5) node[midway,below] {\tiny $\selvariableof{1}$}; 
\node[anchor=center] (text) at (6,0.75) {$\vdots$};
\draw[<-] (5,2)--(7,2) node[midway,above] {\tiny $\selvariableof{\parorder\shortminus1}$}; 

\draw (7,3) rectangle (9,-3);
\node[anchor=center] (text) at (8,0) {$\canparam$};

\end{tikzpicture}
\end{center}


% Representation 
If the number of atoms and parameters gets large, it is important to represent the tensor ${\exformula_{\selindices}}$ efficiently in tensor network format and avoid contractions.
Such efficient decomposition schemes can be found using \decompositionSparsity{}, see \secref{sec:selectionSparsityHLN}.
To avoid inefficiency issues, we also have to represent the parameter tensor $\canparam$ in a tensor network format to improve the variance of estimations (see \charef{cha:concentration}) and provide efficient numerical algorithms.

% Fail of full probability representation
However, when required to instantiate the probability distribution of a \MarkovLogicNetwork{} as a tensor network, we need to exponentiate and normalize the energy tensor, a task for which basis encodings are required.
For such tasks, contractions of formula selecting networks are not sufficient and each formula with a nonvanishing weight needs to be instantiated as a factor tensor of a Markov Network.






\subsect{Expressivity}\label{sec:MLNMaxMintermRep}

Based on \MarkovLogicNetworks{} containing only maxterms and minterms (see \defref{def:clauses}), we now show that any positive probability distribution has a representation by a \MarkovLogicNetwork{}.

\begin{theorem}
    \label{the:maximalClausesRepresentation}\label{the:mintermExpressivityMLN}
    Let there be a positive probability distribution
    \[ \probat{\shortcatvariables} \in \bigotimes_{\atomenumeratorin}\rr^2 \, . \]
    Then the \MarkovLogicNetwork{} of minterms (see \defref{def:clauses})
    \[ \mintermformulaset = \{\mintermof{\atomindices} \wcols \atomindices\in\atomstates \}\]
    with parameters %with nonzero weights at the maxterms indexed by $\atomindicesin$
    \[ \canparamat{\selvariableof{0}=\catindexof{0},\ldots,\selvariableof{\atomorder-1}=\catindexof{\atomorder-1}}% \weightof{\mintermof{\atomindices}}
    = \ln \probat{\indexedcatvariables} \]
    coincides with $\probat{\shortcatvariables}$.

    Further, the \MarkovLogicNetwork{} of maxterms
    \[ \maxtermformulaset = \{\maxtermof{\atomindices} \wcols \atomindices\in\atomstates \}\]
    with parameters
    \[ \canparamat{\selvariableof{0}=\catindexof{0},\ldots,\selvariableof{\atomorder-1}=\catindexof{\atomorder-1}} %\weightof{\maxtermof{\atomindices}}
    = - \ln\probat{\indexedcatvariables} \]
    coincides with $\probat{\shortcatvariables}$.
\end{theorem}
\begin{proof}
    It suffices to show, that in both cases of choosing $\formulaset$ by minterms or maxterms with the respective parameters
    \[ \mlnenergy =  \ln\probat{\shortcatvariables} \]
    and therefore
    \[ \mlnprobat{\shortcatvariables}
    = \normalizationof{\expof{\mlnenergy}}{\shortcatvariables}
    =  \contractionof{\expof{\mlnenergy}}{\shortcatvariables}
    = \probat{\shortcatvariables}\, . \]

    In the case of minterms, we notice that for any $\atomindicesin$
    \[ \mintermof{\atomindices}[\shortcatvariables] = \onehotmapofat{\atomindices}{\shortcatvariables} \]
    and thus with the weights in the claim
    \[ \sum_{\atomindicesin}
    \left( \ln \probat{\indexedcatvariables} \right) \cdot \mintermof{\atomindices}[\shortcatvariables]
    = \ln\probat{\shortcatvariables} \, .
    \]

    For the maxterms we have analogously
    \[ \maxtermof{\atomindices}[\shortcatvariables] = \onesat{\shortcatvariables} - \onehotmapofat{\catindices}{\shortcatvariables} \]
    and thus that the maximal clauses coincide with the one-hot encodings of respective states.
    We thus have
    \begin{align*}
        & \sum_{\atomindicesin}
        \left( - \ln \probat{\indexedcatvariables} \right) \cdot \maxtermof{\atomindices}[\shortcatvariables] \\
        & =
        \left(  \sum_{\nodes_0\subset [\atomorder]}
        \left( - \ln \probat{\indexedcatvariables} \right) \cdot \onesat{\shortcatvariables} \right) \\
        & \quad +
        \left(  \sum_{\nodes_0\subset [\atomorder]}
        \left(  \ln \probat{\indexedcatvariables} \right) \cdot
        \onehotmapofat{\catindices}{\shortcatvariables}
        \right)
        \\
        & = \ln\probat{\shortcatvariables} + \lambda \cdot  \onesat{\shortcatvariables}\,,
    \end{align*}
    where $\lambda$ is a constant.
\end{proof}

We note, that there are $2^{\atomorder}$ maxterms and $2^{\atomorder}$ minterms, which would have to be instantiated by basis encodings to get a tensor network decomposition.
This large number of features originates from the generality of the representation scheme.
As a fundamental tradeoff, efficient representations come at the expense of a smaller expressivity of the representation scheme.


% Redundant parametrization
%In general, this representation is redundant, since any offset of the weight by $\lambda\cdot\ones$ results in the same distribution.
%However, the only $\bar{\canparam}$ are multiples of $\onesat{\shortcatvariables}$.

% Comparison with previous schemes
\theref{the:maximalClausesRepresentation} is the analogue in Markov Logic to \theref{the:tensorToMaxMinTerms}, which shows that any binary tensor has a representation by a logical formula, to probability tensors.
Here we require positive distributions for well-defined energy tensors.

% Markov Networks
Sparser representation formats based on the same principle as used in \theref{the:maximalClausesRepresentation} can be constructed to represent markov networks by \MarkovLogicNetworks{}.
Here, we can separately instantiate the factors by combinations of terms and clauses only involving the variables containted in the factor.
\red{The reason for this is, that minterms form a partition statistics, which will be discussed in more detail in \charef{cha:basisCalculus}, see \defref{def:partitionStatistic}.}

\subsect{Examples}

Let us now provide examples of \MarkovLogicNetworks{}.

\subsubsect{Distribution of independent variables}

We show next, the independent positive distributions are representable by tuning the $\atomorder$ weights of the atomic formulas and keeping all other weights zero.

\begin{theorem}
    \label{the:independentAtomicMLN}
    Let $\probat{\shortcatvariables}$ be a positive probability distribution, such that atomic formulas are independent from each other.
    Then $\probat{\shortcatvariables}$ is the \MarkovLogicNetwork{} of atomic formulas
    \[ \atomformulaset = \{\atomicformulaof{\catenumerator} \wcols \catenumeratorin \} \]
    and parameters
    \[ \canparamat{\selvariable=\catenumerator}
    = \lnof{\frac{
        \contractionof{\probtensor}{\catvariableof{\catenumerator}=1}
    }{
        \contractionof{\probtensor}{\catvariableof{\catenumerator}=0}
    }} \]
%	Any distribution such that the atom satisfaction is independent from each other is reproducible by a MLN with nonzero weights only for the atomic formulas.
\end{theorem}
\begin{proof}
%	Using the independent assumptions, the probability tensor factorizes into normalized vectors to each atom, with are transformed atomic formulas (leaving out the neutral ones tensors).
%	We then find a weight to each atom such that the vector is reproduced by the contraction with the activation core.

    By \theref{the:independenceProductCriterion} we get a decomposition
    \[ \probat{\shortcatvariables} = \bigotimes_{\catenumeratorin} \probofat{\catenumerator}{\catvariableof{\catenumerator}} \,  \]
    where
    \[ \probofat{\catenumerator}{\catvariableof{\catenumerator}} = \contractionof{\probtensor}{\catvariableof{\catenumerator}} \, . \]

    By assumption of positivity, the vector $\probofat{\catenumerator}{\catvariableof{\catenumerator}}$ is positive for each $\catenumeratorin$ and the parameter
    \[ \canparamat{\selvariable=\catenumerator}
    = \lnof{\frac{
        \probofat{\catenumerator}{\catvariableof{\catenumerator}=1}
    }{
        \probofat{\catenumerator}{\catvariableof{\catenumerator}=0}
    }} \]
    well-defined.

    We then notice, that
    \[ \expdistofat{(\{\atomicformulaof{\catenumerator}\},\canparamat{\selvariable=\catenumerator})}{\catvariableof{\catenumerator}}
    = \probofat{\catenumerator}{\catvariableof{\catenumerator}}\]
    and therefore with the parameter vector of dimension $\seldim=\catorder$ defined as
    \[ \canparamat{\selvariable} = \sum_{\catenumeratorin} \canparamat{\selvariable=\catenumerator} \cdot \onehotmapofat{\catenumerator}{\selvariable}  \]
    we have
    \begin{align*}
        \expdistofat{(\{\atomicformulaof{\catenumerator} \wcols \catenumeratorin\},\canparam)}{\shortcatvariables}
        & = \bigotimes_{\catenumeratorin} \expdistofat{(\{\atomicformulaof{\catenumerator}\},\canparamat{\selvariable=\catenumerator})}{\catvariableof{\catenumerator}} \\
        & = \bigotimes_{\catenumeratorin} \probofat{\catenumerator}{\catvariableof{\catenumerator}} \\
        & = \probat{\shortcatvariables} \, . \qedhere
    \end{align*}
\end{proof}

%In general, the statistic to an atomic formula measures the marginal distribution. -> To Parameter Estimation

% Failing to be positive -> Hybrid networks
In \theref{the:independentAtomicMLN} we made the assumption of positive distributions.
If the distribution fails to be positive, we still get a decomposition into distributions of each variable, but there is at least one factor failing to be positive.
Such factors need to be treated by \HybridLogicNetworks{}, that is they are base measure for an exponential family coinciding with a logical literal (see \secref{sec:hardNetworks}).

% Energy representation
All atomic formulas can be selected by a single variable selecting tensor, that is
\[ \energytensorofat{(\{\atomicformulaof{\catenumerator} \wcols \catenumeratorin\},\canparam)}{\shortcatvariables}
= \contractionof{\vselectionmapat{\shortcatvariables,\selvariable},\canparamat{\selvariable}}{\shortcatvariables} \, .
\]

% Holds also more generally for any formula! -> Place it earlier?
In case of negative coordiantes $\canparamat{\selvariable=\catenumerator}$ it is convenient to replace $\atomicformulaof{\catenumerator}$ by $\lnot\atomicformulaof{\catenumerator}$, in order to facilitate the interpretation.
The probability distribution is left invariant, when also replacing $\canparamat{\selvariable=\catenumerator}$ by $-\canparamat{\selvariable=\catenumerator}$.



\subsubsect{Boltzmann machines}\label{sec:boltzmannMachines}

A Boltzmann machine is a distribution of boolean variables $\shortcatvariables$ depending on a weight matrix %(see e.g. Chapter 43 in \cite{mackay_information_2003})
\begin{align*}
    W[\selvariableof{\vselectionsymbol,0},\selvariableof{\vselectionsymbol,1}] \in\rr^{\catorder}\otimes\rr^{\catorder}
    \quad \text{and a bias vector} \quad b[\selvariableof{\vselectionsymbol,0}] \in\rr^{\catorder} \, .
\end{align*}
Its distribution is
\begin{align*}
    \probofat{W,b}{\shortcatvariables} = \normalizationof{\expof{\energytensorofat{W,b}{\shortcatvariables}}}{\shortcatvariables}
\end{align*}
where its energy tensor is
\begin{align*}
    \energytensorofat{W,b}{\indexedshortcatvariables} =
    \sum_{\atomenumerator,\secatomenumerator \in [\atomorder]}
    W[\selvariableof{\vselectionsymbol,0}=\atomenumerator, \selvariableof{\vselectionsymbol,1}=\secatomenumerator] \cdot \catindexof{\atomenumerator} \cdot \catindexof{\secatomenumerator}
    + \sum_{\atomenumerator\in[\atomorder]} b[\selvariableof{\vselectionsymbol,0}=\atomenumerator] \cdot \catindexof{\atomenumerator}\, .
\end{align*}
We notice, that this tensor coincides with the energy tensor of a \MarkovLogicNetwork{} with formula set
\begin{align*}
    \formulaset = \{ \catvariableof{\atomenumerator} \Leftrightarrow \catvariableof{\secatomenumerator} \wcols \atomenumerator,\secatomenumerator \in[\atomorder] \}
    \cup \{ \catvariableof{\atomenumerator}\wcols \atomenumeratorin \} \,
\end{align*}
with cardinality $\atomorder^2+\atomorder$.
For an sparse representation of this energy tensor using a formula selecting network, see \exaref{exa:boltzmannEnergySelection}.
Therefore, Boltzmann machines are specific \MarkovLogicNetworks{} with the statistic being biimplications between atoms and atoms itself.
Generic \MarkovLogicNetworks{} are more expressive than Boltzmann machines, by the flexibility to create further features by propositional formulas.

\sect{\HardLogicNetworks{}}\label{sec:hardNetworks} % To be dropped in the unification with the MLN chapter

% Hard logic vs markov logic
While exponential families are positive distributions, in logics probability distributions can assign states zero probability.
As a consequence, \MarkovLogicNetworks{} have a soft logic interpretation in the sense that violation of activated formulas have nonzero probability.
We now discuss their hard logic counterparts, where worlds not satisfying activated formulas have zero probability.

The probability function of \MarkovLogicNetworks{} with positive weights mimics the tensor network representation of the knowledge base, which is the conjunction of the formulas.
The maxima of the probability function coincide with the models of the corresponding knowledge base, if the latter is satisfiable.
However, since the \MarkovLogicNetwork{} is defined as a normalized exponentiation of the weighted formula sum, it is a positive distribution whereas uniform distributions among the models of a knowledge base assign zero probability to world failing to be a model.
Since both distributions are tensors in the same space to a factored system, we can take the limits of large weights and observe, that \MarkovLogicNetworks{} indeed converge to normalized knowledge bases.

\begin{lemma}
    \label{lem:localHardLimit}
    Given a formula $\exformula$ we have in the limit of $\invtemp\rightarrow\infty$ the coordinatewise convergence
    \begin{align*}
        \normalizationof{\expof{\invtemp\cdot\canparam\cdot\formulaat{\shortcatvariables}}}{\shortcatvariables} \rightarrow
        \begin{cases}
            \normalizationof{\exformula}{\shortcatvariables} & \ifspace \text{$\exformula$ satisfiable and } \canparam>0\\
            \normalizationof{\ones}{\shortcatvariables} & \ifspace \text{$\exformula$ or $\lnot\exformula$ not satisfiable or } \canparam=0 \\
            \normalizationof{\lnot\exformula}{\shortcatvariables} & \ifspace \text{$\lnot\exformula$ satisfiable and } \canparam<0\\
        \end{cases} \, .
    \end{align*}
\end{lemma}
\begin{proof}
    The statement is trivial for the case of $\exformula$ or $\lnot\exformula$ being satisfiable, or $\canparam=0$, since then for all $\invtemp\in\rr$
    \begin{align*}
        \normalizationof{\expof{\invtemp\cdot\canparam\cdot\formulaat{\shortcatvariables}}}{\shortcatvariables} = \normalizationof{\ones}{\shortcatvariables} \, .
    \end{align*}
    In the case of $\exformula$ being satisfiable and $\canparam>0$ we have
    \begin{align*}
        %\partitionfunctionof{\mlnparameters}
        \contraction{\expof{\invtemp\cdot\canparam\cdot\formulaat{\shortcatvariables}}}
        = \left(\prod_{\atomenumeratorin} \catdimof{\atomenumerator}\right) - \contraction{\exformula} + \contraction{\exformula} \cdot \expof{\invtemp\cdot\canparam}
    \end{align*}
    and therefore for any $\shortcatindices\in\atomstates$ with $\formulaat{\indexedshortcatvariables}=1$
    \begin{align*}
        \normalizationof{\expof{\invtemp\cdot\canparam\cdot\exformula}}{\indexedshortcatvariables}
        &= \frac{
            \expof{\invtemp\cdot\canparam}
        }{
            \left(\prod_{\atomenumeratorin} \catdimof{\atomenumerator} \right) - \contraction{\exformula} + \contraction{\exformula} \cdot \expof{\invtemp\cdot\canparam}
        } \\
        & \rightarrow \frac{1}{\contraction{\exformula}}
        = \normalizationof{\exformula}{\indexedcatvariables} \, .
    \end{align*}
    For any $\atomindices\in\atomstates$ with $\formulaat{\indexedshortcatvariables}=0$ we have on the other side
    \begin{align*}
        \normalizationof{\expof{\invtemp\cdot\canparam\cdot\exformula}}{\indexedcatvariables}
        &= \frac{
            1
        }{
            \left(\prod_{\atomenumeratorin} \catdimof{\atomenumerator}\right) - \contraction{\exformula} + \contraction{\exformula} \cdot \expof{\invtemp\cdot\canparam}
        } \\
        & \rightarrow 0
        = \normalizationof{\exformula}{\indexedcatvariables} \, .
    \end{align*}
    This shows the statement in the case of $\exformula$ being satisfiable and $\canparam>0$.
    The case of $\lnot\exformula$ being satisfiable and $\canparam<0$ follows from a similar argument.
\end{proof}

\lemref{lem:localHardLimit} only characterizes the annealing limits in exponential families with statistics by single formulas.
If there are multiple formulas the limiting distribution gets more involved.
We will characterize the generic limiting distributions later in \secref{sec:hardLogicLimit} based on face measures in corresponding mean parameter polytopes.

% Representation by activation cores
We can represent the local limiting distribution in the three cases by an activation core contracted to the computation core $\bencodingofat{\exformula}{\formulavar,\shortcatvariables}$.
When choosing
\begin{align*}
    \kcoreat{\headvariableof{\exformula}} =
    \begin{cases}
        \tbasisat{\headvariableof{\exformula}} & \ifspace \text{$\exformula$ satisfiable and } \canparam>0\\
        \onesat{\headvariableof{\exformula}} & \ifspace \text{$\exformula$ or $\lnot\exformula$ not satisfiable or } \canparam=0 \\
        \fbasisat{\headvariableof{\exformula}} & \ifspace \text{$\lnot\exformula$ satisfiable and } \canparam<0\\
    \end{cases}
\end{align*}
we have
\begin{align*}
    \normalizationof{\expof{\canparam\cdot\formulaat{\shortcatvariables}}}{\shortcatvariables} \rightarrow
    \normalizationof{\bencodingofat{\exformula}{\headvariableof{\exformula},\shortcatvariables},\kcoreat{\headvariableof{\exformula}}}{\shortcatvariables} \, .
\end{align*}

We notice, that these three activation cores are all non-vanishing boolean cores in dimension $2$.
This motivates the definition of \HardLogicNetworks{} as follows.

\begin{definition}[\HardLogicNetwork{}]
    \label{def:hardLogicNetwork}
    Given a boolean statistic $\hlnstat$, a subset $\hardlegset\subset[\seldim]$ and a tuple $\headindexof{\hardlegset}\in\bigtimes_{\selindex\in\hardlegset}[2]$ the \HardLogicNetwork{} is the distribution
    \begin{align*}
        \probofat{\hlnstat,\hardparam}{\shortcatvariables}
        = \normalizationof{\hlnstatccwith,\hardacttensorwith}{\shortcatvariables}
    \end{align*}
    where $\hardacttensorwith=\bigotimes_{\selindexin}\hardactlegwith$ and for $\selindexin$
    \begin{align*}
        \hardactlegwith =
        \begin{cases}
            \onehotmapofat{\headindexof{\selindex}}{\headvariableof{\selindex}} & \ifspace \selindex\in\hardlegset  \\
            \onesat{\headvariableof{\selindex}} & \ifspace \selindex\notin\hardlegset \\
        \end{cases} \, ,
    \end{align*}
    provided that the normalization exists.
\end{definition}

We notice, that with the parametrization by the tuple $\hardparam$, we can represent and non-vanishing elementary boolean tensor in $\bigotimes_{\selindexin}\rr^2$.
Any activation leg core $\onesat{\headvariableof{\selindex}}$ furthermore results in a trivial contraction with the corresponding computation core.
Based on these insights we can characterize \HardLogicNetworks{} by propositional formulas.

\begin{theorem}
    \label{the:hlnFormulaCharacterization}
    For a boolean statistics $\hlnstat$ and parameters $\hardparam$ the \HardLogicNetwork{} exists, if and only if the formula
    \begin{align*}
        \hlnformula =
        \left(\bigwedge_{\selindex\in\hardlegset\wcols\headindexof{\selindex}=1} \enumformulaat{\shortcatvariables}\right)
        \land
        \left(\bigwedge_{\selindex\in\hardlegset\wcols\headindexof{\selindex}=0} \lnot\enumformulaat{\shortcatvariables}\right)
    \end{align*}
    is satisfiable.
    In that case we have
    \begin{align*}
        \probofat{\hlnstat,\hardparam}{\shortcatvariables}
        = \normalizationof{\hlnformula}{\shortcatvariables} \, .
    \end{align*}
\end{theorem}
\begin{proof}
    We apply that
    \begin{align*}
        \contractionof{\bencodingofat{\enumformula}{\headvariableof{\selindex},\shortcatvariables},\hardactlegwith}{\shortcatvariables}
        = \begin{cases}
              \enumformulawith &  \ifspace \hardactlegwith=\tbasisat{\headvariableof{\selindex}} \\% CAN BE SHORTENED BY \selindex\in\hardlegset!
              \lnot\enumformulawith & \ifspace \hardactlegwith=\fbasisat{\headvariableof{\selindex}} \\
              \onesat{\shortcatvariables} & \ifspace \hardactlegwith=\onesat{\headvariableof{\selindex}}
        \end{cases}
    \end{align*}
    With the definition of $\hardacttensorwith$ in \defref{def:hardLogicNetwork} we get
    \begin{align*}
        &\contractionof{\bencodingofat{\hlnstat}{\headvariables,\shortcatvariables},\hardacttensorwith}{\shortcatvariables} \\
        & \quad = \contractionof{
            \bigcup_{\selindexin}\{\bencodingofat{\enumformula}{\headvariableof{\selindex},\shortcatvariables},\hardactlegwith\}
        }{\shortcatvariables} \\
        & \quad =   \contractionof{
            \left\{
            \contractionof{\bencodingofat{\enumformula}{\headvariableof{\selindex},\shortcatvariables},\hardactlegwith}{\shortcatvariables}
            \wcols\selindexin\right\}
        }{\shortcatvariables} \\
        & \quad = \contractionof{
            \{
            \enumformulaat{\shortcatvariables}
            \wcols\hardactlegwith = \tbasisat{\headvariableof{\selindex}}
            \}
            \cup
            \{
            \enumformulaat{\shortcatvariables}
            \wcols\hardactlegwith = \fbasisat{\headvariableof{\selindex}}
            \}
        }{\shortcatvariables} \\
        & \quad = \hlnformulawith \, .
    \end{align*}
    The normalization of this contraction exists if and only if $\contraction{\hlnformula}>0$, that is if and only if $\hlnformula$ is satisfiable.
    In that case we have
    \begin{align*}
        \probofat{\hlnstat,\hardparam}{\shortcatvariables}
        = \frac{\hlnformulawith}{\contraction{\hlnformula}}
        = \normalizationof{\hlnformula}{\shortcatvariables} \, . & \qedhere
    \end{align*}
\end{proof}



\sect{\HybridLogicNetworks{}}\label{sec:hybridNetworks}

\MarkovLogicNetworks{} are by definition positive distributions and are represented by positive activation tensors in $\elrealizabledistsof{\hlnstat}$ (see \figref{fig:elementaryComputableSketch}).
In contrary, \HardLogicNetworks{} model uniform distributions over model sets of the respective Knowledge Base and therefore have vanishing coordinates.
They are represented by boolean activation tensors in $\elrealizabledistsof{\hlnstat}$.
Having discussed these representation approaches, let us now investigate the general case of distributions in $\elrealizabledistsof{\hlnstat}$, which we call \HybridLogicNetworks{}.

\subsect{Definition}

Towards defining \HybridLogicNetworks{}, we represent any distribution computable by $\hlnstat$ and $\elgraph$, as a normalization of a \HardLogicNetwork{} and a \MarkovLogicNetwork{}.

\begin{theorem}
    \label{the:hybridNetworkRepresentation}
    Any distribution $\probwith\in\elrealizabledistsof{\hlnstat}$ is a normalized contraction of a \HardLogicNetwork{} and a \MarkovLogicNetwork{}, that is there are for $\selindexin$ boolean cores $\hardactlegwith$ and a weight vector $\canparamat{\selvariable}$ such that
    \begin{align*}
        \probwith =
        \normalizationof{
            \{\bencodingofat{\hlnstat}{\headvariables,\shortcatvariables}\}
            \cup \left(\bigcup_{\selindexin}\{\hardactlegwith,\softactlegwith\}\right)
        }{\shortcatvariables} \, .
    \end{align*}
\end{theorem}
\begin{proof}
    We show the claim by splitting the activation cores into a contraction of hard (i.e. logical) and soft (i.e. probabilistic) activation cores.
    Since $\probwith\in\elrealizabledistsof{\hlnstat}$ we find for $\selindexin$ non-negative vectors $\acttensorlegwith$ such that
    \begin{align*}
        \probwith =
        \normalizationof{
            \{\bencodingofat{\hlnstat}{\headvariables,\shortcatvariables}\}
            \cup \{\acttensorlegwith\wcols\selindexin\}
        }{\shortcatvariables} \, .
    \end{align*}
    We now define for $\selindexin$ boolean cores by
    \begin{align*}
        \hardactlegwith =
        \greaterzeroof{\acttensorlegwith} \,
    \end{align*}
    and coordinates of the canonical parameter vector $\canparamwith$ by
    \begin{align*}
        \canparamat{\indexedselvariable} =
        \begin{cases}
            0 & \ifspace \hardactlegwith\neq\onesat{\headvariableof{\selindex}} \\
            \lnof{\frac{\acttensorlegat{\headvariableof{\selindex}=1}}{\acttensorlegat{\headvariableof{\selindex}=0}}} & \text{else} \\
        \end{cases} \, .
    \end{align*}
    For all $\selindexin$ we have that $\canparamat{\indexedselvariable}$ is well-defined, since by assumption $\acttensorlegwith$ is positive, if $\hardactlegwith=\onesat{\headvariableof{\selindex}}$.
    If for $\selindexin$ we have $\hardactlegwith\neq\onesat{\headvariableof{\selindex}}$, then $\acttensorlegwith\in\{\tbasisat{\headvariableof{\selindex}},\fbasisat{\headvariableof{\selindex}}\}$, since $\acttensorlegwith=\zerosat{\headvariableof{\selindex}}$ would lead to a vanishing contraction with the basis encoding of $\hlnstat$.
    Then is a real number $\lambda_{\selindex}>0$ such that
    \begin{align*}
        \acttensorlegwith
        = \lambda_{\selindex} \cdot \hardactlegwith
    \end{align*}
    and, since $\softactsymbolofat{\selindex,0}{\headvariableof{\selindex}} = \onesat{\headvariableof{\selindex}}$ also
    \begin{align*}
        \acttensorlegwith
        = \lambda_{\selindex} \cdot \contractionof{\hardactlegwith,\softactlegwith}{\headvariableof{\selindex}} \, .
    \end{align*}
    Conversely, if for $\selindexin$ we have $\hardactlegwith=\onesat{\headvariableof{\selindex}}$, then there is a scalar $\lambda_{\selindex}>0$ such that
    \begin{align*}
        \acttensorlegwith
        &= \lambda_{\selindex} \cdot \softactlegwith \\
        &= \lambda_{\selindex} \cdot \contractionof{\hardactlegwith,\softactlegwith}{\headvariableof{\selindex}} \, .
    \end{align*}
    We therefore have
    \begin{align*}
        &\contractionof{
            \{\bencodingofat*{\hlnstat}{\headvariables,\shortcatvariables}\}
            \cup \{\acttensorlegwith\wcols\selindexin\}
        }{\shortcatvariables}\\
        &\quad =\left(\prod_{\selindexin}\lambda_{\selindex}\right)\cdot\contractionof{
            \{\bencodingofat{\hlnstat}{\headvariables,\shortcatvariables}\}
            \cup \left(\bigcup_{\selindexin}\{\hardactlegwith,\softactlegwith\}\right)
        }{\shortcatvariables}
    \end{align*}
    and the claim follows, since the normalization of a tensor is invariant under multiplication with the positive scalar $\left(\prod_{\selindexin}\lambda_{\selindex}\right)$.
\end{proof}

Motivated by this decomposition, we now define and parametrize \HybridLogicNetworks{}.

\begin{definition}[\HybridLogicNetwork{}]
    \label{def:hybridLogicNetwork}
    Given a boolean statistic $\hlnstat$ we call any element of $\elrealizabledistsof{\hlnstat}$ a \HybridLogicNetwork{} (HLN).
    The extended canonical parameter set to $\hlnstat$ is the set
    \begin{align*}
        \hybridparamset\coloneqq
        \{\hardparam)\wcols \variableset\subset[\seldim]\ncond \headindexof{\variableset}\in\bigtimes_{\selindex\in\variableset}[2]\} \times \parspace \, .
    \end{align*}
    To each \HybridLogicNetwork{} $\hlnwith$ we find a tuple $\hybridparam$ consistent of a subset $\hardlegset\subset[\seldim]$, a tuple $\headindexof{\hardlegset}\in\bigtimes_{\selindex\in\hardlegset}[2]$ and $\canparamwithin$ such that
    \begin{align*}
        \hlnwith
        = \normalizationof{\hlnstatccwith,\paracttensorwith}{\shortcatvariables}
    \end{align*}
    where the activation core is
    \begin{align*}
        \paracttensorwith = \contractionof{\softacttensorwith,\hardacttensorwith}{\headvariables} \, .
    \end{align*}
\end{definition}

Note that by \theref{the:hybridNetworkRepresentation} we find for any \HybridLogicNetwork{} a parametrization by a tuple $\hybridparam$.
We further relate with the formalism of exponential families and determine the base measures which are used to build \HybridLogicNetworks{}.

\begin{theorem}
    Given a boolean statistic $\hlnstat$, we build for any $\hardlegset\subset[\seldim]$ and $\headindexof{\hardlegset}\in\bigtimes_{\selindex\in\hardlegset}[2]$ the formula
    \begin{align*} % Already used to classify Hard Logic Networks
        \hlnformula =
        \left(\bigwedge_{\selindex\in\hardlegset\wcols\headindexof{\selindex}=1} \enumformulaat{\shortcatvariables}\right)
        \land
        \left(\bigwedge_{\selindex\in\hardlegset\wcols\headindexof{\selindex}=1} \lnot\enumformulaat{\shortcatvariables}\right) \, .
    \end{align*}
    Then we have
    \begin{align*}
        \elrealizabledistsof{\hlnstat}
        = \bigcup_{\hardlegset\subset[\seldim]} \,\, \bigcup_{\headindexof{\hardlegset}\in\bigtimes_{\selindex\in\hardlegset}[2]\wcols\contraction{\hlnformula}>0} \expfamilyof{\hlnstat,\hlnformula} \, .
    \end{align*}
\end{theorem}
\begin{proof}
    \theref{the:hybridNetworkRepresentation} implies that $\probwith\in\elrealizabledistsof{\hlnstat}$ is equivalent to the existence of parameters $\hybridparam$ with $\probwith=\hlnwith$.
    We further have that $\hlnwith$ is the member of the exponential family $\expfamilyof{\hlnstat,\hlnformula}$ with canonical parameter $\canparam$, since
    \begin{align*}
        \hlnwith
        &= \normalizationof{\hlnstatccwith,\softacttensorwith,\hardacttensorwith}{\shortcatvariables} \\
        &= \normalizationof{\hlnstatccwith,\softacttensorwith,\contractionof{\hlnstatccwith,\hardacttensorwith}{\shortcatvariables}}{\shortcatvariables} \\
        &= \normalizationof{\hlnstatccwith,\softacttensorwith,\hlnformulawith}{\shortcatvariables} \, ,
    \end{align*}
    where we used \theref{the:hlnFormulaCharacterization} in the last equation.
    We therefore have $\probwith\in\elrealizabledistsof{\hlnstat}$ if and only if there is a tuple $\hardparam$ with $\probwith\in\expfamilyof{\hlnstat,\hlnformula}$.
\end{proof}

% Base Measure trick
We can thus understand \HybridLogicNetworks{}, as a union of exponential families, where the union is over all boolean activation cores understood as generating a base measure.
This trick is known to the field of variational inference, see for Example~3.6 in \cite{wainwright_graphical_2008}.
For an example of a tensor network representation of a \HybridLogicNetwork{} see \figref{fig:hlnRepresentationExample}.

\begin{figure}[t]
    \begin{center}
        \begin{tikzpicture}[scale=0.35, thick, yscale=-1] % , baseline = -3.5pt


    \draw[-<-] (0,-1)--(0,1) node[midway,left] {\colorlabelsize $\catvariableof{a}$};
    \draw[-<-] (1.5,-1)--(1.5,1) node[midway,right] {\colorlabelsize $\catvariableof{b}$};
    \draw[-<-] (3,-1)--(3,1) node[midway,right] {\colorlabelsize $\catvariableof{c}$};
    \draw (-1,-1) rectangle (4, -3);
    \node[anchor=center] (text) at (1.5,-2) {$\partitionfunction \cdot \probtensor$};



    \node[anchor=center] (text) at (5.5,-2) {${=}$};


    \begin{scope}
        [shift={(11,0)}]

        \draw[->-] (0,1)--(0,-1)  node[midway,left] {\colorlabelsize $\catvariableof{a}$};
        \draw[->-]  (3,1)--(3,-1) node[midway,right] {\colorlabelsize $\catvariableof{b}$};
        \draw[->-] (6,1)--(6,-1) node[midway,right] {\colorlabelsize $\catvariableof{c}$};

        \draw (-1,-1) rectangle (4, -3);
        \node[anchor=center] (text) at (1.5,-2) {$\bencodingof{\lor}$};

        \draw[->-] (1.5,-3) --(1.5,-4.5) node[midway,right]{\colorlabelsize $\headvariableof{a \lor b}$};
        \drawvariabledot{1.5}{-4.5}
        \draw[->-] (1.5,-4.5) --(1.5,-6) ;


        \draw[fill, \probcolor] (1.5,-4.5) circle (\dotsize);
        \draw[\probcolor] (1.5,-4.5) -- (-0.25,-4.5);
        \draw[\probcolor]  (-6.75, -3.5) rectangle (-0.25, -6.5);
        \node[anchor=center,\probcolor] (text) at (-3.5,-5) {$\begin{bmatrix}
                                                                  1 \\
                                                                  \expof{\canparam}
        \end{bmatrix}$};

        \draw (5,-1) rectangle (7, -3);
        \node[anchor=center] (text) at (6,-2) {$\bencodingof{\lnot}$};

        \draw[->-] (6,-3) --(6,-4.5) node[midway,right]{\colorlabelsize $\headvariableof{\lnot c}$};
        \drawvariabledot{6}{-4.5}
        \draw[->-] (6,-4.5) --(6,-6);



        \draw (0.5,-6) rectangle (6.5,-8);
        \node[anchor=center] (text) at (3.5,-7) {$\bencodingof{\lor}$};

        \draw[-<-] (4,-9.5) -- (4,-8) node[midway,right] {\colorlabelsize $\headvariableof{a \lor b \lor \lnot c}$};

        \draw[fill,\concolor] (4,-9.5) circle (\dotsize);
        \draw[\concolor] (4,-9.5) -- (2.25,-9.5);
        \draw[\concolor]  (0.25, -8.5) rectangle (2.25, -11.5);
        \node[anchor=center,\concolor] (text) at (1.25,-10) {$\begin{bmatrix}
                                                                  0 \\
                                                                  1
        \end{bmatrix}$};

%\draw (3,-9) rectangle (5,-11);
%\node[anchor=center] (text) at (4,-10) {$\truevectorat{}$};

    \end{scope}

\end{tikzpicture}
    \end{center}
    \caption{Tensor network representation of a \HybridLogicNetwork{} with statistics $\hlnstat=(\catvariableof{a}\lor\catvariableof{b},\catvariableof{a}\lor\catvariableof{b}\lor \lnot c)$, following \theref{the:hybridNetworkRepresentation}.
    Besides the computation cores required to compute the statistics, we have one non-trivial \textcolor{\probcolor}{probabilistic soft activation core $\softactsymbolofat{0,\canparamat{\selvariable=0}}{\headvariableof{a\lor b}}$} and one non-trivial \textcolor{\concolor}{constraint hard activation core $\kcoreofat{1}{\headvariableof{a\lor b \lor \lnot c}}$}.
    The trivial activation cores $\kcoreofat{0}{\headvariableof{a\lor b}}$ and $\softactsymbolofat{1,0}{\headvariableof{a\lor b \lor \lnot c}}$ are omitted from the diagram, since leaving the contraction invariant.}
    \label{fig:hlnRepresentationExample}
\end{figure}

\subsect{Logical Reasoning Properties}

Deciding probabilistic entailment (see \defref{def:probEntailment}) with respect to \HybridLogicNetworks{} can be reduced to the hard logic parts of the network.

\begin{theorem}
    \label{the:hlnEntailmentReduction}
    Let $\hlnwith$ be a \HybridLogicNetwork{}.
    Given a query formula $\exformula$ we have that
    \begin{align*}
        \hlnwith \models \exformula
    \end{align*}
    if and only if
    \begin{align*}
        \hlnformula \models \exformula \, .
    \end{align*}
\end{theorem}
\begin{proof}
    This follows from \theref{the:factorReduction} and
    \begin{align*}
        \greaterzeroof{\hlnwith} = \hlnformula
    \end{align*}
    using the representation of \HybridLogicNetworks{} as exponential distributions in \theref{the:hybridNetworkRepresentation}, and the fact that the support of any member of an exponential family is the base measure.
\end{proof}

Formulas in $\softformulaset$, which are entailed or contradicted by $\hardformulaset$ are redundant, as we show next.

\begin{theorem}%\label{the:hlnRepRedundancy}
    If for a \HybridLogicNetwork{} $\hlnwith$ and any formula $\enumformula\in\formulaset$ we have
    \begin{align*}
        \hlnformula\models\enumformula  \quad \text{or} \quad \hlnformula\models\lnot\enumformula
    \end{align*}
    then
    \begin{align*}
        \hlnwith = \probofat{\hlnstat/\{\enumformula\},(\hardlegset/\{\selindex\},\headindexof{\hardlegset/\{\selindex\}},\tilde{\canparam})}{\shortcatvariables}
    \end{align*}
    where $\tilde{\canparam}$ denotes the tensor $\canparam$, where the coordinate to $\enumformula$ is dropped.
\end{theorem}
\begin{proof}
    The theorem can be shown by isolation of the factor to the hard formula, which is constant for all situations.
\end{proof}

%% Now in the 
A similar statement holds for the hard formulas itself, as shown in \theref{the:ReduncancyOfEntailed}.

% Utility in Contraction KB implementation
These results are especially interesting for the efficient implementation of \algoref{alg:contractionKB}, which has been introduced in \charef{cha:logicalReasoning}.
By \theref{the:hlnEntailmentReduction} only the hard logic parts of a \HybridLogicNetwork{} are required in the ASK operation.


\sect{Sparse Representation Mechanisms}

Let us now investigate sparsity mechanisms, which can be exploited for efficient representations of \HybridLogicNetworks{}.

\subsect{\DecompositionSparsity{}}

\DecompositionSparsity{} aims at a tensor network representation of the computation network.
It is a generalization of the syntactic decomposition of single propositional formulas as introduced in \charef{cha:logicalRepresentation}.
When each formula coincides with some node formula of a node in a decomposition graph (see \defref{def:formulaDecomposition}), then the tensor network representation of the decomposition graph serves as computation network.
Denoting the nodes to $\enumformula$ by $\node_\selindex$ we get
\begin{align*}
    \bencodingofat{\hlnstat}{\headvariables,\shortcatvariables} =
    \breakablecontractionof{
        &\left\{
        \bencodingofat{\connectiveof{\node}}{\headvariableof{\node},\headvariableof{\incomingnodes}} \wcols (\incomingnodes,\{\node\})\in\edges
        \right\} \cup \\
        & \{\identityat{\headvariableof{\atomenumerator},\catvariableof{\atomenumerator}} \wcols \atomenumeratorin\} \cup \\
        &\left\{
        \identityat{\headvariableof{\selindex},\headvariableof{\node_{\selindex}}} \wcols \selindexin
        \right\}
    }{\shortcatvariables} \, .
\end{align*}
In \figref{fig:hlnRepresentationExample} is an example of a \HybridLogicNetwork{}, which computation network is represented using \decompositionSparsity{}.

\subsect{\SelectionSparsity{}}\label{sec:selectionSparsityHLN}

\SelectionSparsity{} exploits formula selecting networks, which expressivity is the statistic $\hlnstat$ to a \HybridLogicNetwork{}.
In this sense \selectionSparsity{} is more flexible than \DecompositionSparsity{}, since it does not demand identical subformulas, but only an identical decomposition graph.
The limitation of this scheme is that it can only represent weighted sums of the formulas by contractions of the formula selecting tensor with a weight tensor.
Such weighted sums appear in the energy tensor of \MarkovLogicNetworks{}, but note that the energy tensor of generic \HybridLogicNetworks{} is not defined.
In \exaref{exa:boltzmannEnergySelection} we now demonstrate this representation mechanism on the energy tensors of Boltzmann machines.

\begin{example}[Energy tensors of Boltzmann machines]
    \label{exa:boltzmannEnergySelection}
    In \secref{sec:boltzmannMachines} we have described Boltzmann machines as an example of \MarkovLogicNetworks{}, with formulas by
    \begin{align*}
        \formulaset = \{ \catvariableof{\atomenumerator} \Leftrightarrow \catvariableof{\secatomenumerator} \wcols \atomenumerator,\secatomenumerator \in[\atomorder] \}
        \cup \{ \catvariableof{\atomenumerator}\wcols \atomenumeratorin \} \,
    \end{align*}
    To demonstrate a representation of the energy tensor of a Boltzmann machine using \SelectionSparsity{}, we now define a formula selecting network to $\formulaset$ (see \figref{fig:boltzmannEnergy}).
    Each formula is in the expressivity of an architecture consisting of a single binary logical neuron selecting any variable of $\shortcatvariables$ in each argument and selecting connectives $\{\eqbincon,\lpasbincon\}$, where by $\lpasbincon$ we refer to a connective passing the first argument, defined for $\catindexofin{0}, \catindexofin{1}$ as
    \begin{align*}
        \lpasbincon[\indexedcatvariableof{0},\indexedcatvariableof{1}] = \vselectionmapat{\indexedcatvariableof{0},\indexedcatvariableof{1},\selvariableof{\vselectionsymbol}=0} \, .
    \end{align*}
    Given a weight matrix $W[\selvariableof{\vselectionsymbol,0},\selvariableof{\vselectionsymbol,1}] \in\rr^{\catorder}\otimes\rr^{\catorder}$ and a bias vector $b[\selvariableof{\vselectionsymbol,0}] \in\rr^{\catorder}$ we choose a canonical parameter as
    \begin{align*}
        \canparamat{\selvariableof{\cselectionsymbol},\selvariableof{\vselectionsymbol,0},\selvariableof{\vselectionsymbol,1}}
        = \onehotmapofat{0}{\selvariableof{\cselectionsymbol}} \otimes W[\selvariableof{\vselectionsymbol,0},\selvariableof{\vselectionsymbol,1}]
        + \tbasisat{\selvariableof{\cselectionsymbol}} \otimes b[\selvariableof{\vselectionsymbol,0}] \otimes  \onehotmapofat{0}{\selvariableof{\vselectionsymbol,1}} \, .
    \end{align*}
    We then have (see \figref{fig:boltzmannEnergy})
    \begin{align*}
        \energytensorofat{W,b}{\shortcatvariables} =
        \contractionof{\fsnnat{\shortcatvariables,\selvariableof{\cselectionsymbol},\selvariableof{\vselectionsymbol,0},\selvariableof{\vselectionsymbol,1}},
            \canparamat{\selvariableof{\cselectionsymbol},\selvariableof{\vselectionsymbol,0},\selvariableof{\vselectionsymbol,1}}}{\shortcatvariables} \, .
    \end{align*}
\end{example}

\begin{figure}[t]
    \begin{center}
        \begin{tikzpicture}[thick, scale=0.35] % , baseline = -3.5pt

\drawatomindices{0}{-4}
\draw (-1,1) rectangle (5, -3);
\node[anchor=center] (text) at (2,-1) {$\sencodingof{\formulaset}$};

%\draw[->] (2,-1)--(2,1) node[midway,right] {\tiny ${\atomicformulaof{\parindexof{1}} \land \atomicformulaof{\parindexof{2}}}$}; 

\draw[<-] (5,0.5) -- (7,0.5) node[midway, above] {\tiny $\selvariableof{\cselectionsymbol}$};
\draw[<-] (5,-1)--(7,-1) node[midway,above] {\tiny $\selvariableof{\vselectionsymbol,1}$}; 
\draw[<-] (5,-2.5)--(7,-2.5) node[midway,below] {\tiny $\selvariableof{\vselectionsymbol,0}$}; 

\draw (7,1) rectangle (9, -3);
\node[anchor=center] (text) at (8,-1) {$\canparam$};


		
\node[anchor=center] (text) at (12,-2) {${=}$};


\begin{scope}[shift={(17,8)}]

% SkeletonCores

%% Would be required to match the formula tensor example, but would get messy!

%\drawatomindices{0}{0}
%\draw (-1,-1) rectangle (5, -3);
%\node[anchor=center] (text) at (1.5,-2) {$\skeleton^{\land}$};
%\drawatomindices{0}{-4}


%\drawatomindices{7}{0}
%\draw (6,-1) rectangle (12, -3);
%\node[anchor=center] (text) at (9,-2) {$\skeleton^{\lnot}$};
%\drawatomindices{7}{-4}

%\drawatomindices{3.5}{-4}	
%\draw (-1,-5) rectangle (12,-7);
%\node[anchor=center] (text) at (5.5,-6) {$\skeletoncoreof{\land}$};
%\drawatomindices{0}{-8}	
%\drawatomindices{7}{-8}	


	\begin{scope}[shift={(0,-10)}]
	
		\draw (4.5,7) rectangle (6.5, 9);	
		\node[anchor=center] (text) at (5.5,8) {$\onehotmapof{1}$};
		
		\draw[->] (5.5,5) -- (5.5,7) node[midway, right] {\tiny $\headvariableof{\lneuron}$};
		\draw (1,3) rectangle (10, 5);
		\node[anchor=center] (text) at (5.5,4) {$\rencodingof{\{\eqbincon,\lpasbincon \}}$};
		\draw (10,4) -- (12,4);
		\draw[<-] (12,4) -- (14,4) node[midway, above] {\tiny $\selvariableof{\cselectionsymbol}$};
			
		% SelectorCores
		\draw[->] (2,1) -- (2,3) node[midway, left] {\tiny $\headvariableof{\vselectionsymbol,0}$};
		\draw (-1,1) rectangle (5, -1);
		\node[anchor=center] (text) at (2,0) {$\selectorcoreof{1}$};
		\draw (5,0) -- (12,0);
		\draw[<-] (12,0) -- (14,0) node[midway, above] {\tiny $\selvariableof{\vselectionsymbol,1}$};
		\begin{scope}[shift={(0,-2)}]
			\draw[<-] (0,1)--(0,-3) node[midway,left] {\tiny $\catvariableof{0}$}; 
			\draw[<-] (1.5,1)--(1.5,-3) node[midway,left] {\tiny $\catvariableof{1}$}; 
			\node[anchor=center] (text) at (3,0) {$\cdots$};
			\draw[<-] (4,1)--(4,-3) node[midway,right] {\tiny $\catvariableof{\atomorder\shortminus1}$}; 
		\end{scope}


		\draw (9,-1) -- (9,1);
		\draw[->] (9,1) -- (9,3) node[midway, left] {\tiny $\headvariableof{\vselectionsymbol,1}$};
		\draw (6,-3) rectangle (12, -1);
		\node[anchor=center] (text) at (9,-2) {$\selectorcoreof{0}$};
		\draw[<-] (12,-2) -- (14,-2) node[midway, above] {\tiny $\selvariableof{\vselectionsymbol,0}$};
		\drawatomindices{7}{-4}	
		
		% ParameterCores
		\draw (14,5) rectangle (16, -3);
		\node[anchor=center] (text) at (15,1) {$\canparam$};
		
		\begin{scope}[shift={(-3.5,8)}]
			\draw[fill] (7.5,-15) circle (0.25cm);
			\draw[] (7.5,-15) to[bend left=25] (3.5,-13);
			\draw[] (7.5,-15) to[bend right=25] (10.5,-13);

			\draw[fill] (9,-15.25) circle (0.25cm);
			\draw[] (9,-15.25) to[bend left=25] (5,-13);
			\draw[] (9,-15.25) to[bend right=25] (12,-13);

			\draw[fill] (11.5,-15) circle (0.25cm);
			\draw[] (11.5,-15) to[bend left=25] (7.5,-13);
			\draw[] (11.5,-15) to[bend right=25] (14.5,-13);

			\drawatomindices{7.5}{-16}

		\end{scope}
	
	\end{scope}

\end{scope}

\end{tikzpicture}
    \end{center}
    \caption{Tensor network representation of the energy of a Boltzmann machine}
    \label{fig:boltzmannEnergy}
\end{figure}

\subsect{\PolynomialSparsity{}}\label{sec:HLNpolyRepresentation}

%% First occurance of bas+ tensors and polynomial sparsity
We now introduce a further mechanism to find sparse representation formats for the introduced \HybridLogicNetwork{}s, which we call \polynomialSparsity{}.
The motivation of this mechanism is the observation, that terms and clauses can be represented storing the signs of the literals only.

This is an application of the basis+ $\cpformat$ format introduced in \charef{cha:sparseRepresentation}, which is connected to polynomial decompositions of boolean functions.
First of all, we establish a sparsity result for terms and clauses (see \defref{def:clauses}).

\begin{lemma}
    \label{lem:clauseTermBasisPlus}
    Any term is representable by a single monomial and any clause is representable by a sum of at most two monomials. %, any term of basis+ with rank 1. %Use also \baspluscprankof{}
\end{lemma}
\begin{proof}
    Let $\nodes_0$ and $\nodes_1$ be disjoint subsets of $\nodes$, then we have
    \begin{align*}
        \termof{\nodes_0}{\nodes_1} = \onehotmapofat{
            \{\catindexof{\atomenumerator} = 0 \wcols \atomenumerator\in\nodes_0\} \cup \{\catindexof{\atomenumerator} = 1 \wcols \atomenumerator\in\nodes_1\}
        }{\catvariableof{\nodes_0\cup\nodes_1}} \otimes \onesat{\catvariableof{\nodes/(\nodes_0\cup\nodes_1)}}
    \end{align*}
    and
    \begin{align*}
        \clauseof{\nodes_0}{\nodes_1} = \onesat{\catvariableof{\nodes}} - \onehotmapofat{
            \{\catindexof{\atomenumerator} = 0 \wcols \atomenumerator\in\nodes_0\} \cup \{\catindexof{\atomenumerator} = 1 \wcols \atomenumerator\in\nodes_1\}
        }{\catvariableof{\nodes_0\cup\nodes_1}}
        \otimes \onesat{\catvariableof{\nodes/(\nodes_0\cup\nodes_1)}} \, .
    \end{align*}
    We notice, that any tensors $\ones$ and $\onehotmapof{\catindex}\otimes \ones$ habe basis+-rank of $1$ and therefore $\termof{\nodes_0}{\nodes_1}$ of $1$ and $\clauseof{\nodes_0}{\nodes_1}$ of at most $2$.
\end{proof}

A formula in conjunctive normal form is a conjunction of clauses, where clauses are disjunctions of literals being atoms (positive literals) or negated atoms (negative literals).
Based on these normal forms, we show representations of formulas as sparse polynomials. %, which will be discussed in more detail in \charef{cha:sparseRepresentation} (see \defref{def:polynomialSparsity}).
We apply \lemref{lem:clauseTermBasisPlus} to show the following sparsity bound. % on the energy tensor of \MarkovLogicNetworks{}.

\begin{theorem}
    \label{the:formulaSlicePolynomialDecomposition}
    Any formula $\exformula$ with a conjunctive normal form of $\clausedim$ clauses satisfies
    \[ \polsparsityof{\exformula} \leq 2^{\clausedim} \, . \]
\end{theorem}
\begin{proof}
    Let $\exformula$ have a conjunctive normal form with clauses indexed by $\clauseenumeratorin$ and each clause represented by subsets $\nodes_0^\clauseenumerator, \nodes_1^\clauseenumerator$, that is
    \[ \exformula = \bigwedge_{\clauseenumeratorin} \clauseof{\nodes_0^\clauseenumerator}{\nodes_1^\clauseenumerator} \, . \]
    We now use the rank bound of \theref{the:CPrankContractionBound} and \lemref{lem:clauseTermBasisPlus} to get
    \begin{align*}
        \polsparsityof{\exformula} \leq \prod_{\clauseenumeratorin} \polsparsityof{\clauseof{\nodes_0^\clauseenumerator}{\nodes_1^\clauseenumerator}} \leq 2^{\clausedim} \, .
    \end{align*}
\end{proof}

We apply this result on the sparse representation of a single formula to derive sparse representations for \HardLogicNetworks{} and the energy tensor of \HybridLogicNetworks{}.
Both results use in addition to \theref{the:formulaSlicePolynomialDecomposition} sparsity bounds, which are shown by explicit representation construction in \charef{cha:sparseRepresentation}.

\begin{corollary}
    Any \HardLogicNetwork{} $\hardformulaset$ obeys
    \begin{align*}
        \polsparsityof{\hardformulaset} \leq \prod_{\exformula\in\hardformulaset} 2^{\clausedimof{\exformula}}
    \end{align*}
\end{corollary}
\begin{proof}
    We apply the contraction bound \theref{the:CPrankContractionBound} for the decomposition
    \begin{align*}
        \kbat{\shortcatvariables} = \contractionof{\{\formulaat{\shortcatvariables} \wcols \formula\in\hardformulaset\}}{\shortcatvariables}
    \end{align*}
    and get
    \begin{align*}
        \polsparsityof{\kb} \leq \prod_{\formula\in\hardformulaset} \polsparsityof{\formula} \, .
    \end{align*}
    The claimed bound follows with \theref{the:formulaSlicePolynomialDecomposition}.
\end{proof}

\begin{corollary}
    The energy tensor of a \HybridLogicNetwork{} with statistic $\hlnstat$
    \begin{align*}
        \polsparsityof{\contractionof{\sencmlnstatwith,\canparamwith}{\shortcatvariables}} \leq \sum_{\selindexin\wcols\canparamat{\indexedselvariable}\neq0} 2^{\clausedimof{\enumformula}} \, .
    \end{align*}
    where $\clausedimof{\enumformula}$ denotes the number of clauses in a conjunctive normal form of $\enumformula$.
\end{corollary}
\begin{proof}
    We decompse the energy into the sum
    \begin{align*}
        \contractionof{\sencmlnstatwith,\canparamwith}{\shortcatvariables}
        = \sum_{\selindexin\wcols\canparamat{\indexedselvariable}\neq0} \canparamat{\indexedselvariable} \cdot \enumformulawith
    \end{align*}
    and apply \theref{the:CPrankSumBound} to get
    \begin{align*}
        \polsparsityof{\contractionof{\sencmlnstatwith,\canparamwith}{\shortcatvariables}}
        \leq \sum_{\selindexin\wcols\canparamat{\indexedselvariable}\neq0} \polsparsityof{\enumformulawith}
        \leq \sum_{\selindexin\wcols\canparamat{\indexedselvariable}\neq0}2^{\clausedimof{\enumformula}} \, .
    \end{align*}
\end{proof}


\sect{Mean parameters of \HybridLogicNetworks{}}

% Polytope
While mean parameter polytopes $\genmeanset$ to generic exponential families have been subject to \charef{cha:probReasoning}, we in this section restrict to the mean polytopes of \HybridLogicNetworks{}, which we characterize using propositional logics.
\HybridLogicNetworks{} are exponential families, which statistic $\sstat$ consists of coordinates with $\imageof{\sstatcoordinateof{\selindex}}\subset\ozset$ and are therefore propositional formulas.
The convex polytope of mean parameters (see \defref{def:meanPolytope}) is for a statistic $\hlnstat$ of propositional formulas and a base measure $\basemeasure$ the set
\begin{align*}
    \hlnmeanset = \left\{ \contractionof{\probwith,\sencmlnstatwith}{\selvariable} \wcols \probwith\in\bmrealprobof{\basemeasure} \right\} \, ,
\end{align*}
where by $\bmrealprobof{\basemeasure}$ we denote the set of all by $\basemeasure$ representable probability distributions.
By \theref{the:meanPolytopeConvHull} the convex polytope has a characterization as a convex hull
\begin{align}
    \label{eq:hlnMeansetConvCharacterization}
    \hlnmeanset
    = \convhullof{\sencodingofat{\formulaset}{\indexedshortcatvariables,\selvariable} \wcols \shortcatindices\in\atomstates, \, \basemeasureat{\indexedshortcatvariables}=1} \, .
\end{align}

% 0/1 Polytopes
We notice, that for boolean statistics $\formulaset$ all $\sencodingofat{\formulaset}{\indexedshortcatvariables,\selvariable}$ are boolean vectors in $\parspace$.
The mean parameter polytopes are thus $0/1$-polytopes \cite{ziegler_lectures_2000,gillmann_01-polytopes_2007}, from which a few obvious properties follow.
Since those are convex subsets of the cube $[0,1]^\seldim$, which vertices are all boolean vectors in $\parspace$, also each $\sencodingofat{\formulaset}{\indexedshortcatvariables,\selvariable}$ (with $\basemeasureat{\indexedshortcatvariables}=1$) is a vertex.
Further, if for any $\selindexin$ we have $\meanparamat{\indexedselvariable}\in\ozset$, then $\meanparamwith$ is on a proper face of the cube and thus also of $\hlnmeanset$.
\red{In the following, we characterize the faces of the mean parameter polytope.}

\subsect{Vertices by \HardLogicNetworks{}} % -> More general: Face measures by \HardLogicNetworks{}

First of all, we show that the vertices of $\hlnmeanset$ are the boolean vectors contained in $\hlnmeanset$.

\begin{theorem}
    \label{the:vertexByHardLogicNetworks}
    Given a boolean statistic $\formulaset$, a base measure $\basemeasurewith$ and let $\meanparamwith\in\hlnmeanset$.
    Then the following are equivalent:
    \begin{itemize}
        \item[(i)] The set $\{\meanparamwith\}$ is a vertex of $\hlnmeanset$
        \item[(ii)] $\meanparamwith$ is boolean
    \end{itemize}
\end{theorem}
\begin{proof}
(i)
    $\Rightarrow$(ii):
    We use that any set of vectors, which convex hull forms a polytope, contains all vertices of that polytope (see Proposition 2.2 in \cite{ziegler_lectures_2013}).
    %For any polytope, and any set of vectors spanning the polytope as convex hull, all vertices are contained in all sets, which convex hull is $\hlnmeanset$,
    Since by definition $\hlnmeanset$ is the convex hull of boolean vectors $\sencmlnstatat{\indexedshortcatvariables,\selvariable}$, all vertices are boolean vectors.

    (ii)$\Rightarrow$(i):
    Since $\hlnmeanset$ is contained in $\fullparcube$ and all boolean vectors in $\fullparcube$ are vertices, whenever a boolean vector is contained in $\hlnmeanset$ it is a vertex.
\end{proof}

We can further provide an equivalent criterion for a vertex by the satisfaction of a corresponding formula.

\begin{theorem}
    Let there be a boolean statistic $\formulaset$ and let $\meanparamwith$ be a boolean vector.
    Then the following are equivalent:
    \begin{itemize}
        \item[(i)] The set $\{\meanparamwith\}$ is a vertex of $\hlnmeanset$.
        \item[(ii)] The formula
        \begin{align*}
            \vertexformula = \bigwedge_{\selindexin} \lnot^{(1-\meanparamat{\indexedselvariable})} \enumformulaat{\shortcatvariables}
        \end{align*}
        is satisfiable.
        \item[(iii)] The \HardLogicNetwork{} $\probofat{\hlnstat,([\seldim],\meanparam)}{\shortcatvariables}$ exists. % There is a \HardLogicNetwork{} $\probwith\in\hlnsetof{\hlnstat}$ such that
%        \begin{align*}
%            \meanparamwith = \contractionof{\probwith,\sencmlnstatwith}{\selvariable} \, .
%        \end{align*}
    \end{itemize}
    Here we denote by $\lnot^0$ the identity connective and by $\lnot^1=\lnot$ the logical negation.
\end{theorem}
\begin{proof}
(i)
    $\Rightarrow$(ii):
    If $\{\meanparamwith\}$ is a vertex of $\hlnmeanset$, then there is $\shortcatindices\in\atomstates$ such that % in particular $\meanparamwith\in$.
    \begin{align*}
        \meanparamwith = \sencodingofat{\formulaset}{\indexedshortcatvariables,\selvariable} \, .
    \end{align*}
    For all $\selindexin$ we have
    \begin{align*}
        \lnot^{(1-\meanparamat{\indexedselvariable})}\enumformulaat{\indexedshortcatvariables} \, .
    \end{align*}
    The index $\shortcatindices$ is therefore a model of $\vertexformula$ and $\vertexformula$ is satisfiable.

    (ii)$\Rightarrow$(iii):
    This follows from \theref{the:hlnFormulaCharacterization}.

    (iii)$\Rightarrow$(i):
    The mean parameter is reproduced by $\probofat{\hlnstat,([\seldim],\meanparam)}{\shortcatvariables}$ and therefore $\meanparamwith\in\hlnmeanset$.
    Since $\meanparamwith$ is further boolean, \theref{the:vertexByHardLogicNetworks} implies that $\{\meanparamwith\}$ is a vertex.
\end{proof}


\subsect{Faces represented by \HardLogicNetworks{}}

With \theref{the:vertexByHardLogicNetworks} we showed that all vertices are reproduced by \HardLogicNetworks{}.
In general, as we show next, \HardLogicNetworks{} are face measures to a set of faces which we characterize in the next theorem.

\begin{theorem}
    \label{the:faceMeasureHardLogicNetworks}
    Let $\hlnfaceset$ be a face of $\hlnmeanset$, then the following are equivalent:
    \begin{itemize}
        \item[(i)] There is a face $\cubeface$ of $\fullparcube$ such that $\hlnfaceset=\hlnmeanset\cap\cubeface$.
        \item[(ii)] There is a \HardLogicNetwork{} which coincides with the normalized face measure $\normalizationof{\basemeasureof{\formulaset,\facecondset}}{\shortcatvariables}$.
        \item[(iii)] The normalized face measure $\normalizationof{\basemeasureof{\formulaset,\facecondset}}{\shortcatvariables}$ is in $\realizabledistsof{\sstat,\elgraph,\basemeasure}$.
    \end{itemize}
\end{theorem}
\begin{proof}
(i)
    $\Rightarrow$(ii):
    Given the face $\cubeface$ such that $\hlnfaceset=\hlnmeanset\cap\cubeface$, we construct a boolean and basis+ elementary tensor (see \charef{cha:sparseRepresentation}), such that
    \begin{align*}
        \basemeasureofat{\sstat,\facecondset}{\shortcatvariables}
        = \contractionof{\bencodingofat{\formulaset}{\headvariables,\shortcatvariables},\kcoreofat{\facecondset}{\headvariables}}{\shortcatvariables} \, .
    \end{align*}
    For each $\selindexin$ we build the vectors
    \begin{align*}
        \hardactlegwith
        = \begin{cases}
              \tbasisat{\headvariableof{\selindex}} & \ifspace \uniquantwrtof{\meanparam\in\cubeface}{\meanparamat{\indexedselvariable}=1} \\
              \fbasisat{\headvariableof{\selindex}} & \ifspace \uniquantwrtof{\meanparam\in\cubeface}{\meanparamat{\indexedselvariable}=0} \\
              \onesat{\headvariableof{\selindex}} & \text{else}
        \end{cases}
    \end{align*}
    and get the activation tensor as
    \begin{align*}
        \kcoreofat{\facecondset}{\headvariables}
        = \bigotimes_{\selindexin} \hardactlegwith \, .
    \end{align*}

    (ii)$\Rightarrow$(iii):
    By definition any \HardLogicNetwork{} is in $\realizabledistsof{\sstat,\elgraph,\basemeasure}$.

    (iii)$\Rightarrow$(i):
    If the normalized face measure $\normalizationof{\basemeasureof{\formulaset,\facecondset}}{\shortcatvariables}$ is in $\realizabledistsof{\sstat,\elgraph,\basemeasure}$,
    Then there is a boolean and elementary tensor $\kcoreofat{\facecondset}{\headvariables}$ such that the face measure is
    \begin{align*}
        \basemeasureofat{\sstat,\facecondset}{\shortcatvariables}
        = \contractionof{\bencodingofat{\formulaset}{\headvariables,\shortcatvariables},\kcoreofat{\facecondset}{\headvariables}}{\shortcatvariables} \, .
    \end{align*}
    Boolean and elementary tensors in leg-dimension two are basis+ elementary tensors.
    Given a basis+ elementary tensor decomposition of $\kcoreof{\facecondset}$ we construct a face of the cube $\fullparcube$ by sets
    \begin{align*}
        \arbsetof{\selindex} = \begin{cases}
                                   \{0\} & \ifspace \hardactlegwith = \fbasisat{\headvariableof{\selindex}} \\
                                   \{1\} & \ifspace \hardactlegwith = \tbasisat{\headvariableof{\selindex}} \\
                                   [0,1] & \text{else}
        \end{cases}
    \end{align*}
    and define a face by the cartesian product of these sets as
    \begin{align*}
        \cubeface = \bigtimes_{\selindexin} \arbsetof{\selindex} \, .
    \end{align*}
    We then notice, that
    \begin{align*}
        \contractionof{\bencodingofat{\formulaset}{\headvariables,\shortcatvariables},\kcoreofat{\facecondset}{\headvariables}}{\indexedshortcatvariables} = 1
    \end{align*}
    if and only if $\sencmlnstatat{\indexedshortcatvariables,\selvariable}\in\cubeface$.
    Therefore, the parametrized face is the intersection of $\hlnmeanset$ with $\cubeface$.
\end{proof}

% \HardLogicNetwork{}
\theref{the:faceMeasureHardLogicNetworks} characterizes the faces, which measures are reproducible by \HardLogicNetworks{}.
The corresponding mean parameters are the face centers.


% Generalization to non-boolean features
\begin{remark}
    For non-boolean statistics $\sstat$, we can derive limited versions of \theref{the:faceMeasureHardLogicNetworks}.
    Any face, which is the intersection with faces of the distorted cube
    \begin{align*}
        \bigtimes_{\selindexin} [\min_{\catindex}\sstatcoordinateofat{\selindex}{\indexedcatvariable},\max_{\catindex}\sstatcoordinateofat{\selindex}{\indexedcatvariable}]
    \end{align*}
    has a representation of its encoded pre-image with an activation tensor of basis+ rank 1.
    Since for leg dimensions larger than 2 the boolean tensors with elementary decomposition contain more tensors than the basis+ tensors of rank 1, we can represent further faces.
\end{remark}

\subsect{Generic faces}

When the normalized face measure to a face is not reproduceable by a \HardLogicNetwork{}, it does not lie in $\realizabledistsof{\sstat,\elgraph,\basemeasure}$.
On the other side, \theref{the:faceMeasureCharacterization} implies, that in general $\sstat$ is a sufficient statistic and it thus lies in $\realizabledistsof{\sstat,\maxgraph,\basemeasure}$.
The core $\kcoreofat{\facecondset}{\headvariables}$ has then $\cpformat$ rank of at least two.
In general, we can express any face measure to faces of $\hlnmeanset$ by a propositional formula.

\begin{theorem}
    \label{the:faceMeasureCharacterizationHLN}
    Let $\hlnfaceset$ be a face of $\hlnmeanset$.
    Then the face measure is given by the formula
    \begin{align*}
        \hlnfacemeasure
        = \bigvee_{\meanparam\in\hlnfaceset\cap\boundaryparcube} \vertexformula
        = \bigvee_{\meanparam\in\hlnfaceset\cap\boundaryparcube}
        \bigwedge_{\selindexin} \lnot^{(1-\meanparamat{\indexedselvariable})} \enumformulaat{\shortcatvariables} \, .
    \end{align*}
\end{theorem}
\begin{proof}
    By \theref{the:faceMeasureCharacterization} we have
    \begin{align*}
        \hlnfacemeasureat{\shortcatvariables}
        =\contractionof{\hlnstatccwith,\kcoreofat{\facecondset}{\headvariables}}{\shortcatvariables}
    \end{align*}
    where
    \begin{align*}
        \kcoreofat{\facecondset}{\headvariables}
        = \sum_{\meanparam\in\hlnfaceset\cap\imageof{\sstatencoding}} \onehotmapofat{\headindex^{\meanparam}_{[\seldim]}}{\headvariables} \, .
    \end{align*}
    We notice, that for each mean parameter $\meanparam\in\hlnfaceset\cap\imageof{\sstatencoding}$ the boolean tensors
    \begin{align*}
        \contractionof{\hlnstatccwith,\onehotmapofat{\headindex^{\meanparam}_{[\seldim]}}{\headvariables}}{\shortcatvariables}
    \end{align*}
    coincide with $\vertexformula$.
    Further, the formulas $\vertexformula$ have disjoint model sets and their disjunction is therefore represented by their sum.
    We therefore have
    \begin{align*}
        \hlnfacemeasureat{\shortcatvariables}
        &=\sum_{\meanparam\in\hlnfaceset\cap\imageof{\sstatencoding}} \vertexformulaat{\shortcatvariables} \\
        &=\bigvee_{\meanparam\in\hlnfaceset\cap\boundaryparcube}
        \bigwedge_{\selindexin} \lnot^{(1-\meanparamat{\indexedselvariable})} \enumformulaat{\shortcatvariables} \, . \qedhere
    \end{align*}
\end{proof}


\subsect{Mean parameters reproducible by \HybridLogicNetworks{}}\label{sec:HLNrepMean} % Generalize to any mean parameter reproduction by \HybridLogicNetworks{}

Based on the representation results for face measures, we now study the mean parameters of \HybridLogicNetworks{}.
Let us find conditions on a mean parameter $\meanparamwith$ for the existence of a \HybridLogicNetwork{} reproducing it. %$\meanparamwith$.

\begin{theorem}
    \label{the:hlnInteriorCharacterization}
    \HybridLogicNetworks{} reproduce exactly those mean parameters $\meanparamwith$, which are in the effective interior of cube faces $\fullparcube$ (see \defref{def:effectiveInterior}), i.e. for which there is a face $\hlnfaceset$ of $\hlnmeanset$ and a face $\cubeface$ of the cube $\fullparcube$ such that
    \begin{align*}
        \meanparamwith\in\sbinteriorof{\hlnfaceset} \andspace \hlnfaceset = \hlnmeanset\cap\cubeface \, .
    \end{align*}
\end{theorem}
\begin{proof}
    \theref{the:faceMeasureHardLogicNetworks} implies that the normalization of the face measure $\basemeasureof{\sstat,\facecondset}$ to $\hlnfaceset$ is a \HardLogicNetwork{} and in $\elrealizabledistsof{\formulaset}$.
    By assumption, $\meanparamwith$ is in the effective interior of $\meansetof{\formulaset,\basemeasureof{\sstat,\facecondset}}$ and therefore reproduced by a distribution $\probwith$ in $\expfamilyof{\formulaset,\basemeasureof{\sstat,\facecondset}}$.
    Since $\probwith\in\elrealizabledistsof{\formulaset}$ we found a \HybridLogicNetwork{} reproducing $\meanparamwith$.
\end{proof}

We now provide a procedure to investigate, whether a mean parameter $\meanparamwith\in\hlnmeanset$ is reproduceable by a \HybridLogicNetwork{} in $\elrealizabledistsof{\formulaset}$.
To this end, we first determine the minimal face with respect to face inclusion (the partial order of the face lattice, see \cite{ziegler_lectures_2013}), which includes $\meanparamwith$.

\begin{lemma}
    \label{lem:minimalContainingFace}
    For each mean parameter $\meanparamwith\in\hlnmeanset$ the minimal face of the cube $\fullparcube$ containing $\meanparamwith$ is
    \begin{align*}
        \cubeface^{\meanparam}
        = \bigtimes_{\selindexin} \arbsetof{\selindex} \, .
    \end{align*}
    where for each $\selindexin$
    \begin{align*}
        \arbsetof{\selindex} = \begin{cases}
                                   \{\meanparamat{\indexedselvariable}\} & \ifspace \meanparamat{\indexedselvariable} \in \ozset \\
                                   [0,1] & \text{else}
        \end{cases} \, .
    \end{align*}
\end{lemma}
\begin{proof}
    Since $\hlnmeanset\subset\fullparcube$, $\meanparamwith$ is always contained in the maximal face $\fullparcube=\bigtimes_{\selindexin}[0,1]$ of the cube, and the smallest face containing $\meanparamwith$ exists.
    Any face of the cube has a representation as a cartesian product of $\arbsetof{\selindex}\in\{\{0\},\{1\},[0,1]\}$ and contains $\meanparamwith$ if and only if
    \begin{align*}
        \uniquantwrtof{\selindexin}{\left(\arbsetof{\selindex}=\{0\} \Rightarrow \meanparamat{\indexedselvariable}=0 \right) \land \left(\arbsetof{\selindex}=\{1\} \Rightarrow \meanparamat{\indexedselvariable}=1 \right)} \, .
    \end{align*}
    The minimal face containing $\meanparamwith$ is therefore $\cubeface^{\meanparam}$.
\end{proof}

% Determine whether reproducible
\begin{lemma}
    $\meanparamwith$ is reproducible by a \HybridLogicNetwork{}, if and only if it is in the effective interior of the set
    \begin{align*}
        \hlnmeanset\cap \cubeface^{\meanparam} \, .
    \end{align*}
    Here $\cubeface^{\meanparam}$ is the minimal face of the cube containing $\meanparamwith$.
\end{lemma}
\begin{proof}
    This follows from \theref{the:hlnInteriorCharacterization} and \lemref{lem:minimalContainingFace}.
\end{proof}

% Extension towards arbitrary hypergraphs
We can extend the discussion to $\realizabledistsof{\formulaset,\graph}$, where $\graph$ is an arbitrary hypergraph.
Whenever a normalized face measure is in $\realizabledistsof{\formulaset,\graph}$, then the all mean parameters on the interior of the face can be reproduced by $\realizabledistsof{\formulaset,\graph}$.

%By \theref{the:meanPolytopeInterior} the interior points are those realizable by a \HybridLogicNetwork{} with statistics $\hlnstat$ and base measure $\basemeasure$, as we state in the following Corollary.
%
%\begin{corollary}
%    \label{cor:interiorCharacterizationHLN}
%    If $\meanparamwith\in\interiorof{\hlnmeanset}$, or equivalently the statistic is minimal and $\meanparamwith$ is reproduceable by a distribution positive with respect to $\basemeasure$, then there is $\canparamat{\selvariable}$ such that $\expdistof{\formulaset,\canparam,\basemeasure}$ reproduces $\meanparamwith$.
%\end{corollary}



\subsect{Expressivity of \HybridLogicNetworks{}}

\begin{figure}[t]
    \begin{center}
        \begin{tikzpicture}[yscale=0.5, xscale=0.9]
	\node[anchor=center] (text) at (-2.25,13) {Distributions reproducing $\hlnmeanset$: $\realizabledistsof{\formulaset,\maxgraph}$};
	\draw[dashed] (-10.5,14) rectangle (5,6);

	\draw[dashed] (-10,12) rectangle (4.5,6.5);
	\node[anchor=center] (text) at (-2.25,11) {Interior of cube faces (\theref{the:hlnInteriorCharacterization}): $\realizabledistsof{\formulaset,\elgraph}$};
	
	\draw[\probcolor] (-9.5,10) rectangle (-3,7);
	\node[\probcolor] (text) at (-6,9) {\MarkovLogicNetworks{}: };
	\node[\probcolor] (text) at (-6,7.9) {Interior $\sbinteriorof{\hlnmeanset}$};

	\draw[\concolor] (-2.5,10) rectangle (4,7);
	\node[\concolor] (text) at (1,9) {\HardLogicNetworks{}:};
	\node[\concolor] (text) at (1,7.9) {Cube face centers};

\end{tikzpicture}
    \end{center}
    \caption{Sketch of distributions and their mean parameters with respect to .
    The \HybridLogicNetworks{} are also maximum entropy distribution, as we will show in \charef{cha:networkReasoning}.}
    \label{fig:reproducibleSketch}
\end{figure}

\begin{figure}[t]
    \begin{center}
        \begin{tikzpicture}[scale=0.35]
    % Define points

    \node[below] at (-1,7) {$\sbinteriorof{\hlnmeanset}$};

    \coordinate (A) at (0,0);

    \node[\concolor, below] at (A) {$\meanparamof{1}$};
    \draw[\concolor, fill] (A) circle (\dotsize);

    \coordinate (B) at (12,2.5);
    \path (A) -- (B) coordinate[pos=0.7] (P1);

    \node[below] at (P1) {$\meanparamof{2}$};
    \draw[fill] (P1) circle (\dotsize);

    \coordinate (P2) at (2,10);
    \node[\probcolor, below] at (P2) {$\meanparamof{3}$};
    \draw[\probcolor, fill] (P2) circle (\dotsize);

    \coordinate (C) at (7.5,12);
    \path (B) -- (C) coordinate[pos=0.5] (P4);
%    \node[left] at (P4) {$\hlnmeanset/\interiorof{\hlnmeanset}$};
    \coordinate (D) at (-3,12);
    \coordinate (E) at (-10,5);

    \draw[thick] (A) -- (B) -- (C) -- (D) -- (E) -- cycle;

    \draw[dashed] (-10,0) -- (12,0) -- (12,12) -- (-10,12) -- (-10,0);
    \node[left] at (-10,10) {$\fullparcube$};
\end{tikzpicture}


    \end{center}
    \caption{Sketch of the mean polytope ${\hlnmeanset}$ to a statistic $\hlnstat$ which is minimal with respect to $\basemeasure$, as a special case of the more generic sketch \figref{fig:meansetSketchGeneric}.
    The mean polytope is a subset of the $\seldim$-dimensional cube $\fullparcube$ (dashed in the sketch), where each mean parameter in one of the three cases $\meanparamof{1},\meanparamof{2}$ or $\meanparamof{3}$.
    \textcolor{\concolor}{Face centers} $\meanparamof{1}$ (in particular the vertices $\hlnmeanset\cap\boundaryparcube$) are reproducible by a \HardLogicNetwork{} given $\hlnstat$.
    Mean points on non-vertex faces outside the interior $\meanparamof{2}\in\hlnmeanset/(\interiorof{\hlnmeanset}\cap\boundaryparcube)$ are reproducible by \HybridLogicNetworks{} with statistic $\hlnstat$ and refined base measure $\secbasemeasure$.
    \textcolor{\probcolor}{Interior points} $\meanparamof{3}\in\interiorof{\hlnmeanset}$ are reproducible by a \MarkovLogicNetwork{} with statistic $\hlnstat$.
    }\label{fig:meansetSketch}
\end{figure}

In \figref{fig:reproducibleSketch} we classify the distributions by their mean parameters in $\hlnmeanset$ and draw a corresponding sketch of the mean parameter polytope in \figref{fig:meansetSketch}.
We have already shown, that the distributions in $\realizabledistsof{\formulaset,\maxgraph}$ suffice to reproduce all mean parameters in $\hlnmeanset$.
The \HybridLogicNetworks{} are the subset $\hlnsetof{\formulaset}\subset\realizabledistsof{\formulaset,\maxgraph}$, which can be computed with elementary activation cores.
% Reduction to \HybridLogicNetworks{}, question of sufficient expressivity
%Let us recall, that the set $\hlnsetof{\formulaset}$ contains all \HybridLogicNetworks{}, which can be realized as tensor networks with the same structure of computation and activation cores.
We now investigate, in which cases also $\hlnsetof{\formulaset}$ suffices to reproduce all mean parameters in $\hlnmeanset$, that is in which cases $\hlnmeanset$ coincides with
%whether we can reduce the set of probability distributions in the definition of the convex polytope of mean parameters to the set $\hlnsetof{\formulaset}$ , that is
\begin{align*}
    %\hlnmeanset|_{\hlnsetof{\formulaset}}
    %=
    \left\{ \contractionof{\probtensor,\sencmlnstat}{\selvariable} \wcols \probtensor\in\hlnsetof{\formulaset} \right\} \, .
\end{align*}
While this set is obviously a subset of $\hlnmeanset$, we investigate, for which $\formulaset$ there is an equivalence.
We will refer to the equality of both sets as sufficient expressivity of $\hlnsetof{\formulaset}$.
In the next example we provide a class of formulas, for which $\hlnsetof{\formulaset}$ does not have sufficient expressivity.

% Insufficient Expressivity
\begin{example}[Insufficient expressivity of $\hlnsetof{\formulaset}$ in cases of disjoint models]
    \label{exa:insufficientHLNsetExpressivity}
    To provide an example, where the set of \HybridLogicNetworks{} does not suffice to reproduce all possible mean parameters, consider the formulas
    \[ \formulaof{0} = \exrandom \land \secexrandom \quad, \quad \formulaof{1} = \lnot\exrandom \land \lnot\secexrandom \, . \]
    The probability distributions on the facet with normal $\canparam=[1\,\, 1]$ are those with support on the models of $\formulaof{0}\lor\formulaof{1}$.
    The \HybridLogicNetworks{} can only reproduce those with are supported on the model of $\formulaof{0}$ or the model of $\formulaof{1}$, but not their convex combinations.

    %% EXTENSION
    %More generally, one can construct similar examples by arbitrary sets of formulas with pairwise disjoint model sets.
    %If they do not sum to $\ones$, i.e. there is a world which is not a model to any formula, the statistic is minimal.
    %The vector $\canparamat{\selvariable} = \onesat{\selvariable}$ is then the normal of the facet with
    %All probabilities supported on the models of the formulas have mean parameters on this facet.
    %Such statistics will be further investigated as partition statistics in \secref{sec:partitionStatistics}. % Most precise: Partition statistics do sum to one
\end{example}


%\subsect{Examples}

We can relate our two standard examples of the atomic and the minterm formula sets to well-studied polytopes, namely the $\catorder$-dimensional hypercube and the standard simplex (see Lecture~0 in \cite{ziegler_lectures_2013}).

\begin{example}[Atomic formulas]
    \label{exa:atomicFormulasHypercube}
    %The assumption of \theref{the:sufficientHLNExpressivity} is satisfied in
    Let us consider the case of atomic formulas. % where the formulas $\formulaof{\formulaset,\canparam}$ are the atoms.
    The mean polytope in this case is the $\catorder$-dimensional hypercube
    \begin{align*}
        \meansetof{\atomformulaset,\ones} = \fullparcube
    \end{align*}
    which is called a simple polytope, since each vertex is contained in the minimal number of $\catorder$ facets.
    %Since the cube $\fullparcube$ is a face of itself, \theref{the:sufficientHLNExpressivity} implies $\hlnmeanset|_{\hlnsetof{\atomformulaset}}=\hlnmeanset$.

    The faces of a hypercube are enumerated in the following way.
    Each face is characterized by the projections onto each variable, which is either $\{0\}$, $\{1\}$ or $[0,1]$.
    The projections are represented by the tuple $\hardparam$ defined in the following way:
    \begin{itemize}
        \item We define the set $\hardlegset\subset[\atomorder]$ of variables, such that the projection onto the variable is $\{0\}$ or $\{1\}$
        \item We define to each $\selindex\in\hardlegset$ an index $\headindexof{\selindex}=0$ if the projection is $\{0\}$ and $\headindexof{\selindex}=1$ if the projection is $\{1\}$.
    \end{itemize}

%    There are thus $2^{\atomorder}$ different sets $\hardlegset$, each with $2^{\hardlegset}$ faces by a choice of $\headindexof{\selindex}$.

    Trivially, each face of the hypercube is a cube face and $\elmeansetof{\atomformulaset}=\meansetof{\atomformulaset}$.
\end{example}

%\subsect{Elementary realizable mean parameters}

With the findings of the above sections and the enumeration of cube faces, we can not characterize the set of mean parameters realizable by \HybridLogicNetworks{}.

\begin{definition}
    \label{def:elementaryRealizableMeanParams}
    The elementary realizable mean parameters to a statistic $\hlnstat$ and a base measure $\basemeasure$ is the set
    \begin{align*}
        \elmeansetof{\hlnstat,\basemeasure}
        = \bigcup_{\hardlegset\subset[\seldim]} \bigcup_{\headindexof{\hardlegset}\in\bigtimes_{\selindex\in\hardlegset}[2]}
        \sbinteriorof{\meansetof{\hlnstat,\basemeasure}\cup\cubeface^{\hardlegset,\headindexof{\hardlegset}}}
    \end{align*}
    where $\cubeface^{\hardlegset,\headindexof{\hardlegset}}$ is the face of the hypercube (see \exaref{exa:atomicFormulasHypercube}).
\end{definition}

We have
\begin{align*}
    \elmeansetof{\hlnstat,\basemeasure}
    = \left\{ \contractionof{\probtensor,\sencmlnstat}{\selvariable} \wcols \probtensor\in\hlnsetof{\formulaset} \right\} \, .
\end{align*}
A further example, where $\elmeansetof{\hlnstat,\basemeasure}=\hlnmeanset$ is the family of minterms.

\begin{example}[Minterm formulas]
    \label{exa:mintermHLNSet}
    The mean polytope is in the case of the minterm statistic (also referred to as universal statistic) and a base measure $\basemeasure$ the $\contraction{\basemeasure}-1$-dimensional standard simplex
    \begin{align*}
        \elmeansetof{\mintermformulaset,\basemeasure}
        = \convhullof{\onehotmapofat{\shortcatindices}{\shortcatvariables}\wcols\shortcatindicesin\ncond\basemeasureat{\indexedshortcatvariables}=1} \, .
    \end{align*}
    In this case, $\hlnsetof{\mintermformulaset}$ contains any distribution and therefore trivially realizes any mean parameter in $\meansetof{\mintermformulaset,\ones}$.

    The faces of the standard simplex are itself standard simplices to base measures $\secbasemeasure$ with $\secbasemeasure\prec\basemeasure$.
    We store them by the tuple $\hardparam$, where $\hardlegset$ is the support of $\secbasemeasure$ and $\headindexof{\selindex}=0$ for $\selindex\in\hardlegset$.
    Each face is a cube face, since it is the intersection of $\hlnsetof{\mintermformulaset}$ with the cube face $\hardparam$.
    In particular, we have $\elmeansetof{\mintermformulaset}=\meansetof{\mintermformulaset}$.
\end{example}


\subsect{The limit of hard logic}\label{sec:hardLogicLimit}

While \lemref{lem:localHardLimit} describes the limiting distributions of a statistic consistent of a single formula under annealing, we now turn towards statistics of multiple formulas.

\begin{theorem}
    \label{the:limitingDistribution}
    Let $\hlnstat$ be a boolean statistics, $\canparamwithin$ and $\hlnfaceset$ the face such that $\canparamwith\in\hlnmaxcone$.
    In the limit $\invtemp\rightarrow\infty$ the distribution $\probofat{\formulaset,\invtemp\cdot\canparam}{\shortcatvariables}$ converges coordinatewise to the normalized face measure $\hlnfacemeasure$, that is for any $\shortcatindices\in\facstates$ we have
    \begin{align*}
        \probofat{\formulaset,\invtemp\cdot\canparam}{\indexedshortcatvariables}
        \rightarrow \frac{1}{\contraction{\hlnfacemeasureat{\shortcatvariables}}} \cdot \hlnfacemeasureat{\indexedshortcatvariables} \, .
    \end{align*}
\end{theorem}
\begin{proof}
    The limit is supported on the set of states, for which $\contraction{\canparamat{\selvariable},\sencodingofat{\formulaset}{\indexedshortcatvariables,\selvariable}}$ is maximal.
    This set has an basis encoding by the face measure (see \defref{def:faceMeasure}) and the limiting distribution is thus the normalized face measure.
\end{proof}

We can now use characterizations of the face measure in case of satisfiable formulas to get a more explicit representation of the limiting distribution.

\begin{theorem}
    If for a $\canparamwithin$ the formula
    \begin{align*}
        \exformula =
        \bigwedge_{\selindex\wcols\canparamat{\indexedselvariable}\neq0} \lnot^{(1-\greaterzeroof{\canparamat{\indexedselvariable}})} \enumformula
    \end{align*}
    is satisfiable, then $\probofat{\formulaset,\invtemp\cdot\canparam}{\shortcatvariables}$ converges for $\invtemp\rightarrow\infty$ coordinatewise to the normalized uniform distribution among the models of $\exformula$.
\end{theorem}
\begin{proof}
    If $\exformula$ is satisfiable, then the face measure $\hlnfacemeasure$ to the face $\hlnfaceset$ with $\canparamwith\in\hlnmaxcone$ is the uniform distribution among the models of $\exformula$.
    The claim follows from \theref{the:limitingDistribution}.
\end{proof}

We notice, that $\exformula$ is computable with elementary activation cores, that is for
\begin{align*}
    \kcoreat{\headvariables}
    = \left(\bigotimes_{\selindex\wcols\canparamat{\indexedselvariable}=0} \onesat{\headvariableof{\selindex}}\right)
    \otimes \left(\bigotimes_{\selindex\wcols\canparamat{\indexedselvariable}>0} \tbasisat{\headvariableof{\selindex}}\right)
    \otimes \left(\bigotimes_{\selindex\wcols\canparamat{\indexedselvariable}<0} \fbasisat{\headvariableof{\selindex}}\right)
\end{align*}
we have
\begin{align*}
    \formulaat{\shortcatvariables} =
    \normalizationof{\bencodingofat{\hlnstat}{\headvariables,\shortcatvariables},\kcoreat{\headvariables}}{\shortcatvariables} \, .
\end{align*}


\sect{Discussion}

We understand \HybridLogicNetworks{} as neuro-symbolic models:
\begin{itemize}
    \item \textbf{Neural Paradigm} is the representation of models as compositions of smaller models.
    For \HybridLogicNetworks{}, the decompositions of logical formulas into their connectives implements this neural paradigm.
    In more generality the decompositions of sufficient statistics into composed functions has a tensor network representation based on basis calculus.
    Deeper nodes as carrying correlations of lower nodes.
    \item \textbf{Symbolic Paradigm} refers to the formalization of reasoning by human-interpretable symbols.
    Since the sufficient statistics of \HybridLogicNetworks{} are propositional formulas, they can be verbalized and interpreted based on their syntactical decompositions.
\end{itemize}

%\HybridLogicNetwork{}s are trainable Machine Learning models:
%\begin{itemize}
%    \item Expressivity: Can represent any positive distribution, as shown by \theref{the:maximalClausesRepresentation}, with $2^d$ formulas.
%    %\item Efficiency: Can only handle small subsets of possible formulas, since their possible number is huge.
%    %Tensor networks provide means to efficiently represent formulas depending on many variables and reason based on contractions.
%    \item Continuous parameterization: Distributions are differentiable functions dependent on their activation cores (e.g. in exponential parametrization).
%    The log-likelihood of data is therefore also differentiable function of their weights and we can exploit first-order methods in their optimization.
%\end{itemize}