\chapter{\chatextfolModels}\label{cha:folModels}

We now extend the tensor representation from to structured representations, whereas we previously focused on factored representation of systems.
%The models/events in this situation are precise relations between objects.


\red{We observe that the more expressive first-order logic bears another tensor structure:
The representation of each world is a boolean tensor.
}


% Formulas in FOL
%Formulas in first order logic can contain variables, which are placeholder for specific individuals.
%Given a model and an assignment of objects to the arguments of a formula, the truth of the formula can be interpreted.
%This truth interpretation defines thus for any model a tensor, which we call the grounding tensor.

%
\sect{World Tensors}

% Index interpretation of world domain
Since first-order logic follows structured representations of a system, a first-order logic world consists in objects and relations between them.
To each world there is a world domain $\worlddomain$ of objects, which we assume to be finite (this is a restrictive assumption).
We exploit the set-encoding formalism discussed in more detail in \charef{cha:basisCalculus} and use bijective index interpretation maps
\begin{align*}
    \indexinterpretation : [\inddim] \rightarrow \worlddomain \, .
\end{align*}
A so-called term variable $\indvariable$ takes states $\indindexin$, which represent objects
\begin{align*}
    \indexinterpretationat{\indindex} \in \worlddomain \, .
\end{align*}

%
The relations between objects are described by $\indorder$-ary predicates $\folpredicate$.
Given a specific world $\dataworld$ the truth of relations is represented by boolean tensors
\begin{align*}
    \groundingof{\folpredicate} : \symindstates\rightarrow\ozset \, .
\end{align*}
Given a tuple $\indindexlist\in\symindstates$ the boolean
\begin{align*}
    \groundingofat{\folpredicate}{\indexedindvariableof{0},\ldots,\indexedindvariableof{\indorder-1}} \in\ozset
\end{align*}
is called a grounding and encodes, whether the relation $\folpredicate$ is satisfied in the world $\dataworld$ for the objects $\invindexinterpretationat{\indindexof{0}},\ldots,\invindexinterpretationat{\indindexof{\indorder-1}}$.

% Assumptions
Let us assume, that we have a function-free theory with $\folpredicateorder$ predicates, where are predicates all of the same arity $\variableorder$.
We then formalize a world in the following based on a selection variable $\selvariable$ selecting a specific predicate and term variables $\shortindvariablelist=\indvariablelist$ representing choices of objects from a given set $\worlddomain$.

\begin{definition}[FOL World]
    \label{def:folWorld}
    Given a set of objects $\worlddomain$ enumerated by an index interpretation function $\indexinterpretation:[\inddim]\rightarrow \worlddomain$ and a finite set $\{\folpredicates\}$ of $\variableorder$-ary predicates a world is a boolean tensor
    \begin{align}
        \dataworldwith : [\catorder] \times \left( \symindstates\right) \rightarrow [2] \, . % ! Selvariable gets catorder !
    \end{align}
    We interpret the world tensor as encoding in the coordinate $\dataworldat{\selvariable=\catenumerator,\indexedshortindvariables}$, whether the $\catenumerator$-th predicate is satisfied on the object tuple $\invindexinterpretationat{\indindexof{0}},\ldots,\invindexinterpretationat{\indindexof{\indorder-1}}$.
\end{definition}


% Inclusion of functions, predicates of differing order
When the assumptions of function-free and constant variable order are not met, we can do the following tricks.
Functions are turned to predicates by their relation interpretation.
If there are predicates of different arity in the theory, we can trivially extend them to $\variableorder$ary predicates by tensor products with the trivial tensor $\ones$.
This can be done by a tensor product with $\onehotmapofat{\inddim}{\indvariable}$, where we add an auxiliary object $\indexinterpretationof{\inddim}$ as a placeholder for predicates with smaller arity.

% Finite worlds -> By database semantics?
While in first order logics, depending on the chosen semantics, worlds can have infinite sets of objects, we here only treat worlds with finite objects.

\subsect{Case of Propositional Logics}

%
Before continuing with the one-hot encoding of first-order logic worlds, let us show that the previously discussed formalism of propositional logics (see \charef{cha:logicalRepresentation}) is a special case of first-order logics, namely when demanding $\indorder=0$.
Consistent with \defref{def:folWorld} we have a propositional logic world by
\begin{align*}
    \dataworld: [\catorder] \rightarrow [2] \, ,
\end{align*}
which we have in \charef{cha:logicalRepresentation} represented by the assignments $\catindexof{\atomenumerator} = \dataworldat{\selvariable=\atomenumerator}$ to the categorical variables $\catvariableof{\atomenumerator}$.

% Comparison with PL
%Compared with propositional formulas, the grounding tensor does not take as input a specific world, but is defined on a given world.
%We show in this chapter, how both tensor interpretations can be transformed, i.e. by extracting samples from a FOL world $\dataworld$ interpretated as an empirical distribution over PL worlds, and by generating FOL worlds by a set of samples generated from a PL distribution.

% One-hot maps
To represent logical formulas as sets of possible worlds, and distributions of worlds, we applied in \parref{par:one} one-hot encodings of possible worlds.
For the case of propositional logics, this is
\begin{align*}
    \onehotmapofat{\dataworld}{\shortcatvariables} = \bigotimes_{\catenumeratorin} \onehotmapofat{\dataworldat{\selvariable=\atomenumerator}}{\catvariableof{\catenumerator}} \, .
\end{align*}

\subsect{One-hot encoding of worlds}

Let us now generalize the one-hot encodings of propositional logic worlds to worlds of first-order logic.
To encode the boolean tensors $\dataworld$ describing first order logics as basis elements of a tensor space, we take the one-hot encoding
\begin{align*}
    \onehotmap :
    \bigtimes_{\atomenumeratorin}\bigtimes_{\indindexofin{0}}\cdots\bigtimes_{\indindexofin{\indorder-1}} [2]
    \rightarrow \bigotimes_{\catenumeratorin}\bigotimes_{\indindexofin{0}}\cdots\bigotimes_{\indindexofin{\indorder-1}} \rr^2
\end{align*}
defined by
\begin{align*}
    \onehotmapofat{\dataworld}{\catvariableof{[\catorder]\times[\inddim]^{\indorder}}}
    = \bigotimes_{\catenumeratorin}\bigotimes_{\indindexofin{0}}\cdots\bigotimes_{\indindexofin{\indorder-1}}
    \onehotmapofat{\dataworldat{\selvariable=\atomenumerator,\indexedshortindvariables}}{\catvariableof{\catenumerator,\shortindindices}} \, .
\end{align*}
This is a tensor of order $\catorder\cdot\inddim^{\indorder}$, in a tensor space of dimension $2^{\left(\catorder\cdot\inddim^{\indorder}\right)}$.
Storage of such tensors in naive formats would not be possible.
However, the basis CP format discussed in \charef{cha:sparseCalculus} still provides storage with demand linear in the order $\catorder\cdot\inddim^{\indorder}$.

% Domain
Another issue when comparing different first-order logic worlds arises in potentially different world domains.
As we have explored, the cardinality of the domain influences the order of the one-hot encoding tensors.
To avoid such issues we here enumerate worlds coinciding in their domains.
This restriction is called database semantics (see e.g. Section 8.2.8 in \cite{russell_artificial_2021}), where only those worlds are considered, which domains have a one-to-one map to the constant symbols appearing in a respective knowledge base. % Unique name assumption + Domain closure!
% Factored systems
When restricting to worlds coinciding in their domain, we still have a factored representation of the system, since we can enumerate the possible worlds by a cartesian product.
However, the number of categorical variables representing the world is $\atomorder\cdot \inddim^{\indorder}$ and tensor representations, even in sparse formats, are not feasible due to the large order required.
These techniques to restrict to comparable factored representations are often refered to propositionalization of a first-order logic knowledge base.

% Propositional 
%Propositional worlds have been enumerated by indices of $\atomorder$ Booleans, that is for a world $\dataworld: [\folpredicateorder] \rightarrow [2]$ we take the index
%	\[ \atomindices \quad \text{where} \quad \atomlegindexof{\atomenumerator} = \dataworld(\atomenumerator) \, .  \]

% FOL
%When we want to enumerate the first order logic worlds to a fixed set of objects $\worlddomain$, we flatten the tensor $\dataworld$ and get indices
%	\[ \{\atomlegindexof{\atomenumerator,\indindexlist} \, : \, \atomenumeratorin, \indindexlist \in[\inddim] \} \quad
%	\text{where} \quad \atomlegindexof{\atomenumerator,\indindexlist} = \dataworld(\atomenumerator,\indindexlist) \, .  \]





\subsect{Probability distributions}

Having established the formalism of one-hot encodings also in the case of first-order logic worlds, we can now proceed with the definition of distributions and formulas, analogously to the development in \parref{par:one}.
Probability distributions over worlds coinciding on their domain are then non-negative and normed tensors
\begin{align*}
    \probat{\catvariableof{[\catorder]\times[\inddim]^{\indorder}}} \in \bigotimes_{\atomenumeratorin,\shortindindices\in[\inddim]^{\indorder}} \rr^2 \, .
\end{align*}
where each coordinate of a world $\dataworld$ is captured by a boolean random variable $\catvariableof{\atomenumerator,\shortindindices}$, indicating whether the $\atomenumerator$-th predicate holds on the object tuple indexed by $\shortindindices$.

% High-dimensional - watch out for repetitions!
We notice, that by definition these probability distributions are distributions of $\atomorder\cdot\inddim^{\indorder}$ Booleans with $2^{\left(\atomorder\cdot\inddim^{\indorder}\right)}$ many states.
% One-hot encodings minimal
Unfortunately, it is not possible to design encoding spaces of smaller dimension, when our aim is to get any distribution over possible worlds by an element in the encoding space.
This is due to the fact, that one-hot encodings provide a basis in the tensor space, as will be shown in \charef{cha:coordinateCalculus}.
The reason for the large encoding space dimension is therefore rooted in the equal number of possible worlds and not in an overhead in the dimension of the one-hot encoding space.
We will later in this chapter investigate methods to handle such high-dimensional distributions in the formalism of exponential families.

\subsect{Semantics of formulas}

Following the development of \charef{cha:logicalRepresentation}, we can choose a semantic approach to the definition of formulas, under the assumption of database semantics.
Since the semantic of a logical formula is the set of its models, we again have a one-to-one correspondence between logical formulas and the boolean tensors in the one-hot encoding space
\begin{align*}
    \bigotimes_{\atomenumeratorin,\shortindindices\in[\inddim]^{\indorder}} \rr^2 \, .
\end{align*}
This correspondence between the semantics and boolean tensor is through a subset encoding (see \defref{def:subsetEncoding}) of the respective formulas.
However, due to the large state dimensions, we will in the following sections choose a syntactical approach to the construction of formulas, which will naturally provide efficient tensor network decompositions.

\subsect{Two levels of tensor representation}

In comparison with propositional logics, first-order logic bears two levels of natural tensor representations.
In the first level, which we call the structured level, each world (see \defref{def:folWorld}) has a natural structure by a tensor, since it encodes relations between objects chosen by assignments to term variables.
This is different to the worlds of a propositional logic theory, which are represented by a boolean vector instead of a tensor.
The second level arises as in propositional logics, by understanding each world as a uncertain state and studying distributions over states, which are understood themself as a tensor (see \defref{def:probabilityDistribution}).
We call this the factored level, since it arises in general in the discussion of factored representations.
As argued above, the assumption of database semantics is central to exploit the tensor structure of the substitution level.
Under this assumption, representation of an uncertain state, or a collection of possible states, is done in the tensor space
\begin{align*}
    \bigotimes_{\atomenumeratorin,\shortindindices\in[\inddim]^{\indorder}} \rr^2 \,
\end{align*}
where the enumeration of the $2$-dimensional axes contains the tensor structure of the substitution level.



\sect{Formulas in a fixed first-order logic world}

Following the argumentation above, we in this section restrict to the exploitation of tensors in the structured level, namely a fixed world represented as a tensor $\dataworldwith$, see \defref{def:folWorld}.
We are specifically interested in the tensor network decomposition of first order formulas, which contain in full generality variables and therefore also have a tensor.
The evaluation of a first-order formula on a specific world is therefore different to the case in propositional logics, where the evaluation was a boolean in $\ozset$ indicating whether the world is a model.
%\red{Here we investigate grounding tensors to formulas with variables, and calculate them in a fixed world.}
% Arbitrary formulas

\subsect{Grounding tensors}

Given a first-order logic world $\dataworldwith$, arbitrary formulas are interpreted in terms of the satisfactions of their groundings.
We define their semantic first, and then relate their syntactical decomposition to tensor networks, similar to our approach to propositional logics in \charef{cha:logicalRepresentation}.

\begin{definition}[Grounding of a first-order formula given a world]
    Given a specific world $\dataworld$, with an domain $\worlddomain$ enumerated by $[\inddim]$, the grounding of a formula $\folexformula$ with variables $\indvariableof{\folexformula}$  is the tensor
    \begin{align*}
        \groundingofat{\folexformula}{\indvariableof{\folexformula}} :
        \bigtimes_{\indenumerator\in[\indvariableof{\folexformula}]} [\inddim] \rightarrow \ozset \, .
    \end{align*}
    Each coordinate represents thereby the boolean, whether the substitution of the variables in the formula is satisfied given a world $\dataworld$, that is
    \begin{align*}
        \groundingofat{\folexformula}{\indexedindvariableof{\folexformula}} = 1
    \end{align*}
    if and only if the substitution of $\folexformula$ with the variables $\indvariableof{\folexformula}$ replaced by the objects $\indexinterpretationat{\indindexof{\indenumerator}}$ is satisfied on the world $\dataworld$.
\end{definition}

% Comment: Formulas as maps to
The grounding tensor formalism can be used to define formulas as a map
\begin{align*}
    \folexformula : \left(\bigotimes_{\atomenumeratorin,\shortindindices\in[\inddim]^{\indorder}}\rr^{2}\right)
    \rightarrow \left(\bigotimes_{\atomenumeratorin,\indindexof{\folexformula}\in[\inddim]^{\cardof{\indvariableof{\folexformula}}}}\rr^{2}\right)
\end{align*}
where each world $\dataworld$ is mapped to a grounding tensor
\begin{align*}
    \folexformula(\dataworld) = \groundingof{\folexformula} \, .
\end{align*}
This would involve the factored level of tensor interpretation, namely representation of all possible worlds.

%% Basis encoding
%When interpreting this map as a basis encoding, formulas are tensors in the tensor space
%\begin{align*}
% 	\left(\bigotimes_{\atomenumeratorin, \indindexlist\in[\inddim]} \rr^{2} \right) \otimes
%	\left(  \bigotimes_{\atomenumeratorin, \indindexof{0},\ldots,\indindexof{\individualorder_{\folexformula}}\in[\inddim]} \rr^{2} \right) \, .
%\end{align*}

\subsect{Atomic Formulas}

Atomic formulas in first-order logic are predicates, which are applied on terms.%, that is constants or variables.
We restrict in this chapter to function-free logic, therefore terms are either constants or variables.
%The predicates itself are the simplest cases of first-order formulas with term variables.
% Atomic
If all arguments of a predicate are assigned by free variables, the corresponding grounding tensor is stored in the slices to the first axis of $\dataworld$ and we have
\begin{align}
    \groundingof{\folpredicateof{\folpredicateenumerator}} =
    \contractionof{\dataworldat{\selvariable,\shortindvariablelist},\onehotmapofat{\folpredicateenumerator}{\selvariable}}{\shortindvariablelist} \, .
\end{align}
In contrast, when a constant object $\indexinterpretationof{\indindex}$ is assigned to an argument of a predicate, the grounding tensor reduced to a slice of the grounding with exclusively free variables.
We capture such slicings by contractions with one-hot encodings of the corresponding constant.

We formalize this approach by atom creating tensors, which contraction with the world tensor results in the grounding of the corresponding atomic formula.

\begin{definition}\label{def:atomCreatingTensor}
    Let there be an atomic formula $\folexformula$, which is constructed using the $\selindex$-th predicate and has constants assigned on the arguments $\arbsetof{C}\subset[\indorder]$ and free variables to the arguments $\arbsetof{V}=[\indorder]/\arbsetof{C}$.
    Let the constant map $C: \arbsetof{C}\subset[\indorder] \rightarrow [\inddim]$ map to the specific objects represented by the constant and $V: \arbsetof{V}\subset[\indorder] \rightarrow \nodes$ to free variables labeled by a set $\nodes$.
    Then the atom creating tensor to $\folexformula$ is
    \begin{align*}
        \atomcreatorofat{\folexformula}{\indvariableof{V(\arbsetof{V})}}
        = \onehotmapofat{\selindex}{\selvariable} \otimes
        \left( \bigotimes_{\indenumerator\in\arbsetof{C}} \onehotmapofat{C(\indenumerator)}{\indvariableof{\indenumerator}} \right) \otimes
        \left( \bigotimes_{\indenumerator\in\arbsetof{V}} \identityat{\indvariableof{V(\indenumerator)},\indvariableof{\indenumerator}} \right) \, .
    \end{align*}
\end{definition}

The ground of the atom is then the contraction of the atom creating tensor with the world tensor, that is
\begin{align*}
    \groundingofat{\folexformula}{\indvariableof{V(\arbsetof{V})}}
    = \contractionof{\dataworldwith, \atomcreatorofat{\folexformula}{\indvariableof{\nodes}}}{\indvariableof{V(\arbsetof{V})}} \, .
\end{align*}


% Predicates as objects
What is more abstract, we can understand the predicate itself as an object, then take the first-order world as a grounding tensor of a more abstract formula.
We will follow this thought in the ternary representation of Knowledge Graphs in \secref{subsec:knowledgeGraphTernaryRep}.

%\subsect{Substitution by slicing}

% Slicing interpretation
%Slicing the grounding tensor of a formula a first-order formula amounts to substitution of the respective variable by the constant at the enumeration index.

%\subsect{Syntactical Decomposition of quantifier-free formulas}

\subsect{Formula synthesis by connectives}\label{sec:folConnectiveRepresentation}

In order to have a sound semantic, the grounding of FOL formulas is determined by the syntax of the formula, i.e. a decomposition of the formula into connectives and quantifiers acting on atomic formulas.

% Formulas as maps from worlds to groundings
Quantifier-free formulas are connectives acting on atomic formulas.
We can describe them as in the case of propositional logics in the $\bencodingof{}$-formalism.
While the atomic formulas where delta tensors copying states, they are more involved here.



\begin{theorem}
    For any connective $\exconnective$ and formulas $\folexformula_1$ and $\folexformula_2$ we have
    \begin{align}
        &\groundingofat{(\folexformula_1\exconnective\folexformula_2)}{\indvariableof{\folexformula_1}\cup\indvariableof{\folexformula_2}} \\
        &\quad=
        \contractionof{
            \bencodingofat{\groundingof{\folexformula_1}}{\headvariableof{\folexformula_1},\indvariableof{\folexformula_1}},
            \bencodingofat{\groundingof{\folexformula_2}}{\headvariableof{\folexformula_2},\indvariableof{\folexformula_2}},
            \bencodingofat{\exconnective}{\headvariableof{\folexformula_1\exconnective\folexformula_2}, \headvariableof{\folexformula_1}, \headvariableof{\folexformula_2}},
            \tbasisat{\headvariableof{\folexformula_1\exconnective\folexformula_2}}
        }
        {\shortindvariablelist} \, .
    \end{align}
\end{theorem}
\begin{proof}
    This directly follows from \theref{the:compositionByContraction}.
%	By the semantic interpretation of the groundings, which has to be sound.
\end{proof}

% Shared variables
Here, variables can be shared by the connected formulas, therefore the variables in the combined formula are unions of the possible not disjoint variables of the connected formulas.

%% Propositional interpretation
%When we understand the head variables in the basis encoding of atoms as the categorical variables, and get a similar interpretation of the tensor network decomposition as in the propositional case.
%\subsect{Propositionalization}

When interpreting the head variables of relational encoded atomic formulas as the atoms of a propositional theory, we find a propositional formula $\exformula$ associated with any decomposable first order logic formula.

\begin{definition}
    \label{def:propositionalEquivalent}
    Given a formula $\folexformula$ in first order logic, we say that a propositional formula $\formulaat{\shortcatvariables}$ is the propositional equivalent to $\folexformula$ given atomic formulas $\extformulaof{\atomenumerator}$ in first order logic, when for any world $\dataworld$ we have
    \begin{align*}
        \groundingofat{\folexformula}{\indvariableof{\folexformula}}
        = \contractionof{
            \{\bencodingofat{\groundingof{\extformulaof{\atomenumerator}}}{\catvariableof{\atomenumerator},\indvariableof{\extformulaof{\atomenumerator}}} : \atomenumeratorin\}
            \cup \{\formulaat{\shortcatvariables}\}
        }{\indvariableof{\folexformula}} \, .
    \end{align*}
    We here denote the head variables of the basis encoding to $\bencodingof{\groundingof{\extformulaof{\atomenumerator}}}$ by $\catvariableof{\atomenumerator}$ to highlight their interpretation as propositional atoms.
\end{definition}

We depict the relation of a grounding tensor to a propositional formula as:
\begin{center}
    \begin{tikzpicture}[scale=0.35, yscale=1, thick] % , baseline = -3.5pt



%\draw[] (2,-1) -- (2,1) node[midway,left] {\tiny $\atomicformulaof{0}$};
%\node[anchor=center] (text) at (4,0) {$\cdots$};
%\draw[] (6,-1) -- (6,1) node[midway,right] {\tiny $\atomicformulaof{\atomorder-1}$};

\draw (1,-1) rectangle (7,-3);
\node[anchor=center] (text) at (4,-2) {$\groundingof{\folexformula}$};

\draw[] (2,-3) -- (2,-5) node[midway,left] {\tiny $\individualvariableof{0}$};
\node[anchor=center] (text) at (4,-4) {$\cdots$};
\draw[] (6,-3) -- (6,-5) node[midway,right] {\tiny $\individualvariableof{\individualorder-1}$};


\node[anchor=center] (text) at (10,-2) {${=}$};



%\draw (1,-1) rectangle (7,-3);
%\node[anchor=center] (text) at (4,-2) {$\rencodingof{\impformula}$};

\begin{scope}[shift={(12,-2)}]

\draw (1,3) rectangle (12,5);
\node[anchor=center] (text) at (6.5,4) {$\exformula$};

\draw[->] (2.5,1) -- (2.5,3) node[midway,right] {\tiny $\atomicformulaof{0}$};
\draw (1,-1) rectangle (4,1);
\node[anchor=center] (text) at (2.5,0) {$\rencodingof{\groundingof{\extformulaof{0}}}$};
\node[anchor=center] (text) at (2.5,-2) {$\cdots$};

\node[anchor=center] (text) at (6.5,0) {$\cdots$};

\draw[->] (10.5,1) -- (10.5,3) node[midway,right] {\tiny $\atomicformulaof{\atomorder-1}$};
\draw (8.75,-1) rectangle (12.25,1);
\node[anchor=center] (text) at (10.5,0) {$\rencodingof{\groundingof{\extformulaof{\atomorder\shortminus1}}}$};
\node[anchor=center] (text) at (10.5,-2) {$\cdots$};

\draw[-] (11.5,-3) -- (3.5,-3) ;
\draw[-] (9.5,-5) -- (1.5,-5) ;

\draw[fill] (11.5,-3) circle (0.25cm);
\draw[->] (11.5,-3) -- (11.5,-1);

\draw[fill] (9.5,-5) circle (0.25cm);
\draw[->] (9.5,-5) -- (9.5,-1);

\draw[fill] (3.5,-3) circle (0.25cm);
\draw[->] (3.5,-3) -- (3.5,-1);

\draw[fill] (1.5,-5) circle (0.25cm);
\draw[->] (1.5,-5) -- (1.5,-1);


\draw[fill] (7.5,-3) circle (0.25cm);
\draw[<-] (7.5,-3) -- (7.5,-7) node[right] {\tiny $\individualvariableof{\individualorder-1}$} ;

\node[anchor=center] (text) at (6.5,-6) {$\cdots$};

\draw[fill] (5.5,-5) circle (0.25cm);
\draw[<-] (5.5,-5) -- (5.5,-7) node[left] {\tiny $\individualvariableof{0}$} ;


%\draw (13,-2) rectangle (15,-6);
%\node[anchor=center] (text) at (14,-4) {$\rencodingof{\impformula}$};
%\node[anchor=center] (text) at (12,-3.75) {$\vdots$};
%\draw[->] (15,-4) -- (16,-4);
%\draw[] (17,-4) -- (16,-4);
%\draw (17,-3) rectangle (19,-5);
%\node[anchor=center] (text) at (18,-4) {$\tbasis$};

\end{scope}




		


\end{tikzpicture}
\end{center}


\subsect{Quantifiers}

Existential and universal quantifiers appear in generic first order logic and are besides substitutions further means to reduce the number of variables in a formula.
%They are not representable as linear transform of the respective quantifier-free formula.


% Definition of existential and universal quantifiction needed!
The semantics of existential quantification consists in a formula being true, if at least one state of the quantified variable is true, as we define next.

\begin{definition}
    Given a grounding tensor
    \begin{align*}
        \groundingofat{\folexformula}{\indvariableof{0},\ldots,\indvariableof{\indorder-1}} \,
    \end{align*}
    the existential and universal quantification with respect to the first variable are the tensors
    \begin{align*}
        \groundingofat{\left(\exists_{\indindexof{0}}\folexformula\right)}{\indvariableof{1},\ldots,\indvariableof{\indorder-1}} \quad \text{and} \quad
        \groundingofat{\left(\forall_{\indindexof{0}}\folexformula\right)}{\indvariableof{1},\ldots,\indvariableof{\indorder-1}} \,
    \end{align*}
    with coordinates as follows.
    For an assignment $\indindexof{1},\ldots,\indindex$ to the non-quantified variables we have
    \begin{align*}
        \groundingofat{\left(\exists_{\indindexof{0}}\folexformula\right)}{\indexedindvariableof{1},\ldots,\indexedindvariableof{\indorder-1}} = 1
    \end{align*}
    if and only if there is an assignment $\indindexofin{0}$ such that
    \begin{align*}
        \groundingofat{\folexformula}{\indexedindvariableof{0},\indexedindvariableof{1},\ldots,\indexedindvariableof{\indorder-1}} = 1 \, .
    \end{align*}
    Conversely, we have for the universal quantification that
    \begin{align*}
        \groundingofat{\left(\forall_{\indindexof{0}}\folexformula\right)}{\indexedindvariableof{1},\ldots,\indexedindvariableof{\indorder-1}} = 1
    \end{align*}
    if and only if for any assignment $\indindexofin{0}$ we have
    \begin{align*}
        \groundingofat{\folexformula}{\indexedindvariableof{0},\indexedindvariableof{1},\ldots,\indexedindvariableof{\indorder-1}} = 1 \, .
    \end{align*}
\end{definition}


Let us now show, that existential and universal quantification are coordinatewise transforms (see \defref{def:coordinatewiseTransform}) of contracted grounding tensors.
To this end, let us introduce the greater-$z$ indicator $\greaterthanfunction{z}$, where $z\in\rr$, as the function
\begin{align*}
    \greaterthanfunction : \rr \rightarrow \ozset
    \quad, \quad \greaterthanfunctionof{z}{x} =
    \begin{cases}
        1 & \quad  \text{if} \quad x > z\\
        0 & else
    \end{cases} \, .
\end{align*}

\begin{theorem}
    For any formula $\folexformula$ with variables $\shortindvariablelist$ we have
    \begin{align*}
        \groundingofat{\left(\exists{\indindexof{0}}\folexformula\right)}{\indvariableof{1},\ldots,\indvariableof{\indorder-1}} =
        \coordinatetrafowrtofat{\existquanttrafo}{\contractionof{\groundingof{\folexformula}}{\indvariableof{1},\ldots,\indvariableof{\indorder-1}}}{\indvariableof{1},\ldots,\indvariableof{\indorder-1}}
    \end{align*}
    and
    \begin{align*}
        \groundingofat{\left(\forall{{\indindexof{0}}} \folexformula\right)}{\indvariableof{1},\ldots,\indvariableof{\indorder-1}}=
        \coordinatetrafowrtofat{\universalquanttrafo}{\contractionof{\groundingof{\folexformula}}{\indvariableof{1},\ldots,\indvariableof{\indorder-1}}}{\indvariableof{1},\ldots,\indvariableof{\indorder-1}}
    \end{align*}
\end{theorem}
\begin{proof}
    We proof the claimed equalities to arbitrary slices of the remaining variables, which amount to arbitrary substitutions of the formulas.
    For any indices $\indindexofin{1},\ldots,\indindexofin{\indorder-1}$ we notice, that
    \begin{align*}
        \sbcontractionof{\groundingof{\folexformula}}{\indexedindvariableof{1},\ldots,\indexedindvariableof{\indorder-1}}
        &= \sum_{\indindexofin{0}} \groundingofat{\folexformula}{\indexedindvariableof{0},\ldots,\indexedindvariableof{\indorder-1}} \\
        &= \cardof{\indindexofin{0} \, : \, \groundingofat{\folexformula}{\indexedindvariableof{0},\ldots,\indexedindvariableof{\indorder-1}}=1} \, .
    \end{align*}
    We can thus understand the contracted grounding tensor as storing in its coordinates the count of the coordinate extensions to the zeroth variable, such that the grounding tensor is satisfied.
    This is analogous to our interpretation of contracted propositional formulas as world counts.
    From this it is obvious, that the existential quantification is satisfied, if the count is different from zero, which is captured by the coordinatewise transform with $\existquanttrafo$.
    We therefore arrive at
    \begin{align*}
        \groundingofat{\left(\exists_{\indindexof{0}}\folexformula\right)}{\indexedindvariableof{1},\ldots,\indexedindvariableof{\indorder-1}} =
        \coordinatetrafowrtofat{\existquanttrafo}{\contractionof{\groundingof{\folexformula}}{\indvariableof{1},\ldots,\indvariableof{\indorder-1}}}{\indexedindvariableof{1},\ldots,\indexedindvariableof{\indorder-1}} \, .
    \end{align*}
    The first claim follows, since the assignment to the non-quantified variables was arbitrary.
    The universal quantification is satisfied, when all extensions are satisfied, and the count is $\inddim$.
    Since $\inddim$ is the maximal count, this is captured by the coordinatewise transform with $\universalquanttrafo$ and we get
    \begin{align*}
        \groundingofat{\left(\forall{\indindexof{0}}\folexformula\right)}{\indexedindvariableof{1},\ldots,\indexedindvariableof{\indorder-1}} =
        \coordinatetrafowrtofat{\universalquanttrafo}{\contractionof{\groundingof{\folexformula}}{\indvariableof{1},\ldots,\indvariableof{\indorder-1}}}{\indexedindvariableof{1},\ldots,\indexedindvariableof{\indorder-1}} \, .
    \end{align*}
    With the same argument, the second claim is established.
\end{proof}

% Customized quantifiers
We can extend this discussion towards more generic counting quantifiers, of which the existential and the universal quantifier are extreme cases.
One can define quantifiers by demanding that at least $z\in\nn$ compatible groundings are satisfied, and show that they amount to coordinatewise transforms with $\greaterthanfunction{z}$.
What is more, quantifiers demanding that at most $z\in\nn$ are satisfied would be representable by transforms with an analogously defined function $\ones_{\leq z}$.
Such customized quantifiers appear for example in the $\mathrm{OWL\,2}$ standard of description logics (see \cite{rudolph_foundations_2011} and \secref{sec:kgRepresentation}).

% basis encodings
As will be discussed in \charef{cha:basisCalculus}, any coordinatewise transform can be performed by a contraction of a basis encoding of the tensor with a head vector prepared by the transform function (see \theref{the:tensorFunctionComposition}).
In the case here, a direct implementation would require a dimension of these head variables by $\inddim$, which can be infeasible when having large object sets.

% Prenex
To summarize, let us assume a formula is in its prenex normal form, that is a collection of quantifiers are acting on a qantifier free part.
We can represent its grounding tensor by
\begin{itemize}
    \item Instantiations of the tom groundings with the assigned variables, as contractions of the basis encoding of the world tensor with atom selecting tensors.
    \item Propositional formula acting on the head variables of the predicate instantiations, representing the connectives combining the formula.
    \item Quantifiers as a composition of contractions closing the quantified variable and coordinatewise transforms with the respective greater-than indicators.
\end{itemize}



\subsect{Storage in basis CP decomposition}\label{sec:basisCPgrounding}

In many situations, grounding cores are sparse and representations as single tensor cores comes with a drastic overhead.
We often encounter sparse grounding tensors, where the number of non-zero coordinates (to be investigated by basis CP ranks in \charef{cha:sparseCalculus}) satisfies
\begin{align*}
    \sparsityof{\groundingof{\folexformula}} << \inddim^{\cardof{\indvariableof{\folexformula}}} \, .
\end{align*}
In this case, since most coordinates vanish, the basis CP decomposition (see \secref{sec:basisCP}) enables a representation of the grounding with significantly lower storage demand, see \theref{the:sparseBasisCP}.
This is particularly useful for representing large relational databases, where each object has only a few relations with others, while the majority of possible relations remains unsatisfied.
We depict such CP decomposition of a formula grounding in \theref{fig:groundingCP}.

% Standard KB Encoding and Assumptions
Most logical syntaxes exploit $\ell_0$-sparsity, explicitly storing only known assertions.
The interpretation of unspecified assertions depends on the underlying assumptions.
Under the Closed World Assumption, for example, all unspecified assertions are assumed to be false.

\begin{figure}[h]
    \begin{center}
        \begin{tikzpicture}[scale=0.35, yscale=-1, thick] % , baseline = -3.5pt





%\drawatomindices{0}{2}


\draw (-1,1) rectangle (5,-1);
\node[anchor=center] (text) at (2,0) {$\groundingof{\exformula}$};

%\draw[->] (2,-1) -- (2,-3) node[midway, right] {\tiny $\atomlegindexof{\exformula}$};
%\draw[] (3,-3) rectangle (1, -5);
%\node[anchor=center] (text) at (2,-4) {\small $\tbasis$};
%\draw[->] (4,-1) -- (4,-3) node[midway, right] {\tiny $\datindex$};
%\draw[dashed] (3,-3) rectangle (5, -5);
%\node[anchor=center] (text) at (4,-4) {\small $\ones$};

\draw[] (0,1) -- (0,3) node[midway,left] {\tiny $\exindividualof{0}$};
\draw[] (1.5,1) -- (1.5,3) node[midway,left] {\tiny $\exindividualof{1}$};
\node[anchor=center] (text) at (2.75,2) {$\cdots$};
\draw[] (4,1) -- (4,3) node[midway,right] {\tiny $\exindividualof{\variableorder\shortminus1}$};

\node[anchor=center] (text) at (7,0) {${=}$};


\begin{scope}[shift={(10,2)}]

\newcommand{\conposseldec}{4.5,-5.5}

\draw[fill] (\conposseldec) circle (0.25cm);
\draw (\conposseldec) -- (4.5,-7.5) node[midway, right] {\tiny $\datindex$};
\draw[] (3.5,-7.5) rectangle (5.5, -9.5);
\node[anchor=center] (text) at (4.5,-8.5) {\small $\ones$};

\draw[<-] (0,1) -- (0,-1) node[midway,left] {\tiny $\exindividualof{0}$};
\draw (-1,-1) rectangle (1, -3);
\node[anchor=center] (text) at (0,-2) {\small $\legcoreof{\exformula,0}$};
\draw[<-] (0,-3) to[bend right=20] (\conposseldec);


\draw[<-] (3,1) -- (3,-1) node[midway,left] {\tiny $\exindividualof{1}$};
\draw (2,-1) rectangle (4, -3);
\node[anchor=center] (text) at (3,-2) {\small $\legcoreof{\exformula,1}$};
\draw[<-] (3,-3) to[bend right=20]  (\conposseldec);

\node[anchor=center] (text) at (6,-2) {$\cdots$};

\draw[<-] (9,1) -- (9,-1) node[midway,left] {\tiny $\exindividualof{\variableorder\shortminus1}$};
\draw (8,-1) rectangle (10, -3);
\node[anchor=center] (text) at (9,-2) {\small $\legcoreof{\exformula,\variableorder\shortminus1}$};
\draw[<-] (9,-3) to[bend left=20]  (\conposseldec);




\end{scope}

		


\end{tikzpicture}
    \end{center}
    \caption{Basis CP Decomposition of the grounding of $\folexformula$, following the scheme of \theref{the:sparseBasisCP}.
    Instead of direct storage of the grounding tensor $\groundingof{\folexformula}$, the non-zero coordinates are enumerated by a variable $\datvariable$ and the corresponding coordinates stored in leg-matrices $\legcoreof{\folexformula,\indenumerator}$.}
    \label{fig:groundingCP}
\end{figure}

\subsect{Queries}

A database is understood as a specific fist order logic world, and are operations on such a single world.
Queries are described by a formula $\impformula$, which are asked against a specific world $\dataworld$ to retrieve the grounding $\groundingof{\impformula}$.
The variables of such formulas are called projection variables.
The answer $\groundingof{\impformula}$ of a query is most conveniently represented as a list of solution mappings from the projection variables to objects in the world, such that the query formula is satisfied.
Answering a query by solution mappings corresponds with finding the basis CP Decomposition (see \secref{sec:basisCP}) of $\groundingof{\impformula}$.
We can understand these solution mappings as stored in the leg-matrices $\legcoreof{\folexformula,\indenumerator}$ (see \figref{fig:gorundingCP}).

Let us give with the outer join an example of a popular operation to define queries, which efficient execution and storage can be improved based on considerations in the tensor network formalism.

\begin{definition}[Outer join]
    Let there be a world $\dataworld$ and formulas $\extformulaof{\selindex}$ depending on variables $\indvariableof{\nodesof{\selindex}}$, which have grounding tensors by
    \begin{align*}
        \groundingofat{\extformulaof{\selindex}}{\indvariableof{\node}} \, : \,  \bigtimes_{\node\in\nodesof{\selindex}}[\inddimof{\node}] \rightarrow \ozset \, .
    \end{align*}
    Then their (outer) $\joinsymbol$ is defined as the grounding of their conjunctions, as
    \begin{align*}
        \groundingofat{\joinsymbol\left(\extformulaof{0},\ldots,\extformulaof{\seldim-1}\right)}{\bigcup_{\selindexin}\indvariableof{\nodesof{\selindex}}}
        = \contractionof{\groundingofat{\extformulaof{\selindex}}{\indvariableof{\nodesof{\selindex}}}\,:\,\selindexin}{\bigcup_{l\in[p]}\indvariableof{\nodesof{\selindex}}} \, .
    \end{align*}
\end{definition}

%Visualization and efficiency
We can understand the $\joinsymbol$ of groundings by a factor graph, where each grounding tensor decorates the hyperedge to the node set $\nodesof{\selindex}$.
The projection variable assignment to each formula combined in a $\joinsymbol$ operation provide a basic tensor network format to store the output of the operation.
There are thus situations, in which the solution map storage corresponding with a CP Decomposition comes with unnecessary overheads compared with other formats.

% Coordinatewise transform
We can also understand the $\joinsymbol$ operation as a coordinatewise transform (see \defref{def:coordinatewiseTransform}) with the product as transform function.
To make this connection solid, one would need to extend each joined formula trivially to the variables appearing in other formulas.

% Evaluation similar constraint propagation
The efficiency of evaluating the contraction to a $\joinsymbol$ operation might be improved by understanding it as an Constraint Satisfaction Problem (see \charef{cha:logicalReasoning}).
When applying efficient Message Passing algorithms such as Knowledge Propagation (see \algoref{alg:knowledgePropagation}), the groundings can be sparsified by local constraint propagation operations before turning to more global and more demanding contraction operations.
Here the groundings $\groundingof{\extformulaof{\selindex}}$ would be used to initialize Knowledge Cores $\kcoreof{\edge}$ and sequentially sparsified during the algorithm.

%\begin{example} % WOULD NEED OVERWORK: DRAW!
%	For example take a query with many basic graph patterns with pairwise different projection variables.
%	The global CP Decomposition would come here with an exponential storage overhead compared with storage as a tensor product of CP Decompositions to each Basic graph pattern.
%\end{example}

%% CONFUSING?
%\begin{remark}[Distinguishing from probabilistic queries]
%	Let us distinguish the discussion here from those of queries in probabilistic reasoning, which have two main differences.
%	First, we ask queries against all possible pairs of variables, instead of asking the probability of satisfaction of a specific formula.
%	Second, since we made the epistemologic assumption of knowing possibilities and not probabilities in logics, a query is answered by a truth value.
%	We then only output in the shape of solution mappings the variable assignments where the query formula is true.
% 	Thus, the queries here can be thought of as a batch of probabilistic queries with Boolean answers.
%	% Alternative -> Later?
%	Probabilistic queries can furthermore be understood in terms of the data extraction process described in this section.
%	We can ask the query in probabilistic form (decomposed into atomic formulas) on the resulting empirical distribution.
%	This results in the ratio of the worlds satisfying the query among those worlds satisfying the extraction query $\impformula$.
%\end{remark}


\sect{Representation of Knowledge Graphs}\label{sec:kgRepresentation}

Let us now represent a specific fragment of first-order logic, namely Description Logics which Knowledge Bases are often refered to as Knowledge Graphs.
We here use the $\mathrm{OWL\,2}$ standard, which encodes the syntax of the description logic $\mathcal{SROIQ(D)}$ \cite{rudolph_foundations_2011}.

\subsect{Representation as unary and binary predicates}

% Reduction to binary
Predicates in knowledge graphs are binary (owl:ObjectProperties) and unary (owl:Class).
%Larger formulas are created by logical connections of these atomic formulas using disjunctions, conjunctions etc.
We enumerate the predicates by $[\folpredicateorder]$, the objects in the domain $\worlddomain$ by $[\inddim]$, and extend the unary predicates to binaries by tensor product with $\onehotmapofat{0}{\indvariableof{1}}$.
A Knowledge Graph on the set $\worlddomain$ of constants (owl:NamedIndividuals) is then the tensor
\begin{align*}
    \kgat{\selvariable,\indvariableof{0},\indvariableof{1}} : [\folpredicateorder] \times [\inddim] \times [\inddim] \rightarrow \ozset \, .
\end{align*}


\subsect{Representation as ternary predicate}\label{subsec:knowledgeGraphTernaryRep}

It has been particulary convenient to represent a Knowledge Graph instead as a grounding of a single ternary predicate $\rdf$.
To this end, the predicates $\folpredicateof{\catenumerator}$ and another object $\mathrdftype$ are added to a domain $\worlddomain$, by extending the $\inddim$ and the index interpretation function accordingly.


% RDF triple: Alternative viewpoint to collection of unary and binary predicates!
Following our notation we understand a Knowledge Graph as a grounding of the rdf triple relation $\rdf$ (being a formula of order 3) on a specific world $\kg$ with individuals $\worlddomain$

We then construct a grounding tensor $\kggroundingof{\rdf}$ out of the world $\kgat{\selvariable,\indvariableof{0},\indvariableof{1}}$ by
\begin{align*}
    \kggroundingof{\rdf} : [\inddim] \times [\inddim] \times [\inddim] \rightarrow \ozset
\end{align*}
where
\begin{align*}
    &\kggroundingofat{\rdf}{\indexedindvariableof{s}, \indexedindvariableof{p}, \indexedindvariableof{o}} \\
    &\quad =
    \begin{cases}
        \kgat{\selvariable=\indindexof{s},\indvariableof{0}=\indindexof{o},\indvariableof{1}=0}
        & \text{if} \quad \indindexof{p} = \invindexinterpretationat{\mathrdftype} \\
        \kgat{\selvariable=\indindexof{p},\indvariableof{0}=\indindexof{s},\indvariableof{1}=\indindexof{o}}
        & \text{if} \quad \indindexof{p} = \invindexinterpretationat{\folpredicateof{\catenumerator}} \quad \text{for some} \quad \catenumerator \\
        0  \quad & \text{else}
    \end{cases} \, .
\end{align*}


Slicing the tensor $\kggroundingof{\rdf}$ along the predicate axis retrieves specific information about roles and can be efficiently be performed on these formats.
The role $\mathrdftype$ has a specific meaning, since it contains from a DL perspective classifications (memberships of named concepts).
Further slicing the tensor along object axis therefore results in membership lists for specific classes (concepts).
One can thus regard $\mathrdftype$ as a placeholder for unitary formulas in a space of binary formulas.

% Triple Stores, sparsity
Exploiting the $\ell_0$-sparsity now leads to a so-called triple store, where $\kggroundingof{\rdf}$ is stored by a listing of those triples $\indindexof{\subsymbol},\indindexof{\predsymbol},\indindexof{\objsymbol}$ such that $\kggroundingofat{\rdf}{\indexedindvariableof{s}, \indexedindvariableof{p}, \indexedindvariableof{o}}=1$
A recent implementation of a triple store exploiting these intuitions is $\mathrm{TENTRIS}$, see \cite{pan_tentris_2020}.
In this work, such decompositions are generalized into more generic CP formats, see \charef{cha:sparseCalculus}.
% Approximation of KG Groundings
Approximations of grounding tensors by decompositions leads to embeddings of the individuals such as $\mathrm{Tucker}$, $\mathrm{ComplEx}$ and $\mathrm{RESCAL}$ (see \cite{nickel_review_2016}).

% Sparse representation
%Sparse representation of the grounding tensor to a knowledge graph is of central importance, as investigated in \cite{pan_tentris_2020}.
%We here do basis CP for sparse representation.


% basis encoding
For our purposes of evaluating logical formulas such as $\sparql$ queries we use the basis encoding of the groundings, which are depicted by
\begin{center}
    \begin{tikzpicture}[scale=0.3, thick] % , baseline = -3.5pt

    \draw[->-] (0,1)--(0,3) node[midway,left] {\tiny $\headvariable$};
    \draw (-3,1) rectangle (3,-1);
    \node[anchor=center] (text) at (0,0) {\small $\rencodingof{\kggroundingof{\rdf}}$};
    \draw[-<-] (-2,-1)--(-2,-3) node[midway,left] {\tiny $\sindvariable$};
    \draw[-<-] (0,-1)--(0,-3) node[midway,left] {\tiny $\pindvariable$};
    \draw[-<-] (2,-1)--(2,-3) node[midway,left] {\tiny $\oindvariable$};

\end{tikzpicture}
\end{center}




\subsect{$\sparql$ Queries}

The $\sparql$ query language is a syntax to express first-order logic formulas $\folexformula$ and intended to be evaluated given a Knowledge Graph.
We here consider tensor network representations of the $\mathrm{WHERE}{\cdot}$ block.
Given a specific knowledge graph $\kggroundingof{\rdf}$, the execution of query is the interpretation $\groundingof{\folexformula}$, typical represented in a sparse basis CP format where each slice represents a solution mapping.

\subsubsect{Triple Patterns}

\red{Central to $\sparql$ queries are triple patterns, which we understand as slicings of the tensor $\kggroundingof{\rdf}$.}
To each so-called triple pattern we build a corresponding atom creating tensor (see \defref{def:atomCreatingTensor}).
The triple pattern is then evaluated by contraction of the atom creating tensor with $\kggroundingof{\rdf}$.

Let us now provide examples of such pattern tensors.
A unary triple patterns contains a single projection variable, typically related with the subject variable $\sindvariable$ of $\kggroundingof{\rdf}$.
The corresponding pattern tensor is then
\begin{align*}
    \atomcreatorofat{\kgtriple{\provariable}{\mathrdftype}{\folpredicateof{\catenumerator}}}{
        \sindvariable, \pindvariable, \oindvariable, \provariable
    }
    = \identityat{\sindvariable,\provariable}
    \otimes \onehotmapofat{\invindexinterpretationat{\mathrdftype}}{\pindvariable}
    \otimes \onehotmapofat{\invindexinterpretationat{\folpredicateof{\atomenumerator}}}{\oindvariable} \, .
\end{align*}

Binary triple patterns come with two projection variables, typically related with the subject and the object variables $\sindvariable$ and $\oindvariable$.
The pattern tensor to the $\catenumerator$-th predicate is then
\begin{align*}
    \atomcreatorofat{\kgtriple{\provariableof{0}}{\folpredicateof{\catenumerator}}{\provariableof{1}}}{
        \sindvariable, \pindvariable, \oindvariable, \provariableof{0}, \provariableof{1}
    }
    = \identityat{\sindvariable,\provariableof{0}}
    \otimes \onehotmapofat{\invindexinterpretationat{\folpredicateof{\atomenumerator}}}{\pindvariable}
    \otimes \identityat{\oindvariable,\provariableof{1}} \, .
\end{align*}

Contraction with these pattern tensor evaluated the specific triple pattern, and outputs in a boolean tensor the indicator, which objects are members of a specific class (for unary patterns) or which pair of objects are related by a specific relation.
Again, the output of such contractions is a subset encodings of the set of solutions (see \defref{def:subsetEncoding}).

%%%%%%%%%%%% END OF FRIDAY 14.3.
%%%%%%%%%%%%

% Examples
Examples of triple patterns, drawn in \figref{fig:triplePatterns} are
\begin{itemize}
    \item Unary triple pattern with one variable, representing a formula with a single projection variable.
    For the example $\exunarytriple$ see Figure~\ref{fig:triplePatterns}a.
    \begin{align*}
        \atomcreatorofat{\kgtriple{\provariable}{\mathrdftype}{\folpredicateof{\catenumerator}}}{
            \sindvariable, \pindvariable, \oindvariable, \provariable
        }
        = \identityat{\sindvariable,\provariable}
        \otimes \onehotmapofat{\invindexinterpretationat{\mathrdftype}}{\pindvariable}
        \otimes \onehotmapofat{\invindexinterpretationat{\exaunaryrelation}}{\oindvariable}
    \end{align*}
    If and only if the output slice is $\tbasis$, then the corresponding object encoded by the input indices is of class $\exaunaryrelation$.
    \item Binary triple pattern with two variables, representing a formula with two projection variables.
    For the example  $\exbinarytriple$ see Figure~\ref{fig:triplePatterns}b.
    If and only if the output slice is $\tbasis$, then the corresponding object tuple encoded by the input indices has a relation $\exabinaryrelation$.
\end{itemize}

% Projection picture
The composition $\psi (\psi^T)$ of the matrification of the tensor $\psi$ is an orthogonal projection.
That means that applying $\psi (\psi^T)$ is the same map as applying once.


\begin{figure}[h]
    \begin{center}
        \begin{tikzpicture}[scale=0.3,thick] % , baseline = -3.5pt

    \begin{scope}
        [shift={(0,0)}]

        \node[anchor=center] (text) at (-12,2) {$a)$};

        \begin{scope}
            [shift={(-7,2)}]

            \draw (0,-3) rectangle (-6,-5);
            \draw[<-] (-3,-1)--(-3,-3) node[midway,right] {\tiny $\headvariable$};
            \node[anchor=center] (text) at (-3,-4) {$\rencodingof{\kggroundingof{\exunarytriple}}$};
            \draw[<-] (-3,-5)--(-3,-7) node[midway,left] {\tiny $\provariableof{0}$};

        \end{scope}

        \node[anchor=center] (text) at (-5.5,-2) {${=}$};

        \draw[->] (0,1)--(0,3) node[midway,left] {\tiny $\headvariable$};
        \draw (-4,1) rectangle (4,-1);
        \node[anchor=center] (text) at (0,0) {\small $\rencodingof{\kggroundingof{\rdf}}$};

        \draw (-2,-3) rectangle (-4,-5);
        \draw[<-] (-3,-1)--(-3,-3) node[midway,left] {\tiny $\sindvariable$};
        \node[anchor=center] (text) at (-3,-4) {$\delta$};
        \draw[<-] (-3,-5)--(-3,-7) node[midway,left] {\tiny $\provariableof{0}$};

        \draw (-1,-3) rectangle (1,-5);
        \draw[<-] (0,-1)--(0,-3) node[midway,left] {\tiny $\pindvariable$};
        \node[anchor=center] (text) at (0,-4) {$\onehotmapof{\invrdftypesymbol}$};

        \draw (2,-3) rectangle (4,-5);
        \draw[<-] (3,-1)--(3,-3) node[midway,left] {\tiny $\oindvariable$};
        \node[anchor=center] (text) at (3,-4) {$\onehotmapof{\exaunaryrelation}$};

    \end{scope}


    \begin{scope}
        [shift={(24,0)}]

        \node[anchor=center] (text) at (-13,2) {$b)$};

        \begin{scope}
            [shift={(-8,2)}]

            \draw (0.5,-3) rectangle (-6.5,-5);
            \draw[<-] (-3,-1)--(-3,-3) node[midway,right] {\tiny $\headvariable$};
            \node[anchor=center] (text) at (-3,-4) {$\rencodingof{\kggroundingof{\exbinarytriple}}$};

            \draw[<-] (-2,-5)--(-2,-7) node[midway,right] {\tiny $\provariableof{0}$};
            \draw[<-] (-4,-5)--(-4,-7) node[midway,left] {\tiny $\provariableof{1}$};

        \end{scope}

        \node[anchor=center] (text) at (-5.5,-2) {${=}$};

        \draw[->] (0,1)--(0,3) node[midway,left] {\tiny $\headvariable$};
        \draw (-4,1) rectangle (4,-1);
        \node[anchor=center] (text) at (0,0) {\small $\rencodingof{\kggroundingof{\rdf}}$};

        \draw (-2,-3) rectangle (-4,-5);
        \draw[<-] (-3,-1)--(-3,-3) node[midway,left] {\tiny $\sindvariable$};
        \node[anchor=center] (text) at (-3,-4) {$\delta$};
        \draw[<-] (-3,-5)--(-3,-7) node[midway,left] {\tiny $\provariableof{1}$};

        \draw (-1,-3) rectangle (1,-5);
        \draw[<-] (0,-1)--(0,-3) node[midway,left] {\tiny $\pindvariable$};
        \node[anchor=center] (text) at (0,-4) {$\onehotmapof{\exabinaryrelation}$};

        \draw (2,-3) rectangle (4,-5);
        \draw[<-] (3,-1)--(3,-3) node[midway,left] {\tiny $\oindvariable$};
        \node[anchor=center] (text) at (3,-4) {$\delta$};
        \draw[<-] (3,-5)--(3,-7) node[midway,right] {\tiny $\provariableof{0}$};

    \end{scope}

\end{tikzpicture}
    \end{center}
    \caption{Triple patterns of $\sparql$ as tensor networks.
    a) Example of unary triple pattern $\exunarytriple$ specifying whether an individual $\indexinterpretationof{\indindexof{1}}$ is a member of class $C$.
    %Here by $0$ we denote the element $\invindexinterpretationat{\mathrdftype}$
        b) Example of a binary triple pattern $\exbinarytriple$ specifying whether individuals $\indexinterpretationof{\indindexof{1}}$ and $\indexinterpretationof{\indindexof{2}}$ have a relation $R$.
        By $\onehotmapof{\invrdftypesymbol},\onehotmapof{\exaunaryrelation},\onehotmapof{\exabinaryrelation}$ we denote the one-hot encodings of the enumeration of the resources $rdf:type, C$ and $R$.
    }
    \label{fig:triplePatterns}
\end{figure}




\subsubsect{Basic Graph Patterns}

Generic $\sparql$ queries are compositions of triple patterns by logical connectives. % Except for some stuff like regex
These triple patterns possibly share projection variables.
Statements in $\sparql$ can be translated into Propositional Logics combining the triple patterns:
\begin{center}
    \begin{tabular}{|c|c|c|}
        \hline
        \textbf{$\sparql$}                & \textbf{Propositional Logics} & \textbf{Tensor Representation}                                                                   \\
        \hline
        $\{f_1, f_2\}$                    & $f_1\land f_2$                & $\bencodingofat{\land}{\headvariableof{f_1\land f_2},\headvariableof{f_1},\headvariableof{f_2}}$ \\
        \hline
        $\mathrm{UNION}\{f_1, f_2\} $     & $f_1\lor f_2$                 & $\bencodingofat{\lor}{\headvariableof{f_1\lor f_2},\headvariableof{f_1},\headvariableof{f_2}}$   \\
        \hline
        $\mathrm{FILTER}\,\,\mathrm{NOT}\,\,\mathrm{EXISTS}\{f\}$ & $\lnot f$                     & $\bencodingofat{\lnot}{\headvariableof{\lnot f},\headvariableof{f}}$                             \\
        \hline
    \end{tabular}
\end{center}

If a $\sparql$ query consists of these keywords, we find a straight forward corresponding network of triple patterns and encoded logical connectives, by applying our findings of \secref{sec:folConnectiveRepresentation}.
To this end, we prepare for each appearing triple pattern the corresponding pattern tensor, and a copy of $\kggroundingof{\rdf}$.
Here we also copy the term variables $\sindvariable,\pindvariable$ and $\oindvariable$, to ensure that each copy of $\kggroundingof{\rdf}$ shares variables with a single pattern tensor.
Projection variables are not copied, since we need to keep track of them shared among triple patterns.
Then we prepare the basis encoding of logical connectives according to the hierarchy specified in the $\sparql$ query.
Finally we add a $\tbasis$-vector to the final head variable representing the complete $\sparql$ query, to restrict the support to coordiantes corresponding with solution mappings.
We then contract the resulting tensor network, leaving all projection variables open.

If a projection variable is not appearing in the $\mathrm{SELECT}$ statement in front of the $\mathrm{WHERE}\{\cdot\}$-block, we simply exclude it from the open variables of the described contraction.
Note that in that case, the coordinates contain solution counts, i.e. how many assignments to the dropped variable have been a $1$ coordinate.
We can drop this additional information simply by performing a coordinatewise transform with the greater zero indicator $\existquanttrafo$.

% Effective calculus alternative
Here we represented a $\sparql$ query $\impformula$ consistent of multiple triple pattern by instantiating a head variables to each triple pattern.
Alternatively, the more direct hybrid calculus developed in \secref{sec:hybridCalculus} can be applied and the additional head variables avoided.
This is especially compelling, when the $\mathrm{WHERE}\{\cdot\}$-block does not contain further keywords, i.e. it is the conjunction of all triple patterns.
In that case, we avoid the instantiation of head variables (i.e. close the head variables separately by $\tbasis$-vectors) and represent the query by a contraction of all triple pattern tensors.

% Expressivity
We further notice, that any propositional formula acting on the head variables of the triple patterns can be expressed by a hierarchical combination of the key words in the above table.
To find the expression, one can transform a given formula into its conjunctive or disjunctive normal form and apply the statements according to the apperaing operations $\land,\lor$ and $\lnot$.


%% Further $\sparql$ features
%Further $\sparql$ features, which cannot be expressed by a tensor network are:
%\begin{itemize}
%    \item $\mathrm{FILTER}\{\cdot\}$ does not depend on triple patterns (e.g. numeric inequalities, regex functions on strings).
%    We can regard it as another basic formula, which does not result from a slicing of the $\rdf$ grounding tensor.
%    Besides that, we can understand it as formulas and include it in compositions.
%    \item $\mathrm{OPTIONAL}\{\cdot\}$ would result in $\ones$ leg vectors, when there is a missing variable assignment resulting.
%\end{itemize}



\sect{Probabilistic Relational Models}

% MLN in FOL and PL
So far we have studied Markov Logic Networks in Propositional Logics as probability distributions over worlds.
In FOL they define probability distributions over relations in worlds with a fixed set of objects.
More generally, such models are probabilistic relational models (see for an overview \cite{getoor_introduction_2019}.


We in this section treat random worlds in first-order logics with fixed domains $\worlddomain$.

%
We in this section show, when and how we can interpret likelihoods of Markov Logic Networks in First Order Logic in terms of samples of a Markov Logic Network in Propositional Logics.

\subsect{Hybrid First-Order Logic Networks}

% Templates
Following \cite{richardson_markov_2006} Markov Logic Networks in first-order logics are templates for distributions, which instantiate random worlds when choosing a set of objects $\worlddomain$.
Given a fixed set of constants, they then define a distribution over the worlds, which objects correspond with the constants. % this is database semantics!
This applies database semantics, where only those worlds are considered, where the unique name and domain closure assumptions given a set of constants are satisfied.
\red{Here we directly define them as exponential families distributing $\randworld$ for a given set of objects $\worlddomain$.}
\red{To avoid a similar discussion as in \charef{cha:networkRepresentation} we directly allow for boolean base measures and call the distributions Hybrid First-Order Logic Networks.}

\begin{definition}[Hybrid First-Order Logic Networks (HFLN)]
%    A Markov Logic Network is a template of probability distributions defined
    Let there be a set $\folformulaset$ of first-order logic formulas with maximal arity $\individualorder$, which is enumerated by a selection variable $\selvariable$ of dimension $\seldim$.
    Further, let there be a set of objects $\worlddomain$ and a boolean base measure $\basemeasureat{\shortindvariables}$.
    The family of Hybrid First-Order Logic Networks $\expfamilyof{\restfolformulaset,\basemeasure}$ defined by the tuple $(\folformulaset,\worlddomain,\basemeasure)$ is the exponential family of joint distributions to the variables $\randworld$ with the statistics
    \begin{align*}
        \sstat^{\restfolformulaset}_{\selindex}\left[\indexedrandworld\right]
        = \sbcontraction{\groundingof{\enumfolformula}}
    \end{align*}
    and the base measure $\basemeasure$.
%    The Markov Logic Network instantiated for a given set of objects $\worlddomain$ and a base measure $\basemeasure$ is the random world, which is a member of the exponential family with sufficient statistics
%    \begin{align*}
%        \sstatcoordinateofat{\selindex}{\indexedrandworld} = \sbcontraction{\groundingof{\enumfolformula}}
%        %\sstat_{\selindex}(\dataworld)  = \sbcontraction{\groundingof{\folexformula_\selindex}} % Formulas can have different
%    \end{align*}
%    and canonical parameters $\weight$.
\end{definition}

Each element of the family $\expfamilyof{\restfolformulaset,\basemeasure}$ is represented by a canonical parameter $\canparamat{\selvariable}$.

The mean parameter polytope is the convex hull of the vectors
\begin{align*}
    \sencodingofat{\folformulaset}{\indexedrandworld,\selvariable}
\end{align*}
to the worlds $\dataworld$ with $\basemeasureat{\indexedrandworld}=1$.
These vectors store are the counts of satisfied groundings to each formula, that is
\begin{align*}
    \sencodingofat{\folformulaset}{\indexedrandworld,\selvariable} = \cardof{
        \indindexof{\enumfolformula} \, : \, \groundingofat{\enumfolformula}{\indexedindvariableof{\enumfolformula}} = 1
    } \, .
\end{align*}
Each substitution of the variables in $\enumfolformula$ by objects in $\worlddomain$, which satisfies the formula in the world $\dataworld$, therefore provides a factor of $\expof{\canparamat{\indexedselvariable}}$ to the probability of $\dataworld$.

Let us notice, that different to the case of Hybrid Logic Networks treated in \charef{cha:networkRepresentation}, the statistic does not consist of boolean features, when formulas contain variables and we have multiple objects.
One could, however, replace each $\enumfolformula$ by the set of the possible groundings, i.e. substitutions of the formulas variables by any tuple of objects in $\worlddomain$.
The resulting distribution would be an Hybrid Logic Network with boolean statistic, which coincides with the HFLN when posing certain weight sharing conditions on $\canparam$.
The downside of this construction is the increase in the number of features from $\seldim$ to $\sum_{\selindexin} \cardof{\worlddomain}^{\cardof{\indvariableof{\enumfolformula}}}$.
This polynomial in the cardinality of the domain set increase poses significant computational challenges, see \cite{richardson_markov_2006}.
We will in the next sections explore an alternative way to apply the theory of \charef{cha:networkRepresentation} and \charef{cha:networkReasoning}, namely based on importance formulas.


%% Interpretation
%The statistics
%\begin{align*}
%    \sbcontraction{\groundingof{\folexformula_\selindex}} % Formulas can have different
%\end{align*}
%can be interpreted as the number of substitutions to a formula, such that the formula ist satisfied.
%Each substitution satisfying a formula adds a factor of $\expof{\canparam_\selindex}$ to the probability of the respective world before normalization.


%
%When constructing a world tensor to a theory with predicates of different order, we already argued that we extend the arity of predicates by tensor products with $\onehotmapof{0}$.
%To define random world tensors, we then restrict the corresponding base measure to be supported only on those worlds where the extended predicates hold only at the individual $\exindividualof{0}$ at the extended axis.


% Comparison with PL MLN
%We choose extraction formulas $\extformulaof{\atomenumerator}$ such that any formula in the FOL MLN has a propositional equivalent (see \defref{def:propositionalEquivalent}).
%The statistic map is then a formula selecting tensor as in the propositional logic case contracted with the groundings of $\extformulaof{\atomenumerator}$.






\subsect{Base measures by importance formulas}

%\red{Analogous to a guard formula in \cite[Definition 6.11]{koller_probabilistic_2009}!}

The boolean base measure $\basemeasure$ of a Hybrid First-Order Logic Network is the subset encoding of the possible worlds which have a non-vanishing probability with respect to any member of the family.
We now construct specific base measures based on a fixed grounding tensor of an importance formula.
This will reduce the number of object tuples influencing the probability distribution in order to arrive at an interpretation of FOL MLNs as likelihoods to datasets of propositional MLNs.

To this end, we mark pairs of term indices relevant to the distributions by an auxiliary index $\datindexin$.
Given a set $\{\indindexof{[\indorder]}^{\datindex} \, : \, \datindexin \}$ of indices to the important tuples we build a set encoding (see \defref{def:subsetEncoding})
\begin{align*}
    \fixedimpformula = \sum_{\datindexin} \left(
    \bigotimes_{\indenumeratorin} \onehotmapofat{\indindexof{\indenumerator}^{\datindex}}{\indvariableof{\indenumerator}}
    \right) \, .
\end{align*}

% Interpretation as grounding
We interpret the tensor $\fixedimpformula$ as the grounding of a formula, which we call the importance formula.

% Restricting to worlds with identical grounding
To have a constant importance formula we define a syntactic representation and restrict the support of the HFLN to those world coinciding with groundings of the importance formula coinciding with $\fixedimpformula$ by designing a base measure
\begin{align*}
    \fixedimpbm
    = \begin{cases}
          1 & \text{if} \quad \groundingofat{\impformula}{\indvariableof{\impformula}} = \fixedimpformula \\
          0 & \text{else}
    \end{cases} \, .
\end{align*}

% Conditioning on exquery
The base measure restricts the HFLN to be those worlds, where $\groundingof{\impformula}$ is coincides with the fixed tensor $\fixedimpformula$.
Intuitively, $\groundingof{\impformula}$ represents certain evidence about a first-order logic world, whereas other formulas are uncertain.


\begin{assumption}
    \label{ass:importanceBasemeasure}
    Given a base measure $\fixedimpbm$, we assume that there is an importance formula $\impformulaat{\shortindvariables}$ such that
    \begin{align*}
        \fixedimpbm
        = \begin{cases}
              1 & \text{if} \quad \groundingofat{\impformula}{\indvariableof{\impformula}} = \fixedimpformula \\
              0 & \text{else}
        \end{cases} \, .
    \end{align*}
\end{assumption}


\subsect{Decomposition of the log likelihood}


% Extraction query
To reduce the likelihood of a world to we make the assumption that all formulas in a HFLN are of the form
\begin{align}
    \label{eq:folImplicationForm}
%    \folexformula_{\selindex}(\individuals) =
    \enumfolformulaat{\indvariableof{\enumfolformula}}
    = \left( \impformulaat{\shortindvariables} \Rightarrow \headfolformulaofat{\selindex}{\indvariableof{\enumfolformula}} \right)
\end{align}
that is a rule with the importance formula being the premise.
In particular, we assume, that they depend on all term variables $\shortindvariables$.
If this is not the case, we extend the formula trivially on the missing term variables.
When this assumption holds, we can think of the importance formula as a conditions on individuals to satisfy a statistical relation given by $\headfolexformula$.

Towards connecting with propositional logics, we further make the assumption, that we can decompose the formula $\headfolformulaof{\selindex}$ in what we will call extraction formulas.

\begin{assumption}
    \label{ass:propositionalHeads}
    We assume that there exist formulas $\{\extformulaofat{\catenumerator}{\shortindvariables} \, : \, \catenumeratorin\}$, which we refer to as atom extraction formulas, and an importance formula $\impformulaat{\shortindvariables}$ such that the following holds.
    To each first-order logic formula $\enumfolformula$ there is another first-order logic formula $\headfolformulaofat{\selindex}{\indvariableof{\enumfolformula}}$ and a propositional formula $\enumformulaat{\shortcatvariables}$ such that
    \begin{align*}
        \enumfolformulaat{\indvariableof{\enumfolformula}}
        = \left( \impformulaat{\shortindvariables} \Rightarrow \headfolformulaofat{\selindex}{\indvariableof{\enumfolformula}} \right)
    \end{align*}
    and
    \begin{align*}
        \headfolformulaofat{\selindex}{\indvariableof{\enumfolformula}} =
        \contractionof{
            \{\enumformulaat{\shortcatvariables}\} \cup \{\bencodingofat{\extformulaof{\catenumerator}}{\catvariableof{\catenumerator},\shortindvariables} \, : \, \catenumeratorin\}
        }{\indvariableof{\enumformula}} \, .
    \end{align*}
\end{assumption}

We depict the assumption, that any formula is of the form \eqref{eq:folImplicationForm} in the diagram
\begin{center}
    \begin{tikzpicture}[scale=0.35, yscale=1, thick] % , baseline = -3.5pt




\draw (1,-1) rectangle (7,-3);
\node[anchor=center] (text) at (4,-2) {$\groundingof{\left(\impformula\Rightarrow\folexformula\right)}$};

\draw[] (2,-3) -- (2,-5) node[midway,left] {\tiny $\individualvariableof{0}$};
\node[anchor=center] (text) at (4,-4) {$\cdots$};
\draw[] (6,-3) -- (6,-5) node[midway,right] {\tiny $\individualvariableof{\individualorder-1}$};


\node[anchor=center] (text) at (10,-2) {${=}$};



%\draw (1,-1) rectangle (7,-3);
%\node[anchor=center] (text) at (4,-2) {$\rencodingof{\impformula}$};

\begin{scope}[shift={(12,-2)}]

\draw (1,3) rectangle (12,5);
\node[anchor=center] (text) at (6.5,4) {$\exformula$};

\draw[->] (2.5,1) -- (2.5,3) node[midway,right] {\tiny $\atomicformulaof{0}$};
\draw (1,-1) rectangle (4,1);
\node[anchor=center] (text) at (2.5,0) {$\rencodingof{\groundingof{\extformulaof{0}}}$};
\node[anchor=center] (text) at (2.5,-2) {$\cdots$};

\node[anchor=center] (text) at (6.5,0) {$\cdots$};

\draw[->] (10.5,1) -- (10.5,3) node[midway,right] {\tiny $\atomicformulaof{\atomorder-1}$};
\draw (8.75,-1) rectangle (12.25,1);
\node[anchor=center] (text) at (10.5,0) {$\rencodingof{\groundingof{\extformulaof{\atomorder\shortminus1}}}$};
\node[anchor=center] (text) at (10.5,-2) {$\cdots$};

\draw[<-] (13,-3) -- (3.5,-3) ;
\draw[<-] (13,-5) -- (1.5,-5) ;

\draw[fill] (11.5,-3) circle (0.25cm);
\draw[->] (11.5,-3) -- (11.5,-1);

\draw[fill] (9.5,-5) circle (0.25cm);
\draw[->] (9.5,-5) -- (9.5,-1);

\draw[fill] (3.5,-3) circle (0.25cm);
\draw[->] (3.5,-3) -- (3.5,-1);

\draw[fill] (1.5,-5) circle (0.25cm);
\draw[->] (1.5,-5) -- (1.5,-1);


\draw[fill] (7.5,-3) circle (0.25cm);
\draw[<-] (7.5,-3) -- (7.5,-7) node[right] {\tiny $\individualvariableof{\individualorder-1}$} ;

\node[anchor=center] (text) at (6.5,-6) {$\cdots$};

\draw[fill] (5.5,-5) circle (0.25cm);
\draw[<-] (5.5,-5) -- (5.5,-7) node[left] {\tiny $\individualvariableof{0}$} ;


\draw (13,-2) rectangle (15,-6);
\node[anchor=center] (text) at (14,-4) {$\rencodingof{\impformula}$};
\node[anchor=center] (text) at (12,-3.75) {$\vdots$};
\draw[->] (15,-4) -- (16,-4);
\draw[] (17,-4) -- (16,-4);
\draw (17,-3) rectangle (19,-5);
\node[anchor=center] (text) at (18,-4) {$\tbasis$};

\end{scope}




\node[anchor=center] (text) at (30,-2) {${+}$};





\begin{scope}[shift={(22,1)}]

\draw (13,0) rectangle (15,2);
\node[anchor=center] (text) at (14,1) {$\fbasis$};

\draw[] (14,-1) --(14,0);
\draw[->] (14,-2) --(14,-1);

\draw (12,-2) rectangle (16,-4);
\node[anchor=center] (text) at (14,-3) {$\rencodingof{\impformula}$};

\draw[<-] (12.5,-4) -- (12.5,-6) node[left] {\tiny $\individualvariableof{0}$} ;
\node[anchor=center] (text) at (14,-5) {$\cdots$};
\draw[<-] (15.5,-4) -- (15.5,-6) node[right] {\tiny $\individualvariableof{\individualorder\shortminus1}$} ;


		
\end{scope}

\end{tikzpicture}
\end{center}
where the second summand depends only on the query $\impformula$ and therefore does not appear in the likelihood.


%\begin{example}[Trivial importance formula]
%	When the importance formula is always satisfied, any tuple of objects contributes to the likelihood. 
%	This original approach to Markov Logic Networks \cite{richardson_markov_2006} however leads to many datapoints which are also dependent on each other.
%\end{example}


% Define probability
Let us now show, how to decompose the probability of a first-order logic world to a HFLN under the above assumptions.
Given a HFLN $\probof{\folmlnparameters}$, the probability of a world $\dataworld$ with $\groundingof{\impformula}=\fixedimpformula$ is % and $\groundingof{\folpredicateof{\folpredicateenumerator}} \prec \fixedimpformula$ as
\begin{align*}
    \probofat{\folmlnparameters}{\indexedrandworld}
    = \frac{1}{\partitionfunctionof{\folmlnparameters}}
    %\expof{\sum_{\folexformulain}\weightof{\folexformula}\contraction{\groundingof{(\impformula\Rightarrow\headfolexformula)}}}
    \expof{\sum_{\selindexin}\canparamat{\indexedselvariable}\contraction{\groundingof{(\impformula\Rightarrow\headfolexformula)}}}
\end{align*}
where the partition function is
\begin{align*}
    \partitionfunctionof{\folmlnparameters} =
    \sum_{\supportedworlds}
    \expof{\sum_{\selindexin}\canparamat{\indexedselvariable}\contraction{\groundingof{(\impformula\Rightarrow\headfolexformula)}}} \, .
    %\expof{\sum_{\folexformulain}  \weightof{\folexformula}  \sum_{\indindexlist\in[\inddim]} \groundingofat{(\impformula\Rightarrow\headfolexformula)}{\indexedshortindvariables} } }
    %\prod_{\individuals\in\worlddomain} \left(\prod_{\folexformulain} \expof{\weightof{\folexformula}(\impformula\Rightarrow\folexformula)(\individuals)} \right)\, .
\end{align*}


Let us now decompose the statistics into constant and varying terms.
We have
\begin{align*}
    \contraction{\groundingof{(\impformula\Rightarrow\headfolexformula)}} =
    \contraction{\groundingof{\impformula\land\headfolexformula}} + \contraction{\groundingof{\lnot\impformula}} \, ,
\end{align*}
where the the second term is constant among the supported worlds and the first can be enumerated by the satisfied substitutions of $\impformula$, that is
\begin{align*}
    \contraction{\groundingof{\impformula\land\headfolexformula}}
    = \sum_{\datindexin}\groundingofat{\headfolexformula}{\indvariableof{[\indorder]} = \indindexof{[\indorder]}^{\datindex}} \, .
\end{align*}


Using these insights we decompose a normalized log likelihood as
\begin{align}
    \label{eq:dataworldLogProb}
    \frac{1}{\datanum} \lnof{\probofat{\folmlnparameters}{\indexedrandworld}}
    = & \frac{1}{\datanum} \sum_{\datindexin} \sum_{\selindexin} \canparamat{\indexedselvariable}
    \groundingofat{\headfolexformula}{\shortindindices = \indindexof{[\indorder]}^{\datindex}} \\
    & - \frac{1}{\datanum} \lnof{
        \frac{\partitionfunctionof{\folmlnparameters}}{
            \expof{\contraction{\weight} \cdot \contraction{\groundingof{\lnot\impformula}}}
        }
    }
%	\sum_{\individuals\in\worlddomain \, : \, \impformula(\individuals)=1} \left(\sum_{\exformula\in\formulaset} \weightof{\exformula} \exformula(\individuals)\right) -  \frac{1}{\datanum}\lnof{\secpartitionfunctionof{\folformulaset,\weight,\worlddomain,\fixedimpformula}} \, . 
\end{align}

% Data identification
We notice a similarity with the likelihood in the case of MLNs in propositional logic.
When we interpret each tuple $\shortindindices\in(\worlddomain)^{\indorder}$ satisfying $\impformulaat{\indexedshortindvariables}=1$ as a datapoint, and choose the formulas
\begin{align*}
    \formulaset = \{\headfolformulaof{\selindex} \, : \, \selindexin \}
\end{align*}
from the propositional equivalents to the formulas $\folformulaset$, the fist term in \eqref{eq:dataworldLogProb} coincides with the first term of the likelihood
\begin{align*}
    \centropyof{\empdistribution}{\expdistof{(\formulaset,\canparam,\ones)}}
    = \dataaverage \sum_{\selindexin} \canparamat{\indexedselvariable} \enumformulaat{\datapoint} - \lnof{\partitionfunctionof{\mlnparameters}}
\end{align*}

However, the partition function couples multiple samples, with possible couplings, and prevents a straight forward interpretation as an empirical dataset.
We in the next section present assumptions on the tuples satisfying $\impformula$, which lead to a factorization of the partition function.

% Partition function simplification
%\subsect{Interpretation as Likelihood of Propositional Dataset}
\subsect{Decomposition of the Partition function}


We now make additional assumptions to decompose the partition function of an HFLN as a product of HLN partition functions.

\begin{assumption}
    \label{ass:independentTuples}
    Let $\fixedimpbm$ be a base measure of worlds such that the vectors
    \begin{align}
        \label{eq:data}
        \left(  \groundingofatwrt{\extformulaof{0}}{\indvariableof{0}=\indindexof{0}^\datindex,\ldots,\indvariableof{\indorder-1}=\indindexof{\indorder-1}^\datindex}{\tilde{\dataworld}}, \ldots,
        \groundingofatwrt{\extformulaof{\atomorder-1}}{\indvariableof{0}=\indindexof{0}^\datindex,\ldots,\indvariableof{\indorder-1}=\indindexof{\indorder-1}^\datindex}{\tilde{\dataworld}}
        \right)
    \end{align}
    for $\datindexin$ are independent and identical distributed by the normation of a boolean base measure $\atombasemeasure$, when drawing $\randworld$ by $\fixedimpbm$.
\end{assumption}

When these assumption holds, we now show that the probability of a first-order logic world coincides with the likelihood of a propositional logic dataset.

\begin{theorem}
    \label{the:FOLworldToPLdataset}
    Let there be a set of formulas $\folformulaset$ and a base measure $\fixedimpbm$ such that \assref{ass:importanceBasemeasure}, \assref{ass:propositionalHeads} and \assref{ass:independentTuples} hold.
    We then have for the likelihood of any by $\fixedimpbm$ supported world $\dataworld$ that
    \begin{align*}
        \frac{1}{\datanum} \lnof{\probofat{\folmlnparameters}{\indexedrandworld}}
        = \centropyof{\empdistribution}{\expdistof{\mlnparameters}}
    \end{align*}
    where $\formulaset$ is the set of propositional equivalents to $\folformulaset$ (see \assref{ass:propositionalHeads}) and $\datamap$ the data map with evaluation at $\datindexin$ by the enumerated non-vanishing coordinates of $\fixedimpformulawith$
    \begin{align*}
        \datapoint
        = \big( \groundingofat{\extformulaof{0}}{\indvariableof{0}=\indindexof{0}^\datindex,\ldots,\indvariableof{\indorder-1}=\indindexof{\indorder-1}^\datindex}, \ldots ,
        \groundingofat{\extformulaof{\atomorder-1}}{\indvariableof{0}=\indindexof{0}^\datindex,\ldots,\indvariableof{\indorder-1}=\indindexof{\indorder-1}^\datindex} \big) \, .
    \end{align*}
\end{theorem}

To show the theorem, we show first in the following lemma the factorization of the partition function of the HFLN.

\begin{lemma}
    \label{lem:FOLpartitionfunctionfactorization}
    Given the assumptions of \theref{the:FOLworldToPLdataset}, we have
    \begin{align*}
        \frac{\partitionfunctionof{\folmlnparameters}}{\expof{\contraction{\canparam} \cdot \contraction{\groundingof{\lnot\impformula}}}}
        = \left(\partitionfunctionof{\mlnparameters,\atombasemeasure}\right)^\datanum \, .
    \end{align*}
\end{lemma}
\begin{proof}
    We have
    \begin{align*}
        \partitionfunctionof{\folmlnparameters}
        &= \expectationofwrt{
            \expof{\sum_{\folexformulain}\weightof{\folexformula}\contraction{\groundingof{(\impformula\Rightarrow\headfolexformula)}}}
        }{\dataworld\sim\fixedimpbm} \\
        &= \expof{\contraction{\weight} \cdot \contraction{\groundingof{\lnot\impformula}}} \cdot
        \expectationofwrt{
            \expof{\sum_{\folexformulain}\weightof{\folexformula}  \sum_{\datindexin} \groundingofat{\headfolexformula}{\datshortindvariables} }
        }{\dataworld\sim\fixedimpbm} \\
        &= \expof{\contraction{\weight} \cdot \contraction{\groundingof{\lnot\impformula}}} \cdot
        \expectationofwrt{
            \prod_{\datindexin} \expof{ \sum_{\folexformulain} \weightof{\folexformula} \cdot \groundingofat{\headfolexformula}{\datshortindvariables} }
        }{\dataworld\sim\fixedimpbm}
    \end{align*}
    Since the substitutions of the atom formulas at the respective object tuples are independent, also the variables
    \begin{align*}
        \expof{\weightof{\folexformula}  \cdot \groundingofat{\headfolexformula}{\datshortindvariables}  }
    \end{align*}
    for $\datindexin$ are independent.
    We therefore get
    \begin{align}
        \label{eq:independentSamplesFOL}
        \partitionfunctionof{\folmlnparameters}
        &= \expof{\contraction{\weight} \cdot \contraction{\groundingof{\lnot\impformula}}} \cdot
        \prod_{\datindexin}
        \expectationofwrt{
            \expof{ \sum_{\folexformulain} \weightof{\folexformula} \cdot \groundingofat{\headfolexformula}{\datshortindvariables} }
        }{\dataworld\sim\fixedimpbm}
    \end{align}
    Each $\groundingofat{\headfolexformula}{\datshortindvariables}$ depends by \assref{ass:propositionalHeads} only on the random tuple $\{\extformulaof{\atomenumerator}[\datshortindvariables] \, : \, \catenumeratorin\}$.
    We build the expectation over all possible values $\shortcatindices$ of this tuple at any $\datindexin$ and get
    \begin{align*}
        & \expectationofwrt{
            \expof{\sum_{\selindexin} \canparamat{\indexedselvariable} \cdot \groundingofat{\headfolformulaof{\selindex}}{\datshortindvariables}}
        }{\dataworld\sim\fixedimpbm} \\
        & \quad = \sum_{\shortcatindices\in\atomstates}
        \probofwrt{\forall{\catenumeratorin} \, : \, \extformulaof{\atomenumerator}[\datshortindvariables] =\catindexof{\atomenumerator}}{\dataworld\sim\fixedimpbm}
        \cdot \expof{\sum_{\selindexin}\canparamat{\indexedselvariable}\cdot\enumformulaat{\indexedshortcatvariables}} \\
        & = \sum_{\shortcatindices\in\atomstates} \atombasemeasure[\indexedshortcatvariables] \cdot
        \expof{\sum_{\selindexin} \canparamat{\indexedselvariable} \cdot \enumformulaat{\indexedshortcatvariables}}
        \\
        & = \partitionfunctionof{\mlnparameters,\atombasemeasure} \, .
    \end{align*}
    We arrive at the claim, when combining this equation with \eqref{eq:independentSamplesFOL}.
\end{proof}

With this lemma, we are now show \theref{the:FOLworldToPLdataset}.

\begin{proof}[Proof of \theref{the:FOLworldToPLdataset}]
    %Use Equation \ref{eq:dataworldLogProb} and the \lemref{lem:FOLpartitionfunctionfactorization}.
    We have for the logarithm of the probability of a world $\dataworld$ given the distribution $\probof{\folmlnparameters}$, that
    \begin{align*}
        \lnof{\probofat{\folmlnparameters}{\indexedrandworld}}
        =  \sum_{\selindexin} \canparamat{\indexedselvariable} \contraction{\groundingof{\enumfolformula}} - \lnof{\partitionfunctionof{\folmlnparameters}}
    \end{align*}
    The first term obeys with \assref{ass:propositionalHeads}
    \begin{align*}
        \sum_{\selindexin} \canparamat{\indexedselvariable} \contraction{\groundingof{\enumfolformula}}
        &=  \contraction{\canparam} \cdot \contraction{\groundingof{\lnot\impformula}}
        + \sum_{\selindexin}\sum_{\datindexin} \canparamat{\indexedselvariable} \cdot \headfolformulaofat{\selindex}{\datshortindvariables} \\
        &= \contraction{\canparam} \cdot \contraction{\groundingof{\lnot\impformula}}
        + \sum_{\selindexin}\sum_{\datindexin} \canparamat{\indexedselvariable} \cdot \enumformulaat{\datshortcatvariables} \, .
    \end{align*}
    With \lemref{lem:FOLpartitionfunctionfactorization} we have under the given assumptions for the second term
    \begin{align*}
        \lnof{\partitionfunctionof{\folmlnparameters}} = \datanum \cdot \lnof{\partitionfunctionof{\mlnparameters,\atombasemeasure}}  + \contraction{\canparam} \cdot \contraction{\groundingof{\lnot\impformula}} \, .
    \end{align*}
    Combining both, we have
    \begin{align*}
        \frac{1}{\datanum} \lnof{\probofat{\folmlnparameters}{\indexedrandworld}}
        =  \dataaverage\sum_{\selindexin} \canparamat{\indexedselvariable} \cdot \enumformulaat{\datshortcatvariables}  - \lnof{\partitionfunctionof{\mlnparameters,\atombasemeasure}}
    \end{align*}
    which coincides with $\centropyof{\empdistribution}{\expdistof{\mlnparameters}}$.
%    Given \assref{ass:propositionalHeads} we can
%    Note that we need to correct the likelihood by the averalge log basemeasure on the data, since that term is appearing in the likelihood of a MLN.
\end{proof}

%%%%%%%%%%%%%%%%%%%%%%%
%%%% END of Monday 17.3.
%%%%%%%%%%%%%%%%%%%%%%

% Independent data investigation
Let us now investigate, in which cases the \assref{ass:independentTuples} of independent data can be matched.

\begin{lemma}
    Let $\impformula$ and $\extformulas$ be quantor and constant free and let the index tuples of the support of $\fixedimpformula$ be pairwise disjoint.
    Then the vectors \eqref{eq:data} are pairwise independent.
\end{lemma}
\begin{proof}
    Then we can reduce each sample as dependent only on an independent random world with domain by the respective objects.
    Quantor and constant-free is needed that this reductions is possible.
\end{proof}


There are situations, where \assref{ass:independentTuples} is violated.
For example, when two object tuples are not disjoint, then some formulas might always coincide on both datapoints, which would violate independence.

In further situations the atom base $\atombasemeasure$ are not the uniform $\ones$:
\begin{itemize}
    \item extraction formula being a) conjunctions of predicates: Probability that they are satisfied decreases
    b) disjunctions of predicates: Probability that they are satisfied increases
    \item extraction formula coinciding with importance formula: Always satisfied, in this case still boolean
    \item extraction formulas contradicting each other, more general not independent from each other
\end{itemize}

Let us notice, that non-boolean base measures could be treated in a same manner, but several developments in this work, such as cross-entropy decompositions in \charef{cha:probReasoning} would receive further terms.



\begin{remark}[Approximation by Independent Samples]
    As observed above, we do not have independent samples in general.
    As a consequence, we cannot apply \lemref{lem:FOLpartitionfunctionfactorization} to decompose the partition function term of the log-probability into factors to each solution map of $\impformula$.
    In this case, it might be still benefitial to use the reduction to the likelihood of a HLN, but needs to understand it as a approximation to the true world probability.

    %
    If the expectations of each sample with respect to the marginalized distributions coincide, the average of empirical distribution also coincides with these (by linearity).
    When the creation of samples has sufficient mixing properties, the empirical distribution converges to this expectation in the asymptotic case of large numbers of samples.

\end{remark}



\sect{Sample extraction from first-order logic worlds}

The decomposition of the likelihood suggests the following approach to generate samples from groundings:
%We propose the following approach to generate datacores from groundings:
\begin{itemize}
    \item Define a query formula $\impformula$, which we decompose in the basis CP decomposition and interpret each slice as the one-hot encoding of the datapoint.
    \item Define for $\atomenumeratorin$ queries $\extformulaof{\atomenumerator}$ generating the the atoms $\atomicformulaof{\atomenumerator}$:
    Predicates along with assignment of variables / constants to its positions.
    \item Contract the groundings of each formula $\extformulaof{\atomenumerator}$ with the grounding of $\impformula$ to build a data core
\end{itemize}


\subsect{Representation by Tensor Networks}

We model the extraction process as a relation between a tuple of individuals and the extracted world in the factored system of atoms $\catvariableof{\atomenumerator}$.

\begin{definition}
    \label{def:extractionRelation}
    Given a first-order logic world $\dataworld$, an importance formula $\impformula$ and extraction formulas $\extformulaof{\catenumerator}$ for $\catenumeratorin$, we define the extraction relation
    \begin{align*}
        \extractionrelation \subset \left(\symindstates\right) \otimes \left(\atomstates\right)
    \end{align*}
    by
    \begin{align*}
        \extractionrelation
        = \{ (\shortindindices, \shortcatindices)
        \, : \,  \groundingofat{\impformula}{\indexedshortindvariables} = 1 \, , \, \forall {\catenumeratorin} : \,  \catindexof{\atomenumerator} = \extformulaofat{\atomenumerator}{\indexedshortindvariables} \} \, .
    \end{align*}
\end{definition}

The encoding of an extraction relation is
\begin{align*}
    \bencodingofat{\extractionrelation}{\shortindvariables,\shortcatvariables} \subset \left(\indspace\right) \otimes \left(\atomspace\right) \,
\end{align*}
and drawn in a contraction diagram by
\begin{center}
    \begin{tikzpicture}[scale=0.35, yscale=1, thick] % , baseline = -3.5pt


    \draw[] (2,-1) -- (2,1) node[midway,left] {\tiny $\catvariableof{0}$};
    \node[anchor=center] (text) at (4,0) {$\cdots$};
    \draw[] (6,-1) -- (6,1) node[midway,right] {\tiny $\catvariableof{\atomorder-1}$};

    \draw (1,-1) rectangle (7,-3);
    \node[anchor=center] (text) at (4,-2) {$\bencodingof{\extractionrelation}$};

    \draw[] (2,-3) -- (2,-5) node[midway,left] {\tiny $\individualvariableof{0}$};
    \node[anchor=center] (text) at (4,-4) {$\cdots$};
    \draw[] (6,-3) -- (6,-5) node[midway,right] {\tiny $\individualvariableof{\individualorder-1}$};


    \node[anchor=center] (text) at (10,-2) {${=}$};


    \begin{scope}
        [shift={(12,0)}]

        \draw[->-] (2.5,1) -- (2.5,3) node[midway,right] {\tiny $\catvariableof{0}$};
        \draw (1,-1) rectangle (4,1);
        \node[anchor=center] (text) at (2.5,0) {$\bencodingof{\groundingof{\extformulaof{0}}}$};
        \node[anchor=center] (text) at (2.5,-2) {$\cdots$};

        \node[anchor=center] (text) at (6.5,0) {$\cdots$};

        \draw[->-] (10.5,1) -- (10.5,3) node[midway,right] {\tiny $\catvariableof{\atomorder-1}$};
        \draw (8.75,-1) rectangle (12.25,1);
        \node[anchor=center] (text) at (10.5,0) {$\bencodingof{\groundingof{\extformulaof{\atomorder\shortminus1}}}$};
        \node[anchor=center] (text) at (10.5,-2) {$\cdots$};

        \draw[-<-] (13,-3) -- (3.5,-3) ;
        \draw[-<-] (13,-5) -- (1.5,-5) ;

        \drawvariabledot{11.5}{-3}
        \draw[->-] (11.5,-3) -- (11.5,-1);

        \drawvariabledot{9.5}{-5}
        \draw[->-] (9.5,-5) -- (9.5,-1);

        \drawvariabledot{3.5}{-3}
        \draw[->-] (3.5,-3) -- (3.5,-1);

        \drawvariabledot{1.5}{-5}
        \draw[->-] (1.5,-5) -- (1.5,-1);

        \drawvariabledot{7.5}{-3}
        \draw[-<-] (7.5,-3) -- (7.5,-7) node[right] {\tiny $\individualvariableof{\individualorder-1}$} ;

        \node[anchor=center] (text) at (6.5,-6) {$\cdots$};

        \drawvariabledot{5.5}{-5}
        \draw[-<-] (5.5,-5) -- (5.5,-7) node[left] {\tiny $\individualvariableof{0}$} ;


        \draw (13,-2) rectangle (17,-6);
        \node[anchor=center] (text) at (15,-4) {$\bencodingof{\groundingof{\impformula}}$};
        \node[anchor=center] (text) at (12,-3.75) {$\vdots$};
        \draw[->-] (17,-4) -- (18,-4);
        \drawvariabledot{18}{-4}
        \draw[] (18,-4) -- (19,-4);
        \draw (19,-3) rectangle (21,-5);
        \node[anchor=center] (text) at (20,-4) {$\tbasis$};

    \end{scope}


\end{tikzpicture}
\end{center}
Here the contraction of $\bencodingof{\impformula}$ with the truth vector $\tbasis$ represents the matching condition posed by $\impformula$ when extracting pairs of individuals.


%% Empirical Distribution
The empirical distribution is then the normalized contraction leaving only the legs to the extracted atomic formulas open, that is
\begin{align*}
    \empdistribution
    = \frac{
        \sbcontractionof{\bencodingof{\extractionrelation}}{\shortcatvariables}
    }{
        \sbcontraction{\bencodingof{\extractionrelation}}
    }  \, .
\end{align*}
Here the number of extracted data is the denominator
\begin{align*}
    \datanum
    = \contraction{\bencodingof{\extractionrelation}}
    = \contraction{\bencodingofat{\impformula}{\headvariableof{\impformula},\shortindvariables},\tbasisat{\headvariableof{\impformula}}}\, .
\end{align*}

We depict this by
\begin{center}
    
\begin{tikzpicture}[scale=0.35, yscale=1, thick] % , baseline = -3.5pt


    \draw[->] (2,-1) -- (2,1) node[midway,left] {\tiny $\catvariableof{0}$};
    \node[anchor=center] (text) at (4,0) {$\cdots$};
    \draw[->] (6,-1) -- (6,1) node[midway,right] {\tiny $\catvariableof{\atomorder-1}$};

    \draw (1,-1) rectangle (7,-3);
    \node[anchor=center] (text) at (4,-2) {$\empdistribution$};
    \node[anchor=center] (text) at (-1,-2) {$\datanum \,\, \cdot $};

    \node[anchor=center] (text) at (10,-2) {${=}$};

    \begin{scope}
        [shift={(12,0)}]

        \draw[->] (2.5,1) -- (2.5,3) node[midway,right] {\tiny $\catvariableof{0}$};
        \draw (1,-1) rectangle (4,1);
        \node[anchor=center] (text) at (2.5,0) {$\rencodingof{\groundingof{\extformulaof{0}}}$};
        \node[anchor=center] (text) at (2.5,-2) {$\cdots$};

        \node[anchor=center] (text) at (6.5,0) {$\cdots$};

        \draw[->] (10.5,1) -- (10.5,3) node[midway,right] {\tiny $\catvariableof{\atomorder-1}$};
        \draw (8.75,-1) rectangle (12.25,1);
        \node[anchor=center] (text) at (10.5,0) {$\rencodingof{\groundingof{\extformulaof{\atomorder\shortminus1}}}$};
        \node[anchor=center] (text) at (10.5,-2) {$\cdots$};

        \draw[<-] (13,-3) -- (3.5,-3) ;
        \draw[<-] (13,-5) -- (1.5,-5) ;

        \drawvariabledot{11.5}{-3}
        \draw[->] (11.5,-3) -- (11.5,-1);

        \drawvariabledot{9.5}{-5}
        \draw[->] (9.5,-5) -- (9.5,-1);

        \drawvariabledot{3.5}{-3}
        \draw[->] (3.5,-3) -- (3.5,-1);

        \drawvariabledot{1.5}{-5}
        \draw[->] (1.5,-5) -- (1.5,-1);

        \draw (13,-2) rectangle (17,-6);
        \node[anchor=center] (text) at (15,-4) {$\rencodingof{\groundingof{\impformula}}$};
        \node[anchor=center] (text) at (12,-3.75) {$\vdots$};
        \draw[->] (17,-4) -- (18,-4);
        \drawvariabledot{18}{-4}
        \draw[] (18,-4) -- (19,-4);
        \draw (19,-3) rectangle (21,-5);
        \node[anchor=center] (text) at (20,-4) {$\tbasis$};

    \end{scope}


\end{tikzpicture}
\end{center}




\subsect{Basis CP Decomposition of extracted data}

To connect with the empirical distribution introduced in \secref{sec:empDistribution} we now show how the empirical distribution extracted from the interpretations of the formulas $\impformula,\extformulas$ on a first-order logic world $\dataworld$ can be represented by tensor networks.

First of all, we decompose the importance formula into a basis CP format (see \charef{cha:sparseCalculus}), that is a decomposition
\begin{align*}
    \groundingofat{\impformula}{\shortindvariables}
    = \contractionof{
        \{\legcoreofat{\indenumerator}{\indvariableof{\indenumerator},\datvariable} \, : \, \indenumeratorin \}
    }{\shortindvariables}
\end{align*}
such that all $\legcoreofat{\indenumerator}{\indvariableof{\indenumerator},\datvariable}$ are directed and boolean tensors.
Here an auxiliary variables $\datvariable$ taking values in $[\datanum]$ is introduced, which we call the data variable, which enumerates the non-vanishing coordinates of $\groundingof{\impformula}$.
With this decomposition, we can understand the decomposition of $\groundingofat{\impformula}{\shortindvariables}$ as a basis encoding of an term selection map $\secdatamap$ with coordinate maps defined such that
\begin{align*}
    \bencodingofat{\secdatamap_{\indenumerator}}{\indvariableof{\indenumerator},\datvariable}
    = \legcoreofat{\indenumerator}{\indvariableof{\indenumerator},\datvariable} \, .
\end{align*}
We depict this decomposition by:
\begin{center}
    \begin{tikzpicture}[scale=0.35, thick] % , baseline = -3.5pt

    \begin{scope}
        [shift={(0,2)}]
        \draw[] (0,1)--(0,-1) node[midway,left] {\tiny $\indvariableof{0}$};
        \draw[] (1.5,1)--(1.5,-1) node[midway,left] {\tiny $\indvariableof{1}$};
        \node[anchor=center] (text) at (3,0) {$\cdots$};
        \draw[] (4,1)--(4,-1) node[midway,right] {\tiny $\indvariableof{\indorder\shortminus1}$};
    \end{scope}


    \draw (-1,1) rectangle (5,-1);
    \node[anchor=center] (text) at (2,0) {\small ${\groundingof{\impformula}}$};


    \node[anchor=center] (text) at (7,0) {${=}$};


    \begin{scope}
        [shift={(10,2)}]



        \coordinate (conposseldec) at (4.5,-5.5);
        \drawvariabledot{4.5}{-5.5}

        \draw (conposseldec) -- (4.5,-7.5) node[midway, right] {\tiny ${\datvariable}$}; % Unclear, whether this is the best notation!
        \draw (3.5,-7.5) rectangle (5.5, -9.5);
        \node[anchor=center] (text) at (4.5,-8.5) {\small $\ones$};

        \draw[-<-] (0,1) -- (0,-1) node[midway,left] {\tiny $\indvariableof{0}$};
        \draw (-1,-1) rectangle (1, -3);
        \node[anchor=center] (text) at (0,-2) {\small $\secdatacoreof{0}$};
        \draw[-<-] (0,-3) to[bend right=20] (conposseldec);


        \draw[-<-] (3,1) -- (3,-1) node[midway,left] {\tiny $\indvariableof{1}$};
        \draw (2,-1) rectangle (4, -3);
        \node[anchor=center] (text) at (3,-2) {\small $\secdatacoreof{1}$};
        \draw[-<-] (3,-3) to[bend right=20]  (conposseldec);

        \node[anchor=center] (text) at (6,-2) {$\cdots$};

        \draw[-<-] (9,1) -- (9,-1) node[midway,left] {\tiny $\indvariableof{\indorder-1}$};
        \draw (7.75,-1) rectangle (10.25, -3);
        \node[anchor=center] (text) at (9,-2) {\small $\secdatacoreof{\indorder-1}$};
        \draw[-<-] (9,-3) to[bend left=20]  (conposseldec);

    \end{scope}


\end{tikzpicture}
\end{center}

Based on these construction, we now provide a tensor network decomposition of the extracted empirical distribution.

\begin{theorem}
    Given a first-order logic world $\dataworld$, an importance formula $\impformula$ and extraction formulas $\extformulaof{\catenumerator}$ for $\catenumeratorin$, we have
    \begin{align*}
        \bencodingofat{\extractionrelation}{\shortindvariables,\shortcatvariables} =
        \contractionof{
            \{\bencodingofat{\groundingof{\extformulaof{\atomenumerator}}}{\catvariableof{\catenumerator},\shortindvariables} \, : \, \catenumeratorin\}
            \cup \{\bencodingofat{\secdatamap_{\indenumerator}}{\indvariableof{\indenumerator},\datvariable} \, : \, \indenumeratorin\}
        }{\shortindvariables,\shortcatvariables}
    \end{align*}
    and thus
    \begin{align*}
        \empdistributionat{\shortcatvariables} =
        \frac{1}{\datanum}  \contractionof{
            \{\bencodingofat{\groundingof{\extformulaof{\atomenumerator}}}{\catvariableof{\catenumerator},\shortindvariables} \, : \, \catenumeratorin\}
            \cup \{\bencodingofat{\secdatamap_{\indenumerator}}{\indvariableof{\indenumerator},\datvariable} \, : \, \indenumeratorin\}
        }{\shortcatvariables} \, .
    \end{align*}
\end{theorem}
\begin{proof}
    To show the first claim, let us choose arbitrary state tuples $\shortindindices$ and $\shortcatindices$.
    We then have
    \begin{align*}
        &\contractionof{
            \{\bencodingofat{\groundingof{\extformulaof{\atomenumerator}}}{\catvariableof{\catenumerator},\shortindvariables} \, : \, \catenumeratorin\}
            \cup \{\bencodingofat{\secdatamap_{\indenumerator}}{\indvariableof{\indenumerator},\datvariable} \, : \, \indenumeratorin\}
        }{\indexedshortindvariables,\indexedshortcatvariables} \\
        & \quad  =  \contraction{
            \{\bencodingofat{\groundingof{\extformulaof{\atomenumerator}}}{\indexedcatvariableof{\catenumerator},\indexedshortindvariables} \, : \, \catenumeratorin\}
            \cup \{\bencodingofat{\secdatamap_{\indenumerator}}{\indexedindvariableof{\indenumerator},\datvariable} \, : \, \indenumeratorin\}
        } \, .
    \end{align*}
    This contraction evaluates to $1$, if and only if for all $\catenumeratorin$ we have $\bencodingofat{\groundingof{\extformulaof{\atomenumerator}}}{\catvariableof{\catenumerator},\shortindvariables}=1$ and
    \begin{align*}
        \contraction{\{\bencodingofat{\secdatamap_{\indenumerator}}{\indexedindvariableof{\indenumerator},\datvariable} \, : \, \indenumeratorin\}}  = 1 \, .
    \end{align*}
    The first condition is equal to $\catindexof{\atomenumerator} = \extformulaofat{\atomenumerator}{\indexedshortindvariables}$ for all $\catenumeratorin$ and the second to
    \begin{align*}
        \groundingofat{\impformula}{\indexedshortindvariables} = 1 \, .
    \end{align*}
    Comparing with the definition of the extraction relation (see \defref{def:extractionRelation}), we notice that these conditions are equal to $(\shortindindices,\shortcatindices)\in\extractionrelation$ and therefore to
    \begin{align*}
        \bencodingofat{\extractionrelation}{\indexedshortindvariables,\indexedshortcatvariables} \, .
    \end{align*}
    The first claim follows, since $\bencodingof{\extractionrelation}$ is boolean, as is the contraction of the cores $\bencodingof{\groundingof{\extformulaof{\atomenumerator}}}$ with the cores $\bencodingof{\secdatamap_{\indenumerator}}$, which leaves the outgoing variables $\shortcatvariables$ open.
    The second claim follows from the first using that $\empdistributionat{\shortcatvariables}=\frac{1}{\datanum}\contractionof{\bencodingof{\extractionrelation}}{\shortcatvariables}$.
\end{proof}

To connect with the representation of empirical distributions based on data cores (see \secref{sec:empDistribution}), we now form data cores by contractions with the grounding of extraction formulas with the cores $\bencodingof{\secdatamap_{\indenumerator}}$ (see \figref{fig:datacoreGeneration}),
\begin{align*}
    \datacoreofat{\atomenumerator}{\catvariableof{\catenumerator},\datvariable}
    = \sbcontractionof{
        \{\bencodingofat{\groundingof{\extformulaof{\atomenumerator}}}{\catvariableof{\catenumerator},\shortindvariables}\}
        \cup \{ \legcoreofat{\indenumerator}{\indvariableof{\indenumerator},\datvariable} \, : \, \indenumeratorin\}
    }{\catvariableof{\atomenumerator},\datvariable} \, .
\end{align*}

% Empirical distribution
The empirical distribution is then a tensor network of these tensors, as we show next.

\begin{theorem}
    \label{the:extractionDataCores}
    We have
    \begin{align*}
        \sbcontractionof{\bencodingof{\extractionrelation}}{\shortcatvariables}
        = \contractionof{\{\datacoreofat{\atomenumerator}{\datvariable,\catvariableof{\atomenumerator}} \, : \, \atomenumeratorin\}}{\shortcatvariables}
    \end{align*}
    and thus
    \begin{align*}
        \empdistributionat{\shortcatvariables}
        = \frac{1}{\datanum} \contractionof{\{\datacoreofat{\atomenumerator}{\datvariable,\catvariableof{\atomenumerator}}  \, : \, \atomenumeratorin\}}{\shortcatvariables} \, .
    \end{align*}
\end{theorem}
\begin{proof}
    By \theref{the:extractionrelationDecomposition} we have
    \begin{align*}
        \bencodingofat{\extractionrelation}{\shortindvariables,\shortcatvariables} =
        \contractionof{
            \{\bencodingofat{\groundingof{\extformulaof{\atomenumerator}}}{\catvariableof{\catenumerator},\shortindvariables} \, : \, \catenumeratorin\}
            \cup \{\bencodingofat{\secdatamap_{\indenumerator}}{\indvariableof{\indenumerator},\datvariable} \, : \, \indenumeratorin\}
        }{\shortindvariables,\shortcatvariables} \, .
    \end{align*}
    Since $\bencodingofat{\secdatamap_{\indenumerator}}{\indvariableof{\indenumerator},\datvariable}$ are directed and boolean, they can be copied and separately contracted with each $\groundingof{\extformulaof{\atomenumerator}}$, without changing the contraction.
    We arrive at
    \begin{align*}
        &\bencodingofat{\extractionrelation}{\shortindvariables,\shortcatvariables} \\
        &\quad = \contractionof{
            \big\{\contractionof{
                \{\bencodingofat{\groundingof{\extformulaof{\atomenumerator}}}{\catvariableof{\catenumerator},\shortindvariables}\}
                \cup \{\bencodingofat{\secdatamap_{\indenumerator}}{\indvariableof{\indenumerator},\datvariable} \, : \, \indenumeratorin\}
            }{\catvariableof{\catenumerator},\datvariable} \, : \, \catenumeratorin \big\}
        }{\shortindvariables,\shortcatvariables} \\
        & \quad =  \contractionof{\{\datacoreofat{\atomenumerator}{\datvariable,\catvariableof{\atomenumerator}}  \, : \, \atomenumeratorin\}}{\shortcatvariables} \, ,
    \end{align*}
    which established the claim.
\end{proof}

% Efficient contraction: Do also basis decomposition of the extraction query and use efficient contraction!
%Towards efficient calculation of the data cores, we build a basis CP decomposition of $\groundingof{\impformula}$, where we further demand $\scalarcore=\ones$.
%This is a collection of basis leg cores $\legcoreof{\fixedimpformula,\indenumerator}$ such that
%\begin{align*}
%    \fixedimpformula[\shortindvariablelist]
%    = \contractionof{ \left\{ \legcoreofat{\fixedimpformula,\indenumerator}{\datvariable,\indvariableof{\indenumerator}} \, : \, \indenumeratorin \right\} }{\shortindvariablelist} \, .
%\end{align*}

% Data enumeration -> To representation
%We can further utilize any decomposition of $\impformula$ into a directed and binary CP Format to enumerate the datapoints by the slice index $\datindex$. % Approaches like SPARQL directly give us these by solution mappings.
%Understanding $\impformula$ as a query on the world being the database, such decomposition is given by the set of solution mappings.


\begin{figure}[h]
    \begin{center}
        \begin{tikzpicture}[scale=0.35, yscale=1, thick] % , baseline = -3.5pt


    \draw[->-] (4,-1) -- (4,1) node[midway, right] {\tiny $\catvariableof{\atomenumerator}$};
    \draw (3,-1) rectangle (5,-3);
    \node[anchor=center] (text) at (4,-2) {$\datacoreof{\atomenumerator}$};
    \draw[-<-] (4,-3) -- (4,-5) node[midway, right] {\tiny $\datvariable$};

    \node[anchor=center] (text) at (7,-2) {${=}$};

    \begin{scope}
        [shift={(10,0)}]

        \draw[->-] (3,1) -- (3,3) node[midway, right] {\tiny $\catvariableof{\atomenumerator}$};
        \draw (-1,1) rectangle (7,-1);
        \node[anchor=center] (text) at (3,0) {$\rencodingof{\groundingof{\extformulaof{\atomenumerator}}}$};

        \draw[->-] (0,-3) -- (0,-1) node[midway,left] {\tiny $\indvariableof{0}$};
        \draw[->-] (3,-3) -- (3,-1) node[midway,left] {\tiny $\indvariableof{1}$};
        \draw[->-] (6,-3) -- (6,-1) node[midway,left] {\tiny $\indvariableof{2}$};


    \end{scope}

    \begin{scope}
        [shift={(10,-2)}]

        \coordinate (conposseldec) at (4.5,-5.5);
        \drawvariabledot{4.5}{-5.5}
        \draw[-<-] (conposseldec) -- (4.5,-7.5) node[midway, right] {\tiny $\indexvariable$};

        \draw (-1,-1) rectangle (1, -3);
        \node[anchor=center] (text) at (0,-2) {\small $\rencodingof{\secdatamap_0}$};%{\small $\legcoreof{\fixedimpformula,0}$};
        \draw[-<-] (0,-3) to[bend right=20] (conposseldec);

        \draw (2,-1) rectangle (4, -3);
        \node[anchor=center] (text) at (3,-2) {\small $\rencodingof{\secdatamap_1}$};%{\small $\legcoreof{\fixedimpformula,1}$};
        \draw[-<-] (3,-3) to[bend right=20]  (conposseldec);

        \draw (5,-1) rectangle (7, -3);
        \node[anchor=center] (text) at (6,-2) {\small $\rencodingof{\secdatamap_2}$};%{\small $\legcoreof{\fixedimpformula,2}$};
        \draw[-<-] (6,-3) to[bend right=-20]  (conposseldec);

        \draw[<-] (9,1) -- (9,-1) node[midway,left] {\tiny $\indvariableof{3}$};
        \draw (8,-1) rectangle (10, -3);
        \node[anchor=center] (text) at (9,-2) {\small $\rencodingof{\secdatamap_3}$};%{\small $\legcoreof{\fixedimpformula,3}$};
        \draw[<-] (9,-3) to[bend right=-20]  (conposseldec);


        \node[anchor=center] (text) at (12,-2) {$\cdots$};

        \draw[<-] (15,1) -- (15,-1) node[midway,left] {\tiny $\indvariableof{\indorder-1}$};
        \draw (13.5,-1) rectangle (16.5, -3);
        \node[anchor=center] (text) at (15,-2) {\small $\rencodingof{\secdatamap_{\indorder-1}}$};%{\small $\legcoreof{\fixedimpformula,\variableorder-1}$};
        \draw[<-] (15,-3) to[bend left=20]  (conposseldec);


        \draw (8,1) rectangle (16, 3);
        \node[anchor=center] (text) at (12,2) {\small $\ones$};


    \end{scope}


    \node[anchor=center] (text) at (29,-2) {${=}$};


    \begin{scope}
        [shift={(32,0)}]

        \draw[->-] (3,1) -- (3,3) node[midway, right] {\tiny $\catvariableof{\atomenumerator}$};
        \draw (-1,1) rectangle (7,-1);
        \node[anchor=center] (text) at (3,0) {$\rencodingof{\groundingof{\extformulaof{\atomenumerator}}}$};

        \draw[->-] (0,-3) -- (0,-1) node[midway,left] {\tiny $\indvariableof{0}$};
        \draw[->-] (3,-3) -- (3,-1) node[midway,left] {\tiny $\indvariableof{1}$};
        \draw[->-] (6,-3) -- (6,-1) node[midway,left] {\tiny $\indvariableof{2}$};


    \end{scope}

    \begin{scope}
        [shift={(32,-2)}]


        \coordinate (conposseldec) at (3,-5.5);
        \drawvariabledot{3}{-5.5}
        \draw[<-] (conposseldec) -- (3,-7.5) node[midway, right] {\tiny $\datvariable$};

        \draw (-1,-1) rectangle (1, -3);
        \node[anchor=center] (text) at (0,-2){\small $\rencodingof{\secdatamap_0}$};%{\small $\legcoreof{\fixedimpformula,0}$};
        \draw[<-] (0,-3) to[bend right=20] (conposseldec);

        \draw (2,-1) rectangle (4, -3);
        \node[anchor=center] (text) at (3,-2) {\small $\rencodingof{\secdatamap_1}$};%{\small $\legcoreof{\fixedimpformula,1}$};
        \draw[<-] (3,-3) to[bend right=0]  (conposseldec);

        \draw (5,-1) rectangle (7, -3);
        \node[anchor=center] (text) at (6,-2) {\small $\rencodingof{\secdatamap_2}$};%{\small $\legcoreof{\fixedimpformula,2}$};
        \draw[<-] (6,-3) to[bend right=-20]  (conposseldec);


    \end{scope}


\end{tikzpicture}
    \end{center}
    \caption{Generation of a data core for the variable $\catvariableof{\catenumerator}$ given an extraction formula $\extformulaof{\catenumerator}$ and an importance formula, which grounding is decomposed into a basis CP format with leg vectors $\bencodingofat{\secdatamap_{\indenumerator}}{\indvariableof{\indenumerator},\datvariable}$.
    Term variables, which are appearing in the importance formula, but not in the extraction formula $\extformulaof{\catenumerator}$ can be treated trivally by contraction with the trivial tensor (here $\indvariableof{4},\ldots,\indvariableof{\indorder-1})$.
    }
    \label{fig:datacoreGeneration}
\end{figure}


% Comment: Exploitation of common structure
When many atom extraction formulas differ only by a constant, we can replace the constant by an auxiliary term variable.
The atoms are then the atomizations of this variable (see \secref{sec:categoricalTN}), treated as a categorical variable, with respect to the constant in the extraction query.
The advantages are that we can avoid the $\bencodingof{}$-formalism and directly model the categorical distributions.

This also enables a batchwise computation of multiple $\sparql$ queries, which differ only in one constant.


%\subsect{Design of the Formulas}
%
%Most intuitive when labeling individuals by classes.
%Extraction formulas $\extformulas$ can then be defined by subclasses of the member of a class and relations between objects of different classes. % Koller calls atomic formulas the template attributes
%We then choose $\formulaset$ as more involved formulas decomposed into connectives acting on these atoms.
%The importance formula $\impformula$ is then designed based on class memberships to ensure, that the arguments of the formulas are always of specific classes. % Koller specifies to each argument of the attributes a class
%
%% Approach
%We propose to
%\begin{itemize}
%    \item Execute an extraction query to get pairs of individuals (the pairDf).
%    \item Propositionalize the FOL Formulas independently on each tuple taking the individuals as a set of constant and filtering on the possible properties of each individuals.
%    (Can understand as adding knowledge that most of the relations do not hold)
%    \item Understand each such generated knowledge base as datapoint and average over them to get the empirical distribution to be fit.
%    \item Fit a MLN describing the statistical relations of unseen results of the extraction query, based on likelihood maximation.
%\end{itemize}




\sect{Generation of first-order logic worlds}

\red{
    So far we have discussed, how MLNs for FOL Knowledge Bases such as Knowledge Graphs can be built by extracting data.
    Conversely, any binary tensor can be interpreted as a Knowledge Graph.
    To be more precise, we follow the intuition that the ones coordinates mark possible worlds compatible with the knowledge about a factored system.
    Each possible world can then be encoded in a subgraph of the Knowledge Graph representing the world.
%
    This amounts to an "inversion" of the data generation process described in the subsection above.
}

In the previous section we have described a way to extract an effective empirical distribution for the likelihood of a first-order logic world given a HFLN.
We now want to investigate methods to reproduce an empirical distribution based on a constructed first-order logic world.

\begin{definition}[Reproduction of Empirical Distributions]
    Given an empirical distribution $\empdistribution\in\atomspace$, we say that a triple $(\dataworld,\impformula,\shortextformulas)$ of a FOL world $\dataworld$ an importance formula $\impformula$ and extraction formulas $\shortextformulas=\{\extformulaof{\atomenumerator}\,:\,\atomenumeratorin\}$ reproduces $\empdistribution$, when
    \begin{align*}
        \empdistribution
        = \normationof{\{\groundingof{\impformula}\}\cup\{\bencodingof{\kggroundingof{\extformulaof{\atomenumerator}}\, : \, \atomenumeratorin}\}}{\shortcatvariables} \, .
    \end{align*}
\end{definition}

% If \datamap is not known
Note that for distribution $\probtensor$ to be reproducable, it needs to have rational coordinates. %, since each coordinate can be interpreted as the frequency of the respective world in the data $\datamap$.
If any only if all coordinates are rational, we find a $\datanum\in\nn$ such that $\imageof{\datanum\cdot\probtensor}\subset\nn$.
We can then interpret $\datanum$ as the number of samples, and construct a sample selector map by understanding each coordinate of $\datanum\cdot\probtensor$ as the number of appearances of the respective world in the samples.

We show different schemes and give examples on Knowledge Graphs, where we provide examples for importance and extraction formulas by $\sparql$ queries.


%\subsect{Example: Generation of Knowledge Graphs} % To generation of FOL worlds?
%
% Having a directed and binary CP decomposition of $\exformula$, each possible world is encoded by a slice.


% Formalization
%\begin{definition}[Reproduction of Empirical Distributions]
%    Given an empirical distribution $\empdistribution\in\bigotimes_{\atomenumeratorin}\rr^2$, we say that a tuple $(\kg,\impformula,\{\extformulas\})$ of a Knowledge Graph $\kg$ and queries $\impformula,\extformulaof{\atomenumerator}$ reproduces $\empdistribution$, when
%    \[\empdistribution = \normationof{\{\kggroundingof{\impformula}\}\cup\{\bencodingof{\kggroundingof{\extformulaof{\atomenumerator}}\, : \, \atomenumeratorin}\}}{\shortcatvariables} \, .  \]
%\end{definition}

%

%In a frequentist interpretation we instantiate each world according to the rate $\probtensor(\atomindices)$.
%This interpretation requires a rounding of the real probabilities by rational numbers.


\subsect{Samples by single objects}

%\subsect{Samples by single objects}

In the first reproduction scheme we construct datapoints by dedicated objects, which represent a sample, that is we choose a domain $\worlddomain=[\datdim]$.

\begin{theorem}
    \label{the:reproducingSingleObjects}
    Let there be an empirical distribution $\empdistribution$ to a sample selector map $\datamap$ (see \defref{def:dataMap}), we construct a world $\dataworld[\selvariable,\indvariable]$ with $\atomorder$ unary predicates by
    \begin{align*}
        \dataworldat{\selvariable,\indvariable}
        = \sum_{\atomenumeratorin} \sum_{\datindexin \, : \datamap_{\atomenumerator}(\datindex)=1} \onehotmapofat{\atomenumerator}{\selvariable} \otimes \onehotmapofat{\datindex}{\indvariable} \, .
    \end{align*}
    We further choose a trivial importance query, that is
    \begin{align*}
        \groundingofat{\impformula}{\indvariable} = \onesat{\indvariable} \, ,
    \end{align*}
    and extraction queries coinciding with the unary predicates, that is for $\atomenumeratorin$
    \begin{align*}
        \extformulaof{\atomenumerator} = \folpredicateof{\atomenumerator} \, .
    \end{align*}
    Then, the triple $(\dataworld,\impformula,\shortextformulas)$ reproduces $\empdistribution$.
%    reproduces with the trivial importance query and extraction queries coinciding with the predicates the dataset $\datamap$.
\end{theorem}
\begin{proof}
    By \theref{the:extractionDataCores} it is enough to show, that the data cores constructed from the data extraction process coincide with those of $\empdistribution$.
    We enumerate to this end the non-vanishing coordinates of $\groundingof{\impformula}$ by the data variable $\datvariable$ taking values $\datindexin$, as
    \begin{align*}
        \groundingofat{\impformula}{\indvariable=\datindex} = 1 \,
    \end{align*}
    and choose
    \begin{align*}
        \secdatamap = \identity \, .
    \end{align*}
    For arbitrary $\atomenumeratorin$ and $\datindexin$ we now have
    \begin{align*}
        \datacoreofat{\atomenumerator}{\catvariableof{\catenumerator},\indexeddatvariable}
        &= \contractionof{
            \bencodingofat{\groundingof{\extformulaof{\atomenumerator}}}{\catvariableof{\catenumerator},\indvariable},
            \legcoreofat{0}{\indvariable,\indexeddatvariable}
        }{\catvariableof{\atomenumerator},\datvariable} \\
        &= \contractionof{
            \bencodingofat{\groundingof{\extformulaof{\atomenumerator}}}{\catvariableof{\catenumerator},\indvariable},
            \onehotmapofat{\secdatamap(\datindex)}{\indvariable}
        }{\catvariableof{\atomenumerator},\indexeddatvariable} \\
        &= \onehotmapofat{\datamap_\atomenumerator(\datindex)}{\catvariableof{\catenumerator}} \, .
    \end{align*}
    This coincides with the slice of the data core of the CP representation of empirical distributions used in \theref{the:empCPRep}.
    Since the slice and the core was arbitrary, the tensor network representations in \theref{the:empCPRep} and \theref{the:extractionDataCores} are equal and thus the triple $(\dataworld,\impformula,\shortextformulas)$ reproduces $\empdistribution$.
\end{proof}


We now give by the next theorem an example of a Knowledge Graph with $\sparql$ queries reproducing and arbitrary empirical distribution.

\begin{theorem}
    \label{the:reproducingKGSingelObjects}
    Let $\empdistribution$ be an empirical distribution to the sample selector $\datamap$.
    We construct a Knowledge Graph of the resources $\worlddomain = \{s_\datindex \, : \, \datindexin\} \cup \{C\} \cup \{C_\atomenumerator \, : \, \atomenumeratorin\}$, where $s_{\datindex}$ represent samples and $C_\atomenumerator$ unary predicates, by
    \begin{align*}
        \kggroundingof{\rdf}
        =
        \sum_{\datindexin}
        \onehotmapof{\indexinterpretationof{s_\datindex}}{\sindvariable}
        \otimes \onehotmapof{\indexinterpretationof{\mathrdftype}}{\pindvariable}
        \otimes \onehotmapof{\indexinterpretationof{C}}{\oindvariable}
        +
        \sum_{\datindexin} \sum_{\atomenumeratorin \, : \, \datamap_{\atomenumerator}(\datindex)=1}
        \onehotmapof{\indexinterpretationof{s_\datindex}}{\sindvariable}
        \otimes \onehotmapof{\indexinterpretationof{\mathrdftype}}{\pindvariable}
        \otimes \onehotmapof{\indexinterpretationof{C_\atomenumerator}}{\oindvariable} \, .
    \end{align*}
    We further define an importance formula by the $\sparql$ query
    \begin{centeredcode}
        \impformula = SELECT \{ ?x \} WHERE \{ ?x \quad \rdftype\quad C \, .\}
    \end{centeredcode}
    and for each $\atomenumeratorin$ an extraction formula by the query
    \begin{centeredcode}
        $\extformulaof{\atomenumerator}$ = SELECT \{ ?x \} WHERE \{ ?x \quad \rdftype \quad $C_\atomenumerator$ \, .\} \, .
    \end{centeredcode}
    Then the triple $(\kg,\impformula,\shortextformulas)$ reproduces $\empdistribution$.
\end{theorem}
\begin{proof}
    We show the theorem analogously to \theref{the:reproducingSingleObjects}, with the slide difference in the importance formula.
    We have for the grounding of $\impformula$ on $\kg$ that
    \begin{align*}
        \kggroundingofat{\impformula}{\indvariable} = \sum_{\datindexin}  \onehotmapof{\indexinterpretationof{s_\datindex}}{\indvariable}
    \end{align*}
    and enumerate the non-vanishing coordinates by $\datvariable$.

    For each extraction formula we have
    \begin{align*}
        \kggroundingofat{\extformulaof{\atomenumerator}}{\indvariable} = \sum_{\datindexin \, : \, \datamap_{\atomenumerator}(\datindex)=1} \onehotmapof{\indexinterpretationof{s_\datindex}}{\indvariable} \,.
    \end{align*}
    It follows that the data cores used in \theref{the:extractionDataCores} are
    \begin{align*}
        \bencodingofat{\datamap_\atomenumerator}{\catvariableof{\atomenumerator},\datindex}
        = \onehotmapofat{0}{\catvariableof{\atomenumerator}} \otimes \left(\sum_{\datindexin \, : \, \datamap_{\atomenumerator}(\datindex)=0} \onehotmapofat{\datindex}{\datvariable}\right)
        +\onehotmapofat{1}{\catvariableof{\atomenumerator}} \otimes \left(\sum_{\datindexin \, : \, \datamap_{\atomenumerator}(\datindex)=1} \onehotmapofat{\datindex}{\datvariable}\right)
    \end{align*}
    and they thus coincide with those in the decomposition in \theref{the:empCPRep}.
    The claim follows therefore with the same argumentation as in the proof of \theref{the:reproducingSingleObjects}.
\end{proof}

%
Let us provide some more insights on the construction of the reproducing Knowledge Graph in \theref{the:reproducingKGSingelObject}.
By the insertions to the one-hot encodings $\onehotmapof{\indexinterpretationof{s_\datindex}}{\sindvariable} \otimes \onehotmapof{\indexinterpretationof{\mathrdftype}}{\pindvariable} \otimes \onehotmapof{\indexinterpretationof{C}}{\oindvariable}$ we mark each sample representing resource by a class and ensure its appearance as a $\mathrm{owl:NamedIndividual}$ in the graph.
The insertions $\onehotmapof{\indexinterpretationof{s_\datindex}}{\sindvariable}\otimes \onehotmapof{\indexinterpretationof{\mathrdftype}}{\pindvariable} \otimes \onehotmapof{\indexinterpretationof{C_\atomenumerator}}{\oindvariable}$ on the other side encode the sample selecting map, by inserting exactly the assertions corresponding with the respective sample.
% 
In this simple Knowledge Graph, Description Logic is expressive enough to represent any formula $\folexformula$ composed of the formulas $\extformulas$.

%
%\begin{theorem}
%    Let there any empirical distribution $\empdistribution\in\bigotimes_{\atomenumeratorin}\rr^2$ and $\datanum\in\nn$ such that $\imageof{\datanum\cdot\empdistribution}\subset\nn$.
%    Then the tuple $(\kg,\impformula,\{\extformulas\})$ defined by a Knowledge Graph
%    \begin{align}
%        \kg =
%        & \bigcup_{\atomindicesin}  \{(
%        s_{j, \atomindices} \quad \mathrm{rdf:type} \quad C ) : j \in [\datanum\cdot\empdistribution(\atomindices)] \}  \\
%        &\bigcup_{\atomindicesin}  \{(
%        s_{j, \atomindices} \quad \mathrm{rdf:type} \quad C_\atomenumerator
%        ) : j \in [\datanum\cdot\empdistribution(\atomindices)], \atomenumeratorin , \atomlegindexof{\atomenumerator}=1\}
%    \end{align}
%    further an importance formula by the query
%    \begin{centeredcode}
%        \impformula = SELECT \{ ?x \} WHERE \{ ?x \quad \rdftype\quad C \, .\}
%    \end{centeredcode}
%    and extraction formulas for each $\atomenumeratorin$ by the query
%    \begin{centeredcode}
%        $\extformulaof{\atomenumerator}$ = SELECT \{ ?x \} WHERE \{ ?x \quad \rdftype \quad $C_\atomenumerator$ \, .\}
%    \end{centeredcode}
%    reproduces $\empdistribution$.
%\end{theorem}
%\begin{proof}
%    With respect to any enumeration of the resources of $\kg$ we have
%    \begin{align}
%        \kggroundingof{\impformula}
%        = \sum_{\atomindicesin} \sum_{j \in [\datanum\cdot\empdistribution(\atomindices)]} \onehotmapof{s_{j, \atomindices} }
%    \end{align}
%    and
%    \begin{align}
%        \kggroundingof{\extformulaof{\atomenumerator}}
%        = \sum_{\atomindicesin \, : \, \atomlegindexof{\atomenumerator} = 1} \sum_{j \in [\datanum\cdot\empdistribution(\atomindices)]} \onehotmapof{s_{j, \atomindices} } \, .
%    \end{align}
%    Summing over the resource variables of these tensors in a contraction we get
%    \begin{align}
%        \contractionof{\{\kggroundingof{\impformula}\}\cup\{\bencodingof{\kggroundingof{\extformulaof{\atomenumerator}}\, : \, \atomenumeratorin}\}}{\shortcatvariables}
%        & = \sum_{\atomenumeratorin}  \datanum\cdot\empdistribution(\atomindices) \cdot \onehotmapof{\atomindices} = \datanum \cdot \empdistribution
%    \end{align}
%    and therefore
%    \begin{align}
%        \normationof{\{\kggroundingof{\impformula}\}\cup\{\bencodingof{\kggroundingof{\extformulaof{\atomenumerator}}}\, : \, \atomenumeratorin\}}{\shortcatvariables} = \empdistribution \, .
%    \end{align}
%\end{proof}







\subsect{Samples by pairs of objects}

%\paragraph{TBox:} The categorical variables of the factored system are the classes.
%We define atomic formulas by the state indicators of each categorical variable as in \secref{sec:categoricalTN}.
%Each such atomic formula corresponds with a sub-class of the classes.
%By definition, each collection of state indicators define thus pairwise disjoint subclasses.
%
%\paragraph{ABox:} The samples are represented by single individuals in the Knowledge Graph.
%Their sub-class memberships corresponding with the categorical variables of the system are instantiated whenever the atom is true in the sample.
%%\subsubsect{Samples by pairs of resources}
%
%\begin{remark}[Refinement of the Samples]
%    We can split each sample node into a pair of individuals.
%    For this we need to specify, which each class membership will be encoded in a unary or binary attribute of the splitted individuals.
%    This specification is possible based on the extraction query and the atomic formulas.
%\end{remark}
%
%%
%Taking any importance query $\impformula$, which has no permutation symmetries, we can instantiate each projection variable for each sample and prepare the links according to the triple patterns.
%When the atom queries $\extformulas$ have different triple patterns compared with $\impformula$, we instantiate those in cases where $\atomlegindexof{\atomenumerator}=1$.


%
We now instantiate multiple objects for each datapoint, one for each variable of the importance formula, i.e. $\worlddomain=[\datdim]\times[\indorder]$
Label individuals $s_{\datindex,\indenumerator}$ by data index and variable index.

\begin{lemma}
    Let there a data map $\datamap$, queries $\impformula,\shortextformulas$ and a first-order logic world containing objects $s_{\datindex,\indenumerator}$ for $\datindexin$ and $\indenumeratorin$
    If
    \begin{align*}
        \kggroundingof{\impformula}
        = \sum_{\datindexin} \bigotimes_{\indenumeratorin} \onehotmapofat{\indexinterpretationof{s_{\datindex,\indenumerator}}}{\indvariableof{\indenumerator}}
    \end{align*}
    and for any $\atomenumeratorin$
    \begin{align*}
        \kggroundingof{\extformulaof{\atomenumerator}}
        = \sum_{\datindex : \datamap_{\atomenumerator}(\datindex)=1} \bigotimes_{\indvariableof{\indenumerator} \in \indvariableof{\extformulaof{\atomenumerator}}}
        \onehotmapofat{\indexinterpretationof{s_{\datindex,\indenumerator}}}{\indvariableof{\indenumerator}} \, .
    \end{align*}
%    \[ \kggroundingof{\extformulaof{\atomenumerator}}
%    = \sum_{\datindex : \datamap^{\atomenumerator}(\datindex)=1} \bigotimes_{\indenumerator \in \extformulaof{\atomenumerator}} \onehotmapof{\datindex,\indenumerator} \, . \]
    Then the tuple $(\kg,\impformula,\{\extformulas\})$ reproduces $\empdistribution$.
\end{lemma}
\begin{proof}
    We notice, that the grounding of the importance formula is in a basis CP format, since by assumption
    \begin{align*}
        \kggroundingof{\impformula}
        = \sum_{\datindexin} \bigotimes_{\indenumeratorin} \onehotmapofat{\indexinterpretationof{s_{\datindex,\indenumerator}}}{\indvariableof{\indenumerator}} \, .
    \end{align*}
    We choose $\datvariable$ to enumerate the non-vanishing entries and get a term selecting map
    \begin{align*}
        \secdatamap_{\indenumerator}(\datindex) = \indexinterpretationof{s_{\datindex,\indenumerator}} \, .
    \end{align*}
    From this we have
    \begin{align*}
        \contractionof{
            \{\bencodingofat{\kggroundingof{\extformulaof{\atomenumerator}}}{\catvariableof{\atomenumerator},\indvariableof{\extformulaof{\atomenumerator}}}\} \cup
            \{\bencodingofat{\secdatamap_{\indenumerator}}{\indvariableof{\indenumerator},\datvariable} \, : \, \indenumeratorin\}
        }{\catvariableof{\catenumerator},\datvariable}
        = \bencodingofat{\datamap_{\atomenumerator}}{\catvariableof{\catenumerator},\datvariable}
    \end{align*}
    and the claim follows with the same argumentation as in the proof of \theref{the:reproducingSingleObjects}.
\end{proof}


%Let us construct a Knowledge Graph
%\begin{align*}
%        \kggroundingof{\rdf}
%        =
%        \sum_{\datindexin}\sum_{\indenumeratorin}
%        \onehotmapof{\indexinterpretationof{s_{\datindex,\indenumerator}}}{\sindvariable}
%        \otimes \onehotmapof{\indexinterpretationof{\mathrdftype}}{\pindvariable}
%        \otimes \onehotmapof{\indexinterpretationof{C}}{\oindvariable}
%        +
%        \sum_{\datindexin} \sum_{\atomenumeratorin \, : \, \datamap_{\atomenumerator}(\datindex)=1} \sum_{\indvariableof{\indenumerator}\in\indvariableof{}}
%        \onehotmapof{\indexinterpretationof{s_{\datindex,\indenumerator}}}{\sindvariable}
%        \otimes \onehotmapof{\indexinterpretationof{\mathrdftype}}{\pindvariable}
%        \otimes \onehotmapof{\indexinterpretationof{C_\atomenumerator}}{\oindvariable} \, .
%\end{align*}
%    We further define an importance formula by the $\sparql$ query
%\begin{centeredcode}
%        \impformula = SELECT \{ ?x_0 \cdots ?x_{\indorder-1} \} WHERE \{ ?x_0 \quad \rdftype\quad C \, .\}
%\end{centeredcode}
%    and for each $\atomenumeratorin$ an extraction formula by the query
%\begin{centeredcode}
%        $\extformulaof{\atomenumerator}$ = SELECT \{ ?x \} WHERE \{ ?x \quad \rdftype \quad $C_\atomenumerator$ \, .\} \, .
%\end{centeredcode}
%    Then the triple $(\kg,\impformula,\shortextformulas)$ reproduces $\empdistribution$.



\sect{Discussion}


% Probabilistic Relational Models
Statistical Models are called Probabilistic Relational Models. % (RUSSELL - Chapter Probabilistic Programming).
Extensions are models that also handle structural uncertainty, i.e. distributions of worlds with varying $\worlddomain$.

% Comparison with network science
In the emerging area of network science \cite{barabasi_network_2016, giovanni_russo_vito_latora_complex_2017}, statistical models for random graphs are investigated.
Statistical Models of first-order logic go beyond the typical single edge type perspective of network science.


%
\begin{remark}[Alternative Representation of empirical distributions]
    So far, we have motivated the representation of empirical distributions based on basis CP decompositions based on data maps.
    In this section, based on the extraction queries, we have observed that empirical distributions might have more efficient representation formats.
    In many applications such as the computation of log-likelihoods we can use any representation of the empirical distribution by tensor networks.
    It is thus not necessary to compute the data cores as above, unless one requires a list of the extracted samples.
\end{remark}