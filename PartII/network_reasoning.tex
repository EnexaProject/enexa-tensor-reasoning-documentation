\chapter{\chatextnetworkReasoning}\label{cha:networkReasoning}

In this chapter we investigate the inference properties of \HybridLogicNetworks{}, exploiting the characterizations of the corresponding mean parameter polytopes in \charef{cha:networkRepresentation}.
We investigate unconstrained parameter estimated for \MarkovLogicNetworks{} and \HybridLogicNetworks{}, which are special cases of the backward maps introduced in \charef{cha:probRepresentation}.
We then motivate structure learning based on sparsity constraints on the parameters on the minterm exponential family and present heuristic strategies leading to efficient structure learning algorithms.




\sect{Entropy Optimization} \label{sec:parameterEstimation} % on \HybridLogicNetworks{} Check for redundancy with the mln introduction chapter!

We now motivate \HybridLogicNetworks{} as distributions with maximum entropy under a moment constraint.


%We will investigate two entropic approaches towards unconstrained parameter estimation.
%First of all, we characterize Maximum Entropy distributions with a moment constraint.
%Maximum Likelihood Estimation restricts the distributions optimized over to a specific set, here the \HybridLogicNetworks{}.

\subsect{Entropy Maximization}

The Maximum Entropy Problem $\probtagtypeinst{\entropysymbol}{\mlnstat,\basemeasure,\genmean}$ for a boolean statistic $\mlnstat$ %and a base measure $\basemeasure$ is %\MarkovLogicNetworks{} is
\begin{align}
    \label{prob:maxEntropyHLN}\tag{$\probtagtypeinst{\entropysymbol}{\mlnstat,\genmean}$}
    \argmax_{\probtensor\in\alldists} \quad \sentropyof{\probtensor}
    \stspace
    \contractionof{\probtensor,\sencmlnstat}{\selvariable}
    =  \genmeanwith
\end{align}
where by $\alldists$ we denote all probability distributions.

\begin{theorem}
    \label{the:maxEntropyCharacterizationHLN}
    Let $\genmeanwith\in\genmeanset$ and the minimal face of $\genmeanset$, which includes $\genmeanwith$, be $\genfaceset$, and let
    \begin{align*}
        \gencanparam = \backwardmapwrtof{\mlnstat,\hlnfacemeasure}{\genmeanwith} \, ,
    \end{align*}
    where $\backwardmapwrt{\mlnstat,\hlnfacemeasure}$ is a backward map in the exponential family $\expfamilyof{\mlnstat,\hlnfacemeasure}$ and $\hlnfacemeasure$ is the face measure to $\genfaceset$ (see \theref{the:faceMeasureCharacterizationHLN}).
    The solution of the Maximum Entropy \probref{prob:maxEntropyHLN} is then
    \begin{align*}
        \probwith = \expdistofat{(\mlnstat,\gencanparam,\hlnfacemeasure)}{\shortcatvariables} \, .
    \end{align*}
\end{theorem}
\begin{proof}
    Any feasible distribution has to be representable by $\hlnfacemeasure$, since $\genmeanwith\in\hlnfaceset$.
    The solution therefore coincides with the solution of the maximum entropy problem with respect to $\mlnstat$ and $\hlnfacemeasure$ as a base measure.
    Since $\genfaceset$ is minimal, $\meanparamwith$ is on the effective interior of the face and the claim follows with \theref{the:maxEntropyFace}.
\end{proof}

\theref{the:maxEntropyCharacterizationHLN} characterizes the solution of the Maximum Entropy Problem for arbitrary positions of the mean parameter.
Note, that if $\genmeanwith\notin\genmeanset$ then no distribution is feasible for \probref{prob:maxEntropyHLN} and there is no solution.
We are especially interested in situations, where the solution is a Hybrid Logic Network.
As we state next, this is exactly the case if the mean parameter is reproducable by a Hybrid Logic Network (see \secref{sec:HLNrepMean}).

\begin{theorem}
    %Let $\mlnstat$ be a boolean statistic and $\genmeanwith\in\hlnmeanset$.
    The solution of the Maximum Entropy \probref{prob:maxEntropyHLN} is a Hybrid Logic Network, if and only if $\genmeanwith$ is reproducable by a Hybrid Logic Network.
\end{theorem}
%\begin{proof}
%
%\end{proof}
%
%
%\begin{theorem}
%    Let $\genmeanwith$ be a mean parameter reproducable by a Hybrid Logic Network. % (see \defref{def:HLNrepMean}).
%    Then the Hybrid Logic Network reproducing $\genmeanwith$ is the solution of the Maximum Entropy Problem \eqref{prob:maxEntropyHLN}.
%\end{theorem}
\begin{proof}
    We notice, that if and only if the solution of the Maximum Entropy Distribution is a \HybridLogicNetwork{}, then the normalization of the corresponding face measure $\hlnfacemeasure$ of the minimal face containing $\genmeanwith$ is in $\elrealizabledistsof{\mlnstat}$.
    This is equivalent to $\genmeanwith$ being reproducable by a Hybrid Logic Network.
    %The solution of \probref{prob:maxEntropyHLN} characterized in \theref{the:maxEntropyCharacterizationHLN} is then also in $\elrealizabledistsof{\mlnstat}$.
\end{proof}


\subsect{Cross Entropy Minimization}

Different to Maximum Entropy Problems, we formulate the Maximum Likelihood Problem as cross entropy minimization with respect to \HybridLogicNetworks{}, that is
\begin{align}
    \label{prob:minCrossEntropyHLN}\tag{$\probtagtypeinst{\mathrm{M}}{\Gamma,\gendistribution}$}
    \argmin_{\probtensor\in\elrealizabledistsof{\mlnstat}} \centropyof{\gendistribution}{\probtensor}
\end{align}
When choosing $\gendistribution$ by an empirical distribution, this minimization problem is the Maximum Likelihood Problem (see \secref{cha:probReasoning}).
% LEMMA TO CHA:PROBREASONING?
In order to characterize the solution of the cross entropy minimization problem on \HybridLogicNetworks{}, we first characterize the solution of the cross entropy minimization on exponential families.


To solve the maximum likelihood problems on \HybridLogicNetworks{} we choose the parametrization by tuples $\hybridparamin$ and
\begin{align*}
    \min_{\hybridparamin} \centropyof{\gendistribution}{\probtensorof{\mlnstat,\hybridparam}}
\end{align*}



\begin{lemma}
    \label{lem:minCrossEntropyHardparam}
    For any $\canparam$ we have
    \begin{align*}
        \min_{\hardparam} \centropyof{\gendistribution}{\probtensorof{\mlnstat,\hybridparam}}
        = \centropyof{\gendistribution}{\probtensorof{\mlnstat,(\hardlegsetto{\genmean},\hardlegindicesto{\genmean},\canparam)}}
    \end{align*}
    where $(\hardlegsetto{\genmean},\hardlegindicesto{\genmean})$ parametrize the smallest cube face containing $\genmeanwith=\contractionof{\gendistributionwith,\sencmlnstatwith}{\selvariable}$.
\end{lemma}
\begin{proof}
    We decompose the cross entropy in three terms
    \begin{align*}
        \centropyof{\gendistribution}{\probtensorof{\mlnstat,\hybridparam}}
        =& \contraction{\gendistributionwith,-\lnof{\hlnformulawith}} \\
        &+ \contraction{\gendistributionwith,\sencmlnstatwith,-\lnof{\actcorewith}} \\
        &+ \lnof{\contraction{\mlnstatccwith,\kcoreofat{\hardparam}{\shortcatvariables},\actcoreofat{\canparam}{\shortcatvariables}}} \, .
    \end{align*}
    The first term can be characterized by
    \begin{align*}
        \contraction{\gendistributionwith,-\lnof{\hlnformulawith}}
        =\begin{cases}
             0 & \ifspace \gendistribution\models\hlnformula \\
             \infty & \ifspace \gendistribution\not\models\hlnformula
        \end{cases} \, .
    \end{align*}

    The cross entropy is therefore finite, if and only if $\hardlegset\subset\hardlegsetto{\genmean}$ and $\headindexof{\hardlegset}=\restrictionofto{\hardlegindicesto{\genmean}}{\hardlegset}$.
    Among those tuples, only the third term of the cross-entropy varies, where we have
    \begin{align*}
        \lnof{\contraction{\mlnstatccwith,\kcoreof{\hardparam},\actcoreof{\canparam}}} \leq \lnof{\contraction{\mlnstatccwith,\kcoreof{(\hardlegsetto{\genmean},\hardlegindicesto{\genmean})},\actcoreof{\canparam}}}
    \end{align*}
    The minimum of the cross entropy is therefore taken at $\hardlegset=\hardlegsetto{\genmean}$ and $\headindexof{\hardlegset}=\hardlegindicesto{\genmean}$.
\end{proof}

We therefore know
\begin{align*}
    \min_{\canparam\in\parspace} \centropyof{\gendistribution}{\probtensorof{\mlnstat,(\hardlegsetto{\genmean},\hardlegindicesto{\genmean},\canparam)}}
\end{align*}

We can therefore characterize the minimum using \lemref{lem:minCrossEntropyExponential}.

\begin{theorem}
    \label{the:minCrossEntropyHLN}
    Let $\gendistributionwith$ be a distribution, $\mlnstat$ a boolean statistic.%, and choose a subset $\variableset\subset[\seldim]$ and a boolean tuple $\headindexof{\variableset}$.
    We build the mean parameter $\genmeanwith=\contractionof{\gendistributionwith,\sencmlnstatwith}{\selvariable}$ and have the following:
%    For any $\variableset\subset[\seldim]$ and boolean tuple $\headindexof{\variableset}$ we have
    \begin{itemize}
        \item[(1)] If $\genmeanwith \in \sbinteriorof{\meansetof{\mlnstat,\trivbm}}$ then
        \begin{align*}
            \min_{\hybridparamin} \centropyof{\gendistribution}{\probtensorof{\mlnstat,\hybridparam}}
            = \sentropyof{\probtensorof{\mlnstat,(\hardlegsetto{\genmean},\hardlegindicesto{\genmean},\estcanparam)}}
        \end{align*}
        where $\estcanparam=\backwardmapwrtof{\mlnstat,\hlnformulato{\genmeanwith}}{\genmean}$.
        \item[(2)] If $\genmeanwith \notin \sbinteriorof{\meansetof{\mlnstat,\trivbm}}$ and $\genmeanwith\in\closureof{\meansetof{\mlnstat,\trivbm}}$ then there is a sequence $\left(\meanparamofat{n}{\selvariable}\right)_{n\in\nn}\subset\hlnmeanset$ converging coordinatewise to $\genmeanwith$ and
        \begin{align*}
            \min_{\hybridparamin} \centropyof{\gendistribution}{\probtensorof{\mlnstat,\hybridparam}}
            = \lim_{\meanparamofat{n}{\selvariable}\rightarrow\genmeanwith}
            \sentropyof{\probtensorof{\mlnstat,(\hardlegsetto{\meanparamof{n}},\hardlegindicesto{\meanparamof{n}},\canparamof{n})}}\,
        \end{align*}
        where $\canparamof{n}=\backwardmapwrtof{\mlnstat,\hlnformulato{\meanparamof{n}}}{\meanparamof{n}}$.
        \item[(3)] If $\genmeanwith\notin\closureof{\meansetof{\mlnstat,\trivbm}}$ then
        \begin{align*}
            \min_{\hybridparamin} \centropyof{\gendistribution}{\probtensorof{\mlnstat,\hybridparam}}
            = \infty \, .
        \end{align*}
    \end{itemize}
\end{theorem}
\begin{proof}
    This follows from \lemref{lem:minCrossEntropyExponential}.
\end{proof}

%Based on \lemref{lem:minCrossEntropyExponential} we can characterize the solution of the Maximum Likelihood Problem \probref{prob:minCrossEntropyHLN} with respect to \HybridLogicNetworks{}.
%
%\begin{theorem}
%    Let $\gendistribution$ be any distribution, such that $\genmeanwith$ is reproducable by a Hybrid Logic Network.
%    Then the solution of the Maximum Likelihood Problem \probref{prob:minCrossEntropyHLN} is the Hybrid Logic Network $\probtensorof{\mlnstat,\backwardmapwrtof{\mlnstat,\hlnformula}{\genmeanwith},\hlnformula}$.
%\end{theorem}
%\begin{proof}
%    We exploit the parameterization of \HybridLogicNetworks{} by tuples $\hlnformulaparams$ (parametrizing the corresponding \HardLogicNetwork{}) and canonical parameters $\canparamwith$ (parametrizing the corresponding \MarkovLogicNetwork{}).
%    The cross entropy minimization is then
%    \begin{align*}
%        \min_{\variableset\subset[\seldim]}\min_{\headindexof{\variableset}\in\bigtimes_{\selindex\in\variableset}[2]} \left(\min_{\canparamwithin} \centropyof{\gendistribution}{\probtensorof{\mlnstat,\canparamwith,\hlnformula}}\right) \, .
%    \end{align*}
%    For any pair $\variableset,\headindexof{\variableset}$ we apply the lemma above and get a characterization of the inner minimum
%    \begin{align*}
%        \min_{\canparamwithin} \centropyof{\gendistribution}{\probtensorof{\mlnstat,\canparamwith,\hlnformula}}
%    \end{align*}
%    dependent on the stated three cases.
%    There is exactly one face $\genfaceset$, for which $\genmeanwith\in\sbinteriorof{\genfaceset}$ and by assumption of reproducability there is a parameter tuple $\hlnformulaparams$ such that $\hlnformula$ is the face measure of $\genfaceset$.
%    For this parameter tuple we have
%    \begin{align*}
%        \min_{\canparamwithin} \centropyof{\gendistribution}{\probtensorof{\mlnstat,\canparamwith,\hlnformula}}
%        = -\sentropyof{\probtensorof{\mlnstat,\backwardmapwrtof{\mlnstat,\hlnformula}{\genmean},\hlnformula}} \, .
%    \end{align*}
%    The following two arguments show that this is the minimum among all other faces.
%    If for another tuple $\sechlnformulaparams$ we have $\genmeanwith\notin\bmrealprobof{\hlnformulaof{\sechlnformulaparams}}$ (i.e. the third case in \lemref{lem:minCrossEntropyExponential}) then the respective inner minimum is $\infty$ and the outer minimum is not taken.
%    If for another tuple $\sechlnformulaparams$ we have $\genmeanwith\in\bmrealprobof{\hlnformulaof{\sechlnformulaparams}}$ but $\genmeanwith\notin\sbinteriorof{\genmeanset\cup\cubeface^{\sechlnformulaparams}}$ (i.e. the second case in \lemref{lem:minCrossEntropyExponential}), then the face $\genmeanset\cap\cubeface^{\hlnformulaparams}$ is contained in $\genmeanset\cap\cubeface^{\sechlnformulaparams}$ and the minimum is taken on that face.
%\end{proof}


When $\genmeanwith$ is not reproduceable by a \HybridLogicNetwork{}, we are in the case where the smallest face, such that $\genmeanwith$ is contained is not an intersection of $\genmeanset$ with a cube face.
In this case, there is no solution of the Maximum Likelihood \probref{prob:minCrossEntropyHLN} in the set of \HybridLogicNetworks{}, since the minimum is not taken.
The reason for this lies in the expressivity problem of \HybridLogicNetworks{}, which do not reproduce the interior of such faces, but tend in a limit of large canonical parameters to any mean parameter on such faces.

%%% GENERALIZATION: TO BE SHOWN
%\begin{theorem}
%    Let $\gendistribution$ be any distribution, $\genmeanwith=\contractionof{\gendistribution,\sencmlnstatwith}{\selvariable}$ and $\hlnfaceset$ be the smallest face of $\hlnmeanset$, which contains $\genmeanwith$.
%    Then the cross entropy minimizer over
%    \begin{align*}
%        \probtensorset = \bigcup_{\facecondset} \expfamilyof{\mlnstat,\hlnfacemeasure}
%    \end{align*}
%    is the distribution $\probtensorof{\mlnstat,\backwardmapwrtof{\mlnstat,\hlnfacemeasure}{\genmeanwith},\hlnformula}$, where $\hlnfacemeasure$ is the face measure of $\hlnfaceset$.
%\end{theorem}
%\begin{proof}
%
%\end{proof}


%We can instead state a Maximum Likelihood Problem over $\maxrealizabledistsof{\mlnstat}$, which provides enough expressivity to represent all faces.
%\begin{theorem}
%    Let $\gendistribution$ be any distribution, $\genmeanwith=\contractionof{\gendistribution,\sencmlnstatwith}{\selvariable}$ and $\genfaceset$ be the smallest face of $\genmeanset$, which contains $\genmeanwith$.
%    Then the solution of the Maximum Likelihood Problem \probref{prob:minCrossEntropyHLN} over $\realizabledistsof{\mlnstat,\maxgraph}$ is $\probtensorof{\mlnstat,\backwardmapwrtof{\mlnstat,\hlnfacemeasure}{\genmeanwith},\hlnfacemeasure}$, where $\hlnfacemeasure$ is the face measure of $\genfaceset$.
%\end{theorem}
%\begin{proof}
%    Again by application of \lemref{lem:minCrossEntropyExponential} on each face of $\genmeanset$.
%    %The outer minimum is taken at the smallest face
%\end{proof}







\sect{Forward and backward mappings}

Forward and backward mappings have been introduced for exponential families in \charef{cha:probRepresentation}.
We now generalize them to \HybridLogicNetworks{}, which are parametrized by tuples $\hybridparamin$.

%\subsect{Representation as Computation Activation Networks} %% Now in Represeentation Chapter
%
%We parametrize each elementary activation tensors $\extnetat{\headvariables}=\bigotimes_{\selindexin}\hypercoreofat{\selindex}{\headvariableof{\selindex}}$ by a tuple
%$\hybridparam$ as follows
%\begin{itemize}
%    \item The axis set of hard leg vectors
%    \begin{align*}
%        \variableset = \Big\{ \selindex \wcols \selindexin \ncond \hypercoreofat{\selindex}{\indexedheadvariableof{\selindex}=0}=0 \,\,\text{or}\,\,  \hypercoreofat{\selindex}{\indexedheadvariableof{\selindex}=1}=0  \Big\}
%    \end{align*}
%    \item The indices of the hard leg vector $\headindexof{\variableset}$ where for $\selindex\in\variableset$
%    \begin{align*}
%        \headindexof{\selindex} =
%        \begin{cases}
%            0 & \ifspace \hypercoreofat{\selindex}{\headvariableof{\selindex}=1}=0 \\
%            1 & \ifspace \hypercoreofat{\selindex}{\headvariableof{\selindex}=0}=0
%        \end{cases}
%    \end{align*}
%    Note, that exactly one of the cases is true, since both would imply a vanishing activation tensor.
%    \item A canonical parameter $\canparamat{\selvariable}$ where $\canparamat{\indexedselvariable}=0$ for $\selindex\in\variableset$ and for $\selindex\notin\variableset$
%    \begin{align*}
%        \canparamat{\indexedselvariable} = \lnof{
%            \frac{\hypercoreofat{\selindex}{\headvariableof{\selindex}=1}}{\hypercoreofat{\selindex}{\headvariableof{\selindex}=0}}
%        }
%    \end{align*}
%\end{itemize}
%Conversely, for a tuple $\hybridparam$ we define leg vectors
%\begin{align*}
%    \hypercoreofat{\selindex}{\headvariableof{\selindex}} =
%    \begin{cases}
%        \onehotmapofat{\headindexof{\selindex}}{\headvariableof{\selindex}} & \text{if} \quad \selindex\in\variableset \\
%        \expof{\canparamat{\indexedselvariable}\cdot \indexinterpretationof{\headvariableof{\selindex}}} & \text{else}
%    \end{cases}
%\end{align*}
%and an activation tensor
%\begin{align*}
%    \tnetofat{\hybridparam}{\headvariables}
%    =\bigotimes_{\selindexin}\hypercoreofat{\selindex}{\headvariableof{\selindex}} \, .
%\end{align*}
%
%Note that the tuple $\hybridparam$ to a elementary activation tensor determines the corresponding \HybridLogicNetwork{}, since it determines the activation tensor up to an irrelevant scalar multiplication.


%\subsect{Definition}

\begin{definition}
    The extended canonical parameter set to a boolean statistic is the set
    \begin{align*}
        \hybridparamset\coloneqq
        \{\hardparam)\wcols \variableset\subset[\seldim]\ncond \headindexof{\variableset}\in\bigtimes_{\selindex\in\variableset}[2]\} \times \parspace \, .
    \end{align*}
    The forward map for a \HybridLogicNetwork{} is
    \begin{align*}
        \forwardmapwrt{\mlnstat} :  \hybridparamset
        \rightarrow \hlnmeanset \subset \parspace
    \end{align*}
    where for $\hybridparamin$
    \begin{align*}
        \forwardmapwrtof{\mlnstat}{\hybridparam}
        = \contractionof{
            \normalizationof{\tnetofat{\hybridparam}{\headvariables},\bencodingofat{\mlnstat}{\headvariables,\shortcatvariables}}{\shortcatvariables}
            ,\sencodingofat{\mlnstat}{\shortcatvariables,\selvariable}}{\selvariable} \, .
    \end{align*}

    A backward map for a \HybridLogicNetwork{} is any map
    %\begin{align*}
    %    \backwardmapwrt{\mlnstat} :  \imageof{\forwardmapwrt{\mlnstat}} \rightarrow \eltnset
    %\end{align*}
    such that for any $\hybridparam\in\hybridparamset$ we have $\backwardmapwrtof{\mlnstat}{\forwardmapwrtof{\mlnstat}{\tnetof{\elgraph}}}$.

\end{definition}

% Expressivity implying image
From the expressivity study in \charef{cha:networkRepresentation} we know that for any $\meanparamwith$ there is a $\hybridparam\in\hybridparamset$ with $\forwardmapwrtof{\mlnstat}{\hybridparam}=\meanparamwith$, if and only if $\meanparamwith\in\elhlnmeanset$.
In particular, the this implies that the image of $\forwardmapwrt{\mlnstat}$ is the subset $\elhlnmeanset\subset\hlnmeanset$, which is the union of cube face interiors.

A backward map can be constructed as follows:
\begin{itemize}
    \item Choose $\variableset=\{\selindex\wcols\meanparamat{\indexedselvariable}\in\{0,1\}\}$, and for $\selindex\in\variableset$ $\headindexof{\selvariable}=\meanparamat{\indexedselvariable}$
    %\item and build
    %\begin{align*}
    %    \kcoreofat{\selindex}{\headvariableof{\selindex}}
    %\end{align*}
    \item Use the backward map of the exponential family $\expfamilyof{\mlnstat,\hlnformula}$ to compute
    \begin{align*}
        \canparamat{\selvariable} = \backwardmapwrtof{\mlnstat,\hlnformula}{\meanparamwith}
    \end{align*}
    %\item Set
    %\begin{align*}
    %    \hypercoreofat{\selindex}{\headvariableof{\selindex}} = \contractionof{\actcorewith,\kcoreofat{\selindex}{\headvariableof{\selindex}}}{\headvariableof{\selindex}}
    %\end{align*}
\end{itemize}

\subsect{Backward Maps in Closed Form}

\red{For the universal and the atomic family we have $\elhlnmeanset=\hlnmeanset$.
Furthermore, we can provide the backward mapping in a closed form expression.}

% Closed form availability
We recall from \charef{cha:probReasoning}, that while forward mappings are always in closed form by contractions, backward mapping in general do not have a closed form representation.
Instead, the backward map is in general implicitly characterized by a maximum entropy problem constrained to matching expected sufficient statistics.
We investigate in this section specific examples, where closed forms are available for both.
In these cases, parameter estimation can thus be solved by application of the inverse on the expected sufficient statistics with respect to the empirical distribution, and iterative algorithms can be avoided.

% Usage
%When the backward map $\backwardmap$ is available in closed form, we directly get optimal parameters by the inversion acting on the satisfaction rate and can avoid iterative algorithms of parameter estimation.

\subsubsect{Maxterms and Minterms}

Minterms (respectively maxterms) are ways in propositional logics to get a syntactical formula representation based on a formula to each world which is a model (respectively fails to be a model).
We have already studied in \secref{sec:MLNMaxMintermRep} how to represent any positive distribution by a distribution in the family of minterms (respectively maxterms), see \theref{the:maximalClausesRepresentation}.
Here we extend to the representation of distributions with arbitrary supports and provide forward and backward maps.

We use the tuple enumeration of the maxterms and minterms by $\atomstates$ introduced in \secref{sec:termClauseDecomposition}.
With respect to this enumeration the canonical parameters and mean parameters are tensors in $\bigotimes_{\atomenumeratorin}\rr^2$.

\begin{theorem}
    For the \HybridLogicNetworks{} to the minterm and maxterm statistics
    \begin{align*}
        \mintermformulaset \coloneqq \{ \mintermof{\shortcatindices} \wcols \shortcatindices\in\atomstates\}
        \andspace
        \maxtermformulaset \coloneqq \{ \maxtermof{\shortcatindices} \wcols \shortcatindices\in\atomstates\}
    \end{align*}
    we have the forward maps
    \begin{align*}
        \forwardmapwrtof{\mlnmintermsymbol}{\hybridparam}
        = \normalizationof{\bencodingofat{\mintermformulaset}{\headvariableof{\mlnmintermsymbol},\shortcatvariables},
            \tnetofat{\hybridparam}{\headvariableof{\mlnmintermsymbol}},\identityat{\shortcatvariables,\selvariableof{[\atomorder]}}}{\selvariableof{[\atomorder]}}
    \end{align*}
    and
    \begin{align*}
        \forwardmapwrtof{\mlnmaxtermsymbol}{\hybridparam}
        = \onesat{\selvariableof{[\atomorder]}} - \normalizationof{\bencodingofat{\maxtermformulaset}{\headvariableof{\mlnmaxtermsymbol},\shortcatvariables},
            \tnetofat{\hybridparam}{\headvariableof{\mlnmaxtermsymbol}},\identityat{\shortcatvariables,\selvariableof{[\atomorder]}}}{\selvariableof{[\atomorder]}} \, .
    \end{align*}

    Further, the map
    \begin{align*}
        \backwardmapwrtof{\mlnmintermsymbol}{\meanparamat{\selvariableof{[\atomorder]}}}
        = \Big(\Big\{ \selindex \wcols \selindexin \ncond \meanparamat{\indexedselvariable}=0 \Big\},0_{\variableset},\lnof{\meanparamat{\selvariableof{[\atomorder]}}}\Big)
    \end{align*}
    (we set here $\lnof{0}=0$) is a backward map for the minterm statistic.

    The map
    \begin{align*}
        \backwardmapwrtof{\mlnmaxtermsymbol}{\meanparamat{\selvariableof{[\atomorder]}}}
        = \Big(\Big\{ \selindex \wcols \selindexin \ncond \meanparamat{\indexedselvariable}=1 \Big\},1_{\variableset},-\lnof{\meanparamat{\selvariableof{[\atomorder]}}}\Big)
    \end{align*}
    (we set here $\lnof{0}=0$) is a backward map for the minterm statistic.
\end{theorem}
\begin{proof}
    The minterm statistic $\mintermformulaset$ has a selection encoding
    \begin{align*}
        \sencodingofat{\mintermformulaset}{\shortcatvariables,\selvariableof{[\atomorder]}}
        = \identityat{\shortcatvariables,\selvariableof{[\atomorder]}}
    \end{align*}
    and the mean parameter to any distribution $\probwith$ has therefore the coordinates to $\selindexof{[\atomorder]}\in\atomstates$ by
    \begin{align*}
        \meanparamat{\indexedselvariableof{[\atomorder]}}
        = \contractionof{\identityat{\shortcatvariables,\selvariableof{[\atomorder]}}}{\indexedselvariableof{[\atomorder]}}
        = \probat{\shortcatvariables=\selindexof{[\atomorder]}} \, .
    \end{align*}
    For the maxterm statistic $\maxtermformulaset$ we analogously have
    \begin{align*}
        \sencodingofat{\maxtermformulaset}{\shortcatvariables,\selvariableof{[\atomorder]}}
        = \onesat{\shortcatvariables,\selvariableof{[\atomorder]}}-\identityat{\shortcatvariables,\selvariableof{[\atomorder]}}
    \end{align*}
    and thus the mean parameter to any distribution $\probwith$ has therefore the coordinates to $\selindexof{[\atomorder]}\in\atomstates$ by
    \begin{align*}
        \meanparamat{\indexedselvariableof{[\atomorder]}}
        = 1-\probat{\shortcatvariables=\selindexof{[\atomorder]}} \, .
    \end{align*}
    The claim on the forward maps follows for
    \begin{align*}
        \probwith=\normalizationof{\bencodingofat{\maxtermformulaset}{\headvariableof{\mlnmaxtermsymbol},\shortcatvariables},
            \tnetofat{\elgraph}{\headvariableof{\mlnmaxtermsymbol}}}{\shortcatvariables} \, . & \qedhere
    \end{align*}
\end{proof}

% Fitting arbitrary distributions
Any probability distribution can thus be represented by a \HybridLogicNetwork{} in the minterm statistic, as well as in the maxterm statistic.
Thus, we have identified sets of $2^{\atomorder}$ formulas, which is rich enough to fit any distribution.

\subsubsect{Atomic formulas}

% Repeat atomic formulas
Let us now derive a closed form backward mapping for the statistic
\begin{align*}
    \atomformulaset \coloneqq \{\atomicformulaof{\atomenumerator} \wcols \atomenumeratorin\}
\end{align*}
of atomic formulas, which coincides with a variable selection map.
The selection encoding of this statistic is the tensor
\begin{align*}
    \sencodingof{\atomformulaset}{\shortcatvariables,\selvariable}
    = \sum_{\atomenumeratorin} \onehotmapofat{1}{\catvariableof{\atomenumerator}} \otimes \onesat{\catvariableof{[\atomorder]/\{\atomenumerator\}}} \otimes \onehotmapofat{\atomenumerator}{\selvariable} \, .
\end{align*}
For each probability distribution $\probwith$ we the corresponding mean parameter has coordinates at $\atomenumeratorin$ by
\begin{align*}
    \meanparamat{\selvariable=\atomenumerator} = \probat{\catvariableof{\atomenumerator}=1}  \, .
\end{align*}

\begin{theorem}
    For the \HybridLogicNetworks{} to the statistic of atomic formulas $\atomformulaset$ we have the forward map
    \begin{align*}
        \forwardmapwrtof{\mlnatomsymbol}{\hybridparam}[\selvariable=\atomenumerator]
        = \begin{cases}
              \headindexof{\atomenumerator} & \text{if} \quad \atomenumerator\in\variableset \\
              \frac{\expof{\canparamat{\selvariable=\atomenumerator}}}{1+\expof{\canparamat{\selvariable=\atomenumerator}}} & \text{if} \quad \atomenumerator\notin\variableset % sigmoid
        \end{cases}
    \end{align*}
\end{theorem}
\begin{proof}
    Let $\hybridparam\in\hybridparamsetofdim{\atomorder}$ and denote the
    \begin{align*}
        \meanparamat{\selvariable} = \forwardmapwrtof{\mlnatomsymbol}{\hybridparam} \, .
    \end{align*}
    By definition we have for any $\atomenumeratorin$
    \begin{align*}
        \meanparamat{\selvariable=\atomenumerator}
        &= \contractionof{\sencodingof{\atomformulaset}{\shortcatvariables,\selvariable},
            \normalizationof{\bencodingofat{\atomformulaset}{\headvariableof{[\atomorder]},\shortcatvariables},\hypercoreofat{\hybridparam}{\headvariableof{[\atomorder]}}}{\shortcatvariables}
        }{\selvariable=\atomenumerator} \\
        &= \contraction{\onehotmapofat{1}{\catvariableof{\atomenumerator}},
            \normalizationof{
                \bencodingofat{\atomicformulaof{\atomenumerator}}{\headindexof{\atomenumerator},\catvariableof{\atomenumerator}},
                \hypercoreof{\selindex}{\headvariableof{\atomenumerator}}}{\catvariableof{\atomenumerator}}
        }
    \end{align*}
    Now, if $\atomenumerator\in\variableset$ we have $\hypercoreof{\selindex}{\headvariableof{\atomenumerator}}=\onehotmapofat{\headindexof{\atomenumerator}}{\headvariableof{\atomenumerator}}$ and $\meanparamat{\selvariable=\atomenumerator}=\headindexof{\atomenumerator}$.
    If $\atomenumerator\notin\variableset$ then $\hypercoreof{\selindex}{\headvariableof{\atomenumerator}}=\actcoreofat{\atomenumerator,\canparamat{\selvariable=\atomenumerator}}{\headvariableof{\atomenumerator}}$ and
    \begin{align*}
        \meanparamat{\selvariable=\atomenumerator} = \frac{\expof{\canparamat{\selvariable=\atomenumerator}}}{1+\expof{\canparamat{\selvariable=\atomenumerator}}} \, . & \qedhere
    \end{align*}
\end{proof}

A backward map to the atomic statistic is given by
\begin{align*}
    \backwardmapwrtof{\mlnatomsymbol}{\meanparamat{\selvariable}}
    =\hybridparam
    = \Big(\{\atomenumerator\wcols\atomenumeratorin\meanparamat{\selvariable=\atomenumerator}\in\{0,1\}\}, [\meanparamat{\selvariable=\atomenumerator}\wcols ] \Big)
\end{align*}
where
\begin{itemize}
    \item $\variableset = \{\atomenumerator\wcols\atomenumeratorin\meanparamat{\selvariable=\atomenumerator}\in\{0,1\}\}$
    \item For $\atomenumerator\in\variableset$ $\headindexof{\atomenumerator}=\meanparamat{\selvariable=\atomenumerator}$
    \item For $\atomenumerator\in\variableset$ we have $\canparamat{\selvariable=\atomenumerator}=0$ and for $\atomenumerator\notin\variableset$
    \begin{align*} % logit
        \canparamat{\selvariable=\atomenumerator}
        = \lnof{\frac{\meanparamat{\selvariable=\atomenumerator}}{1-\meanparamat{\selvariable=\atomenumerator}}}
    \end{align*}
\end{itemize}

The forward and backward map on the soft atomic formulas are the coordinatewise sigmoid and logit, respectively.

%\begin{theorem}
%    Given a \MarkovLogicNetwork{} with the statistic $\atomformulaset$ of atomic formulas, the forward mapping from canonical parameters to mean parameters is the coordinatewise sigmoid, that is
%    \[ \forwardmapwrtof{\mlnatomsymbol}{\canparamat{\selvariable}} = \frac{\expof{\canparamat{\selvariable}}}{\onesat{\selvariable}+\expof{\canparamat{\selvariable}}}   \]
%    where the quotient is performed coordinatewise.
%
%    A backward mapping is the coordinatewise logit, that is
%    \[ \backwardmapwrt{\mlnatomsymbol}(\meanparamwith)
%    = \lnof{\frac{
%        \meanparamwith
%    }{
%        \onesat{\selvariable}-\meanparamwith
%    }}  \, . \]
%\end{theorem}
%\begin{proof}
%    We have for any $\canparamat{\selvariable}\in\rr^{\atomorder}$
%    \[ \probofat{(\atomformulaset,\canparam)}{\shortcatvariables}
%    = \bigotimes_{\atomenumeratorin} \normalizationof{\expof{\canparamat{\selvariable=\atomenumerator}\cdot \atomicformulaof{\atomenumerator}}}{\catvariableof{\atomenumerator}}  \, . \]
%
%
%    For any $\atomenumeratorin$ it therefore holds, that
%    \begin{align*}
%        \forwardmapwrtof{\mlnatomsymbol}{\canparamat{\selvariable}}[\selvariable=\atomenumerator]
%        &=\contraction{\atomicformulaof{\atomenumerator},  \probofat{(\atomformulaset,\canparam)}{\shortcatvariables}} \\
%        &=\contraction{\atomicformulaof{\atomenumerator},  \normalizationof{\expof{\canparamat{\selvariable=\atomenumerator}\cdot \atomicformulaof{\atomenumerator}}}{\catvariableof{\atomenumerator}}} \\
%        & = \frac{\expof{\canparamat{\selvariable=\atomenumerator}}}{1+\expof{\canparamat{\selvariable=\atomenumerator}}} \, .
%    \end{align*}
%
%    Since the coordinatewise logit is the inverse function of the coordinatewise sigmoid the map
%    \begin{align*}
%        \backwardmapwrtof{\mlnatomsymbol}{\meanparamwith}[\selvariable=\atomenumerator]
%        & = \lnof{\frac{\meanparamat{\selvariable=\atomenumerator}}{1- \meanparamat{\selvariable=\atomenumerator}}}
%    \end{align*}
%    satisfies for any $\meanparam$ in the image of the forward map
%    \begin{align*}
%        \forwardmapwrt{\mlnatomsymbol}(\backwardmapwrt{\mlnatomsymbol}(\meanparam)) = \meanparam
%    \end{align*}
%    and is therefore a backward map.
%\end{proof}
%
%
%% Representation by selection tensor networks
%In a selection tensor networks they are represented by a single neuron with identity connective and variable selection to all atoms.
%We will investigate such examples in more detail in \charef{cha:sparseRepresentation}, where atomic formulas \MarkovLogicNetworks{} are specific cases of monomial decomposition of order 1.

% Interpretation of the result as independence approximation
%The maximum likelihood estimator of a positive probability distribution by the MLN of atomic formulas is therefore the tensor product of the marginal distributions.

%The Kullback-Leibler divergence between the distribution and its projection is the mutual information of the atoms, see for example Chapter~8 in \cite{mackay_information_2003}.

%\begin{remark}[Decomposition into systems of atomic networks]
%    \red{By Independence Decomposition we reduce to a system of atomic MLN.
%    The minterms of such MLNs are the literals.
%    By redundancy (literals sum up to $\ones$), it suffices to take only the positive or the negative literal.
%    }
%%	We set the weights of $\weightof{\lnot\atomicformulaof{\atomenumerator}}=0$ (corresponding with a gauge normalization of the energy offset symmetry). % Not needed!
%\end{remark}





\sect{Alternating Moment Matching}\label{sec:alternatingParEstMLN}

We now derive an algorithm for the Maximum Likelihood Estimation in case of \HybridLogicNetworks{}.
To this end, we first solve local cross-entropy minimization problems, which are then alternated to find global solutions.

%Parameter estimation is the Maximum Likelihood Problem for exponential families.
%It is solved by the backward map, when the mean parameter is in the interior of the mean parameter polytope.

\subsect{Local updates}

Let us now varying a distribution $\secprobwith$ by adding an additional feature $\formula$
\begin{align*}
    \probofat{\hypercore}{\shortcatvariables}
    \coloneqq \normalizationof{\secprobwith,\formulaccwith,\hypercoreat{\formulavar}}{\shortcatvariables} \, .
\end{align*}
Note, that the normalization exists for positive $\hypercoreat{\formulavar}$ and if $\contraction{\secprobtensor,\exformula}\notin\{0,1\}$ then also for any non-vanishing $\hypercoreat{\formulavar}$.
If $\contraction{\secprobtensor,\exformula}\in\{0,1\}$, then the $\probof{\hypercore}$ is constant among $\hypercoreat{\formulavar}$, when it exists.

We want to solve the local cross-entropy minimization problem
\begin{align*}
    \min_{\hypercore} \centropyof{\empdistribution}{\probof{\hypercore}} \,.
\end{align*}
If $\contraction{\secprobtensor,\exformula}\in\{0,1\}$ then the minimum is taken at any $\hypercoreat{\formulavar}$ such that $\probof{\hypercore}$ exists.

\begin{lemma}
    \label{lem:localHybridParamUpdate}
    Let $\empdistribution$, $\secprobtensor$ be distributions and $\exformula$ a formula such that $\contraction{\secprobtensor,\exformula}\notin\{0,1\}$.
    If $\contraction{\empdistribution,\exformula}\in\{0,1\}$ the solution of the local cross-entropy minimization is
    \begin{align*}
        \hypercoreat{\formulavar} = \onehotmapofat{\contraction{\empdistribution,\exformula}}{\formulavar}
    \end{align*}
    If $\contraction{\empdistribution,\exformula}\notin\{0,1\}$ the solution is
    \begin{align*}
        \hypercoreat{\formulavar} = \actcoreofat{\canparam}{\formulavar}
    \end{align*}
    where
    \begin{align*}
        \canparam = \lnof{
            \frac{\contraction{\empdistribution,\exformula}}{(1-\contraction{\empdistribution,\exformula})}
            \cdot \frac{1-\contraction{\secprobtensor,\exformula}}{\contraction{\secprobtensor,\exformula}}
        } \, .
    \end{align*}
\end{lemma}
\begin{proof}
    The cross entropy is decomposed into
    \begin{align*}
        \centropyof{\empdistribution}{\probof{\hypercore}}
        &= \centropyof{\empdistribution}{\secprobtensor} \\
        &\quad + \contraction{\empdistributionwith,\formulaccwith,-\lnof{\hypercoreat{\formulavar}}} \\
        &\quad + \lnof{\contraction{\secprobwith,\formulaccwith,\hypercoreat{\formulavar}}}\, .
    \end{align*}
    Since the first term is constant among $\hypercore$, we focus on the minimization of the second and third term.
    For each $\hypercoreat{\formulavar}$ and its boolean support $\kcoreat{\formulavar}\coloneqq\greaterzeroof{\hypercoreat{\formulavar}}$ we find $\lambda>0$ and $\canparam\in\rr$ such that
    \begin{align*}
        \hypercoreat{\formulavar}
        = \lambda\cdot\contractionof{\kcoreat{\formulavar},\actcoreof{\canparam}}{\formulavar} \, .
    \end{align*}
    Given this parametrization we have
    \begin{align*}
        & \contraction{\empdistributionwith,\formulaccwith,-\lnof{\hypercoreat{\formulavar}}}
        + \lnof{\contraction{\secprobwith,\formulaccwith,\hypercoreat{\formulavar}}} \\
        & \quad  = \contraction{\empdistributionwith,\formulaccwith,-\lnof{\kcoreat{\formulavar}}} - \canparam\cdot\contraction{\empdistributionwith,\formulawith} \\
        & \quad \quad  + \lnof{\contraction{\secprobwith,\formulaccwith,\kcoreat{\formulavar},\actcoreofat{\canparam}{\formulavar}}} \, .
    \end{align*}
    The minimum over $\kcore$ is taken at
    \begin{align*}
        \kcoreat{\formulavar} =
        \begin{cases}
            \tbasisat{\formulavar} & \ifspace \contraction{\empdistribution,\exformula}=1 \\
            \fbasisat{\formulavar} & \ifspace \contraction{\empdistribution,\exformula}=0 \\
            \onesat{\formulavar} & \text{else}
        \end{cases} \, .
    \end{align*}
    If the optimal $\kcore$ is not the trivial vector $\onesat{\formulavar}$, the parameter $\canparam\in\rr$ does not influence the distribution and we can arrive at the claim when choosing $\canparam=0$.
    If the optimal $\kcore$ is trivial, we optimize further over $\actcoreof{\canparam}$
    \begin{align*}
        \min_{\canparam\in\rr} \contraction{\empdistributionwith,\formulaccwith,-\lnof{\hypercoreat{\formulavar}}}
        + \lnof{\contraction{\secprobwith,\formulaccwith,\actcoreofat{\canparam}{\formulavar}}} \, .
        \, .
    \end{align*}
    The derivation of the objective is
    \begin{align*}
        & \difofwrt{\contraction{\empdistributionwith,\formulaccwith,-\lnof{\hypercoreat{\formulavar}}}
        + \lnof{\contraction{\secprobwith,\formulaccwith,\actcoreofat{\canparam}{\formulavar}}}}{\canparam} \\
        & \quad = \contraction{\empdistributionwith,\formulawith}
        - \contraction{\probofat{\actcoreof{\canparam}}{\shortcatvariables},\formulawith} \\
        & \quad = \contraction{\empdistribution,\exformula}
        - \frac{\expof{\canparam} \cdot \contraction{\secprobwith,\formulawith}}{
            \expof{\canparam} \cdot \contraction{\secprobwith,\formulawith} + (1-\contraction{\secprobwith,\formulawith})
        } \, .
    \end{align*}
    The derivative vanished thus at the unique minimum of the cross entropy at
    \begin{align*}
        \canparam = \lnof{
            \frac{\contraction{\empdistribution,\exformula}}{(1-\contraction{\empdistribution,\exformula})}
            \cdot \frac{(1-\contraction{\secprobtensor,\exformula})}{\contraction{\secprobtensor,\exformula}}
        } \, . & \qedhere
    \end{align*}
\end{proof}


\subsubsect{\MarkovLogicNetworks{}}

In case of boolean statistics, we can provide a particular simple implementation of the Alternating Moment Matching Algorithm~\ref{alg:AMM}.
In the following section we will then generalize to \HybridLogicNetworks{} by allowing hard cores.
%The moment matching condition of \lemref{lem:mmContractionEquation} simplifies as follows.

Iteratively we update leg vectors of the activation tensor.
This is the above local variation with
\begin{align*}
    \secprobwith = \normalizationof{\{\bencodingof{\enumformula} \wcols \selindexin\}
    \cup\{\actcoreof{\tilde{\selindex},\canparamat{\selvariable=\tilde{\selindex}}} \wcols \tilde{\selindex} \in [\seldim], \tilde{\selindex}\neq\selindex\}
    \cup\{\basemeasure\}}{\headvariableof{\selindex}}
\end{align*}

To solve the moment matching condition at a formula $\enumformula$ we do not have to compute the normalization constant, since we only require the quotient
\begin{align*}
    \frac{1-\contraction{\secprobtensor,\exformula}}{\contraction{\secprobtensor,\exformula}} \, .
\end{align*}
The updated canonical coordinate is thus computed as
%$\secprobtensor$
\begin{align}
    \label{sol:momentMatchingExformula}
    \indexedcanparam = \lnof{
        \frac{\datameanat{\indexedselvariable}}{(1-\datameanat{\indexedselvariable})}
        \cdot \frac{\hypercoreat{\headvariableof{\selindex}=0}}{\hypercoreat{\headvariableof{\selindex}=1}}
    }
\end{align}
where by $\hypercoreat{\headvariableof{\selindex}}$ we denote the contraction
\begin{align*}
    \hypercoreat{\headvariableof{\selindex}}
    = \contractionof{\{\bencodingof{\enumformula} \wcols \selindexin\}
    \cup\{\actcoreof{\tilde{\selindex},\canparamat{\selvariable=\tilde{\selindex}}} \wcols \tilde{\selindex} \in [\seldim], \tilde{\selindex}\neq\selindex\}
    \cup\{\basemeasure\}}{\headvariableof{\selindex}} \, .
\end{align*}

%%refine \lemref{lem:mmContractionEquation} in the following.
%\begin{lemma}
%    \label{ref:lemMMinMLN}
%    Let there be a base measure $\basemeasure$, a formula selecting map $\formulaset=\{\enumformula \wcols \selindexin\}$ and a canonical parameter $\canparam$, and choose $\selindexin$ such that $\enumformula  \notin \{\onesat{\shortcatvariables},\zerosat{\shortcatvariables}\}$.
%    The moment matching condition relative to $\canparamwith$, and $\datameanat{\indexedselvariable}\in(0,1)$ is then satisfied, if for all $\selindexin$
%    \begin{align}
%        \label{sol:momentMatchingExformula}
%        \indexedcanparam = \lnof{
%            \frac{\datameanat{\indexedselvariable}}{(1-\datameanat{\indexedselvariable})}
%            \cdot \frac{\hypercoreat{\headvariableof{\selindex}=0}}{\hypercoreat{\headvariableof{\selindex}=1}}
%        }
%    \end{align}
%    where by $\hypercoreat{\headvariableof{\selindex}}$ we denote the contraction
%    \begin{align*}
%        \hypercoreat{\headvariableof{\selindex}}
%        = \contractionof{\{\bencodingof{\enumformula} \wcols \selindexin\}
%        \cup\{\actcoreof{\tilde{\selindex},\canparamat{\selvariable=\tilde{\selindex}}} \wcols \tilde{\selindex} \in [\seldim], \tilde{\selindex}\neq\selindex\}
%        \cup\{\basemeasure\}}{\headvariableof{\selindex}} \, .
%    \end{align*}
%\end{lemma}
%\begin{proof}
%    Since $\imageof{\enumformula}\subset[2]$ we have
%    \begin{align*}
%        \idrestrictedto{\imageof{\enumformula}} = \onehotmapofat{1}{\headvariableof{\selindex}}
%    \end{align*}
%    and the moment matching condition is by \lemref{lem:mmContractionEquation} satisfied if
%    \begin{align*}
%        \contraction{\actcorewith, \onehotmapofat{1}{\headvariableof{\selindex}}, \hypercoreat{\headvariableof{\selindex}}}
%        = \contraction{\actcorewith,\hypercoreat{\headvariableof{\selindex}}} \cdot \datameanat{\indexedselvariable} \, .
%    \end{align*}
%    This is equal to
%    \begin{align*}
%        \expof{\canparamat{\indexedselvariable}} \cdot \hypercoreat{\headvariableof{\selindex}=1}
%        = \left( \expof{\canparamat{\indexedselvariable}} \cdot \hypercoreat{\headvariableof{\selindex}=1} + \hypercoreat{\headvariableof{\selindex}=0} \right) \cdot \datameanat{\indexedselvariable} \, .
%    \end{align*}
%    Rearranging the equations this is equal to
%    \begin{align*}
%        \hypercoreat{\headvariableof{\selindex}}
%        = \contractionof{\{\bencodingof{\enumformula}\}
%        \cup\{\actcoreof{\tilde{\selindex}} : \tilde{\selindex} \in [\seldim], \tilde{\selindex}\neq\selindex\}
%        \cup\{\basemeasure\}}{\selvariable} \, .
%    \end{align*}
%    We notice that the right side is well defined, since we have by assumption $\datameanat{\indexedselvariable}, (1- \datameanat{\indexedselvariable}) \neq 0$ and $\hypercoreat{\headvariableof{\selindex}=0}, \hypercoreat{\headvariableof{\selindex}=1} \neq 0$ since \MarkovLogicNetworks{} are positive distributions and $\enumformula \notin \{\onesat{\shortcatvariables},\zerosat{\shortcatvariables}\}$.
%\end{proof}

\begin{algorithm}[hbt!]
    \caption{Alternating Moment Matching for Markov Logic Networks}\label{alg:AMM_MLN}
    \begin{algorithmic}
        \Require Mean parameter $\meanparamwith$ with $\uniquantwrtof{\selindexin}{\meanparamat{\indexedselvariable}\notin\{0,1\}}$, boolean statistic $\sstat$, base measure $\basemeasure$
        \Ensure Canonical parameter $\canparamwith$, such that $\expdist$ is the (approximative) moment projection of $\empdistribution$ onto $\expfamily$
        \hrule
        \State Set $\canparamwith=\zerosat{\selvariable}$
        \While{Convergence criterion is not met}
            \For{$\selindex\in\secnodes$}
                \State Compute
                \begin{align*}
                    \hypercoreat{\headvariableof{\selindex}}
                    = \contractionof{\{\bencodingof{\enumformula} \wcols \selindexin\}
                    \cup\{\actcoreof{\tilde{\selindex},\canparamat{\selvariable=\tilde{\selindex}}} \wcols \tilde{\selindex}\in[\seldim]\ncond\tilde{\selindex}\neq\selindex\}
                    \cup\{\basemeasure\}}{\headvariableof{\selindex}}
                \end{align*}
                \State Set
                \begin{align*}
                    \canparamat{\indexedselvariable} = \lnof{
                        \frac{\meanparamat{\indexedselvariable}}{(1-\meanparamat{\indexedselvariable})}
                        \cdot \frac{\hypercoreat{\headvariableof{\selindex}=0}}{\hypercoreat{\headvariableof{\selindex}=1}}
                    }
                \end{align*}
            \EndFor
        \EndWhile
        \State \Return $\canparamwith$
    \end{algorithmic}
\end{algorithm}

Note that while $\uniquantwrtof{\selindexin}{\meanparamat{\indexedselvariable}\notin\{0,1\}}$ ensures the well-definedness of the update equations in \algoref{alg:AMM_MLN} (otherwise the update could not be computed), it is only a necessary but not always sufficient criterion for the existence of a solution.
The moment matching conditions are simultaneously satisfiable, if and only if $\meanparamwith\in\sbinteriorof{\meansetof{\mlnstat,\basemeasure}}$.


The algorithm would be finished after a single pass through the loop, if the variables $\catvariableof{\exformula}$ are independent.
This would be the case, if the \MarkovLogicNetwork{} consists of atomic formulas only.
When they fail to be independent, the adjustment of the weights influence the marginal distribution of other formulas and we need an alternating optimization.
%
This situation corresponds with couplings of the weights by a partition contraction, which does not factorize into terms to each formula.

% Concave likelihood
Since the likelihood is concave (see \cite{koller_probabilistic_2009}), there are not local maxima the coordinate descent could run into and coordinate descent will give a monotonic improvement of the likelihood.

% Inference in inner loop
Solving Equation~\ref{sol:momentMatchingExformula} requires inference of a current model by answering a query.
This can be a bottleneck and circumvented by approximative inference, see e.g. CAMEL \cite{ganapathi_constrained_2008}.

\subsubsect{\HybridLogicNetworks{}}

We now extend the alternating parameter estimation algorithm to \HybridLogicNetworks{}, by allowing for mean parameters on the interior of cube faces.
To this end, we first find the smallest cube face containing the mean parameter $\meanparamwith$ (see \lemref{lem:minimalContainingFace}) and then run the algorithm on the exponential family on that face.
In an alternative perspective, we first optimize the leg vectors to features with $\meanparamat{\indexedselvariable}\in\{0,1\}$, which are then constant and left out in the further update sweeps.
The local moment matching conditions of are satisfied simultanously for some $\canparamwith$, if and only if $\meanparamwith$ is elementarily realizable (see \defref{def:elementaryRealizableMeanParams}), that is $\meanparamwith$ is on the interior of a cube face of $\hlnmeanset$.

\begin{algorithm}[hbt!]
    \caption{Alternating Moment Matching for Hybrid Logic Networks}\label{alg:AMM_HLN}
    \begin{algorithmic}
        \Require Mean parameter $\meanparamwith$
        \Ensure Canonical parameter $\canparamwith$, such that $\expdist$ is the (approximative) moment projection of $\empdistribution$ onto $\hlnsetof{\formulaset}$
        \hrule
        \State Set
        \begin{align*}
            \variableset = \Big\{ \selindex \wcols \selindexin \ncond \meanparamat{\indexedselvariable}\in\{0,1\} \Big\}
        \end{align*}
        and a tuple $\headindexof{\variableset}$ with $\headindexof{\selindex}=\meanparamat{\indexedselvariable}$ for $\selindex\in\variableset$ .
        \State Run \algoref{alg:AMM_MLN} (Alternating Moment Matching for Markov Logic Networks) with base measure $\exformulaof{\variableset,\headindexof{\variableset}}$ and statistic $\mlnstat=\{\enumformula\wcols\selindex\in\hardlegset\}$ to get $\canparamwith$
        \State \Return $(\variableset,\headindexof{\variableset},\canparamwith)$
    \end{algorithmic}
\end{algorithm}


\subsect{Iterative Proportional Fitting} % TO BE FORMULATED

In a special case of partition statistics we can find simultaneous updates to the canonical parameters.
Partition statistics are boolean statistics $\mlnstat$ such that
\begin{align*}
    \contractionof{\sencodingofat{\mlnstat}{\shortcatvariables,\selvariable}}{\shortcatvariables} \, .
\end{align*}
We can thus understand them as a disjoint partition of the state set $\facstates$ into sets
\begin{align*}
    \arbsetof{\selindex} \coloneqq \{\shortcatindices\wcols\shortcatindicesin\ncond\enumformulaat{\indexedshortcatvariables}=1\} \, .
\end{align*}
These statistics will be investigated in more detail in \secref{sec:partitionStatistics}.

\begin{lemma}
    Given distributions $\empdistributionwith,\secprobwith$ we build
    \begin{align*}
        \datameanat{\selvariable} = \contractionof{\empdistributionwith,\sencmlnstatwith}{\selvariable}
        \andspace
        \currentmeanat{\selvariable} = \contractionof{\secprobwith,\sencmlnstatwith}{\selvariable}
    \end{align*}
    and assume that $\greaterzeroof{\datamean}\models\greaterzeroof{\currentmean}$. % and $\equaloneof{\currentmean}\models\equaloneof{\datamean}$.
    Let us vary the distribution $\secprobwith$ by elementary tensors $\hypercoreat{\headvariables}$ as
    \begin{align*} % NEED TO RESTRICT TO ELEMENTARY HYPERCORE?
        \probofat{\hypercore}{\shortcatvariables}
        \coloneqq \normalizationof{\secprobwith,\mlnstatccwith,\hypercoreat{\headvariables}}{\shortcatvariables} \, .
    \end{align*}

    Then
    \begin{align*}
        \argmin_{\hypercore} \centropyof{\empdistribution}{\probof{\hypercore}}
    \end{align*}
    is solved at
    \begin{align*}
        \hypercoreat{\headvariables}
        = \contractionof{\kcoreofat{(\hardlegset,0_{\hardlegset})}{\headvariables},\actcoreofat{\canparam}{\headvariables}}{\headvariables}
    \end{align*}
    where $\hardlegset=\{\selindex\wcols\datameanat{\indexedselvariable}=0\}$ and for $\selindex\in[\seldim]/\hardlegset$
    \begin{align*}
        \canparamat{\indexedselvariable} = \lnof{\frac{\datameanat{\indexedselvariable}}{\currentmeanat{\indexedselvariable}}} \, .
    \end{align*}
\end{lemma}
\begin{proof}
    We use \theref{the:selectionRepresentationPartitionStatistics} to compute
    \begin{align*}
        \contractionof{\secprobwith,\mlnstatccwith,\hypercoreat{\headvariables}}{\selvariable}
        = \contractionof{\secprobwith,\sencmlnstatwith,\expof{\canparamwith},\onehotmapofat{[\seldim]/\hardlegset}{\selvariable}}{\selvariable} \\
        = \contractionof{\currentmeanat{\selvariable},\expof{\canparamwith},\onehotmapofat{[\seldim]/\hardlegset}{\selvariable}}{\selvariable}
        = \datameanat{\selvariable}
    \end{align*}
    Here $\onehotmapofat{[\seldim]/\hardlegset}{\selvariable}$ is the indicator vector, whether $\selindex\notin\hardlegset$ (see for more details \defref{def:subsetEncoding}).

    Using that $\mlnstat$ is a partition statistic we get
    \begin{align*}
        \contraction{\datameanat{\selvariable}}
        &= \contraction{\empdistribution,\sencmlnstatwith} \\
        &= \contraction{\empdistribution} \\
        &= 1 \, .
    \end{align*}
    It follows that
    \begin{align*}
        \contraction{\secprobwith,\mlnstatccwith,\hypercoreat{\headvariables}} = 1
    \end{align*}
    and due to trivial partition function term that
    \begin{align*}
        \probofat{\hypercore}{\shortcatvariables} = \contractionof{\secprobwith,\mlnstatccwith,\hypercoreat{\headvariables}}{\shortcatvariables} \, .
    \end{align*}
    We conclude
    \begin{align*}
        \contractionof{\probofat{\hypercore}{\shortcatvariables},\sencmlnstatwith}{\selvariable}
        = \datameanat{\selvariable} \, .
    \end{align*}
    Thus, the moment matching condition is satisfied for each $\selindexin$ and the cross-entropy is minimized.
\end{proof}

If the $\greaterzeroof{\datamean}\models\greaterzeroof{\currentmean}$ is violated, then there is no solution to the moment matching condition, since the support of the mean parameter cannot be increased by an activation tensor.

The assumptions of a partition statistic are met when taking all features to any hyperedge in a Markov Network seen as an exponential family.
In that case, the update algorithm is refered to as Iterative Proportional Fitting \cite{wainwright_graphical_2008}.

%Further, when activating both $\exformula$ and $\lnot\exformula$.
%\begin{remark}[Grouping of partition statistics]
%    When having a set of coordinates, such that the coordinate functions are boolean and sum to the trivial tensor, one can find simultaneous updates to the canonical parameters, such that the partition function is staying invariant.
%    Given a parameter $\canparam^t$ we compute
%    \[ \meanparam^t = \contractionof{\expdistof{(\sstat,\canparam^t)}, \sstat}{\selvariable} \]
%    and build the update
%    \[ \canparam^{t+1} = \canparam^t + \lnof{\meanparam^{\datamap}}{\meanparam^t} \, . \]
%    Then, $\canparam^{t+1}$ satisfies the moment matching equations for all coordinates in the set.
%\end{remark}


\sect{Structure Learning}

Structure learning refers to the learning of the statistic $\mlnstat$ itself.
Let there be a set $\formulasuperset$ of statistics we build the set of parametrized distributions
\begin{align*}
    \probtensor = \bigcup_{\mlnstat} \elrealizabledistsof{\mlnstat}
\end{align*}
and pose the structure learning problem as the minimization of the cross entropy
\begin{align*}
    \min_{\mlnstat\in\formulasuperset} \min_{\hybridparamin} \centropyof{\empdistribution}{\probof{\mlnstat,\hybridparam}} \, .
\end{align*}
It can be impracticle to learn all formulas at once, since the set $\formulasuperset$ often grows combinatorically, for example when choosing as a powerset of formulas.
To avoid intractabilities, one can choose a greedy approach and learn in addition formulas $\exformula$ when already having learned a set $\formulaset$ of formulas.
%We in this section assume a current model $\currentdistribution$, which is a generic positive distribution not necessarily a \MarkovLogicNetwork{}. % or Hybrid Logic Network.

\subsect{Greedy formula inclusions}

Having a current set of formulas $\formulaset$ we want to choose the best $\formula\in\greedyhypothesis$ to extend the set of formulas to $\formulaset\cup\{\formula\}$ in a way minimizing the cross entropy.
Given this, add each step we solve the greedy cross entropy minimization
\begin{align}
    \label{prob:perfectGreedy}\tag{$\probtagtypeinst{\mathrm{greedy}}{\datamap,\mlnstat,\greedyhypothesis}$}
    \min_{\formula\in\greedyhypothesis} \min_{\hybridparam\in\hybridparamsetofdim{\cardof{\mlnstat}+1}}
    \centropyof{\empdistribution}{\probof{\mlnstat\cup\{\formula\},\hybridparam}} \, .
\end{align}
A brute force solution of Problem~\eqref{prob:perfectGreedy} would require parameter estimation for each candidate in $\greedyhypothesis$.
We provide two more efficient approximative heuristics in the following (see Chapter~20 in \cite{koller_probabilistic_2009}), which are faster to compute estimates of the cross entropy improvement from adding a formula to an existing statistic.

\subsect{Gain Heuristic}

In the gain heuristic, only the parameters of the new formula are optimized and the others left unchanged.
Let $\sechybridparam$ be the canonical parameter of the reference distribution on the statistic $\mlnstat$.
When adding a feature $\formula\in\greedyhypothesis$ we extend $\mlnstat$ by $\formula$ and restrict the parameters to coincide with $\sechybridparam$ on the first $\cardof{\mlnstat}$ coordinates.
We denote this constraint by
\begin{align*}
    \restrictionofto{\hybridparam}{[\mlnstat]} = \sechybridparam \, .
\end{align*}
The greedy gain heuristic is then the problem
\begin{align}
    \label{prob:greedyGain}\tag{$\probtagtypeinst{\gainsymbol}{\datamap,\formulaset,\sechybridparam,\greedyhypothesis}$}
    \min_{\formula\in\greedyhypothesis} \min_{\hybridparamsetofdim{\cardof{\mlnstat}+1} \wcols \restrictionofto{\hybridparam}{[\mlnstat]} = \sechybridparam }
    \centropyof{\empdistribution}{\probof{\hybridparam}} \, .
\end{align}
%Here we denote by $\canparam$ the first $\cardof{\formulaset}$ coordinates of the M-projection $\currentdistribution$  of $\empdistribution$ onto $\formulaset$ and the variable new coordinate at position $\canparamat{\cardof{\formulaset}}$.
To provide further insight into the solution of the gain heuristic, let us quantify the improvement of the cross entropy when addign a feature $\exformula$.

%% FOR GAIN HEURISTIC
\begin{lemma}
    Let $\empdistribution$, $\secprobtensor$ be distributions and $\exformula$ a formula such that $\contraction{\secprobtensor,\exformula}\notin\{0,1\}$.
    Then we have
    \begin{align*}
        \centropyof{\empdistribution}{\probof{\mlnstat,\sechybridparam}}
        - \min_{\hypercore} \centropyof{\empdistribution}{\probof{\hypercore}}
        = \kldivof{\bernoulliof{\contraction{\empdistribution,\exformula}}}{\bernoulliof{\contraction{\secprobtensor,\exformula}}}
    \end{align*}
    where by $\bernoulliof{p}$ we denote the Bernoulli distribution with parameter $p\in[0,1]$.
\end{lemma}
\begin{proof}
    We use the characterization of the local update by \lemref{lem:localHybridParamUpdate} for $\secprobtensor=\probof{\mlnstat,\sechybridparam}$.
    The update on the added feature is then parametrized by the two-dimensional vector $\hypercoreat{\formulavar}$ and we have
    \begin{align*}
        \probof{\mlnstat,\sechybridparam} = \probof{\ones} \, .
    \end{align*}
    Let us abbreviate $\datamean\coloneqq\datamean$ and $\currentmean\coloneqq\contraction{\probof{\mlnstat,\sechybridparam},\exformula}$.
    We distinguish the cases $\datamean\in(0,1)$ and $\datamean\in\{0,1\}$.

    In the case $\datamean\in(0,1)$, we have $\hardlegset=\sechardlegset$, $\headindexof{\hardlegset}=\secheadindexof{\hardlegset}$ and $\canparamat{\indexedselvariable}=\seccanparamat{\indexedselvariable}$ for $\selindex\in[\cardof{\mlnstat}]$.
    The additional coordinate of the canonical parameter is then
    \begin{align*}
        \canparamat{\selvariable=\cardof{\mlnstat}} = \lnof{
            \frac{\datamean}{(1-\datamean)}
            \cdot \frac{1-\currentmean}{\currentmean}
        } \, .
    \end{align*}
    Based on this characterization, the cross entropy difference is
    \begin{align*}
        \centropyof{\empdistribution}{\probof{\ones}} - \min_{\hypercore} \centropyof{\empdistribution}{\probof{\hypercore}} \\
        &= \datamean \cdot \canparamat{\selvariable=\cardof{\mlnstat}}
        - \lnof{\contraction{\secprobtensor,\formulaccwith,\actcoreof{\canparamat{\selvariable=\cardof{\mlnstat}}}}} \, .
    \end{align*}
    We simplify
    \begin{align*}
        \contraction{\secprobwith,\formulaccwith,\actcoreofat{\canparamat{\selvariable=\cardof{\mlnstat}}}{\headvariableof{\formulavar}}}
        = (1-\currentmean) + \currentmean \cdot \expof{\canparamat{\selvariable=\cardof{\mlnstat}}}
        = \frac{(1-\currentmean)}{(1-\datamean)} \, .
    \end{align*}
    We further have
    \begin{align*}
        \datamean \cdot \canparamat{\selvariable=\cardof{\mlnstat}}
        = \datamean \cdot \left[ \lnof{\frac{\datamean}{(1-\datamean)}\cdot \frac{(1-\currentmean)}{\currentmean}}  \right]
        = \datamean \lnof{\datamean} - \datamean \lnof{1-\datamean} + \datamean \lnof{1-\currentmean} - \datamean \lnof{\currentmean}
    \end{align*}
    and arrive at
    \begin{align*}
        &  \centropyof{\empdistribution}{\probof{\ones}} - \min_{\hypercore} \centropyof{\empdistribution}{\probof{\hypercore}} \\
        & \quad =  \datamean \lnof{\datamean} - \datamean \lnof{1-\datamean} + \datamean \lnof{1-\currentmean} - \datamean \lnof{\currentmean}
        -  \lnof{1-\currentmean} - \lnof{1-\datamean} \\
        & \quad = \left( -\datamean \lnof{\currentmean} - (1-\datamean) \lnof{1-\currentmean} \right)  - \left( -\datamean \lnof{\datamean} - (1-\datamean) \lnof{1-\datamean} \right) \\
        & \quad = \kldivof{\bernoulliof{\datamean}}{\bernoulliof{\currentmean}} \, .
    \end{align*}

    In the case $\datamean\in\{0,1\}$, the optimal $\hypercore$ is boolean and only the partition function term is changed by the update.
    We then have
    \begin{align*}
        \centropyof{\empdistribution}{\probof{\ones}} - \min_{\hypercore} \centropyof{\empdistribution}{\probof{\hypercore}}
        %&= \lnof{\contraction{\secprobwith}} - \lnof{\contraction{\secprobwith,\formulaccwith,\hypercoreat{\formulavar}}} \\
        &= - \lnof{\contraction{\secprobwith,\formulaccwith,\hypercoreat{\formulavar}}} \\
        &= \begin{cases}
               \lnof{\currentmean} & \ifspace \datamean=0 \\
               \lnof{1-\currentmean} & \ifspace \datamean=1 \\
        \end{cases} \\
        &= \kldivof{\bernoulliof{\datamean}}{\bernoulliof{\currentmean}} \, . \qedhere
    \end{align*}
\end{proof}

Problem \eqref{prob:greedyGain} is thus solved for
\begin{align*}
    \hat{\formula} \in \argmax_{\formula\in\formulaset} \kldivof{\bernoulliof{\contraction{\empdistribution,\formula}}}{\bernoulliof{\contraction{\currentdistribution,\formula}}}
\end{align*}
and $\hybridparam$ characterized in \lemref{lem:localHybridParamUpdate}.
The minimum is taken at
\begin{align*}
    \centropyof{\empdistribution}{\probof{\mlnstat,\sechybridparam}}
    - \kldivof{\bernoulliof{\contraction{\empdistribution,\hat{\formula}}}}{\bernoulliof{\contraction{\currentdistribution,\hat{\formula}}}} \, .
\end{align*}

% Algorithmic
The gain heuristic thus searches for the mode of a coordinatewise transform of the mean parameter tensors to $\empdistribution$ and $\currentdistribution$, using the Bernoulli Kullback-Leibler divergence as transform function.

% Interpretation
One therefore takes the formula, which marginal distribution in the current model and the targeted distribution are differing at most, measured in the KL divergence.

% Optimization method
%One optimization method would thus be the computation of the mean parameters to both distribution, building the coordinatewise KL divergence and choosing the maximum.
%Since we need to evaluate each coordinate, this can be intractable for large sets of formulas.


% Further weight optimization
Further improvement of the model can be achieved by iteratively optimizing the other weights as well, since their corresponding moment matching conditions might be violated after the integration of a new formula.
Unfortunately, backward maps cannot be expressed in closed form in general.
%This would require the computation of backward mappings for each candidate formula, for which we only have an alternating approach in general.



\subsect{Gradient heuristic}

Gradient heuristic is another approach to select a feature.
We show here how the effective selection tensor network representation of exponentially many formulas described in \charef{cha:formulaSelection} can be utilized. % Or later in proposal

In the gradient heuristic, we estimate the cross entropy improvement by the gradient of the cross entropy with respect to varying with another feature.
Given a distribution $\secprobwith$, we vary % more general than
\begin{align*}
    \probofat{\actcoreof{\canparam}}{\shortcatvariables} \coloneqq \normalizationof{\secprobwith,\formulaccwith,\actcoreofat{\canparam}{\headvariableof{\formulavar}}}{\shortcatvariables} \, .
\end{align*}
In the gradient heuristic we choose the steepest gradient direction
\begin{align}
    \label{prob:greedyGrad}\tag{$\probtagtypeinst{\gradientsymbol}{\datamap,\secprobtensor,\greedyhypothesis}$}
    \min_{\formula\in\greedyhypothesis} \difofwrt{\centropyof{\empdistribution}{\probofat{\actcoreof{\canparam}}{\shortcatvariables}}}{\canparam}   \, .
\end{align}

%% FOR GRADIENT HEURISTIC
\begin{lemma}
    \begin{align*}
        \difofwrt{\centropyof{\empdistribution}{\probof{\actcoreof{\canparam}}}}{\canparam} =
        -\contraction{\empdistribution,\formula} + \contraction{\probof{\actcoreof{\canparam}},\formula} \, .
    \end{align*}
\end{lemma}
\begin{proof}
    We have
    \begin{align*}
        \centropyof{\empdistribution}{\probof{\actcoreof{\canparam}}}
        & = \contraction{\empdistribution,\formulaccwith,-\lnof{\actcoreofat{\canparam}{\headvariableof{\formulavar}}}}
        + \lnof{\contraction{\secprobwith,\formulaccwith,\actcoreofat{\canparam}{\headvariableof{\formulavar}}}}
    \end{align*}
    The derivation of the first term at $\canparam=0$ is $-\contraction{\empdistribution,\formula}$ and of the second $\contraction{\probof{\actcoreof{\canparam}},\formula}$.
\end{proof}

Problem~\eqref{prob:greedyGrad} is thus
\begin{align}
    %\label{prob:greedyGrad}\tag{$\probtagtypeinst{\gradsymbol}{\datamap,\secprobtensor,\greedyhypothesis}$}
    \min_{\formula\in\greedyhypothesis} \contraction{\secprobtensor,\formula} - \contraction{\empdistribution,\formula}   \, .
\end{align}
%% Positive and Negative Search
The gradient shows the typical decomposition into a positive and a negative phase.
While the positive phase comes from the data term and prefers directions of large data support, the negative phase originates in the partition function and draws the gradient away from directions already supported by the current model $\expdistof{(\naivestat, \naivecanparam)}$.
%% Regularization functionality
The negative phase is a regularization, by comparing with what has already been learned.
When nothing has been learned so far, we can take the current model to be the uniform distribution, which is the naive exponential family with vanishing canonical parameters.






\subsect{Proposal distributions}

We now frame the selection of a formula as a sampling problem of proposal distributions.
%All the costs and the exact greedy objective
Each of the scores to candidate formulas are understood as coordinates of an energy tensor.
Among the heuristics, the gradient heuristic has the most accessible form, since the energy tensor is available as a tensor network of the selection encoding of the formulas.

%We can choose selection architectures to efficiently parametrize the formulas in the hypothesis $\greedyhypothesis$ and rewrite the problem as
%\begin{align*}
%	\argmax_{\selindexin} \contractionof{ \gradwrtat{\canparam}{\canparam=0} \lossof{\expdist}}{\indexedselvariable}
%\end{align*}
%This is thus equivalent to the problem \ref{prob:greedyGrad}, when taking all formulas selectable by $\formulaset$ as the hypothesis $\Gamma$.

% Proposal distribution
%Let us now understand the likelihood gradient as the energy tensor of a probability distribution, which we call the proposal distribution.

\begin{definition}[Gradient Heuristic Proposal Distribution]
    Let there be a base distribution $\currentdistribution$, a targeted distribution $\empdistribution$ and a formula selecting map $\greedyhypothesis$.
    The proposal distribution at inverse temperature $\invtemp>0$ is the distribution of $\selvariable$ defined by
    \begin{align*}
        \normalizationof{\expof{\contractionof{\invtemp\cdot(\empdistribution[\shortcatvariables]-\currentdistribution[\shortcatvariables]),\greedyhypothesis\left[\shortcatvariables,\selvariable\right]}{\selvariable}} }{\selvariable} \, .
    \end{align*}
    The proposal distribution is the member of the exponential family with statistics $\greedyhypothesis$ and canonical parameter $\invtemp\cdot(\empdistribution-\currentdistribution)$.
\end{definition}


%. Exponential family
The proposal distribution is in the exponential family with sufficient statistic by the formula selecting map $\greedyhypothesis$, namely the member with the canonical parameters $\canparam=\empdistribution-\currentdistribution$.
Of further interest are tempered proposal distributions, which are in the same exponential family with canonical parameters $\invtemp\cdot(\empdistribution-\currentdistribution)$ where $\invtemp>0$ is the inverse temperature parameter.

% MLN
As \MarkovLogicNetworks{}, the proposal distributions are in exponential families with the sufficient statistic defined in terms of formula selecting maps.
While \MarkovLogicNetworks{} contract the maps on the selection variables $\selvariable$, the proposal distributions contract them along the categorical variables $\catvariable$ to define energy tensors.

% Methods to solve mode search
The gradient heuristic optimization Problem~\eqref{prob:greedyGrad} is the search for the mode of the proposal distribution.
To solve the gradient heuristic, we thus need to answer a mode query, for which we can apply the methods introduced in \charef{cha:probReasoning}, such as Gibbs Sampling or Mean Field Approximations in combination with annealing.


%\subsect{Mean parameter polytope}
The mean parameter polytope of any proposal distribution to statistic $\proposalstat$ is the convex hull of the formulas in $\formulaset$, that is
\begin{align*}
    \meansetof{\proposalstat}
    = \convhullof{\sencodingof{\proposalstat}{\indexedselvariable,\shortcatvariables} \wcols \selindexin}
    = \convhullof{\formulaat{\shortcatvariables} \wcols \formula\in\greedyhypothesis}
\end{align*}
% 0/1
As it was the case for \MarkovLogicNetworks{}, the mean parameter polytopes are instances of a $0/1$-polytopes \cite{ziegler_lectures_2000,gillmann_01-polytopes_2007}.
% Interpretation as formulas
The vertices are the formulas selectable by the formula selecting map $\greedyhypothesis$.

\subsect{Iterations}

Let us now iterate the search for a best formula at a current model with the optimization of weights after each step.
The result is Algorithm~\ref{alg:greedyStructureLearning}, which is a greedy algorithm adding iteratively the currently best feature.

\begin{algorithm}[hbt!]
    \caption{Greedy Structure Learning}\label{alg:greedyStructureLearning}
    \begin{algorithmic}
        \Require Empirical distribution $\empdistribution$, hypothesis $\greedyhypothesis$ of formulas
        \Ensure Distribution $\expdist$ approximating $\empdistribution$
        \hrule
        \State Initialize
        \[ \currentdistribution \algdefsymbol \frac{1}{\prod_{\catenumeratorin}\catdimof{\atomenumerator}} \cdot \onesat{\shortcatvariables} \quad, \quad \formulaset = \varnothing \]
        \While{Stopping criterion is not met}
        % REFINE! Work in data
            \State
            \begin{itemize}
                \item \textbf{Structure Learning:} Compute a (approximative) solution $\hat{\formula}$ to Problem~\eqref{prob:perfectGreedy} and add the formula to $\formulaset$, i.e.
                \[ \formulaset \algdefsymbol \formulaset \cup\{\hat{\formula}\} \]
                Extend dimension of $\selvariable$ by one, by $\formulaof{\seldim}=\hat{\formula}$ and $\canparamat{\selindex=\seldim}=0$
                \item \textbf{Weight Estimation:} Estimate the best weights for the added formula and recalibrate the weights of the previous formulas, by calling Algorithm~\ref{alg:AMM_HLN}.
                \[ \currentdistribution \algdefsymbol \probof{\formulaset,\hybridparam} \]
            \end{itemize}
        \EndWhile
        \State \Return $\formulaset$, $\hybridparam$ %, $\kb$
    \end{algorithmic}
\end{algorithm}


%% Energy Storage -> Useful after learning for energy-based inference
When having used the same learning architecture multiple times, the energy of the corresponding formulas are all representable by a formula selecting architecture.
Their energy term is therefore a contraction of the selecting tensor with a parameter tensor $\canparam$ in a basis $\cpformat$ decomposition with rank by the number of learned formulas.
When mutiple selection architectures have been used, the energy is a sum of such contractions.
% 
Let us note, that this representation is useful after learning, when performing energy-based inference algorithms on the result.
During learning, one needs to instantiate the proposal distribution, which requires instantiation of the probability tensor.
\red{However, one could alternate data energy-based and use this as a particle-based proxy for the probability tensor.}


\begin{remark}[Sparsification by Thresholding]
    To maintain a small set of active formulas, one could combine greedy learning approaches with thresholding on the coordinates of $\canparam$.
    This is a standard procedure in Iterative Hard Thresholding algorithms of Compressed Sensing, but note that here we do not have a linear in $\canparam$ objective.
\end{remark}









\sect{Discussion}

\begin{remark}[Bayesian approach]
    We only treated the estimation of a single resulting distribution by the data, while in a Bayesian approach one typically considers an uncertainty over possible distributions.
    % MAP
    \red{When treating $\canparam$ as a random tensor, which prior distribution is given and posteriori distribution wanted, we have a more involved Bayesian approach.}
    When having a prior $\probat{\mlnparameters}$ over the \MarkovLogicNetworks{} we alternatively want to find the parameters $\mlnparameters$ solving the maximum a posteriori problem
    \begin{align}
        \argmax_{\mlnparameters} \mlnprobat{\data}\cdot \probat{\mlnparameters}\, .
    \end{align}
\end{remark}

%To summarize some insights on the mean polytopes $\hlnmeanset$:
%\begin{itemize}
%    \item If and only íf all coordinates are in $\{0,1\}$ then an extreme points, then $\meanparam$ is reproduced by a \HardLogicNetwork{}.
%    \item If some mean params in $\{0,1\}$, then not in the interior, and not reproduced by a \MarkovLogicNetwork{}.
%    Back direction not correct: There are interior points where no coordinate in $\{0,1\}$.
%    \item If not in the interior, we can identify with base measure refinement a base measure, such that reproducable by a distribution representable by the base measure.
%\end{itemize}


% Polytopes - MLN 
The polytopes of mean parameters to \HybridLogicNetworks{} and proposal distributions are an interesting connection between the fields of combinatorical optimization and the study of expressivity of tensor networks.
% Minimal Connectivity: Local consitency - Hierarchical Tucker
This is of special interest, when the computation cores of a hybrid logic network are minimally connected, the mean parameters are captured by local consistencies.
Similar investigations have been made in the field of tensor networks, where minimal connected tensor networks are refered to by Hierarchical Tucker formats (HT).
Minimal connection is exploited in the tensor network community to show numerical properties of the format, such as closedness and existence of best approximators.














