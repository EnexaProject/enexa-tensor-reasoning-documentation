\chapter{Introduction to \parref{par:two}}

\begin{highlight}
	Research [...] should bring together logic`s aptitude for handling the visible and probability`s ability to summarize the invisible. - \text{Judea Pearl \cite{pearl_probabilistic_1988}}
\end{highlight}

After having explored the tensor formalism in the logical and probabilistic foundations of artificial intelligence in \parref{par:one}, we now investigate \HybridLogicNetworks{}, which unify the two approaches.

\sect{\HybridLogicNetworks{} as \ComputationActivationNetworks{}}

%We unify the discussion on exponential families and propositional formulas, by constructing sufficient statistics and base measures based on propositional formulas.
%This restricts the sufficient statistics to boolean valued functions, which can be decomposed into connectives.

% Computation Activation Networks in part \parref{par:one}:
The unification is along the framework of \ComputationActivationNetworks{} (see \defref{def:realizableStatDistributions}), which is common to the tensor network representations in probabilistic and logical models:
\begin{itemize}
    \item Probability distributions, which have a sufficient statistics can be represented by a computation network of the statistic and an activation tensor.
    We are especially interested in distributions with elementary activation tensor, which is the case for exponential families.
    \item Propositional formulas can be computed based on their syntactical decompositions (see \charef{cha:logicalRepresentation}).
    The uniform distributions over their models has the formula itself as a sufficient statistic and can be instantiated by a boolean activation cores.
\end{itemize}
By allowing for arbitrary elementary activation tensors, we can unify both approaches, when focusing on boolean valued statistics.
The resulting \ComputationActivationNetworks{}are called \HybridLogicNetworks{}, since they unify logical and probabilistic models.
In \charef{cha:networkRepresentation} we focus on the representation of such networks and introduce the following (see \figref{fig:elementaryComputableSketch}):
\begin{itemize}
    \item \textbf{\MarkovLogicNetworks{}:} By demanding positive and elementary activation tensors, we parametrize positive distributions interpreted as uncertainty-tolerant soft logical knowledge bases.
    \item \textbf{\HardLogicNetworks{}:} By demanding boolean and elementary activation tensors, we parametrize propositional formulas interpreted by hard logical knowledge bases.
    \item \textbf{\HybridLogicNetworks{}:} Both the soft and the hard parametrizations are unified when admitting generic elementary activation tensors.
\end{itemize}
In \charef{cha:networkReasoning} we characterize these networks as maximum entropy distributions and then focus on the inference properties of such networks.

Regarding the representation of \HybridLogicNetworks{}, we investigate sparsity mechanisms to result in efficient tensor network representations:
\begin{itemize}
	\item \textbf{\DecompositionSparsity{}:} When the sufficient statistics is decomposable into logical connectives, we find tensor network representation by basis encodings of the component functions.
    This mechanism has been exploited in \charef{cha:logicalRepresentation} for a single propositional formula and will in \charef{cha:networkRepresentation} be extended to sets of propositional formulas.
    \item \textbf{\SelectionSparsity{}:} We will investigate representation schemes for sets of formulas, which share a common structure, in \charef{cha:formulaSelection}.
%    To capture sparse mechanisms we define formula selecting networks in \charef{cha:formulaSelection}.
%    This allows for sparse representation of weighted sums of formulas, which appear in energy tensors.
    \item \textbf{\PolynomialSparsity{}:} Tensors are decomposed into sums of restricted elementary tensors, which are interpreted by monomials.
    We will describe such decompositions in \secref{sec:HLNpolyRepresentation} and relate it later in \charef{cha:sparseRepresentation} to restricted $\cpformat$ decompositions, which we call basis+.
\end{itemize}

\begin{figure}[t]
    \begin{center}
        \begin{tikzpicture}[yscale=0.5, xscale=0.9]
	\node[anchor=center] (text) at (-2.25,13) {Distributions with sufficient statistic $\formulaset$: $\realizabledistsof{\formulaset,\maxgraph}$};
	\draw[dashed] (-10.5,14) rectangle (5,6);

	\draw[dashed] (-10,12) rectangle (4.5,6.5);
	\node[anchor=center] (text) at (-2.25,11) {Hybrid Logic Networks: $\realizabledistsof{\formulaset,\elgraph}$};
	
	\draw[\probcolor] (-9.5,10) rectangle (-3,7);
	\node[\probcolor] (text) at (-6,9) {Markov Logic Networks: };
	\node[\probcolor] (text) at (-6,7.9) {Positive activation cores};

	\draw[\concolor] (-2.5,10) rectangle (4,7);
	\node[\concolor] (text) at (1,9) {Hard Logic Networks:};
	\node[\concolor] (text) at (1,7.9) {Boolean activation cores};

\end{tikzpicture}
    \end{center}
    \caption{Sketch of distributions with sufficient statistics by a set of propositional formulas $\hlnstat$.
    In most generality, any distribution with a sufficient statistics can be represented by an arbitrary activation tensor and is therefore in $\realizabledistsof{\hlnstat,\maxgraph}$.
    The \HybridLogicNetworks{} are those with elementary activation tensor, that is the elements of $\realizabledistsof{\hlnstat,\elgraph}$.
    By further demanding positive activation tensors we characterize the \MarkovLogicNetworks{} $\hlnstat$ and by demanding boolean activation tensors we the \HardLogicNetworks{} $\hlnstat$.
    }
    \label{fig:elementaryComputableSketch}
\end{figure}

\sect{Extensions towards First-Order Logics}

While so-far the focus had been on propositional logic as an explainable framework in machine learning, we show in \charef{cha:folModels} extenstions towards more expressive first-order logics.
We therein encounter two mechanisms in logic introducing tensor structures:
\begin{itemize}
    \item \textbf{\SubstitutionStructure{}:} Assigning objects to variables results in a vector of possible substitutions. When there are formulas with multiple variables, these lists of substitutions carry a tensor structure.
    \item \textbf{\SemanticStructure{}:} Mapping all possible worlds defined by $\atomorder$ atoms requires $2^{\atomorder}$ dimensions, which are at best represented by a space of order $\atomorder$ tensors.
    This is analogous to the factored representations of a system.
\end{itemize}

\sect{Probabilistic Guarantees}

Besides that, probabilistic guarantees on the success of the learning problems are derived in \charef{cha:concentration}.
Here we also focus on boolean statistics, which coordinates are Bernoulli variables.
Due to their boundedness, they and their averages are sub-Gaussian variables with favorable concentration properties.

\sect{Outline}

We first investigate in \charef{cha:formulaSelection} efficient representation schemes for propositional formulas.
In \charef{cha:networkRepresentation} and \charef{cha:networkReasoning} we develop the hybrid reasoning scheme for logical and probabilistic approaches, which we call \HybridLogicNetworks{}.
We extend the scheme towards first-order logics in \charef{cha:folModels} and derive probabilistic guarantees on the success of the learning problems in \charef{cha:concentration}.