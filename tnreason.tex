\documentclass[aps,onecolumn,nofootinbib,pra]{article}

\usepackage{arxiv}
\usepackage{amsmath,amsfonts,amssymb,amsthm,bbm,graphicx,enumerate,times}
\usepackage{mathtools}
\usepackage[usenames,dvipsnames]{color}
\usepackage{hyperref}
\hypersetup{
    breaklinks,
    colorlinks,
    linkcolor=gray,
    citecolor=gray,
    urlcolor=gray,
    pdftitle={The Tensor Network Approach to Efficient and Explainable AI},
    pdfauthor={Alex Goessmann}
}

\usepackage{tikz}
\usepackage{graphicx}
\usepackage{float}
\usepackage{comment}
\usepackage{csquotes}

\usepackage{listings}
\usepackage{verbatim}
\usepackage{etoolbox}
\usepackage{braket}
\usepackage[utf8]{inputenc}
\usepackage[english]{babel}
\usepackage[T1]{fontenc}
\usepackage{amsmath}
\usepackage{amsfonts}
\usepackage{amssymb}
\usepackage{amsthm}
\usepackage{titlesec}
\usepackage{tikz}
\usepackage{mathtools}
\usepackage{fancyhdr}
\usepackage{bbm}
\usepackage{bm}
\usepackage{algpseudocode}
\usepackage{algorithm}
\usepackage{lipsum}

\newtheorem{remark}{Remark}
\newtheorem{theorem}{Theorem}
\newtheorem{lemma}{Lemma}
\newtheorem{corollary}{Corollary}
\newtheorem{definition}{Definition}
\newtheorem{example}{Example}

\newcommand{\var}[1]{\text{\emph{#1}}}


\newcommand{\synencodingof}[1]{S\left(#1\right)} % Syntax encoding!
\newcommand{\stringof}[1]{"#1"}

% Text Macros
\newcommand{\python}{$\mathrm{python}$ }
\newcommand{\tnreason}{$\mathrm{tnreason}$ }
\newcommand{\spengine}{$\mathrm{engine}$ }
\newcommand{\spencoding}{$\mathrm{encoding}$ }
\newcommand{\spalgorithms}{$\mathrm{algorithms}$ }
\newcommand{\spknowledge}{$\mathrm{knowledge}$ }


\newcommand{\layeronespec}{\textbf{Layer 1}: Storage and manipulations}
\newcommand{\layertwospec}{\textbf{Layer 2}: Specification of workload}
\newcommand{\layerthreespec}{\textbf{Layer 3}: Applications in reasoning}

\newcommand{\rdf}{\mathrm{RDF}}
\newcommand{\rdftype}{$\mathrm{rdf}$:$\mathrm{type}$}

\newcommand{\true}{$\mathrm{true}$}
\newcommand{\truesymbol}{\mathrm{True}}
\newcommand{\false}{$\mathrm{false}$}
\newcommand{\falsesymbol}{\mathrm{False}}

\newenvironment{centeredcode}
    {\begin{center}\begin{algorithmic}\hspace{1cm}}
    {\end{algorithmic}\end{center}}

\newcommand{\algdefsymbol}{\leftarrow}

\newcommand{\probcolor}{red}
\newcommand{\concolor}{blue}
\newcommand{\inactivecolor}{gray}

\newcommand{\conjunctioncolor}{red}
\newcommand{\negationcolor}{blue}
\newcommand{\nodeminsize}{0.8cm}
\newcommand{\nodegrayscale}{gray!50}

% Basic Symbols
\newcommand{\sentropyof}[1]{\mathbb{H}\left[#1\right]}
\newcommand{\centropyof}[2]{\mathbb{H}\left[#1,#2\right]}

\newcommand{\subsphere}{\mathcal{S}}
\newcommand{\rr}{\mathbb{R}}
\newcommand{\nn}{\mathbb{N}}

\newcommand{\closureof}[1]{\overline{#1}}
\newcommand{\interiorof}[1]{{#1}^{\circ}}

\newcommand{\difofwrt}[2]{\frac{\partial #1}{\partial #2}}
\newcommand{\difwrt}[1]{\difofwrt{}{#1}}
\newcommand{\gradwrt}[1]{\nabla_{#1}}
\newcommand{\gradwrtat}[2]{\nabla_{#1}|_{#2}}

\newcommand{\cardof}[1]{\left|#1\right|}
\newcommand{\absof}[1]{\left|#1\right|}

\newcommand{\imageof}[1]{\mathrm{im}\left(#1\right)}

\newcommand{\argmin}{\mathrm{argmin}}
\newcommand{\argmax}{\mathrm{argmax}}


% Help functions
\newcommand{\chainingfunction}{h}
\newcommand{\longchainingfunction}{h}

\newcommand{\greaterzerofunction}{\psi}
\newcommand{\greaterzeroof}[1]{\greaterzerofunction\left(#1\right)}

\newcommand{\nonzerofunction}{\chi}
\newcommand{\nonzeroof}[1]{\nonzerofunction\left(#1\right)}


% Probability distributions
\newcommand{\expof}[1]{\mathrm{exp}\left[#1\right]}
\newcommand{\probtensor}{\mathbb{P}}
\newcommand{\probtensorof}[1]{\probtensor^{#1}}
\newcommand{\secprobtensor}{\tilde{\mathbb{P}}}
\newcommand{\secprobat}[2]{\secprobtensor[#1]}

\newcommand{\probtensorset}{\Gamma}

\newcommand{\gendistribution}{\probtensor^*}
\newcommand{\currentdistribution}{\tilde{\probtensor}}

\newcommand{\probof}[1]{\probat{#1}} % Bad usage!
\newcommand{\probat}[1]{\probtensor\left[#1\right]} 
\newcommand{\probofat}[2]{\probtensor^{#1}\left[#2\right]} 

\newcommand{\condprobof}[2]{\mathbb{P}\left[#1|#2\right]}
\newcommand{\margprobof}[2]{\probof{#1}}%{\mathbb{P}^{#2}\left[#1\right]} % OLD -> Use probof
\newcommand{\expectationof}[1]{\mathbb{E}\left[#1\right]}
\newcommand{\expectationofwrt}[2]{\mathbb{E}_{#2}\left[#1\right]}
\newcommand{\lnof}[1]{\ln \left[ #1 \right] }
\newcommand{\sgnormof}[1]{\left\|#1\right\|_{\psi_2}} % subgaussian
\newcommand{\normof}[1]{\left\|#1\right\|_{2}}


\newcommand{\distof}[1]{\mathbb{P}^{#1}}

\newcommand{\ones}{\mathbb{I}}
\newcommand{\onesof}[1]{\ones^{#1}}
\newcommand{\onesat}[1]{\ones\left[#1\right]}
\newcommand{\zerosat}[1]{0\left[#1\right]}
\newcommand{\identity}{\delta}
\newcommand{\identityat}[1]{\identity\left[#1\right]}
\newcommand{\dirdeltaof}[1]{\delta^{#1}}

\newcommand{\exmatrix}{M}
\newcommand{\matrixat}[1]{\exmatrix[#1]}

\newcommand{\idrestrictedto}[1]{\restrictionofto{\mathrm{Id}}{#1}}



%% KNOWLEDGE GRAPH
\newcommand{\kg}{kg}
\newcommand{\kgreptensor}{\rencodingof{\kg}}

\newcommand{\kgtriple}[3]{\braket{#1,#2,#3}}
\newcommand{\exunarytriple}{\kgtriple{?\variableof{0}}{a}{C}}
\newcommand{\exbinarytriple}{\kgtriple{?\variableof{0}}{R}{?\variableof{1}}}

\newcommand{\sparql}{\mathrm{SPARQL}}

\newcommand{\sindex}{s}
\newcommand{\pindex}{p}
\newcommand{\oindex}{o}

\newcommand{\cindex}{c}
\newcommand{\rindex}{r}

\newcommand{\resourcenumber}{\variablelegdim}
\newcommand{\indexedkgreptensor}{\kgreptensor_{:\sindex\pindex\oindex}}

\newcommand{\exreptensor}{\ftensor}


% Propositional Logics: New square bracket notation
\newcommand{\formula}{f}
\newcommand{\formulaat}[1]{\formula\left[#1\right]}
\newcommand{\formulaof}[1]{\formula_{#1}}
\newcommand{\formulaofat}[2]{\formulaof{#1}\left[#2\right]}
\newcommand{\enumformula}{\formulaof{\selindex}}
\newcommand{\enumformulaat}[1]{\formulaof{\selindex}\left[#1\right]}

\newcommand{\exformula}{\formula}
\newcommand{\formulavar}{\catvariableof{\exformula}}

\newcommand{\texformula}{\tilde{\exformula}}
\newcommand{\secexformula}{h} % Since g is atom
\newcommand{\exformulain}{\exformula\in\formulaset}
\newcommand{\exformulaof}[1]{\exformula\left(#1\right)}

\newcommand{\formulasuperset}{\mathcal{H}}


\newcommand{\exindividual}{a}
\newcommand{\secindividual}{b}
\newcommand{\exindividualof}[1]{\exindividual_{#1}}

% First order Logics
\newcommand{\folexformula}{q}
\newcommand{\headfolexformula}{\tilde{\folexformula}} % When representing \folexformula as \importancequery \rightarrow \headfolexformula
\newcommand{\folformulaset}{\mathcal{Q}}
\newcommand{\folexformulain}{\folexformula\in\folformulaset}

\newcommand{\folpredicate}{g}
\newcommand{\folpredicateof}[1]{\folpredicate_{#1}}
\newcommand{\folpredicates}{\folpredicateof{0},\ldots,\folpredicateof{\folpredicateorder-1}}
\newcommand{\folpredicateenumerator}{\atomenumerator} % Due to PL being a special case
\newcommand{\folpredicateorder}{\atomorder}

\newcommand{\worlddomain}{A}

\newcommand{\atombasemeasure}{\mu}

%\newcommand{\individualset}{\variableset}

\newcommand{\individuals}{\exindividualof{\indindexof{0}},\ldots,\exindividualof{\indindexof{\individualorder-1}}}


\newcommand{\individualsof}[1]{\exindividualof{0}^{#1},\ldots,\exindividualof{\individualorder-1}^{#1}} % Do not use, index already in individuals

%\newcommand{\individualsspace}{\bigotimes_{\individualenumeratorin}\rr^{\cardof{\individualset}}}







%% Redundant to individual variables
\newcommand{\individualvariable}{\indvariable}
\newcommand{\individualvariableof}[1]{\indvariableof{#1}}
\newcommand{\individualvariables}{\indvariablelist}


\newcommand{\individualorder}{\indorder}
\newcommand{\individualenumerator}{\indenumerator}
\newcommand{\individualenumeratorin}{\indenumeratorin}

\newcommand{\variableindex}{\indindex}
\newcommand{\variableindexof}[1]{\indindexof{#1}}
\newcommand{\variableenumerator}{\indenumerator}
\newcommand{\variableorder}{\indorder}
\newcommand{\variableenumeratorin}{\indenumeratorin}
\newcommand{\variableindices}{\indindexof{0}\ldots\indindexof{\indorder-1}}



% Tensor Representaion of functions: Outdated, since functions by definition tensors
\newcommand{\ftensor}{\theta}
\newcommand{\ftensorof}[1]{\rencodingof{#1}}

% Vector Representation of functions
%\newcommand{\vtensor}{\phi}
%\newcommand{\vtensorof}[1]{\vtensor^{#1}}



%\newcommand{\concore}{\theta} -> Relational Encoding in macros_tc
\newcommand{\concoreof}[1]{\concore^{#1}}

\newcommand{\exconnective}{\circ}
\newcommand{\connectiveof}[1]{\exconnective_{#1}}
\newcommand{\connectiveofat}[2]{\connectiveof{#1}\left[#2\right]}


\newcommand{\chadamard}{\circ}

%\newcommand{\gtensorspace}{ \rr^{\cardof{\indsetof{1}} \times \ldots \times \cardof{\indsetof{\variableorder}}}}

%\newcommand{\grounding}{\rho}

\newcommand{\dataworld}{W}
\newcommand{\dataworldat}[1]{\dataworld[#1]}

\newcommand{\groundingofatwrt}[3]{{#1}|_{#3} \left[#2\right]}
\newcommand{\groundingofat}[2]{{#1}|_{\dataworld} \left[#2\right]}
\newcommand{\groundingof}[1]{{#1}|_{\dataworld}}
\newcommand{\kggroundingof}[1]{{#1}|_{\kg}}

\newcommand{\restrictionofto}[2]{{#1}|_{#2}}

\newcommand{\gtensor}{\rho} % For decompositions
\newcommand{\gtensorof}[1]{\gtensor^{#1}}


\newcommand{\parametrization}{\rho}

%% For the TCalculus Theorem
\newcommand{\coordinatetrafo}{h}
\newcommand{\gentensor}{T}
\newcommand{\basisslices}{U}




% Parameters 


\newcommand{\candidatelist}{\mathcal{M}}
\newcommand{\candidatelistof}[1]{\candidatelist^{#1}}

\newcommand{\aselectionvariable}{L}
\newcommand{\aselectionvariableof}[1]{\aselectionvariable_{#1}}


%% OUTDATED IN MAINLINE
\newcommand{\atomlegdim}{p} %% Not needed: This is 2
\newcommand{\atomlegdimof}[1]{\atomlegdim_{#1}}
\newcommand{\atomlegset}{\mathcal{M}}
\newcommand{\atomlegsetof}[1]{\atomlegset^{#1}}



% Data Extraction Spec
\newcommand{\impformula}{p}
\newcommand{\fixedimpformula}{\underline{\impformula}}
\newcommand{\fixedimpbm}{\basemeasure_{\fixedimpformula}}
\newcommand{\supportedworlds}{\dataworld \, : \, \groundingof{\impformula} = \fixedimpformula}

\newcommand{\extformula}{q}
\newcommand{\extformulaof}[1]{\extformula_{#1}}
\newcommand{\extformulas}{\extformulaof{0},\ldots,\extformulaof{\atomorder-1}}



% Old, redundant
\newcommand{\exquery}{\impformula}
\newcommand{\fixedexquery}{\fixedimpformula}
\newcommand{\atomicqueryof}[1]{\extformulaof{#1}}
\newcommand{\atomicqueries}{\extformulas}



\newcommand{\extractionrelation}{\exrelation}

\newcommand{\variableset}{A} % Still used in monomial decomposition, NOT for object sets!
\newcommand{\variablesetof}[1]{\variableset_{#1}}


\newcommand{\variable}{\exindividual}
\newcommand{\variableof}[1]{\exindividualof{#1}}



%\newcommand{\variablelegdim}{m} % \inddim?
%\newcommand{\variablelegdimof}[1]{\variablelegdim_{#1}}
%\newcommand{\variablespace}{\rr^{\variablelegdimof{1}\times\ldots\times\variablelegdimof{\variableorder}}}

%\newcommand{\selectororder}{\atomorder}
%\newcommand{\selectorenumerator}{\atomenumerator}
%\newcommand{\selectorenumeratorin}{\selectorenumerator\in\selectororder}

%\newcommand{\datapoint}{\datamap}
%\newcommand{\atomicdatapoint}{\datapoint^{(\secexformula)}}

%\newcommand{\tensordataof}[1]{\datapointof{#1}}
%\newcommand{\evidencecore}{E}
%\newcommand{\evidencecoreof}[1]{\evidencecore^{#1}}

\newcommand{\formulaset}{\mathcal{F}}
\newcommand{\formulasetof}[1]{\formulaset_{#1}}
\newcommand{\hardformulaset}{\kb}
\newcommand{\hfbasemeasure}{\basemeasureof{\hardformulaset}}
\newcommand{\hfbasemeasureat}[1]{\hfbasemeasure\left[#1\right]}
\newcommand{\softformulaset}{\formulaset}


% Formula Selecting
\newcommand{\fselector}{\fselectionmap} % OLD, use \fselectionmap

\newcommand{\larchitecture}{\mathcal{A}}
\newcommand{\larchitectureat}[1]{\larchitecture\left[#1\right]}

\newcommand{\inneuronset}{\mathcal{A}^{\mathrm{in}}}
\newcommand{\outneuronset}{\mathcal{A}^{\mathrm{out}}}

\newcommand{\lneuron}{\sigma}
\newcommand{\lneuronof}[1]{\lneuron_{#1}}
\newcommand{\lneuronat}[1]{\lneuron\left[#1\right]}
\newcommand{\lneuractivation}{\lneuron^{\larchitecture}}
\newcommand{\lneuractivationat}[1]{\lneuractivation\left[#1\right]}

\newcommand{\fsnn}{\fselectionmapof{\larchitecture}}
\newcommand{\fsnnat}[1]{\fsnn\left[#1\right]}


\newcommand{\skeleton}{S}
\newcommand{\skeletonof}[1]{\skeleton\left(#1\right)}
\newcommand{\skeletontensor}{\ftensorof{\skeleton}} %OLD! Use skeleton

%\newcommand{\parameterspace}{\bigotimes_{\parenumeratorin} \rr^{\parlegdimof{\parenumerator}}}
\newcommand{\modelspace}{\bigotimes_{\atomenumeratorin} \rr^2}

\newcommand{\skeletoncore}{S}
\newcommand{\skeletoncoreof}[1]{\skeletoncore^{#1}}

\newcommand{\cselectionsymbol}{C}
\newcommand{\vselectionsymbol}{V}
\newcommand{\sselectionsymbol}{S}

\newcommand{\selinputvariable}{\selvariable} 
\newcommand{\cselinputvariable}{\selvariableof{\cselectionsymbol}}
\newcommand{\vselinputvariable}{\selvariableof{\vselectionsymbol}}

\newcommand{\fselectionmap}{\mathcal{H}}
\newcommand{\fselectionmapof}[1]{\fselectionmap_{#1}}
\newcommand{\fselectionmapat}[1]{\fselectionmap\left[#1\right]}
\newcommand{\fselectionmapofat}[2]{\fselectionmap_{#1}\left[#2\right]}

\newcommand{\cselectionmap}{\fselectionmapof{\cselectionsymbol}}
\newcommand{\cselectionmapat}[1]{\fselectionmapofat{\cselectionsymbol}{#1}}

\newcommand{\vselectionmap}{\fselectionmapof{\vselectionsymbol}}
\newcommand{\vselectionmapat}[1]{\fselectionmapofat{\vselectionsymbol}{#1}}

\newcommand{\sselectionmap}{\fselectionmapof{\sselectionsymbol}}
\newcommand{\sselectionmapat}[1]{\fselectionmapofat{\sselectionsymbol}{#1}}


\newcommand{\vselectionmapof}[1]{\fselectionmapof{\vselectionsymbol,#1}} % tb deleted!


\newcommand{\tranfselectionmap}{\fselectionmap^T}


% Input variables -> Use generic for target coordinates of sufficient statistics!
\newcommand{\selvariables}{\selvariableof{0},\ldots,\selvariableof{\selorder-1}}
\newcommand{\shortselvariables}{\selvariableof{[\selorder]}}

% OLD: To selection variables
\newcommand{\parorder}{\selorder} 
\newcommand{\parenumerator}{\selenumerator}
\newcommand{\parenumeratorin}{\selenumeratorin}
\newcommand{\parindex}{\selindex}
\newcommand{\parindexof}[1]{\selindexof{#1}}
\newcommand{\parindexin}{\selindex\in[\seldim]}
\newcommand{\parlegdim}{\seldim} 
\newcommand{\parlegdimof}[1]{\seldimof{#1}}



% Output variables - Following the catvariable convention
\newcommand{\seloutputvariable}{\randomx}
\newcommand{\cseloutputvariable}{\catvariableof{\cselectionsymbol}}
\newcommand{\vseloutputvariable}{\catvariableof{\vselectionsymbol}}

% Tensor Core Representation
\newcommand{\selectorcore}{{\rencodingof{\vselectionsymbol}}}
%\newcommand{\selectorcoreof}[1]{\rencodingof{\vselectionsymbol_{#1}}} 
\newcommand{\selectorcoreof}[1]{\rencodingof{\vselectionmapof{#1}}} 

\newcommand{\selectorcomponentof}[1]{\hypercoreof{\vselectionsymbol_{#1}}} % Since not an relational encoding!
\newcommand{\selectorcomponentofat}[2]{\selectorcomponentof{#1}\left[#2\right]} 

%% OLD
\newcommand{\fselectionvariable}{\selvariable}
\newcommand{\vselectionvariable}{\fselectionvariable} % For variable selection
\newcommand{\vselectionvariables}{\fselectionvariable_0,\ldots,\fselectionvariable_{\parorder-1}}
\newcommand{\cselectionvariable}{\fselectionvariable_{\exconnective}} % For connective selection
\newcommand{\selectionvariables}{\{\fselectionvariable_0,\ldots,\fselectionvariable_{\parorder-1}\}}

\newcommand{\parametercore}{\canparam}
\newcommand{\parametertensor}{\parametercore}
\newcommand{\parametertensorof}[1]{\parametertensor^{#1}}
\newcommand{\parametercoreof}[1]{\parametertensor^{#1}}

\newcommand{\parindices}{\parindexof{0}\ldots\parindexof{\parorder-1}}
\newcommand{\parindicesin}{\{\parindexof{\parenumerator} \in [\parlegdimof{\parenumerator}] \, : \, \parenumeratorin\}}
\newcommand{\parstates}{\bigtimes_{\parenumeratorin}[\parlegdimof{\parenumerator}]}
\newcommand{\parspace}{\bigotimes_{\parenumeratorin}\rr^{\parlegdimof{\parenumerator}}}
\newcommand{\simpleparspace}{\rr^{\parlegdim}}


\newcommand{\parametrizingtensor}{P^{\skeleton}}
\newcommand{\parametrizingspace}{\rr^{\atomlegdimof{1}\times\ldots\times\atomlegdimof{\atomorder}}}

\newcommand{\unitvectoratof}[2]{e^{(#1)}_{#2}}
\newcommand{\parametrizingunittensor}{e_{\atomindices}} % Not required?


\newcommand{\fulltensor}{T}
\newcommand{\fullspace}{\parametrizingspace \otimes \variablespace}


\newcommand{\placeholder}{Z} %% When not used in formulas, take the set for it
\newcommand{\placeholderof}[1]{\placeholder^{#1}}

\newcommand{\atomicformula}{\catvariable}
\newcommand{\atomicformulaof}[1]{\catvariableof{#1}}
\newcommand{\atomicformulas}{\catvariableof{[\atomorder]}} %{\{\atomicformulaof{\atomenumerator} :  \atomenumeratorin \}}
\newcommand{\enumeratedatoms}{\atomicformulaof{0},\ldots,\atomicformulaof{\atomorder-1}}
\newcommand{\atomformulaset}{\formulasetof{\mlnatomsymbol}}


\newcommand{\clause}{Z^{\lor}}
\newcommand{\clauseof}[2]{\clause_{#1,#2}}
\newcommand{\maxtermof}[1]{\clause_{#1}}
\newcommand{\maxtermformulaset}{\formulasetof{\mlnmaxtermsymbol}}

\newcommand{\term}{Z^{\land}}
\newcommand{\termof}[2]{\term_{#1,#2}}
\newcommand{\mintermof}[1]{\term_{#1}}
\newcommand{\mintermofat}[2]{\mintermof{#1}\left[#2\right]}
\newcommand{\mintermformulaset}{\formulasetof{\mlnmintermsymbol}}

\newcommand{\atomstates}{\bigtimes_{\atomenumeratorin}[2]}
\newcommand{\atomspace}{\bigotimes_{\atomenumeratorin}\rr^2}
%\newcommand{\atomsspace}{\atomspace}

\newcommand{\indexedplaceholderof}[1]{\placeholderof{#1}_{\atomlegindexof{#1}}}
\newcommand{\indexedplaceholders}{\indexedplaceholderof{1},\ldots,\indexedplaceholderof{\atomorder}}


\newcommand{\atomorder}{d}
\newcommand{\secatomorder}{r}
\newcommand{\atomenumerator}{k}
\newcommand{\secatomenumerator}{l}

\newcommand{\atomenumeratorin}{\atomenumerator\in[\atomorder]}
\newcommand{\secatomenumeratorin}{\secatomenumerator\in[\secatomorder]}
\newcommand{\atomlegindex}{\catindex}
\newcommand{\tatomlegindex}{\tilde{\atomlegindex}}
\newcommand{\atomlegindexof}[1]{\atomlegindex_{#1}}
\newcommand{\tatomlegindexof}[1]{\tatomlegindex_{#1}}
\newcommand{\atomindices}{{\atomlegindexof{0},\ldots,\atomlegindexof{\atomorder-1}}}
\newcommand{\atomindicesin}{\atomindices\in\atomstates}

\newcommand{\atombasisvector}{e}
\newcommand{\indexedatombasisvectorof}[1]{\atombasisvector^{(#1)}_{\atomlegindexof{#1}}}

\newcommand{\truevectorat}[1]{\atombasisvector_1^{#1}}
\newcommand{\falsevectorat}[1]{\atombasisvector_0^{#1}}



%% OPTIMIZATION

\newcommand{\targettensor}{Y}
\newcommand{\importancetensor}{I}




\newcommand{\affineoperator}{X}
\newcommand{\affineoperatorof}[1]{\affineoperator^{(#1)}}

%% MARKOV LOGIC NETWORK
\newcommand{\loss}{\mathcal{L}}
\newcommand{\lossof}[1]{\loss_{\datamap}\left(#1\right)}
\newcommand{\mlnformulaset}{\mathcal{F}}
\newcommand{\mlnformulain}{\exformula\in\mlnformulaset}
\newcommand{\weight}{\theta}
\newcommand{\weightof}[1]{\weight_{#1}}
\newcommand{\weightat}[1]{\weight[#1]}
\newcommand{\polynomial}{p}
\newcommand{\polynomialof}[1]{\polynomial^{#1}}

\newcommand{\mlnparameters}{(\formulaset,\canparam)}
\newcommand{\mlnparameterswithout}{(\tilde{\formulaset},\canparamt)}
\newcommand{\mlntrueparameters}{(\formulaset^*,\weight^*)}


% Examples
\newcommand{\mlnatomsymbol}{[\catorder]}
\newcommand{\mlnmintermsymbol}{\land}
\newcommand{\mlnmaxtermsymbol}{\lor}




\newcommand{\mlntensor}{\theta^*} % Should be dropped?

\newcommand{\partitionfunction}{\mathcal{Z}}
\newcommand{\secpartitionfunction}{\tilde{\mathcal{Z}}}
\newcommand{\partitionfunctionof}[1]{\partitionfunction{\left(#1\right)}}
\newcommand{\secpartitionfunctionof}[1]{\secpartitionfunction{\left(#1\right)}}


\newcommand{\mlnprob}{\probtensorof{\mlnparameters}}
\newcommand{\mlnprobat}[1]{\expdistofat{\mlnparameters}{#1}}
\newcommand{\mlnenergy}{\energytensorof{\mlnparameters}}

\newcommand{\folmlnparameters}{\folformulaset |_{\worlddomain},\weight , \basemeasure_{\fixedexquery}}

%% OLD
\newcommand{\mlnprobof}[2]{\probtensor^{(#2)}\left[#1\right]}
%\newcommand{\mlnprobabilityof}[1]{\mlnprobat{#1}}
%\newcommand{\mlnprobabilityexpof}[1]{\frac{1}{\partitionfunctionof{\mlnparameters}} \expof{\sum_{\mlnformulain}\exformula(#1)\cdot \weightof{\exformula}}}



\newcommand{\activationof}[1]{A^{#1}}




\newcommand{\variablesum}{\frac{1}{\datanum}\sum_{\variableindex=1}^\datanum}
\newcommand{\formulasum}{\sum_{\mlnformulain}}


%\newcommand{\acore}{A}
%\newcommand{\acoreof}[1]{\acore^{#1}}
%\newcommand{\aleg}{a}
%\newcommand{\alegof}[1]{\aleg_{#1}}

\newcommand{\mlntn}{\Theta}




% For Probabilistic Analysis

\newcommand{\kldivof}[2]{\mathrm{D}_{\mathrm{KL}}\left[ #1 || #2 \right]}
\newcommand{\kllossof}[1]{\kldivof{\expdistof{\ftensor^*}}{\expdistof{#1}}}

\newcommand{\expsolution}{\theta^*}
\newcommand{\empsolution}{\hat{\theta}}

\newcommand{\noisetensor}{\eta}
\newcommand{\fluctuationtensor}{\eta}
\newcommand{\expfamfluctuation}{\fluctuationtensor^{\sstat, \gendistribution, \datamap}}
\newcommand{\naivefluctuation}{\fluctuationtensor^{\identity, \gendistribution, \datamap}}
\newcommand{\proposalfluctuation}{\fluctuationtensor^{\proposalstat, \gendistribution, \datamap}}
\newcommand{\mlnfluctuation}{\fluctuationtensor^{\mlnstat, \gendistribution, \datamap}}

\newcommand{\fprob}{p}
\newcommand{\fprobof}[1]{\fprob^{(#1)}}
\newcommand{\bidistof}[1]{B\left(#1\right)}
\newcommand{\widthwrtof}[2]{\omega_{#1}(#2)}
\newcommand{\widthatof}[2]{\widthwrtof{#1}{#2}}

\newcommand{\failprob}{\epsilon}
\newcommand{\precision}{\tau}
\newcommand{\maxgap}{\delta}

% For Basis Calculus
%\newcommand{\booleanvariable}{X}
%\newcommand{\booleanvariableof}[1]{\booleanvariable^{(#1)}}

%\newcommand{\legspace}{\mathcal{H}}
%\newcommand{\legspaceof}[1]{\legspace^{(#1)}}


%% CONTRACTION 

\newcommand{\cost}{c}
\newcommand{\costof}[1]{\cost\left( #1 \right)}

\newcommand{\temperature}{t}
\newcommand{\invtemp}{\beta}
\newcommand{\currentstate}{\Theta}
\newcommand{\newstate}{\Theta^{\rm new}}


%% Hard Logic

\newcommand{\kb}{\mathcal{KB}}
\newcommand{\kbvar}{\catvariableof{\kb}}
\newcommand{\kbat}[1]{\kb\left[#1\right]}

\newcommand{\seckb}{\tilde{\kb}}
%\newcommand{\qvariable}{q} % query variable
\newcommand{\thing}{\mathrm{Thing}}
\newcommand{\nothing}{\mathrm{Nothing}}

%% TIKZ




%% Tensor Network Formats

\newcommand{\elformat}{\mathrm{EL}}
\newcommand{\cpformat}{\mathrm{CP}}

\newcommand{\htformat}{\mathrm{HT}}
\newcommand{\ttformat}{\mathrm{TT}}

\newcommand{\subspaceof}[1]{V^{#1}}

%\newcommand{\contractionof}[2]{\mathcal{C}\left(#1,#2\right) }
%\newcommand{\sbcontractionof}[2]{\contractionof{\{#1\}}{\{#2\}}}
\newcommand{\contractionof}[2]{\left\langle #1\right\rangle \left[ #2 \right]}
\newcommand{\sbcontractionof}[2]{\contractionof{#1}{#2}}
%\newcommand{\sbcontractionof}[2]{\contractionof{\left\{#1\right\}}{#2}}

\newcommand{\contraction}[1]{\contractionof{#1}{\varnothing}}
\newcommand{\sbcontraction}[1]{\contraction{#1}}
%\newcommand{\sbcontraction}[1]{\contraction{\left\{#1\right\}}}

%\newcommand{\normationofwrt}[3]{\mathcal{N}\left(#1,#2,#3\right)}
%\newcommand{\sbnormationofwrt}[3]{\normationofwrt{\{#1\}}{\{#2\}}{\{#3\}}}
\newcommand{\normationofwrt}[3]{\left\langle #1\right\rangle \left[ #2 | #3 \right]}
\newcommand{\sbnormationofwrt}[3]{\normationofwrt{\left\{#1\right\}}{#2}{#3}}

\newcommand{\normationof}[2]{\normationofwrt{#1}{#2}{\varnothing}}
\newcommand{\sbnormationof}[2]{\sbnormationofwrt{#1}{#2}{\varnothing}}

\newcommand{\extnet}{\mathcal{T}^{\graph}} % [\catvariableof{\nodes}]
\newcommand{\secextnet}{\mathcal{T}^{\tilde{\graph}}} %[\catvariableof{\nodes}]
\newcommand{\extnetat}[1]{\extnet[#1]}

\newcommand{\objof}[1]{O\left(#1\right)}

\newcommand{\nodevariables}{\catvariableof{\nodes}}
\newcommand{\edgevariables}{\catvariableof{\edge}}
\newcommand{\extnetdist}{\normationof{\extnet}{\nodevariables}}


\newcommand{\extnetasset}{\{\hypercoreof{\edge}\, : \, \edge\in\edges\}}
\newcommand{\tnetof}[1]{\mathcal{T}^{#1}}
\newcommand{\tnetofat}[2]{\tnetof{#1}\left[#2\right]}
\newcommand{\exvariable}{\randomx}

%% Probability Representation
\newcommand{\randomx}{\catvariable}

\newcommand{\exrandom}{\catvariableof{0}}
\newcommand{\secexrandom}{\catvariableof{1}}
\newcommand{\thirdexrandom}{\catvariableof{2}}

\newcommand{\exrandind}{\catindexof{0}}
\newcommand{\exranddim}{\catdimof{0}}

\newcommand{\secexrandind}{\catindexof{1}}
\newcommand{\secexranddim}{\catdimof{1}}

\newcommand{\thirdexrandind}{\catindexof{2}}

\newcommand{\randomxof}[1]{\randomx_{#1}} % In combination with atomenumerator or tenumerator
\newcommand{\randome}{E}
\newcommand{\randomeof}[1]{\randome_{#1}}
\newcommand{\tenumerator}{t}
\newcommand{\tdim}{T}
\newcommand{\tenumeratorin}{\tenumerator\in[\tdim]}

%% Exponential families
\newcommand{\expdistof}[1]{\probtensorof{#1}}
\newcommand{\expdistofat}[2]{\expdistof{#1}[#2]}
\newcommand{\expdist}{\probtensorof{(\sstat,\canparam,\basemeasure)}}
\newcommand{\expfamily}{\Gamma^{\sstat,\basemeasure}}

\newcommand{\mnexpfamily}{\Gamma^{\graph}} % The exponential family of Markov Networks on \graph
\newcommand{\mlnexpfamily}{\Gamma^{\formulaset}}

\newcommand{\basemeasure}{\nu}
\newcommand{\basemeasureof}[1]{\basemeasure^{#1}}
\newcommand{\basemeasureofat}[2]{\basemeasure^{#1}\left[#1\right]}

%\newcommand{\sstat}{\tau} % sufficent statistics
\newcommand{\sstat}{\phi}
\newcommand{\proposalstat}{\fselectionmap^T}
\newcommand{\mlnstat}{\fselectionmap}
\newcommand{\naivestat}{\mathrm{Id}}

\newcommand{\sstatcoordinateof}[1]{\sstat_{#1}}
\newcommand{\sstatcoordinateofat}[2]{\sstat_{#1}{#2}}

\newcommand{\sstatcatof}[1]{\catvariableof{\sstatcoordinateof{#1}}}
\newcommand{\sstatindof}[1]{\catindexof{\sstatcoordinateof{#1}}}
\newcommand{\sencsstat}{\sencodingof{\sstat}}
\newcommand{\sencsstatat}[1]{\sencodingof{\sstat}\left[#1\right]}

\newcommand{\canparam}{\theta}
\newcommand{\canparamat}[1]{\canparam\left[#1\right]}
\newcommand{\canparamwrtat}[2]{\canparam^{#1}\left[#2\right]}
\newcommand{\estcanparam}{\hat{\canparam}}
\newcommand{\naivecanparam}{\tilde{\canparam}}
\newcommand{\naivecanparamat}[1]{\naivecanparam\left[#1\right]}

\newcommand{\meanparam}{\mu}
\newcommand{\meanparamof}[1]{\meanparam_{#1}}
\newcommand{\meanparamat}[1]{\meanparam\left[#1\right]}
\newcommand{\meanparamofat}[2]{\meanparamof{#1}\left[#2\right]}


\newcommand{\meanrepprob}{\probtensor^{\meanparam}}

\newcommand{\meanset}{\mathcal{M}}
\newcommand{\meansetof}[1]{\meanset_{#1}}

\newcommand{\datamean}{\meanparamof{\datamap}}
\newcommand{\datameanat}[1]{\datamean\left[#1\right]}

\newcommand{\genmean}{\meanparam^*}
\newcommand{\genmeanat}[1]{\genmean[#1]}

\newcommand{\currentmean}{\tilde{\meanparam}}


\newcommand{\cumfunctionwrt}[1]{A^{#1}}
\newcommand{\cumfunctionwrtof}[2]{\cumfunctionwrt{#1}(#2)}
\newcommand{\cumfunction}{\cumfunctionwrt{(\sstat,\basemeasure)}}
\newcommand{\cumfunctionof}[1]{\cumfunction(#1)}
\newcommand{\dualcumfunction}{\big(\cumfunction\big)^*}

\newcommand{\forwardmapwrt}[1]{F^{#1}}
\newcommand{\forwardmap}{\forwardmapwrt{(\sstat,\basemeasure)}}
\newcommand{\forwardmapwrtof}[2]{\forwardmapwrt{#1}(#2)}
\newcommand{\forwardmapof}[1]{\forwardmapwrtof{(\sstat,\basemeasure)}{#1}}

\newcommand{\backwardmapwrt}[1]{B^{#1}}
\newcommand{\backwardmap}{\backwardmapwrt{(\sstat,\basemeasure)}}
\newcommand{\backwardmapwrtof}[2]{\backwardmapwrt{#1}(#2)}
\newcommand{\backwardmapof}[1]{\backwardmapwrtof{(\sstat,\basemeasure)}{#1}}


\newcommand{\energytensor}{E}
\newcommand{\energytensorofat}[2]{\energytensor^{#1}[#2]}
\newcommand{\energytensorof}[1]{\energytensor^{#1}}
\newcommand{\energytensorat}[1]{\energytensor\left[#1\right]}
\newcommand{\expenergy}{\energytensorofat{(\sstat,\canparam,\basemeasure)}{\shortcatvariables}}
\newcommand{\expenergyat}[1]{\energytensorofat{(\sstat,\canparam,\basemeasure)}{#1}}

\newcommand{\energyhypothesis}{\Theta}
%\newcommand{\energyhypothesisat}[1]{\energyhypothesis^{#1}}
\newcommand{\energyhypothesisof}[1]{\energyhypothesis^{#1}}

\newcommand{\statenumerator}{\selindex} % Since statistics are selected by a single selection variable!
\newcommand{\statorder}{\seldim}  
\newcommand{\statenumeratorin}{\selindexin}
\newcommand{\parameterspace}{\rr^\seldim}

\newcommand{\essdistof}[1]{\mathbb{E}\sstat_{#1}} % expected sufficient statistics on distributions
\newcommand{\esspar}{\mathrm{ess}}
\newcommand{\essparof}[1]{\esspar\left(#1\right)} % expected sufficient statistics on distributions


%% Logical Reasoning
\newcommand{\kcore}{K}
\newcommand{\kcoreof}[1]{\kcore^{#1}}

\newcommand{\tbasis}{e_1}
\newcommand{\fbasis}{e_0}
\newcommand{\nbasis}{\ones}




\newcommand{\graph}{\mathcal{G}}
\newcommand{\graphof}[1]{\graph^{#1}}
\newcommand{\secgraph}{\tilde{\graph}}
\newcommand{\nodes}{\mathcal{V}}
\newcommand{\nodesof}[1]{\nodes^{#1}}
\newcommand{\innodes}{\nodesof{\mathrm{in}}}
\newcommand{\outnodes}{\nodesof{\mathrm{out}}}

\newcommand{\prenodes}{\{\secnode \, : \, \secnode \prec \node, \secnode\neq\node\}}
\newcommand{\afternodes}{\{\secnode \, : \, \node \prec \secnode, \secnode\neq\node\}}

\newcommand{\incomingnodes}{\edge^{\mathrm{in}}}
\newcommand{\outgoingnodes}{\edge^{\mathrm{out}}}

\newcommand{\nodesa}{A}
\newcommand{\nodesb}{B}
\newcommand{\nodesc}{C}

\newcommand{\secnodes}{\tilde{\nodes}}
\newcommand{\thirdnodes}{\bar{\nodes}}
\newcommand{\node}{v}
\newcommand{\nodein}{\node\in\nodes}
\newcommand{\secnode}{\tilde{\node}}
\newcommand{\thirdnode}{\bar{\node}}

\newcommand{\edges}{\mathcal{E}}
\newcommand{\edgesof}[1]{\edges^{#1}}
\newcommand{\secedges}{\tilde{\edges}}
\newcommand{\edge}{e}
\newcommand{\secedge}{\tilde{\edge}}
\newcommand{\thirdedge}{\hat{\edge}}
\newcommand{\edgein}{\edge\in\edges}

\newcommand{\parentsof}[1]{\mathrm{Pa}(#1)}
\newcommand{\nondescendantsof}[1]{\mathrm{NonDes}(#1)}

\newcommand{\extensor}{T}
\newcommand{\extensorspace}{\bigotimes_{\node\in\nodes}\rr^{\catdimof{\node}}}
\newcommand{\hypercore}{\extensor}
\newcommand{\hypercoreat}[1]{\extensor\left[#1\right]}

\newcommand{\hypercoreof}[1]{\hypercore^{#1}}
\newcommand{\hypercoreofat}[2]{\hypercoreof{#1}\left[#2\right]}
\newcommand{\sechypercore}{\tilde{\extensor}}
\newcommand{\sechypercoreat}[1]{\sechypercore\left[#1\right]}

%% Factored System

\newcommand{\onehotmap}{e}
\newcommand{\onehotmapof}[1]{\onehotmap_{#1}}
\newcommand{\onehotmapofat}[2]{\onehotmap_{#1}\left[#2\right]}
\newcommand{\onehotmapto}[1]{\onehotmapof{#1}} % For encoding of sets, relations
\newcommand{\invonehotmapof}[1]{\onehotmap^{-1}(#1)}

\newcommand{\sembedding}{\phi} % Selection Embedding
\newcommand{\sembeddingof}[1]{\sembedding^{#1}}

\newcommand{\facsystem}{\mathcal{F}}





\newcommand{\exfunction}{f}
\newcommand{\exfunctiontargetspace}{\bigotimes_{l\in[p]}\rr^{\catdimof{l}}}
\newcommand{\exfunctiontargetvariables}{Y_0,\ldots,Y_{p-1}}

\newcommand{\secexfunction}{g}

\newcommand{\catleg}{\randomx}
\newcommand{\catlegof}[1]{\catleg_{#1}}


%% Message Passing
\newcommand{\cluster}{C}
\newcommand{\clusterof}[1]{\cluster_{#1}}
\newcommand{\clusterenumerator}{i}
\newcommand{\secclusterenumerator}{j}
\newcommand{\thirdclusterenumerator}{\tilde{j}}

\newcommand{\enc}{\clusterof{\clusterenumerator}}
\newcommand{\secenc}{\clusterof{\secclusterenumerator}}
\newcommand{\thirdenc}{\clusterof{\thirdclusterenumerator}}

\newcommand{\clusterorder}{n}
\newcommand{\clusterenumeratorin}{\clusterenumerator\in[\clusterorder]}

\newcommand{\upmes}[2]{\delta_{#1 \rightarrow #2}}
\newcommand{\downmes}[2]{\delta_{#2 \leftarrow #1}}

% OLD and not used in main document
%\newcommand{\resources}{R}
%\newcommand{\stortensor}{T}
%\newcommand{\cmmatrix}{C}
%\newcommand{\emb}{\tau}

% Binary connective symbols
\newcommand{\eqbincon}{\Leftrightarrow}
\newcommand{\lpasbincon}{\triangleleft}

\newcommand{\indexinterpretationof}[1]{I^{#1}}
\newcommand{\indexinterpretationofat}[2]{\indexinterpretationof{#1}(#2)}

%% CONTRACTIONS
\newcommand{\contractionof}[2]{\left\langle #1\right\rangle \left[ #2 \right]}

\newcommand{\breakablecontractionof}[2]{\big\langle #1 \big\rangle \big[ #2 \big]}
\newcommand{\sbcontractionof}[2]{\contractionof{#1}{#2}}
\newcommand{\contraction}[1]{\contractionof{#1}{\varnothing}}
\newcommand{\sbcontraction}[1]{\contraction{#1}}
\newcommand{\normationofwrt}[3]{\left\langle #1\right\rangle \left[ #2 | #3 \right]}
\newcommand{\sbnormationofwrt}[3]{\normationofwrt{#1}{#2}{#3}}
\newcommand{\normationof}[2]{\normationofwrt{#1}{#2}{\varnothing}}
\newcommand{\sbnormationof}[2]{\sbnormationofwrt{#1}{#2}{\varnothing}}

\newcommand{\nzcontractionof}[2]{\nonzerocirc\contractionof{#1}{#2}}

%% ENCODING SCHEMES
\newcommand{\rencodingof}[1]{\rho^{#1}}
\newcommand{\rencodingofat}[2]{\rencodingof{#1}\left[#2\right]}

\newcommand{\sencodingof}[1]{\gamma^{#1}}
\newcommand{\sencodingofat}[2]{\sencodingof{#1}\left[#2\right]}  

\newcommand{\bencodingof}[1]{\beta^{#1}}
\newcommand{\bencodingofat}[2]{\bencodingof{#1}\left[#2\right]}  

\newcommand{\linmap}{F}
\newcommand{\linmapof}[1]{\linmap^{#1}}
\newcommand{\linmapofat}[2]{\linmapof{#1}\left(#2\right)}
\newcommand{\linmapspace}{\mathbb{L}}

%% Directed Tensor Calculus
\newcommand{\exrelation}{\mathcal{R}}
\newcommand{\exrelationof}[1]{\exrelation^{#1}}
\newcommand{\arbset}{\mathcal{U}}
\newcommand{\arbsetof}[1]{\arbset^{#1}}
\newcommand{\arbsubset}{\mathcal{V}} % Conflicts with nodes!
\newcommand{\arbelement}{u}
\newcommand{\arbelementin}{\arbelement\in\arbset}

\newcommand{\insymbol}{\mathrm{in}}
\newcommand{\outsymbol}{\mathrm{out}}
\newcommand{\inset}{\arbsetof{\insymbol}}
\newcommand{\outset}{\arbsetof{\outsymbol}}

\newcommand{\incatindex}{\catindexof{\insymbol}}
\newcommand{\outcatindex}{\catindexof{\outsymbol}}

%% Sparse Tensor Calculus
\newcommand{\sparsityof}[1]{\ell_0\left(#1\right)}


%% MAIN VARIABLES
\newcommand{\indvariable}{O} 
\newcommand{\inddim}{r}
\newcommand{\indindex}{o} % was s
\newcommand{\indenumerator}{l}
\newcommand{\indorder}{n} % number of variables 

\newcommand{\selvariable}{L} 
\newcommand{\seldim}{p}
\newcommand{\selindex}{l}
\newcommand{\selenumerator}{s}
\newcommand{\selorder}{n}

\newcommand{\catvariable}{X} 
\newcommand{\catdim}{m}
\newcommand{\catindex}{x} % was i
\newcommand{\catenumerator}{\atomenumerator}
\newcommand{\catorder}{\atomorder}

\newcommand{\headvariable}{Y} % head of a relational encoding
\newcommand{\headdim}{n}
\newcommand{\headindex}{y}

\newcommand{\datvariable}{J} % Can be understood as selvariable selecting a datapoint, catvariable as a random datapoint, indvariable as a abstract object representing the sample, also used at indexvariable!
\newcommand{\datdim}{m}
\newcommand{\datindex}{j}

%% Syntactic Sugar
\newcommand{\indvariableof}[1]{\indvariable_{#1}}
\newcommand{\selvariableof}[1]{\selvariable_{#1}}
\newcommand{\catvariableof}[1]{\catvariable_{#1}}
\newcommand{\headvariableof}[1]{\headvariable_{#1}}

\newcommand{\indvariablelist}{\indvariableof{0},\ldots,\indvariableof{\individualorder-1}}
\newcommand{\catvariablelist}{\catvariableof{0},\ldots,\catvariableof{\atomorder-1}}
\newcommand{\selvariablelist}{\selvariableof{0},\ldots,\selvariableof{\selorder-1}}

\newcommand{\shortindvariablelist}{\indvariableof{[\individualorder]}}
\newcommand{\shortcatvariablelist}{\catvariableof{[\atomorder]}}
\newcommand{\shortselvariablelist}{\selvariableof{[\selorder]}}

\newcommand{\selindices}{\selindexof{0},\ldots,\selindexof{\selorder-1}}

\newcommand{\shortindindices}{\indindexof{[\indorder]}}
\newcommand{\shortcatindices}{\catindexof{[\catorder]}}
\newcommand{\shortselindices}{\selindexof{[\selorder]}}

\newcommand{\inddimof}[1]{\inddim_{#1}}
\newcommand{\seldimof}[1]{\seldim_{#1}}
\newcommand{\catdimof}[1]{\catdim_{#1}}
\newcommand{\headdimof}[1]{\headdim_{#1}}

\newcommand{\indindexof}[1]{\indindex_{#1}}
\newcommand{\selindexof}[1]{\selindex_{#1}}
\newcommand{\catindexof}[1]{\catindex_{#1}} 
\newcommand{\headindexof}[1]{\headindex_{#1}}

\newcommand{\indindexin}{\indindex\in[\inddim]}
\newcommand{\selindexin}{\selindex\in[\seldim]}
\newcommand{\catindexin}{\catindex\in[\catdim]}
\newcommand{\datindexin}{\datindex\in[\datdim]}

\newcommand{\indindexofin}[1]{\indindexof{#1}\in[\inddimof{#1}]}
\newcommand{\catindexofin}[1]{\catindexof{#1}\in[\catdimof{#1}]}
\newcommand{\selindexofin}[1]{\selindexof{#1}\in[\seldimof{#1}]}

\newcommand{\indindexlist}{\indindexof{0},\ldots,\indindexof{\indorder-1}}
\newcommand{\catindexlist}{\catindexof{0},\ldots,\catindexof{\atomorder-1}}
\newcommand{\selindexlist}{\selindexof{0},\ldots,\selindexof{\selorder-1}}

\newcommand{\indenumeratorin}{\indenumerator\in[\indorder]}
\newcommand{\selenumeratorin}{\selenumerator\in[\selorder]}
\newcommand{\catenumeratorin}{\catenumerator\in[\catorder]}

\newcommand{\indexedindvariableof}[1]{\indvariableof{#1}=\indindexof{#1}}
\newcommand{\indexedcatvariableof}[1]{\catvariableof{#1}=\catindexof{#1}}
\newcommand{\indexedselvariableof}[1]{\selvariableof{#1}=\selindexof{#1}}
\newcommand{\indexedheadvariableof}[1]{\headvariableof{#1}=\headindexof{#1}}

\newcommand{\indexedindvariable}{\indexedindvariableof{}}
\newcommand{\indexedcatvariable}{\indexedcatvariableof{}}
\newcommand{\indexedselvariable}{\indexedselvariableof{}}

\newcommand{\catstatesof}[1]{[\catdimof{#1}]}

\newcommand{\catspaceof}[1]{\rr^{\catdimof{#1}}}

\newcommand{\indspace}{\bigotimes_{\indenumeratorin}\rr^{\inddim}}
\newcommand{\catspace}{\bigotimes_{\atomenumeratorin} \rr^{\catdimof{\atomenumerator}}}

\newcommand{\selstates}{\bigtimes_{\selenumeratorin}[\seldimof{\selenumerator}]}
\newcommand{\selvectorspace}{\rr^{\seldim}}
\newcommand{\selspace}{\bigotimes_{\selenumeratorin}\rr^{\seldimof{\selenumerator}}}
%%

\newcommand{\datanum}{\datdim}

\newcommand{\datain}{\datindex\in[\datanum]}
\newcommand{\data}{\{\datapointof{\datindex}\}_{\datindexin}}
\newcommand{\dataaverage}{\frac{1}{\datanum}\sum_{\datindexin}}

\newcommand{\catvariables}{\catvariablelist}
\newcommand{\shortcatvariables}{\shortcatvariablelist}
\newcommand{\indexedshortcatvariables}{\shortcatvariables=\shortcatindices}
\newcommand{\shortcatindicesin}{\shortcatindices\in\facstates}
\newcommand{\shortatomindicesin}{\shortcatindices\in\atomstates}
\newcommand{\datshortcatvariables}{\shortcatvariables=\shortcatindices^{\datindex}}

\newcommand{\shortindvariables}{\shortindvariablelist}
\newcommand{\indexedshortindvariables}{\shortindvariables=\shortindindices}
\newcommand{\datshortindvariables}{\shortindvariables=\shortindindices^{\datindex}}

\newcommand{\selvariables}{\selvariableof{0},\ldots,\selvariableof{\selorder-1}}
\newcommand{\shortselvariables}{\selvariableof{[\selorder]}}
\newcommand{\indexedshortselvariables}{\shortselvariables=\shortselindices}
\newcommand{\secselenumerator}{\tilde{\selenumerator}}

\newcommand{\nodestatesof}[1]{\bigtimes_{\node\in#1}\catstatesof{\node}}
\newcommand{\atomstates}{\bigtimes_{\atomenumeratorin}[2]}


\newcommand{\symindstates}{\bigtimes_{\indenumeratorin}[\inddim]}

\newcommand{\facstates}{\bigtimes_{\atomenumeratorin}\catstatesof{\atomenumerator}}
\newcommand{\facdim}{\prod_{\atomenumeratorin}\catdimof{\atomenumerator}}
\newcommand{\secfacstates}{\bigtimes_{\secatomenumerator\in[\secatomorder]}\catstatesof{\secatomenumerator}}

\newcommand{\atomspace}{\bigotimes_{\atomenumeratorin}\rr^2}
\newcommand{\facspace}{\catspace}
\newcommand{\secfacspace}{\bigotimes_{\secatomenumerator\in[\seccatorder]} \rr^{\catdimof{\secatomenumerator}}}

\newcommand{\indexedcatvariables}{\indexedcatvariableof{0},\ldots,\indexedcatvariableof{\atomorder-1}} 
\newcommand{\tildeindexedcatvariables}{\catvariableof{0}=\tilde{\catindex}_0,\ldots,\catvariableof{\atomorder-1}=\tilde{\catindex}_{\atomorder-1}} 

\newcommand{\seccatenumerator}{\tilde{\catenumerator}}
\newcommand{\seccatenumeratorin}{\seccatenumerator\in[\catorder]}

\newcommand{\seccatvariable}{Y} % used as differentiation variable
\newcommand{\seccatindex}{y}
\newcommand{\seccatorder}{p} % Has to coincide with seldim for relational encoding def

\newcommand{\seccatvariableof}[1]{\seccatvariable_{#1}}
\newcommand{\indexedseccatvariableof}[1]{\seccatvariableof{#1}=\seccatindexof{#1}}
\newcommand{\seccatvariables}{\seccatvariableof{0},\ldots,\seccatvariableof{\seccatorder\shortminus1}}
\newcommand{\secshortcatvariables}{\seccatvariableof{[\seccatorder]}}
\newcommand{\indexedseccatvariables}{\indexedseccatvariableof{0}\ldots,\indexedseccatvariableof{\seccatorder-1}} 
\newcommand{\indexedsecshortcatvariables}{\indexedseccatvariableof{[\seccatorder]}}

\newcommand{\catvariablesinset}[1]{\catvariableof{#1}}%{\catvariableof{\node} \, : \, \node \in #1}
\newcommand{\seccatindexof}[1]{\seccatindex_{#1}}

\newcommand{\catindices}{\catindexlist}
\newcommand{\tildecatindexof}[1]{\tilde{\catindex}_{#1}}
\newcommand{\tildecatindices}{\tildecatindexof{0},\ldots,\tildecatindexof{\atomorder-1}}
\newcommand{\seccatindices}{{\seccatindexof{0},\ldots,\seccatindexof{\secatomorder-1}}}
\newcommand{\tildeshortcatindices}{\tildecatindexof{[\catorder]}}

\newcommand{\catindicesof}[1]{{\catindexof{0}^{#1},\ldots,\catindexof{\atomorder-1}^{#1}}}

\newcommand{\catzeropositions}{\{\atomenumerator : \catindexof{\atomenumerator}=0\}}
\newcommand{\catonepositions}{\{\atomenumerator : \catindexof{\atomenumerator}=0\}}

%% Cores
\newcommand{\categoricalmap}{Z}
\newcommand{\categoricalmapat}[1]{\categoricalmap\left[#1\right]}
\newcommand{\categoricalmapof}[1]{\categoricalmap^{#1}}
\newcommand{\categoricalmapofat}[2]{\categoricalmap^{#1}\left[#2\right]}

\newcommand{\categoricalcore}{\rencodingof{\categoricalmap}}
\newcommand{\categoricalcoreof}[1]{\rencodingof{\categoricalmapof{#1}}}
\newcommand{\categoricalcoreofat}[2]{\rencodingof{\categoricalmapof{#1}}\left[#2\right]}

\newcommand{\datamap}{D}
\newcommand{\datamapat}[1]{\datamap(#1)}
\newcommand{\datamapof}[1]{\datamap_{#1}}
\newcommand{\datapointof}[1]{\datamapat{#1}}
\newcommand{\datapoint}{\datapointof{\datindex}}
\newcommand{\dataset}{\left((\catindicesof{\datindex})\,:\,\datindexin\right)}

\newcommand{\secdatamap}{\tilde{\datamap}}
\newcommand{\datacore}{\rencodingof{\datamap}}
\newcommand{\datacoreat}[1]{\datacore\left[#1\right]}
\newcommand{\datacoreof}[1]{\rencodingof{\datamap_{#1}}}
\newcommand{\datacoreofat}[2]{\rencodingof{\datamap_{#1}}[#2]}

\newcommand{\secdatacoreof}[1]{\rencodingof{\secdatamap_{#1}}}
\newcommand{\empdistribution}{\probtensor^{\datamap}}
\newcommand{\empdistributionat}[1]{\empdistribution\left[#1\right]}
\newcommand{\empdistributionwith}{\empdistributionat{\shortcatvariables}}

\newcommand{\varcore}[1]{U^{#1}} % For optimization of tensor network
\newcommand{\varspace}[1]{\rr^{p_{#1}}}
\newcommand{\varcollection}{\big\{\varcore{\atomenumerator}\, :\, \atomenumeratorin \big\}}

\newcommand{\headcore}{W} % activation of a formula, typical exp
\newcommand{\headcoreof}[1]{\headcore^{#1}}
\newcommand{\headcoreat}[1]{\headcore\left[#1\right]}
\newcommand{\headcoreofat}[2]{\headcore^{#1}[#2]}

\newcommand{\actcore}{W} % activation of a formula, typical exp
\newcommand{\actcoreof}[1]{\actcore^{#1}}
\newcommand{\actcoreofat}[2]{\actcore^{#1}[#2]}

%% CP Decomposition
\newcommand{\legcore}{V}
\newcommand{\legcoreof}[1]{\legcore^{#1}}
\newcommand{\legcoreofat}[2]{\legcoreof{#1}\left[#2\right]}

\newcommand{\scalarcore}{\sigma}
\newcommand{\scalarcoreof}[1]{\scalarcore[#1]}
\newcommand{\scalarcoreat}[1]{\scalarcore[#1]}
\newcommand{\scalarcoreofat}[2]{\scalarcore^{#1}[#2]}

% DecompositionIndex 
\newcommand{\decvariable}{I}
\newcommand{\decvariableof}[1]{\decvariable_{#1}}
\newcommand{\decindex}{i} % Needs to be different to datindex!
\newcommand{\decindexof}[1]{\decindex_{#1}}
\newcommand{\indexeddecvariableof}[1]{\decvariableof{#1}=\decindexof{#1}}
\newcommand{\decdim}{n}
\newcommand{\decdimof}[1]{\decdim_{#1}}
\newcommand{\decindexin}{\decindex\in[\decdim]}
\newcommand{\indexeddecvariable}{\decvariable=\decindex}
\newcommand{\inddecvar}{\indexeddecvariable}

\newcommand{\indexeddatvariable}{\datvariable=\datindex}

% Used in poly sparsity
\newcommand{\indexvariable}{\datvariable} % for datacores used
\newcommand{\indexset}{J}
\newcommand{\indexsetof}[1]{\indexset^{#1}}

\newcommand{\slackvariable}{z}
\newcommand{\slackindex}{z}
\newcommand{\slackindexof}[1]{\slackindex_{#1}}

\newcommand{\rankofat}[2]{\mathrm{rank}^{#1}\left(#2\right)}
\newcommand{\cprankof}[1]{\mathrm{rank}\left(#1\right)}
\newcommand{\bincprankof}[1]{\mathrm{rank}^{\mathrm{bin}}\left(#1\right)}
\newcommand{\slicesparsityof}[1]{\slicerankwrtof{\catorder}{#1}} % former {\tilde{\ell} \left(#1\right)}


\newcommand{\dircprankof}[1]{\mathrm{rank}^{\mathrm{dir}}\left(#1\right)}
\newcommand{\bascprankof}[1]{\mathrm{rank}^{\mathrm{bas}}\left(#1\right)}
\newcommand{\baspluscprankof}[1]{\mathrm{rank}^{\mathrm{bas+}}\left(#1\right)}
\newcommand{\quacprankof}[1]{\mathrm{rank}^{\mathrm{qua}}\left(#1\right)}

\newcommand{\sliceset}{\mathcal{M}}
\newcommand{\slicescalar}{\lambda}
\newcommand{\slicescalarof}[1]{\slicescalar^{#1}}
\newcommand{\slicetupleof}[1]{(\slicescalar^{#1}, \variablesetof{#1}, \catindexof{\variablesetof{#1}}^{#1})}
\newcommand{\enumeratedslices}{\{\slicetupleof{\decindex} \, : \, \decindexin\}}

\newcommand{\sliceorder}{r}
\newcommand{\slicerankwrtof}[2]{\mathrm{rank}^{#1}\left(#2\right)}

\usetikzlibrary {arrows.meta} 
\usetikzlibrary{shapes,positioning}
\usetikzlibrary{decorations.markings}
\tikzset{
    midarrow/.style={
        postaction={decorate},
        decoration={markings, mark=at position 0.5 with {\arrow{>}}}
    },
    midbackarrow/.style={
        postaction={decorate},
        decoration={markings, mark=at position 0.5 with {\arrow{<}}}
    },
     ->/.style={midarrow},
     <-/.style={midbackarrow}
}

\newcommand{\skeletoncolor}{blue}
\newcommand{\arrowstyle}{->}

\newcommand{\shortminus}{\scalebox{0.4}[1.0]{$-$}}

\newcommand{\drawvariabledot}[2]{
	\draw[fill] (#1,#2) circle (0.15cm);
}

% Draws indices and below the indices the core
\newcommand{\drawatomindices}[2]{
	\begin{scope}[shift={(#1,#2)}]
		\draw[<-] (0,1)--(0,-1) node[midway,left] {\tiny $\catvariableof{0}$}; 
		\draw[<-] (1.5,1)--(1.5,-1) node[midway,left] {\tiny $\catvariableof{1}$}; 
		\node[anchor=center] (text) at (3,0) {$\cdots$};
		\draw[<-] (4,1)--(4,-1) node[midway,right] {\tiny $\catvariableof{\atomorder\shortminus1}$}; 
	\end{scope}
}
\newcommand{\drawundiratomindices}[2]{
	\begin{scope}[shift={(#1,#2)}]
		\draw[] (0,1)--(0,-1) node[midway,left] {\tiny $\catvariableof{0}$}; 
		\draw[] (1.5,1)--(1.5,-1) node[midway,left] {\tiny $\catvariableof{1}$}; 
		\node[anchor=center] (text) at (3,0) {$\cdots$};
		\draw[] (4,1)--(4,-1) node[midway,right] {\tiny $\catvariableof{\atomorder\shortminus1}$}; 
	\end{scope}
}
\newcommand{\drawparindices}[2]{
	\begin{scope}[shift={(#1,#2)}]
		\draw (0,1)--(0,-1) node[midway,left] {\tiny $\parindexof{1}$}; 
		\draw (1.5,1)--(1.5,-1) node[midway,left] {\tiny $\parindexof{2}$}; 
		\node[anchor=center] (text) at (3,0) {$\cdots$};
		\draw (4,1)--(4,-1) node[midway,right] {\tiny $\parindexof{\parorder}$}; 
	\end{scope}
}
\newcommand{\drawatomcore}[3]{
	\begin{scope}[shift={(#1,#2)}]
		\draw (-1,-1) rectangle (5,-3);
		\node[anchor=center] (text) at (2,-2) {#3};
	\end{scope}
}
\newcommand{\drawsmallcore}[3]{
	\begin{scope}[shift={(#1,#2)}]
		\draw (-1,1) rectangle (1,-1);
		\node[anchor=center] (text) at (0,0) {#3};
	\end{scope}
}


\newcommand{\drawformulatensor}{
    \drawformulatensorof{\ftensorof{\exformula}}
    }
    
\newcommand{\drawformulatensorof}[1]{
    \draw (-5,1) rectangle (5,-1);
    \node at (0,-1) [above] {$#1$} ;
    \draw (-4,-1)--(-4,-3) node[midway,left] {$\variableindexof{1}$};
    \draw (4,-1)--(4,-3) node[midway,left] {$\variableindexof{\variableorder}$};
    \node at (0,-3.1) [above] {$\cdots$};
    }
       
    
\newcommand{\drawskeleton}{
	\renewcommand{\skeletoncolor}{}
    	\draw[\skeletoncolor,\arrowstyle] (-4,-1)--(-4,-3) node[midway,left] {$\variableindexof{1}$};
	\draw[\skeletoncolor,\arrowstyle] (-2,-1)--(-2,-3) node[midway,right] {$\variableindexof{2}$};
    	\draw[\skeletoncolor,\arrowstyle] (4,-1)--(4,-3) node[midway,left] {$\variableindexof{\variableorder}$};

	\draw[\skeletoncolor]  (-5,-3) rectangle (5,-5);
	\node[\skeletoncolor]  at (0,-5.1) [above] {$\gtensorof{\skeleton}$} ;
    	\draw[\skeletoncolor,\arrowstyle] (-4,-5)--(-4,-7) node[midway,left] {$\variableindexof{1}$};
	\draw[\skeletoncolor,\arrowstyle]  (-2,-5)--(-2,-7) node[midway,right] {$\variableindexof{2}$};
   	\draw[\skeletoncolor,\arrowstyle]  (4,-5)--(4,-7) node[midway,left] {$\variableindexof{\variableorder}$};
	\node[\skeletoncolor] at (1,-7) [above] {$\ldots$};
}
    
\newcommand{\drawatomdecomposition}{
	\draw (-5,1) rectangle (-1,-1);
    	\node at (-3,-1.1) [above] {$\gtensorof{\atomicformulaof{1}}$} ;
	\node at (0.65,-1.1) [above] {$\cdots$};
	\draw (2,1) rectangle (6,-1);
    	\node at (4,-1.1) [above] {$\gtensorof{\atomicformulaof{\atomorder}}$} ;
	\drawskeleton
  }
  


\newcommand{\drawchadamardcore}{
	\node at (0,0) [left,\skeletoncolor] {$\delta$};
	\draw[\skeletoncolor]  (0,2) -- (0,0);% node[midway,left] {$\placeholderof{1}$};
	\draw[fill,\skeletoncolor] (0,0) circle (0.25cm);
	\draw[\skeletoncolor,\arrowstyle] (0,0) -- (0,-2) node[midway,right] {$\variableindexof{2}$};
	\draw[\skeletoncolor] (4,2) to[bend left=20] (0,0);
	\draw[\skeletoncolor,\arrowstyle] (-2,2) -- (-2,-2) node[midway,left] {$\variableindexof{1}$};
}


\newcommand{\drawchadamardcoretwocontractions}{
	\node at (0,0) [left,\skeletoncolor] {$\delta$};
	\draw[\skeletoncolor]  (0,2) -- (0,0);% node[midway,left] {$\placeholderof{1}$};
	\draw[fill,\skeletoncolor] (0,0) circle (0.25cm);
	\draw[\skeletoncolor,\arrowstyle] (0,0) -- (0,-2) node[midway,right] {$\variableindexof{2}$};
	\draw[\skeletoncolor] (5,2) to[bend left=20] (0,0);


	\draw[fill,\skeletoncolor] (-2,0.5) circle (0.25cm);
	\draw[\skeletoncolor] (3,2) to[bend left=20] (-2,0.5);
	\draw[\skeletoncolor,\arrowstyle] (-2,2) -- (-2,-2) node[midway,left] {$\variableindexof{1}$};
}

  
\newcommand{\drawchadamard}{
	\begin{scope}[shift={(-7,0)}]
	  	\draw[dashed]  (-3,2) rectangle (1,4);
		\node at (-1,1.9) [above] {$\gtensorof{\exformula\land\secexformula}$};
		\draw (-2,2) -- (-2,-2) node[midway,left] {$\variableindexof{1}$};
		\draw (0,2) -- (0,-2) node[midway,right] {$\variableindexof{2}$};	
	\end{scope}
	\node at (-4.5,1.9) [above] {$=$};	
  	\draw[dashed] (-3,2) rectangle (1,4);
	\node at (-1,1.9) [above] {$\gtensorof{\exformula}$};
  	\draw[dashed] (3,2) rectangle (5,4);
	\node at (4,1.9) [above] {$\gtensorof{\secexformula}$};
  	\drawchadamardcore
  }
  
  
  \newcommand{\drawanegationcore}{
    	\begin{scope}[shift={(-2,-2)}]
  		\draw[\skeletoncolor] (-8,2) rectangle (-4,4);
		\node at (-6,2) [above,\skeletoncolor] {$\ones$};
		\draw[\skeletoncolor,\arrowstyle] (-7,2) -- (-7,0) node[midway,left] {$\variableindexof{1}$};
		\draw[\skeletoncolor,\arrowstyle] (-5,2) -- (-5,0) node[midway,right] {$\variableindexof{2}$};
	\end{scope}
	
	\draw[\skeletoncolor,\arrowstyle] (-2,2) -- (-2,-2) node[midway,left] {$\variableindexof{1}$};
	\draw[\skeletoncolor,\arrowstyle](0,2) -- (0,-2) node[midway,right] {$\variableindexof{2}$};
	\node[\skeletoncolor] at (-4.5,0) [above] {$-$};
  }
  
  \newcommand{\drawanegation}{

  	\begin{scope}[shift={(-14,0)}]
	  	\draw[dashed]  (-3,2) rectangle (1,4);
		\node at (-1,1.9) [above] {$\gtensorof{\lnot\exformula}$};
		\draw (-2,2) -- (-2,-2) node[midway,left] {$\variableindexof{1}$};
		\draw (0,2) -- (0,-2) node[midway,right] {$\variableindexof{2}$};	
	\end{scope}

	\drawanegationcore



	\node at (-11.5,1.9) [above] {$=$};	
	
  	\draw[dashed]  (-3,2) rectangle (1,4);
	\node at (-1,1.9) [above] {$\gtensorof{\exformula}$};	
  
  	  }

\pretolerance=500
\tolerance=100
\emergencystretch=10pt

% Bibliography
\DeclareUnicodeCharacter{FB01}{fi}
\usepackage[round]{natbib}


\newcommand{\red}[1]{\textcolor{red}{#1}}

\begin{document}
    \title{The Tensor Network Approach to Efficient and Explainable AI}
    \author{Alex Goessmann, DATEV eG}

    \maketitle
    \date{\today}

    \begin{abstract}
        While tensors appear naturally in artificial intelligence as factored representations of systems, their decompositions into networks improve the efficiency and explainability of several approaches.
        Since the curse of dimensionality prevents feasible generic representations and reasoning, logical and probabilistic reasoning focuses on tradeoffs between efficiency and generality.
        In this work we present these tradeoffs based on the tensor network formalism and formulate feasible reasoning algorithms based on tensor network contractions.
        We review the classical logical and probabilistic approaches to reasoning in the first part and develop applications in neuro-symbolic AI in the second part.
        In the third part we investigate in more detail schemes to exploit tensor network contractions for calculus.
    \end{abstract}

    \tableofcontents

    \section{Introduction}

% Explaining the title
Artificial intelligence is a long-standing dream of humanity, which has in recent years received enourmous attention, driven by breakthroughs in large language models.
Among the key priorities towards an economic and trustworthy usage remain the creation of efficient and the explainable models.

% Explainability
Instead of post-hoc explainability of a models inference given specific data, our aim in this work is the intrinsic human understandability of a model.
We are motivated by the theory of logic, which formalization of human thoughts serves as an interface between mechanized reasoning on a machine and human understandability.
Having established this advanced form of explainability enables novel forms of human interactions with a model based on verbalizations, manipulations and guarantees on the models inference output.

% Efficiency
The desire of an efficient model originates more from an economic perspective on the realizability of a model and its power consumption.
Tensors appear naturally as representations of a system with multiple variables, both in logical and probabilistic approaches towards artificial intelligence. % avoid factored at this point!
However, already for moderate numbers of variables, the curse of dimensionality prevents a typical machines memory to store a generic representation.
The careful design of representation formats is therefore a necessary task to avoid the exponential increase of storage demands and balance the expressivity and the efficiency of representation formats.

% Tensor Networks
We in this work exploit the formalism of tensor networks in the creation of efficient representation schemes.
The chosen tensor network formats are motivated from explainable learning architectures and provide a synergy between the aims of efficiency and explainability.
Tensor networks appear as the natural numerical structures in probabilistic graphical models and logical knowledge bases.
After presenting the probabilistic and logical approaches based on the tensor network formalism we develop novel applications schemes towards neuro-symbolic artificial intelligence.

\subsection{Background}

Before presenting an overview over the contents, we further motivate this work based on the broach approaches towards artificial intelligence and more recent developments.

\subsubsection{Classical Approached towards AI}

We start with ontological commitments in the description of a system and follow the book \cite{russell_artificial_2021} distinguishing atomic, factored and structured representations.
While in atomic representation, the states of a systems are enumerated and represented in a single variable, factored representations describe a systems state based on a collection of variables.
In the tensor formalism, each state of a system corresponds with a coordinate of a representing tensor.
The order of the tensor coincides therefore with the number of variables in a system.
In an atomic representation, where there is a single coordinate, each state corresponds with a coordinate of the representing vector being a tensor of order one.
Having a factored representation with two variables requires order two tensors or matrices, where a coordinate is specified by a row and a column index.
Given larger numbers of coordinates now extends this representation picture to tensors of larger orders, which have more abstract axes besides rows and columns.
The generalization of the atomic representation to a factored system thus corresponds with the generalization of vectors towards matrices and tensors of larger orders.
Along this line, we can always transform a factored representation of a system to an atomic one, just by enumerating the states of the factored system and interpreting them by a single variable.
This amounts to the flattening of a representing tensor to a vector.
However, by doing so, we would loose much of the structure of the representation, which we would like to exploit in reasoning processes.

% Structured Representations
A more generic representation of systems are structured representation.
Structured representations involve objects of differing numbers and relations between them.
As a consequence the numbers of variables can differ depending on the state of a system.
This poses a challenge to the tensor representation, since a fixed number of variables is required to motivate a tensor space of representations.
There are approaches to circumvent these difficulty by the development of template models such as Markov Logic Networks \cite{richardson_markov_2006}, which are instantiated on systems with differing number of objects.
We will discuss those in \charef{cha:folModels}.

% Continous vs discrete
In this work we treat discrete systems, where the number of states is finite.
One can understand them as a discretization of continuous variables and many results will generalize by the converse limit to the situation of continuous variables.

% Epistemologic
Besides ontological commitments in the choice of a representation scheme, modelling a system also requires epistemologic commitments, by defining what properties are to be reasoned about.
In logical approaches the properties of states are boolean values representing whether a state is consistent with known constraints.
Probabilistic approaches assign to the coordinates of the tensors numbers in $[0,1]$ encoding the probability of a state.
Compared with logical approached to reasoning, probabilistic approaches thus bear a more expressive modelling.

\subsubsection{Logic and Explainability in AI}

\textbf{Inductive Logic Programming:}
\begin{itemize}
    \item ILP is a classical task \cite{muggleton_inductive_1994}
    \item Amie \cite{galarraga_amie_2013} is a method of learning Horn clauses using a refinement operator.
    \item Class Expression Learning \cite{lehmann_class_2011} is a more recent approach to assist in the design of reasoning capabilities in Knowledge Graphs.
        However, problems arise from the expressivity of description logics and the efficient choice of formulas from exponentially large hypothesis sets.
    \item CEL has therefore recently received further popularity in combination with reinforcement learning \cite{demir_drill-_2021} and neural networks \cite{kouagou_neural_2022, pesquita_neural_2023}, which are methods searching efficienctly in exponentially large spaces of formulas.
\end{itemize}

\textbf{Statistical Relational AI:} \cite{getoor_introduction_2019}
\begin{itemize}
    \item Classical combination of logical and probabilistic approaches to reasoning
\end{itemize}

\textbf{Knowledge Graphs}
\cite{hogan_knowledge_2021}
\begin{itemize}
    \item The advent of large Knowledge Graphs enables explainable reasoning methods on structured data.
    \item Knowledge Graphs are stored in a sparse format, i.e. only true atoms instead of all + truth label.
\end{itemize}


\subsubsection{Tensor Networks in AI}

\textbf{Tensor Network formats}
\begin{itemize}
    \item HT Format \cite{hackbusch_new_2009}
    \item CP Format
\end{itemize}

\textbf{Tensor Networks as Regressors}
\begin{itemize}
    \item Dynamical Systems learning \cite{gels_multidimensional_2019, goesmann_tensor_2020}
    \item Supervised learning CITE: Stoudenmire etc.
\end{itemize}

\textbf{Tensor Representation of Logics}
\begin{itemize}
    \item Tensor Networks have been applied in the automatization of logic reasoning \cite{li_linear_2017, sato_linear_2017} apply Matrix multiplication in reasoning.
    \item \cite{nickel_review_2016} review over relational machine learning and latent features via matrix embeddings.
\end{itemize}

\textbf{Tensor Representation of Knowledge Graphs}
\begin{itemize}
    \item Effective representation of queries
    \item Usage of tensor networks in embeddings \cite{yang_embedding_2015} and using complex extensions \cite{trouillon_complex_2017, trouillon_knowledge_2017}
\end{itemize}

\textbf{Tensor Representation of Graphical Models}
\begin{itemize}
    \item Duality of Graphical Models and Tensor Networks:
    \cite{robeva_duality_2019}
    \item Expressivity studies \cite{glasser_expressive_2019}
\end{itemize}

\subsubsection{Infrastructure of AI}

The formalism of tensors and their network decompositions and contractions bears the potential of parallel computations exploited in the AI-dedicated soft and hardware.
\begin{itemize}
    \item Hardware: TPUs beyond GPUs
    \item Software: Tensors as basic data structure in TensorFlow, pyTorch etc., storing neural activations and model weights.
\end{itemize}



\subsection{Structure of the work}

The chapters are structured into three parts, and two focuses, as sketched by:
\newcommand{\horDistChapter}{6}
\newcommand{\verDistChapter}{2.25}
\newcommand{\parBlockDistance}{0.25}
\newcommand{\drawchapter}[4]{
    \coordinate (logRepStart) at (#1, #2);
    \draw (logRepStart) rectangle ($(logRepStart) + (blockDiagonal)$);
    \node[anchor=center] (text) at ($(logRepStart) + (toTop) +0.5*(blockDiagonal)$) {\small #3};
    \node[anchor=center] (text) at ($(logRepStart) + (toBottom) +0.5*(blockDiagonal)$) {\small #4};
}

\begin{tikzpicture}[scale=1]
    \coordinate (blockDiagonal) at (5,1.75);
    \coordinate (toTop) at (0,0.35);
    \coordinate (toBottom) at (0,-0.35);

    % Representation
    \node[anchor=center] (text) at (-0.5, 2*\verDistChapter+0.35) {Focus~I: Representation};
    \draw[dashed]  (-0.25-1*\horDistChapter, 2*\verDistChapter+1) -- (1*\horDistChapter-0.75, 2*\verDistChapter+1) -- (1*\horDistChapter-0.75, -4*\verDistChapter-1)
    -- (-0.25-1*\horDistChapter, -4*\verDistChapter-1) -- (-0.25-1*\horDistChapter, 2*\verDistChapter+1);

    % Reasoning
    \node[anchor=center] (text) at (-0.5+1.5*\horDistChapter, 2*\verDistChapter+0.35) {Focus~II: Reasoning};
    \draw[dashed]  (-0.25+1*\horDistChapter, 2*\verDistChapter+1) -- (2*\horDistChapter-0.75, 2*\verDistChapter+1) -- (2*\horDistChapter-0.75, -4*\verDistChapter-1)
    -- (-0.25+1*\horDistChapter, -4*\verDistChapter-1) -- (-0.25+1*\horDistChapter, 2*\verDistChapter+1);

    % Part I
    \node[anchor=center] (text) at (-0.5-0.5*\horDistChapter, 1*\verDistChapter-0.25) {\parref{par:one}: Factored Systems};
    \draw (-0.5-1*\horDistChapter,-0.25) -- (2*\horDistChapter-0.5, -0.25) -- (2*\horDistChapter-0.5, 2*\verDistChapter-0.25) --
    (-0.5-1*\horDistChapter, 2*\verDistChapter-0.25) -- (-0.5-1*\horDistChapter,-0.25);
    \foreach \x/\y/\key/\name in {
        0/1/cha:probRepresentation/Probability Representation,
        0/0/cha:logicalRepresentation/Logical Representation,
        1/1/cha:probReasoning/Probabilistic Reasoning,
        1/0/cha:logicalReasoning/Logical Reasoning
    } {
        \drawchapter{\x * \horDistChapter}{\y *\verDistChapter}{\charef{\key}}{\name}
    }

    % Part II
    \node[anchor=center] (text) at (-0.5-0.5*\horDistChapter, -1.5*\verDistChapter-0.5) {\parref{par:two}: Neuro-Symbolic AI};
    \draw (-0.5-1*\horDistChapter,-0.5) -- (2*\horDistChapter-0.5, -0.5) -- (2*\horDistChapter-0.5, -2*\verDistChapter-0.5) --
    (-0.5-1*\horDistChapter, -2*\verDistChapter-0.5) -- (-0.5-1*\horDistChapter,-0.5);
    \foreach \x/\y/\key/\name in {
        -1/-1/cha:formulaSelection/Formula Selecting Networks,
        0/-1/cha:networkRepresentation/Logic Network Representation,
        0/-2/cha:folModels/Statistical Models of FOL,
        1/-1/cha:networkReasoning/Logic Network Reasoning,
        1/-2/cha:mlnConcentration/Probabilistic Success Guarantees
    } {
        \drawchapter{\x * \horDistChapter}{\y *\verDistChapter - \parBlockDistance}{\charef{\key}}{\name}
    }

    % Part III
    \node[anchor=center] (text) at (-0.5-0.5*\horDistChapter, -3.5*\verDistChapter-0.5) {\parref{par:three}: Contraction Calculus};
    \draw (-0.5 - 1*\horDistChapter, -2*\verDistChapter-0.75) -- (2*\horDistChapter-0.5, -2*\verDistChapter-0.75) -- (2*\horDistChapter-0.5, -4*\verDistChapter-0.75) --
    (-0.5-1*\horDistChapter, -4*\verDistChapter-0.75) -- (-0.5-1*\horDistChapter,-2*\verDistChapter-0.75);
    \foreach \x/\y/\key/\name in {
        -1/-3/cha:coordinateCalculus/Coordinate Calculus,
        0/-3/cha:basisCalculus/Basis Calculus,
        0/-4/cha:sparseCalculus/Sparse Calculus,
        1/-3/cha:approximation/Approximation,
        1/-4/cha:messagePassing/Message Passing
    } {
        \drawchapter{\x * \horDistChapter}{\y *\verDistChapter - 2 * \parBlockDistance}{\charef{\key}}{\name}
    }

\end{tikzpicture}

\textbf{\parref{par:one}: \partonetext} \\
\ \\
The probabilistic and logical approaches towards artificial intelligence are reviewed in the tensor network formalism. \\

Tensors appear naturally in
\begin{itemize}
    \item Logics: Boolean tensors indicating models (propositional case) and interpretation tensors in first order logics
    \item Probability theory: Truth tables, which are tensors of probabilities for joint distibutions of categorical variables.
\end{itemize}

% Classical usage of tensor network decompositions
Tensor network decompositions as representation schemes appear in
\begin{itemize}
    \item Logics: Conjunctions of formulas are Hadamard products of the tensor representation of formulas (Coordinate Calculus/ Effective Calculus)
    \item Probability theory: Graphical models are tensor networks of the factors. Further sparsity schemes apply, when placing restrictions on the structure of each factor.
    \item Data bases: Relations encoded by lists as storage of nonvanishing coordinates of a relation encoding
\end{itemize}

% Classical usage of tensor network contractions
Tensor network contractions as reasoning schemes appear in
\begin{itemize}
    \item Logics: Model counts, used for satisfiablility decisions and entailment
    \item Probability theory: Marginal probability distributions, extended to conditional probability distributions through normations
\end{itemize}

\ \\
\textbf{\parref{par:two}: \parttwotext} \\
\ \\
Motivated by the classical approaches we apply the tensor network formalism towards learning and infering neuro-symbolic models. \\

\textbf{Neurosymbolic AI}
\begin{itemize}
    \item Required for more advanced AI \cite{hochreiter_toward_2022}
    \item Add the paradigm of neural computing to logical reasoning
    \item Potential benefits from Statistical Relational AI \cite{marra_statistical_2024}
    \item Tensor based approaches \cite{cohen_tensorlog_2020}
    \item \cite{badreddine_logic_2022} representation of logic using tensor networks and automated differentiation to optimize.
\end{itemize}

%\subsubsection{Neuro-Symbolic AI}

\textbf{Tensor Approaches to Neuro-Symbolic AI}
\begin{itemize}
    \item TensorLog \cite{cohen_tensorlog_2020}
    \item \cite{badreddine_logic_2022} representation of logic using tensor networks and automated differentiation to optimize.
\end{itemize}

%% Decomposition of Neural Networks
In Deep Neural Networks, functions between the input layer and the output layer are decomposed into neurons.
Typical neurons are linear transforms with an activation function.

%% Sparsity by fixed architecture
Sparsity means restriction to functions, which are decomposable into a small number of neurons.
Approximations of generic functions (see the universal approximation theorems) would require large amounts of neurons. % CITE!
When restricting to functions based on a fixed architecture, we restrict to a certain set of functions called the inductive bias of the architecture.

\ \\
\textbf{\parref{par:three}: \partthreetext}\\
\ \\
The applied schemes of calculus using tensor network contractions are investigated in more detail.
\ \\
% Representation
\textbf{\focusonespec}\\
\\\
Here we motivate and investigate the efficient representation of tensors based on tensor network decompositions. \\
\ \\
% Reasoning
\textbf{\focustwospec}\\
\ \\
We develop schemes to efficiently perform inductive and deductive reasoning based on information stored in decomposed tensor.


%\subsubsection{\focusonespec}
%
%\subsubsection{\focustwospec}
%
%\subsubsection{\parref{par:one}: \partonetext}
%
%\subsubsection{\parref{par:two}: \parttwotext}
%
%\subsubsection{\parref{par:three}: \partthreetext}
    \chapter{Notation and Basic Concepts}\label{cha:TensorNetworks}\label{cha:notation}

We here provide the fundamental definitions of tensors, which are essentiell for the content in \parref{par:one} and \parref{par:two}.
In \parref{par:three} we will further investigate the properties of tensors focusing on their contractions.

\sect{\bncategoricals}

We will in this work investigate systems, which are described by a set of properties, each called categorical variables.
This is called an ontological commitment, since it defines what properties a system has.

\begin{definition}
	An atomic representation of a system is described by a categorical variables $\catvariable$ taking values $\catindex$ in a finite set
		\[  [\catdim]\coloneqq \{0,\ldots, \catdim-1\} \]
	of cardinality $\catdim$.
\end{definition}

% Notation: Large and small literals
We will in this work always notate categorical variables by large literals and indices by small literals, possible with other letters such as $\catvariable,\selvariable,\indvariable,\datvariable$ and corresponding values $\catindex,\selindex,\indindex,\datindex$.

\begin{definition}
	A factored representation of a system is a set of categorical variables $\catvariableof{\atomenumerator}$, where $\atomenumeratorin$, taking values in $[\catdimof{\atomenumerator}]$.
\end{definition}

\sect{\bntensors}

% Gentle introduction sentences
Tensors are multiway arrays and a generalization of vectors and matrices to higher orders.
We will first provide a formal definition as real maps from index sets enumerating the coordinates of vectors, matrices and larger order tensors.

\begin{definition}[Tensor]\label{def:tensor}
	Let there be numbers $\catdimof{\atomenumerator}\in\nn$ for $\atomenumeratorin$ and categorical variables $\catvariableof{\atomenumerator}$ taking their values in $[\catdimof{\atomenumerator}]$.
	We call maps
	\begin{align*}
		\hypercoreat{\catvariables} : \bigtimes_{\atomenumeratorin} [\catdimof{\atomenumerator}] \rightarrow \rr
	\end{align*}
	tensor of order $\atomorder$ and leg dimensions $\catdimof{0},\ldots,\catdimof{\atomorder-1}$.
	Evaluations of these maps at indices $\catindices$ are denoted by
	\begin{align*}
		\hypercoreat{\indexedcatvariables} = \hypercoreat{\catvariables}(\catindices)\, .
	\end{align*}
%	with coordinates denoted by $\hypercore_{\catindices}$ is called a tensor of order $\atomorder$ and legs with the dimensions $\catdimof{0},\ldots,\catdimof{\atomorder-1}$.
	Tensors $\hypercoreat{\catvariables}$ are elements of the space
	\begin{align*}
		\bigotimes_{\atomenumeratorin} \rr^{\catdimof{\atomenumerator}} \,
	\end{align*}
	which is, with the operations of coordinatewise summation and scalar multiplication, a linear space called a tensor space.
\end{definition}

% Non-canonical 
We here introduced tensors in a non-canonical way based on categorical variables assigned to its axis.
While coming as syntactic sugar at this point, this will allow us to define contractions without further specification of axes, based on comparisons of shared categorical variables.
Especially, this eases the implementation of tensor network contractions without the need to further specify a graph (see Appendix~\ref{cha:implementation}).

% Further abbreviations
We abbreviate lists $\catvariables$ of categorical variables by $\shortcatvariables$, that is denote $\hypercoreat{\catvariables}$ by $\hypercoreat{\shortcatvariables}$.
Occasionally, when the categorical variables of a tensor are clear from the context, we will omit the notation of the variables. %further abbreviate $\hypercoreat{\catvariables}$ by $\hypercore$.

\begin{example}[Trivial Tensor]\label{exa:trivialTensor}
	The trivial tensor is defined as the map
		\[ \onesat{\shortcatvariables} : \facstates \rightarrow \{1\} \subset \rr \]
	with all coordinates being $1$, that is for all $\catindices\in\facstates$
		\[ \onesat{\indexedshortcatvariables} = 1 \, . \]
\end{example}

%\sect{Properties of Tensors}

%% Boolean
We will often encounter situations, where the coordinates of tensors are in $\{0,1\}=[2]$.

\begin{definition}\label{def:booleanTensor}
	We call a tensor $\hypercoreat{\shortcatvariables}$ boolean, when $\imageof{\hypercore}\subset[2]$, i.e. all coordinates are either $0$ or $1$.
\end{definition}

%\sect{One-hot encodings}

We are now ready to provide the link between tensors and states of systems with factored representations.
To this end, we define the one-hot encoding of a state, which is a bijection between the states and the basis elements of a tensor space.

\begin{definition}[One-hot encodings to Atomic Representations]
	Given an atomic system described by the categorical variable $\catvariable$, we define for each $\catindex\in[\catdim]$ the basis vector $\onehotmapofat{\catindex}{\catvariable}$ by
	\begin{align}
		\onehotmapofat{\catindex}{\catvariable=\tilde{\catindex}} = \begin{cases}
			1 & \text{if} \quad \catindex=\tilde{\catindex} \\
			0 & \text{else} \, .
		\end{cases}
	\end{align}
	The one-hot encoding of states $\catindex\in[\catdim]$ of the atomic system described by the categorical variable $\catvariable$ is the map
		\[ \onehotmap: [\catdim] \rightarrow \rr^\catdim \]
	which maps $\catindex \in [\catdim]$ to the basis vectors $\onehotmapofat{\catindex}{\catvariable}$.
\end{definition}

% Coordinatewise representation
The basis vectors $\onehotmapofat{\catindex}{\catvariable}$ are tensors of order $1$ and leg dimension $\catdim$ of the structure
\begin{align}
	\onehotmapofat{\catindex}{\catvariable} = \begin{bmatrix}
	0 & \cdots & 0 & 1 & 0 & \cdots & 0
	\end{bmatrix}^T \, ,
\end{align}
where the $1$ is at the $\catindex$th coordinate of the vector.

% Atomic -> Factored system
We have so far described one-hot representations of the states of a single categorical variable, which would suffice to encode the state of an atomic system.
In a factored system on the other side, we are dealing with multiple categorical variables.

\begin{definition}[One-hot encodings to Factored Representations]\label{def:oneHotEncoding}
	Let there be a factored system defined by a tuple $(\catvariables)$ of variables taking values in $\facstates$.
	The one-hot encoding of its states is the tensor product of the one-hot encoding to each categorical variables, that is the map
		\[ \onehotmap : \facstates \rightarrow  \facspace \]
	defined by mapping $\catindices=\shortcatindices$ to
	\begin{align*}
		 \onehotmapofat{\shortcatindices}{\shortcatvariables}
		=: \bigotimes_{\atomenumeratorin} \onehotmapofat{\catindexof{\atomenumerator}}{\catvariableof{\atomenumerator}} \, .
	\end{align*}
	We will call one-hot representations \emph{tensor representations} and depict them as
	\begin{center}
		\input{OtherContent/tikz_pics/notation/one_hot_tensorproduct.tex}
	\end{center}
\end{definition}

In \charef{cha:coordinateCalculus} we will investigate the image of $\onehotmap$ in more detail and show that it is an orthonormal basis of the tensor space $\facspace$.

\begin{remark}[Flattening of Tensors]
	The use the tensor product to represent states of factored systems can be motivated by the reduction to atomic systems by enumeration of the states.
	We have this property reflected in the state encoding of factored systems, since the tensor space $\facspace$ is isomorphic to the vector spaces $\rr^{\prod_{\atomenumeratorin}\catdimof{\atomenumerator}}$.
	This operation is called flattening (or unfolding) of tensors with many axes to tensors of less axes.
\end{remark}

\sect{\bncontractions}

Contractions are the central manipulation operation on sets of tensors.
To introduce them, we will develop a graphical illustration of sets of tensors, which we also call tensor networks.
In \parref{par:three} we will further investigate the utility of contractions in representing specific calculations, which demand different encoding schemes.


\subsect{Graphical Illustrations}

% Hypergraph as capturing the categorical variable assignment to tensors
Sets of tensor with categorical variables assigned to each legs implicitly carry a notion of a hypergraph.
This perspective is especially useful, when some categorical variables are assigned to axis of multiple tensors, as it will often be the case in the applications considered in this work.
Each variable can then be labeled by a node and each tensor as a hyperedge containing the nodes to its axis variables.
Let us first formally introduce hypergraphs, which are generalizations of graphs allowing edges to be arbitrary nonempty subsets of the nodes, whereas canonical graphs demand a cardinality of two.

\begin{definition}\label{def:hypergraphs}
	A hypergraph is a pair $\graph=(\nodes,\edges)$ of a set of nodes $\nodes$ and a set of edges $\edges$, where each hyperedge $\edge\in\edges$ is a subset of the nodes $\nodes$.
	A directed hypergraph is a pair $\graph=(\nodes,\edges)$, such that each hyperedge $\edge\in\edges$ is the tuple of two disjoint sets $\incomingnodes,\outgoingnodes\subset\nodes$, that is
		\[ \edge = (\incomingnodes,\outgoingnodes)  \, . \]
\end{definition}

% Diagrammatic representation in factor graphs
We will use the standard visualization by factor graphs as a diagrammatic illustration of sets of tensors, where tensors are represented by block nodes and each axis assigned with by a categorical variable $\catvariableof{\atomenumerator}$ represented by a node, see Figure~\ref{fig:tensors}a).
%We further denote on Each axis of the tensor is represented by a node representing the variable $\catvariableof{\atomenumerator}$ and the tensor $\hypercore$ is associated with the hyperedge $\edge$ connecting all variables.
 %representing the choice of an element in the set $[\catdimof{\atomenumerator}]$
% Hyperedge view
Different simplifications of these factor graph depictions have been evolved in different research fields.
In the tradition of graphical models, which started with the work \cite{pearl_probabilistic_1988}, the categorical variables are highlighted and the tensor blocks just depicted by hyperedges.
To depict dependencies with causal interpretations, the edges are further decorated by directions in the depiction of Bayesian networks, see for example \cite{pearl_causality_2009}.

In the tensor network community on the other hand, a simplification scheme highlighting the tensors as blocks and omitting the depiction of categorical variables has been evolved.
The variables, or sometimes their index or dimension, are then directly assigned to the lines depicting the axes of the tensor blocks.
This depiction scheme has been established in the literature as wiring diagrams (see \cite{landsberg_tensors_2011} and dates back at least to the work \cite{penrose_spinors_1987}.

Both depiction schemes are simplifications of factor graphs, by highlighting the categorical variables in the depiction in Figure~\ref{fig:tensors}b) and the tensors in the depiction in Figure~\ref{fig:tensors}c).
We in this work will prefer the simplification of the tensor network community, depicted in Figure~\ref{fig:tensors}b).

% Duality
In another interpretation (see \cite{robeva_duality_2019}), both simplification schemes are different hypergraphs, which are dual to each other.

\begin{figure}[h!]
	\begin{center}
		\input{OtherContent/tikz_pics/notation/hypercore.tex}
	\end{center}
	\caption{Depiction of Tensors
	a) As a factor in a factor graph, depicted by a block, and connected to categorical variables assigned to nodes.
	b) Highlighting only the variable dependencies by a hyperedge connecting the variables $\catvariableof{\atomenumerator}$ to each axis $\atomenumeratorin$.
	c) Highlighting the tensor by a blockwise notation with axes denoted by open legs represented by the variables $\catvariableof{\atomenumerator}$.
	}\label{fig:tensors}
\end{figure}


% Diagramatic representation of vectors
To depict vector calculus and its generalizations, we will apply the graphical notation (mainly version b) introduced in \charef{cha:TensorNetworks}.
Along this line, we represent vectors and their generalization to tensors by blocks with legs representing its indices.
The basis vectors being one-hot encodings of states are in this scheme represented by
	\begin{center}
		\input{OtherContent/tikz_pics/notation/one_hot_atomic.tex}
	\end{center}
where $\tilde{\catindex}$ is an indexed represented by an open leg.
Assigning $\catindex$ to this index will retrieve the $\catindex$th coordinate (with value $1$), whereas all other assignments will retrieve the coordinate values $0$.


Drawing on the interpretation of tensors by hyeredges we can continue with the definition of tensor networks.

\begin{definition}\label{def:tensorNetwork}
	Let $\graph=(\nodes,\edges)$ be a hypergraph with nodes decorated by categorical variables $\catvariableof{\node}$ with dimensions
		\[ \catdimof{\node} \in \nn \]
	and hyperedges $\edge\in\edges$ decorated by core tensors
		\[ \hypercoreofat{\edge}{\catvariableof{\edge}} \in \bigotimes_{\node\in\edge}\rr^{\catdimof{\node}} \, , \]
	where we denote by $\catvariableof{\edge}$ the set of categorical variables $\catvariableof{\node}$ with $\node\in\edge$.
	Then we call the set
		\[ \tnetofat{\graph}{\catvariableof{\nodes}} = \{\hypercoreofat{\edge}{\catvariableof{\edge}}  \, : \, \edge\in\edges\} \]
	the Tensor Network of the decorated hypergraph $\graph$.
\end{definition}


\begin{figure}
	\begin{center}
		\input{OtherContent/tikz_pics/notation/network.tex}
	\end{center}
	\caption{
	Example of a tensor network on a
	a) hypergraph with edges $\edge_0=\{\catvariableof{0},\catvariableof{1},\catvariableof{2}\}$, $\edge_1=\{\catvariableof{1},\catvariableof{2}\}$ and $\edge_2=\{\catvariableof{2},\catvariableof{3}\}$,
	which is decorated by the tensor cores b), representing a contraction with leaving all variables open.
	}\label{fig:network}
\end{figure}

%%
%Diagrammatic notation: Best to do version a) as used in the definition, highlighting that tensors have shared categorical variables with fixed dimensions.




\subsect{Tensor Product}

% Diagrams -> Contractions
Let us now exploit the developed graphical representations to define contractions of tensor networks.
The simplest contraction is the tensor product, which maps a pair of two tensors with distinct variables onto a third tensor and has an interpretation by coordinatewise products.
Such a contraction corresponds with a tensor network of two tensors with disjoint variables, depicted as:
\begin{center}
	\begin{tikzpicture}[scale=0.35,thick] % , baseline = -3.5pt


\begin{scope}[shift={(-15,0)}]



\draw (-1,1) rectangle (10,-1);
\node[anchor=center] (text) at (4.5,0) {\small $\hypercoreof{\edge_0}$};

\draw (0,-1)--(0,-3) node[midway,left] {\tiny $\catvariableof{0}$}; 
\draw (3,-1)--(3,-3) node[midway,left] {\tiny $\catvariableof{1}$}; 
\node[anchor=center] (text) at (3,-4) {$\cdots$};
\draw (9,-1)--(9,-3) node[midway,right] {\tiny $\catvariableof{\atomorder\shortminus1}$}; 

\node [circle, draw, thick, fill=gray!50, minimum size = \nodeminsize] (P1) at (0,-4) {\tiny $\catvariableof{0}$};	
\node [circle, draw, thick, fill=gray!50, minimum size = \nodeminsize] (P2) at (3,-4) {\tiny $\catvariableof{1}$};
\node[anchor=center] (text) at (6,-4) {$\cdots$};

\node [circle, draw, thick, fill=gray!50, minimum size = \nodeminsize] (P3) at (9,-4) {};

\node[anchor=center] (text) at (9,-4) {\tiny $\catvariableof{\atomorder-1}$};



\end{scope}




\draw (-1,1) rectangle (10,-1);
\node[anchor=center] (text) at (4.5,0) {\small $\hypercoreof{\edge_1}$};

\draw (0,-1)--(0,-3) node[midway,left] {\tiny $\seccatvariableof{0}$}; 
\draw (3,-1)--(3,-3) node[midway,left] {\tiny $\seccatvariableof{1}$}; 
\node[anchor=center] (text) at (3,-4) {$\cdots$};
\draw (9,-1)--(9,-3) node[midway,right] {\tiny $\seccatvariableof{\seccatorder\shortminus1}$}; 

\node [circle, draw, thick, fill=gray!50, minimum size = \nodeminsize] (P1) at (0,-4) {\tiny $\seccatvariableof{0}$};	
\node [circle, draw, thick, fill=gray!50, minimum size = \nodeminsize] (P2) at (3,-4) {\tiny $\seccatvariableof{1}$};
\node[anchor=center] (text) at (6,-4) {$\cdots$};

\node [circle, draw, thick, fill=gray!50, minimum size = \nodeminsize] (P3) at (9,-4) {};

\node[anchor=center] (text) at (9,-4) {\tiny $\seccatvariableof{\seccatorder-1}$};




\end{tikzpicture}
\end{center}

\begin{definition}[Tensor Product]\label{def:tensorProduct}
	Let there be two tensor
	\begin{align*}
		\hypercoreofat{\edge_0}{\shortcatvariables} : \facstates \rightarrow \rr \quad \text{and} \quad  \hypercoreofat{\edge_1}{\secshortcatvariables} : \secfacstates \rightarrow \rr \,
	\end{align*}
	with different categorical variables assigned to its axes.
	Then there tensor product is the map
	\begin{align*}
		\contractionof{\hypercoreofat{\edge_0}{\shortcatvariables},\hypercoreofat{\edge_1}{\secshortcatvariables}}{\shortcatvariables,\secshortcatvariables} :  \left(\facstates\right) \times \left(\secfacstates\right) \rightarrow \rr
	\end{align*}
	defined for $\catindices\in\facstates$ and $\seccatindices\in\secfacstates$ as
	\begin{align*}
		& \contractionof{\hypercore,\sechypercore}{\indexedcatvariables,\indexedseccatvariables} \\
		&\quad\quad :=  \hypercoreofat{\edge_0}{\indexedcatvariables}\cdot \hypercoreofat{\edge_1}{\indexedseccatvariables} \, .
	\end{align*}
\end{definition}

% Other notations
Other popular standard notations of tensor products (see \cite{kolda_tensor_2009,hackbusch_tensor_2012,cichocki_tensor_2015})
	\[ \left(\hypercore \otimes \sechypercore\right) = \left(\hypercore \circ \sechypercore\right)
	= \contractionof{\hypercoreofat{\edge_0}{\shortcatvariables},\hypercoreofat{\edge_1}{\secshortcatvariables}}{\shortcatvariables,\secshortcatvariables}  \, . \]
We will avoid these notations in this work in favor of a consistent notation capable of depicting generic tensor network contractions.

When the tensor $\hypercoreofat{\edge_1}{\secshortcatvariables}$ coincides with the trivial tensor $\onesat{\secshortcatvariables}$ (see Example~\ref{exa:trivialTensor}), we further make a notation convention to omit that tensor, that is
\begin{align*}
	\contractionof{\hypercoreofat{\edge_0}{\shortcatvariables},\onesat{\secshortcatvariables}}{\shortcatvariables,\secshortcatvariables}
	= \contractionof{\hypercoreofat{\edge_0}{\shortcatvariables}}{\shortcatvariables,\secshortcatvariables} \, .
\end{align*}


\subsect{Generic Contractions}


Contractions of Tensor Networks $\extnet$ are operations to retrieve single tensors by summing products of tensors in a network over common indices.
We will define contractions formally by specifying just the indices not to be summed over.

When some of the variables are not appearing as leg variables, we define the contraction as being a tensor product with the trivial tensor $\ones$ carrying the legs of the missing variables.

\begin{definition}\label{def:contraction}
	Let $\tnetof{\graph}$ be a tensor network on a decorated hypergraph $\graph=(\nodes,\edges)$.
	For any subset $\secnodes\subset\nodes$ we define the contraction  to be the tensor
	\begin{align}
		\contractionof{\tnetof{\graph}}{\secnodevariables} \in \bigotimes_{\node\in\secnodes} \rr^{\catdimof{\node}}
	\end{align}
	defined coordinatewise by the sum
	\begin{align}
		\contractionof{\tnetof{\graph}}{\indexedcatvariableof{\secnodes}} =
		\sum_{\catindexof{\nodes/\secnodes} \in\,\nodestatesof{\nodes/\secnodes}}
		\left( \prod_{\edge\in\edges}\hypercoreofat{\edge}{\indexedcatvariableof{\edge}} \right) \, .
	\end{align}
	We call $\secnodevariables$ the open variables of the contraction.
\end{definition}

To ease notation, we will often omit the set notation by brackets $\{\cdot\}$ and specify the tensors to be contracted with the delimiter "," (see e.g. \exaref{exa:matrixProduct}).

\begin{figure}
	\begin{center}
		\input{OtherContent/tikz_pics/notation/contraction.tex}
	\end{center}
	\caption{
		Example of a tensor network contraction of all but the variables $\catvariableof{1},\catvariableof{3}$.
		Contraction of variables can always be depicted by closing the open legs with trivial tensors $\ones$ performing index sums.
	}\label{fig:contraction}
\end{figure}

%%
%Diagrammatic notation: Best to do version b), since this is easiest to see how tensors combine to new tensors by contractions.

\begin{remark}[Alternative Notations]
	% Einstein summations
	Contractions can also denoted by the Einstein summations of the indices along connected edges, understood as scalar product in each subspace.
	This is as in \defref{def:contraction}, just omitting the sums.
	We found it useful in this work to do the diagrammatic representation instead, since it offers a better possibility to depict hierarchical arrangements of shared variables.
\end{remark}


% Mode products
Further notations without usage of axis variables are mode products (see \cite{kolda_tensor_2009,hackbusch_tensor_2012,cichocki_tensor_2015}), often denoted by the operation $\times_n$.
With our more generic variable-based notations, we can capture these more specific contractions by coloring the tensor axes, that is assignment of axis variables.

% Examples
To further gain familiarity with the generic contractions, we show the connection to two more popular examples.

%% Diagrammatic representation of Matrix Vector
\begin{example}{Matrix Vector Products}\label{exa:matrixProduct}
	The matrix vector product is a special case of tensor contractions, where a matrix $\matrixat{\exrandom,\secexrandom}$ shares a categorical variable with a vector $\vectorat{\secexrandom}$.
	When leaving the variable unique to the matrix open we get the matrix vector product as
		\[ \contractionof{\matrixat{\exrandom,\secexrandom},\vectorat{\secexrandom}}{\exrandom=\exrandind} = \sum_{\secexrandind\in[\secexranddim]} \matrixat{\exrandom=\exrandind,\secexrandom=\secexrandind} \cdot \vectorat{\secexrandom=\secexrandind} \, .  \]

	Exploiting the diagramatic tensor network visualization we depict matrix vector by: %in \figref{fig:matrixProduct}.
	\begin{center}
		\begin{tikzpicture}[scale=0.3,thick,xscale=-1] % , baseline = -3.5pt

\draw (-9,2)--(-7,2) node[midway,above] {\tiny $\exrandom$};
\draw (-21,1) rectangle (-9,3);
\node[anchor=center] (text) at (-15,2) {\small $\contractionof{\matrixat{\exrandom,\secexrandom},\vectorat{\secexrandom}}{\exrandom}$};

\node[anchor=center] (text) at (-5,2) {\small ${=}$};

\draw (3,2)--(5,2) node[midway,above] {\tiny $\exrandom$};
\draw (1,1) rectangle (3,3);
\node[anchor=center] (text) at (2,2) {\small $\exmatrix$};
\draw (1,2)--(-1,2) node[midway,above] {\tiny $\secexrandom$};
\draw (-1,1) rectangle (-3,3);
\node[anchor=center] (text) at (-2,2) {\small $\exvector$};

%\node[anchor=center] (text) at (7,1) {$\cdot$};


\end{tikzpicture}
	\end{center}
%	Here the index $j$ is represented by a closed edge, which means that it is eliminated by a sum.
\end{example}

%% Hadamard Product 
\begin{example}{Hadamard products of vectors}\label{exa:hadamard}
	A node appearing in arbitrary many hyperedges denotes a Hadamard product of the axis of the respective decorating tensors.
	To give an example, let $\vectorofat{\catenumerator}{\catvariable}\in\rr^\catdim$ be vectors for $\catenumeratorin$. Their hadamard product is the vector
		\[ \contractionof{\{\vectorofat{\catenumerator}{\catvariable} \, : \, \catenumeratorin\}}{\catvariable}  \in \rr^\catdim \]
	defined by
		\[ \contractionof{\{\vectorofat{\catenumerator}{\catvariable} \, : \, \catenumeratorin\}}{\indexedcatvariable}
		= \prod_{\atomenumeratorin} \vectorofat{\atomenumerator}{\indexedcatvariable}\, . \]
	In a contraction diagram the Hadamard product is depicted by: % in \figref{fig:hadamard}.
	\begin{center}
		\begin{tikzpicture}[scale=0.3,thick] % , baseline = -3.5pt


\begin{scope}[shift={(-10,0)}]

\draw (-5,1) rectangle (7,3);
\node[anchor=center] (text) at (1,2) {\small $\contractionof{\{\vectorofat{\catenumerator}{\catvariable} \, : \, \catenumeratorin\}}{\catvariable}$};
\draw (1,-1)--(1,1) node[midway,right] {\tiny $\catleg$};

\node[anchor=center] (text) at (9,2) {${=}$};

\end{scope}



\draw (1,1) rectangle (3,3);
\node[anchor=center] (text) at (2,2) {\small $\vectorof{0}$};
\draw (2,-1)--(2,1) node[midway,right] {\tiny $\catvariable$};


\begin{scope}[shift={(5,0)}]

\draw (1,1) rectangle (3,3);
\node[anchor=center] (text) at (2,2) {\small $\vectorof{1}$};
\draw (2,-1)--(2,1) node[midway,right] {\tiny $\catvariable$};

\end{scope}

\node[anchor=center] (text) at (11.5,2) {\small $\cdots$};


\begin{scope}[shift={(15,0)}]

\draw (0.75,1) rectangle (3.25,3);
\node[anchor=center] (text) at (2,2) {\small $\vectorof{\atomorder\shortminus1}$};
\draw (2,-1)--(2,1) node[midway,right] {\tiny $\catvariable$};

\end{scope}


\draw[fill] (9.125,-4.5) circle (0.25cm);

\draw (9.125,-4.5) to[bend right=-20] (2,-1); 
\draw (9.125,-4.5) to[bend right=-20] (7,-1); 
\draw (9.125,-4.5) to[bend right=20] (17,-1); 

\draw (9.125,-4.5) -- (9.125,-6.5) node[midway,right] {\tiny $\catvariable$};; 

\end{tikzpicture}
	\end{center}
\end{example}



\subsect{Decompositions}

Tensors can be represented by tensor network decompositions, when the contraction of the network retrieves the tensor.

\begin{definition}\label{def:tnDecomposition}
	%Let $\hypercoreat{\nodevaraibles}$ be a tensor in $\extensorspace$.
	A Tensor Network Decomposition of a tensor $\hypercoreat{\nodevariables}$ is a Tensor Network $\tnetof{\graph}$ such that
		\[ \hypercoreat{\nodevariables}= \contractionof{\tnetof{\graph}}{\nodevariables} \, . \]
	We call the hypergraph $\graph$ the format of the decomposition.
\end{definition}





\subsect{Directed Tensors and normalizations}

%% Directionality
Directionality represents constraints on the structure of tensors, namely that the sum over outgoing trivializes the tensor.

\begin{definition}\label{def:directedTensor}
	A Tensor
		\[ \hypercoreat{\nodevariables} \in\bigotimes_{\nodein}\rr^{\catdimof{\node}} \]
	is said to be directed with incoming variables $\innodes$ and outgoing variables $\outnodes$, where $\nodes=\innodes\dot{\cup}\outnodes$, when
		\[ \contractionof{\hypercore}{\catvariablesinset{\outnodes}} =  \onesat{\catvariablesinset{\innodes}} \]
	where $\onesat{\catvariablesinset{\innodes}}$ denoted the trivial tensor in  $\bigotimes_{\node\in\innodes}\rr^{\catdimof{\node}}$ which coordinates are all $1$.
\end{definition}

While by default all legs are outgoing, we can change the direction by normalization.

\begin{definition}\label{def:normalization}
	A tensor $\hypercoreat{\nodevariables}$ is said to be normable on $\innodes\subset\nodes$, if for any $\catindexof{\innodes}\in\nodestatesof{\innodes}$ we have
		\[ \contraction{\hypercoreat{\nodevariables},\onehotmapofat{\atomlegindexof{\innodes}}{\catvariableof{\innodes}}} > 0 \, . \]
	The normalization of a on $\innodes\subset\nodes$ normable tensor is the tensor
	\begin{align*}
		\normalizationofwrt{\hypercoreat{\nodevariables}}{\catvariableof{\outnodes}}{\catvariableof{\innodes}} =
		\sum_{\catindexof{\innodes}\in\nodestatesof{\innodes}}
		\onehotmapofat{\atomlegindexof{\innodes}}{\catvariableof{\innodes}} \otimes \frac{
		\contractionof{\hypercoreat{\nodevariables},\onehotmapofat{\catindexof{\innodes}}{\catvariableof{\innodes}}}{\catvariableof{\outnodes}}
		}{
		\contraction{\hypercoreat{\nodevariables},\onehotmapofat{\catindexof{\innodes}}{\catvariableof{\innodes}}}
		}
	\end{align*}
	where $\outnodes = \nodes/\innodes$.
\end{definition}

We will investigate the contractions of directed tensors in \parref{par:three}, where we show in Theorem~\ref{the:normalizationDirected} that normalizations are directed tensors.


%% Diagrammatic notation
In our graphical tensor notation, we depict directed tensors by directed hyperedges (a), which are decorated by directed tensors (b), for example:
	\begin{center}
		\input{OtherContent/tikz_pics/notation/directed_core.tex}
	\end{center}



\sect{\bnencoding}

Tensors are defined here as real-valued functions on the state set of a system described by categorical variables.
We provide further schemes to represent functions in order to perform sparse calculus and to handle more generic functions.



%
%\subsect{Real-valued functions}
%\begin{example}[Uncertainty about States]\label{exa:onehotUncertainty}
%	The uncertainty about the state of a categorical variable $\catvariable$ can be expressed in vectors.
%	For example let there be real numbers $\probat{\catvariable=\catindex} \in [0,1]$ for $\catindex\in[\catdim]$ with $\sum_{\catindex\in[\catdim]}\probat{\catvariable=\catindex}=1$ with the interpretation that $\probat{\catvariable=\catindex}$ is the probability of a system being in state $\catindex$.
%	We can represent this uncertain state simply by a vector 
%		\[ \probat{\catvariable}\in\rr^{\catdim} \]
%	defined as the sum of one-hot representations weighted by $\probat{\catvariable=\catindex}$
%	\[ \sum_{\catindex\in[\catdim]} \probat{\catvariable=\catindex} \cdot \onehotmapofat{\catindex}{\catvariable} =
%		\begin{bmatrix}
%		\probat{\catvariable=0} & \probat{\catvariable=1} & \cdots & \probat{\catvariable=\catdim-1}
%		\end{bmatrix} \, . 
%	\]
%\end{example}



\subsect{Basis encodings}

%We have already observed in Example~\ref{exa:atomicFunction}, that any function of a categorical variable has a representation as a linear function acting on the one-hot encoding of the variable.
Let us now show how we can encode maps between factored systems.
The scheme is described in more generality and detail (encoding of subsets and relations) in \charef{cha:basisCalculus}, see \defref{def:functionRelationEncoding}.

\begin{definition}[Relation encoding of maps between Factored Systems]\label{def:functionRepresentation}
	Let $\exfunction$ be a function
		\[ \exfunction : \facstates \rightarrow  \secfacstates \]
	which maps the states of a factored system to variables $\catvariables$ to the states of another factored system with variables $\seccatvariables$.
	Then the tensor representation of $\exfunction$ is a tensor
		\[ \bencodingofat{\exformula}{\catvariables,\seccatvariables} \in  \left(\secfacspace\right) \otimes \left(\facspace\right)  \]
	defined by
	\begin{align*}
		& \bencodingofat{\exformula}{\seccatvariables,\catvariables} \\
		& \quad = \sum_{\catindices\in\facstates}
		\onehotmapofat{\exfunction(\catindices)}{\seccatvariables} \otimes  \onehotmapofat{\catindices}{\catvariables} \, .
	\end{align*}
\end{definition}

We depict basis encodings by directed tensors:
\begin{center}
	\input{OtherContent/tikz_pics/notation/bencoding.tex}
\end{center}


% Notation with image categorical variable
%When the categorical variables of the image factored system to a map $\exfunction$ are not specified otherwise, we will denote them by $\catvariableof{\exfunction}$.




\subsect{Tensor-valued functions}


%% TO DETAILLED HERE -> Part III?
\begin{definition}[Selection encoding of Maps between Factored Systems]\label{def:selectionEncoding}
	Given a tensor space $\selspace$ described by categorical variables $\selvariables$ and a tensor-valued function
		\[ \exfunction : \facstates \rightarrow \selspace \]
	the selection encoding of $\exfunction$ is a tensor
		\[ \sencodingofat{\exfunction}{\shortcatvariables,\shortselvariables} \in \left(\facspace\right) \otimes \left(\selspace\right) \]
	defined by the basis decomposition
		\[ \sencodingofat{\exfunction}{\shortcatvariables,\shortselvariables} = \sum_{\shortcatindicesin} \onehotmapofat{\catindices}{\shortcatvariables} \otimes \exfunction(\catindices)[\shortselvariables] \, .  \]
\end{definition}

%%
We call these tensor representation of maps selection encodings, since the coordinate of a function $\exfunction$ to be processed is selected by another argument to $\sencodingof{\exfunction}$.

%\begin{example}[Vector valued functions]\label{exa:atomicFunction} %% CONFUSIN, since already needs selection variables?
%	When using a one-hot representation of the state of a categorical variable, any real valued function has a representation by a real valued matrix acting on the one-hot encoding. 
%	Let there be a vector valued function
%		\[ \exformula : [\catdim] \rightarrow \rr^p \]
%	which maps $\catindex\in[\catdim]$ to the vector
%		\[ \exformula(\catindex)[\selvariable] \in \rr^p \, , \]
%	where we introduced the variable $\selvariable\in[p]$ selecting a coordinate of the image vector.
%	The 
%		\[ \exformula(\catindex)[\selvariable] = 
%		\contractionof{\{\onehotmapof{\catindex}[\catvariable] , \,\concore_{\exformula}[\catvariable,\selvariable]\}}{\selvariable}  \]
%	where $\concore_{\exformula} \in \rr^{\catdim \times p} $ is the matrix defined by the function evaluation vectors of $\exformula$ as
%		\[ \concore_{\exformula}[\catvariable,\selvariable] = \begin{bmatrix}
%			-- & \exformula(0) & -- \\
%			-- & \exformula(1) & -- \\
%			& \vdots &  \\
%			-- & \exformula(\catdim-1) & -- 
%		\end{bmatrix} \, . 
%		\]
%	This can easily be verified, since matrix multiplication with basis vectors amounts to selection of rows (when the basis vector is acting from the left) or columns (when the basis vector is acting from the right).
%	Thus, linear transforms (matrices) acting on the one-hot representation are sufficient to represent any vector valued function of the states of a categorical variable.
%\end{example} 


%% 
We will provide more detail to the tensor representation of functions in \parref{par:three}, where we distinguish between embeddings for basis and coordinate calculus. %where we show that domain encodings coincide with selection encodings.







    \part{\partonetext}\label{par:one}

    The computational automation of reasoning is rooted both in the probabilistic and the logical reasoning tradition.
    Both draw on the same ontological commitment that systems have a factored structure, that is their states are described by assignments to a set of variables.
    Based on this commitment both approaches bear a natural tensor representation of their states and a formalism of the respective reasoning algorithms based on multilinear methods.

    \input{PartI/probability_representation.tex}
    \section{Probabilistic Reasoning}\label{cha:probReasoning} 

We have investigated means to store the knowledge about a system and now turn to the retrieval of information, a process called inference.

% 
Contraction of the relational encoding of a function with a Markov Network gives the statistics over the values of the functions.
When contracting the function directly, we get the expectation.

% Message passing
%Another approximation comes from an approximation of the contractions itself. 
One can increase the efficiency of inference algorithms by using approximative contractions.
Here, message passing schemes can be applied as to be introduced in \charef{cha:messagePassing}.


\subsection{Queries}

% Motivation of queries: Avoid distribution instantiation
In the previous chapter, we have derived efficient representation schemes of probability distributions based on tensor network decompositions.
We have argued that one should avoid naive instantiation of these distributions based on an storage of each coordinates.
In the task of reasoning, we want to retrieve information encoded in the probability distribution.
To derive an efficient approach one therefore needs to avoid instantiating the distribution in a coordiantewise manner in an intermediate step.
We thus formalize a basic reasoning scheme by contractions of the decomposed distributions with query tensors.

\subsubsection{Querying by functions}

We can formalize queries by retrieving expectations of functions given a distribution specified by probability tensors. 
We exploit basis calculus in defining categorical variables $\catvariableof{\exfunction}$ to tensors $\exfunction$, which are enumerating the set $\imageof{\exfunction}$.
More details on this scheme are provided in \charef{cha:basisCalculus}, see \defref{def:functionRelationEncoding} therein.

\begin{definition}\label{def:queries}
	The marginal query of a probability distribution $\probat{\shortcatvariables}$ by a tensor
		\[ \exfunction : \facstates \rightarrow \rr \]
	is the vector $\probat{\catvariableof{\exfunction}} \in \rr^{\cardof{\imageof{\exfunction}}}$ defined as the contraction
	\begin{align*}
		\probat{\catvariableof{\exfunction}} = \contractionof{\probat{\shortcatvariables},\rencodingofat{\exfunction}{\shortcatvariables,\catvariableof{\exfunction}}}{\catvariableof{\exfunction}} \, .
	\end{align*}
	
	% Used in connection to mean parameters
	The expectation query of $\probtensor$ by $\exfunction$ is 
	\begin{align*}
		\expectationof{\exfunction} = \sbcontraction{\exfunction, \probtensor} \, . 
	\end{align*}
	
	% Used for sampling
	Given another tensor $\secexfunction: \facstates \rightarrow \rr $ the conditional query of the probability distribution $\probat{\shortcatvariables}$ by the tensor $\exfunction$ conditioned on the tensor $\secexfunction$ is the matrix $\condprobof{\catvariableof{\exfunction}}{\catvariableof{\secexfunction}}\in\rr^{\cardof{\imageof{\exfunction}}}\otimes \rr^{\cardof{\imageof{\secexfunction}}}$ defined as the normation
	\begin{align*}
		\condprobof{\catvariableof{\exfunction}}{\catvariableof{\secexfunction}} 
		= \normationofwrt{\{
		\probat{\shortcatvariables},\rencodingofat{\exfunction}{\shortcatvariables,\catvariableof{\exfunction}},\rencodingofat{\secexfunction}{\shortcatvariables,\catvariableof{\secexfunction}}
		\}}{
		\catvariableof{\exfunction}}{\catvariableof{\secexfunction}
		} \, . 
	\end{align*}
\end{definition}

%% Relation of queries and expectation queries
Expectation queries are contractions of marginal queries with identities, that is
	\[ \expectationof{\exfunction} = \sbcontraction{\probat{\catvariableof{\exfunction}} \idrestrictedto{\imageof{\exfunction}}{\catvariableof{\exfunction}} } \, . \]
This will be shown in more detail in \charef{cha:basisCalculus} in Corollary~\ref{cor:rhoToNormal}.

%% Conditional Probabilities and conditional queries
Conditional probabilities are queries, where the tensors $\exfunction$ and $\secexfunction$ are identity mappings in the respective variable state spaces.
Conversely, we can understand the conditional query $\condprobof{\exfunction}{\secexfunction}$ as the conditional probability of $\exfunction$ conditioned on $\secexfunction$, of the underlying Markov Network with cores $\{\probtensor, \rencodingof{\exfunction}, \rencodingof{\secexfunction} \}$ and variables $\catvariableof{\exfunction},\catvariableof{\secexfunction}$ besides the variables distributed by $\probtensor$.

%% Expectations as event queries -> Consistency with $\probat{X=i}$?
We further denote event queries by
	\[  \expectationof{\exfunction=z} = \sbcontraction{\probtensor,\rencodingof{\exfunction},\onehotmapof{z}}\]
where by $\onehotmapof{z}$ be denote the one hot encoding of the state $z$ with respect to some enumeration.
Let us note that they are further contraction of the queries in \defref{def:queries} since by \theref{the:splittingContractions}
\begin{align*}
	 \expectationof{\exfunction=z} 
	& =  \sbcontraction{ \sbcontractionof{\probtensor,\rencodingof{\exfunction}}{\catvariableof{\exfunction}} ,\onehotmapof{z}}\\
	& =  \sbcontraction{ \probat{\exfunction} ,\onehotmapof{z}} \, .
\end{align*}

%% OLD: Defining queries by 
%\begin{definition}
%	The expectation of functions $\exfunction$ given a probability tensor is the contraction
%		\[ \expectationofwrt{\exfunction(\catvariables)}{\catvariables\sim\probtensor} = 
%			\contractionof{\{\probtensor,\rencodingof{\exfunction}\}}{\{\exfunctiontargetvariables \}} \, . 
%		\]
%\end{definition}
%This is the canonical definition of expectations, since summing function values weighted by the probability of the argument.
%When we have an unnormalized probability distribution $\phi$ the expectation is the quotient
%\begin{align*}
%	\expectationofwrt{\exfunction(\catvariables)}{\catvariables\sim\phi}  = \frac{
%		\contractionof{\{\phi,\rencodingof{\exfunction}\}}{\{\exfunctiontargetvariables \}}
%	}{
%		\contractionof{\{\phi\}}{\varnothing} 
%	} \, . 
%\end{align*}

%\subsubsection{Conditional Probability Queries}
%
%Typical queries are the computation of an a posteriori distribution given evidence.
%This is just the contraction.
%
%%% As expectation
%The query consists of the one-hot encoding of the evidence and Ids elsewhere.
%The result is then interpreted as another probability distribution, defined as a Markov network and the possible need to normalize with the partition function.
%
%Given evidence, condition the probability tensor on that evidence.






\subsubsection{MAP Queries}

Find the maximal variable of a tensor is a problem, which can be approached by sampling methods as we discuss here.

\begin{definition}
	Given a tensor $\hypercore$ the MAP query is the problem 
	\begin{align}
		\argmax_{\catindices} \hypercoreat{\indexedcatvariables} \, .
	\end{align}
\end{definition}

%Often, the generation of a full (conditioned) probability tensor can be infeasible, if too many variables are queries.
%Having a tensor network decomposition of the probability tensor avoids this generation.

% One hot perspective
By coordinate calculus, we notice that
\begin{align}
	\hypercoreat{\indexedcatvariables} 
	\sbcontraction{\hypercore, \onehotmapof{\catindices}} \, .
\end{align}
Given the image $\Gamma^{\elformat}$ of one-hot encodings, the MAP query problem is equivalent to 
\begin{align}
	\max_{\catindices} \hypercoreat{\indexedcatvariables} 
	= \max_{\theta\in\Gamma^{\elformat}} \sbcontraction{\hypercore, \theta} \, .
\end{align}
We can thus understand MAP queries as a Tensor Network approximation problem, where the approximating tensor are the one-hot encodings of states.

\begin{remark}[MAP queries on energy and probability tensors]
% Usage on energies and probabilities
	Since the exponential function is monotonic, MAP queries on the energy tensor of an exponential family with uniform base measure are equivalent to MAP queries of their energies.
\end{remark}


\subsubsection{Answering queries by energy contractions}

Let us now interpret a probability tensor at hand as a member of an exponential family (see \secref{sec:exponentialFamilies}), which is always possible when taking the naive exponential family.

\begin{lemma}\label{lem:energyContractionQueries} % This is a statement about "full" queries.
	For any probability distribution $\probtensor$ with $\probtensor= \normationof{\expof{\energytensorat{\shortcatvariables}}}{\shortcatvariables}$, disjoint subsets $\nodesa,\nodesb \subset [\catorder]$ with $\nodesa\cup\nodesb=[\catorder]$  and any $\catindexof{\nodesb}$ we have
		\[ \condprobof{\catvariableof{\nodesa}}{\indexedcatvariableof{\nodesb}} 
			= \normationof{
				\expof{\energytensorat{\catvariableof{\nodesa},\indexedcatvariableof{\nodesb}}}
		}{\catvariableof{\nodesa}} \, .\]
\end{lemma}
\begin{proof}
	Since no summation is commuted.
\end{proof}

Thus, it suffices to build the selection encoding of the statistics, and we can avoid the usage of the relational encoding.

% 
We notice, that \lemref{lem:energyContractionQueries} does not generalize to situations, where $\nodesa\cup\nodesb\neq[\catorder]$, since summation over the indices of the variables $[\catorder]/\nodesa\cup\nodesb$ and contraction do not commute.
%\red{In that case, each summed index produces a factor.}


\begin{lemma}  %\red{TRUE?}
	For any probability distribution $\probtensor$ with $\probtensor= \normationof{\expof{\energytensorat{\shortcatvariables}}}{\shortcatvariables}$, disjoint subsets $\nodesa,\nodesb \subset [\catorder]$ and any $\catindexof{\nodesb}$ we have
		\[ \condprobof{\catvariableof{\nodesa}}{\indexedcatvariableof{\nodesb}} 
			=
			\normationof{
			 \sum_{\catindexofin{[\catorder]/\nodesa\cup\nodesb}} 
				 \expof{\energytensorat{\catvariableof{\nodesa},\indexedcatvariableof{\nodesb},\indexedcatvariableof{[\catorder]/\nodesa\cup\nodesb}}}
		}{\catvariableof{\nodesa}} \, .\]
\end{lemma}
\begin{proof}
	By splitting the contraction into terms to $\nodesa\cup\nodesb$. % and using \lemref{lem:energyContractionQueries}.
\end{proof}




\subsection{Sampling based on queries}


Let us here investigate how to draw samples from distributions $\probtensor$, based on queries on $\probtensor$.

%Need to generate the full conditional probability distribution by contraction and then sample from it.
Since there are $\prod_{\node\in\nodes}\catdimof{\node}$ coordinates stored in $\probtensor$, naive methods are often infeasible.
One can instead exploit a representation of $\probtensor$ by a Markov network or the energy term in an exponential family for efficient algorithms and sample from local proxy distributions resulting from contractions and interpreted as marginal and conditional probabilities.

\subsubsection{Exact Methods}

Forward Sampling (see Algorithm~\ref{alg:ForwardSampling}) uses a chain decomposition (see \theref{the:chainRule}) of a probability distribution to iteratively sample the variables.

\begin{algorithm}[hbt!]
\caption{Forward Sampling}\label{alg:ForwardSampling}
\begin{algorithmic}
\For{$\catenumeratorin$}
	\State Draw $\catindexof{\catenumerator}\in[\catdimof{\catenumerator}]$ from the conditional query
		\[ \condprobof{\catvariableof{\catenumerator}}{\indexedcatvariableof{\seccatenumerator} \, : \, \seccatenumerator < \catenumerator} \]
\EndFor
\end{algorithmic}
\end{algorithm}

%
Forward Sampling is especially efficient, when sampling from a Bayesian Network respecting the topological order of its nodes.
The reason for this lies in trivilizations of all conditional distributions, which heads are not included in the evidence of previously sampled variables.
More technically, we can show that
	\[ \condprobof{\catvariableof{\catenumerator}}{\indexedcatvariableof{\seccatenumerator} \, : \, \seccatenumerator < \catenumerator}  
	= \condprobof{\catvariableof{\catenumerator}}{\indexedcatvariableof{\parentsof{\catenumerator}}} \, , \]
which is only involving a single core of a Bayesian network.
\red{This can be shown using Corollary~\ref{cor:onesHead} to be derived in \charef{cha:basisCalculus}.}


%% Comment on rejection Sampling 
%When sampling from conditional probability distributions, one can sample from the conditioned distribution instead.
%However, the conditioning changes the structure of the distribution, and conditioned Bayesian Networks are not Bayesian Networks on the same graph.
%One ways around is rejection sampling, where one samples from the unconditioned distribution and rejects samples not satisfying the event conditioned on.
%When the event conditioned on is of small probability, methods like rejection sampling will come with large runtimes.

\subsubsection{Approximate Methods}

% Problem of many variables
When there are many variables to be sample, the computation of the conditional probability to all variables can be infeasible.
One way to overcome this is Gibbs Sampling: Iteratively resemble single variables given the rest as evidence.

%\subsubsection{Gibbs Sampling}

% Still old: Sample from Marginal
Sample each variable independent from the marginal distribution.
Then, alternate through the variables and sample each variable from the conditional distribution taking the others as evidence.

\begin{algorithm}[hbt!]
\caption{Gibbs Sampling}\label{alg:Gibbs}
\begin{algorithmic}
\For{$\catenumeratorin$}
	\State Draw State for atom $\catenumerator$ from initialization distributions. % In implementation: Initialize with ones and draw -> Avoids zero probability state
\EndFor
\While{Stopping criterion is not met}
\For{$\catenumeratorin$}
	\State Draw $\catindexof{\catenumerator}\in[\catdimof{\catenumerator}]$ from the conditional query
		\[ \condprobof{\catvariableof{\catenumerator}}{\indexedcatvariableof{\seccatenumerator}\, : \seccatenumerator \neq \catenumerator} \]
\EndFor
\EndWhile
\end{algorithmic}
\end{algorithm}


% Energy

Gibbs can be implemented based on the energy tensor $\energytensor$ of the probability tensor, as follows form the \lemref{lem:energyContractionQueries}.



%	\[ \condprobof{\catvariableof{\catenumerator}}{\{\catvariableof{\seccatenumerator}=\catindexof{\seccatenumerator} \, : \seccatenumerator \neq \catenumerator\}} 
%	= \normationofwrt{\expof{\contractionof{\{\energytensor\}\cup\{\onehotmapof{\catindexof{\seccatenumerator}} \, : \seccatenumerator \neq \catenumerator \}}{\catvariableof{\catenumerator}}}}{\catvariableof{\catenumerator}}{\varnothing}  \, .\]
	


\red{This is in contrast with forward sampling, where we need to sum over many coordinates of the exponentiated energy tensor, which amounts to the representation of the probability distribution as a tensor network using relational encodings.
}
%where the operation with energy tensors and selection encodings is not efficient.}




\subsubsection{Simulated Annealing}

\red{MAP queries are approximated by sampling from annealed distributions: Use $\hypercore$ as the energy tensor, e.g. as parameter tensor to the naive exponential family.}

%\begin{remark}\label{rem:simulatedAnnealing}
% Simulated annealing
	\red{Here by the naive exponential family!}
	Simulated annealing manipulates the probability used to sample $\catindexof{\catenumerator}$ in terms of an inverse temperature parameter $\invtemp$, by
		\[ \probtensor \rightarrow \frac{\expof{\invtemp\cdot\lnof{\probtensor}}}{\contraction{\expof{\invtemp\cdot\lnof{\probtensor}}} } \, . \]
	When the temperature is larger than $1$, the probability of states with low probability increases while the probability of states with large probability decreases and for low temperatures the opposite.
	Simulated annealing, that is the decrease of the temperature to $0$ during Gibbs sampling biases the algorithm towards states with large probability.
%	Tuning this parameter can improve the convergence of Gibbs Sampling.

	% On exponential families
	For any exponential family the transformation 
		\[ \energytensor \rightarrow \invtemp \cdot \energytensor  \]
	can be performed by rescaling the canonical parameters as
		\[ \canparam \rightarrow \invtemp \cdot \canparam \, . \]
%\end{remark}







\subsection{Maximum Likelihood Estimation} % Stuff from Parameter Estimation - Problem that Part I is called inference?

Let us now turn to inductive reasoning tasks, where a probabilistic model is trained on given data.

\subsubsection{Likelihood and Loss}

Given a datapoint $\datamapof{\datindex}$ consisting of the images of the data selecting map $\datamap$ (see \defref{def:dataMap}), the likelihood given a Markov Logic Network is denoted as
	\[ \probat{\shortcatvariables = \datamapof{\datindex}} \, . \]
	
% Independent assumption
When all $\datamapof{\datindex}$ are drawn independently from $\probat{\shortcatvariablelist}$, we can factorize into
	\[ \probat{\data}  = \prod_{\datindexin} \probat{\shortcatvariables=\datamapof{\datindex}} \, . \]

% Logarithm
It is convenient to apply a logarithm on the objective, which does not influence the optimum when optimizing this quantity.
This is especially useful, when investigating the convergence of the objective for $\datanum\rightarrow\infty$ (see \charef{cha:mlnConcentration}).

\begin{definition}\label{def:loss}
	We define the loss of a distribution $\probtensor$ as
	\begin{align*}%\label{eq:defLikelihoodLossPL}
		\lossof{\probtensor} 
		= \frac{1}{\datanum} \lnof{\probat{\data}}
	\end{align*}
\end{definition}

We now state the Maximum Likelihood Problem in the form
\begin{align}\tag{$\probtagtypeinst{\loss}{\Gamma,\empdistribution}$}\label{prob:parameterMaxLikelihood}
	\argmin_{\probtensor\in\Gamma} \lossof{\probtensor} \, . % Naive Exponential Family perspective!
\end{align}



\subsubsection{Entropic Interpretation}



\begin{definition}[Shannon entropy]
	The information content or the Shannon entropy of a distribution is defined as
		\[ \sentropyof{\probtensor} 
		:= \expectationofwrt{-\lnof{\probat{\shortcatvariables}}}{\shortcatvariables\sim\probtensor}
		= \sbcontraction{\probtensor,-\lnof{\probtensor}} \, . \]
	%	= - \sum_{\shortcatindices} \probat{\indexedshortcatvariables} \cdot \lnof{\probat{\indexedshortcatvariables}} \, . \]
	We depict this in a tensor network diagram with an ellipsis denoting a coordinatewise transform (see \charef{cha:coordinateCalculus}) with a natural logarithm $\ln$ as:
	\begin{center}
		\begin{tikzpicture}[scale=0.3,thick] % , baseline = -3.5pt

\node[anchor=center] (text) at (-8,-5) {\small $\sentropyof{\probtensor}$};

\node[anchor=center] (text) at (-5,-5) {\small ${=}$};

\node[anchor=center] (text) at (-3,-2) {\small $\mathrm{ln}$};
\draw (2,-2) ellipse (6 and 2.75);

\draw (-1,-1) rectangle (5,-3);
\node[anchor=center] (text) at (2,-2) {\small $\probtensor$};
\draw (-1,-7) rectangle (5,-9);
\node[anchor=center] (text) at (2,-8) {\small $\probtensor$};
\draw (0,-5)--(0,-3); 
\draw (0,-5)--(0,-7) node[midway,left] {\tiny $\atomlegindexof{1}$}; 
\draw (1.5,-5)--(1.5,-3); 
\draw (1.5,-5)--(1.5,-7) node[midway,left] {\tiny $\atomlegindexof{2}$}; 
\node[anchor=center] (text) at (3,-4) {$\cdots$};
\draw (4,-5)--(4,-3);
\node[anchor=center] (text) at (3,-6) {$\cdots$};
\draw (4,-5)--(4,-7) node[midway,right] {\tiny $\atomlegindexof{\atomorder}$}; 

%\drawatomcore{3.5}{-8}{$\probtensor$}
%\drawatomindices{3.5}{-12}	
%\draw (5.5,-9)--(5.5,-7) node[midway,right] {\tiny $\atomlegindexof{\exformula}$};

\end{tikzpicture}
	\end{center}
\end{definition}

\begin{definition}[Cross entropy]\label{def:crossEntropy}
	The cross entropy between two distributions is defined as 
		\[ \centropyof{\probtensor}{\secprobtensor} 
		:=  \expectationofwrt{-\lnof{\secprobtensor[\shortcatvariables]}}{\shortcatvariables\sim\probtensor} 
		= \sbcontraction{\probtensor,-\lnof{\secprobtensor}} \, . \]
		%- \sum_{\catindices}  \probat{\indexedcatvariables} \cdot \lnof{\secprobtensor[\indexedshortcatvariables]}  \, . \]
	We depict this in a tensor network diagram with an ellipsis denoting a coordinatewise transform (here the $\ln$) as :
	\begin{center}
		\input{PartI/tikz_pics/probability_reasoning/cross_entropy.tex}
	\end{center}
\end{definition}

%% Vanishing coordinates case
We here use $\lnof{0}=-\infty$ and $0\cdot \lnof{0} = 0$. 
Then we have $\centropyof{\probtensor}{\secprobtensor} = \infty$ if and only if there is a $\shortcatindices$ such that $\probat{\indexedshortcatvariables}>0$ and $\secprobtensor[\indexedshortcatvariables]=0$.


% KL Divergence
The Gibbs inequality states that
		\[ \centropyof{\probtensor}{\secprobtensor} \geq \sentropyof{\probtensor} \, . \]
The difference between both sides is called the Kullback Leibler Divergence and a useful metric in reasoning, since it vanishes for $\probtensor=\secprobtensor$.

\begin{definition}[Kullback Leibler Divergence]\label{def:KLDivergence}
	The KL divergence between two distributions is defined as 
		\[ \kldivof{\probtensor}{\secprobtensor} = \centropyof{\probtensor}{\secprobtensor} - \sentropyof{\probtensor}  \, . \]
\end{definition}

We are now ready to provide an entropic interpretation of the loss introduced in \defref{def:loss}.

\begin{theorem}\label{the:lossCentropy}
	Given a data selecting map $\datamap$ and a distribution $\probtensor$ we have
	\begin{align}
		\lossof{\probtensor} =  \centropyof{\empdistribution}{\probtensor} \, . % \sbcontraction{\empdistribution,\lnof{\probtensor}} \, . 
	\end{align}
\end{theorem}
\begin{proof}
	We have
	\begin{align*}
		\lossof{\probtensor} 
		& = \frac{1}{\datanum} \lnof{\probat{\data}}
		= \frac{1}{\datanum} \sum_{\datain} \lnof{\probat{\shortcatvariables =\datamap(\datindex)}}
		= \frac{1}{\datanum} \sum_{\datain} \contraction{\{\lnof{\probtensor},\onehotmapof{\datamap(\datindex)}\}} \\ 
		& = \sbcontraction{\empdistribution,\lnof{\probtensor}} \, .
	\end{align*}
	Comparing with the negative log likelihood we notice that that loss coincides with the cross-entropy between the empirical distribution $\empdistribution$ and $\probtensor$, i.e.
		\[ \lossof{\probtensor} = \centropyof{\empdistribution}{\probtensor} \, . \]
\end{proof}


% Interpretation of MLE as Cross-Entropy Minimization

We can therefore rewrite Problem~\ref{prob:parameterMaxLikelihood} as minimization of cross-entropies and of Kullback Leibler divergences as
\begin{align*}
	\argmin_{\probtensor\in\Gamma} \lossof{\probtensor} 
	= \argmin_{\probtensor\in\Gamma} \centropyof{\empdistribution}{\probtensor} 
	= \argmin_{\probtensor\in\Gamma} \kldivof{\empdistribution}{\probtensor} \, .
\end{align*}
	


% M-Projection -> A projection since P^2 = P, i.e. P applied on the image is id
Most general, the Maximum Likelihood Problem is the M-Projection of a distribution $\gendistribution$ onto a set $\Gamma$ of probability tensors is
\begin{align}\tag{$\mathrm{P}_{\Gamma, \gendistribution}$}\label{prob:mProjection}
	\argmax_{\probtensor\in\Gamma} \centropyof{\gendistribution}{\probtensor} 
\end{align}
where the Maximum Likelihood Estimation is the special case $\gendistribution=\empdistribution$.


\begin{example}[Cross entropy with respect to exponential families]\label{exa:cEntropyExp}
	If $\secprobtensor$ from an exponential family with boolean base measure, have with the representation from \lemref{lem:energyCumulantRepresentation}
	\begin{align*}
		\centropyof{\probtensor}{\expdist} 
		= \sbcontraction{\probtensor,\lnof{\expdist}} 
		= \sbcontraction{\probtensor,\sencsstat} - \cumfunctionof{\canparam} + \sbcontraction{\probtensor,\lnof{\basemeasure}} \, . 
	\end{align*}
	For the trivial base measure we can further exploit the existence of the energy tensor and have the representation
		\[ \centropyof{\probtensor}{\expdist} = \sbcontraction{\probtensor,(\expenergy-\cumfunctionof{\canparam}\cdot \ones)}
		=   \sbcontraction{\probtensor,\expenergy} -\cumfunctionof{\canparam} \, .   \]
\end{example}




\subsection{Forward Mapping in Exponential Families} 


%\red{Integrate: 
%Selection encodings suffice for variational methods, relational encodings of statistics are required for markov network instantiations of exponential families.}


%% Mean parameters are expectation queries
Mean parameter coordinates are expectation queries to $\sstatcoordinateof{\selindex}$, by 
	\[ \meanparamat{\indexedselvariable} = \expectationof{\sstatcoordinateof{\selindex}} \, . \]
	
%% Forward mappings are contractions, variational formulation as an alternative to avoid inefficiencies
Forward mappings have a closed form representation by
	\[ \forwardmapof{\canparam}
	= \sbcontractionof{\sencodingof{\sstat},\normationof{\basemeasure,\expof{\contraction{\sencodingof{\sstat},\canparam}}}{\shortcatvariables}}{\selvariable} \, . \]
% Infeasibility and turn to variational alternatives with selection encodings.
This contraction can, however, be infeasible, since it requires the instantiation of the probability tensor, which can be done by basis encodings of the statistic.
We in this section provide alternative characterization of the forward map and approximations of it, which can be computed based on the selection encoding instead.
Following \cite{wainwright_graphical_2008}, we can characterize the forward mapping to exponential families as a variational problem and provide an alternative characterization to this contraction.



\subsubsection{Variational Formulation}

Besides the direct computation of the mean parameter tensor we can give a variational characterization of the forward mapping.
This is especially useful, when the contraction is intractable, for example because the tensor $\expdist$ is infeasible to create.

\begin{theorem}
	We have
	\begin{align*}
		\forwardmapof{\canparam}
		  = \argmax_{\meanparam\in\genmeanset}  \sbcontraction{\meanparam,\canparam} + \sentropyof{\meanrepprob} 
	\end{align*}
	where by $\meanrepprob$ we denote a probability distribution with respect to a base measure $\basemeasure$, which reproduces the mean parameter $\meanparam$.
\end{theorem}
\begin{proof}
	Theorem~3.4 in \cite{wainwright_graphical_2008}.
\end{proof}

Let us now characterize the image of the forward map, which turns out to be the interior of the mean polytope, if the statistic is minimal (see \defref{def:minimalStatistics}).

\begin{theorem}\label{the:meanPolytopeInterior}
	For any statistics $\sstat$, which is minimal with respect to a base measure $\basemeasure$, the image $\imageof{\forwardmap}$ of the forward map is the interior of the convex polytope $\genmeanset$.
\end{theorem}
\begin{proof}
	Theorem 3.3 in \cite{wainwright_graphical_2008}.
\end{proof}

For the practicle usage of this theorem, we need a characterization of the interior of $\genmeanset$.

\begin{theorem}\label{the:meanPolytopeInteriorCharacterization}
	For any minimal statistics $\sstat$ and boolean base measure $\basemeasure$ we have for some $\meanparamat{\selvariable}$ that $\meanparamat{\selvariable}\in\genmeanset$ if and only if there is a positive distribution with respect to $\basemeasure$ such that
		\[ \meanparamat{\selvariable} = \sbcontractionof{\probtensor,\sencsstat}{\selvariable} \, . \]
%	If $\meanparamat{\selvariable}$ is in an minimal exponential family with boolean base measure $\basemeasure$, then it is in the interior of $\genmeanset$ if and only if it is representable by a positive distribution with respect to $\basemeasure$.
\end{theorem}
\begin{proof} 
	\proofrightsymbol: 
		By \theref{the:meanPolytopeInterior} we find a canonical parameter $\canparamat{\selvariable}$ such that
		\begin{align*}
			\meanparamat{\selvariable} = \sbcontractionof{\expdistat{\shortcatvariables},\sencsstatwith}{\selvariable} \, .
		\end{align*}
		We notice, that $\expdist$ is positive with respect to $\basemeasure$, as is any member of an exponential family with base measure $\basemeasure$.
		
	\proofleftsymbol: % Orient on proof of Theorem~3.3 
		Since by assumption the statistics is minimal, the convex set $\genmeanset$ is full dimensional (see e.g. Appendix B in \cite{wainwright_graphical_2008}). 
		We thus use a well-known property for full-dimensional convex sets (see \cite{rockafellar_convex_1997,hiriart-urruty_convex_1993}), that $\meanparam\in\interiorof{\genmeanset}$ if for any non-vanishing vector $\vectorat{\selvariable}$ there is a  % citations from Wainwright - Appendix B	
		there is a $\tilde{\meanparam}[\selvariable]$ with
			\[ \contraction{\vectorat{\selvariable},\meanparamat{\selvariable}} <  \contraction{\vectorat{\selvariable},\tilde{\meanparam}[\selvariable]} \, . \]
		It thus suffices to show for an arbitrary non-vanishing vector $\vectorat{\selvariable}$ the existence of a distribution $\tilde{\probtensor}$, such that
		\begin{align*}
			\contraction{\vectorat{\selvariable},\meanparamat{\selvariable}} < \contraction{\vectorat{\selvariable},\sencsstatwith,\secprobat{\shortcatvariables}} \, .
		\end{align*}
		We define for $\epsilon\in\rr$
		\begin{align*}
			\probofat{\epsilon}{\shortcatvariables} 
			= \sbnormationof{\probat{\shortcatvariables},\expof{\epsilon\cdot\contractionof{\sencsstatwith,\vectorat{\selvariable}}{\shortcatvariables}}}{\shortcatvariables}
		\end{align*}
		The derivation of this map at $\epsilon=0$ is 
		\begin{align*}
			\difwrt{\epsilon}\probofat{\epsilon}{\shortcatvariables}|_{\epsilon=0}
			= \contractionof{\probwith,\sencsstatwith,\vectorat{\selvariable}}{\shortcatvariables} - \contraction{\probwith,\sencsstatwith,\vectorat{\selvariable}} \cdot \probwith 
		\end{align*}
		and thus
		\begin{align*}
			\difwrt{\epsilon} \contraction{\probofat{\epsilon}{\shortcatvariables},\sencsstatwith,\vectorat{\selvariable}}|_{\epsilon=0}
			&= \contractionof{\probwith,(\contractionof{\sencsstatwith,\vectorat{\selvariable}})^2}{\shortcatvariables} \\
			 & \quad - \left(\contractionof{\probwith,\sencsstatwith,\vectorat{\selvariable}}{\shortcatvariables}\right)^2 \, . 
		\end{align*}
		We can interpret this quantity as the variance of the random variable $\contractionof{\sencsstatwith,\vectorat{\selvariable}}{\indexedshortcatvariables}$, where $\shortcatindices$ is drawn from $\probwith$.
		The variance is greater than zero, if this random variable is not constant.
		But from the minimality of $\sstat$ with respect to $\basemeasure$ it follows, that this variable is not constant and we therefore have
		\begin{align*}
			0 < \difwrt{\epsilon} \contraction{\probofat{\epsilon}{\shortcatvariables},\sencsstatwith,\vectorat{\selvariable}}|_{\epsilon=0} \, . 
		\end{align*}
		Thus, there is a $\epsilon>0$ with 
		\begin{align*}
			\contraction{\vectorat{\selvariable},\meanparamat{\selvariable}} < \contraction{\vectorat{\selvariable},\sencsstatwith,\probofat{\epsilon}{\shortcatvariables}} \, .
		\end{align*}

\end{proof}


\subsubsection{Boundary of convex polytopes}

For mean parameters $\meanparamat{\selvariable}$ outside the interior of $\genmeanset$ we know by \theref{the:meanPolytopeInteriorCharacterization}, that any distribution with mean parameter $\meanparamat{\selvariable}$ is not positive with respect to $\basemeasure$ and is therefore not in the exponential family.
We investigate this situation further and provide here a construction scheme to adapt the base measure such that there are exponential families containing these boundary distributions.

\begin{theorem}\label{the:faceToArgmax}
	Let there be a minimal $\sstat$ with respect to the base measure $\basemeasure$ and $\meanparamat{\selvariable}\notin\interiorof{\genmeanset}$.
	Then there is a $\canparamat{\selvariable}$ with 
		\[ \meanparamat{\selvariable} \in \argmax_{\meanparam\in\genmeanset} \contraction{\canparamat{\selvariable},\meanparamat{\selvariable}} \,  \]
	and all distributions with mean parameter $\meanparamat{\selvariable}$ are representable with respect to the base measure
		\[ \secbasemeasureat{\shortcatvariables} = \contractionof{\basemeasure, \indicatorofat{\arbset}{\shortcatvariables}}{\shortcatvariables} \, , \]
	where the indicator is on the set
		\[ \arbset = \argmax_{\shortcatindices} \contraction{\canparam,\sstat(\shortcatindices)}  \, . \]
\end{theorem}
\begin{proof}
	When $\meanparam\notin\interiorof{\genmeanset}$ we find a face such that $\meanparam\in\genfacesetof{\facecondset}$.
	The existence of $\canparamat{\selvariable}$ follows from \theref{the:faceNormal}, in which also a construction procedure is provided given a half-space representation (see \theref{the:meanPolytopeHalfspaces}).
	
	Now, we have 
	\begin{align*}
		 \meanparamat{\selvariable} \in \argmax_{\meanparam\in\genmeanset} \contraction{\canparamat{\selvariable},\meanparamat{\selvariable}} 
	\end{align*}
	and thus 
	\begin{align*}
		 \meanparamat{\selvariable} \in \convhullof{ \sencsstat{\indexedshortcatvariables,\selvariable} \, : \, 
		 \shortcatindices \in \argmax_{\shortcatindices \, : \, \basemeasureat{\indexedshortcatvariables}=1} \contraction{\canparamat{\selvariable},\sencsstat{\indexedshortcatvariables,\selvariable}} }
	\end{align*}	
	Thus, any distribution reproducing meanparam is a convex combination of the one-hot encodings of the states in $\argmax_{\shortcatindices} \contraction{\canparamat{\selvariable},\sencsstat{\indexedshortcatvariables,\selvariable}}$, and therefore representable with respect to the base measure $\secbasemeasure$.
\end{proof}

Each face of $\genmeanset$ thus defines a refinement of a base measure, which is sufficient to reproduce the mean parameters on that face.

\begin{definition}\label{def:faceBaseMeasure}
	The base measure to the face of $\meanset$ with normal $\canparam$ is
		\[ \basemeasureof{\sstat,\canparam} = \indicatorofat{\argmax_{\shortcatindices} \contraction{\canparam,\sstat(\shortcatindices)}}{\shortcatvariables} \, . \]
\end{definition}

\theref{the:faceToArgmax} therefore states, that when a mean parameter is on a face of $\genmeanset$, then each distribution reproducing the mean parameter has a representation with respect to the refined base measure
\begin{align*}
	\secbasemeasureat{\shortcatvariables} = \contractionof{\basemeasure,\basemeasureof{\sstat,\canparam}}{\shortcatvariables} \, . 
\end{align*}

% Base Measure Refinement algorithm
We now utilize these findings and provide in \algoref{alg:baseMeasureRefinement} a procedure to refine the base measure until the reduced mean parameter is in the open set of a reduced mean parameter polytope.

\begin{algorithm}[h!]
\caption{Base Measure Refinement}\label{alg:baseMeasureRefinement}
\begin{algorithmic}
\State \textbf{Input}: Base measure $\basemeasure$, statistic $\sstat$ and mean parameter $\meanparam\in\genmeanset$
\State \textbf{Output}: Refined base measure $\secbasemeasure$, remaining statistic $\secsstat$ and remaining mean parameter $\secmeanparam$
\hrule
%\State \noindent\rule{\linewidth}{0.4pt}
\While{$\meanparam\notin\sbinteriorof{\genmeanset}$}
	\While{$\sstat$ not minimal with respect to $\basemeasure$ (see \defref{def:minimalStatistics})}
		\State Find non-vanishing vector $\vectorat{\selvariable}$ and scalar $\lambda\in\rr$ such that 
			\[ \contractionof{\sencsstatat{\shortcatvariables,\selvariable},\vectorat{\selvariable},\basemeasureat{\shortcatvariables}}{\shortcatvariables} = \lambda\cdot\basemeasureat{\shortcatvariables} \, . \]
		\State Choose a coordinate $\selindexin$ with $\vectorat{\indexedselvariable}\neq0$ and drop it from $\sstat$ and $\meanparam$
	\EndWhile
	\State Find a non-trivial face (i.e. a non-empty face, which is a proper subset of $\genmeanset$) with normal $\canparam$, such that
		\[ \meanparam\in\genfacesetof{\canparam} \]
	\State Refine base measure
		\[ \basemeasure \algdefsymbol \contractionof{\basemeasure,\basemeasureof{\sstat,\canparam}}{\shortcatvariables} \]
\EndWhile
\State \textbf{return} $\basemeasure, \, \sstat,\,\meanparam$
\end{algorithmic}
\end{algorithm}

\begin{theorem}\label{the:baseMeasureRefinement}
	For arbitrary inputs $\basemeasure,\sstat$ and $\meanparam\in\genmeanset$, \algoref{alg:baseMeasureRefinement} terminates in finite time and outputs a triple of base measure $\secbasemeasure$, statistic $\secsstat$ and mean parameter $\secmeanparam$ such that the following holds.
	Any probability tensor $\probtensor$ reproducing $\meanparam$ is representable with respect to $\secbasemeasure$ and $\secmeanparam\in\sbinteriorof{\meansetof{\secsstat,\secbasemeasure}}$.
%	Thus, there is a member of the exponential family $\expfamilyof{\secsstat,\secbasemeasure}$ reproducing $\meanparam$.
\end{theorem}
\begin{proof}
	Let us first show, that \algoref{alg:baseMeasureRefinement} always terminates.
	The inner while loop of \algoref{alg:baseMeasureRefinement} always terminates, since $\sstat$ has a finite number of coordinates, and in each iteration one of the coordinates is dropped.
	To show that the outer while loop also terminates, it suffices to show, that the non-vanishing coordinates of the refined base measure are a proper subset of the base measure before refinement.
	But if this would not be the case, we would have 
		\[ \basemeasureat{\shortcatvariables} = \contractionof{\basemeasure,\basemeasureof{\sstat,\canparam}}{\shortcatvariables} \]
	and thus $\genfacesetof{\canparam}=\genmeanset$, which is a contradiction with the assumption of a non-trivial face.
	
	The second claim follows from an iterative application of \theref{the:faceToArgmax} and the fact, that a probability distribution reproduces $\meanparam$ in a non-minimal representation, if and only if it reproduces the corresponding reduced $\meanparam$ with respect to the reduced statistics.
\end{proof}


\begin{example}[Faces with normals parallel to one-hot encodings]
	To get some intuition how to represent face base measures, let us consider face normals $\canparam\in\{\lambda\cdot\onehotmapofat{\selindex}{\selvariable} \, : \, \selindexin, \, \lambda\in\rr/\{0\}\}$.
	We use relational encodings of the coordinates $\sstatcoordinateof{\selindex}$ of the statistic $\sstat$, with head variables $\catvariableof{\sstatcoordinateof{\selindex}}$ with dimension $\catdimof{\sstatcoordinateof{\selindex}}$ enumerating the image $\imageof{\sstatcoordinateof{\selindex}}\subset\rr$ in an ascending order.
	If $\canparamat{\selvariable}=\lambda\cdot\onehotmapofat{\selindex}{\selvariable}$ with $\lambda>0$, then $\argmax_{\shortcatindices} \contraction{\canparam,\sstat(\shortcatindices)}$ consists of states $\shortcatindices$ with minimal statistic $\sstatcoordinateofat{\selindex}{\indexedshortcatvariables}$, that is
		\[  \basemeasureofat{\sstat,\lambda\cdot\onehotmapof{\selindex}}{\shortcatvariables}
		 = \contractionof{\rencodingofat{\sstatcoordinateof{\selindex}}{\shortcatvariables,\catvariableof{\sstatcoordinateof{\selindex}}},
		 \onehotmapofat{\catdimof{\sstatcoordinateof{\selindex}}-1}{\catvariableof{\sstatcoordinateof{\selindex}}}}{\shortcatvariables}  \, . \]		
	If $\canparamat{\selvariable}=\lambda\cdot\onehotmapofat{\selindex}{\selvariable}$ with $\lambda<0$, then at the states with minimal statistic $\sstatcoordinateofat{\selindex}{\indexedshortcatvariables}$, that is
		\[  \basemeasureofat{\sstat,\lambda\cdot\onehotmapof{\selindex}}{\shortcatvariables}
		 = \contractionof{\rencodingofat{\sstatcoordinateof{\selindex}}{\shortcatvariables,\catvariableof{\sstatcoordinateof{\selindex}}},
		 \onehotmapofat{0}{\catvariableof{\sstatcoordinateof{\selindex}}}}{\shortcatvariables}  \, . \]
\end{example}


% Define sets of realizable distributions
\begin{theorem}
	For the maximal graph $\maxgraph=([\seldim],\{[\seldim]\})$, which has a single hyperedge containing all head variables we have
	\begin{align*}
		\genmeanset = \left\{ \contractionof{\probat{\shortcatvariables},\sencodingofat{\sstat}{\shortcatvariables,\selvariable}}{\shortcatvariables} \, , \, \probtensor \in \realizabledistsof{\sstat,\maxgraph} \right\}
	\end{align*}
\end{theorem}
\begin{proof}
	It is enough show, that for any output tuples $\secbasemeasure$, $\secsstat$ of the Base Measure Refinement \algoref{alg:baseMeasureRefinement} we have
		\[ \expfamilyof{\secbasemeasure,\secsstat} \subset  \realizabledistsof{\sstat,\maxgraph} \, . \]
	We notice, that the normation of any face base measure is realizable by $\realizabledistsof{\sstat,\maxgraph}$, since the objective in the maximation problem in \defref{def:faceBaseMeasure} depends only on $\sstat$.
	Providing a more technical argument, we have
	\begin{align*}
		\indicatorofat{\argmax_{\shortcatindices} \contraction{\canparam,\sstat(\shortcatindices)}}{\shortcatvariables}
		= \contractionof{
			\sstatcc,
			\sum_{\sstat(\shortcatindices) \, : \, \shortcatindices \in \argmax_{\shortcatindices} \contraction{\canparam,\sstat(\shortcatindices)} }
			\onehotmapofat{\indexinterpretationat{\sstat(\shortcatindices)}}{\sstatheadvariables}
		}{\shortcatvariables} \, .
	\end{align*}
	Since during the execution of \algoref{alg:baseMeasureRefinement}, $\secsstat$ is a subset of $\sstat$, we can find a corresponding $\canparamof{i}$ extending the face normal by vanishing coordinates to $\sstat$.
	We then have, that 
	\begin{align*}
		\secbasemeasure = \contractionof{
			\{\rencodingofat{\sstat}{\sstatheadvariables,\shortcatvariables}\} \cup
			\left\{\sum_{\sstat(\shortcatindices) \, : \, \shortcatindices \in \argmax_{\shortcatindices} \contraction{\canparamof{i},\sstat(\shortcatindices)} } 
			\onehotmapofat{\indexinterpretationat{\sstat(\shortcatindices)}}{\sstatheadvariables}
			: i \in [n] \right\}
			}{\sstatheadvariables} 
	\end{align*}
	represents the output base measure, where $i\in[n]$ label the faces chosen during in the loop of \algoref{alg:baseMeasureRefinement}.
	Now, any member $\expdistof{\secbasemeasure,\canparam,\secsstat}\in\expfamilyof{\secsstat,\secbasemeasure}$ can be represented by a member of  $\realizabledistsof{\sstat,\maxgraph}$, by contracting these base measure representing cores with the activation cores $\bigotimes_{\selindexin}\sstatacwith$.
\end{proof}

\subsubsection{Mode Search by annealing}

%% ANNEALING
Finding the mode of a distribution is related to the forward mapping of $\invtemp\cdot\canparam$: 
$\meanparam$ to a delta distribution (or in the convex hull of multiple maxima) in the limit.

% Annealing effect on the optimization problem
This is because 
\begin{align*}
	\argmax_{\meanparam\in\genmeanset}  \sbcontraction{\meanparam,\canparam}
\end{align*}
is taken at an extreme point in $\genmeanset$ (since linear objective over closed convex set), which is a delta distribution of a set and
\begin{align*}
	\argmax_{\meanparam\in\genmeanset}  \sbcontraction{\meanparam,\invtemp\cdot\canparam}+ \sentropyof{\meanrepprob} 
	= 
	\argmax_{\meanparam\in\genmeanset}  \sbcontraction{\meanparam,\canparam} + \frac{1}{\invtemp} \cdot \sentropyof{\meanrepprob} 	
\end{align*}
thus the entropy term is neglectible for large $\invtemp$.
A more precise argument is using a limit of the maxima and can be found in Theorem~8.1 in \cite{wainwright_graphical_2008}





\subsubsection{Mean Field Method}

We rewrite 
\begin{align*}
	\max_{\meanparam\in\genmeanset}  \sbcontraction{\meanparam,\canparam} + \sentropyof{\meanrepprob} 
	=
	\max_{\probtensor} \sbcontraction{\energytensor,\probtensor} + \sentropyof{\probtensor}
\end{align*}
where
	\[ \energytensor = \sbcontractionof{\sencsstat,\canparam}{\shortcatvariables} \, . \]

We now restrict the distributions in the maximum.
Typically we use the family of independent distributions, also called naive mean field method.
The naive mean field is the approximation by distributions of independent random variables $\legcoreof{\catenumerator}$, that is
\begin{align*}
	\argmax_{\legcoreof{\catenumerator} \, : \, \catenumeratorin} \contraction{\{\energytensor\} \cup \{\legcoreof{\catenumerator} \, : \, \catenumeratorin\}}
	+ \sum_{\catenumeratorin} \sentropyof{\legcoreof{\catenumerator}} \, . 
\end{align*}


\begin{theorem}[Update equations for the mean field approximation]
	Keeping all legs but one constant, the problem
	\begin{align*}
		\argmax_{\legcoreof{\catenumerator}} \contraction{\{\energytensor\} \cup \{\legcoreof{\catenumerator} \, : \, \catenumeratorin\}}
		+ \sum_{\catenumeratorin} \sentropyof{\legcoreof{\catenumerator}} 
	\end{align*}
	is solved at 
		\[ \legcoreofat{\catenumerator}{\catvariableof{\catenumerator}} 
			= \normationof{ \expof{ \contractionof{ \{\energytensor[\shortcatvariables] \} \cup
				\{\legcoreofat{\seccatenumerator}{\catvariableof{\seccatenumerator}} \, : \, \seccatenumerator\neq\catenumerator\} }{\shortcatvariables} }
			}{\catvariableof{\catenumerator}} \, . \]
\end{theorem}
\begin{proof}
	We have
	\begin{align*}
		 \difofwrt{\sentropyof{\legcoreof{\catenumerator}}}{\legcoreof{\catenumerator}}
		=  - \lnof{\legcoreofat{\catenumerator}{\catvariableof{\catenumerator}}}
		+ \onesat{\catvariableof{\catenumerator}}
	\end{align*}
	and by multilinearity of tensor contractions
	\begin{align*}
		\difofwrt{\contraction{\{\energytensor\}\cup\{\legcoreof{\seccatenumerator} \, : \, \seccatenumeratorin \}}}{\legcoreof{\catenumerator}}
		=  \contractionof{\{\energytensor\}\cup\{\legcoreof{\seccatenumerator} \, : \, \seccatenumeratorin ,\, \seccatenumerator\neq\catenumerator \}}{\catvariableof{\catenumerator}} \, . 
	\end{align*}
	Combining both, the condition
	\begin{align*}
		0 = \difofwrt{
			\left( \contraction{\{\energytensor\}\cup\{\legcoreof{\seccatenumerator} \, : \, \seccatenumeratorin \}} + \sum_{\catenumeratorin} \sentropyof{\legcoreof{\catenumerator}} \right)
		}{\legcoreof{\catenumerator}}
	\end{align*}
	is equal to
	\begin{align*}
		\lnof{\legcoreofat{\catenumerator}{\catvariableof{\catenumerator}}} =
		 \onesat{\catvariableof{\catenumerator}} + \contractionof{\{\energytensor\}\cup\{\legcoreof{\seccatenumerator} \, : \, \seccatenumeratorin ,\, \seccatenumerator\neq\catenumerator \}}{\catvariableof{\catenumerator}} \, .
	\end{align*}
	Together with the condition $\sbcontractionof{\legcoreof{\catenumerator}}=1$ this is satisfied at
		\[ \legcoreofat{\catenumerator}{\catvariableof{\catenumerator}} 
			= \normationof{ \expof{ \contractionof{ \{\energytensor\} \cup
				\{\legcoreof{\seccatenumerator} \, : \, \seccatenumerator\neq\catenumerator\} }{\catvariableof{\catenumerator}} }
			}{\catvariableof{\catenumerator}} \, . \]
\end{proof}



Algorithm~\ref{alg:NMF} is the alternation of legwise updates until a stopping criterion is met.

\begin{algorithm}[h!]
\caption{Naive Mean Field Approximation}\label{alg:NMF}
\begin{algorithmic}
\For{$\catenumeratorin$}
	\State 
		\[ \legcoreofat{\catenumerator}{\catvariableof{\catenumerator}} 
		\algdefsymbol \normationof{\ones}{\catvariableof{\catenumerator}}  \]
\EndFor
\While{Stopping criterion is not met}
	\For{$\catenumeratorin$}
		\State 
			\[ \legcoreofat{\catenumerator}{\catvariableof{\catenumerator}} 
			\algdefsymbol \normationof{ \expof{ \contractionof{ \{\energytensor[\shortcatvariables] \} \cup
				\{\legcoreofat{\seccatenumerator}{\catvariableof{\seccatenumerator}} \, : \, \seccatenumerator\neq\catenumerator\} }{\catvariableof{\catenumerator}} }
			}{\catvariableof{\catenumerator}} \]
\EndFor
\EndWhile
\end{algorithmic}
\end{algorithm}


\subsubsection{Structured Variational Approximation}

%% Structured Variational approximation
More generically, we restrict the maximum over the mean parameters of efficiently contractable distributions and get a lower bound.
In this section we use any Markov Network as the approximating family. 

Let $\graph$ be any hypergraph, we define the problem
\begin{align}\tag{$\mathrm{P}_{\mnexpfamily, \probtensor}$}\label{prob:structuredApproximation}
	\argmax_{\probtensor\in \mnexpfamily} \sbcontraction{\energytensor,\probtensor} + \sentropyof{\probtensor}
\end{align}

We approximate the solution of this problem again by an alternating algorithm, which iteratively updates the cores of the approximating Markov Network. 

\begin{theorem}[Update equations for the structured variational approximation]\label{the:updateEquationStructuredVariational}
	The Markov Network $\extnet$ with hypercores $\extnetasset$ is a stationary point for Problem~\ref{prob:structuredApproximation}, if for all $\edgein$
	\begin{align*}
	\hypercoreofat{\edge}{\edgevariables}
	= \lambda\cdot \expof{
	\frac{
		\contractionof{\{\energytensor\}\cup\{
		\hypercoreof{\secedge} : \secedge\neq\edge
		\}}{\edgevariables} 
	}{
		\contractionof{\{
		\hypercoreof{\secedge} : \secedge\neq\edge
		\}}{\edgevariables} 
	}
	- \sum_{\thirdedge\neq\edge} 
		\frac{
		\contractionof{\{\lnof{\hypercoreof{\thirdedge}}\}\cup\{
		\hypercoreof{\secedge} : \secedge\neq\thirdedge
		\}}{\edgevariables} 
	}{
		\contractionof{\{
		\hypercoreof{\secedge} : \secedge\neq\thirdedge
		\}}{\edgevariables} 
	}
	}
	\end{align*}
	for any $\lambda>0$ (e.g. by the norm).
	Here, the quotient denotes the coordinatewise quotient.
\end{theorem}
\begin{proof}%[Proof of \theref{the:updateEquationStructuredVariational}]
	We proof the theorem by first order condition on the objective $\objof{\extnet} = \sbcontraction{\energytensor,\extnetdist} + \sentropyof{\extnetdist}$.
	
	To proof the theorem, we use \lemref{lem:difMNExpectation}, which shows a characterization of the derivative of functions
	
	%% Energy Contraction Term
	We have %for $\probtensor\in\mnexpfamily$
	\begin{align*}
		\sbcontraction{\energytensor,\normationof{\extnet}{\shortcatvariables}} 
		=  \frac{
			\contraction{\{\energytensor\}\cup\extnet} 
		}{
			\contraction{\extnet} 			
		} \, . 
	\end{align*}
	
	%% Entropy Term Decomposition
	Further we have
	\begin{align*}
		\sentropyof{\normationof{\extnet}{\shortcatvariables}}
		= \left(\sum_{\secedge\in\edges} \contraction{-\lnof{\hypercoreof{\secedge}},\normationof{\extnet}{\shortcatvariables}} \right)
		+ \lnof{\contraction{\extnet}}	
	\end{align*}
	
	We define the tensor
		\[ \sechypercore[\catvariableof{\nodes}] = \energytensorat{\catvariableof{\nodes}} 
		- \sum_{\secedge\neq\edge} \lnof{\hypercoreofat{\secedge}{\catvariableof{\secedge}}} \otimes \onesat{\catvariableof{\nodes/\secedge}} \]
	and notice, that $\sechypercore$ does not depend on $\hypercoreof{\edge}$.	

	The objective has then a representation as
	\begin{align*}
		\objof{\extnet} = \sbcontraction{\sechypercore[\catvariableof{\nodes}], \extnetdist} - \sbcontraction{ \lnof{\hypercoreof{\edge}}, \extnetdist} +  \lnof{\contraction{\extnet}}
	\end{align*}
	
	Let us now differentiate all terms.
	With \lemref{lem:difMNExpectation} we now get
	\begin{align*}
		\difwrt{\hypercoreofat{\edge}{\seccatvariableof{\edge}}} \sbcontraction{\sechypercore[\catvariableof{\nodes}], \extnetdist}
		& = \sbcontractionof{\sechypercoreat{\nodevariables},
	 	\identityat{\seccatvariableof{\edge},\edgevariables}, 
		\frac{\contractionof{\extnet}{\edgevariables}}{\hypercoreofat{\edge}{\edgevariables}}, 
		\normationofwrt{\extnet}{\catvariableof{\nodes/\edge}}{\edgevariables} }{\seccatvariableof{\edge},\nodevariables} \\
		& \quad -  \contraction{\sechypercoreat{\nodevariables},\extnetdist}
		 \otimes \sbcontractionof{\frac{\contractionof{\extnet}{\seccatvariableof{\edge}}}{\hypercoreofat{\edge}{\seccatvariableof{\edge}}}
		}{\seccatvariableof{\edge}} \, .
	\end{align*}
	
	Further we have
	\begin{align*}
		\difwrt{\hypercoreofat{\edge}{\seccatvariableof{\edge}}} \sbcontraction{ \lnof{\hypercoreof{\edge}}, \extnetdist} 
		& = \sbcontractionof{\lnof{\hypercoreofat{\edge}{\edgevariables}},
	 	\identityat{\seccatvariableof{\edge},\edgevariables}, 
		\frac{\contractionof{\extnet}{\edgevariables}}{\hypercoreofat{\edge}{\edgevariables}}, 
		\normationofwrt{\extnet}{\catvariableof{\nodes/\edge}}{\edgevariables} }{\seccatvariableof{\edge},\nodevariables} \\
		& \quad -  \contraction{\lnof{\hypercoreofat{\edge}{\edgevariables}},\extnetdist}
		 \otimes \sbcontractionof{\frac{\contractionof{\extnet}{\seccatvariableof{\edge}}}{\hypercoreofat{\edge}{\seccatvariableof{\edge}}}
		}{\seccatvariableof{\edge}} \\
		& \quad\quad - \sbcontraction{ \frac{1}{\hypercoreofat{\edge}{\edgevariables}}, \extnetdist}
	\end{align*}
	and (see Proof of \ref{lem:difMNprob})
	\begin{align*}
		\difwrt{\hypercoreofat{\edge}{\seccatvariableof{\edge}}} \lnof{\contraction{\extnet}}
		 = \frac{\difwrt{\hypercoreofat{\edge}{\seccatvariableof{\edge}}} \contraction{\extnet}}{\contraction{\extnet}} 		
		 = \frac{\contractionof{\extnet}{\seccatvariableof{\edge}}}{\hypercoreofat{\edge}{\seccatvariableof{\edge}}} \, .
	\end{align*}
	
	Together, the first order condition
	\begin{align*}
		0 = \difwrt{\hypercoreofat{\edge}{\seccatvariableof{\edge}}} \objof{\extnet}
	\end{align*}
	is equal to all $\seccatindexof{\edge}$ satisfying% here drop seccatvariable to catvariable by slicing 
	\begin{align*}
		0 & = \frac{\contractionof{\extnet}{\indexedseccatvariableof{\edge}}}{\hypercoreofat{\edge}{\indexedseccatvariableof{\edge}}}
		 \Big(
		 	\sbcontraction{\sechypercoreat{\catvariableof{\nodes/\edge},\catvariableof{\edge}=\seccatindexof{\edge}}, \normationofwrt{\extnet}{\catvariableof{\nodes/\edge}}{\catvariableof{\edge}=\seccatindexof{\edge}}} \\
			&\quad \quad - \sbcontraction{\sechypercoreat{\nodevariables}, \extnetdist}  \\
			&\quad \quad - \sbcontraction{\lnof{\hypercoreofat{\edge}{\edgevariables=\seccatindexof{\edge}}}, \normationofwrt{\extnet}{\catvariableof{\nodes/\edge}}{\catvariableof{\edge}=\seccatindexof{\edge}}} \\
			&\quad \quad + \sbcontraction{\lnof{\hypercoreofat{\edge}{\edgevariables}}, \extnetdist} 
		 \Big) \, . 
	\end{align*}
	
	We notice, that by normation
		\[ \sbcontraction{\lnof{\hypercoreofat{\edge}{\edgevariables=\seccatindexof{\edge}}}, \normationofwrt{\extnet}{\catvariableof{\nodes/\edge}}{\catvariableof{\edge}=\seccatindexof{\edge}}} =  \lnof{\hypercoreofat{\edge}{\edgevariables=\seccatindexof{\edge}}} \]
	and that the scalar
		\[ \lambda_1 = \sbcontraction{\sechypercoreat{\nodevariables},\normationof{\extnet}{\catvariableof{\nodes}}}	
		- \sbcontraction{\lnof{\hypercoreofat{\edge}{\edgevariables}},\normationof{\extnet}{\catvariableof{\nodes}}}	\]
	is the constant for all $\seccatindexof{\edge}$.
	
	The first order condition is therefore equal to the existence of a $\lambda_1\in\rr$ such that for all $\seccatindexof{\edge}$ 
	\begin{align*}
		\lnof{\hypercoreofat{\edge}{\catvariableof{\edge}=\seccatindexof{\edge}}}
		= 	\sbcontraction{\sechypercoreat{\catvariableof{\nodes/\edge},\catvariableof{\edge}=\seccatindexof{\edge}}, 
		\normationofwrt{\extnet}{\catvariableof{\nodes/\edge}}{\catvariableof{\edge}=\seccatindexof{\edge}}} + \lambda_1 \, . 
	\end{align*}
	The claim follows when applying the exponential on both sides and with the observation, that 
	\begin{align*}
	\sbcontraction{\sechypercoreat{\catvariableof{\nodes/\edge},\catvariableof{\edge}=\seccatindexof{\edge}}, 
		\normationofwrt{\extnet}{\catvariableof{\nodes/\edge}}{\catvariableof{\edge}=\seccatindexof{\edge}}}
		= 
		\frac{\contractionof{\{\sechypercore\}\cup\{\hypercoreof{\secedge} \, : \, \secedge\neq \edge\}}{\catvariableof{\edge}=\seccatindexof{\edge}} }{
		\contractionof{\{\hypercoreof{\secedge} \, : \, \secedge\neq \edge\}}{\catvariableof{\edge}=\seccatindexof{\edge}} 
		}
	\end{align*}
	and reparametrization of $\lambda_1$ to
		\[ \lambda = \expof{\lambda_1} \, . \]
\end{proof}

%% KL Divergence
The mean field method corresponds with minimization of the KL Divergence to the efficiently contractable family, i.e. the I-projection onto the family.

\begin{theorem}
	For any hypergraph $\graph$ and energy tensor $\energytensor$ we have 
	\begin{align*}
		\argmax_{\probtensor\in \mnexpfamily} \sbcontraction{\energytensor, \probtensor}+ \sentropyof{\probtensor}
		= \argmax_{\probtensor\in \mnexpfamily} \kldivof{\expdistof{(\graph,\canparam)}}{\normationof{\expof{\energytensor}}{\shortcatvariables}}
	\end{align*}
	Problem~\ref{prob:structuredApproximation} is thus the I-projection onto the exponential family $\mnexpfamily$.
\end{theorem}
\begin{proof}
%	This follows from the fact, that the objective is the cross-entropy and the position of the maximum is invariant under substracting $\sentropyof{\probtensor}$.
	By rearranging the objective to the KL divergence.
\end{proof}








\subsection{Backward Mapping in Exponential Families}

%% FROM NETWORK LEARNING
The parameters optimizing the likelihood, will be shown to coincide with the backward mapping evaluated on the expectation of the sufficient statistics (see \theref{the:parEstToBackwardMap}).
This is in most generality true for the parameters of the M-projection of any distribution onto the exponential family.
We therefore investigate methods to compute the backward mapping, in most generality by alternating algorithms and in the special case of Markov Logic Networks by closed form representations.




%\begin{theorem}[Moment Matching Criteria]\label{the:MM}
	We have that $\canparam$ is a solution of the backward problem at $\genmean$, if and only if 
		\[ \sbcontractionof{\expdist,\sencsstat}{\selvariable} = \genmeanat{\selvariable} \, . \]
%\end{theorem}

This contraction equation is called moment matching, since the moment of the empirical distribution is matched by the moment of the fitting distribution.

We find one backward mapping as the dual problem to the forward mapping.


\subsubsection{Variational Formulation}

The backward mapping to $\datameanat{\selvariable} = \sbcontractionof{\empdistribution,\sencsstat}{\selvariable}$ is Maximum Likelihood estimation and the solution of the maximum entropy problem.

\begin{theorem}\label{the:varBackward}
	Let there be a sufficient statistic $\sstat$.
	The map $\backwardmap: \rr^{\seldim}\rightarrow \rr^{\seldim}$ defined as
	\begin{align*}
		\backwardmapof{\meanparam}
		= \argmax_{\canparam\in\rr^{\seldim}}  \sbcontraction{\meanparam,\canparam} - \cumfunctionof{\canparam} \, . 
	\end{align*}
	is a backward mapping.
\end{theorem}
\begin{proof}
	%\red{From duality, see Theorem~3.4 in \cite{wainwright_graphical_2008}.}
	We show the claim can be shown by the first order condition on the objective.	
	It holds that
	\begin{align*}
		\difwrt{\canparamat{\selvariable}}  \cumfunctionof{\canparam}  
		 & = \difwrt{\canparamat{\selvariable}}  \lnof{\contraction{\expof{\contractionof{\sencsstat,\canparam}{\shortcatvariables}}}} \\
		 & = \difwrt{\canparamat{\selvariable}} \frac{\contraction{\sencsstat[\selvariable],\expof{\contractionof{\sencsstat,\canparam}{\shortcatvariables}}}}{\contraction{\expof{\contractionof{\sencsstat,\canparam}{\shortcatvariables}}}}   \\
		 & = \forwardmapof{\canparam}[\selvariable]
	\end{align*}
	and thus
	\begin{align*}
		\difwrt{\canparamat{\selvariable}} \left( \sbcontraction{\meanparam,\canparam} - \cumfunctionof{\canparam}  \right) 
		= \meanparamat{\selvariable} -  \forwardmapof{\canparam}[\selvariable] \, . 
	\end{align*}
	
	The first order condition is therefore 
		\[ \meanparamat{\selvariable} =  \forwardmapof{\canparam}[\selvariable] \]
	and any $\canparam$ satisfies this condition exactly when $\canparam=\backwardmapof{\meanparam}$ for a backward map.
\end{proof}


\subsubsection{Interpretation by Maximum Likelihood Estimation}

% Backward mapping
Backward mapping coincides with the Maximum Likelihood Estimation Problem \eqref{prob:parameterMaxLikelihood}, when we take $\Gamma$ to the distributions in an exponential family $\expfamily$ for a sufficient statistic $\sstat$.

% Cross entropy
The loss is the cross entropy between a distribution with $\meanparam$ and the distribution $\expdistof{(\sstat,\canparam,\basemeasure)}$.


\begin{theorem}
	Let there be any exponential family, a mean parameter vector $\genmean\in\imageof{\forwardmap}$ and a backward map $\backwardmap$.
	Then $\estcanparam=\backwardmapof{\genmean}$ is the parameter of the M-projection (Problem~\ref{prob:mProjection}) of any $\gendistribution$ with $\sbcontractionof{\sencsstat,\gendistribution}{\selvariable}=\genmeanat{\selvariable}$ on to $\expfamily$, that is
		\[ \expdistof{(\sstat,\estcanparam,\basemeasure)} \in \argmax_{\probtensor\in\expfamily} \centropyof{\gendistribution}{\probtensor}  \, . \]
	In particular, if $\meanparam=\datamean$ for a data map $\datamap$, the backward map is a maximum likelihood estimator.
\end{theorem}
\begin{proof}
	We exploit the variational characterization of the backward map by \theref{the:varBackward}, and first show that the objective coincides with the cross entropy between the distribution $\gendistribution$ and the respective member of the exponential family.
	For any $\gendistribution$ and $\canparam$ we have with Example~\ref{exa:cEntropyExp}
	\begin{align*}
		\centropyof{\gendistribution}{\expdistof{(\sstat,\canparam,\basemeasure)}} 
		=   \sbcontraction{\gendistribution,\sencsstat,\canparam} -\cumfunctionof{\canparam} \, .  
	\end{align*}
	We use that by assumption $\sbcontractionof{\gendistribution,\sencsstat}{\selvariable}=\genmeanat{\selvariable}$ and thus
	\begin{align*}
		\centropyof{\gendistribution}{\expdistof{(\sstat,\canparam,\basemeasure)}} 
		=   \sbcontraction{\genmean,\canparam} -\cumfunctionof{\canparam} \, .  
	\end{align*}
	This shows, that the backward map coincides with the M-projection onto $\Gamma=\expfamily$.

	Further, if $\meanparam=\datamean$ for a data map $\datamap$, we have that the corresponding empirical distribution $\empdistribution$ satisfies $\sbcontractionof{\sencsstat,\empdistribution}{\selvariable}=\meanparamat{\selvariable}$.
	The backward map of $\meanparam$ is therefore the M-projection of $\empdistribution$, which is with \theref{the:lossCentropy} the maximum likelihood estimator.
\end{proof}


%\begin{lemma}
%	Let $\sstat\in\facspace\otimes\rr^{\seldim}$ be a sufficient statistic and $\gendistribution\in\facspace$ a probability distribution.
%	For any member $\expdist\in\expfamily$ we have
%		\[ \centropyof{\gendistribution}{\expdist} = \sbcontraction{\canparam,\genmean} - \cumfunctionof{\canparam} \]
%	where 
%		\[ \genmean = \sbcontractionof{\gendistribution,\sencsstat}{\selvariableof{\sstat}} \,  \]
%	and 
%		\[ \cumfunctionof{\canparam} = \lnof{\contraction{\expof{\expenergy}}} \, . \]
%	The M-projection of $\gendistribution$ onto $\expfamily$ is  $\expdistof{(\sstat,\estcanparam,\basemeasure)}$ for
%		\[ \estcanparam\in \argmax_{\canparam}  \sbcontraction{\canparam,\genmean} - \cumfunctionof{\canparam} \, .  \]
%\end{lemma}
%\begin{proof}
%	By decomposing 
%	\begin{align*}
%		\expdist 	& = \normationof{\expof{\sbcontractionof{\sencsstat,\canparam}{\shortcatvariables}}}{\shortcatvariables} \\
%				& = \frac{\expof{\expenergy}}{\sbcontraction{\expof{\expenergy}}}
%	\end{align*}
%	we get
%	\begin{align*}
%		\lnof{\expdist} & = \lnof{\expof{\expenergy}} - \onesat{\shortcatvariables} \cdot \sbcontraction{\expof{\expenergy}} \\ 
%		& = \expenergy - \cumfunction(\canparam) \cdot \onesat{\shortcatvariables}  \, .
%	\end{align*}
%	If follows that
%	\begin{align*}
%		\centropyof{\gendistribution}{\expdist} 
%		&=  \sbcontraction{\gendistribution,\lnof{\expdist}} \\
%		&=  \sbcontraction{\gendistribution,\expenergy} - \cumfunction(\canparam) \cdot \sbcontraction{\gendistribution}   \\
%		&= \sbcontraction{\canparam, \genmean} - \cumfunction(\canparam) \, . 
%	\end{align*}
%\end{proof}




%%\subsection{Maximum Likelihood and Maximum Entropy for Exponential Families}
%
%Parameter Estimation is the M-Projection of a distribution onto the exponential family.
%

%% DONE BEFORE!
%\begin{theorem}[\cite{wainwright_graphical_2008}]\label{the:parEstToBackwardMap}
%	Given any probability distribution $\probat{\shortcatvariables}$ and a exponential family defined by the sufficient statistic $\sstat$, the M-Projection onto the family is the distribution $\probtensorof{(\sstat,\estcanparam,\basemeasure)}$ where
%	\begin{align*}
%		\estcanparam = \backwardmapof{\contractionof{\probtensor,\sencsstat}{\selvariable}} \, .
%	\end{align*}
%\end{theorem}
%\begin{proof}
%	$\contractionof{\probtensor,\sencsstat}{\selvariable}$ is in $\imageof{\forwardmap}$ and MLE has a variational characterization with maximum at the dual $\estcanparam$, see \cite{wainwright_graphical_2008}.
%\end{proof}





\subsubsection{Connection with Maximum Entropy}\label{sec:maxEntDuality}


The Maximum entropy problem with respect to matching expected statistics $\genmean\in\genmeanset$ 
\begin{align}\tag{$\probtagtypeinst{\entropysymbol}{\sstat,\basemeasure,\genmean}$}\label{prob:maxEntropy}
	\argmax_{\probtensor\in\Gamma^{\basemeasure}} \sentropyof{\probtensor} \quad \text{subject to} \quad 
	 \sbcontractionof{\probtensor,\sencsstat}{\selvariable} =  \genmeanat{\selvariable}
\end{align}
where the optimization is over all the distributions $\Gamma^{\basemeasure}$, which are representable with respect to the base measure $\basemeasure$.

\begin{theorem}\label{the:maxEntInterior}
	Let $\sstat$ be a statistic and $\basemeasure$ a base measure.
 	For any $\genmean\in\sbinteriorof{\genmeanset}$ the solution of \probref{prob:maxEntropy} is the distribution $\expdistof{(\secsstat,\estcanparam,\secbasemeasure)}$, where $\estcanparam=\backwardmapwrtof{\secsstat,\secbasemeasure}{\secmeanparam}$.
\end{theorem}
\begin{proof}
	Since $\genmean\in\sbinteriorof{\genmeanset}$, \theref{the:meanPolytopeInteriorCharacterization} implies the existence of $\estcanparam$ such that 
		\[ \genmeanat{\selvariable} = \sbcontractionof{\expdistof{(\sstat,\estcanparam,\basemeasure)},\sencsstat}{\selvariable}   \, . \]
	We now follow the argumentation of the proof of Theorem~20.2 in \cite{koller_probabilistic_2009}.
	Let $\secprobtensor$ further be an arbitrary distribution, possibly different from $\expdistof{(\sstat,\estcanparam,\basemeasure)}$, such that
		\[ \genmeanat{\selvariable} = \sbcontractionof{\secprobtensor,\sencsstat}{\selvariable}  \, . \]
	We then have
	\begin{align*}
		\sentropyof{\expdistof{(\sstat,\estcanparam,\basemeasure)}}
		= \centropyof{\secprobtensor}{\expdistof{(\sstat,\estcanparam,\basemeasure)}}
	\end{align*}
	
	With the Gibbs inequality we have if $\secprobtensor\neq\expdistof{(\sstat,\estcanparam,\basemeasure)}$
	\begin{align*}
		\sentropyof{\expdistof{(\sstat,\estcanparam,\basemeasure)}} - \sentropyof{\secprobtensor}
		= \centropyof{\secprobtensor}{\expdistof{(\sstat,\estcanparam,\basemeasure)}} - \sentropyof{\secprobtensor} > 0 \, . 
	\end{align*}	
	
	Therefore, if $\secprobtensor$ does not coincide with$\expdistof{(\sstat,\estcanparam,\basemeasure)}$, it is not a solution of Problem~\ref{prob:maxEntropy}.
	%Classical result based on duality of maximum entropy and maximum likelihood, shown e.g. in Koller Book.
\end{proof}

% Interpretation
Let us highlight the fact, that in \probref{prob:maxEntropy} we did not restrict to distributions in an exponential family and only demanded representability with respect to the base measure.
When choosing the trivial base measure, this does not pose a restriction on the distributions.
\theref{the:maxEntInterior} states, that when the maximum entropy problem has a solution (i.e. $\genmean\in\genmeanset$), then the solution is in the exponential family to the statistic $\sstat$.

% Generalization
When $\genmean\notin\sbinteriorof{\genmeanset}$, the mean paramater is by \theref{the:meanPolytopeInteriorCharacterization} not reproducable by a member of the exponential family $\expfamilyof{\sstat,\basemeasure}$. 
Instead, in combination with the base measure refinement \algoref{alg:baseMeasureRefinement}, we show that the solution is in a refined exponential family.
% dropping the assumption that the mean parameters are in the interior of the mean parameter polytope.

\begin{theorem}\label{the:maxEntMaxLikeDuality} 
	Let $\sstat$ be a statistic and $\basemeasure$ a base measure.
	For any $\genmean\in\genmeanset$, let $\secsstat,\secbasemeasure$ and $\secmeanparam$ be the outputs of \algoref{alg:baseMeasureRefinement} when passing $\sstat,\basemeasure$ and $\genmean$ as input.
	Then, the distribution $\expdistof{(\secsstat,\estcanparam,\secbasemeasure)}$, where $\estcanparam=\backwardmapwrtof{\secsstat,\secbasemeasure}{\secmeanparam}$, solves \probref{prob:maxEntropy}.
\end{theorem}
\begin{proof}
	\theref{the:baseMeasureRefinement} and the above Lemma.
\end{proof}

% Minimality of the refined base measure
\theref{the:maxEntMaxLikeDuality} further implies, that the base measure $\secbasemeasure$ identified by \algoref{alg:baseMeasureRefinement} is minimal for the maximum entropy problem, in the sense that the solving distribution is positive with respect to it and all feasible distributions have to be representable by it.
This highlights the fact, that the maximum entropy distribution does not vanish beyond those states, which are necessary by \theref{the:baseMeasureRefinement}.


%\begin{theorem}\label{the:maxEntMaxLikeDuality} % In Koller Book, Theorem 20.2
%	If $\genmean\in\imageof{\forwardmap}$, we have that any distribution solving Problem~\ref{prob:maxEntropy} has a representation by $\expdistof{(\sstat,\estcanparam,\basemeasure)}$, 
%	where $\estcanparam=\backwardmapof{\genmean}$ for any backward map of the exponential family. 
%	%where $\estcanparam$ is the Maximum Likelihood Estimate with respect to any $\probtensor$ with $\sbcontractionof{\secprobtensor,\sencsstat}{\selvariable} =\genmean$.
%%
%%	Let $\sstat$ be a map and $\gendistribution$ be any distribution of $\atomstates$ and define
%%		\[ \genmeanat{\selvariable} = \sbcontractionof{\gendistribution,\sencsstat}{\selvariable} \, .  \]
%%	Then the solution of \ref{prob:maxEntropy} coincides with the member $\expdistof{(\sstat,\estcanparam,\basemeasure)}$ of the exponential family $\expfamily$ where
%%		\[ \estcanparam = \backwardmapof{\genmean} \]
%%	for a backward map $\backwardmap$ of $\expfamily$.
%\end{theorem}
%\begin{proof}
%	Since $\genmean\in\imageof{\forwardmap}$, there is a parameter $\estcanparam$ such that 
%		\[ \genmeanat{\selvariable} = \sbcontractionof{\expdistof{(\sstat,\estcanparam,\basemeasure)},\sencsstat}{\selvariable}   \, . \]
%	Let $\secprobtensor$ further be an arbitrary distribution such that
%		\[ \genmeanat{\selvariable} = \sbcontractionof{\secprobtensor,\sencsstat}{\selvariable}  \, . \]
%	We then have
%	\begin{align*}
%		\sentropyof{\expdistof{(\sstat,\estcanparam,\basemeasure)}}
%		= \centropyof{\secprobtensor}{\expdistof{(\sstat,\estcanparam,\basemeasure)}}
%	\end{align*}
%	
%	With the Gibbs inequality we have if $\secprobtensor\neq\expdistof{(\sstat,\estcanparam,\basemeasure)}$
%	\begin{align*}
%		\sentropyof{\expdistof{(\sstat,\estcanparam,\basemeasure)}} - \sentropyof{\secprobtensor}
%		= \centropyof{\secprobtensor}{\expdistof{(\sstat,\estcanparam,\basemeasure)}} - \sentropyof{\secprobtensor} > 0 \, . 
%	\end{align*}	
%	
%	Therefore, if $\secprobtensor$ does not coincide with$\expdistof{(\sstat,\estcanparam,\basemeasure)}$, it is not a solution of Problem~\ref{prob:maxEntropy}.
%	%Classical result based on duality of maximum entropy and maximum likelihood, shown e.g. in Koller Book.
%\end{proof}




\subsubsection{Alternating Algorithms to Approximate the Backward Map}\label{sec:alternatingBackwardMap}


\red{While the forward map always has a representation in closed form by contraction of the probability tensor, the backward map in general fails to have a closed form representation.
Computation of the Backward map can instead be performed by alternating algorithms, as we show here.} % Are these fixpoint iterations?


Alternate through the coordinates of the statistics and adjust $\canparamat{\indexedselvariable}$ to a minimum of the likelihood, i.e. where for any $\selindexin$
\begin{align*}
	0 = \frac{\partial}{\partial \canparamat{\indexedselvariable}} \lossof{\expdist} \, . 
\end{align*}

% Moment matching
This condition is equal to the collection of moment matching equations % (see \theref{the:mm})
\begin{align*}
	\sbcontractionof{\expdist,\sencsstat}{\indexedselvariable} = \sbcontraction{\empdistribution,\sencsstat}{\indexedselvariable} \, . 
\end{align*}


\begin{lemma}\label{lem:mmContractionEquation}
	For any sufficient statistic $\sstat$ a parameter vector $\canparam$ and a $\selindexin$ we define
	\begin{align*}
	 	\hypercoreat{\catvariableof{\sstatcoordinateof{\selindex}}} 
		= \contractionof{\{\sstatcc\}\cup\{\headcoreof{\tilde{\selindex}} : \tilde{\selindex} \in [\seldim], \tilde{\selindex}\neq\selindex\}}{\catvariableof{\sstatcoordinateof{\selindex}}} \, . 
	\end{align*}
	Then the moment matching condition for $\sstatcoordinateof{\selindex}$ relative to $\canparam$ and $\meanparam$ is satisfied for any $\canparamat{\indexedselvariable}$ with
	\begin{align*}
		\sbcontraction{\headcoreof{\selindex}, \idrestrictedto{\imageof{\sstatcoordinateof{\selindex}}}, \hypercoreat{\selvariable_\sstat}}
		= \sbcontraction{\headcoreof{\selindex}, \hypercoreat{\selvariable_\sstat}} \cdot \meanparamat{\indexedselvariable} \, . 
	\end{align*}
\end{lemma}
\begin{proof}
	We have
	\begin{align*}
		\expdist = \frac{
			\sbcontractionof{\headcoreof{\selindex}, \hypercore}{\shortcatvariables}
		}{
			\sbcontraction{\headcoreof{\selindex}, \hypercore}
		}
	\end{align*}
	and 
	\begin{align*}
		\sbcontraction{\expdist, \sstatcoordinateof{\selindex}}
		= \frac{
			\sbcontractionof{\headcoreof{\selindex}, \idrestrictedto{\imageof{\sstatcoordinateof{\selindex}}}, \hypercore}{\shortcatvariables}
		}{
			\sbcontraction{\headcoreof{\selindex}, \hypercore}
		} \, . 
	\end{align*}
	Here we used
		\[ \sstatcoordinateof{\selindex} = \sbcontractionof{\headcoreof{\selindex}, \idrestrictedto{\imageof{\sstatcoordinateof{\selindex}}}}{\shortcatvariables} \]
	and redundancies of copies of relational encodings.
	It follows that 
	\begin{align*}
		\sbcontraction{\expdist,\sstatcoordinateof{\selindex}} = \contraction{\empdistribution,\sstatcoordinateof{\selindex}}
	\end{align*}
	is equal to
	\begin{align*}
		\sbcontraction{\headcoreof{\selindex}, \idrestrictedto{\imageof{\sstatcoordinateof{\selindex}}}, \hypercoreat{\catvariableof{\sstatcoordinateof{\selindex}}}}
		= \sbcontraction{\headcoreof{\selindex},\hypercoreat{\catvariableof{\sstatcoordinateof{\selindex}}}} \cdot \meanparamat{\indexedselvariable} \, . 
	\end{align*}	
\end{proof}

% Alternation necessary
The steps have to be alternated until sufficient convergence, since matching the moment to $\selindex$ by modifying $\canparamat{\indexedselvariable}$ will in general change other moments, which will have to be refit.


%Coordinate descent
An alternating optimization is the coordinate descent of the negative likelihood, seen as a function of the coordinates of $\canparam$, see Algorithm~\ref{alg:AMM}.
Since the log likelihood is concave, the algorithm converges to a global minimum.



\begin{algorithm}[h!]
\caption{Alternating Moment Matching}\label{alg:AMM}
\begin{algorithmic}
\State Set $\canparamat{\selvariable}=0$
\State Compute $\datameanat{\selvariable}= \sbcontractionof{\empdistribution,\sencsstat}{\selvariable}$
%\For{$\selindexin$}
%	\State Set $\canparamat{\indexedselvariable}=0$ 
%	\State Compute $\meanparamat{\indexedselvariable}^{\datamap} = \contractionof{\{\empdistribution,\sstatcoordinateof{\selindex}\}}{\varnothing} $ % Or give those as input!
%\EndFor
\While{Stopping criterion is not met}
\For{$\selindexin$}
	\State Compute 
		\begin{align*}
			\hypercoreofat{\selindex}{\catvariableof{\sstatcoordinateof{\selindex}}} 
			\algdefsymbol \contractionof{\{\sstatcc\}\cup\{\headcoreof{\tilde{\selindex}} : \tilde{\selindex} \in [\seldim], \tilde{\selindex}\neq\selindex\}}{\catvariableof{\sstatcoordinateof{\selindex}}} 
		\end{align*}
	\State Set $\canparamat{\indexedselvariable}$ to a solution of 
	\begin{align*}
		\sbcontraction{\headcoreof{\selindex},\idrestrictedto{\imageof{\sstatcoordinateof{\selindex}}},\hypercoreof{\selindex}}
		\algdefsymbol \sbcontraction{\headcoreof{\selindex},\hypercoreof{\selindex}} \cdot \datameanat{\indexedselvariable} \, . 
	\end{align*}
\EndFor
\EndWhile
\end{algorithmic}
\end{algorithm}


% 
In general, if $\imageof{\sstatcoordinateof{\selindex}}$ contains more than two elements, there exists no closed form solutions.
We will investigate the case of binary images, where there are closed form expressions, later in \secref{sec:alternatingParEstMLN}.


%
The computation of $\hypercoreof{\selindex}$ in Algorithm~\ref{alg:AMM} can be intractable and be replaced by an approximative procedure based on message passing schemes.

\subsection{Discussion}

% Forward mapping as gradient of A
Further in \cite{wainwright_graphical_2008}: Convex Duality.
Forward mapping coincides with gradient, i.e. $\meanparam = \nabla \cumfunction(\canparam)$.

% Gradient property of the backward mapping
In \cite{wainwright_graphical_2008}:
The objective is the conjugate dual $\dualcumfunction$ of $\cumfunction$, and backward mapping has an expression by the gradient, i.e. $\canparam = \nabla \dualcumfunction(\meanparam)$.




    \chapter{\chatextlogicalRepresentation}\label{cha:logicalRepresentation}

Propositional logics describes systems with $\atomorder$ boolean variables, which are called atoms and denoted by $\atomicformulaof{\atomenumerator}$ for $\atomenumeratorin$.
Indices $\catindexof{\atomenumerator}\in[2]$ to the atoms $\atomenumeratorin$ enumerate the $2^\atomorder$ states of these systems, which are called worlds.
In each world indexed by $\shortcatindices=\catindices$ the indices $\atomicformulaof{\atomenumerator}$ encode whether the corresponding variable is $\truesymbol$. 

% Propositional logics
The epistemological commitments of propositional logics are whether the state is $\truesymbol$ or $\falsesymbol$ reflected by the coordinate of the one-hot encoding being $1$ or $0$.
Intuitively this describes, whether a specific world can be the state of a factored system.
Propositional logic amounts to reason about boolean variables, which are categorical variables with $2$ possible values.
Such boolean tensors have already appeared as base measures in the representation of probability distributions in \charef{cha:probRepresentation}.

Before discussing the semantics and syntax of propositional formulas, we first investigate how Boolean can be represented by vectors in order to mechanize their processing based on contractions.



\sect{Encoding of Booleans}\label{sec:booleanEncoding}

Booleans are variables valued by $\truthset$ and consist a basic data structure.

\subsect{Representation by coordinates}

To represent Booleans by categorical variables $\catvariable$ with two states we use the index interpretation function
\begin{align*}
	\indexinterpretation:[2]\rightarrow\truthset%\rightarrow\ozset
\end{align*}
defined as
\begin{align*}
	\indexinterpretationat{1} = \truesymbol%} = 1
	\quad \text{and} \quad \indexinterpretationat{0} = \falsesymbol \, .
\end{align*}

% Outlook: General interpretation maps
In \defref{def:subsetEncoding} in \parref{par:three} will define encodings of arbitrary sets based on index interpretation maps.

% Group homomorphism
One motivation for this particular choice of the interpretation function $\indexinterpretation$ is the effective execution of the conjunction as we show in the next Lemma.

\begin{lemma}
	$\indexinterpretation$ is a homomorphism between the groups
	\begin{align*}
		\big(\ozset,\cdot\big)  \quad \text{and} \quad \big(\truthset,\land\big) \, .
	\end{align*}
\end{lemma}
\begin{proof}
	It suffices to notice, that for arbitrary $\truthstateof{0},\truthstateof{1}\in\ozset$ we have
	\begin{align*}
		\indexinterpretationat{\truthstateof{0} \cdot \truthstateof{1}}
		= \indexinterpretationat{\truthstateof{0}} \land \indexinterpretationat{\truthstateof{1}}  \, .
	\end{align*}
\end{proof}
	
% Interpretation of boolean contraction and type conversion application
Based on this homomorphism, contractions of boolean tensors, in which all variables are kept open, can be regarded as parallel calculations of the conjunction $\land$ encoded by $\indexinterpretation$.
This homomorphism is further applied in type conversion in dynamically-typed languages (e.g. in $\mathrm{python}$ \cite{python_software_foundation_python_2025}).

% Nonlinearity issues
Operations like the negation fail to be linear and are only affine linear, since for $\truthstate\in\truthset$ we have
\begin{align}\label{eq:affineLinearNegation}
 	\invindexinterpretationat{\lnot\truthstate} = 1 - \invindexinterpretationat{\truthstate}  \, .
\end{align}
Since any logical connective can be represented as a composition of conjunctions and negations, any logical connective corresponds with an affine linear function on the interpreted truth values.
Direct applications of this insight to execute logical calculus will be discussed later in \secref{sec:effectiveCalculus}.
For our purposes here, we would like to execute logical connective based on single contractions and avoid summations over them.
This is why we call the negation representation as in \eqref{eq:affineLinearNegation} the affine representation problem, which we in the following want to resolve.

% Disjunction central interpretation
While in this work, we will always encode boolean states by $\indexinterpretation$, other index interpretation functions could be chosen.
For example, the interpretation
	\[ \indexinterpretationof{\lor}:\ozset\rightarrow\truthset \]
defined as
    	\[ \indexinterpretationofat{\lor}{0} = \truesymbol \quad \text{and} \quad \indexinterpretationofat{\lor}{1} = \falsesymbol \, , \]
results is a homomorphism between the groups	
	\[ \big(\ozset,\cdot\big) \quad \text{and} \quad \big(\truthset,\lor\big)  \, . \]
While placing the disjunction $\lor$ as the logical connective effectively executed by contractions, the negation will for arbitrary interpretations mapping onto $\ozset$ remain the function %\eqref{eq:affineLinearNegation}
\begin{align*}
 	\invindexinterpretationofat{\lor}{\lnot\truthstate} = 1 - \invindexinterpretationofat{\lor}{\truthstate}  \, .
\end{align*}
Thus, the problem of affine linear operations cannot be resolved by a clever choice of an interpretation function with image in $\ozset$. 


\subsect{Representation by basis vectors} % This is what Basis Calculus does! Refer to that here?

While contractions can just perform conjunctions, we need a representation trick to extend the contraction expressivity to arbitrary connectives and resolve the affine representation problem.
To this end we now compose $\indexinterpretation$ with the one-hot encoding $\onehotmap$ and get an encoding
\begin{align*}
	\onehotmap\circ\invindexinterpretation : \truthset \rightarrow \ozbasisset \, ,
\end{align*}
where $\catvariable$ is a categorical variable with $\catdim=2$.
For any $\truthstate\in\truthset$ we have
\begin{align*}
	\onehotmap\circ\invindexinterpretationat{\truthstate} =
	\begin{bmatrix}
		\invindexinterpretationat{\lnot\truthstate} \\
		\invindexinterpretationat{\truthstate}
	\end{bmatrix}  \, .
\end{align*}


% Resolving the affine representation problem
Performing the negation now amounts to switching the coordinates of the encoded vector, which can be performed by contraction with a transposition matrix
\begin{align*}
	\rencodingofat{\lnot}{\headvariableof{\lnot},\catvariable} = 
	\begin{bmatrix}
		0 & 1 \\
		1 & 0
	\end{bmatrix} \, ,
\end{align*}
where in this notation we always understand the first variable $\catvariable$ as the row index selector and the second variable $\headvariableof{\notucon}$ as the column index selector.
We then have
\begin{align*}
	\onehotmap\circ\invindexinterpretationat{\lnot\truthstate}[\headvariableof{\lnot}]
	= \contractionof{\rencodingofat{\notucon}{\headvariableof{\notucon},\catvariable},\onehotmap\circ\invindexinterpretationat{\truthstate}[\catvariable]}{\headvariableof{\notucon}} \, .
\end{align*}
We therefore arrived at our aim to resolve the affine representation problem and have found a procedure to represent logical negations by a contraction, which is a linear operation.
Besides negations, we will show in this chapter, that arbitrary logical formulas can be represented by contractions.

\subsect{Coordinate and Basis Calculus}

Our findings on the encoding of booleans hint towards more general schemes to encode information into boolean tensors, which will be explored in more detail in \charef{cha:coordinateCalculus} and \charef{cha:basisCalculus}.
When each coordinate in a boolean tensor represents one in $\ozset$ interpreted boolean we call the scheme coordinate calculus.
In basis calculus on the other hand, booleans are represented by elements of $\ozbasisset$.
In that scheme, there are pairs of two coordinates (building slice vectors of the tensors), which are restricted to be different from each other.
This amounts to posing a global directionality constraint on the boolean tensor, as will be shown in \theref{the:rencodingDirected}.

\sect{Semantics of Propositional Formulas}

% Structure
We now choose a semantic centric approach to propositional logic, by defining formulas as boolean tensors.
Then we investigate the corresponding syntax of formulas as specification of a tensor network decomposition of the relational encoding of formulas.

\subsect{Formulas}

% Intro of formulas
Logics is especially useful in interpreting boolean tensors representing Propositional Knowledge Bases, based on connections with abstract human thinking.
To make this more precise, we associate each such tensor is associated with a formula $\exformula$ being a composition of the atomic variables with logical connectives as we proof next.

\begin{definition}\label{def:formulas}
	A propositional formula $\formulaat{\shortcatvariables}$ depending on $\atomorder$ atoms $\catvariableof{\atomenumerator}$ is a boolean-valued tensor
	\begin{align*}
		\formulaat{\shortcatvariables} : \atomstates \rightarrow \ozset \subset \rr \, .
	\end{align*}
	We call a state $\shortcatindices \in \atomstates$ a model of a propositional formula $\formula$, if
	\begin{align*}
		\formulaat{\indexedshortcatvariables}=1 \, .
	\end{align*}
	If there is a model to a propositional formula, we say the formula is satisfiable.
\end{definition}

% Boolean Tensors
The propositional formulas coincide therefore with the boolean tensors (see \defref{def:booleanTensor}).


% Decomposition into model sums
Since propositional formulas are binary valued tensors, the generic decomposition of \lemref{lem:tensorBasisDecomposition} simplifies to
\begin{align}\label{eq:formulaModelDecomposition}
	\formulaat{\shortcatvariables} = \sum_{\catindices\in\atomstates} \formulaat{\indexedcatvariables} \cdot \onehotmapofat{\shortcatindices}{\shortcatvariables} \\
	= \sum_{\catindices\in\atomstates \, : \, \formulaat{\indexedcatvariables}=1}  \onehotmapofat{\catindices}{\shortcatvariables} \, .
\end{align}
Thus, any propositional formula is the sum over the one-hot encodings of its models.
This is equal to the encoding of the set of models, which will be introduced in \charef{cha:basisCalculus} (see \defref{def:subsetEncoding}).

We depict this decomposition in the diagrammatic notation by
\begin{center}
	\begin{tikzpicture}[scale=0.35, thick] % , baseline = -3.5pt

%\draw[->-] (2,-1)--(2,1) node[midway,right] {\tiny $\catvariableof{\exformula}$};
\draw (-1,-1) rectangle (5,-3);
\node[anchor=center] (text) at (2,-2) {\small ${\exformula}$};
\draw[] (0,-3)--(0,-5) node[midway,left] {\tiny $\randomxof{0}$}; 
\draw[] (1.5,-3)--(1.5,-5) node[midway,left] {\tiny $\randomxof{1}$}; 
\node[anchor=center] (text) at (3,-4) {$\cdots$};
\draw[] (4,-3)--(4,-5) node[midway,right] {\tiny $\randomxof{\atomorder\shortminus1}$}; 


\node[anchor=center] (text) at (7,-2) {${=}$};

\node[anchor=center] (text) at (12,-2.5) {${\sum\limits_{\atomindices\in\atomstates}}$};
\node[anchor=center] (text) at (12,-4) {\tiny $\exformula(\atomindices)=1$};

\begin{scope}[shift={(19.5,1)}]

%\draw (-2,1) rectangle (4,-1);
%\node[anchor=center] (text) at (1,0) {\small $\onehotmapof{\exformula(\atomindices)}$};
%\draw[->-] (1,1)--(1,3) node[midway,right] {\tiny $\catvariableof{\exformula}$};

\draw (-3,-2) rectangle (-1,-4);
\node[anchor=center] (text) at (-2,-3) {\small $\onehotmapof{\atomlegindexof{0}}$};
\draw[->-] (-2,-4)--(-2,-6) node[midway,right] {\tiny $\catvariableof{0}$};

\node[anchor=center] (text) at (1,-3) {\small $\cdots$};

\draw (3,-2) rectangle (5,-4);
\node[anchor=center] (text) at (4,-3) {\small $\onehotmapof{\atomlegindexof{\atomorder\shortminus1}}$};
\draw[->-] (4,-4)--(4,-6) node[midway,right] {\tiny $\catvariableof{\atomorder\shortminus1}$};

\end{scope}

\end{tikzpicture}
\end{center}




% Maps to multiple formulas -> Later?
%We can extend the map to factored systems of multiple formulas, by using \defref{def:formulas} as coordinate maps.
%This is exactly what we will study by Bayesian Propositional Networks.
%We will make use of redundancies in the maps to get an efficient representation based on decompositions.





%% Semantic approach
We here chose a semantic approach to propositional logic in contrary to the standard syntactical approach.
Instead of defining formulas by connectives acting on atomic formulas, we define them here as binary valued functions of the states of a factored system.
They are interpreted by marking possible states as models, given the knowledge of $\exformula$.
The syntactical side will then be introduced later by studying decompositions of formulas.


%\begin{figure}[h]
%\begin{center}
%	\begin{tikzpicture}[scale=0.35, thick] % , baseline = -3.5pt

%\draw[->-] (2,-1)--(2,1) node[midway,right] {\tiny $\catvariableof{\exformula}$};
\draw (-1,-1) rectangle (5,-3);
\node[anchor=center] (text) at (2,-2) {\small ${\exformula}$};
\draw[] (0,-3)--(0,-5) node[midway,left] {\tiny $\randomxof{0}$}; 
\draw[] (1.5,-3)--(1.5,-5) node[midway,left] {\tiny $\randomxof{1}$}; 
\node[anchor=center] (text) at (3,-4) {$\cdots$};
\draw[] (4,-3)--(4,-5) node[midway,right] {\tiny $\randomxof{\atomorder\shortminus1}$}; 


\node[anchor=center] (text) at (7,-2) {${=}$};

\node[anchor=center] (text) at (12,-2.5) {${\sum\limits_{\atomindices\in\atomstates}}$};
\node[anchor=center] (text) at (12,-4) {\tiny $\exformula(\atomindices)=1$};

\begin{scope}[shift={(19.5,1)}]

%\draw (-2,1) rectangle (4,-1);
%\node[anchor=center] (text) at (1,0) {\small $\onehotmapof{\exformula(\atomindices)}$};
%\draw[->-] (1,1)--(1,3) node[midway,right] {\tiny $\catvariableof{\exformula}$};

\draw (-3,-2) rectangle (-1,-4);
\node[anchor=center] (text) at (-2,-3) {\small $\onehotmapof{\atomlegindexof{0}}$};
\draw[->-] (-2,-4)--(-2,-6) node[midway,right] {\tiny $\catvariableof{0}$};

\node[anchor=center] (text) at (1,-3) {\small $\cdots$};

\draw (3,-2) rectangle (5,-4);
\node[anchor=center] (text) at (4,-3) {\small $\onehotmapof{\atomlegindexof{\atomorder\shortminus1}}$};
\draw[->-] (4,-4)--(4,-6) node[midway,right] {\tiny $\catvariableof{\atomorder\shortminus1}$};

\end{scope}

\end{tikzpicture}
%\end{center}
%\caption{Direct interpretation of a propositional formula $\exformula$ as a tensor.
%	The tensor is the sum of the one hot encodings of its models.
%	While the one hot encodings are directed, their sum is not.}
%\label{fig:formulaDirect} 
%\end{figure}

%% Intro of connectives
%Logical connectives are basic building blocks of such formulas and can be understood by simple computations represented in truth tables.
% Here truth tables?
%We call each combination of atomic formulas with connectives a formula.

\subsect{Relational encoding of formulas}


%% Direct and Relational interpretation of $\exformula$
There are two ways to represent formulas by tensors.
One way is to understand $[2]$ as subset of $\rr$ and interpreting the formula directly as a tensor (as in \defref{def:formulas}).
Another way is to understand $[2]$ as the possible values of a categorical variable.
% Maps between factored systems
Following this second perspective, formulas are maps between factored systems, where the image system is the factored systems of atoms and the target system the atomic system defined by a variable $\formulavar$ representing the formula satisfaction.
%Following this perspective, formulas are maps between the factored systems of atoms and the atomic system of the formula.
We can then build the relational encoding (\defref{def:functionRepresentation}) of that map to represent the formula (see Figure~\ref{fig:formulaRencoding}).

%\begin{definition}[Relation Encoding of Formulas] % Own definition, since a reinterpretation of the formula
Given a factored system with $\atomorder$ atoms $\shortcatvariables$ and a propositional formula $\formula$, the relational encoding of $\formula$ (see \defref{def:functionRepresentation}) is the tensor
\begin{align*}
	\rencodingofat{\formula}{\formulavar,\shortcatvariables} \in  \left(\atomspace \right) \otimes \rr^2
\end{align*}
decomposable as
\begin{align} 
	\rencodingofat{\formula}{\formulavar,\shortcatvariables} 
	= & \sum_{\shortcatindices\in\atomstates}  \onehotmapofat{\shortcatindices}{\shortcatvariables} \otimes \onehotmapofat{\formulaat{\indexedshortcatvariables}}{\formulavar} \, . 
\end{align}
%\end{definition}

%% More general relational encodings
We can build relational encodings more generally of any tensors, where we identify the image of the tensor with the states of a categorical variable.
Exactly for propositional formulas, this construction will lead to Boolean image variables.


\begin{lemma}\label{lem:formulaEncodingDecomposition}
	For any formula $\formula$ we have
	\begin{align*}
		\rencodingofat{\formula}{\formulavar,\shortcatvariables}
		= \formulaat{\shortcatvariables} \otimes \tbasisat{\formulavar}
		+ \lnot\formulaat{\shortcatvariables} \otimes  \fbasisat{\formulavar} \, .
	\end{align*}
	In particular
	\begin{align*}
		 \formulaat{\shortcatvariables} = \contractionof{
		\rencodingofat{\formula}{\formulavar,\shortcatvariables},\tbasisat{\formulavar}
		}{\shortcatvariables} \, .
	\end{align*}
\end{lemma}
\begin{proof}
%% Decomposition
	We can decompose relational encodings of formulas into the sum (see Figure~\ref{fig:formulaRencoding}) % ! Not a tensor network decomposition !
	\begin{align} 
		 \rencodingofat{\formula}{\formulavar,\shortcatvariables}  
		 = & \fbasisat{\formulavar} \otimes \left( \sum_{\formulazerocoordinates}  \onehotmapofat{\shortcatindices}{\shortcatvariables} \right) \\
		 + & \tbasisat{\formulavar} \otimes \left( \sum_{\formulaonecoordinates}  \onehotmapofat{\shortcatindices}{\shortcatvariables} \right)
	\end{align}
	where the second term sums up the models of $\formula$ and the first one the models of $\lnot\formula$.
\end{proof}


% Comparison with direct interpretation
Compared with the direct interpretation of a formula as a tensor and the decomposition into models in Equation~\ref{eq:formulaModelDecomposition}, we notice that the relational encoding also represents encoding of worlds where the formula is not satisfied.
This representation is required to represent arbitrary propositional formulas by contracted tensor networks of its components, as will be investigated in the following sections.

%% Coordinatewise 
The relational encoding $\rencodingof{\exformula}$ has slices
\begin{align*}
	\contractionof{\rencodingof{\exformula},\onehotmapof{\shortcatindices}}{\formulavar} 
		\rencodingofat{\exformula}{\indexedshortcatvariables,\formulavar}
	= \begin{cases}
		\tbasis[\formulavar] & \text{if the world $\shortcatindices$ is a model of $\exformula$}  \\
		\fbasis[\formulavar] & \text{else}\, .
		\end{cases}
\end{align*}
The contractions of the relational encoding therefore calculate whether an assignment of atoms is a model of the formula, using basis calculus (see \theref{the:basisCalculus}).

\begin{figure}[h]
\begin{center}
	\input{./PartI/tikz_pics/logic_representation/formula_rencoding.tex}
\end{center}
\caption{Relational encoding of a propositional formula. 
The encoding is a sum of the one hot encodings of all states of the factored system in a tensor product with basis vectors, which encode whether the state is a model of the formula.
The tensor is directed, since any contraction with an encoded state results in the basis vector evaluating the formula, which we called basis calculus.
}
\label{fig:formulaRencoding} 
\end{figure}


\sect{Syntax of Propositional Formulas}

Relational encodings of propositional formulas are especially useful when representing function compositions by the representation of their components (see \theref{the:compositionByContraction}).
In propositional logics, the syntax of defining propositional formulas is oriented on compositions of formulas by connectives. % Quantifications will be studied in the FOL Chapter.
We in this section investigate the decomposition schemes of relational encodings into tensor networks of component encodings for binary tensors following propositional logic syntax.

\subsect{Atomic Formulas}

We call atomic formulas the most granular formulas, which are not splitted into compositions of other formulas.
Our syntactic decomposition of propositional formulas will then investigate, how any propositional formula can be represented by these.

\begin{definition}
	The tensors $\formulaofat{\atomenumerator}{\shortcatvariables}$ defined for $\shortcatindices\in\atomstates$ as
	\begin{align*}
		\formulaofat{\atomenumerator}{\indexedshortcatvariables}
		= \catindexof{\atomenumerator}
	\end{align*}
	are called atomic formulas.
\end{definition}

Atomic formulas and their relational encodings have an especially compelling representation.

\begin{theorem}
	Any atomic formula $\formulaofat{\atomenumerator}{\shortcatvariables}$ is represented as
	\begin{align*}
		\formulaofat{\atomenumerator}{\indexedshortcatvariables}
		= \contractionof{\tbasisat{\catvariableof{\atomenumerator}}}{\shortcatvariables}
		= \tbasisat{\catvariableof{\atomenumerator}} \otimes \onesat{\catvariableof{[\catorder]/\{\atomenumerator\}}}  \, .
	\end{align*}

	The relational encoding of any atomic formula $\atomicformulaofat{\atomenumerator}{\shortcatvariables}$ has a tensor decomposition by
	\begin{align*}
		\rencodingofat{\atomicformulaof{\atomenumerator}}{\headvariableof{\atomorder},\shortcatvariables}
		= \contractionof{\identityat{\catvariableof{\atomenumerator},\headvariableof{\atomenumerator}}}{\shortcatvariables}
		= \identityat{\catvariableof{\atomenumerator},\headvariableof{\atomenumerator}} \otimes \onesat{\catvariableof{[\catorder]/\{\atomenumerator\}}} \, .
	\end{align*}
	The decomposition is depicted in a network diagram as
	\begin{center}
		\begin{tikzpicture}[scale=0.35,thick] % , baseline = -3.5pt

\drawatomcore{3.5}{-8}{$\bencodingof{\formulaof{\atomenumerator}}$}
\drawatomindices{3.5}{-12}	
\draw[->-] (5.5,-9)--(5.5,-7) node[midway,right] {\tiny $\headvariableof{\atomenumerator}$};

\node[anchor=center] (text) at (10,-10) {${=}$};

\draw (12,-9) rectangle (15,-11); 
\node[anchor=center] (text) at (13.5,-10) {\small $\ones$}; 
\draw[-<-] (12.5,-11)--(12.5,-13) node[midway,left] {\tiny $\catvariableof{0}$};
\node[anchor=center] (text) at  (13.5,-12) {$\cdots$};
\draw[-<-] (14.5,-11)--(14.5,-13) node[midway,right] {\tiny $\catvariableof{\atomenumerator\shortminus1}$};

\node[anchor=center] (text) at (16.25,-10) {\small $\otimes$}; 

\draw[->-] (18.5,-9)--(18.5,-7) node[midway,right] {\tiny $\headvariableof{\atomenumerator}$};
\draw (17.5,-9) rectangle (19.5,-11);
\node[anchor=center] (text) at (18.5,-10) {\small $\delta$}; 
\draw[-<-]  (18.5,-11)--(18.5,-13) node[midway,right] {\tiny $\catvariableof{\atomenumerator}$};

\node[anchor=center] (text) at (20.75,-10) {\small $\otimes$}; 

\begin{scope}[shift={(10,0)}]

\draw (12,-9) rectangle (15,-11); 
\node[anchor=center] (text) at (13.5,-10) {\small $\ones$}; 
\draw[-<-]  (12.5,-11)--(12.5,-13) node[midway,left] {\tiny $\catvariableof{\atomenumerator+1}$};
\node[anchor=center] (text) at  (13.5,-12) {$\cdots$};
\draw[-<-]  (14.5,-11)--(14.5,-13) node[midway,right] {\tiny $\catvariableof{\atomorder\shortminus1}$};

\node[anchor=center] (text) at  (16.5,-13) {$.$};

\end{scope}

\end{tikzpicture}
	\end{center}
\end{theorem}
\begin{proof}
	We have by definition
	\begin{align*}
		\rencodingofat{\atomicformulaof{\atomenumerator}}{\headvariableof{\atomenumerator},\shortcatvariables}
		=& \sum_{\catindices\in\atomstates} \onehotmapofat{\catindices}{\shortcatvariables} \otimes \onehotmapofat{\formulaofat{\atomenumerator}{\indexedcatvariables}}{\headvariableof{\atomenumerator}} \\
		=& \left( \onehotmapofat{0,0}{\catvariableof{\atomenumerator},\headvariableof{\atomenumerator}} +
		\onehotmapofat{1,1}{\catvariableof{\atomenumerator},\headvariableof{\atomenumerator}} \right) \otimes \onesat{\catvariableof{\secatomenumerator}\, : \, \secatomenumerator \neq \atomenumerator} \\
		=& \contractionof{\identityat{\catvariableof{\atomenumerator},\headvariableof{\atomenumerator}}}{\shortcatvariables,\headvariableof{\atomenumerator}} \, .
	\end{align*} 
\end{proof}

\subsect{Syntactical combination of formulas}

Propositional formulas are elements of tensor spaces with $\atomorder$ axis. 
The number of coordinates thus grows exponentially with the number of atoms, which is
\begin{align*}
	\dimof{\atomspace} = 2^{\atomorder} \, .
\end{align*}
When the number of atoms is large, the naive representation of formula tensors will be thus intractable.
In contrast, typcial logical formulas appearing in practical knowledge bases are sparse in the sense that they have short representations in a logical syntax.
Motivated by this consideration we now discuss propositional syntax and investigate the sparse decomposition of formula tensors along their formula structure to avoid the curse of dimensionality.

%% Propositional Syntax
In logical syntax formulas are described by atomic formulas recursively connected via connectives. 
We show, that representations of logical connectives can be represented by feasible tensor cores $\rencodingof{\exconnective}$ contracted along a tensor network.
Let us first provide in \exaref{exa:connectives} unary ($\atomorder=1$) and binary ($\atomorder=2$) connectives.

\begin{example}\label{exa:connectives}
	We use the following connectives:
	\begin{itemize}
	\item negation $\notucon: [2]\rightarrow [2]$ by the vector
	\begin{align*}
		\notucon[\exformulavar] = \begin{bmatrix}
		0  \\
		1  
		\end{bmatrix} 
	\end{align*}
	\item conjunctions $\land:  [2]\times[2] \rightarrow[2]$
		\begin{align*}
			\land[\exformulavar,\secexformulavar]
			 = \begin{bmatrix}
			0 & 0 \\
			0 & 1 
			\end{bmatrix}
		\end{align*}
	\item disjunctions $\lor : [2]\times[2] \rightarrow[2]$
		\begin{align*}
			\lor[\exformulavar,\secexformulavar]
			 = \begin{bmatrix}
			0 & 1 \\
			1 & 1 
			\end{bmatrix}
		\end{align*}
	\item exact disjunction $\oplus:  [2]\times[2] \rightarrow[2]$	
		\begin{align*}
			\oplus[\exformulavar,\secexformulavar]
			 = \begin{bmatrix}
			0 & 1 \\
			1 & 0 
			\end{bmatrix}
		\end{align*}
	\item implications $\impbincon:  [2]\times[2] \rightarrow[2]$ 
		\begin{align*}
			\impbincon[\exformulavar,\secexformulavar]
			 = \begin{bmatrix}
			1 & 1 \\
			0 & 1 
			\end{bmatrix}
		\end{align*}
	\item biimplication $\eqbincon:  [2]\times[2] \rightarrow[2]$ 
		\begin{align*}
			\eqbincon[\exformulavar,\secexformulavar]
			 = \begin{bmatrix}
			1 & 0 \\
			0 & 1 
			\end{bmatrix}
		\end{align*}
	\end{itemize}
\end{example}

\begin{figure}[h]
\begin{center}
	\begin{tikzpicture}[scale=0.35, thick] % , baseline = -3.5pt

\node[anchor=center] (text) at (2,-4) {$a)$};

\draw[->] (5.5,-5)--(5.5,-3) node[midway,right] {\tiny $\headvariableof{\neg\exformula}$};

\node[anchor=center] (text) at (5.5,-6) {$\rencodingof{\lnot}$};
\draw (4.5,-7) rectangle (6.5,-5);

\draw[->] (5.5,-9)--(5.5,-7) node[midway,right] {\tiny $\formulavar$};


\drawatomcore{3.5}{-8}{$\rencodingof{\exformula}$}
\drawatomindices{3.5}{-12}	




\begin{scope}[shift={(15,0)}]

\node[anchor=center] (text) at (2,-4) {$b)$};

\draw[->] (9.5,-5)--(9.5,-3) node[midway,right] {\tiny $\headvariableof{\exformula\circ\secexformula}$};

\node[anchor=center] (text) at (9.5,-6) {$\rencodingof{\circ}$};
\draw (4.5,-7) rectangle (14.5,-5);

\draw[->] (5.5,-9)--(5.5,-7) node[midway,right] {\tiny $\formulavar$};

\drawatomcore{3.5}{-8}{$\rencodingof{\exformula}$}
\drawatomindices{3.5}{-12}	

\begin{scope}[shift={(8,0)}]

	\draw[->] (5.5,-9)--(5.5,-7) node[midway,right] {\tiny $\secexformulavar$};

	\drawatomcore{3.5}{-8}{$\rencodingof{\secexformula}$}
	\drawatomindices{3.5}{-12}	

\end{scope}

\draw[fill] (7.5,-15) circle (0.25cm);
\draw[] (7.5,-15) to[bend left=25] (3.5,-13);
\draw[] (7.5,-15) to[bend right=25] (11.5,-13);

\draw[fill] (9,-15.25) circle (0.25cm);
\draw[] (9,-15.25) to[bend left=25] (5,-13);
\draw[] (9,-15.25) to[bend right=25] (13,-13);

\draw[fill] (11.5,-15) circle (0.25cm);
\draw[] (11.5,-15) to[bend left=25] (7.5,-13);
\draw[] (11.5,-15) to[bend right=25] (15.5,-13);



\draw[] (7.5,-15)--(7.5,-17) node[midway,left] {\tiny $\catvariableof{0}$}; 
\draw[] (9,-15.25)--(9,-17) node[midway,left] {\tiny $\catvariableof{1}$}; 
\node[anchor=center] (text) at (10.5,-16.5) {$\cdots$};
\draw[] (11.5,-15)--(11.5,-17) node[midway,right] {\tiny $\catvariableof{\atomorder-1}$}; 

\end{scope}

\end{tikzpicture} 
\end{center}
\caption{a) Relational encoding of a negated formula $\exformula$ as a tensor network of the encoded formula and the encoded connective $\notucon$.
	b) Relational encoding of a composition of formulas $\exformula, \secexformula$ by a connective $\exconnective\in\{\land,\lor,\oplus,\impbincon,\eqbincon\}$. 
	The encoding is a contraction of encodings to  $\exformula, \secexformula$ and $\exconnective$.}
	\label{fig:unaryBinaryComposition} 
\end{figure}

We now show how formulas consisting of connectives acting on other formulas can be represented by basis calculus.
Let there be formulas $\exformula$ and $\secexformula$ depending on categorical variables $\shortcatvariables$ and a binary connective
	\[ \exconnective: [2]\times[2] \rightarrow[2] \, . \]
Then we can show as a special case of the next theorem, that (see \figref{fig:unaryBinaryComposition})
\begin{align*}
	\rencodingofat{\exformula\exconnective\secexformula}{\shortcatvariables,\catvariableof{\exformula\exconnective\secexformula}}
	= \contractionof{
	\rencodingofat{\exconnective}{\formulavar,\secexformulavar,\catvariableof{\exformula\exconnective\secexformula}},
	\rencodingofat{\exformula}{\shortcatvariables,\formulavar},
	\rencodingofat{\secexformula}{\shortcatvariables,\secexformulavar} 
	}{
	\shortcatvariables,\catvariableof{\exformula\exconnective\secexformula}
	} \, . 
\end{align*}

For any unary connective $\exconnective: [2] \rightarrow[2]$ we have
\begin{align*}
	\rencodingofat{\exconnective\exformula}{\shortcatvariables,\catvariableof{\exconnective\exformula}}
	= \contractionof{
	\rencodingofat{\exconnective}{\formulavar,\catvariableof{\exconnective\exformula}},
	\rencodingofat{\exformula}{\shortcatvariables,\formulavar}
	}{
	\shortcatvariables,\catvariableof{\exconnective\exformula}} \, . 
\end{align*}

Let us now generalize this observation to arbitrary arity of connectives and provide a proof of its correctness.

% CLASH OF NOTATION: HEADVARIABLES - CATVARIABLES
\begin{theorem}[Composition of Formulas]\label{the:formulaDecomposition}
	Let there be a formula $\formulaat{\shortcatvariables}$, which has a syntactical decomposition into connectives $\{\connectiveofat{\selindex}{\headvariableof{\nodesof{\selindex}}} : \selindexin\}$ taking their inputs by variables $\headvariableof{\nodesof{\selindex}}\subset \headvariableof{\nodes}$ and output by a variable $\headvariableof{\connectiveof{\selindex}}$
	%Let there be a set of boolean variables $\headvariableof{\formulaset}$, where $\formulaset$ is  and boolean atom variables $\shortcatvariables$.
	We here denote by $\formulaset$ the set of sub-formulas and use a boolean variable $\headvariableof{\secexformula}$ for each $\secexformula\in\formulaset$.
	In particular, we denote for each atom in $\formulaset$ the corresponding boolean variable by $\headvariableof{\atomenumerator}$.
	It then holds
	\begin{align*}
		\rencodingofat{\formula}{\formulavar,\shortcatvariables} =
		\contractionof{\left\{
		\rencodingofat{\connectiveof{\selindex}}{\headvariableof{\connectiveof{\selindex}},\headvariableof{\nodesof{\selindex}}} : \selindexin
		\right\} \cup \{\identityat{\headvariableof{\atomenumerator},\catvariableof{\atomenumerator}} \, : \, \atomenumeratorin\}}
		{\formulavar,\shortcatvariables} \, . 
	\end{align*}
\end{theorem}
\begin{proof}
	When a variable in $\headvariableof{\formulaset}$ appears multiple times as input to connectives, we replace it by a set of copies (which wont change the contraction, since all tensors are binary and \theref{the:invarianceAddingSubcontractions} can be applied).
	This follows from an iterative application of \theref{the:compositionByContraction} to be shown in \charef{cha:basisCalculus}.
\end{proof}

\begin{remark}[$\atomorder$-ary connectives such as $\land$ and $\lor$]\label{rem:naryConnectives}
	Since the decomposition of relational encoding can be applied to generic function compositions (see \theref{the:compositionByContraction}), we can also allow for $\atomorder$-ary connectives
		\[ \exconnective : \bigtimes_{\atomenumeratorin} [2] \rightarrow [2] \, . \]
	The connectives $\land$ and $\lor$ satisfy associativity and have thus straightforward generalizations to the $\atomorder$-ary case.
	This is because associativity can be exploited to represent the relational encoding by any tree-structured composition of binary $\land$ and $\lor$ connectives.
\end{remark}

%Propositional Syntax describes generic formulas $\exformula$ based on the composition of these maps starting with atomic formulas.
%We can thus apply \theref{the:compositionByContraction} recursively to decompose the formula tensors $\rencodingof{\exformula}$ into cores $\rencodingof{\exconnective}$.

%% Construction from atomic formula tensors
Propositional syntax consists in the application of connectives on atomic formulas, and recursively on the results of such constructions.
When passed towards connective cores, atomic formula tensors act trivial on the legs and just identify the corresponding atomic formula index $\catindexof{\atomicformulaof{\atomenumerator}}$ with $\catindexof{\atomenumerator}$.
This is due to the fact, that contractions with the trivial tensor $\ones$ leaves any tensor invariant, and the contraction with the elementary matrix $\identity$ identifies indices with each other.
We can thus savely ignore the atomic formula tensors appearing in the decomposition of formula tensors to non-atomic formulas.
An example of such a decomposition is depicted in \figref{fig:decompositionExample}.

\begin{figure}[h]
\begin{center}
	\begin{tikzpicture}[scale=0.35, yscale=-1, thick] % , baseline = -3.5pt

\begin{scope}[shift={(-15,0)}]

\node[anchor=center] (text) at (-3,-6) {${a)}$};

	\node [circle, draw, thick, fill=gray!50] (T1) at (0,0) {\tiny $\catvariableof{a}$};
	\node [circle, draw, thick, fill=gray!50] (T2) at (3,0) {\tiny $\catvariableof{b}$};
	\node [circle, draw, thick, fill=gray!50] (T3) at (6,0) {\tiny $\catvariableof{c}$};
	
	\node [circle, draw, thick, fill=gray!50] (and) at (1.5,-3) {\tiny $\land$};
	\node [circle, draw, thick, fill=gray!50] (not) at (6,-3) {\tiny $\lnot$};	
	
	\draw [->] (T1) -- (and);
	\draw [->] (T2) -- (and);
	
	\draw [->] (T3) -- (not);	
	
	\node [circle, draw, thick, fill=gray!50] (head) at (3.25,-6) {\tiny $\land$};
	
	\draw [->] (and) -- (head);
	\draw [->] (not) -- (head);			
\end{scope}

\node[anchor=center] (text) at (-3,-6) {${b)}$};

\draw[->] (0,1)--(0,-1) node[midway,left] {\tiny $\catvariableof{a}$}; 
\draw[->] (1.5,1)--(1.5,-1) node[midway,right] {\tiny $\catvariableof{b}$}; 
\draw[->] (3,1)--(3,-1) node[midway,right] {\tiny $\catvariableof{c}$}; 
\draw (-1,-1) rectangle (4, -3);
\node[anchor=center] (text) at (1.5,-2) {\small $\rencodingof{a \land b \land \lnot c}$};
\draw[->] (1.5,-3)--(1.5,-5) node[midway,right] {\tiny $\headvariableof{a \land b \land \lnot c}$}; 

\node[anchor=center] (text) at (5,-2) {${=}$};


\begin{scope}[shift={(7,0)}]

\draw[->] (0,1)--(0,-1) node[midway,left] {\tiny $\catvariableof{a}$}; 
\draw[->] (3,1)--(3,-1) node[midway,right] {\tiny $\catvariableof{b}$}; 
\draw[->] (6,1)--(6,-1) node[midway,right] {\tiny $\catvariableof{c}$}; 
	
\draw (-1,-1) rectangle (4, -3);
\node[anchor=center] (text) at (1.5,-2) {\small $\rencodingof{\land}$};

\draw[->] (1.5,-3) --(1.5,-5) node[midway,right]{\tiny $\headvariableof{a \land b}$};

\draw (5,-1) rectangle (7, -3);
\node[anchor=center] (text) at (6,-2) {\small $\rencodingof{\lnot}$};

\draw[->] (6,-3) --(6,-5) node[midway,right]{\tiny $\headvariableof{\lnot c}$};
	
\draw (0.5,-5) rectangle (6.5,-7);
\node[anchor=center] (text) at (3.5,-6) {\small $\rencodingof{\land}$};
	
\draw[->] (4,-7) -- (4,-9) node[midway,right] {\tiny $\headvariableof{a \land b \land \lnot c}$};

%\draw (3,-9) rectangle (5,-11);
%\node[anchor=center] (text) at (4,-10) {$\truevectorat{}$};

\end{scope}

\end{tikzpicture}
\end{center}
\caption{Decomposition of the formula tensor to $\exformula = a \land b \land \lnot c$ into unary (matrix) and binary (third order tensor) cores.
	a) Visualization of $\exformula$ as a graph.
	b) Tensor Network decomposition of $\exformula$.
	We can make use of the invariance of a Hadamard product with a constant tensor $\ones$ and thus not draw axis to atoms not affected by a formula.}
\label{fig:decompositionExample}
\end{figure}

\begin{remark}[Tensor Network Decomposition of Formulas]
	The decomposition of the propositional into a tensor network is a hierarchical decomposition of the formula tensor, which we will describe in more detail in \secref{sec:HT}.
	Of special interest are tree hypergraphs, where the format is called Hierarchical Tucker.
	At each decomposition of a formula into sub-formulas, two subspaces spanned by the respective atomic spaces are selected. 
\end{remark}


\subsect{Syntactical decomposition of formulas}\label{sec:termClauseDecomposition}

% Decomposition in case of missing 
We have seen how the decomposition of complex formulas into connectives acting on the component formulas can be exploited to find effective representations of the semantics by tensor networks.
Here the question arises here, how to perform such decompositions in case of a missing syntactical representation of a formula.
By \defref{def:formulas} any binary tensor is a formula.
We show in the following, how we can find a syntactic specification of a formula given its tensor.

%
%Let us now show that any formula tensor can be decomposed into a network of these connective symbols and atomic formula tensors.


\begin{definition}[Terms and Clauses]\label{def:clauses}
	Given two disjoint subsets $\nodesof{0}$ and $\nodesof{1}$ of $[\atomorder]$, the corresponding term is the formula defined on the indices $\shortcatindices\in\atomstates$ by
		\[ \termofat{\nodesof{0}}{\nodesof{1}}{\shortcatvariables}
		=\left( \bigwedge_{\atomenumerator\in\nodesof{0}} \lnot\formulaof{\atomenumerator} \right)  \land \left( \bigwedge_{\atomenumerator\in\nodesof{1}} \formulaof{\atomenumerator} \right)  \]
	and the corresponding clause is the formula defined on the indices $\catindices\in\atomstates$ by
		\[ \clauseofat{\nodesof{0}}{\nodesof{1}}{\shortcatvariables}
		=\left( \bigvee_{\atomenumerator\in\nodesof{0}} \formulaof{\atomenumerator} \right)  \lor \left( \bigvee_{\atomenumerator\in\nodesof{1}} \lnot\formulaof{\atomenumerator} \right)  \, , \]
	where by $\land_{\atomenumerator\in\nodes}$ and $\lor_{\atomenumerator\in\nodes}$ we refer to the $n$-ary connectives $\land$ and $\lor$.
	%We call the clause a minterm, if $\nodesof{0}\cup\nodesof{1} = [\atomorder]$.
	We call the term a minterm and the clause a maxterm, if $\nodesof{0}\cup\nodesof{1} = [\atomorder]$.
\end{definition}

%% 
Terms and Clauses have for any index tuple $\shortcatindices$ a polynomial representation by
		\[ \termof{\nodesof{0}}{\nodesof{1}}[\indexedshortcatvariables] 
		= \left( \prod_{\atomenumerator \in \nodesof{0}} (1-\catindexof{\atomenumerator}) \right)
		\left(  \prod_{\atomenumerator \in \nodesof{1}} \catindexof{\atomenumerator} \right) \]
and
		\[ \clauseof{\nodesof{0}}{\nodesof{1}}[\indexedshortcatvariables] 
		= 1 - \left( \prod_{\atomenumerator \in \nodesof{0}} (1-\catindexof{\atomenumerator})\right)
		\left(  \prod_{\atomenumerator \in \nodesof{1}} \catindexof{\atomenumerator} \right) \, . \]


\begin{lemma}\label{lem:termClauseOneHot}
	Terms are contractions of one-hot encodings, that is for any disjoint subsets $\nodesof{0},\nodesof{1}\subset[\atomorder]$ we have
		\[ \termof{\nodesof{0}}{\nodesof{1}}[\shortcatvariables] = \contractionof{\onehotmapof{\{\catindexof{\atomenumerator}=0 : \atomenumerator\in\nodesof{0} \} \cup \{\catindexof{\atomenumerator}=1 : \atomenumerator\in\nodesof{1}\}}}{\shortcatvariables} \, . \]
	Clauses are substractions of one-hot encodings from the trivial tensor, that is for any disjoint subsets $\nodesof{0},\nodesof{1}\subset[\atomorder]$ we have
		\[ \clauseof{\nodesof{0}}{\nodesof{1}}[\shortcatvariables] = 
		\onesat{\shortcatvariables} -
		\contractionof{\onehotmapof{\{\catindexof{\atomenumerator}=0 : \atomenumerator\in\nodesof{0} \} \cup \{\catindexof{\atomenumerator}=1 : \atomenumerator\in\nodesof{1}\}}}{\shortcatvariables} \, . \]
\end{lemma}


	
%
The reference of the formulas in the case $\nodesof{0}\dot{\cup}\nodesof{1} = [\atomorder]$ as minterms and maxterms is due to the fact, that minterms are formulas with unique models and maxterms are formulas with a unique world not satisfying the formula.
% Enumeration by $\atomstates$
We use this insight and enumerate maxterms and minterms by the index $\catindex\in\atomstates$ of the unique world where the minterm is satisfied, respectively the maxterm is not satisfied.
For any $\nodesof{0}\dot{\cup}\nodesof{1} = [\atomorder]$ we take the index tuple $\catindices$ where $\catindexof{\atomenumerator}=0$ if $\atomenumerator\in\nodesof{0}$ and $\catindexof{\atomenumerator}=1$ if $\atomenumerator\in\nodesof{1}$ and define
\begin{align*}
	\maxtermof{\catindices} = \clauseof{\nodesof{0}}{\nodesof{1}} \quad \text{and} \quad \mintermof{\catindices} = \termof{\nodesof{0}}{\nodesof{1}} \, .
\end{align*}


\begin{corollary}
	Minterms are basis elements of the tensor space, that is for any $\shortcatindices\in\atomstates$ we have
	\begin{align*}
		\mintermof{\shortcatindices} = \onehotmapofat{\shortcatindices}{\shortcatvariables}
	\end{align*}
	Maxterms are substraction of basis elements from the trivial tensor, that is for any $\shortcatindices\in\atomstates$ we have
	\begin{align*}
		\maxtermof{\shortcatindices} = \onesat{\shortcatvariables} - \onehotmapofat{\shortcatindices}{\shortcatvariables}  \, .
	\end{align*}
\end{corollary}
\begin{proof}
	Follows from \lemref{lem:termClauseOneHot}, since when $\nodesof{0}\cup\nodesof{1} = [\atomorder]$ the contraction of the one-hot encodings coincides with the one-hot encoding of a fully specified state.
\end{proof}


Based on this insight, we can decompose any propositional formula into a conjunction of maxterms or a disjunction of minterms as we show next.

\begin{theorem}\label{the:tensorToMaxMinTerms}
	For any boolean tensor $\hypercoreat{\shortcatvariables}\in\atomspace$ with leg-dimensions two we have
	\begin{align*}
		\hypercoreat{\shortcatvariables} = \left( \bigvee_{\hyperonecoordinates} 
		\termof{\catzeropositions}{\catonepositions} 
		\right)
		[\shortcatvariables] 
	\end{align*}
	and
	\begin{align*}
		\hypercoreat{\shortcatvariables} = \left( \bigwedge_{\hyperzerocoordinates} 
		\clauseof{\catzeropositions}{\catonepositions} 
		\right)
		[\shortcatvariables] \, .
	\end{align*}
\end{theorem}
\begin{proof}
	To show the representation by minterms we use the decomposition
	\begin{align*}
		\hypercoreat{\shortcatvariables}  = \sum_{\hyperonecoordinates} \onehotmapofat{\shortcatindices}{\shortcatvariables}
	\end{align*}
	and notice that each term in the disjunction modifies the formula by adding respective world $\shortcatindices$ to the models of the formula.
	To show the representation by maxterms we use the decomposition
	\begin{align*}
		\hypercoreat{\shortcatvariables}  = \onesat{\shortcatvariables} \quad - \sum_{\hyperzerocoordinates} \onehotmapofat{\shortcatindices}{\shortcatvariables}
	\end{align*}
	and notice that each term in the conjunction modifies the formula by removing the respective world $\shortcatindices$ from the models of the formula.	
	Thus, both decompositions are propositional formulas with the same set of models as the formula $\hypercore$ and are thus identical to $\hypercore$.
\end{proof}


% Canonical normal forms
The decompositions found in \theref{the:tensorToMaxMinTerms} are also called canonical normal forms to propositional formulas $\hypercoreat{\shortcatvariables}$.

%% Universality of representations
\begin{remark}[Efficient Representation in Propositional Syntax]
	% Relation with binary CP
	The decomposition in \theref{the:tensorToMaxMinTerms} is a basis CP decomposition of the binary tensor and will further be investigated in \charef{cha:sparseCalculus}.
	The formulas constructed in the proof of \theref{the:tensorToMaxMinTerms} are however just one possibility to represent a formula tensor in propositional syntax.
	Typically there are much sparser representations for many formula tensors, in the sense that less connectives and atomic symbols are required.
	Having such a sparser syntactical description of a propositional formula can be exploited to find a shorter conjunctive normal form of the formula and construct a sparse polynomial based on similar ideas as in \theref{the:tensorToMaxMinTerms}.
	%One way to eliminate syntactical redundancies are through schemes for decompositions called normal forms, for example the Conjunctive Normal Form (CNF) or the Disjunctive Normal Form (DNF).
	We will provide such constructions in \charef{cha:sparseCalculus}, where we show that dropping the demand of directionality and investigating binary CP Decompositions will improve the sparsity of the polynomial formula representation.
\end{remark}

\subsect{Comparing with probabilistic approaches }

Both probability and logic provide a human-understandable interface to machine learning.
As we will describe in \parref{par:two}, they can be combined in one formalism providing efficient reasoning.

% Same thesises repeated??
\textbf{Probability} represents the uncertainty of states.
The categorical variables are called random variables and their joint distribution is represented by a probability tensor.
Humans interpret probabilities by Bayesian and frequentist approaches.
Reasoning based on Bayes Theorem has an intuitive interpretation in terms of evidence based update of prior distributions to posterior distributions.
However it is based on interpreting (large amounts) of numbers, which makes it hard for humans to assess the probabilistic reasoning process.

\textbf{Logics} explains relations between sets of worlds in a human understandable way.
Categorical variables have dimension $2$, where the first is interpreted as indicating a $\falsesymbol$ state and the second as a $\truesymbol$ state.
We mainly restrict to propositional logics, where there are finite sets of such variables called atomic formulas.
Using model-theoretic semantics it defines entailment of sets by other sets, which is understandable as a consequence relation.

\textbf{Tensors} unify both approaches since they are natural numerical structures to represent properties of states in factored systems.
The potential is then based in employing scalable multilinear algorithms to solve reasoning problems.
Further, algorithms formulated in tensor networks have a high parallelization potential, which is why they are of central interest in the development of AI-dedicated software and hardware.

The different areas have developed separated languages to describe similar objects.
Here we want to provide a rough comparison of those in a dictionary.

\begin{tabular}{l|l|l|l}
    & \textbf{Probability Theory} & \textbf{Propositional Logic} & \textbf{Tensors}   \\
    \hline
    \textit{Atomic System}        & Random Variable             & Atomic Formula               & Vector             \\
    \textit{Factored System}      & Joint Distribution          & Knowledge Base               & Tensor             \\
    \textit{Categorical Variable} & Random Variable             & Atomic Formula               & Axis of the Tensor
\end{tabular}

While the probability theory lacks to provide an intuition about sets of events, propositional syntax has limited functionality to represent uncertainties.
Tensors on the other side can build a bridge by representing both functionalities and relying on probability theory and logics for respective interpretations.


\sect{Outlook}

While we in this chapter investigated representation schemes for single propositional formulas, we will further study the representation of knowledge bases consisting in multiple formulas in \secref{sec:hardNetworks}.
Further, we will build hybrid models bridging the concepts of probability distributions and propositional logics in \secref{sec:hybridNetworks}.
Propositional formulas will therein serve as features and base measures for exponential families.
%Further study of representing Knowledge Bases based on Tensor Networks of its formulas in \secref{sec:hardNetworks} (see \theref{the:conDecKB}).
    \section{Logical Inference} \label{cha:logicalReasoning}

We approach logical inference by defining probability distributions based on propositional formulas and then apply the methodology introduced in the more generic situation of probabilistic inference.
Logical approaches pay here special attention to situations of certainty, where a state of a variable has probability $1$.
In this situation, we say that the corresponding formula is entailed.
%Such situations are called entailment, and we will investigate how we can find these by contractions.


% From Probabilistic 
We start the discussion by showing how formulas can be interpreted by distributions and define logical entailment based on corresponding probabilistic queries.
This enables us to define logical entailment based on the resulting conditional distributions.


%% Where to put?
\begin{remark}[Interpretation of Contractions in Logical Reasoning]
	The coordinates of contracted binary tensor networks describe whether the by the coordinate indexed world is a model of the Knowledge Base at hand.
	Contractions, which only leave a part variables open, store the counts of the world respecting conditions given by the choice of slices. 
	When contracting without open variables, we thus get the total worldcount.
	
	This is consistent with the probabilistic interpretation of contractions, when applying the frequentist interpretation of probability and defining normed worldcounts as probabilities.
\end{remark}


\subsection{Entailment in Propositional Logics}

\begin{definition}[Entailment of propositional formulas]\label{def:logicalEntailment}
	Given two propositional formulas $\kb$ and $\exformula$ we say that $\kb$ entails $\exformula$, denoted by $\kb\models\exformula$, if any model of $\kb$ is also a model of $\exformula$, that is
		\[ \forall_{\shortcatindices\in\atomstates} \big(\kbat{\indexedcatvariableof{[\catorder]}}=1\big) \rightarrow \big(\formulaat{\indexedcatvariableof{[\catorder]}}=1\big) \, . \]
	If $\kb\models\lnot\exformula$ holds, we say that $\kb$ contradicts $\exformula$.
\end{definition}

%
\red{Entailment can be understood by subset relations of the models of formulas.
This perspective can be applied with subset encodings in \charef{cha:basisCalculus}.
}





\subsubsection{Deciding Entailment by contractions}

\begin{theorem}[Contraction Criterion of Entailment]\label{the:contCriterionLogEntailment}
	We habe $\kb\models\exformula$ if and only if 
		\[ \sbcontraction{\kb,\lnot\exformula} = 0 \, . \]
\end{theorem}
\begin{proof}
	% <= 
	If for a $\atomindices$ we have $\kbat{\indexedcatvariableof{[\catorder]}}=1$ but not $\big(\exformula(\atomindices)=1\big)$, the contraction would be at least $1$.
	% =>
	Conversely, if the contraction is at least one, we would find $\atomindices$ with $\kbat{\indexedcatvariableof{[\catorder]}}=1$ and $\lnot\formulaat{\indexedcatvariableof{[\catorder]}}=1$, therefore $\formulaat{\indexedcatvariableof{[\catorder]}}=0$. 
	It follows that $\kb\models\exformula$ does not hold.
\end{proof}

% Can use relational encoding
To decide whether a formula is entailed, or its negation is entailed (in which case one says that the formula is contradicted) by a single contraction, one can perform the contraction
\begin{align*}
	\hypercore = \sbcontractionof{\kbat{\shortcatvariables},\formulaat{\shortcatvariables,\exformulavar}}{\exformulavar}
\end{align*}
and use that
\begin{align*}
	 \sbcontraction{\kb,\lnot\exformula} = \hypercoreat{\exformulavar=0} 
\end{align*}
and 
\begin{align*}
	 \sbcontraction{\kb,\exformula} = \hypercoreat{\exformulavar=1} \, .  
\end{align*}






\subsubsection{Contraction Knowledge Base}

We now show how to implement a propositional Knowledge Base with the TELL and ASK operations based on \theref{cor:parallelCriterion}.

% Works also for Markov Networks!
\begin{algorithm}[hbt!]
\caption{Contraction Knowledge Base}\label{alg:TensorKB}
ASK(formula $\exformula$)
\begin{algorithmic}
	\State{$\hypercoreat{\formulavar} \algdefsymbol \sbcontractionof{\kb,\rencodingof{\exformula}}{\formulavar}$}
	\If{$\hypercoreat{\formulavar=0}=0$} 
		\State{return Entailed}
	\EndIf
	\If{$\hypercoreat{\formulavar=1}=0$} 
		\State{return Contradicted}
	\EndIf
	\State{return Contingent}
\end{algorithmic}
TELL(formula $\exformula$)
\begin{algorithmic}
	\If{ASK($\exformula$) returns Contingent:}
%	\If{$\sbcontractionof{\kb,\exformula}>0$ and $\sbcontraction{\kb,\lnot\exformula}>0$} %Consistency + Redundancy check
	\State $\kb \algdefsymbol \kb\land\exformula$%Add cores of formula tensor $\exformula$ to $\rencodingof{\kb}$ in order to represent $\rencodingof{\kb\cup\{\exformula\}}$
	\EndIf
\end{algorithmic}

\end{algorithm}

\red{
Comment: TELL checks whether the formula to be added is entailed, in which case it is redundant to add, and whether the formula to be added is contradicted, in which case the knowledge base would become unsatisfiable.
}



\subsubsection{Sparse Representation of a Knowledge Base}

Let us now investigate how to sparsely represent a Knowledge Base.
Towards getting insights on this we first show that entailed formulas can be dropped from the Knowledge Base.

\begin{theorem}\label{the:ReduncancyOfEntailed}
	If and only if $\kb \models \exformula$ we have
		\[ \kbat{\shortcatvariables}= \sbcontractionof{\kb,\exformula}{\shortcatvariables}  \, . \]
\end{theorem}
\begin{proof}
	For any world indexed by a coordinate $\atomindices$, $\kbat{\indexedcatvariableof{[\catorder]}}$ indicates whether the world is a model of $\kb$.
	We have entailment, when the models of $\kb\cup\exformula$ coincide with those of $\kb$.
\end{proof}


\begin{remark}[Sparsest Description of a Knowledge Base]
	Given a set of worlds indexed by $\hypercore$, find the sparsest set of formulas $\kb$ such that
		\[ \hypercore = {\kb} \]
	would be benefitial for small computational complexity.
	Since the formula tensors are invariant under entailment, we can drop entailed formulas.
\end{remark}	





\subsection{Formulas as Random Variables}

\red{Aim here: Relate with the probabilistic reasoning concepts of marginal and conditional distributions.}

\red{Given a probability distribution $\probtensor$ of atoms we add a variable by building the Markov Network of $\probtensor$ and $\rencodingof{\exformula}$ to get a joint distribution of the atoms and a query formula $\exformula$}

There are two ways of interpreting formula tensors as conditional probabilities.
The standard one, which we also used above, understands the atomic legs as conditions and calculates the truth of the formula.
Another understands a formula as a condition.

\subsubsection{Conditioning on the atoms}

%% Conditional interpretation -> Formulas as conditional probability ("local")
Our main interpretation understands each tuple of indices $\atomindices$ as conditions of a probability tensor.
Given a truth assignment to the atomic variables $\atomicformulaof{\atomenumerator}$, that is a choice of indices $\atomlegindexof{\atomenumerator}$, determines the truth of the formula.
We thus interpret the formula tensors as defining a conditional probability of $\exformula$ given the atoms $\atomicformulaof{\atomenumerator}$ indexed by $\atomlegindexof{\atomenumerator}$.

\begin{theorem}\label{the:conditionByAtoms}
	The relational encoding of any propositional formula $\exformula$ coincides with the conditional probability of that formula conditioned on the identity on the atoms, that is
		\[ \rencodingof{\exformula} = \condprobof{\formulavar}{\atomicformulas} \, . \]
	We depict this by
	\begin{center}
		\begin{tikzpicture}[scale=0.35,thick] % , baseline = -3.5pt



\draw[->] (2,-1)--(2,1) node[midway,right] {\tiny $\catvariableof{\exformula}$};

\draw (-3,-1) rectangle (7,-3);
\node[anchor=center] (text) at (2,-2) {\small $\condprobof{\exformula}{\atomicformulas}$};
\draw[<-] (0,-3)--(0,-5) node[midway,left] {\tiny $\catvariableof{0}$}; 
\draw[<-] (1.5,-3)--(1.5,-5) node[midway,left] {\tiny $\catvariableof{1}$}; 
\node[anchor=center] (text) at (3,-4) {$\cdots$};
\draw[<-] (4,-3)--(4,-5) node[midway,right] {\tiny $\catvariableof{\atomorder\shortminus1}$}; 


\node[anchor=center] (text) at (9,-2) {${=}$};


\begin{scope}[shift={(12,0)}]

\draw[->] (2,-1)--(2,1) node[midway,right] {\tiny $\catvariableof{\exformula}$};
\draw (-1,-1) rectangle (5,-3);
\node[anchor=center] (text) at (2,-2) {\small $\rencodingof{\exformula}$};
\draw[<-] (0,-3)--(0,-5) node[midway,left] {\tiny $\catvariableof{0}$}; 
\draw[<-] (1.5,-3)--(1.5,-5) node[midway,left] {\tiny $\catvariableof{1}$}; 
\node[anchor=center] (text) at (3,-4) {$\cdots$};
\draw[<-] (4,-3)--(4,-5) node[midway,right] {\tiny $\catvariableof{\atomorder\shortminus1}$}; 

\node[anchor=center] (text) at (7,-5) {${\cdot}$};

\end{scope}


\end{tikzpicture}
	\end{center}
\end{theorem}
\begin{proof}
	The distribution $\probtensor$ does not influence the conditional query, since the normation acts on any state.
\end{proof}


% Interpretation of directionality as 
The conditional query $\condprobof{\formulavar}{\shortcatvariables}$ provides an interpretation of $\rencodingof{\exformula}$ as a conditional probability. 
This is also reflected in the fact that both $\condprobof{\formulavar}{\shortcatvariables}$ and $\rencodingof{\exformula}$ are directed, since the first is a normation by Defintion~\ref{def:queries} and the second a encoding of a function.

%The direction of the legs in the formula tensor diagram in Figure~\ref{fig:FormulaTensor} is chosen to highlight the conditional probability interpretation.


This directly implies using \theref{the:conditionalMarginalization}  the trivialization of the formula tensor when contracting its head axis indexed by $\atomlegindexof{\exformula}$ with the trivial vector $\ones$, depicted as
\begin{center}
	\input{./PartI/tikz_pics/logic_reasoning/ones_property_ft.tex}
\end{center}



\subsubsection{Conditioning on the formula}

% Defining probability distribution by formulas
Let us now converse the order of conditioning from $\condprobof{\exformula}{\atomicformulas}$ to $\condprobof{\atomicformulas}{\exformula}$.
In this way, we have propositonal formulas defining probability distributions on the factored system of atoms.

Given a Markov Network $\probtensor$ with a single core $\rencodingof{\exformula}$ for a propositional formula $\exformula$.
By definition we have
\begin{align*}
	\condprobof{\shortcatvariables}{\formulavar} 
	= \sbnormationofwrt{\rencodingof{\exformula}}{\shortcatvariables}{\formulavar} \, .  
\end{align*}
\begin{center}
	\begin{tikzpicture}[scale=0.35,thick] % , baseline = -3.5pt


\begin{scope}[shift={(-2,0)}]

%\draw[dashed] (1,1) rectangle (3,3);
%\node[anchor=center] (text) at (2,2) {\small $\onehotmapof{\atomlegindexof{\exformula}}$};

\draw[-<-] (2,-1)--(2,1) node[midway,right] {\tiny $\formulavar$};

\draw (-1,-1) rectangle (5,-3);
\node[anchor=center] (text) at (2,-2) {\small $\condprobof{\atomicformulas}{\formulavar}$};
\draw[->-] (0,-3)--(0,-5) node[midway,left] {\tiny $\catvariableof{0}$};
\draw[->-] (1.5,-3)--(1.5,-5) node[midway,left] {\tiny $\catvariableof{1}$};
\node[anchor=center] (text) at (3,-4) {$\cdots$};
\draw[->-] (4,-3)--(4,-5) node[midway,right] {\tiny $\catvariableof{\atomorder\shortminus1}$};


\node[anchor=center] (text) at (7,-2) {${=}$};

\end{scope}


\node[anchor=center] (text) at (8,-2.25) {$\sum\limits_{\headindexof{\exformula}\in[2]}$};

\draw[] (10,-1) rectangle (12,-3);
\node[anchor=center] (text) at (11,-2) {\small $\onehotmapof{\headindexof{\exformula}}$};

\draw[->-] (11,-3)--(11,-5) node[midway,right] {\tiny $\formulavar$};



\begin{scope}[shift={(15,0)}]

\draw[] (1,1) rectangle (3,3);
\node[anchor=center] (text) at (2,2) {\small $\onehotmapof{\headindexof{\exformula}}$};

\draw[->-] (2,-1)--(2,1) node[midway,right] {\tiny $\formulavar$};
\draw (-1,-1) rectangle (5,-3);
\node[anchor=center] (text) at (2,-2) {\small $\rencodingof{\exformula}$};
\draw[-<-] (0,-3)--(0,-5) node[midway,left] {\tiny $\catvariableof{0}$};
\draw[-<-] (1.5,-3)--(1.5,-5) node[midway,left] {\tiny $\catvariableof{1}$};
\node[anchor=center] (text) at (3,-4) {$\cdots$};
\draw[-<-] (4,-3)--(4,-5) node[midway,right] {\tiny $\catvariableof{\atomorder\shortminus1}$};


\end{scope}


\begin{scope}[shift={(25,0)}]

\draw (-5,-7) -- (0,3);
	
	\draw[] (1,1) rectangle (3,3);
	\node[anchor=center] (text) at (2,2) {\small $\onehotmapof{\headindexof{\exformula}}$};

	\draw[->-] (2,-1)--(2,1) node[midway,right] {\tiny $\formulavar$};
\draw (-1,-1) rectangle (5,-3);
\node[anchor=center] (text) at (2,-2) {\small $\rencodingof{\exformula}$};
\draw[-<-] (0,-3)--(0,-5) node[midway,left] {\tiny $\catvariableof{0}$};
\draw[-<-] (1.5,-3)--(1.5,-5) node[midway,left] {\tiny $\catvariableof{1}$};
\node[anchor=center] (text) at (3,-4) {$\cdots$};
\draw[-<-] (4,-3)--(4,-5) node[midway,right] {\tiny $\catvariableof{\atomorder\shortminus1}$};

\draw (-1,-5) rectangle (5,-7);
\node[anchor=center] (text) at (2,-6) {\small $\ones$};

\end{scope}

\node[anchor=center] (text) at (32,-5) {$.$};

\end{tikzpicture}
\end{center}

% Conditioning on the formula being true
Let us further investigate the slices of $\condprobof{\shortcatvariables}{\exformula}$ with respect to $\exformula$, which define distributions of the states of the factored system.
To this end, let us condition on the event of $\exformula=1$, for which we have the distribution
\begin{align}\label{eq:eventFormulaProb}
	\condprobof{\shortcatvariables}{\formulavar=1} = \frac{1}{\sbcontraction{\exformula}} \sum_{\shortcatindices\in\atomstates \, : \, \formulaat{\indexedshortcatvariables}=1} \onehotmapofat{\shortcatindices}{\shortcatvariables} \, .
\end{align}
With $\sbcontraction{\exformula}$ being the number of models of $\exformula$,  this is the uniform distribution among the models of $\exformula$.
Conversely, when conditioning on the event $\formulavar=0$ we get a uniform distribution of the models of $\lnot\exformula$.

% 
The probability distribution in Equation~\eqref{eq:eventFormulaProb} is well defined except for the case that $\sbcontraction{\exformula}=0$.
In this case we have $\exformula=0$ and call $\exformula$ unsatisfiable, since it has no models.

%The probability tensor is well-defined except for the case that $\theta_{1,:}$ contains just $0$ coordinates (respectively for $\theta_{0,:}$).
%This is an exceptionous situation in logics and called unsatisfiability of the knowledge base.
%\begin{definition}
%	A propositional formula $\exformula$ with $\sbcontraction{\exformula}=0$ is called unsatisfiable.
%\end{definition}
%If the Knowledge Base is inconsistent, the probabilistic interpretation breaks down.
%Thus we will always assume a consistent Knowledge Base when doing probabilistic reasoning.
%An alternative interpretation of formula tensors is the conditional probability of the atomic formulas given the formula at hand.
%To derive the conditional probability tensor we apply the Bayes Theorem
%\begin{align}
%	\condprobof{\{\atomicformulaof{\atomenumerator} = \atomlegindexof{\atomenumerator}\}}{\exformula=\atomlegindexof{\exformula}} 
%	=\frac{
%	\condprobof{\exformula}{\{\atomicformulaof{\atomenumerator} = \atomlegindexof{\atomenumerator}\}}
%	}
%	{
%	\sum_{\atomindices}\condprobof{\exformula}{\{\atomicformulaof{\atomenumerator} = \atomlegindexof{\atomenumerator}\}} \, .
%	}
%\end{align}



%% Uniform interpretation -> KB as probability distribution over its models ("global")
From an epistemological point of view, probability theory is a generalization of logics, since we allow for probability values in the interval $[0,1]$.
The set of distributions being constructed by conditioning on propositional formulas as in Equation~\eqref{eq:eventFormulaProb} correspond within the set of probability distributions with those having constant coordinates on their support.
% More specific
While the probability tensors with nonvanishing coordinates build a $2^\atomorder-1$-dimensional manifold, where the formulas parametrize $2^{2^\atomorder}$ probability tensors, most of which having vanishing coordinates.




\subsubsection{Probability of a function given a Knowledge Base}

% Both directions for entailment
We can now combine the ideas of the previous two subsections and define probabilities of formulas $\exformula$ given the satisfaction of another formula $\kb$, which we call a Knowledge Base.
We have by \theref{the:conditionByAtoms} % Again, Markov Network with rencoding of \exformula, \kb build the precise \probtensor
\begin{align*}
	\condprobof{\formulavar}{\kbvar} 
	& = \sbcontractionof{
	\condprobof{\formulavar}{\atomicformulas}, \condprobof{\atomicformulas}{\kbvar}
	}{\formulavar,\kbvar} \\
	& = \sbnormationofwrt{\rencodingof{\exformula} , \rencodingof{\kb}}{\formulavar}{\kbvar}
\end{align*}

% 
Of special interest is the marginal probability of $\formulavar$ given that $\kbvar$ is satisfied, that is
\begin{align*}
	\condprobof{\formulavar}{\kbvar=1} 
	& = \normationof{\{\rencodingof{\exformula} ,\kb\}}{\formulavar}\\
	& = \frac{\contractionof{\{\rencodingof{\exformula},\kb\}}{\formulavar}}{\contraction{\{\kb\}}} \, . 
\end{align*}


% Knowledge Base as Probability
\begin{remark}[Case of Unsatisfiable Knowledge Bases]
	When the Knowledge Base is not satisfiable, one cannot normate it and the probability distribution is not dedfined.
%	We notice, that by the criterion provided by \theref{cor:parallelCriterion} we can decide entailment also in the cases where $\kb$ is unsatisfiable.
%	In that case the contraction is the zero tensor, which is parallel to $\tbasis$ and $\fbasis$ and thus entailed and contradicting at the same time.
\end{remark}

%We will now define entailment based on this quantity. 
%\begin{definition}[Entailment]
%	We say that a not unsatisfiable Knowledge Base $\kb$ entails a formula $\exformula$, denoted by $\kb\models\exformula$, if $\condprobof{\formulavar=1}{\kbvar=1}=1$. 
%	If $\kb$ entails $\lnot\exformula$ we say that $\kb$ contradicts $\exformula$.
%	If the Knowledge Base $\kb$ is unsatisfiable, it entails any formula.
%	%
%	More generally, we say that a probability distribution $\probtensor$ entails a formula $\exformula$ if $\probat{\exformula=1}=1$.
%\end{definition}



% 
\begin{theorem}\label{the:probEntailment}
	Given a satisfiable formula $\kb$, we have $\kb\models\exformula$, if and only if 
		\[ \condprobof{\formulavar=0}{\kbvar=1} = 0 \, .  \]
\end{theorem}
\begin{proof}
	Since $\kb$ is satisfiable, we have $\sbcontraction{\kb}>0$ and
		\[ \condprobof{\formulavar=0}{\kbvar=1} = \frac{\sbcontraction{\lnot\exformula, \kb}}{\sbcontraction{\kb}} \, .  \]
	This term vanishes if and only if $\sbcontraction{\lnot\exformula, \kb}$ vanish.
	Thus, the condition is equivalent to the condition in \theref{the:contCriterionLogEntailment}.
\end{proof}

Given that $\kb$ is satisfiable, we therefore have $\kb\models\exformula$ if and only if
\begin{align}
	\condprobof{\formulavar}{\kbvar=1} = %\begin{cases}
	\tbasis \, .  %& \text{if }\kb \models \lnot\exformula \\
	%\tbasis & \text{if }\kb \models \exformula \\
	%\notin \{\fbasis,\tbasis\} & \text{else}
	%\end{cases} \, .
\end{align}
We depict this condition by the contraction diagram
%It suffices to check, whether the contraction with the normed Knowledge Base is the basis vector $\tbasis$, respectively $\fbasis$, that is
\begin{center}
	\begin{tikzpicture}[scale=0.35,thick] % , baseline = -3.5pt




\draw[->] (2,-1)--(2,1) node[midway,right] {\tiny $\catvariableof{\exformula}$};
\draw (-1,-1) rectangle (5,-3);
\node[anchor=center] (text) at (2,-2) {\small $\ftensorof{\exformula}$};
\draw[<-] (0,-3)--(0,-5) node[midway,left] {\tiny $\catvariableof{0}$}; 
\draw[<-] (1.5,-3)--(1.5,-5) node[midway,left] {\tiny $\catvariableof{1}$}; 
\node[anchor=center] (text) at (3,-4) {$\cdots$};
\draw[<-] (4,-3)--(4,-5) node[midway,right] {\tiny $\catvariableof{\atomorder\shortminus1}$}; 

\draw (-1.5,-5) rectangle (5.5,-7);
\node[anchor=center] (text) at (2,-6) {\small $\sbnormationof{\kb}{\shortcatvariables}$};

\node[anchor=center] (text) at (9,-4) {\small ${=}$};
%\draw (9,-3) -- (9,-5);
%\draw (9.15,-3) -- (9.15,-5);

\draw[->] (13,-3)--(13,-1) node[midway,right] {\tiny $\catvariableof{\exformula}$};
\draw (12,-5) rectangle (14,-3);
\node[anchor=center] (text) at (13,-4) {\small $\onehotmapof{1}$};

\node[anchor=center] (text) at (16,-6) {$\cdot$};

\end{tikzpicture}
\end{center}


We can omit the normation by $\sbcontraction{\kb}$ when deciding entailment, as we state next.

\begin{corollary}\label{cor:parallelCriterion}
	Given a satisfiable formula $\kb$, we have $\kb\models\exformula$ (respectively $\kb\models\lnot\exformula$), if and only if 
		\[ \sbcontractionof{\kb,\rencodingof{\exformula}}{\formulavar=0} = 0 
		 \quad \text{( respectively }
		 \sbcontractionof{\kb,\rencodingof{\exformula}}{\formulavar=1} = 0 \, . \]
\end{corollary}




%We will draw on this interpretation in the following, when investigating contraction equation equivalent to entailment.


%
Relating entailment to probability distributions motivates an extension of Definition\ref{def:logicalEntailment} of entailment to arbitrary probability distributions.


\begin{definition}\label{def:probEntailment}
	For any propositional formula $\exformula$ and a probability distribution $\probtensor$ we say that $\probtensor$ probabilistically entails $\exformula$, denoted as $\probtensor\models\exformula$, if
		\[ \sbcontractionof{\probtensor,\rencodingof{\exformula}}{\formulavar=0} = 0 . \]
	If $\probtensor\models\lnot\exformula$ we say that $\probtensor$ probabilistically contradicts $\exformula$.
%	Conversely, we 
%		\[ \sbcontractionof{\probtensor,\rencodingof{\exformula}}{\formulavar=1} = 0 . \]
\end{definition}

%
By \theref{the:probEntailment} the definition of entailment reduces to propositional formulas by choosing $\probtensor=\sbnormationof{\kb}{\shortcatvariables}$









\subsubsection{Knowledge Bases as Base Measures for Probability Distributions}



% Generic Probability Tensors
Let us now relate the probabilistic entailment definition \ref{def:probEntailment} with the logical entailment.
Given a generic probability distribution $\probtensor$ we can build a Knowledge Base by the indicator function of the support as 
	\[ \kb^{\probtensor} = \nonzerofunction \circ \probtensor \]
where $\nonzerofunction:\rr\rightarrow \rr$ is defined as $\nonzeroof{x}=1$ if $x\neq0$ and $\nonzeroof{x}=0$ else.

% Generic case of distributions
\begin{theorem}\label{the:entailmentProbToLogical}
	Any probability distribution $\probtensor$ probabilistically entails a formula $\exformula$, if and only if the Knowledge Base $\kb^{\probtensor}$ logically entails $\exformula$.
\end{theorem}
\begin{proof}
	Whenever $\probtensor$ does not entail $\exformula$ probabilistically we find a state $\shortcatindices\in\atomstates$ such that
		\[ \probat{\shortcatvariables=\shortcatindices} >0 \quad\text{and} \quad \formulaat{\shortcatvariables=\shortcatindices} = 0 \, . \]
	We further have $\probat{\shortcatvariables=\shortcatindices} >0$ if and only if $\kb^{\probtensor}[\shortcatvariables=\shortcatindices]=1$ and
		\[ \big((\kb^{\probtensor}[\indexedcatvariableof{[\catorder]}]=1\big) \rightarrow \big(\formulaat{\indexedcatvariableof{[\catorder]}}=1\big) \, . \]
	is not satisfied.
	Together, $\probtensor\models\exformula$ does not holds if and only if
		\[ \forall \shortcatvariables (\kb^{\probtensor}[\shortcatvariables=\shortcatindices]=1) \rightarrow \big(\formulaat{\indexedcatvariableof{[\catorder]}}=1\big) \,  \]
	is not satisfied. 
	Therefore, probabilistic entailment of $\exformula$ by $\probtensor$ is equivalent to logical entailment of $\exformula$ by $\kb^{\probtensor}$.
\end{proof}

Let us use this to connect the entailment formalism with the representability (see \defref{def:representationBaseMeasure}) and positivity (see \defref{def:positivityBaseMeasure}) of distributions with respect to boolean base measures.

\begin{theorem}\label{the:minimalRepPosBaseMeasure}
	A distribution $\probtensor$ of boolean variables is representable with respect to $\basemeasure$, if and only if $\nonzerofunction\circ\probtensor\models\basemeasure$.
	A distribution $\probtensor$ of boolean variables is positive with respect to $\basemeasure$, if and only if $\basemeasure=\nonzerofunction\circ\probtensor$.
\end{theorem}
\begin{proof}
	To show the first claim, let $\probtensor$ be a distribution and $\basemeasure$ be a base measure.
	With \defref{def:representationBaseMeasure}, $\probtensor$ is representable with respect to $\basemeasure$, if and only if
		\[ \forall_{\shortcatindices\in\atomstates} \big(\basemeasureat{\indexedshortcatvariables}=0\big) \rightarrow \big(\probat{\indexedshortcatvariables}=0\big) \, .  \]
	This is equal to
		\[ \forall_{\shortcatindices\in\atomstates} \big(\nonzerofunction\circ\probat{\indexedshortcatvariables}=1\big) \rightarrow \big(\basemeasureat{\indexedshortcatvariables}=1\big)    \]
	and by definition \defref{def:logicalEntailment} equal to $\basemeasure\models\nonzerofunction\circ\probtensor$.
	
	To show the second claim, we show that when $\probtensor$ is in addition positive with respect to $\basemeasure$, then also $\basemeasure\models\nonzerofunction\circ\probtensor$ and thus $\basemeasure=\nonzerofunction\circ\probtensor$.
	Let $\probtensor$ be a distribution, which is representable with respect to $\basemeasure$.
	Then $\probtensor$ is positive with respect to $\basemeasure$, if and only if
		\[ \forall_{\shortcatindices\in\atomstates} \big(\basemeasureat{\indexedshortcatvariables}=1\big) \rightarrow \big(\probat{\indexedshortcatvariables}>0\big)   \]
	This is equal to
		\[ \forall_{\shortcatindices\in\atomstates} \big(\basemeasureat{\indexedshortcatvariables}=1\big) \rightarrow \big(\nonzerofunction\circ\probat{\indexedshortcatvariables}=1\big)   \]
	and thus $\basemeasure\models\nonzerofunction\circ\probtensor$.
\end{proof}



\subsubsection{Deciding entailment on Markov Networks}



\begin{theorem}\label{the:factorReduction}
	Let $\extnet=\extnetasset$ be a non-negative Tensor Network on a hypergraph $\graph=(\nodes,\edges)$, $\secnodes\subset\nodes$ be a subset and
		\[ \probtensor[\catvariableof{\secnodes}] = \normationof{\{\hypercoreat{\edge} \, : \, \edge\in\edges \}}{\catvariableof{\secnodes}} \]
	and
		\[ \tilde{\probtensor}[\catvariableof{\secnodes}] = \normationof{\{\nonzerofunction \circ \hypercoreat{\edge} \, : \, \edge\in\edges \}}{\catvariableof{\secnodes}} \]
	Then we have for any $\exformula$ that $\probtensor\models\exformula$ if and only if $\tilde{\probtensor}\models\exformula$.
\end{theorem}
\begin{proof}
	We first show
	\begin{align}\label{eq:proofFacReduction}
		 \nonzerofunction\circ\probtensor = \nonzerofunction\circ\tilde{\probtensor} \, . 
	\end{align}
	The claim follows then from \theref{the:entailmentProbToLogical}.
	To show \eqref{eq:proofFacReduction} let there be $\indexedcatvariableof{\secnodes}$ such that $\probtensor[\indexedcatvariableof{\secnodes}]=0$.
	Then for any $\indexedcatvariableof{\nodes}$ extending  $\indexedcatvariableof{\secnodes}$ we have $\contractionof{\{\hypercoreat{\edge} \, : \, \edge\in\edges \}}{\indexedcatvariableof{\nodes}} = 0$ and thus also $\contractionof{\{\nonzerofunction\circ\hypercoreat{\edge} \, : \, \edge\in\edges \}}{\indexedcatvariableof{\nodes}} = 0$ and $\tilde{\probtensor}[\indexedcatvariableof{\secnodes}]=0$.
	One can similarly show, that when $\tilde{\probtensor}[\indexedcatvariableof{\secnodes}]=0$ then also ${\probtensor}[\indexedcatvariableof{\secnodes}]=0$. 
	The support of the distributions $\probtensor$ and $\tilde{\probtensor}$ is thus identical and \eqref{eq:proofFacReduction} holds.
\end{proof}

% Consequence: Reduction of probabilitic entailment to logical entailment.
For any positive tensor $\hypercore$ we have
	\[ \nonzerofunction\circ\hypercoreat{\catvariableof{\edge}} = \onesat{\catvariableof{\edge}} \, , \]
which does not influence the distribution and can be omitted from the Markov Network.
By \theref{the:factorReduction}, when deciding eintailment, we can reduce all tensors of a Markov Network to their support and omit those with full support.
Since the support indicating tensors $\nonzerofunction\circ\hypercoreat{\catvariableof{\edge}}$ are Boolean, each is a propositional formula and the Markov Network is turned into a Knowledge Base of their conjunctions.
Deciding probabilstic entailment is thus traced back to logical entailment.

%\subsubsection{Queries by Formulas}
%We have investigated a specific type of query for the definition of entailment.
%More generally, the semantic of logics thus offer a method to state generic queries on arbitrary probability distributions in an interpretable way.

%%  TO DO: Give examples, e.g. correlations by formulas
%
%% Probabilistic Queries
%Deciding entailment is a specific form of a query:
%\begin{itemize}
%	\item Query function is a formula, the one-hot encoding the formula tensor
%	\item Expectations, which are the output of the query, are interpreted whether they are parallel to $\fbasis$ (contradiction), $\tbasis$ (entailment), both (inconsistent KB) or neither (contingent)
%\end{itemize}
%
%
%%We now apply the developed formalism of formula tensors to design a knowledge base.
%Deciding entailment can be done by efficient tensor network contractions of the knowledge base sentences and the query formula in tensor network representation.








%\subsection{Deciding Entailment by Contractions}
%
%In the next Theorem we show how the normations required in the computation of $\condprobof{\exformula}{\kb=1}$ can be avoided when deciding entailment.
%

%\begin{proof}
%	The claim holds in case of unsatisfiable $\kb$, since any $\exformula$ is entailed and $\sbcontractionof{\kb,\rencodingof{\exformula}}{\formulavar}$ is parallel to both $\tbasis$ and $\fbasis$.
%	Let us thus assume, that $\kb$ is satisfiable, in which case we have
%	\begin{align*}
%		 \sbcontractionof{\kb,\rencodingof{\exformula}}{\formulavar}  
%		 & = \onehotmapof{0} \otimes \sbcontractionof{\kb,{\lnot\exformula}}{\formulavar} 
%		 + \onehotmapof{1} \otimes \sbcontractionof{\kb,{\exformula}}{\formulavar} \\
%		 & = \contractionof{\{\kb\}}{\varnothing} \cdot \left(  
%		 \onehotmapof{0} \otimes \frac{\sbcontractionof{\kb,{\lnot\exformula}}{\formulavar}}{\contractionof{\{\kb\}}{\varnothing}}
%		 + \onehotmapof{1} \otimes \frac{
%		 \sbcontractionof{\kb,{\exformula}}{\formulavar}
%		 }{\contractionof{\{\kb\}}{\varnothing}}
%		 \right) \\
%		 & = \contractionof{\{\kb\}}{\varnothing} \cdot \left(  
%		 	\onehotmapof{0} \otimes \condprobof{\exformula=0}{\kb=1}
%		 	+ \onehotmapof{1} \otimes \condprobof{\exformula=1}{\kb=1}
%		 \right) \\
%	\end{align*}
%	Now, if and only if $\kb\models\exformula$ we have $\condprobof{\exformula=1}{\kb=1}=1$ and
%		\[ \condprobof{\exformula=0}{\kb=1} = 1 - \condprobof{\exformula=1}{\kb=1}=0\]
%	and the term $\onehotmapof{0} \otimes \condprobof{\exformula=0}{\kb=1}$.
%	Exactly in this case, we then have $ \sbcontractionof{\kb,\rencodingof{\exformula}}{\formulavar}  \parallel \tbasis$.
%	By the same argument concerning the term $\onehotmapof{1} \otimes \condprobof{\exformula=1}{\kb=1}$, we get  $\sbcontractionof{\kb,\rencodingof{\exformula}}{\formulavar}\parallel\fbasis$ if and only if $\kb\models\lnot\exformula$.
%\end{proof}











%\subsection{Deciding Entailment by Contractions}
%
%\begin{theorem}
%	Given a Knowledge Base $\kb$ and a formula $\exformula$, we have $\kb\models\exformula$ if and only if
%		\[ \contractionof{\{\kb,\rencodingof{\exformula}\}}{\randomxof{\exformula}}  \parallel \tbasis \, . \]
%\end{theorem}
%\begin{proof}
%	We note that  $\contractionof{\{\kb,\rencodingof{\exformula}\}}{\randomxof{\exformula}}  \parallel \tbasis $ is equal to
%		\[ \contractionof{\{\kb,{\lnot\exformula}\}}{\varnothing}  = 0  \, . \]
%	This is equal to $\kb\land\not\exformula$ being inconsistent and therefore equal to $\kb\models\exformula$.
%\end{proof}












\subsection{Deciding Entailment by partial ordering}

% Classical definition of entailment
Classically entailment in propositional logics is defined by a model-theoretic approach.
According to that approach, the entailment statement $\kb\models\exformula$ holds, whenever any model of $\kb$ is also a model of $\exformula$.
We will in the following show, that this is equal to our definition based on probabilistic queries.


% Here for general tensors, not just propositional formulas!
\begin{definition}[Partial ordering of tensors]\label{def:partialFTOrder}
	We say that two tensors $\exformula$ and $\secexformula$ in a tensor space $\facspace$ are partially ordered, denoted by
		\[ {\exformula}\prec{\secexformula} \, , \]
	if for all $\catindices\in\facstates$
		\[ {\exformula}(\catindices) \leq {\secexformula}(\catindices) \, .\]
\end{definition}

We notice, that whenever ${\exformula} \prec{\secexformula}$ holds, for any model $\atomindices$ of $\exformula$ we have
\begin{align*}
	1 = \exformula(\atomindices) \leq \secexformula(\atomindices)
\end{align*}
and thus $\secexformula(\atomindices)=1$.
Therefore any model of $\exformula$ is also a model of $\secexformula$.
We show in the next theorem, that this is equivalent to the entailment statement $\exformula\models\secexformula$.

\begin{theorem}[Partial Ordering Criterion] \label{the:orderingEntailmentCriterion}
	We have $\kb\models\exformula$ if and only if $\kb\prec\exformula$.
\end{theorem}
\begin{proof}
	Directly by definition, since both $\kb$ and $\exformula$ are Boolean and therefore for any $\shortcatindices\in\atomstates$ we have that
		\[ \kbat{\indexedcatvariableof{[\catorder]}} \leq \formulaat{\indexedcatvariableof{[\catorder]}} \]
	is equivalent to 
		\[ \big(\kbat{\indexedcatvariableof{[\catorder]}}=1\big) \rightarrow \big(\formulaat{\indexedcatvariableof{[\catorder]}}=1\big) \, .  \]
	Therefore, $\kb\prec\exformula$ is equivalent to
		\[ \forall_{\shortcatindices\in\atomstates} \big(\kbat{\indexedcatvariableof{[\catorder]}}=1\big) \rightarrow \big(\formulaat{\indexedcatvariableof{[\catorder]}}=1\big) \, , \]
	which is equal to $\kb\models\exformula$.
%\red{Relate to other Theorem!}
%	By \theref{cor:parallelCriterion} suffices to show that $\contractionof{\{\kb,\rencodingof{\exformula}\}}{\formulavar}  \parallel \tbasis$ is equivalent to $\kb\prec\exformula$.
%	To show this equivalence we observe
%		\[ \sbcontractionof{\kb,\rencodingof{\exformula}}{\formulavar}(0) = 
%		\contractionof{\{\kb,{\lnot\exformula}\}}{\varnothing} =
%		\# \left\{ i \in\facstates : \kb(i)= 1 \land \exformula(i) = 0 \right\} \, . \]
%	If and only if $\contractionof{\{\kb,\rencodingof{\exformula}\}}{\formulavar}  \parallel \tbasis$ we have $\sbcontraction{\kb,\exformula}=0$, which is equivalent to 
%		\[ \forall i \in\facstates : \lnot\kb(i)= 1 \land \exformula(i) = 0)  \, . \]
%	This is further equivalent to 
%		\[ \forall i \in\facstates : \kb(i) = 1 \rightarrow \exformula(i) = 1)  \]
%	and 
%		\[ \kb \prec \exformula \, . \]
\end{proof}

% Semantic Interpretation
%The partial ordering criterion offers a model-theoretic proof of entailment, since partial ordering is defined through comparison of all models:
%We have
%	\[ \exformula \prec \secexformula \]
%if and only if 
%	\[ \forall i\in \facstates : \exformula_i=1 \rightarrow \secexformula_i = 1 \, , \]
%that is any model of $\exformula$ is also a model of $\secexformula$.



% Partial Ordering
%We can therefore understand partial ordering as a generalization of entailment?





\subsubsection{Monotonicity of Entailment}


%\red{When defining entailment based on Markov Networks, would have clearer statement!}
Vanishing local contractions provide sufficient but not necessary criterion to decide entailment, as we show in the next theorem.

\begin{theorem}[Monotonicity of Entailment]\label{the:monotonEntailment}
	For any Markov Network on the decorated hypergraph $\graph$ and any subgraph $\secgraph$, we have for any formula that $\probtensor^{\graph}\models\exformula$ if $\probtensor^{\secgraph}\models\exformula$.
\end{theorem}	
\begin{proof}
	Based on the reduction to Knowledge Bases by \theref{the:entailmentProbToLogical} and the monotonocity of binary contractions as shown in \theref{the:monotonicityBinaryContractions}.
\end{proof}



\begin{remark}
	To make use of \theref{the:monotonEntailment} we can exploit any entailment criterion.
	However, there is no claim about entailment being false, when the entailment 
	\theref{the:monotonEntailment} therefore just provides a sufficient but not necessary criterion of entailment with respect to $\probtensor^{\graph}$.
\end{remark}



\subsection{Deciding Entailment by local contractions}\label{subsec:LocalEntailment}


Global entailment can become inefficient, when
\begin{itemize}
	\item we are interested in batches of entailment checks. Here we can make use of dynamic programming (store partial contraction results in the Knowledge Cores).
	\item the network is large. Although efficient tensor network contraction often work, they might get infeasible when the tensor network has a large connectivity. For many 
\end{itemize}
An alternative to deciding entailment by global operations is the use of local operations.
Here we interpret a part of the network (for example a single core) as an own knowledge base (with atomic formulas being the roots of the directed subgraph, that is potentially differing with the atoms in the global perspective) and perform entailment with respect to that.

\begin{remark}{Tradeoff between generality and efficiency}
	While generic entailment decision algorithms (those by the full network) can decide any entailment, local algorithms as presented here can only perform some, but therefore more effectively as operating batchwise (dynamically deciding entailment for many leg variables).
	This is a typical phenomenon in logical reasoning and related to decidability.
\end{remark}


\subsubsection{Knowledge Propagation}

Let us now draw on these insights and store partial entailment results in Knowledge Cores, which is a use of the dynamic programming paradigm.
We then iterate over local entailment checks, where we recursively add further entailment checks to be redone due to additional knowledge.
We then call the local checks until convergence Entailment Propagation, since different stadia of knowledge are propagated through the network.
We describe local Knowledge Propagation in a generic way in Algorithm~\ref{alg:KP}.

\begin{algorithm}[hbt!]
\caption{Knowledge Propagation (KP)}\label{alg:KP}
\begin{algorithmic}
\State Tensor Network $\extnet$, $\kcoreof{\edge}=\onesat{\catvariableof{\edge}}$
\While{Stopping Criterion is not met}
	\State Choose $\edge$, subset $M$ of $\extnet$ and of $\{\kcoreof{\edge} : \edge\in\edges \}$ containing $\kcoreof{\edge}$
	\State Update 
		\[ \kcoreof{\edge} \leftarrow \nonzerofunction\circ\contractionof{M}{\catvariableof{\edge}} \]
\EndWhile
\end{algorithmic}
\end{algorithm}

% Interpretation
Each chosen subset $M$ is understood as a local knowledge base, which is then applied for local entailment.

%
%The Knowledge Cores 

%Implementation
There are different ways of implementing Algorithm~\ref{alg:KP}, by choosing an order of local knowledge bases $M$ and a stopping criterion.

\begin{theorem}
	In Entailment Propagation Algorithm~\ref{alg:KP}, $\kcoreof{\edge}$ is monotonically decreasing with respect to the partial ordering and greater than $\tilde{\kcoreof{\edge}}$ defined as
			\[ \tilde{\kcoreof{\edge}} = \nonzeroof{\contractionof{\extnet}{\edge}} \, . \]
\end{theorem}
\begin{proof}
	We deduce the theorem from generic properties of the support of contractions, see \secref{sec:supportContractionEquations}.
	Monotonic decreasing follows from montonocity of tensor contractions, see \theref{the:monotonicityBinaryContractions}.
	By \theref{the:invarianceAddingSubcontractions} we have during any state of the algorithm
		\[ \nonzerofunction\circ\contractionof{\extnet}{\catvariableof{\nodes}}  =  
		\nonzerofunction\circ\contractionof{\extnet\cup\{\kcoreof{\edge} : \edge\in\edges\}}{\catvariableof{\nodes}}  \, . 
		\]
	If follows that
		\[ \tilde{\kcoreof{\edge}} =  \nonzerofunction\circ\contractionof{\extnet\cup\{\kcoreof{\edge} : \edge\in\edges\}}{\catvariableof{\edge}} \]
	and by \theref{the:monotonicityBinaryContractions}
		\[  \tilde{\kcoreof{\edge}}  \prec \kcoreof{\edge} \, . \]
\end{proof}


\begin{corollary}
	Whenever for a formula $\formulaat{\catvariableof{\secnodes}}$ and a $\kcoreof{\edge}$ we have
		\[ \contractionof{\kcoreof{\edge},\rencodingof{\exformula}}{\exformulavar=0} =0  \]
	then the Markov Network $\extnet$ probabilistically entails $\exformula$.
\end{corollary}


%% Interpretation for leg dimension two
Another way to use Algorithm~\ref{alg:KP} to decide entailement of formulas $\exformula$ is adding each $\rencodingof{\exformula}$ to $\extnet$ and defining Knowledge Cores $\kcoreof{\exformula}[\exformulavar]$.
Since then the Knowledge Core has only two dimensions, there are only four possible cores with the interpretation
\begin{itemize}
	\item $\tbasis$: the formula is known to be true
	\item $\fbasis$: the formula is known to be false
	\item $\ones$: the formula is not known
	\item $0$: the knowledge base is inconsistent
\end{itemize}


%% OLD Criteria 

%\subsection{Deciding entailment by Global Operations}
%
%\begin{corollary}\label{cor:SatisfiabilityCheck}
%	A Knowledge Base $\kb$ is satisfiable, if and only if
%		\[ \contractionof{\rencodingof{\kb}\cup\tbasis^{\kb}}{[]} \geq 0 \, .\]
%	Here $\tbasis^{\kb}$ denotes the tensor with values $\tbasis$ and leg variable $\kb$.
%\end{corollary}
%\begin{proof}
%	It would not be satisfiable if and only if $\formulaset\models\nothing$ and $\rencodingof{\nothing}=0$.
%\end{proof}
%
%More precisely, $\contractionof{\rencodingof{\kb}}{[]} $ is the count of models.
%
%
%%% Satisfaction Check
%
%\begin{theorem}\label{the:EntailmentCheck}
%	We have $\kb\models\exformula$ if and only if
%		\[ \contractionof{\rencodingof{\kb}\cup\tbasis^{\kb} \cup \rencodingof{\exformula} \cup \fbasis^{\exformula}}{[]} = 0 \, .\]
%\end{theorem}
%\begin{proof}
%	It is known that $\kb\models\exformula$ if and only if $\kb\cup\{\lnot\exformula\}$ unsatisfiable (proof by contradiction).
%	Using Corollary \ref{cor:SatisfiabilityCheck} we have $\kb\cup\{\lnot\exformula\}$ unsatisfiable if and only if
%		\[  \contractionof{\rencodingof{\kb\cup\{\lnot\exformula\}}\cup\tbasis^{\kb\cup\{\lnot\exformula\}}}{[]} = 0 \, .\]
%	The claim thus follows from noticing 
%		\[ \contractionof{\rencodingof{\kb\cup\{\lnot\exformula\}}\cup\tbasis^{\kb\cup\{\lnot\exformula\}}}{[]}  =
%		\contractionof{\rencodingof{\kb}\cup\tbasis^{\kb} \cup \rencodingof{\exformula} \cup \fbasis^{\exformula}}{[]} \, .
%		\]
%\end{proof}


%% Contraction based criterion % But obvious from Monotonocity!! % Best when having a Markov Network definition of entailment.
%\begin{theorem}[Local Contraction Criterion]\label{the:localEntailmentCriterion}
%	For any subset $\seckb$ of the binary tensor cores of $\kb$, we have $\kb\models\exformula$ (respectively $\kb\models\lnot\exformula$), if the contraction of $\seckb$ with leaving $\atomlegindexof{\exformula}$ open is parallel to $\fbasis$ (respectively parallel to $\fbasis)$, denoted by 
%		\[ \contractionof{\rencodingof{\seckb}}{[\atomlegindexof{\exformula}]}  \parallel \tbasis \quad \text{( respectively }\contractionof{\rencodingof{\seckb}}{[\atomlegindexof{\exformula}]}  \parallel \fbasis \text{)} \, . \]
%%	 whenever the contraction of the subset with leaving $\atomlegindexof{\exformula}$ open is parallel to $\fbasis$ (respectively parallel to $\fbasis)$, then $\kb\models\exformula$ (respectively $\kb\models\lnot\exformula)$.
%\end{theorem}
%\begin{proof}
%	From the monotonicity of binary tensor contractions with respect to the partial ordering it follows with \theref{the:monotonicityBinaryContractions} that
%		\[ \rencodingof{\kb} \prec \rencodingof{\seckb} \, . \]
%	By \theref{the:orderingEntailmentCriterion} we further have $\kb\models\seckb$.
%	Theorem~\ref{cor:parallelCriterion} now implies, that $\seckb\models\exformula$ (respectively $\seckb\models\lnot\exformula$), when the in the claim assumed contraction criterion is satisfied.
%	By monotonicity of entailment we in that case further have $\kb\models\exformula$ (respectively $\seckb\models\lnot\exformula$).
%%	For any knowledge base $\seckb$ such that $\kb\models\seckb$ (seen as sublist of its formulas represented by knowledge cores).
%%	Follows from the monotonicity of propositional logic: When entailment by subset, then also entailment by the full, but not the other way around.
%\end{proof}


    \part{\parttwotext}\label{par:two}

    We now employ tensor networks to define architectures and algorithms for neuro-symbolic reasoning based on the logical and probabilistic foundations.
    Markov Logic Networks will be taken as generative models to be learned from data, using formula selecting tensor networks and likelihood optimization algorithms.

    \chapter{\chatextformulaSelection}\label{cha:formulaSelection}

In this chapter we will investigate efficient schemes to represent collections of formulas with similar structure in one tensor network.

% basis encoding of the selection map
\begin{definition}
    Given a set of $\seldim$ formulas $\{\formulaof{\selindex} : \selindexin\}$, the formula selecting map is the map
    \begin{align*}
        \fselectionmap : \atomstates \rightarrow \bigtimes_{\selindex\in\seldim} [2]
    \end{align*}
    defined for $\shortcatindices\in\atomstates$ as
    \begin{align*}
        \fselectionmapat{\shortcatindices}
        = \bigtimes_{\selindex\in\seldim} \formulaofat{\selindex}{\shortcatindices} \, .
    \end{align*}
\end{definition}

% Selection Variables
A tensor representation of a formula selecting map is provided by the selection encoding (see \defref{def:selectionEncoding})
\begin{align*}
    \sencfselectionmapat{\shortcatvariables,\selvariable}
\end{align*}
where the selection variable $\selvariable$ takes values in $[\seldim]$ and selects specific formulas in the set $\{\formulaof{\selindex} : \selindexin\}$.
By definition, we have for any $\shortcatindices\in\atomstates$ and $\selindexin$
\begin{align*}
    \sencfselectionmapat{\indexedshortcatvariables,\indexedselvariable}
    =  \formulaofat{\selindex}{\shortcatvariables=\atomindices} \, .
\end{align*}
% Decomposition
This selection encoding is thus the sum
\begin{align*}
    \sencfselectionmapat{\shortcatvariables,\selvariable}
    = \sum_{\selindexin} \formulaofat{\selindex}{\shortcatvariables}
    \otimes \onehotmapofat{\selindex}{\selvariable} \, .
\end{align*}
Such a representation scheme requires linear resources in the number of formulas.
We will show in the following, that we can exploit common structure in formulas to drastically reduce this resource consumption.
Central to these sparse representation scheme are basis encodings $\bencodingof{\fselectionmap}$ of the selection encodings $\sencodingof{\fselectionmap}$, which we depict in \figref{fig:formulaSelectionMap}.

% Depiction
\begin{figure}[h]
    \begin{center}
        \begin{tikzpicture}[scale=0.35, thick] % , baseline = -3.5pt

    \begin{scope}
        [shift={(-20,0)}]
        \node[anchor=center] (text) at (-1,3) {${a)}$};

        \node [circle, draw, thick, fill=gray!50, minimum size = \nodeminsize] (T1) at (0,0) {\tiny $\catvariableof{0}$};
        \node [circle, draw, thick, fill=gray!50, minimum size = \nodeminsize] (T2) at (3,0) {\tiny $\catvariableof{1}$};
        \node[anchor=center] (text) at (6,0) {${\cdots}$};
        \node [circle, draw, thick, fill=gray!50, minimum size = \nodeminsize] (T3) at (9,0) {};
        \node[anchor=center] (text) at (9,0) {\tiny $\catvariableof{\seldim\shortminus1}$};

        \node [circle, draw, thick, fill=gray!50, minimum size = \nodeminsize] (T4) at (12,3) {};
        \node[anchor=center] (text) at (12,3) {\tiny $\selvariable$};

        \draw[->-] (6,3) -- (6,6);

        \node [circle, draw, thick, fill=gray!50, minimum size = \nodeminsize] (S) at (6,6) {};
        \node[anchor=center] (text) at (6,6) {\tiny $\headvariableof{\fselectionmap}$};

        \draw[->-] (T1) -- (6,3);
        \draw[->-] (T2) -- (6,3);
        \draw[->-] (T3) -- (6,3);
        \draw[->-] (T4) -- (6,3);

    \end{scope}


    \node[anchor=center] (text) at (-1,3) {${b)}$};


    \begin{scope}
        [shift={(0,-2)}]
        \draw[-<-] (0,1)--(0,-1) node[midway,left] {\tiny $\catvariableof{0}$};
        \draw[-<-] (1.5,1)--(1.5,-1) node[midway,left] {\tiny $\catvariableof{1}$};
        \node[anchor=center] (text) at (3,0) {$\cdots$};
        \draw[-<-] (4,1)--(4,-1) node[midway,right] {\tiny $\catvariableof{\seldim\shortminus1}$};
    \end{scope}

    \draw (-1,1) rectangle (5,-1);
    \node[anchor=center] (text) at (2,0) {$\bencodingof{\fselectionmap}$};
    \draw[->-] (2,1) -- (2,3) node[midway, right]  {\tiny $\headvariableof{\fselectionmap}$};
    \draw[-<-] (5,0) -- (7,0) node[midway, above] {\tiny $\selvariable$};


\end{tikzpicture}
    \end{center}
    \caption{Representation of the basis encoding $\bencodingof{\fselectionmap}$ to a selection encoded formula selecting map as an
    a) Factor of a Graphical Model with a selection variable $\selvariable$ and a computed variable $\headvariableof{\fselectionmap}$.
    b) Decorating Tensor Core with selection variable corresponding with an additional axis.}
    \label{fig:formulaSelectionMap}
\end{figure}


\sect{Construction schemes}

% Naturality of folding
Let us now investigate efficient schemes to define sets of formulas to be used in the definition of $\fselectionmap$.
We will motivate the folding of the selection variable into multiple selection variables by compositions of selection maps.


\subsect{Connective Selecting Maps} % ! This is more or less the same as the general formula selecting map

We represent choices over connectives with a fixed number of arguments by adding a selection variable to the cores and defining each slice by a candidate connective.

% Formal map
\begin{definition}
    \label{def:connectiveSelector}
    Let $\{\connectiveof{0},\ldots,\connectiveof{\seldimof{\cselectionsymbol}-1}\}$ be a set of connectives with $\atomorder$ arguments.
    The associated connective selection map is
    \begin{align*}
        \cselectionmap : \atomstates \rightarrow \bigtimes_{\selindex\in[\seldimof{\cselectionsymbol}]}[2]
    \end{align*}
    defined for $\shortcatindices\in\atomstates$ as
    \begin{align*}
        \cselectionmapat{\shortcatindices} = \bigtimes_{\selindex\in[\seldimof{\cselectionsymbol}]} \connectiveofat{\selindex}{\shortcatvariables=\shortcatindices} \, .
    \end{align*}
%    \[ \cselectionmapat{\shortcatvariables,\selvariableof{\cselectionsymbol}}
%    : \atomstates \times [\seldimof{\cselectionsymbol}] \rightarrow [2] \]
%    defined for each $\selindexofin{\cselectionsymbol}$ and $\shortcatindices\in\atomstates$ by
%    \[ \cselectionmapat{\shortcatvariables=\shortcatindices,\indexedselvariableof{\cselectionsymbol}}
%    = \connectiveofat{\selindexof{\cselectionsymbol}}{\shortcatvariables=\shortcatindices}  \, . \]
\end{definition}

%We depict the basis encoding of connective selection maps in \figref{fig:connectiveSelector}.
%
%\begin{figure}[h] % Already drawn
%    \begin{center}
%        \input{PartII/tikz_pics/formula_selection/connective_selector.tex}
%    \end{center}
%    \caption{Connective Selector.}
%    \label{fig:connectiveSelector}
%\end{figure}


\subsect{Variable Selecting Maps}

Choices of connectives can be combined with selections of variables assigned building the arguments of a connective.
To this end, we introduce variable selecting maps.

%% Definition
\begin{definition}
    \label{def:variableSelector}
    The selection of one out of $\seldim$ variables in a list $\catvariableof{[\seldim]}$ is done by variable selecting maps
    \begin{align}
        \vselectionmap : \bigtimes_{\selindex\in[\seldim]}[2] \rightarrow \bigtimes_{\selindex\in[\seldim]}[2] \, .
        % \vselectionmapat{\catvariableof{[\seldim]},\selvariableof{\vselectionsymbol}}:  \left(\bigtimes_{\selindex\in[\seldim]}[2]\right) \times [\seldim]  \rightarrow [2]
    \end{align}
    defined as the identity map
    \begin{align*}
        \vselectionmapat{\catvariableof{[\seldim]}} = \catvariableof{[\seldim]} \, .
    \end{align*}
%    are defined coordinatewise by
%    \begin{align}
%        \vselectionmapat{\indexedcatvariableof{0},\ldots,\indexedcatvariableof{\seldim-1},\indexedselvariableof{\vselectionsymbol}} = \catindexof{\selindex} \, .
%    \end{align}
\end{definition}

The selection encoding of the variable selecting map is the tensor $\sencvselectionmapat{\catvariableof{[\seldim]},\selvariableof{\vselectionsymbol}}$
\begin{align*}
    \sencvselectionmapat{\indexedcatvariableof{0},\ldots,\indexedcatvariableof{\seldim-1},\indexedselvariableof{\vselectionsymbol}} = \catindexof{\selindexof{\vselectionsymbol}} \, .
\end{align*}


% Interpretation as multiplex gate
Selection encodings of variable selecting maps appear in the literature as multiplex gates (see e.g. Definition 5.3 in \cite{koller_probabilistic_2009}).

The basis encoding of the variable selection map has a decomposition
\begin{align*}
    \bencodingofat{\vselectionmap}{\vselectionheadvar,\catvariableof{[\seldimof{\vselectionsymbol}]}}
    = \sum_{\selindexofin{\vselectionsymbol}}
    \bencodingofat{\atomicformulaof{\selindexof{\vselectionsymbol}}}{\vselectionheadvar,\catvariableof{\selindexof{\vselectionsymbol}}} \otimes  \onehotmapofat{\selindexof{\vselectionsymbol}}{\selvariableof{\vselectionsymbol}} \, .
\end{align*}
This structure is exploited in the next theorem to derive a tensor network decomposition of $\bencodingof{\vselectionmap}$.

\begin{theorem}[Decomposition of Variable Selecting Maps]
    \label{the:varSelectorDecomposition}
    Given a list $\catvariableof{[\seldimof{\vselectionsymbol}]}$ of variables, we define for each $\selindexofin{\vselectionsymbol}$ the tensors
    \begin{align*}
        \selectorcomponentofat{\selindexof{\vselectionsymbol}}{\catvariableof{\selindexof{\vselectionsymbol}},\selvariableof{\vselectionsymbol}}
        = \identityat{\vselectionheadvar,\catvariableof{\selindexof{\vselectionsymbol}}} \otimes \onehotmapofat{\selindexof{\vselectionsymbol}}{\selvariableof{\vselectionsymbol}}
        + \onesat{\vselectionheadvar,\catvariableof{\selindexof{\vselectionsymbol}}}
        \otimes \left(\onesat{\selvariableof{\vselectionsymbol}} - \onehotmapofat{\selindexof{\vselectionsymbol}}{\selvariableof{\vselectionsymbol}} \right) \, .
    \end{align*}
    Then we have (see Figure~\ref{fig:SelectorDecomposition})
    \[ \bencodingofat{\vselectionmap}{\vselectionheadvar,\catvariableof{[\seldim]},\selvariableof{\vselectionsymbol}}
    = \contractionof{
        \{\selectorcomponentofat{\selindexof{\vselectionsymbol}}{\vselectionheadvar,\catvariableof{\selindexof{\vselectionsymbol}},\selvariableof{\vselectionsymbol}} \, : \, \selindexofin{\vselectionsymbol}\}
    }{\vselectionheadvar,\catvariableof{[\seldim]},\selvariableof{\vselectionsymbol}} \, .
    \]
\end{theorem}
\begin{proof}
    We show the equivalence of the tensors on an arbitrary coordinates.
    For $\tilde{\selindex}_{\vselectionsymbol}\in[\seldimof{\vselectionsymbol}]$, $\vselectionheadvar\in[2]$ and $\catindexof{[\seldimof{\vselectionsymbol}]}\in\bigtimes_{\catenumerator\in[\seldimof{\vselectionsymbol}]}[2]$ we have
    \begin{align*}
        & \contractionof{
            \{\selectorcomponentofat{\selindexof{\vselectionsymbol}}{\vselectionheadvar,\catvariableof{\selindexof{\vselectionsymbol}},\selvariableof{\vselectionsymbol}} \, : \, \selindexofin{\vselectionsymbol}\}
        }{\indexedheadvariableof{\vselectionsymbol},\indexedcatvariableof{[\seldim]},\selvariableof{\vselectionsymbol} = \tilde{\selindex}_{\vselectionsymbol}} \\
        & \quad =
        \prod_{\selindexofin{\vselectionsymbol}} \selectorcomponentofat{\selindexof{\vselectionsymbol}}{
            \indexedheadvariableof{\vselectionsymbol},\indexedcatvariableof{\selindexof{\vselectionsymbol}},\selvariableof{\vselectionsymbol}=\tilde{\selindex}_{\vselectionsymbol}
        } \\
        & \quad = \selectorcomponentofat{\tilde{\selindex}_{\vselectionsymbol}}{
            \indexedheadvariableof{\vselectionsymbol},\indexedcatvariableof{\selindexof{\vselectionsymbol}},\selvariableof{\vselectionsymbol}=\tilde{\selindex}_{\vselectionsymbol}
        } \\
        & \quad =
        \begin{cases}
            1 & \text{if} \quad \headindexof{\vselectionsymbol} = \catindexof{\selindexof{\vselectionsymbol}} \\
            0 & \text{else}
        \end{cases} \\
        & = \bencodingofat{\vselectionmap}{\indexedheadvariableof{\vselectionsymbol},\indexedcatvariableof{[\seldim]},\selvariableof{\vselectionsymbol}=\tilde{\selindex}_{\vselectionsymbol}}
    \end{align*}
    In the second equality, we used that the tensor $\selectorcomponentof{\selindexof{\vselectionsymbol}}$ have coordinates $1$ whenever $\tilde{\selindex}_{\vselectionsymbol}\neq\selindexof{\vselectionsymbol}$.
\end{proof}

The decomposition provided by \theref{the:varSelectorDecomposition} is in a $\cpformat$ format (see \charef{cha:sparseCalculus}).
The introduced tensors $\selectorcomponentof{\selindexof{\vselectionsymbol}}$ are Boolean, but not directed and therefore basis encodings of relations but not functions (see \charef{cha:basisCalculus}).

%% Interpretation
%The selectorcores $\selectorcoreof{\selindexof{1}}$ are contracted with the parameter cores and select the respective atom when contracted with truth vector tensormultiplied by constant cores (as placeholder for the other possible atoms).
%Decomposed into disconnected strands for each atomkey, which connect on the selection axis and on the atom truth axis.

\begin{figure}[h]
    \begin{center}
        \input{PartII/tikz_pics/formula_selection/variable_selector.tex}
    \end{center}
    \caption{Decomposition of the basis encoding of a variable selecting tensor into a network of tensors defined in \theref{the:varSelectorDecomposition}.
    The decomposition is in a $\cpformat$ format (see \charef{cha:sparseCalculus}). %, when grouping the indices  $\selindexof{\selenumerator}$ and $\atomlegindexof{\atomicformulaof{\selindexof{\selenumerator}}}$).
    %To ease the notation, we here use $\bencodingof{\selenumerator}$ to denote $\bencodingof{\bencodingof{\selenumerator}}$.
    }
    \label{fig:SelectorDecomposition}
\end{figure}




\sect{State Selecting Maps}

When the variables to be selected are the atomization variables to the same categorical variable (see \secref{sec:categoricalTN}), one can avoid the instantiation of all atomization cores and instead represent the variable selecting map using the categorical variable only.
To show this we introduce the state selecting map to a categorical variable $\catvariable$.

\begin{definition}
    \label{def:stateSelector}
    Given a categorical variable $\catvariableof{\sselectionsymbol}$ with dimension $\catdimof{\sselectionsymbol}$ and a selection variable $\selvariableof{\sselectionsymbol}$ with dimension $\seldimof{\sselectionsymbol}=\catdimof{\sselectionsymbol}$ the state selecting map is the map
    \begin{align*}
        \sselectionmap : [\catdimof{\sselectionsymbol}] \rightarrow \bigtimes_{\catenumerator\in[\catdimof{\sselectionsymbol}]} [2]
    \end{align*}
    defined for $\catindexofin{\sselectionsymbol}$ as
    \begin{align*}
        \sselectionmapat{\catindexof{\sselectionsymbol}} = \onehotmapofat{\catindexof{\sselectionsymbol}}{\selvariableof{\sselectionsymbol}} \, .
    \end{align*}
%    \[ \sselectionmapat{\catvariableof{\sselectionsymbol},\selvariableof{\sselectionsymbol}} : [\catdimof{\sselectionsymbol}] \times [\seldimof{\sselectionsymbol}] \rightarrow [2] \]
%    is defined on $\catindexofin{\sselectionsymbol}$ and $\selindexofin{\sselectionsymbol}$ by
%    \begin{align*}
%        \sselectionmapat{\indexedcatvariableof{},\indexedselvariableof{\sselectionsymbol}} =
%        \begin{cases}
%            1 & \text{if} \quad \catindex = \selindexof{\sselectionsymbol} \\
%            0 & \text{else}
%        \end{cases} \, .
%    \end{align*}
\end{definition}

The selection encoding of the state selecting map coincides with the dirac delta tensor, that is for $\catindexofin{\sselectionsymbol}$ and $\selindexofin{\sselectionsymbol}$ we have
\begin{align*}
    \sencsselectionmapat{\catvariableof{\sselectionsymbol},\selvariableof{\sselectionsymbol}} =
    \begin{cases}
        1 & \text{if} \quad \catindex = \selindexof{\sselectionsymbol} \\
        0 & \text{else}
    \end{cases} \, .
\end{align*}

The relation of the variable selecting map and the state selecting map is shown in the next lemma.

\begin{lemma}
    \label{lem:stateSelectorVsVariableSelector}
    Let $\catvariableof{[\seldim]}$ be a collection of atomization variables to a categorical variable $\catvariable$ taking values in $[\seldim]$.
    Then we have for
    \begin{align*}
        \contractionof{\bencodingofat{\categoricalmap}{\catvariableof{[\seldim]},\catvariable},
            \sencvselectionmapat{\catvariableof{[\seldim]},\selvariable}
        }{\catvariable,\selvariable}
        =
        \sencsselectionmapat{\catvariable,\selvariable} \, .
    \end{align*}
\end{lemma}
\begin{proof}
    For each $\catindex,\selindexin$ we have that
    \begin{align*}
        \contractionof{\bencodingofat{\categoricalmap}{\catvariableof{[\seldim]},\catvariable},
            \sencvselectionmapat{\catvariableof{[\seldim]},\selvariable}
        }{\indexedcatvariable,\indexedselvariable}
        &=
        \contraction{\bencodingofat{\categoricalmap}{\catvariableof{[\seldim]},\indexedcatvariable},
            \sencvselectionmapat{\catvariableof{[\seldim]},\indexedselvariable}
        } \\
        &= \begin{cases}
               1 & \text{if} \quad \catindex = \selindex \\
               0 & \text{else}
        \end{cases}
    \end{align*}
    which is thus equal to $\sencsselectionmapat{\indexedcatvariable,\indexedselvariable}$.
\end{proof}

\lemref{lem:stateSelectorVsVariableSelector} shows that when the variables to be selected are the atomization variables to a categorical variable, the state selecting map can thus be used instead of the variable selecting map.
The state selecting map has the advantage, that the instantiation of the tensor $\bencodingofat{\categoricalmap}{\catvariableof{[\seldim]},\catvariable}$ enforcing the categorical constraint can be avoided.

% Comment: Alternative based on categorical constraints to be introduced later
%State selecting tensors can also be realized by variable selecting tensors.
%In \secref{sec:categoricalTN} we have described methods to build atomic variables indicating the states of a categorical variable.
%This would, however, increase the number of variables in a tensor network and can thus lead to an exponential overhead of dimensions.
%State selecting tensors can therefore be seen as a mean to avoid such dimension increases.

\sect{Composition of formula selecting maps}
%\sect{Folding of the Selection Variable}

We will now parametrize the sets $\formulaset$ with additional indices and define formula selector maps subsuming all formulas.
To handle large sets of formulas, we further fold the selection variable into tuples of selection variables.

\begin{definition}
    \label{def:formulaSelector}
    Let there be a formula $\formulaof{\selindexlist}$ for each index tuple in $\selindexlist\in\selstates$, where $\selorder,\seldimof{0},\ldots,\seldimof{\selorder-1}\in\nn$.
    The folded formula selection map (see \figref{fig:foldedSelector}) is the map
    \begin{align*}
        \fselectionmap : \atomstates \rightarrow \bigtimes_{\selindexofin{0}} \cdots \bigtimes_{\selindexofin{\selorder-1}} [2]
    \end{align*}
    defined as
    \begin{align*}
        \fselectionmapat{\shortcatindices} = \left( \formulaofat{\shortselindices}{\shortcatvariables=\shortcatindices} \right)_{\seldimof{0},\ldots,\seldimof{\selorder-1}\in\selstates} \, .
    \end{align*}
%
%    \[ \fselectionmapat{\shortcatvariables,\shortselvariables} : \left(\atomstates\right) \times \left(\selstates\right) \rightarrow [2] \]
%    with the coordinates at the indices $\shortcatindices\in\atomstates$, $\shortselindices\in\selstates$
%    \[  \fselectionmapat{\shortcatvariables=\shortcatindices,\shortselvariables=\shortselindices}
%    = \formulaofat{\shortselindices}{\shortcatvariables=\shortcatindices} \, . \]
\end{definition}

Folding the selection variable into multiple selection variables is especially useful to find efficient decomposition schemes of the formula selecting maps.
In the reminder of this section we will provide an example, where each selection variable is constructed to parameterize a local change to the formula with the result that the basis encoding of the global formula selecting map decomposes into local formula selecting maps.


\begin{figure}[h]
    \begin{center}
        \input{PartII/tikz_pics/formula_selection/folded_selector.tex}
    \end{center}
    \caption{Basis encoding of the folded map $\fselectionmap$.}
    \label{fig:foldedSelector}
\end{figure}




\subsect{Formula Selecting Neuron}


% Motivating foldings by composition
The folding of the selection variable is motivated by the composition of selection maps.
We call the composition of a connective selection (see \defref{def:connectiveSelector}) with variable selection maps (see \defref{def:variableSelector}) for each argument a formula selecting neuron.


\begin{definition}
    \label{def:fsNeuron}
    Given an order $\selorder\in\nn$ let there be a connective selector $\selvariable_{\exconnective}$ selecting connectives of order $\selorder$ and let $\vselectionmapof{0},\ldots,\vselectionmapof{\selorder-1}$ be a collection of variable selectors.
    The corresponding logical neuron is the map
    \begin{align*}
        \lneuron : \atomstates \rightarrow \bigtimes_{\selindexofin{\cselectionsymbol}} \bigtimes_{\selindexofin{0}} \cdots \bigtimes_{\selindexofin{\selorder-1}} [2]
    \end{align*}
    defined for $\shortcatindices\in\atomstates$ by
    \begin{align*}
        \lneuronat{\shortcatindices}
        = \left( \connectiveof{\selindexof{\cselectionsymbol}(\catvariableof{\selindexof{0}},\ldots,\catvariableof{\selindexof{\selorder-1}})}\right)_{\selindexofin{\cselectionsymbol},\selindexofin{0},\ldots,\selindexofin{\selorder-1}}
    \end{align*}
%    \begin{align*}
%        \lneuronat{\shortcatvariablelist,\shortselvariablelist}
%        : \left(\atomstates\right) \times [\seldimof{\cselectionsymbol}] \times \left( \bigtimes_{\selenumeratorin} [\seldimof{\selenumerator}]\right) \rightarrow [2]
%    \end{align*}
%    defined for $\shortcatindices\in\atomstates$, $\selindexof{\cselectionsymbol}\in[\seldimof{\cselectionsymbol}]$ and
%    $\selindices\in \bigtimes_{\selenumeratorin} [\seldimof{\selenumerator}]$ by
%    \begin{align*}
%        \lneuron(\atomindices, \selindexof{\cselectionsymbol}, \selindices) =
%        \cselectionmap(\vselectionmapof{0}(\atomindices, \selindexof{0}),\ldots,\vselectionmapof{\selorder-1}(\atomindices,\selindexof{\selorder-1}), \selindexof{\cselectionsymbol}) \, .
%    \end{align*}
\end{definition}

% Tensor Network Decomposition
Each neuron has a tensor network decomposition by a connective selector tensor and a variable selector tensor network for each argument, as we state in the next theorem.

\begin{theorem}{Decomposition of formula selecting neurons}
    \label{the:neuronDecomposition}
    Let $\lneuron$ a logical neuron, defined for a connective selector $\selvariable_{\exconnective}$ and variable selectors $\vselectionmapof{0},\ldots,\vselectionmapof{\selorder-1}$.
    Then we have (see Figure~\ref{fig:neuronDecomposition} for the example of $\selorder=2$):
    \begin{align*}
        &\bencodingofat{\lneuron}{\headvariableof{\lneuron},\shortcatvariables,\selvariableof{\cselectionsymbol},\selvariableof{\vselectionsymbol,0},\ldots,\selvariableof{\vselectionsymbol,\selorder-1}} \\
        &\quad = \langle\{\bencodingofat{\cselectionmap}{
            \headvariableof{\lneuron},\headvariableof{\vselectionsymbol,0},\ldots,\headvariableof{\vselectionsymbol,\selorder-1}}, \\
        & \quad\quad\quad\bencodingofat{\vselectionmapof{0}}{
            \headvariableof{\vselectionsymbol,0},\shortcatvariables,\selvariableof{\vselectionsymbol,0}},\ldots,
        \bencodingofat{\vselectionmapof{\selorder-1}}{
            \headvariableof{\vselectionsymbol,\selorder-1},\shortcatvariables,\selvariableof{\vselectionsymbol,\selorder-1}}
        \} \rangle
        \left[\headvariableof{\lneuron},\shortcatvariables, \selvariableof{\cselectionsymbol},\selvariableof{\vselectionsymbol,0},\ldots,\selvariableof{\vselectionsymbol,\selorder-1}\right] \, .
    \end{align*}
\end{theorem}
\begin{proof}
    By composition \theref{the:compositionByContraction}.
\end{proof}


%% Example of a FSNN: A skeleton expression, where only the atoms are varied.
%Given a skeleton expression and a set of candidates at each placeholder, we parameterize a set of formulas by the assignment of candidate atoms to each placeholder position.
%Let us denote the set of formulas, which are generated through choosing atoms from $\candidatelistof{\selenumerator}$ for the skeleton formula $\skeleton$ by
%		\[ \formulasetof{\skeleton} \coloneqq
%	 \left\{ \skeletonof{\placeholderof{1},\ldots,\placeholderof{\atomorder}} \, : \, \placeholderof{\atomenumerator} \in \candidatelistof{\atomenumerator} \right\} \]

%We now enumerate at each position $\selenumerator$ the list of candidates $\candidatelistof{\selenumerator}$ using an index $\selindexof{\selenumerator}\in[\seldimof{\selenumerator}]$ and parametrize the choice of the $\selindexof{\selenumerator}$ for the placeholder $\placeholderof{\selenumerator}$ by unit vectors
%	\[ \unitvectoratof{\selenumerator}{\selindexof{\selenumerator}} \in \rr^{\seldimof{\selenumerator}} \, . \]
%We thus have a parameter space $\rr^{\seldim}$ parametrizing the possible assignments to the skeleton in its basis vectors.

\begin{figure}[h]
    \begin{center}
        \input{PartII/tikz_pics/formula_selection/logical_neuron.tex}
    \end{center}
    \caption{Example of a logical neuron $\lneuron$ of order $\selorder=2$.
    a) Selection and categorical variables and their interdependencies visualized in a hypergraph.
    b) Basis encoding of the logical neuron and tensor network decomposition into variable selecting and connective selecting tensors.
    }
    \label{fig:neuronDecomposition}
\end{figure}


\subsect{Formula Selecting Neural Network}

% Enhancement of the Expressivity
Single neurons have a limited expressivity, since for each choice of the selection variables they can just express single connectives acting on atomic variables.
The expressivity is extended to all propositional formulas, when allowing for networks of neurons, which can select each others as input arguments.


\begin{definition}
    \label{def:fsNeuralNetwork}

%	We call a graph consistent of nodes decorated by formula selecting neurons and directed edges representing the argument dependencies of the neuron on other neurons, an architecture graph.
%	An acyclic architecture graph is called a formula selecting neural network.	
%	Formula selecting neurons, which are not included by other formula selecting neurons are called output neurons and collected in the variables $\catvariableof{\larchitecture}$. 
%	A logical neural network is a collection of logical neurons, such that the network graph (nodes: neurons, edges: directed representing argument dependencies) is acyclic (a DAG).

    An architecture graph $\graphof{\larchitecture}=(\nodesof{\larchitecture},\edgesof{\larchitecture})$ is an acyclic directed hypergraph with nodes appearing at most once as outgoing nodes.
    Nodes appearing only as outgoing nodes are input neurons and are labeled by $\inneuronset$ and nodes not appearing as outgoing nodes are the output neurons in the set $\outneuronset$ (see Figure~\ref{fig:architectureGraph} for an example).

    Given an architecture graph $\graphof{\larchitecture}=(\nodesof{\larchitecture},\edgesof{\larchitecture})$, a \emph{formula selecting neural network} $\fsnn$ is a tensor network of logical neurons at each $\lneuron\in\nodesof{\larchitecture}/\inneuronset$, such that each neuron depends on variables $\catvariableof{\parentsof{\lneuron}}$ and on selection variables $\selvariableof{\lneuron}$.
    The collection of all selection variable is notated by $\selvariableof{\larchitecture}$.

    The activation tensor of each neuron $\lneuron\in\nodesof{\larchitecture}/\inneuronset$ is
    \begin{align*}
        \lneuractivationat{\catvariableof{\inneuronset},\selvariableof{\larchitecture}}
        = \contractionof{
            \{\bencodingof{\tilde{\lneuron}} \, : \, \tilde{\lneuron}\in\nodesof{\larchitecture}/\inneuronset \} \cup \{\onehotmapofat{1}{\headvariableof{\lneuron}}\}
        }{\catvariableof{\inneuronset},\selvariableof{\larchitecture}} \, .
    \end{align*}

    The activation tensor of the formula selecting neural network is the contraction
    \begin{align*}
        \fsnnat{\catvariableof{\inneuronset},\selvariableof{\larchitecture}}
        = \contractionof{
            \{\bencodingofat{\lneuractivation}{\headvariableof{\lneuron},\catvariableof{\parentsof{\lneuron}},\selvariableof{\larchitecture}} \, : \, \lneuron\in\nodesof{\larchitecture}/\inneuronset \} \cup \{\onehotmapofat{1}{\headvariableof{\lneuron}} \, : \, \lneuron\in\outneuronset\}
        }{\catvariableof{\inneuronset},\selvariableof{\larchitecture}} \, .
    \end{align*}

    The expressivity of a formula selecting neural network $\fsnn$ is the formula set
    \begin{align*}
        \formulasetof{\larchitecture} = \left\{ \fsnnat{\catvariableof{\inneuronset},\indexedselvariableof{\larchitecture}}  : \selindexof{\larchitecture}\in\selstates \right\} \, .
    \end{align*}

\end{definition}

% ? Extend by activation cone stuff
The activation tensor of each neuron depends in general on the activation tensor of its ancestor neurons with respect to the directed graph $\graphof{\larchitecture}$, and thus inherits the selection variables.

% Architecture graph -> Tensor Network
We notice that the architecture graph is a scheme to construct the variable dependency graph of the tensor network $\formulasetof{\larchitecture}$.
To this end, we replace each neuron $\lneuron\in\nodesof{\larchitecture}/\inneuronset$ by an output variable $\headvariableof{\lneuron}$ and further add selection variables $\selvariableof{\lneuron}$ to the directed edges, that is to each directed hyperedge $(\{\lneuron\}, \parentsof{\lneuron})\in\edgesof{\larchitecture}$ we construct a directed hyperedge $(\{\headvariableof{\lneuron}\}, \catvariableof{\parentsof{\lneuron}}\cup\selvariableof{\lneuron})$.

\begin{figure}[h]
    \begin{center}
        \begin{tikzpicture}[scale=0.45, yscale=1.2, thick] % , baseline = -3.5pt


	\node [circle, draw, thick, fill=gray!50] (T1) at (0,0) {\tiny $\lneuronof{0}$};
	\node [circle, draw, thick, fill=gray!50] (T2) at (3,0) {\tiny $\lneuronof{1}$};
	\node [circle, draw, thick, fill=gray!50] (T3) at (6,0) {\tiny $\lneuronof{2}$};
	\node [circle, draw, thick, fill=gray!50] (T4) at (9,0) {\tiny $\lneuronof{3}$};
	
	\node [circle, draw, thick, fill=gray!50] (ph) at (1.5,3.5) {\tiny $\lneuronof{4}$};
	\node [circle, draw, thick, fill=gray!50] (ph2) at (6,2.5) {\tiny  $\lneuronof{5}$};	
	
	\node [circle, draw, thick, fill=gray!50] (head) at (3.25,6) {\tiny $\lneuronof{6}$};
	\node [circle, draw, thick, fill=gray!50] (head2) at (0,6) {\tiny $\lneuronof{7}$};

	\draw[->-] (ph) -- (head2);

	\coordinate (S3) at (3.25,4.5);
	\draw[->-] (S3) -- (head);
	%\draw [->-] (sel3) -- (S3);
	\draw [->-] (ph2) -- (S3);
	\draw [->-] (ph) -- (S3);
	
	\coordinate (S1) at (1.5,1.5);
	\draw[->-] (S1) -- (ph);
	\draw [->-] (T1) -- (S1);
	\draw [->-] (T2) -- (S1);
	\draw [->-] (T3) -- (S1);
	\draw [->-] (ph2) -- (S1);
	
	\coordinate (S2) at (6,1.5);
	\draw[->-] (S2) -- (ph2);
	%\draw [->-] (sel) -- (S1);
	%\draw [->-] (sel2) -- (S2);

	\draw [->-] (T2) -- (S2);
	\draw [->-] (T3) -- (S2);

	\draw [->-] (T4) -- (S2);



\end{tikzpicture}
    \end{center}
    \caption{Example of an architecture graph $\graphof{\larchitecture}$ with input neurons $\inneuronset=\{\lneuronof{0},\lneuronof{1},\lneuronof{2},\lneuronof{3}\}$ and output neurons $\outneuronset=\{\lneuronof{6},\lneuronof{7}\}$
    }
    \label{fig:architectureGraph}
\end{figure}


\begin{theorem}
    Given fixed selection variables $\selvariableof{\larchitecture}$, the formula selecting neural network is the conjunction of output neurons, that is
    \begin{align*}
        \fsnnat{\catvariableof{\inneuronset},\selvariableof{\larchitecture}} %= \bigwedge_{\lneuron\in\outneuronset} \lneuronat{\catvariableof{\inneuronset},\selvariableof{\larchitecture}}
        = \contractionof{\left\{\lneuractivationat{\catvariableof{\inneuronset},\selvariableof{\larchitecture}} : \lneuron\in\outneuronset\right\}}{\catvariableof{\inneuronset},\selvariableof{\larchitecture}} \, .
    \end{align*}
\end{theorem}
\begin{proof}
    By effective calculus (see \theref{the:effectiveConjunction}), we have
    \[ \contractionof{\bencodingofat{\land}{\catvariableof{\land},\shortcatvariables},\onehotmapofat{1}{\catvariableof{\land}}}{\shortcatvariables} = \bigotimes_{\catenumeratorin} \onehotmapofat{1}{\catvariableof{\catenumerator}} \]
    and thus
    \begin{align*}
        \fsnnat{\catvariableof{\inneuronset},\selvariableof{\larchitecture}}
        = \contractionof{
            \{\bencodingof{\lneuron} \, : \, \lneuron\in\nodesof{\larchitecture}/\inneuronset \} \cup \{\bencodingofat{\land}{\catvariableof{\land},\headvariableof{\lneuron}  \, : \, \lneuron\in\outneuronset}, \onehotmapofat{1}{\catvariableof{\land}}\}
        }{\catvariableof{\inneuronset},\selvariableof{\larchitecture}} \, .
    \end{align*}
\end{proof}


% Combination of decompositions
By the commutation of contractions, we can further use \theref{the:neuronDecomposition} to decompose each tensor $\bencodingof{\lneuron}$ into connective and variable selecting components to get a sparse representation of a formula selecting neural network $\fsnn$.

%% Now as the definition!
%\begin{theorem}{Decomposition of formula selecting neural networks}\label{the:architectureDecomposition}
%	We have
%		\[ \bencodingof{\larchitecture} = \contractionof{\{\bencodingof{\lneuron} \, : \, \lneuron \in \larchitecture\}}{\catvariableof{\larchitecture},\shortcatvariables,\selvariableof{\larchitecture}} \]
%\end{theorem}
%\begin{proof}
%	By composition \theref{the:compositionByContraction}.
%	%\red{In addition: $X_{\larchitecture}$ specifying the headneurons! }
%\end{proof}

%% Now as the definition!
%% Relation between $\lneuron$ and $\bencodingof{\larchitecture}$
%Another useful property of encoded formula selecting architecture, is that we can retrieve any neuron by a simple contraction, as we show next.
%
%\begin{theorem}\label{the:formulaRetrieval}
%	Any neuron $\lneuron\in\larchitecture$ is retrieved by the contraction 
%		\[ \lneuron = \contractionof{\bencodingof{\larchitecture},\onehotmapof{1}[X_{\lneuron}]}{X\cup Z} \, . \]
%\end{theorem}
%\begin{proof}
%	First use the head neutralization property (Corollary~\ref{cor:onesHead}) in a parent stripping argument.
%	Then we are left with an architecture with $\lneuron$ being the only output neuron and use Corollary~\ref{cor:rhoToNormal} (we have $\restrictionofto{\mathrm{Id}}{[2]}=\onehotmapof{1}$).
%\end{proof}

% Alternative: Headneuron retrieval
%In case of multiple output neurons, the retrieval needs to be performed separately as in \theref{the:formulaRetrieval}, since contracting basis vectors $\onehotmapof{1}$ at multiple output neurons will retrieve the conjunction of those output neurons.


%\subsect{Skeleton Expressions}
%
%When only allowing for argument selections at the leaf level of the network, we get a skeleton expression.
%
%\begin{definition}\label{def:skeleton}
%	A skeleton expression
%		\[ \skeleton(\placeholderof{0},\ldots,\placeholderof{\selorder-1}) \]
%	is a composition of atom and connective selector maps, which are denoted by placeholders $\placeholderof{\selenumerator}$, where $\selenumerator\in[\selorder]$..
%	Each placeholder has a by $\selindexof{\selenumerator}$ enumerated list $\candidatelistof{\selenumerator}$ with cardinality $\seldimof{\selenumerator}= \cardof{\candidatelistof{\selenumerator}}$ of possible symbols denoting atoms or connectives to be placed in at this position.
%	This defines a map
%		\[ \skeleton : \left(\facstates\right) \times \left(\secfacstates\right) \rightarrow \{0,1\} \]
%	where $\skeleton(\atomindices,\selindices)$ denotes the formula given the selection of placeholders by $\selindices$, which is evaluated at the atoms $\atomindices$.
%\end{definition}

%\begin{definition}
%	A skeleton expression is a formula
%		\[ \skeleton(\placeholderof{0},\ldots,\placeholderof{\selorder-1}) \]
%	where instead of atoms and connectives there are placeholders $\placeholderof{\selenumerator}$, where $\selenumerator\in[\selorder]$.
%	Each skeleton has for each placeholder $\placeholderof{\selenumerator}$ a set $\candidatelistof{\selenumerator}$ of candidate atoms to be plugged in the placeholders.
%	We denote its cardinality to be $\seldimof{\selenumerator}= \cardof{\candidatelistof{\selenumerator}}$ and enumerate the elements $\placeholderof{\selenumerator}_{\selindexof{\selenumerator}}$ in each candidates list by an index $\selindexof{\selenumerator}\in[\seldimof{\selenumerator}]$.
%\end{definition}


%% ANOTHER EXAMPLE:


%\begin{definition}
%	Given a skeleton expression, the skeleton tensor is the map from the parameter space to the space of formula tensors, defined by
%	\begin{align}
%		 \skeletontensor : \rr^{\seldim} \rightarrow  \modelspace \quad , \quad
%		 \skeletontensor\left( \bigotimes_{\selenumeratorin}\unitvectoratof{\selenumerator}{\selindexof{\selenumerator}} \right) = \bencodingof{\skeletonof{\placeholderof{\selenumerator}_{\selindexof{\selenumerator}}\, : \, \selenumeratorin}}
%	\end{align}
%\end{definition}
%
%Using the canonical duality of tensors as maps and elements of tensor spaces, we can reinterpret is as a tensor \red{Domain representation of skeleton map}
%	\[ \skeletontensor \in \bigotimes_{\selenumerator\in[\selorder]} \rr^{\seldimof{\selenumerator}} \otimes  \modelspace  \, , \]
%which is the superposed formula tensor to a skeleton based parametrization.
%In the following, we investigate how to efficiently represent the skeleton tensor $\bencodingof{\skeleton}$ as a tensor network.









\sect{Application of Formula Selecting Networks}

There are two main applications of formula selecting networks.
First, when contracting the selection variables with a weight tensor we get a weighted sum of the parametrized formulas.
Second, when contracting the categorical variables with a distribution or a knowledge base, we get a tensor storing the satisfaction rates respectively the world counts of the parametrized formulas.

\subsect{Representation of selection encodings}

The main application of formula selecting networks in this work is the efficient representation of selection encodings.
This will be exploited in the sparse representation of exponential families by energies and in structure learning.
In the next lemma we will show the correspondence of formula selecting networks and selection encodings.

\begin{lemma}
    \label{lem:relToSelFSN}
    Given a set $\{\formulaof{\selindexlist} : \selindexlist\in\selstates\}$ of propositional formulas we define the statistic
    \[ \formulaset : \catindices \rightarrow (\formulaof{\selindexlist}(\catindices))_{\selindexlist} \, . \]
    and the formula selecting map
    \[ \fselectionmap: \catindices , \selindexlist \rightarrow \formulaof{\selindexlist} (\catindices) \, . \]
    Then
    \[ \sencodingofat{\formulaset}{\shortcatvariablelist, \shortselvariablelist} = \fselectionmap\left[\shortcatvariablelist, \shortselvariablelist \right] \, .  \]
\end{lemma}
\begin{proof}
    For any indices $\shortselindices\in\selstates$ and $\shortcatindices\in\atomstates$ we have
    \begin{align*}
        \sencodingofat{\formulaset}{\shortcatvariablelist=\shortcatindices, \shortselvariablelist=\shortselindices}
        =  \formulaof{\selindexlist}(\catindices) =  \fselectionmap\left[\shortcatvariablelist=\shortcatindices, \shortselvariablelist=\shortselindices \right] \, .
    \end{align*}
\end{proof}

%% Reason for basis encodings and selection encodings.
Technically, basis encodings have been exploited to derive decompositions based on basis calculus.
Selection encodings on the other hand enable the application of formula selecting networks as superpositions of formulas.



\subsect{Efficient Representation of Formulas}

% Exponentially many formulas represented by linear demand
Formula Selecting Neural Networks are means to represent exponentially many formulas with linear (in sum of candidates list lengths) storage.
Their contraction with probability tensor networks, is thus a batchwise evaluation of exponentially many formulas.
This is possible due to redundancies in logical calculus due to modular combinations of subformulas.

% Retrieve functions
We can retrieve specific formulas by slicing the selection variables, i.e. for $\selindices$ we have
\[ \exformula_{\selindices}[\shortcatvariables] = \fselectionmapat{\shortcatvariables,\selvariable=\selindices} \, .  \]

In a tensor network diagram we depict this by
\begin{center}
    \begin{tikzpicture}[thick, scale=0.35] % , baseline = -3.5pt

\node[anchor=east] (text) at (-3,0) {${\exformula_{\selindices}} \quad\quad {=}$};

\drawatomindices{0}{-4}
\draw (-1,3) rectangle (5, -3);
\node[anchor=center] (text) at (2,0) {$\bsencodingof{\fselectionmap}$};

\draw[->-] (2,3)--(2,5) node[midway,right] {\colorlabelsize $\headvariableof{\fselectionmap}$};
\draw (1,5) rectangle (3,7);
\node[anchor=center] (text) at (2,6) {$\tbasis$};

\draw[-<-] (5,-2.25)--(7,-2.25) node[midway,below] {\colorlabelsize $\selvariableof{0}$};
\draw[-<-] (5,-0.5)--(7,-0.5) node[midway,below] {\colorlabelsize $\selvariableof{1}$};
\node[anchor=center] (text) at (6,1) {$\vdots$};
\draw[-<-] (5,2.25)--(7,2.25) node[midway,above] {\colorlabelsize $\selvariableof{\selorder\shortminus1}$};

\draw (7,1.5) rectangle (9,3);
\node[anchor=center] (text) at (8,2.25) {$\onehotmapof{\selindexof{\selorder\shortminus1}}$};

\draw (7,0.25) rectangle (9,-1.25);
\node[anchor=center] (text) at (8,-0.5) {$\onehotmapof{\selindexof{1}}$};

\draw (7,-1.5) rectangle (9,-3);
\node[anchor=center] (text) at (8,-2.25) {$\onehotmapof{\selindexof{0}}$};

\end{tikzpicture}
\end{center}

% Interpretation by dynamic programming
Another perspective on the efficient formula evaluation by selection tensor networks is dynamic computing.
Evaluating a formula requires evaluations of its subformulas, which are done by subcontractions and saved for different subformulas due to the additional selection legs.

% Storage problem of solutions
However, we need to avoid contracting the tensor with leaving all selection legs open, since this would require exponential storage demand.
% Sparse algorithm
We can avoid this storage bottleneck by contraction of parameter cores $\canparam$ with efficient network decompositions along the selection variables. %extending the contractions by additional cores leaving less variable legs open.

% Gibbs sampling
In Gibbs Sampling (\algoref{alg:Gibbs}), one can use the energy-based approach to queries \theref{the:energyContractionQueries}, and contract basis vectors on all but one selection variables.

%\red{This is the case when contracting gradients of the parameter tensor networks in alternating least squares approaches.
%Other methods avoiding the bottleneck can be constructed by MCMC sampling, for example Gibbs Sampling.
%Here we only need to vary local components of the formula reflected in keeping only single variable legs open.}



\subsect{Batch contraction of parametrized formulas}

Given a set $\formulaset$ of formulas, we build a formula selecting network parametrizing the formulas.
The contraction
\begin{align*}
    \contractionof{\extnet,\fselectionmap}{\shortselvariables}
\end{align*}
is a tensor containing the contractions of the formulas $\formulaof{\shortselindices}$ with an arbitrary tensor network $\extnet$ as
\begin{align*}
    \contraction{\extnet,\formulaof{\shortselindices}} = \contractionof{\extnet,\fselectionmap}{\shortselvariables=\shortselindices} \, .
\end{align*}


\subsect{Average contraction of parametrized formulas}

We show in the next two examples, how a full contraction of the formula selecting map with a probability distribution or a knowledge base can be interpreted.

\begin{example}[Average satisfaction of formulas]
    The average of the formula satisfactions in $\formulaset$ giben a probability tensor $\probtensor$ is
    \[ \frac{1}{\prod_{\selenumeratorin}\seldimof{\selenumerator}} \cdot \contraction{\probtensor,\sencodingof{\formulaset}} \, . \]
\end{example}


\begin{example}[Deciding whether any formula is not contradicted]
    For example: We want to decide, whether there is a formula in $\formulaset$ not contradicted by a Knowledge base $\kb$.
    This is the case if and only if
    \[ \contraction{\kb,\sencodingof{\formulaset}} = 0 \, .  \]
    We use \lemref{lem:relToSelFSN} to get that $\sencodingof{\formulaset}=\fselectionmap$.
    When the formulas are representable in a folded scheme, we find tensor network decompositions of $\fselectionmap$ and exploit them along efficient representations of $\kb$ in an efficient calculation of $\contraction{\kb,\sencodingof{\formulaset}} $.
    This is further equal to
    \[ \kb \models \lnot \left( \bigvee_{\exformula\in\formulaset} \exformula\right) \, . \]
\end{example}


%\subsect{Neuro-Symbolic Architectures}
%
%%% Neuro-Symbolic Architecture
%We understand selector tensor networks as a neuro-symbolic architecture, where the selector variables are understood as parameters and the processed variables as neural activation variables.
%The orientation of the tensor network organizes the variables in layers.




\sect{Examples of formula selecting neural networks}




\subsect{Correlation}


For example (see Figure \ref{fig:AndSupFTDecomposition}) consider the logical neuron with single activation candidate $\{\land\}$ and two variable selectors selecting $\catorder$ atomic variables $\shortcatvariables$.
The expressivity of this network is the set of all conjunctions of the atoms
\[ \{\catvariableof{\atomenumerator} \land \catvariableof{\secatomenumerator} \, : \, \atomenumerator,\secatomenumerator\in[\atomorder] \} \]


% Covariance measure
Contracting with a probability distribution, we use the tensor
\[ \hypercoreat{\selvariableof{\vselectionsymbol,0},\selvariableof{\vselectionsymbol,1}} = \contractionof{\fsnn}{\selvariableof{\vselectionsymbol,0},\selvariableof{\vselectionsymbol,1}} \]
to read of covariances as
\[ \mathrm{Cov}(\catvariableof{\atomenumerator},\catvariableof{\secatomenumerator}) = \hypercoreat{\selvariableof{\vselectionsymbol,0}=\atomenumerator,\selvariableof{\vselectionsymbol,1}=\secatomenumerator}  -
\hypercoreat{\selvariableof{\vselectionsymbol,0}=\atomenumerator,\selvariableof{\vselectionsymbol,1}=\atomenumerator}  \cdot \hypercoreat{\selvariableof{\vselectionsymbol,0}=\secatomenumerator,\selvariableof{\vselectionsymbol,1}=\secatomenumerator} \, .  \]


%	\[ \skeleton = \placeholderof{1} \land \placeholderof{2} \]
%with the candidates for each placeholder being a set of $\atomorder$ atoms.

\begin{figure}[h]
    \begin{center}
        \begin{tikzpicture}[thick, scale=0.35] % , baseline = -3.5pt

\drawatomindices{0}{-4}
\draw (-1,-1) rectangle (5, -3);
\node[anchor=center] (text) at (2,-2) {$\rencodingof{\atomicformulaof{\parindexof{1}} \land \atomicformulaof{\parindexof{2}}}$};

\draw[->] (2,-1)--(2,1) node[midway,right] {\tiny ${\headvariableof{\parindexof{1}} \land \headvariableof{\parindexof{2}}}$};

\draw[<-] (5,-2.5)--(7,-2.5) node[midway,below] {\tiny $\fselectionvariable_{1}$}; 
\draw[<-] (5,-1.5)--(7,-1.5) node[midway,above] {\tiny $\fselectionvariable_{0}$}; 

\draw (7,-1) rectangle (9, -3);
\node[anchor=center] (text) at (8,-2) {$\canparam$};


		
\node[anchor=center] (text) at (12,-2) {${=}$};


\begin{scope}[shift={(17,8)}]
	\begin{scope}[shift={(0,-10)}]

		\draw[->] (5.5,5) -- (5.5,7) node[midway, right] {\tiny ${\headvariableof{\parindexof{1}} \land \headvariableof{\parindexof{2}}}$};
		\draw (1,3) rectangle (10, 5);
		\node[anchor=center] (text) at (5.5,4) {$\rencodingof{\land}$};

			
		% SelectorCores
		\draw[->] (2,1) -- (2,3) node[midway, left] {\tiny ${\headvariableof{\parindexof{2}}}$};
		\draw (-1,1) rectangle (5, -1);
		\node[anchor=center] (text) at (2,0) {$\selectorcoreof{2}$};
		\draw (5,0) -- (12,0);
		\draw[<-] (12,0) -- (14,0) node[midway, above] {\tiny $\fselectionvariable_{1}$};
		\begin{scope}[shift={(0,-2)}]
			\draw[<-] (0,1)--(0,-3) node[midway,left] {\tiny $\catvariableof{0}$}; 
			\draw[<-] (1.5,1)--(1.5,-3) node[midway,left] {\tiny $\catvariableof{1}$}; 
			\node[anchor=center] (text) at (3,0) {$\cdots$};
			\draw[<-] (4,1)--(4,-3) node[midway,right] {\tiny $\catvariableof{\atomorder\shortminus1}$}; 
		\end{scope}


		\draw (9,-1) -- (9,1);
		\draw[->] (9,1) -- (9,3) node[midway, left] {\tiny ${\headvariableof{\parindexof{1}}}$};
		\draw (6,-3) rectangle (12, -1);
		\node[anchor=center] (text) at (9,-2) {$\selectorcoreof{1}$};
		\draw[<-] (12,-2) -- (14,-2) node[midway, above] {\tiny $\fselectionvariable_{0}$};
		\drawatomindices{7}{-4}	
		
		% ParameterCores
		\draw (14,1) rectangle (16, -3);
		\node[anchor=center] (text) at (15,-1) {$\canparam$};
		
		\begin{scope}[shift={(-3.5,8)}]
			\draw[fill] (7.5,-15) circle (0.25cm);
			\draw[] (7.5,-15) to[bend left=25] (3.5,-13);
			\draw[] (7.5,-15) to[bend right=25] (10.5,-13);

			\draw[fill] (9,-15.25) circle (0.25cm);
			\draw[] (9,-15.25) to[bend left=25] (5,-13);
			\draw[] (9,-15.25) to[bend right=25] (12,-13);

			\draw[fill] (11.5,-15) circle (0.25cm);
			\draw[] (11.5,-15) to[bend left=25] (7.5,-13);
			\draw[] (11.5,-15) to[bend right=25] (14.5,-13);

			\drawatomindices{7.5}{-16}

		\end{scope}

	\end{scope}
\end{scope}

\end{tikzpicture}
    \end{center}
    \caption{Superposition of the encoded formulas $\bencodingof{\atomicformulaof{\selindexof{1}} \land \atomicformulaof{\selindexof{2}}}$ with weight $\canparam_{\selindexof{1} \selindexof{2}}$}
    \label{fig:AndSupFTDecomposition}
\end{figure}



\subsect{Conjunctive and Disjunctive Normal Forms}%\label{sec:CNFasFormulaSelection}

% Architecture
\red{
    We can represent any propositional knowledge base by the following scheme:
    Literal selecting neurons are logical neurons with connective identity/negation (selecting positive/negative literal) and selecting neurons select for each an atom.
    The single output neuron represents the disjunction, respectively the conjunction, combining the literal selecting neurons.
    The number of neurons defined by the maximal clause size plus one.
    Smaller clauses can be covered when adding False as a possible choice (The respective neuron has to choose the identity, otherwise the full clause will be trivial).
    This architecture will be discussed in more detail in \charef{cha:approximation} as $\cpformat$ selecting networks.
% Parameter
    The parameter core is in the basis $\cpformat$ format and each slice selects a clause of the knowledge base.
    In combination with polynomial decompositions, which will be provided in \charef{cha:networkRepresentation}, one can exploit this architecture to find sparse formula decompositions.
%When taking the slice values to infinity (e.g. by an annealing procedure), the represented member of the exponential family converges to the uniform distribution of the models of the knowledge base.
}

% Representation by selection tensor networks
\begin{remark}[Minterms and Maxterms]
    All minterms and maxterms can be represented by a two layer selection tensor networks without variable selection in two layers.
    The bottom layer has an $\lnot/\mathrm{Id}$ connective selection neuron to each atom and the upper layer consists of a single $\atomorder$ary conjunction.
\end{remark}




\sect{Extension to variables of larger dimension}

While we here restricted on boolean variables, formula selecting networks can be extended to variables of larger cardinality.
\begin{itemize}
    \item Connective selecting tensors: Can encode arbitrary functions $h_{\selindex}$ of discrete variables, but need $\catvariableof{\cselectionmap}$ to be an enumeration of the states, in particular to be of dimension
    \[ \catdimof{\cselectionmap} = \cardof{ \cup_{\selindexin} \imageof{h_{\selindex}} } \, . \]
    \item Variable selecting tensors can be understood as specific cases of connective selecting tensors and can thus also be generalized in a straight forward manner by
    \[ \catdimof{\cselectionmap} = \cardof{ \cup_{\selindexin} \imageof{h_{\selindex}} } \, .  \]
    \item State selecting tensors are directly defined for larger dimensions
\end{itemize}


An example of such a more generic usage is a discretization scheme for continuous neurons.

\begin{example}[Discretization of a continuous neuron]
    Let there be a neuron by a map of weight vectors and input vectors to $\rr$, that is
    \[ \sigma( w, x) : \rr^{\catorder} \times \rr^{\catorder} \rightarrow \rr \, .\]
%	When $w \in \arbsetof{weight}\subset \rr^{\catorder}$ and $x \in \arbsetof{x}\subset \rr^{\catorder}$ have
    We restrict the weights to a subset $\arbsetof{weight}\subset\rr^{\catorder}$ and the input vectors to $\arbsetof{x}\subset\rr^{\catorder}$,
    If follows that
    \[ \cardof{\imageof{\restrictionofto{\sigma}{\arbsetof{weight}\times\arbsetof{x}}}} \leq \cardof{\arbsetof{weight}} \cdot \cardof{\arbsetof{x}} \, . \]
    To discretize the neuron, we use the subset encoding scheme of \defref{def:subsetEncoding} and define enumeration variables $\indvariableof{weight}$, $\indvariableof{x}$ and $\indvariableof{\sigma}$ enumerating $\arbsetof{weight}$, $\arbsetof{x}$ and $\imageof{\restrictionofto{\sigma}{\arbsetof{weight}\times\arbsetof{x}}}$, which are accompanied by respective index interpretation functions.
    Then the basis encoding of the discretized neuron is
    \begin{align*}
        \bencodingofat{\sigma}{\indvariableof{\sigma},\indvariableof{weight},\indvariableof{x}} \, .
        = \sum_{\indindexofin{weight},\indindexofin{x}}
        \onehotmapof{\invindexinterpretationofat{\sigma(\indexinterpretationofat{weight}{\indindexof{weight}},\indexinterpretationofat{x}{\indindexof{x}})}{\sigma}}{\indvariableof{\sigma}}
        \otimes \onehotmapofat{\indindexof{weight}}{\indvariableof{weight}}
        \otimes \onehotmapofat{\indindexof{x}}{\indvariableof{x}} \, .
    \end{align*}
    If the neuron is of the form
    \[ \sigma(w,x) = \psi(\sum_i w_i \cdot x_i)\]
    a decomposition into multiplication at each coordinate and summation of the results, with basis encodings for each, can be done.
    \red{Here the index interpretation variables are split into a selection enumerated by $i$ and each variable gets assigned to single cores in the decomposition.}
\end{example}


    \section{Logic Network Representation}\label{cha:networkRepresentation}

Logic networks are graphical models with an interpretation by propositional logics.
We first distinguish between Markov Logic Networks, which are an approach to soft logics in the framework of exponential families, and Hard Logic Networks, which correspond with propositional knowledge bases.
Then we exploit non-trivial boolean base measures to unify both approaches by Hybrid Logic Networks, which are itself in exponential families.




%Markov Logic Networks are probability functions of truth assignments to logical functions.
%They respect propositional logic as hard constraints, but have beyond that freedom to shape probability distributions on possible situations.
%To capture these properties, we define them as graphical models with structure cores representing propositional logics and activation cores representing the specification of probability distributions.
% We in this part employ them to combine the probabilistic and the logical paradigm.


\subsection{Markov Logic Networks}

Markov Logic Networks exploit the efficiency and interpretability of logical calculus as well as the expressivity of graphical models. 

\subsubsection{Markov Logic Networks as Exponential Families}

We introduce Markov Logic Networks in the formalism of exponential families (see Section~\ref{sec:exponentialFamilies}).

\begin{definition}[Markov Logic Networks]
	Markov Logic Networks are exponential families $\mlnexpfamily$ with sufficient statistics by functions
		\[ \mlnstat : \atomstates \rightarrow \bigtimes_{\exformulain}[2] \subset \rr^{\cardof{\formulaset}} \]
	defined coordinatewise by propositional formulas $\exformulain$.
\end{definition}

% Binary Statistics as propositional formula
Since the image of each feature is contained in $[2]$, they are propositional formulas (see Def.~\ref{def:formulas}).

% Characterization of MLNs among exponential families: When choosing binary features
Conversely, any binary feature $\sstatcoordinateof{\statenumerator}$ of an exponential family defines a propositional formula (see Definition~\ref{def:formulas}).
Thus, any exponential family of distributions of $\atomstates$, such that $\imageof{\sstatcoordinateof{\statenumerator}}\subset\ozset$ for all $\statenumeratorin$ is a set of Markov Logic Networks with fixed formulas.

% Formula Selecting Networks
%We will further study the sparse representation of formula sets in Chapter~\ref{cha:architectures}.


The sufficient statistics consistent in a map $\formulaset$ of formulas brings the following advantages:
\begin{itemize}
	\item Numerical Advantage: The sufficient statistics is decomposable into logical connectives. 
	If the formulas are sparse (in the sense of limited number of connectives necessary in their representation), this gives rise to efficient tensor network decompositions of the relational encoding.
	\item Statistical Advantage: Since each formula is Boolean valued, the coordinates of the sufficient statistic are Bernoulli variables. 
	Due to their boundedness, they and their averages (by Hoeffdings inequality) are sub-Gaussian variables with favorable concentration properties (absence of heavy tails).
\end{itemize}


\begin{remark}[Alternative Definitions]
	We here defined MLNs on propositional logic, while originally they are defined in FOL.
	The relation of both frameworks will be discussed further in Chapter~\ref{cha:folModels}.
\end{remark}



\subsubsection{Tensor Network Representation}

Based on the previous discussion on the representation of exponential families by tensor networks in Section~\ref{sec:exponentialFamilies} we now derive a representation for Markov Logic Networks.

\begin{theorem}[Relational Encodings for Markov Logic Networks]\label{the:mlnTensorRep}
	A Markov Logic Network to a set of formulas $\formulaset = \{\enumformula \, : \, \selindexin\}$ is represented as
	\begin{align*}
		\mlnprobat{\shortcatvariables} = 
		\normationof{\{\enumformulaccwith : \selindexin \} \cup \{\enumformulaacwith : \selindexin \}
		}{\shortcatvariables}
	\end{align*}
	where we denote for each $\selindexin$ an activation core
	\begin{align*}
		\enumformulaacwith
		= \begin{bmatrix} 1 \\
		 \expof{\canparamat{\indexedselvariable}} 
		 \end{bmatrix}[\enumformulavar] \, .
	\end{align*}
\end{theorem}
\begin{proof}
	The claim follows from Theorem~\ref{def:expFamilyTensorRep} and the following contraction equations.
	We have with the grouped variable $\headvariableof{\formulaset} = \{\enumformulavar\, : \, \selindexin\}$
	\begin{align*}
		\rencodingofat{\formulaset}{\headvariableof{\formulaset},\shortcatvariables}
		= \contractionof{\{\enumformulaccwith : \selindexin \}}{\headvariableof{\formulaset},\shortcatvariables} \, .
	\end{align*}
	Since we have a Markov Logic Network we have $\imageof{\enumformula}\subset [2]$ and thus
	\begin{align*}
		\enumformulaac\left[\headvariableof{\selindex}=\headindex_{\selindex}\right]
		= \begin{cases}
			1 & \text{for} \quad \headindex_{\selindex} = 0 \\
			\expof{\canparamat{\indexedselvariable}} & \text{for} \quad \headindex_{\selindex}  = 1
		\end{cases}  
	\end{align*}
	Using these equations, the claim follows from Theorem~\ref{def:expFamilyTensorRep}.
\end{proof}

\begin{figure}[h]
\begin{center}
	\begin{tikzpicture}[thick, scale=0.35] % , baseline = -3.5pt

\drawundiratomindices{0}{-4}
\draw (-2,-1) rectangle (6, -3);
\node[anchor=center] (text) at (2,-2) {$\expof{\canparamat{\selvariable=\selindex}\cdot\formulaof{\selindex}}$};

		
\node[anchor=center] (text) at (10,-2) {${=}$};


\begin{scope}[shift={(15,-2)}]

		\draw (-0.5,3) rectangle (4.5, 5);
		\node[anchor=center] (text) at (2,4) {$\headcoreof{\formulaof{\selindex},\canparamat{\selvariable=\selindex}}$};

		\draw[fill] (2,2.25) circle (0.25cm);
		\draw[] (2,2.25) -- (2,3);
		\draw[->] (2,1) -- (2,2.5) node[midway, left] {\tiny $\catvariableof{\formulaof{\selindex}}$};
		
		\draw (-1,1) rectangle (5, -1);
		\node[anchor=center] (text) at (2,0) {$\rencodingof{\formulaof{\selindex}}$};

		\drawatomindices{0}{-2}

\end{scope}

\end{tikzpicture}
\end{center}
\caption{Factor of a Markov Logic Network to a formula $\enumformula$, represented as the contraction of a computation core $\enumformulacc$ and an activation core $\enumformulaac$.
	While the computation core $\enumformulacc$ prepares based on basis calculus a categorical variable representing the value of the statistic formula $\enumformula$ dependent on assignments to the distributed variables, the activation core multiplies an exponential weight to coordinates satisfying the formula.
}
\label{fig:mlnFactor}
\end{figure}

% 
Since any member of an exponential family is a Markov Network with tensors to each coordinate of the statistic, also Markov Logic Networks are Markov Networks.

\begin{corollary}\label{cor:MLNasMN}
	Given a set $\formulaset$ of formulas on atomic variables $\catvariableof{\nodes}$, we construct a $\graph=(\nodes,\edges)$, where $\nodes$ are decorated by the atoms and
		\[ \edges = \{ \nodesof{\formula}: \formula\in\formulaset \} \, , \]
	where by $\nodesof{\formula}$ we denote the minimal set such that there exists a tensor $\hypercoreat{\catvariableof{\nodesof{\formula}}}$ with
		\[ \formulaat{\catvariableof{\nodes}} = \hypercoreat{\catvariableof{\nodesof{\formula}}} \otimes \onesat{\catvariableof{\nodes/\nodesof{\formula}}} \, . \]		
	Any Markov Logic Network $\mlnparameters$ is then a Markov Network given the graph $\graphof{\formulaset}$
	$\{\expof{\canparamat{\indexedselvariable}\cdot\enumformula}
\, :\,\selindexin\}$.
\end{corollary}


% MLN as graphical models
Markov Logic Networks are Markov Networks with the factors given in a restricted form from the weighted truth of a formula.
Each formula is seen as a factor of the graphical model.

There are two sparsity mechanisms drastically reducing the number of parameters (and loosing generality):
\begin{itemize}
	\item Factors/Formulas contain only subsets of atoms (already in Corollary~\ref{cor:MLNasMN} exploited):
		The underlying assumptions of conditional independence loss generality.
	\item Structure in the factors: In MLN each factor corresponds with a formula evaluated on possible worlds.
		Again, any possible factor can be represented by a formula, but we concentrate on small formulas (see Theorem \ref{the:FormulaToTensor}).
\end{itemize}


% 
\red{We can extend the set of variables, by including the hidden formulas, and get a Markov Network of the relational encodings of connectives and headcores.
Here hidden variables are additional variables facilitating the decomposition, but not appearing in open variables of contractions when doing reasoning.
One can then exploit redundancies and make sure that every subresult is computed just once, by dropping relational encodings with identical head functions.
}


\begin{figure}[h]
\begin{center}
	\input{PartII/tikz_pics/network_representation/decomposed_representation.tex}
\end{center}
\caption{Example of a decomposed Markov Network representation of a Markov Logic Network with formulas $\{\formulaof{0} = a\lor b, \formulaof{1} = a \lor b \lor \lnot c\}$.
	Since both formulas share the subformula $a\lor b$, their contracted factors have a representation by a connected tensor network.}
% Where $\actcoreofat{\enumformula}{\enumformulavar} =\begin{bmatrix} 1 & \expof{\weightof{\exformula}} \end{bmatrix}[\formulavar] $}
\label{fig:mlnDecRep}
\end{figure}


%\begin{theorem}[Selection encodings for Energy representation]
%	\red{More the definition of exponential families.}
%	The energy of Markov Logic Networks is the contraction
%		\[ \mlnenergy = \sbcontractionof{\sencodingof{\formulaset},\canparam}{\shortcatvariables} \, . \]
%\end{theorem}


\subsubsection{Energy tensors}

%% Tensor Representation of MLN
With the energy tensor
\begin{align}
	\mlnenergy\left[\shortcatvariables\right]
	= \sum_{\selindexin} \canparamat{\indexedselvariable} \cdot \enumformulaat{\shortcatvariables} 
	= \sbcontractionof{\sencodingofat{\formulaset}{\shortcatvariables,\selvariable},\canparamat{\selvariable}}{\shortcatvariables} 
\end{align}
the MLN is the distribution
\begin{align}
	\mlnprobat{\shortcatvariables} = \normationof{\expof{\mlnenergy}}{\shortcatvariables} \, . 
\end{align}

In case of a common structure of the formulas in a Markov Logic Network, Formula selecting networks can be applied to represent their energies.

% Energy representation
%The weighted sum of formulas is then the energy of the Markov Logic Network.
We represent the superposition of formulas as a contraction with s parameter tensor.
Given a factored parametrization of formulas $\exformula_{\parindices}$ with indices $\selindexof{\parenumerator}$ we have the superposition by the network representation:
\begin{center}
	\begin{tikzpicture}[thick, scale=0.35] % , baseline = -3.5pt

\node[anchor=east] (text) at (-3,0) {$\sum_{\parindexof{[\parorder]}\in\parstates} \canparamat{\selvariableof{[\parorder]}=\parindexof{[\parorder]}} {\exformula_{\parindexof{[\parorder]}}} \quad {=}$};

%\node[anchor=center] (text) at (0.5,-8) {$\mathrm{log}$};

%\drawatomcore{3.5}{-8}{$\rencodingof{\fselectionmap}$}
%\drawatomindices{3.5}{-12}	
%
%
%\drawatomcore{3.5}{-4}{$\canparam$}
%\drawparindices{3.5}{-8}	



\drawatomindices{0}{-4}
\draw (-1,3) rectangle (5, -3);
\node[anchor=center] (text) at (2,0) {$\rencodingof{\fselectionmap}$};

\draw[->] (2,3)--(2,5) node[midway,right] {\tiny $\catvariableof{\fselectionmap}$}; 
\draw (1,5) rectangle (3,7);
\node[anchor=center] (text) at (2,6) {$\tbasis$};

\draw[<-] (5,-2)--(7,-2) node[midway,below] {\tiny $\selvariableof{0}$}; 
\draw[<-] (5,-0.5)--(7,-0.5) node[midway,below] {\tiny $\selvariableof{1}$}; 
\node[anchor=center] (text) at (6,0.75) {$\vdots$};
\draw[<-] (5,2)--(7,2) node[midway,above] {\tiny $\selvariableof{\parorder\shortminus1}$}; 

\draw (7,3) rectangle (9,-3);
\node[anchor=center] (text) at (8,0) {$\canparam$};

\end{tikzpicture}
\end{center}


% Representation 
If the number of atoms and parameters gets large, it is important to represent the tensor ${\exformula_{\parindices}}$ efficiently in tensor network format and avoid contractions.
To avoid inefficiency issues, we also have to represent the parameter tensor $\canparam$ in a tensor network format to improve the variance of estimations (see Chapter~\ref{cha:mlnConcentration}) and provide efficient numerical algorithms.

% Fail of full probability representation
However, when required to instantiate the probability distribution of a Markov Logic Network as a tensor network, we need to exponentiate and normate the energy tensor, a task for which relational encodings are required.
For such tasks, contractions of formula selecting networks are not sufficient and each formula with a nonvanishing weight needs to be instantiated as a factor tensor of a Markov Network. 






\subsubsection{Expressivity}\label{sec:MLNMaxMintermRep}

Based on Markov Logic Networks containing only maxterms and minterms (see Definition~\ref{def:clauses}), we here provide an expressivity study.
There are $2^{\atomorder}$ maxterms and $2^{\atomorder}$ minterms which are enough to represent any probability distribution as we show next.

\begin{theorem}\label{the:maximalClausesRepresentation}\label{the:mintermExpressivityMLN}
	Let there be a positive probability distribution
		 \[ \probof{\shortcatvariables} \in \bigotimes_{\atomenumeratorin}\rr^2 \, . \] 
	Then the Markov Logic Network of minterms (see Definition~\ref{def:clauses})
		\[ \mintermformulaset = \{\mintermof{\atomindices} \, : \, \atomindices\in\atomstates \}\]
	with parameters %with nonzero weights at the maxterms indexed by $\atomindicesin$
		\[ \canparamat{\selvariableof{0}=\catindexof{0},\ldots,\selvariableof{\atomorder-1}=\catindexof{\atomorder-1}}% \weightof{\mintermof{\atomindices}} 
		= \ln \probof{\indexedcatvariables} \]
	coincides with $\probof{\shortcatvariables}$.

	Further, the Markov Logic Network of maxterms
		\[ \maxtermformulaset = \{\maxtermof{\atomindices} \, : \, \atomindices\in\atomstates \}\]
	with wparameters
		\[ \canparamat{\selvariableof{0}=\catindexof{0},\ldots,\selvariableof{\atomorder-1}=\catindexof{\atomorder-1}} %\weightof{\maxtermof{\atomindices}} 
		= - \ln\probof{\indexedcatvariables} \]
	coincides with $\probof{\shortcatvariables}$.
\end{theorem}
\begin{proof}
	It suffices to show, that in both cases of choosing $\formulaset$ by minterms or maxterms with the respective parameters
		\[ \mlnenergy =  \ln\probof{\shortcatvariables} \]
	and therefore
		\[ \mlnprobat{\shortcatvariables} 
		= \sbnormationof{\expof{\mlnenergy}}{\shortcatvariables} 
		=  \sbcontractionof{\expof{\mlnenergy}}{\shortcatvariables} 
		= \probof{\shortcatvariables}\, . \]
	
	In the case of minterms, we notice that for any $\atomindicesin$
		\[ \mintermof{\atomindices}[\shortcatvariables] = \onehotmapofat{\atomindices}{\shortcatvariables} \]
	and thus with the weights in the claim
		\[ \sum_{\atomindicesin} 
		\left( \ln \probof{\indexedcatvariables} \right) \cdot \mintermof{\atomindices}[\shortcatvariables] 
		= \ln\probof{\shortcatvariables} \, .
		 \]

	For the maxterms we have analogously
		\[ \maxtermof{\atomindices}[\shortcatvariables] = \onesat{\shortcatvariables} - \onehotmapofat{\catindices}{\shortcatvariables} \]
	and thus that the maximal clauses coincide with the one-hot encodings of respective states.
	We thus have
	\begin{align*}
		& \sum_{\atomindicesin} 
		\left( - \ln \probof{\indexedcatvariables} \right) \cdot \maxtermof{\atomindices}[\shortcatvariables] \\
		& =
		\left(  \sum_{\nodes_0\subset [\atomorder]} 
		\left( - \ln \probof{\indexedcatvariables} \right) \cdot \onesat{\shortcatvariables} \right) \\
		& \quad + 
		\left(  \sum_{\nodes_0\subset [\atomorder]} 
		\left(  \ln \probof{\indexedcatvariables} \right) \cdot 
		\onehotmapofat{\catindices}{\shortcatvariables} 
		\right) 
		 \\
		 & = \ln\probof{\shortcatvariables} + \lambda \cdot  \onesat{\shortcatvariables}\,,
	\end{align*}
	where $\lambda$ is a constant.
\end{proof}

% Redundant parametrization
In general, this representation is redundant, since any offset of the weight by $\lambda\cdot\ones$ results in the same distribution.
However, the only $\bar{\canparam}$ are multiples of $\onesat{\shortcatvariables}$.

% Comparison with previous schemes
Theorem~\ref{the:maximalClausesRepresentation} is the analogue in Markov Logic to Theorem~\ref{the:tensorToMaxMinTerms}, which shows that any binary tensor has a representation by a logical formula, to probability tensors.
Here we require positive distributions for well-defined energy tensors.


\begin{remark}[Representation of Markov Networks]
% Composition of Markov Networks
	If a probability distribution is representable as a Markov Network, we only need to activate clauses and terms, which variables are contained in factors.
	\red{Make a theorem out of that?}
\end{remark}



%\subsubsection{Examples}


\subsubsection{Distribution of independent variables}

We show next, the independent positive distributions are representable by tuning the $\atomorder$ weights of the atomic formulas and keeping all other weights zero.

\begin{theorem}\label{the:independentAtomicMLN}
	Let $\probat{\shortcatvariables}$ be a positive probability distribution, such that disjoint subsets of atoms are independent from each other.
	Then $\probat{\shortcatvariables}$ is the Markov Logic Network of atomic formulas
		\[ \atomformulaset = \{\atomicformulaof{\catenumerator} \, : \, \catenumeratorin \} \]
	and parameters
		\[ \canparamat{\selvariable=\catenumerator} 
		= \lnof{\frac{
		\contractionof{\probtensor}{\catvariableof{\catenumerator}=1}
		}{
		\contractionof{\probtensor}{\catvariableof{\catenumerator}=0}
		}} \]
%	Any distribution such that the atom satisfaction is independent from each other is reproducable by a MLN with nonzero weights only for the atomic formulas.
\end{theorem}
\begin{proof}
%	Using the independent assumptions, the probability tensor factorizes into normed vectors to each atom, with are transformed atomic formulas (leaving out the neutral ones tensors).
%	We then find a weight to each atom such that the vector is reproduced by the contraction with the activation core.
	
	By Theorem~\ref{the:independenceProductCriterion} we get a decomposition 
		\[ \probat{\shortcatvariables} = \bigotimes_{\catenumeratorin} \probofat{\catenumerator}{\catvariableof{\catenumerator}} \,  \]
	where 
		\[ \probofat{\catenumerator}{\catvariableof{\catenumerator}} = \sbcontractionof{\probtensor}{\catvariableof{\catenumerator}} \, . \]
	
	By assumption of positivity, the vector $\probofat{\catenumerator}{\catvariableof{\catenumerator}}$ is positive for each $\catenumeratorin$ and the parameter
		\[ \canparamat{\selvariable=\catenumerator} 
		= \lnof{\frac{
		\probofat{\catenumerator}{\catvariableof{\catenumerator}=1}
		}{
		\probofat{\catenumerator}{\catvariableof{\catenumerator}=0}
		}} \]
	well-defined.
	
	We then notice, that 
		\[ \expdistofat{(\{\atomicformulaof{\catenumerator}\},\canparamat{\selvariable=\catenumerator})}{\catvariableof{\catenumerator}} 
		= \probofat{\catenumerator}{\catvariableof{\catenumerator}}\]
	and therefore with the parameter vector of dimension $\seldim=\catorder$ defined as
		\[ \canparamat{\selvariable} = \sum_{\catenumeratorin} \canparamat{\selvariable=\catenumerator} \cdot \onehotmapofat{\catenumerator}{\selvariable}  \]
	we have
	\begin{align*}
	 	 \expdistofat{(\{\atomicformulaof{\catenumerator} \, : \, \catenumeratorin\},\canparam)}{\shortcatvariables} 
		& = \bigotimes_{\catenumeratorin} \expdistofat{(\{\atomicformulaof{\catenumerator}\},\canparamat{\selvariable=\catenumerator})}{\catvariableof{\catenumerator}} \\
		& = \bigotimes_{\catenumeratorin} \probofat{\catenumerator}{\catvariableof{\catenumerator}} \\
		& = \probat{\shortcatvariables} \, . 
	\end{align*}
\end{proof}

%In general, the statistic to an atomic formula measures the marginal distribution. -> To Parameter Estimation

% Failing to be positive -> Hybrid networks
In Theorem~\ref{the:independentAtomicMLN} we made the assumption of positive distributions.
If the distribution fails to be positive, we still get a decomposition into distributions of each variable, but there is at least one factor failing to be positive.
Such factors need to be treated by hybrid logic networks, that is they are base measure for an exponential family coinciding with a logical literal (see Section~\ref{sec:hardNetworks}).

% Energy representation
All atomic formulas can be selected by a single variable selecting tensor, that is
	\[ \energytensorofat{(\{\atomicformulaof{\catenumerator} \, : \, \catenumeratorin\},\canparam)}{\shortcatvariables}
	= \sbcontractionof{\vselectionmapat{\shortcatvariables,\selvariable},\canparamat{\selvariable}}{\shortcatvariables} \, . 
	\]
	
% Holds also more generally for any formula! -> Place it earlier?
In case of negative coordiantes $\canparamat{\selvariable=\catenumerator}$ it is convenient to replace $\atomicformulaof{\catenumerator}$ by $\lnot\atomicformulaof{\catenumerator}$, in order to facilitate the interpretation.
The probability distribution is left invariant, when also replacing $\canparamat{\selvariable=\catenumerator}$ by $-\canparamat{\selvariable=\catenumerator}$.



\subsubsection{Boltzmann machines}

%\red{Add sufficient statistics?}

A Boltzmann machine is a member of an exponential family with the energy tensor (see e.g. Chapter 43 in \cite{mackay_information_2003})
	\[ \energytensorofat{W,b}{\indexedcatvariables} = 
	\sum_{\atomenumerator,\secatomenumerator \in [\atomorder]} 
		W[\selvariableof{\vselectionsymbol,0}=\atomenumerator, \selvariableof{\vselectionsymbol,1}=\secatomenumerator] \cdot \catindexof{\atomenumerator} \cdot \catindexof{\secatomenumerator} 
	+ \sum_{\atomenumerator\in[\atomorder]} b[\selvariableof{\vselectionsymbol,0}=\atomenumerator] \cdot \catindexof{\atomenumerator}\, . \]


%sufficient statistic 
%	\[ \sstat : \atomstates \rightarrow (\rr^{\catorder}\otimes \rr^{\catorder}) \times \rr^{\catorder} \]
%by interaction term
%	\[ \sstat(\shortcatindices) = (\catindexof{\atomenumerator} \Leftrightarrow \catindexof{\secatomenumerator})_{\atomenumerator,\secatomenumerator \in[\atomorder]} \]
%and by potential term
%	\[ \sstat(\shortcatindices) =  (\catvariableof{\atomenumerator})_{\atomenumeratorin} \, . \]

We notice, that this coincides with the energy tensor of a Markov Logic Network with formula set 
	\[ \formulaset = \{ \catvariableof{\atomenumerator} \Leftrightarrow \catvariableof{\secatomenumerator} \, : \, \atomenumerator,\secatomenumerator \in[\atomorder] \} 
	\cup \{ \catvariableof{\atomenumerator}\, : \, \atomenumeratorin \} \, \]
with cardinality $\atomorder^2+\atomorder$.

Each formula is in the expressivity of an architecture consisting of a single binary logical neuron selecting any variable of $\shortcatvariables$ in each argument and selecting connectives $\{\eqbincon,\lpasbincon\}$, where by $\lpasbincon$ we refer to a connective passing the first argument, defined for $\catindexofin{0}, \catindexofin{1}$ as 
	\[ \lpasbincon[\indexedcatvariableof{0},\indexedcatvariableof{1}] = \vselectionmapat{\indexedcatvariableof{0},\catvariableof{1},\selvariableof{\vselectionsymbol}=0} \, . \]

The weight is
	\[ \canparam 
	= \onehotmapofat{0}{\selvariableof{\cselectionsymbol}} \otimes W 
	+ \onehotmapofat{1}{\selvariableof{\cselectionsymbol}} \otimes b[\selvariableof{\vselectionsymbol,0}] \otimes  \onehotmapofat{0}{\selvariableof{\vselectionsymbol,0}} 
	\]
	
And we have
	\[ \energytensorofat{W,b}{\shortcatvariables} = 
	\sbcontractionof{\fsnnat{\shortcatvariables,\selvariableof{\cselectionsymbol},\selvariableof{\vselectionsymbol,0},\selvariableof{\vselectionsymbol,1}}, \canparamat{\selvariableof{\cselectionsymbol},\selvariableof{\vselectionsymbol,0},\selvariableof{\vselectionsymbol,1}}}{\shortcatvariables} \, . \]


\begin{figure}[h]
\begin{center}
	\input{PartII/tikz_pics/network_representation/boltzmann_energy.tex}
\end{center}
\caption{Tensor network representation of the energy of a Boltzmann machine}
\label{fig:boltzmannEnergy}
\end{figure}


%where by $(\cdot,\cdot)|_{0}$
%To connect with the formalism of Boltzmann machines, let us identify the visible units of a Boltzmann machines with the atoms in a propositional theory.

%Boltzmann machines are then reproduced by taking $\atomorder^2+\atomorder$ many formulas, namely those measuring the correlations and the marginal distributions.
%To be more precise, the correlation between atom $\atomicformulaof{\atomenumerator}$ and $\atomicformulaof{\secatomenumerator}$ is measured by the satisfaction rate of the formula 
%	\[ \exformula_{\atomenumerator,\secatomenumerator} = \atomicformulaof{\atomenumerator} \leftrightarrow \atomicformulaof{\secatomenumerator}\]

%\begin{theorem}
%	Any Boltzmann machine over $\atomorder$ units with interaction matrix $U\in\rr^{\atomorder\times\atomorder}$ and potential term $b\in\rr^{\atomorder}$ (MacKay Book notation) is a MLN where the only nonzero weights are 
%		\[ \weightof{\atomicformulaof{\atomenumerator} } = b_{\atomenumerator} \quad, \quad \atomenumeratorin \]
%	and 
%		\[ \weightof{ \exformula_{\atomenumerator,\secatomenumerator} } = U_{\atomenumerator, \secatomenumerator} \quad , \quad \atomenumerator,\secatomenumerator \in [\atomorder]\] 
%\end{theorem}

\red{
Often Boltzmann machines are formulated with hidden variables.
To average those out, one needs to instantiate the probability distribution instead of the energy tensor and leave only visible variables open in a contraction.
}


Markov Logic Networks go beyond the Boltzmann machines already for binary formulas, by the flexibility to capture further dependencies beyond the correlation.
We can use any binary logical connective and have an associated formula where we can put a weight on.














\subsection{Hard Logic Networks}\label{sec:hardNetworks} % To be dropped in the unification with the MLN chapter

% Hard logic vs markov logic
While exponential families are positive distributions, in logics probability distributions can assign states zero probability.
As a consequence, Markov Logic Networks have a soft logic interpretation in the sense that violation of activated formulas have nonzero probability.
We here discuss their hard logic counterparts, where worlds not satisfying activated formulas have zero probability.

\subsubsection{The limit of hard logic}\label{sec:hardLogicLimit} % To be merged with the above

The probability function of Markov Logic Networks with positive weights mimiks the tensor network representation of the knowledge base, which is the conjunction of the formulas. 
The maxima of the probability function coincide with the models of the corresponding knowledge base, if the latter is satisfiable.
However, since the Markov Logic Network is defined as a normed exponentiation of the weighted formula sum, it is a positive distribution whereas uniform distributions among the models of a knowledge base assign zero probability to world failing to be a model.
Since both distributions are tensors in the same space to a factored system, we can take the limits of large weights and observe, that Markov Logic Networks indeed converge to normed knowledge bases.


\begin{lemma}
	For any satisfiable formula $\formulaat{\shortcatvariables}$ and a variable weight $\canparam\in\rr$, we have for $\canparam\rightarrow\infty$
	\begin{align*}
		\normationof{\expof{\canparam\cdot\formulaat{\shortcatvariables}}}{\shortcatvariables} \rightarrow \normationof{\exformula}{\shortcatvariables}
	\end{align*}
	and for $\canparam\rightarrow-\infty$
	\begin{align*}
		\normationof{\expof{\canparam\cdot\formulaat{\shortcatvariables}}}{\shortcatvariables} \rightarrow \normationof{\lnot\exformula}{\shortcatvariables} \, .
	\end{align*}
	Here we denote the understand the convergence of tensors as a convergence of each coordinate.
\end{lemma}
\begin{proof}
	We have 
	\begin{align*}
		\partitionfunctionof{\mlnparameters} = \left(\prod_{\atomenumeratorin} \catdimof{\atomenumerator}\right) - \contraction{\exformula} + \contraction{\exformula} \cdot \expof{\canparam}
	\end{align*}
	and therefore for any $\shortcatindices\in\atomstates$ with $\formulaat{\indexedshortcatvariables}=1$
	\begin{align*}
		\normationof{\expof{\canparam\cdot \exformula}}{\indexedshortcatvariables}
		&= \frac{
			\expof{\canparam}
			}{
			\left(\prod_{\atomenumeratorin} \catdimof{\atomenumerator} \right) - \contraction{\exformula} + \contraction{\exformula} \cdot \expof{\canparam}
			} \\
		& \rightarrow \frac{1}{\contraction{\exformula}} 
		= \normationof{\exformula}{\indexedcatvariables} \, . 
	\end{align*}
	For any $\atomindices\in\atomstates$ with $\formulaat{\indexedshortcatvariables}=0$ we have on the other side
	\begin{align*}
		\normationof{\expof{\canparam\cdot \exformula}}{\indexedcatvariables}
		&= \frac{
			1
			}{
			\left(\prod_{\atomenumeratorin} \catdimof{\atomenumerator}\right) - \contraction{\exformula} + \contraction{\exformula} \cdot \expof{\canparam}
			} \\
		& \rightarrow 0
		= \normationof{\exformula}{\indexedcatvariables} \, . 
	\end{align*}
\end{proof}



% Limit on the activation core
We can by the above Lemma represent both the situation of non-asymptotic weights and the limit for diverging weights by the same computation core $\formulaccwith$, with different activation cores, since
\begin{align*}
	\normationof{\expof{\canparam\cdot\formulaat{\shortcatvariables}}}{\shortcatvariables} 
	= \contractionof{\formulaccwith,\actcoreof{\formula,\canparam}}{\shortcatvariables}
\end{align*}
and 
\begin{align*}
	\normationof{\formula}{\shortcatvariables}
	= \contractionof{\formulaccwith,\tbasisat{\formulavar}}{\shortcatvariables} \quad \text{respectively} \quad
	\normationof{\lnot\formula}{\shortcatvariables}
	= \contractionof{\formulaccwith,\fbasisat{\formulavar}}{\shortcatvariables} \, . 
\end{align*}



\begin{theorem}
	Let $\formulaset$ be a formulaset and $\canparam$ a positive parameter vector.
	If the formula
		\[ \kb = \bigwedge_{\exformulain} \exformula \]
	is satisfiable we have in the limit $\invtemp\rightarrow\infty$ the coordinatewise convergence
		\[ \expdistofat{(\formulaset,\invtemp\cdot\canparam)}{\shortcatvariables} \rightarrow \normationofwrt{\kb}{\shortcatvariables} \, . \]
\end{theorem}
\begin{proof}
	Since $\kb$ is satisfiable we find $\catindices\in\atomstates$ with
		\[  \contractionof{\expof{\sum_{\exformulain}\invtemp\cdot \weightof{\exformula} \cdot \exformula}}{\indexedcatvariables} = \expof{\invtemp \cdot \sum_{\exformulain}\weightof{\exformula}}  \]
	and the partition function obeys
		\[ \contractionof{\expof{\sum_{\exformulain}\invtemp\cdot \weightof{\exformula} \cdot \exformula}}{\varnothing} \geq  \expof{\invtemp \cdot \sum_{\exformulain}\weightof{\exformula}}  \, . \]
	For any state $\catindices\in\atomstates$ with $\kb(\catindices)=0$ we find $\secexformula\in\formulaset$ with $\secexformula(\catindices)=0$ and have
	\begin{align*}
	 	\frac{
		\contractionof{\expof{\sum_{\exformulain}\invtemp\cdot \weightof{\exformula} \cdot \exformula}}{\indexedcatvariables}
		}{
		\contractionof{\expof{\sum_{\exformulain}\invtemp\cdot \weightof{\exformula} \cdot \exformula}}{\varnothing}
		} 
		\leq  
	 	\frac{
		\expof{\invtemp\cdot \sum_{\exformulain : \exformula\neq \secexformula}\weightof{\exformula}}
		}{
		\expof{\invtemp\cdot \sum_{\exformulain}\weightof{\exformula}}
		} 
		= \expof{\invtemp \cdot \weightof{\secexformula}} \rightarrow 0 \, . 
	\end{align*}
	The limit of the distribution has thus support only on the models of $\kb$. 
	Since any model of $\kb$ has same energy at any $\invtemp$ the limit is a uniform distribution and coincides therefor with
		\[ \normationofwrt{\kb}{\shortcatvariables} \, . \]
\end{proof}


\begin{remark}[More generic situation of simulated annealing]
	The process of taking $\invtemp\rightarrow\infty$ is known as simulated annealing, see Chapter~\ref{cha:probReasoning}.
	From the discussion there we have the more general statement, that the limiting distribution is the uniform distribution among the maxima of $\expdistofat{(\formulaset,\canparam)}{\shortcatvariables}$.
	If the formula $\kb$ is not satisfiable the normation $\normationofwrt{\kb}{\shortcatvariables}{\varnothing}$ does not exist and the limit distribution has another syntactical representation, to be gained e.g. by minterm or maxterm representation (see Theorem~\ref{the:tensorToMaxMinTerms}).
\end{remark}





%To make this convergence precise, we define the uniform distribution 
%\begin{align}
%	\expdistofat{\kb}{\datapoint}
%	= \begin{cases} 
%	\frac{1}{\braket{\ftensorof{\kb},\ones}} & \text{if } \braket{\ftensorof{\kb},\datapoint} = 1 \\
%	0 &  \text{if } \braket{\ftensorof{\kb},\datapoint} = 0
%	\end{cases}
%\end{align}

%\begin{theorem}
%	Given a MLN parameterized by $\mlnparameters$, we have for $\lambda\rightarrow\infty$
%		\[ \kldivof{\expdistofat{(\formulaset,\lambda\cdot\weight)}{\datapoint}}{\expdistofat{\kb}{\datapoint}} \rightarrow 0 \, .\]
%\end{theorem}
%\begin{proof}
%	Follows directly from the convergence at each core.
%\end{proof}





\subsubsection{Tensor Network Representation}

Hard Logic Network coincide with Knowledge Bases and are thus representable by contractions of formulas (which can be interpreted as an effective calculus scheme, see Section~\ref{sec:effectiveCalculus}).
We use $\land$ symmetry to represent them as a contraction of the formulas building the Knowledge Base as conjunction.

\begin{theorem}[Conjunction Decomposition of Knowledge Bases]\label{the:conDecKB}
	For a Knowledge Base
		\[ \kb = \bigwedge_{\exformula\in\formulaset} \exformula \]
	we have
		\[ \kbat{\shortcatvariables} = \contractionof{\formulaat{\shortcatvariables}}{\shortcatvariables}   \]
	and
		\[ \kbat{\shortcatvariables} = \contractionof{\{\formulaccwith \, : \, \formula\in\formulaset\} \cup \{\tbasisat{\formulavar} \, : \, \formula\in\formulaset\} }{\shortcatvariables} \, .  \]
\end{theorem}
\begin{proof}
	By the $\land$-symmetry, see effective calculus and 
		\[ \formulaat{\shortcatvariables} =  \contractionof{\{\formulaccwith,\tbasisat{\formulavar}\}}{\shortcatvariables} \]
\end{proof}

\begin{remark}{$\land$ symmetry does not generalize to Markov Logic Networks}
	% Comparison to Markov Logic
	In Markov Logic, similar decompositions are not possible.
	For example, consider a MLN with a single formula $\atomicformulaof{0}\land\atomicformulaof{1}$ and nonvanishing weight $\canparam$.
	This does not coincide with the distribution of a MLN of two formulas $\atomicformulaof{0}$ and $\atomicformulaof{1}$.
	To see this, we notice that with respect to the distribution of the first MLN, both variables are not independent, while for any MLN constructed by the two atomic formulas they are.
\end{remark}


%It is known, that there are symmetries in the syntactical represention of Knowledge Bases.
%
%There is a lot of redundancy in the activation of Knowledge Cores describing exactly the same models.
%
%\begin{theorem}[$\land$-symmetry]\label{the:landSymmetry}
%	We observe that the contraction of an $\land$ core with $\tbasis$  is equivalent with $\tbasis$ cores on all the connected subformulas.
%\end{theorem}
%\begin{proof}
%	By equality of the Knowledge Base contraction in both ways: The missing subformulas behave the same if they are activated, since they then are contrained to the same subnetworks somewhere else. 
%	%\red{Find better arguments for missing subformulas when having the larger core.}
%\end{proof}
%
%\begin{theorem}[$\lnot$-symmetry]
%	Similarly the contraction of an $\lnot$ core with $\tbasis$ or $\tbasis$ has the same result as with $\tbasis$ or $\tbasis$ on the subformula.
%\end{theorem}
%
%We call the application of these in changing the Knowledge Cores without changing the contracted network as the representation symmetry.


%\subsubsection{Conjunctive Normal Representation}

%One tensor representation of a Knowledge Base is the association of the Knowledge Core $\tbasis$ at the formula being the Knowledge Base itself.
%We can use the $\land$ symmetry (Theorem~\ref{the:landSymmetry}) to propagate $\tbasis$ to all clause cores and get an alternative representation.
%Those are especially interesting when using Modus Ponens/Resolution as local sub-KB reasoners (see Section~\ref{subsec:LocalEntailment}).


\subsubsection{Polynomial Representation}

We now apply the representation symmetries to represent a propositional Knowledge Base in conjunctive normal form.
A Knowledge Base in Conjunctive Normal Form is a conjunction of clauses, where clauses are disjunctions of literals being atoms (positive literals) or negated atoms (negative literals).

Formulas can be represented as sparse polynomials, which will be discussed in more detail in Chapter~\ref{cha:sparseTC} (see Definition~\ref{def:polynomialSparsity}).

\begin{lemma}\label{lem:clauseTermBasisPlus}
	Any term is representable by a single monomial and any clause is representable by at most two monomials. %, any term of basis+ with rank 1. %Use also \baspluscprankof{}
\end{lemma}
\begin{proof}
	Let $\nodes_0$ and $\nodes_1$ be disjoint subsets of $\nodes$, then we have
	\begin{align*}
		\termof{\nodes_0}{\nodes_1} = \onehotmapofat{
			\{\catindexof{\atomenumerator} = 0 : \atomenumerator\in\nodes_0\} \cup \{\catindexof{\atomenumerator} = 1 : \atomenumerator\in\nodes_1\}
		}{\catvariableof{\nodes_0\cup\nodes_1}} \otimes \onesat{\catvariableof{\nodes/(\nodes_0\cup\nodes_1)}}
	\end{align*}
	and
	\begin{align*}
		\clauseof{\nodes_0}{\nodes_1} = \onesat{\catvariableof{\nodes}} - \onehotmapofat{
			\{\catindexof{\atomenumerator} = 0 : \atomenumerator\in\nodes_0\} \cup \{\catindexof{\atomenumerator} = 1 : \atomenumerator\in\nodes_1\}
		}{\catvariableof{\nodes_0\cup\nodes_1}}
		\otimes \onesat{\catvariableof{\nodes/(\nodes_0\cup\nodes_1)}} \, . 
	\end{align*}
	We notice, that any tensors $\ones$ and $\onehotmapof{\catindex}\otimes \ones$ habe basis+-rank of $1$ and therefore $\termof{\nodes_0}{\nodes_1}$ of $1$ and $\clauseof{\nodes_0}{\nodes_1}$ of at most $2$.
\end{proof}


We apply Lemma~\ref{lem:clauseTermBasisPlus} to show the following sparsity bound on the energy tensor of Markov Logic Networks.

\begin{theorem}
	Any formula $\exformula$ with a conjunctive normal form of $n$ clauses satisfies
		\[ \slicesparsityof{\exformula} \leq 2^{n} \, . \]
	For any set $\formulaset$ of formulas each with a conjunctive normal form of $n_{\exformula}$ clauses satisfies for any $\weight$
		\[ \slicesparsityof{\sum_{\exformulain}\weightof{\exformula}\cdot\exformula} \leq \sum_{\exformulain}2^{n_{\exformula}} \, . \]
\end{theorem}
\begin{proof}
	Let $\exformula$ have a CNF with clauses indexed by $l\in[n]$ and each clause represented by subsets $\nodes_0^l, \nodes_1^l$, that is
		\[ \exformula = \bigwedge_{l \in [n]}  \clauseof{\nodes_0^l}{\nodes_1^l} \, . \]
	We now use the rank bound of Theorem~\ref{the:CPrankContractionBound} and Lemma~\ref{lem:clauseTermBasisPlus} to get
	\begin{align*}
		\slicesparsityof{\exformula} \leq \prod_{l \in [n]}  \slicesparsityof{\clauseof{\nodes_0^l}{\nodes_1^l}} \leq 2^n \, . 
	\end{align*}
	
	Given a collection of formulas $\formulaset$, each with a CNF of $n_{\exformula}$ clauses we apply Theorem~\ref{the:CPrankSumBound} and get
	\begin{align*}
		\slicesparsityof{\sum_{\exformulain}\weightof{\exformula}\cdot\exformula} \leq \sum_{\exformulain}\slicesparsityof{\exformula} \leq \sum_{\exformulain}2^{n_{\exformula}} \, . 
	\end{align*}
\end{proof}


\subsubsection{Categorical Constraints}\label{sec:categoricalTN}

%% Categorical variables with more possibilities
We made the assumption that all categorical variables in factored systems to be represented by propositional logics take binary values (i.e. $\catdim=2$).
In cases where a categorical variable $\catvariable$ takes multiple values we define for each $\catindex$ an atomic formula $\catvariableof{\catindex}$ representing whether $\catvariable$ is assigned by $\catindex$ in a specific state.
	%\[ \catvariableof{\catindex} =  (\catvariable = \catindex \, . \] Confusing notation
Following this construction we have the constraint that exactly one of the atoms $\catvariableof{\catindex}$ is $1$ at each state.

%% Capture constraint
To capture the constraints resulting from this construction we introduce auxiliary parts. % of Bayesian Propositional Networks.
Such constraints can also be expressed by a formula but would result in an unnecessary large tensor network.


%% Categorical selection map
\begin{definition}[Categorical Constraint and Atomization Variables]
	Given a list $\catvariableof{0},\ldots,\catvariableof{\catdim-1}$ of boolean variables and a categorical variable $\catvariable$ with dimension $\catdim$ a categorical constraint is a tensor $\categoricalmap[\catvariable,\catvariableof{[\catdim]}]$ defined as
	\begin{align*}
		 \categoricalmap(\catindex,\catindexof{\variableset}) 
		 = \begin{cases} 
		 	1 & \text{if} \quad \catindexof{[\catdim]} = \onehotmapof{\catindex} \\
			0 & \text{else} \, . 
		 \end{cases}
	\end{align*}
	We then call the variables  $\catvariableof{0},\ldots,\catvariableof{\catdim-1}$ the atomization variables to the categorical variable $\catvariable$.
\end{definition}

%% Decomposition
With Theorem~\ref{the:functionDecompositionBasisCP} the tensor representation of $\categoricalcore$ decomposes in a basis CP format (see Figure~\ref{fig:CategoricalDecomposition}b) of if its coordinate maps $\categoricalmap_{\catindex}$, where $\catindex\in[\catdim]$.
For the cores
\begin{align}
	\categoricalcoreof{\catindex} = \onehotmapofat{\catindex}{\catvariable} \otimes \onehotmapofat{1}{\catvariableof{\catindex}} + (\onesat{\catvariable}- \onehotmapofat{\catindex}{\catvariable} ) \otimes \onehotmapofat{0}{\catvariableof{\catindex}} 
\end{align}	
we have by Theorem~\ref{the:functionDecompositionBasisCP}
\begin{align*}
	\rencodingofat{\categoricalmap}{\catvariable, \catvariableof{0}, \ldots, \catvariableof{\catdim-1}} 
	= \contractionof{\{\rencodingof{\categoricalmap(\catindex)} \, : \, \catindex\in[\catdim]\}}{\catvariable, \catvariableof{0}, \ldots, \catvariableof{\catdim-1}} \, . 
\end{align*}


In the next theorem we show how a categorical constraint can be enforced in a tensor network by adding the tensor $\categoricalmap$ to a contraction.

\begin{theorem}
	For any tensor $\hypercoreat{\shortcatvariables}$ and a categorical constraint defined by an ordered subset $\catvariableof{\variableset}\subset\shortcatvariables$, a variable $\catvariable\in\shortcatvariables$ we have
	\begin{align*}
	 	\contractionof{\{\hypercoreat{\shortcatvariables}, \categoricalmap\}}{\indexedcatvariables} 
		= \begin{cases}
			\hypercoreat{\indexedcatvariables} & \text{if} \quad \catindexof{\variableset} = \onehotmapof{\catindex} \\
			0 & \text{else} \, . 
		\end{cases}
	\end{align*}
\end{theorem}
\begin{proof}
	For any $\catindexof{[\atomorder]}$ we have
		\[ \contractionof{\{\hypercoreat{\shortcatvariables}, \categoricalmap\}}{\indexedcatvariables}  = 
			\hypercoreat{\shortcatindices} \cdot \categoricalmap[\indexedcatvariableof{},\indexedcatvariableof{\variableset}] \, . 
		\]
	If $\catindexof{\variableset} = \onehotmapof{\catindex}$ we have $\categoricalmap[\indexedcatvariableof{},\indexedcatvariableof{\variableset}] = 1$ and thus
		\[ \contractionof{\hypercoreat{\shortcatvariables},\categoricalmap}{\indexedcatvariables}  =  \hypercoreat{\shortcatindices}  \, . \]
	If $\catindexof{\variableset} \neq \onehotmapof{\catindex}$ then $\categoricalmap[\indexedcatvariableof{},\indexedcatvariableof{\variableset}] = 0$ and  
		\[ \contractionof{\hypercoreat{\shortcatvariables},\categoricalmap}{\indexedcatvariables}  = 0 \, . \]
\end{proof}

\begin{figure}[h]
\begin{center}
	\begin{tikzpicture}[scale=0.35, thick] % , baseline = -3.5pt

\begin{scope}[shift={(-15,2)}]

\node[anchor=center] (text) at (-1,3) {${a)}$};


\node [circle, draw, thick, fill=gray!50] (T1) at (0,0) {\tiny $\randomxof{0}$};	
\node [circle, draw, thick, fill=gray!50] (T2) at (3,0) {\tiny $\randomxof{1}$};	
\node[anchor=center] (text) at (6,0) {\small $\cdots$};
\node [circle, draw, thick, fill=gray!50] (T3) at (9,0) {\tiny $\randomxof{\atomorder-1}$};	

\node [circle, draw, thick, fill=gray!50] (C) at (4.5,-5) {\tiny $\randomxof{\categoricalmap}$};	

\draw[->] (C) -- (T1);
\draw[->] (C) -- (T2);
\draw[->] (C) -- (T3);

\end{scope}

\node[anchor=center] (text) at (-1,5) {${b)}$};


\drawatomindices{0}{2}
\draw (-1,1) rectangle (5,-1);
\node[anchor=center] (text) at (2,0) {\small $\categoricalcore$};
\draw[->] (2,-3) -- (2,-1) node[midway,left] {\tiny $\randomxof{\categoricalmap}$};

\node[anchor=center] (text) at (7,0) {${=}$};


\begin{scope}[shift={(10,2)}]

\newcommand{\conposseldec}{4.5,-5.5}

\draw[fill] (\conposseldec) circle (0.15cm);
\draw (\conposseldec) -- (4.5,-7.5) node[midway, right] {\tiny $\randomxof{\categoricalmap}$};
%!TEX encoding = UTF-8 Unicode\draw[dashed] (3.5,-7.5) rectangle (5.5, -9.5);
%\node[anchor=center] (text) at (4.5,-8.5) {\small $\ones$};

\draw[<-] (0,1) -- (0,-1) node[midway,left] {\tiny $\randomxof{0}$};
\draw (-1,-1) rectangle (1, -3);
\node[anchor=center] (text) at (0,-2) {\small $\categoricalcoreof{0}$};
\draw[<-] (0,-3) to[bend right=20] (\conposseldec);


\draw[<-] (3,1) -- (3,-1) node[midway,left] {\tiny $\randomxof{1}$};
\draw (2,-1) rectangle (4, -3);
\node[anchor=center] (text) at (3,-2) {\small $\categoricalcoreof{1}$};
\draw[<-] (3,-3) to[bend right=20]  (\conposseldec);

\node[anchor=center] (text) at (6,-2) {$\cdots$};

\draw[<-] (9,1) -- (9,-1) node[midway,left] {\tiny $\randomxof{\atomorder-1}$};
\draw (7.75,-1) rectangle (10.25, -3);
\node[anchor=center] (text) at (9,-2) {\small $\categoricalcoreof{\atomorder-1}$};
\draw[<-] (9,-3) to[bend left=20]  (\conposseldec);




\end{scope}

		


\end{tikzpicture}
\end{center}
\caption{Representation of a categorical constraint in a $\cpformat$ Format tensor network.
	a) Representation of the dependency of the graphical model.
	b) Tensor Representation with further network decomposition.
	We average by contraction with the dashed tensor $\ones$, if we do not specify the active atom.
	}
\label{fig:CategoricalDecomposition}
\end{figure}

\begin{remark}[Combination of Constraints]
	We can combine constraint cores by Hadamard products in the dual tensor network representation, as long as they can be satisfied together.
	An example, where this is not the case, are the categorical constraints to the three sets
		\[ \{\randomxof{0},\randomxof{1},\randomxof{2},\randomxof{3}\} \, , \, \{\randomxof{0},\randomxof{1}\}\, ,\,\{\randomxof{2},\randomxof{3}\} \, . \] 
	Besides the categorical cores also the datacores have a similar bayesian network affecting the atoms by another hidden variable.
	Combining both is welldefined, only when all datapoints satisfy the categorical constraints (that is only one of the atoms in each constraint is active).
\end{remark}


\begin{example}[Sudoku]\label{exa:sudoku}
	An interesting example, where categorical constraints are combined is Sudoku, the game of assigning numbers to a grid (see for example Section~5.2.6 in \cite{russel_artificial_2021}).
	The basic variables therein are $\catvariableof{i,j}$, with $\catdimof{i,j}=n^2$ and $i,j\in[n^2]$.
	By understanding $i$ as a line index and $j$ as a column index, they are ordered in a grid as sketched in Figure~\ref{fig:sudokuGrid} in the case $n=3$.
		
	For a $n\in\nn$ we further define the atomization variables $\catvariableof{i,j,k}$ where $i,j,k\in[n^2]$ and $\catdimof{i,j,k}=2$.
	These $n^6$ variables are the booleans indicating whether a specific position has a specific number assigned.
	The consistency of the atomization variables to the basic variables is then for each $i,j\in[n^2]$ ensured by the constraints
		\[ \{\catvariableof{i,j,k} \, : \, k \in [n^2] \} \, . \]
	
	We further have $3\cdot n^2$ constraints by the
	\begin{itemize}
		\item Row constraints: Each number $k$ appears exactly once in each row $i\in[n^2]$, captured by the constraints
			\[ \{\catvariableof{i,j,k}  \, : \, j \in [n^2] \} \, . \]
		\item Column constraints: Each number $k$ appears exactly once in each column $j\in[n^2]$, captured by the constraints
			\[ \{\catvariableof{i,j,k}  \, : \, i \in [n^2] \} \, . \]
		\item Square constraints: Each number appears exactly once in each square $s,r\in[n]$, captured by the constraints
			\[ \{\catvariableof{i+n\cdot s,j+n\cdot r,k}  \, : \, i,j \in [n] \} \, . \]
	\end{itemize}
	
	In total we have $3\cdot n^2 + n^4$ constraints for $n^6$ variables.

	\begin{figure}\label{fig:sudokuGrid} 
	\begin{center}
		\input{PartII/tikz_pics/network_representation/sudoku_grid.tex}
	\end{center}
	\caption{
	Sudoku grid of basic categorical variables $\catvariableof{i,j}$, here drawn in the standard case of $n=3$, each with dimension $\catdim=n^2=9$.
	Each basic categorical variables has $n^2$ corresponding atomization variables, which are further atomization variables to the row, column and squares constraints.
	Instead of depicting those constraints by hyperedges in a variable dependency graph, we here just indicate their existence through row, column and squares blocks.
	}
	\end{figure}

	\red{Reasoning by Entailment propagation! 
	Also, probabilistic choices possible when exact (!) contraction at a position not a basis vector, then can choose one possibility.}

\end{example}



\subsection{Hybrid Logic Network}

Markov Logic Networks are by definition positive distributions.
In contrary, Hard Logic Networks model uniform distributions over model sets of the respective Knowledge Base and therefore have vanishing coordinates.
We now show how to combine both approaches by defining Hybrid Logic Networks, when understanding Hard Logic Networks as base measures.
This trick is known to the field of variational inference, see for Example~3.6 in \cite{wainwright_graphical_2008}. 

\begin{definition}\label{def:hln}
	Given a set of formulas $\softformulaset$ with weights $\canparam$ and set $\hardformulaset$ of formulas, which conjunction is satisfiable, the hybrid logic network is the probability distribution
	\begin{align*}
		\probtensorof{(\softformulaset,\canparam,\hfbasemeasure)}[\shortcatvariables] 
		= \normationof{
		\{\exformula : \exformula\in\hardformulaset\} \cup \{\expof{\weightof{\exformula}\cdot\exformula} : \exformula\in\softformulaset\}
		}{\shortcatvariables} \, ,
	\end{align*}
	which is the member of the exponential family with statistic by $\softformulaset$ and the base measure
		\[ \hfbasemeasure[\shortcatvariables] = \contractionof{\{\formula : \formula \in \hardformulaset\}}{\shortcatvariables} \, .\]
		
	Given a set of formulas $\formulaset$, we define the set of hybrid logic networks realizable with $\formulaset$ and elementary activation cores as
	\begin{align*}
		\hlnsetof{\formulaset} 
		= \bigcup_{\secformulaset\subset\formulaset \, , \, \meanparam\in\{0,1\}^{\cardof{\formulaset}}}
		\expfamilyof{\formulaset/\secformulaset,\basemeasureof{\secformulaset,\meanparam}}
	\end{align*}
	where we denote base measures
	\begin{align*}
		\basemeasureofat{\secformulaset,\meanparam}{\shortcatvariables}
		= \bigwedge_{\enumformula\in\secformulaset} \lnot^{(1-\meanparamat{\indexedselvariable})} \enumformulaat{\shortcatvariables} \, . 
	\end{align*}
\end{definition}

The assumption of a satisfiable set $\hardformulaset$ is necessary, as we show next.

\begin{theorem}
	If any only if $\bigwedge_{\formula\in\hardformulaset}\formula$ is satisfiable, the tensor 
		\[  \contractionof{
		\{\exformula : \exformula\in\hardformulaset\} \cup \{\expof{\weightof{\exformula}\cdot\exformula} : \exformula\in\softformulaset\}
		}{\shortcatvariables} \]
	is normable.
\end{theorem}
\begin{proof}
	We need to show that
	\begin{align}\label{eq:tbsWellDefinedHLN}
		\contraction{\{\exformula : \exformula\in\hardformulaset\} \cup \{\expof{\weightof{\exformula}\cdot\exformula} : \exformula\in\softformulaset\}} > 0 \, . 
	\end{align}
	Since the conjunction of $\hardformulaset$ is satisfiable we find a $\shortcatindices$ with $\formulaat{\indexedcatvariableof{[\catorder]}}=1$ for all $\exformula\in\hardformulaset$.
	Then 
	\begin{align*}
		 \contractionof{\{\exformula : \exformula\in\hardformulaset\} \cup \{\expof{\weightof{\exformula}\cdot\exformula} : \exformula\in\softformulaset\}}{\indexedcatvariableof{[\catorder]}}  
		 & = \left( \prod_{\exformula\in\hardformulaset}\formulaat{\indexedcatvariableof{[\catorder]}} \right) 
		 \cdot \left( \prod_{\exformula\in\softformulaset}\expof{\weightof{\exformula}\cdot\exformula}[\indexedcatvariableof{[\catorder]}] \right) \\
		 & =  \left( \prod_{\exformula\in\softformulaset}\expof{\weightof{\exformula}\cdot\exformula}[\indexedcatvariableof{[\catorder]}] \right) \\
		 & > 0 \, . 
	\end{align*}
	Condition \eqref{eq:tbsWellDefinedHLN} follows from this and the Hybrid Logic Network is well-defined.
\end{proof}


\subsubsection{Tensor Network Representation}

We can employ the formula decompositions to represent both probabilistic facts of the MLN and hard facts (seen as the limit of large weights).

\begin{theorem}\label{the:hybridNetworkRepresentation}
	For any hybrid logic network we have
	\begin{align*}
		\probtensorof{(\softformulaset,\canparam,\hardformulaset)}[\shortcatvariables] 
		= \normationof{
		\{\formulaccwith : \exformula\in\softformulaset\cup\hardformulaset \}
		\cup \{\tbasisat{\formulavar} : \exformula\in\hardformulaset \}
		\cup \{\actcoreofat{\exformula}{\formulavar} : \exformula\in\softformulaset \}
		}{\shortcatvariables} \, . 
	\end{align*}
\end{theorem}
\begin{proof}
	By Lemma~\ref{lem:formulaEncodingDecomposition}.
\end{proof}

% Discussion
While the statistics computing cores in the relational encoding are shared to compute the soft and the hard logic formulas, their activation cores differ.
While probabilistic soft formulas get activation cores (see Theorem~\ref{the:mlnTensorRep})
\begin{align*}
	\actcoreofat{\exformula}{\formulavar} 
	= \begin{bmatrix} 1 \\
		 \expof{\canparamat{\indexedselvariableof{}}} 
		 \end{bmatrix}[\enumformulavar] \,
\end{align*}
the hard formulas get activation cores by unit vectors
\begin{align*}
	\tbasisat{\formulavar} 
	= \begin{bmatrix} 0 \\
		 1
	 \end{bmatrix}[\enumformulavar]  \, .
\end{align*}
As shown in Section~\ref{sec:hardLogicLimit}, the soft activation cores converge to these hard activation cores in the limit of large parameters, when imposing a local normation.
%
We further notice, that the probabilistic activation cores are trivial tensors if and only if the corresponding parameter coordinate vanishes.


%The reason for this is the Slicing Theorem, enabling the operations by both (exponentiation and selection of one slice) by the activation cores.
For an example see Figure~\ref{fig:ActivatedHeads}.

\begin{figure}[h]
\begin{center}
	\input{PartII/tikz_pics/network_representation/activated_heads.tex}
\end{center}
\caption{Diagram of a formula tensor with activated heads, containing \textcolor{\concolor}{hard constraint cores} and \textcolor{\probcolor}{probabilistic weight cores} .} %along \textcolor{\inactivecolor}{inactive cores}.}
\label{fig:ActivatedHeads} 
\end{figure}



\begin{remark}{Probability interpretation using the Partition function}
	The tensor networks here represent unnormalized probability distributions.
	The probability distribution can be normed by the quotient with the naive contraction of the network, the partition function.
\end{remark}


\subsubsection{Reasoning Properties}

Deciding probabilistic entailment (see Definition~\ref{def:probEntailment}) with respect to Hybrid Logic Networks can be reduced to the hard logic parts of the network.

\begin{theorem}\label{the:hlnEntailmentReduction}
	Let $(\softformulaset,\canparam,\hardformulaset)$ define a Hybrid Logic Network.
	Given a query formula $\exformula$ we have that 
		\[ \probtensorof{(\softformulaset,\canparam,\hardformulaset)} \models \exformula \]
	if and only if
		\[ \hardformulaset \models \exformula \, . \]
\end{theorem}
\begin{proof}
	This follows from Theorem~\ref{the:factorReduction} on the representation of Hybrid Logic Networks as Markov Networks in Theorem~\ref{the:hybridNetworkRepresentation}.
\end{proof}


Formulas in $\softformulaset$, which are entailed or contradicted by $\hardformulaset$ are redundant, as we show next.

\begin{theorem}%\label{the:hlnRepRedundancy}
	If for a formula $\exformula$ and $\hardformulaset$ we have 
		\[ \hardformulaset \models \exformula \, \quad \text{or} \quad \hardformulaset \models \lnot\exformula \]
	then for any $(\softformulaset,\canparam,\hardformulaset)$
		\[ \probofat{(\softformulaset/\{\exformula\},\tilde{\canparam},\hardformulaset)}{\shortcatvariables} =  \probofat{(\softformulaset,\canparam,\hardformulaset)}{\shortcatvariables}  \, , \]
	where $\tilde{\canparam}$ denotes the tensor $\canparam$, where the coordinate to $\exformula$ is dropped, if $\exformula\in\softformulaset$.
\end{theorem}
\begin{proof}
	Isolate the factor to the hard formula, which is constant for all situations.
\end{proof}

%% Now in the 
A similar statement holds for the hard formulas itself, as shown in Theorem~\ref{the:ReduncancyOfEntailed}.
However, notice that if $\hardformulaset/\{\exformula\}\models\lnot\exformula$, then $\hardformulaset\cup\{\exformula\}$ is not satisfiable and a hybrid logic network cannot be defined for $\hardformulaset\cup\{\exformula\}$ as hard logic formulas.

%If the conjunction of $\hardformulaset/\{\exformula\}$ entails $\exformula$, we can erase $\exformula$ from $\hardformulaset$ without changing the contraction, therefore without changing the base measure of the Hybrid Logic Network.

% Utility in Contraction KB implementation
These results are especially interesting for the efficient implementation of Algorithm~\ref{alg:TensorKB}, which has been introduced in Chapter~\ref{cha:logicalReasoning}.
By Theorem~\ref{the:hlnEntailmentReduction} only the hard logic parts of a Hybrid Logic Network are required in the ASK operation.
%Theorem~\ref{the:hlnRepRedundancy} for the TELL operation, but already discussed in more generality

\subsubsection{Expressivity}

Hybrid Logic Networks extend the expressivity result of Theorem~\ref{the:mintermExpressivityMLN} to arbitrary probability tensors, dropping the positivity constraints for Markov Logic Networks.

\begin{theorem}\label{the:mintermExpressivityHLN}
	Let $\probat{\shortcatvariables}$ a possibly not positive probability tensor we build a base measure
		\[ \hfbasemeasure = \nonzeroof{\probat{\shortcatvariables}} \]
	and a parameter tensor
	\begin{align*}
		\canparamat{\selvariableof{[\catorder]}=\shortcatindices}
		= \begin{cases}
			0 & \text{if} \quad \probat{\indexedshortcatvariables} = 0  \\
			\lnof{\probat{\indexedshortcatvariables}} & \text{else} 
		\end{cases} \, . 
	\end{align*}
	Then the probability tensor is the member of the minterm exponential family with base measure $\hardformulaset$ and parameter $\canparam$, that is
		\[ \probof{(\mintermformulaset,\canparam,\hfbasemeasure)}\]
\end{theorem}
\begin{proof}
	It suffices to show that 
		\[ \sbcontractionof{\hfbasemeasure, \expof{\contractionof{
		\sencodingof{\mintermformulaset}\canparam
		}{
		\shortcatvariables
		}}}{\shortcatvariables} = \probat{\shortcatvariables} \, . \]
	For indices $\shortcatindices$ with $\probat{\indexedshortcatvariables}=0$ we have $\hfbasemeasureat{\indexedshortcatvariables}=0$ and thus also 
		\[ \sbcontractionof{\hfbasemeasure, \expof{\contractionof{
		\sencodingof{\mintermformulaset}\canparam
		}{
		\shortcatvariables
		}}}{\indexedshortcatvariables} = 0 \, . \]
	For indices $\shortcatindices$ with $\probat{\indexedshortcatvariables}>0$ we have $\hfbasemeasureat{\indexedshortcatvariables}=1$ and
	\begin{align*}
		 \sbcontractionof{\hfbasemeasure, \expof{\contractionof{
		\sencodingof{\mintermformulaset},\canparam
		}{
		\shortcatvariables
		}}}{\indexedshortcatvariables} 
		&= \prod_{\selindexof{[\catorder]}} \expof{\canparamat{\selvariableof{[\catorder]}=\selindexof{[\catorder]}} \cdot \mintermofat{\selindexof{[\catorder]}}{\indexedshortcatvariables}} \\
		&=  \expof{\canparamat{\selvariableof{[\catorder]}=\shortcatindices}} \\
		&=  \probat{\indexedshortcatvariables} \, .
	\end{align*}
\end{proof}

\subsection{Applications}

Hybrid Logic Networks as neuro-symbolic architectures:
\begin{itemize}
	\item Neural Paradigm here by decompositions of logical formulas into their connectives.
		In more generality by decompositions of sufficient statistics into composed functions, using Basis Calculus.
		Deeper nodes as carrying correlations of lower nodes.
	\item Symbolic Paradigm by interpretability of propositional logics.
\end{itemize}


Hybrid Logic Networks as trainable Machine Learning models:
\begin{itemize}
	\item Expressivity: Can represent any positive distribution, as shown by Theorem~\ref{the:maximalClausesRepresentation}, with $2^d$ formulas.
	\item Efficiency: Can only handle small subsets of possible formulas, since their possible number is huge.
		Tensor networks provide means to efficiently represent formulas depending on many variables and reason based on contractions.
	\item Differentiability: Distributions are differentiable functions of their weights, see Parameter Estimation Chapter. 
		The log-likelihood of data is therefore also differentiable function of their weights and we can exploit first-order methods in their optimization.
	\item Structure Learning: We need to find differentiable parametrizations of logical formulas respecting a chosen architecture.
		In Chapter~\ref{cha:formulaBatches} such representations are described based on Selector Tensor Networks.
\end{itemize}
Differentiability and structure learning will be investigated in more detail in the next chapter.

\red{When understanding atoms as observed variables, and the computed as hidden, Hybrid Logic Networks are deep higher-order boltzmann machines: 
More generic correlations can be captured by a logical connective, calculated by a relational encoding and activated by an activation core.}

\red{Hybrid Logic Networks as bridging soft and hard logics within the formalism of exponential families.}

%\begin{remark}[Hopfield networks]
%	Also interesting for MLNs is a Hopfield perspective.
%	Having an initialization the update can be interpreted as a Gibbs sampling step at temperature $0$ (since deterministic update).
%\end{remark}


% CSP
A more general class of problems, which have natural representations by Hard Logic Networks are Constraint Satisfaction Problems (see Chapter~5 in \cite{russell_artificial_2021}).
Solving such problems is then equivalent to sampling from the worlds in a logical interpretation, and can be approached by the methods of Chapter~\ref{cha:logicalReasoning}.
Among these classed, we have only discussed the Sudoku game in Example~\ref{exa:sudoku}.
Extensions by Hybrid Logic Networks can be interpreted as implementations of preferences among possible solutions by probabilities.





    \chapter{\chatextnetworkReasoning}\label{cha:networkReasoning}

In this chapter we investigate the inference properties of \HybridLogicNetworks{}, exploiting the characterizations of the corresponding mean parameter polytopes in \charef{cha:networkRepresentation}.
We investigate unconstrained parameter estimated for \MarkovLogicNetworks{} and \HybridLogicNetworks{}, which are special cases of the backward maps introduced in \charef{cha:probRepresentation}.
We then motivate structure learning based on sparsity constraints on the parameters on the minterm exponential family and present heuristic strategies leading to efficient structure learning algorithms.




\sect{Entropy Optimization} \label{sec:parameterEstimation} % on \HybridLogicNetworks{} Check for redundancy with the mln introduction chapter!

We now motivate \HybridLogicNetworks{} as distributions with maximum entropy under a moment constraint.


%We will investigate two entropic approaches towards unconstrained parameter estimation.
%First of all, we characterize Maximum Entropy distributions with a moment constraint.
%Maximum Likelihood Estimation restricts the distributions optimized over to a specific set, here the \HybridLogicNetworks{}.

\subsect{Entropy Maximization}

The Maximum Entropy Problem $\probtagtypeinst{\entropysymbol}{\mlnstat,\basemeasure,\genmean}$ for a boolean statistic $\mlnstat$ %and a base measure $\basemeasure$ is %\MarkovLogicNetworks{} is
\begin{align}
    \label{prob:maxEntropyHLN}\tag{$\probtagtypeinst{\entropysymbol}{\mlnstat,\genmean}$}
    \argmax_{\probtensor\in\alldists} \quad \sentropyof{\probtensor}
    \stspace
    \contractionof{\probtensor,\sencmlnstat}{\selvariable}
    =  \genmeanwith
\end{align}
where by $\alldists$ we denote all probability distributions.

\begin{theorem}
    \label{the:maxEntropyCharacterizationHLN}
    Let $\genmeanwith\in\genmeanset$ and the minimal face of $\genmeanset$, which includes $\genmeanwith$, be $\genfaceset$, and let
    \begin{align*}
        \gencanparam = \backwardmapwrtof{\mlnstat,\hlnfacemeasure}{\genmeanwith} \, ,
    \end{align*}
    where $\backwardmapwrt{\mlnstat,\hlnfacemeasure}$ is a backward map in the exponential family $\expfamilyof{\mlnstat,\hlnfacemeasure}$ and $\hlnfacemeasure$ is the face measure to $\genfaceset$ (see \theref{the:faceMeasureCharacterizationHLN}).
    The solution of the Maximum Entropy \probref{prob:maxEntropyHLN} is then
    \begin{align*}
        \probwith = \expdistofat{(\mlnstat,\gencanparam,\hlnfacemeasure)}{\shortcatvariables} \, .
    \end{align*}
\end{theorem}
\begin{proof}
    Any feasible distribution has to be representable by $\hlnfacemeasure$, since $\genmeanwith\in\hlnfaceset$.
    The solution therefore coincides with the solution of the maximum entropy problem with respect to $\mlnstat$ and $\hlnfacemeasure$ as a base measure.
    Since $\genfaceset$ is minimal, $\meanparamwith$ is on the effective interior of the face and the claim follows with \theref{the:maxEntropyFace}.
\end{proof}

\theref{the:maxEntropyCharacterizationHLN} characterizes the solution of the Maximum Entropy Problem for arbitrary positions of the mean parameter.
Note, that if $\genmeanwith\notin\genmeanset$ then no distribution is feasible for \probref{prob:maxEntropyHLN} and there is no solution.
We are especially interested in situations, where the solution is a Hybrid Logic Network.
As we state next, this is exactly the case if the mean parameter is reproducable by a Hybrid Logic Network (see \secref{sec:HLNrepMean}).

\begin{theorem}
    %Let $\mlnstat$ be a boolean statistic and $\genmeanwith\in\hlnmeanset$.
    The solution of the Maximum Entropy \probref{prob:maxEntropyHLN} is a Hybrid Logic Network, if and only if $\genmeanwith$ is reproducable by a Hybrid Logic Network.
\end{theorem}
%\begin{proof}
%
%\end{proof}
%
%
%\begin{theorem}
%    Let $\genmeanwith$ be a mean parameter reproducable by a Hybrid Logic Network. % (see \defref{def:HLNrepMean}).
%    Then the Hybrid Logic Network reproducing $\genmeanwith$ is the solution of the Maximum Entropy Problem \eqref{prob:maxEntropyHLN}.
%\end{theorem}
\begin{proof}
    We notice, that if and only if the solution of the Maximum Entropy Distribution is a \HybridLogicNetwork{}, then the normalization of the corresponding face measure $\hlnfacemeasure$ of the minimal face containing $\genmeanwith$ is in $\elrealizabledistsof{\mlnstat}$.
    This is equivalent to $\genmeanwith$ being reproducable by a Hybrid Logic Network.
    %The solution of \probref{prob:maxEntropyHLN} characterized in \theref{the:maxEntropyCharacterizationHLN} is then also in $\elrealizabledistsof{\mlnstat}$.
\end{proof}


\subsect{Cross Entropy Minimization}

Different to Maximum Entropy Problems, we formulate the Maximum Likelihood Problem as cross entropy minimization with respect to \HybridLogicNetworks{}, that is
\begin{align}
    \label{prob:minCrossEntropyHLN}\tag{$\probtagtypeinst{\mathrm{M}}{\Gamma,\gendistribution}$}
    \argmin_{\probtensor\in\elrealizabledistsof{\mlnstat}} \centropyof{\gendistribution}{\probtensor}
\end{align}
When choosing $\gendistribution$ by an empirical distribution, this minimization problem is the Maximum Likelihood Problem (see \secref{cha:probReasoning}).
% LEMMA TO CHA:PROBREASONING?
In order to characterize the solution of the cross entropy minimization problem on \HybridLogicNetworks{}, we first characterize the solution of the cross entropy minimization on exponential families.


To solve the maximum likelihood problems on \HybridLogicNetworks{} we choose the parametrization by tuples $\hybridparamin$ and
\begin{align*}
    \min_{\hybridparamin} \centropyof{\gendistribution}{\probtensorof{\hlnparameters}}
\end{align*}



\begin{lemma}
    \label{lem:minCrossEntropyHardparam}
    For any $\canparam$ we have
    \begin{align*}
        \min_{\hardparam} \centropyof{\gendistribution}{\probtensorof{\hlnparameters}}
        = \centropyof{\gendistribution}{\probtensorof{\mlnstat,(\hardlegsetto{\genmean},\hardlegindicesto{\genmean},\canparam)}}
    \end{align*}
    where $(\hardlegsetto{\genmean},\hardlegindicesto{\genmean})$ parametrize the smallest cube face containing $\genmeanwith=\contractionof{\gendistributionwith,\sencmlnstatwith}{\selvariable}$.
\end{lemma}
\begin{proof}
    We decompose the cross entropy in three terms
    \begin{align*}
        \centropyof{\gendistribution}{\probtensorof{\hlnparameters}}
        =& \contraction{\gendistributionwith,-\lnof{\hlnformulawith}} \\
        &+ \contraction{\gendistributionwith,\sencmlnstatwith,-\lnof{\actcorewith}} \\
        &+ \lnof{\contraction{\mlnstatccwith,\kcoreofat{\hardparam}{\shortcatvariables},\actcoreofat{\canparam}{\shortcatvariables}}} \, .
    \end{align*}
    The first term can be characterized by
    \begin{align*}
        \contraction{\gendistributionwith,-\lnof{\hlnformulawith}}
        =\begin{cases}
             0 & \ifspace \gendistribution\models\hlnformula \\
             \infty & \ifspace \gendistribution\not\models\hlnformula
        \end{cases} \, .
    \end{align*}

    The cross entropy is therefore finite, if and only if $\hardlegset\subset\hardlegsetto{\genmean}$ and $\headindexof{\hardlegset}=\restrictionofto{\hardlegindicesto{\genmean}}{\hardlegset}$.
    Among those tuples, only the third term of the cross-entropy varies, where we have
    \begin{align*}
        \lnof{\contraction{\mlnstatccwith,\kcoreof{\hardparam},\actcoreof{\canparam}}} \leq \lnof{\contraction{\mlnstatccwith,\kcoreof{(\hardlegsetto{\genmean},\hardlegindicesto{\genmean})},\actcoreof{\canparam}}}
    \end{align*}
    The minimum of the cross entropy is therefore taken at $\hardlegset=\hardlegsetto{\genmean}$ and $\headindexof{\hardlegset}=\hardlegindicesto{\genmean}$.
\end{proof}

We therefore know
\begin{align*}
    \min_{\canparam\in\parspace} \centropyof{\gendistribution}{\probtensorof{\mlnstat,(\hardlegsetto{\genmean},\hardlegindicesto{\genmean},\canparam)}}
\end{align*}

We can therefore characterize the minimum using \lemref{lem:minCrossEntropyExponential}.

\begin{theorem}
    \label{the:minCrossEntropyHLN}
    Let $\gendistributionwith$ be a distribution, $\mlnstat$ a boolean statistic.%, and choose a subset $\variableset\subset[\seldim]$ and a boolean tuple $\headindexof{\variableset}$.
    We build the mean parameter $\genmeanwith=\contractionof{\gendistributionwith,\sencmlnstatwith}{\selvariable}$ and have the following:
%    For any $\variableset\subset[\seldim]$ and boolean tuple $\headindexof{\variableset}$ we have
    \begin{itemize}
        \item[(1)] If $\genmeanwith \in \sbinteriorof{\meansetof{\mlnstat,\trivbm}}$ then
        \begin{align*}
            \min_{\hybridparamin} \centropyof{\gendistribution}{\probtensorof{\hlnparameters}}
            = \sentropyof{\probtensorof{\mlnstat,(\hardlegsetto{\genmean},\hardlegindicesto{\genmean},\estcanparam)}}
        \end{align*}
        where $\estcanparam=\backwardmapwrtof{\mlnstat,\hlnformulato{\genmeanwith}}{\genmean}$.
        \item[(2)] If $\genmeanwith \notin \sbinteriorof{\meansetof{\mlnstat,\trivbm}}$ and $\genmeanwith\in\closureof{\meansetof{\mlnstat,\trivbm}}$ then there is a sequence $\left(\meanparamofat{n}{\selvariable}\right)_{n\in\nn}\subset\hlnmeanset$ converging coordinatewise to $\genmeanwith$ and
        \begin{align*}
            \min_{\hybridparamin} \centropyof{\gendistribution}{\probtensorof{\hlnparameters}}
            = \lim_{\meanparamofat{n}{\selvariable}\rightarrow\genmeanwith}
            \sentropyof{\probtensorof{\mlnstat,(\hardlegsetto{\meanparamof{n}},\hardlegindicesto{\meanparamof{n}},\canparamof{n})}}\,
        \end{align*}
        where $\canparamof{n}=\backwardmapwrtof{\mlnstat,\hlnformulato{\meanparamof{n}}}{\meanparamof{n}}$.
        \item[(3)] If $\genmeanwith\notin\closureof{\meansetof{\mlnstat,\trivbm}}$ then
        \begin{align*}
            \min_{\hybridparamin} \centropyof{\gendistribution}{\probtensorof{\hlnparameters}}
            = \infty \, .
        \end{align*}
    \end{itemize}
\end{theorem}
\begin{proof}
    This follows from \lemref{lem:minCrossEntropyExponential}.
\end{proof}

%Based on \lemref{lem:minCrossEntropyExponential} we can characterize the solution of the Maximum Likelihood Problem \probref{prob:minCrossEntropyHLN} with respect to \HybridLogicNetworks{}.
%
%\begin{theorem}
%    Let $\gendistribution$ be any distribution, such that $\genmeanwith$ is reproducable by a Hybrid Logic Network.
%    Then the solution of the Maximum Likelihood Problem \probref{prob:minCrossEntropyHLN} is the Hybrid Logic Network $\probtensorof{\mlnstat,\backwardmapwrtof{\mlnstat,\hlnformula}{\genmeanwith},\hlnformula}$.
%\end{theorem}
%\begin{proof}
%    We exploit the parameterization of \HybridLogicNetworks{} by tuples $\hlnformulaparams$ (parametrizing the corresponding \HardLogicNetwork{}) and canonical parameters $\canparamwith$ (parametrizing the corresponding \MarkovLogicNetwork{}).
%    The cross entropy minimization is then
%    \begin{align*}
%        \min_{\variableset\subset[\seldim]}\min_{\headindexof{\variableset}\in\bigtimes_{\selindex\in\variableset}[2]} \left(\min_{\canparamwithin} \centropyof{\gendistribution}{\probtensorof{\mlnstat,\canparamwith,\hlnformula}}\right) \, .
%    \end{align*}
%    For any pair $\variableset,\headindexof{\variableset}$ we apply the lemma above and get a characterization of the inner minimum
%    \begin{align*}
%        \min_{\canparamwithin} \centropyof{\gendistribution}{\probtensorof{\mlnstat,\canparamwith,\hlnformula}}
%    \end{align*}
%    dependent on the stated three cases.
%    There is exactly one face $\genfaceset$, for which $\genmeanwith\in\sbinteriorof{\genfaceset}$ and by assumption of reproducability there is a parameter tuple $\hlnformulaparams$ such that $\hlnformula$ is the face measure of $\genfaceset$.
%    For this parameter tuple we have
%    \begin{align*}
%        \min_{\canparamwithin} \centropyof{\gendistribution}{\probtensorof{\mlnstat,\canparamwith,\hlnformula}}
%        = -\sentropyof{\probtensorof{\mlnstat,\backwardmapwrtof{\mlnstat,\hlnformula}{\genmean},\hlnformula}} \, .
%    \end{align*}
%    The following two arguments show that this is the minimum among all other faces.
%    If for another tuple $\sechlnformulaparams$ we have $\genmeanwith\notin\bmrealprobof{\hlnformulaof{\sechlnformulaparams}}$ (i.e. the third case in \lemref{lem:minCrossEntropyExponential}) then the respective inner minimum is $\infty$ and the outer minimum is not taken.
%    If for another tuple $\sechlnformulaparams$ we have $\genmeanwith\in\bmrealprobof{\hlnformulaof{\sechlnformulaparams}}$ but $\genmeanwith\notin\sbinteriorof{\genmeanset\cup\cubeface^{\sechlnformulaparams}}$ (i.e. the second case in \lemref{lem:minCrossEntropyExponential}), then the face $\genmeanset\cap\cubeface^{\hlnformulaparams}$ is contained in $\genmeanset\cap\cubeface^{\sechlnformulaparams}$ and the minimum is taken on that face.
%\end{proof}


When $\genmeanwith$ is not reproduceable by a \HybridLogicNetwork{}, we are in the case where the smallest face, such that $\genmeanwith$ is contained is not an intersection of $\genmeanset$ with a cube face.
In this case, there is no solution of the Maximum Likelihood \probref{prob:minCrossEntropyHLN} in the set of \HybridLogicNetworks{}, since the minimum is not taken.
The reason for this lies in the expressivity problem of \HybridLogicNetworks{}, which do not reproduce the interior of such faces, but tend in a limit of large canonical parameters to any mean parameter on such faces.

%%% GENERALIZATION: TO BE SHOWN
%\begin{theorem}
%    Let $\gendistribution$ be any distribution, $\genmeanwith=\contractionof{\gendistribution,\sencmlnstatwith}{\selvariable}$ and $\hlnfaceset$ be the smallest face of $\hlnmeanset$, which contains $\genmeanwith$.
%    Then the cross entropy minimizer over
%    \begin{align*}
%        \probtensorset = \bigcup_{\facecondset} \expfamilyof{\mlnstat,\hlnfacemeasure}
%    \end{align*}
%    is the distribution $\probtensorof{\mlnstat,\backwardmapwrtof{\mlnstat,\hlnfacemeasure}{\genmeanwith},\hlnformula}$, where $\hlnfacemeasure$ is the face measure of $\hlnfaceset$.
%\end{theorem}
%\begin{proof}
%
%\end{proof}


%We can instead state a Maximum Likelihood Problem over $\maxrealizabledistsof{\mlnstat}$, which provides enough expressivity to represent all faces.
%\begin{theorem}
%    Let $\gendistribution$ be any distribution, $\genmeanwith=\contractionof{\gendistribution,\sencmlnstatwith}{\selvariable}$ and $\genfaceset$ be the smallest face of $\genmeanset$, which contains $\genmeanwith$.
%    Then the solution of the Maximum Likelihood Problem \probref{prob:minCrossEntropyHLN} over $\realizabledistsof{\mlnstat,\maxgraph}$ is $\probtensorof{\mlnstat,\backwardmapwrtof{\mlnstat,\hlnfacemeasure}{\genmeanwith},\hlnfacemeasure}$, where $\hlnfacemeasure$ is the face measure of $\genfaceset$.
%\end{theorem}
%\begin{proof}
%    Again by application of \lemref{lem:minCrossEntropyExponential} on each face of $\genmeanset$.
%    %The outer minimum is taken at the smallest face
%\end{proof}







\sect{Forward and backward mappings}

Forward and backward mappings have been introduced for exponential families in \charef{cha:probRepresentation}.
We now generalize them to \HybridLogicNetworks{}, which are parametrized by tuples $\hybridparamin$.

%\subsect{Representation as Computation Activation Networks} %% Now in Represeentation Chapter
%
%We parametrize each elementary activation tensors $\extnetat{\headvariables}=\bigotimes_{\selindexin}\hypercoreofat{\selindex}{\headvariableof{\selindex}}$ by a tuple
%$\hybridparam$ as follows
%\begin{itemize}
%    \item The axis set of hard leg vectors
%    \begin{align*}
%        \variableset = \Big\{ \selindex \wcols \selindexin \ncond \hypercoreofat{\selindex}{\indexedheadvariableof{\selindex}=0}=0 \,\,\text{or}\,\,  \hypercoreofat{\selindex}{\indexedheadvariableof{\selindex}=1}=0  \Big\}
%    \end{align*}
%    \item The indices of the hard leg vector $\headindexof{\variableset}$ where for $\selindex\in\variableset$
%    \begin{align*}
%        \headindexof{\selindex} =
%        \begin{cases}
%            0 & \ifspace \hypercoreofat{\selindex}{\headvariableof{\selindex}=1}=0 \\
%            1 & \ifspace \hypercoreofat{\selindex}{\headvariableof{\selindex}=0}=0
%        \end{cases}
%    \end{align*}
%    Note, that exactly one of the cases is true, since both would imply a vanishing activation tensor.
%    \item A canonical parameter $\canparamat{\selvariable}$ where $\canparamat{\indexedselvariable}=0$ for $\selindex\in\variableset$ and for $\selindex\notin\variableset$
%    \begin{align*}
%        \canparamat{\indexedselvariable} = \lnof{
%            \frac{\hypercoreofat{\selindex}{\headvariableof{\selindex}=1}}{\hypercoreofat{\selindex}{\headvariableof{\selindex}=0}}
%        }
%    \end{align*}
%\end{itemize}
%Conversely, for a tuple $\hybridparam$ we define leg vectors
%\begin{align*}
%    \hypercoreofat{\selindex}{\headvariableof{\selindex}} =
%    \begin{cases}
%        \onehotmapofat{\headindexof{\selindex}}{\headvariableof{\selindex}} & \text{if} \quad \selindex\in\variableset \\
%        \expof{\canparamat{\indexedselvariable}\cdot \indexinterpretationof{\headvariableof{\selindex}}} & \text{else}
%    \end{cases}
%\end{align*}
%and an activation tensor
%\begin{align*}
%    \tnetofat{\hybridparam}{\headvariables}
%    =\bigotimes_{\selindexin}\hypercoreofat{\selindex}{\headvariableof{\selindex}} \, .
%\end{align*}
%
%Note that the tuple $\hybridparam$ to a elementary activation tensor determines the corresponding \HybridLogicNetwork{}, since it determines the activation tensor up to an irrelevant scalar multiplication.


%\subsect{Definition}

\begin{definition}
    The extended canonical parameter set to a boolean statistic is the set
    \begin{align*}
        \hybridparamset\coloneqq
        \{\hardparam)\wcols \variableset\subset[\seldim]\ncond \headindexof{\variableset}\in\bigtimes_{\selindex\in\variableset}[2]\} \times \parspace \, .
    \end{align*}
    The forward map for a \HybridLogicNetwork{} is
    \begin{align*}
        \forwardmapwrt{\mlnstat} :  \hybridparamset
        \rightarrow \hlnmeanset \subset \parspace
    \end{align*}
    where for $\hybridparamin$
    \begin{align*}
        \forwardmapwrtof{\mlnstat}{\hybridparam}
        = \contractionof{
            \normalizationof{\tnetofat{\hybridparam}{\headvariables},\bencodingofat{\mlnstat}{\headvariables,\shortcatvariables}}{\shortcatvariables}
            ,\sencodingofat{\mlnstat}{\shortcatvariables,\selvariable}}{\selvariable} \, .
    \end{align*}

    A backward map for a \HybridLogicNetwork{} is any map
    %\begin{align*}
    %    \backwardmapwrt{\mlnstat} :  \imageof{\forwardmapwrt{\mlnstat}} \rightarrow \eltnset
    %\end{align*}
    such that for any $\hybridparam\in\hybridparamset$ we have $\backwardmapwrtof{\mlnstat}{\forwardmapwrtof{\mlnstat}{\tnetof{\elgraph}}}$.

\end{definition}

% Expressivity implying image
From the expressivity study in \charef{cha:networkRepresentation} we know that for any $\meanparamwith$ there is a $\hybridparam\in\hybridparamset$ with $\forwardmapwrtof{\mlnstat}{\hybridparam}=\meanparamwith$, if and only if $\meanparamwith\in\elhlnmeanset$.
In particular, the this implies that the image of $\forwardmapwrt{\mlnstat}$ is the subset $\elhlnmeanset\subset\hlnmeanset$, which is the union of cube face interiors.

A backward map can be constructed as follows:
\begin{itemize}
    \item Choose $\variableset=\{\selindex\wcols\meanparamat{\indexedselvariable}\in\{0,1\}\}$, and for $\selindex\in\variableset$ $\headindexof{\selvariable}=\meanparamat{\indexedselvariable}$
    %\item and build
    %\begin{align*}
    %    \kcoreofat{\selindex}{\headvariableof{\selindex}}
    %\end{align*}
    \item Use the backward map of the exponential family $\expfamilyof{\mlnstat,\hlnformula}$ to compute
    \begin{align*}
        \canparamat{\selvariable} = \backwardmapwrtof{\mlnstat,\hlnformula}{\meanparamwith}
    \end{align*}
    %\item Set
    %\begin{align*}
    %    \hypercoreofat{\selindex}{\headvariableof{\selindex}} = \contractionof{\actcorewith,\kcoreofat{\selindex}{\headvariableof{\selindex}}}{\headvariableof{\selindex}}
    %\end{align*}
\end{itemize}

\subsect{Backward Maps in Closed Form}

\red{For the universal and the atomic family we have $\elhlnmeanset=\hlnmeanset$.
Furthermore, we can provide the backward mapping in a closed form expression.}

% Closed form availability
We recall from \charef{cha:probReasoning}, that while forward mappings are always in closed form by contractions, backward mapping in general do not have a closed form representation.
Instead, the backward map is in general implicitly characterized by a maximum entropy problem constrained to matching expected sufficient statistics.
We investigate in this section specific examples, where closed forms are available for both.
In these cases, parameter estimation can thus be solved by application of the inverse on the expected sufficient statistics with respect to the empirical distribution, and iterative algorithms can be avoided.

% Usage
%When the backward map $\backwardmap$ is available in closed form, we directly get optimal parameters by the inversion acting on the satisfaction rate and can avoid iterative algorithms of parameter estimation.

\subsubsect{Maxterms and Minterms}

Minterms (respectively maxterms) are ways in propositional logics to get a syntactical formula representation based on a formula to each world which is a model (respectively fails to be a model).
We have already studied in \secref{sec:MLNMaxMintermRep} how to represent any positive distribution by a distribution in the family of minterms (respectively maxterms), see \theref{the:maximalClausesRepresentation}.
Here we extend to the representation of distributions with arbitrary supports and provide forward and backward maps.

We use the tuple enumeration of the maxterms and minterms by $\atomstates$ introduced in \secref{sec:termClauseDecomposition}.
With respect to this enumeration the canonical parameters and mean parameters are tensors in $\bigotimes_{\atomenumeratorin}\rr^2$.

\begin{theorem}
    For the \HybridLogicNetworks{} to the minterm and maxterm statistics
    \begin{align*}
        \mintermformulaset \coloneqq \{ \mintermof{\shortcatindices} \wcols \shortcatindices\in\atomstates\}
        \andspace
        \maxtermformulaset \coloneqq \{ \maxtermof{\shortcatindices} \wcols \shortcatindices\in\atomstates\}
    \end{align*}
    we have the forward maps
    \begin{align*}
        \forwardmapwrtof{\mlnmintermsymbol}{\hybridparam}
        = \normalizationof{\bencodingofat{\mintermformulaset}{\headvariableof{\mlnmintermsymbol},\shortcatvariables},
            \tnetofat{\hybridparam}{\headvariableof{\mlnmintermsymbol}},\identityat{\shortcatvariables,\selvariableof{[\atomorder]}}}{\selvariableof{[\atomorder]}}
    \end{align*}
    and
    \begin{align*}
        \forwardmapwrtof{\mlnmaxtermsymbol}{\hybridparam}
        = \onesat{\selvariableof{[\atomorder]}} - \normalizationof{\bencodingofat{\maxtermformulaset}{\headvariableof{\mlnmaxtermsymbol},\shortcatvariables},
            \tnetofat{\hybridparam}{\headvariableof{\mlnmaxtermsymbol}},\identityat{\shortcatvariables,\selvariableof{[\atomorder]}}}{\selvariableof{[\atomorder]}} \, .
    \end{align*}

    Further, the map
    \begin{align*}
        \backwardmapwrtof{\mlnmintermsymbol}{\meanparamat{\selvariableof{[\atomorder]}}}
        = \Big(\Big\{ \selindex \wcols \selindexin \ncond \meanparamat{\indexedselvariable}=0 \Big\},0_{\variableset},\lnof{\meanparamat{\selvariableof{[\atomorder]}}}\Big)
    \end{align*}
    (we set here $\lnof{0}=0$) is a backward map for the minterm statistic.

    The map
    \begin{align*}
        \backwardmapwrtof{\mlnmaxtermsymbol}{\meanparamat{\selvariableof{[\atomorder]}}}
        = \Big(\Big\{ \selindex \wcols \selindexin \ncond \meanparamat{\indexedselvariable}=1 \Big\},1_{\variableset},-\lnof{\meanparamat{\selvariableof{[\atomorder]}}}\Big)
    \end{align*}
    (we set here $\lnof{0}=0$) is a backward map for the minterm statistic.
\end{theorem}
\begin{proof}
    The minterm statistic $\mintermformulaset$ has a selection encoding
    \begin{align*}
        \sencodingofat{\mintermformulaset}{\shortcatvariables,\selvariableof{[\atomorder]}}
        = \identityat{\shortcatvariables,\selvariableof{[\atomorder]}}
    \end{align*}
    and the mean parameter to any distribution $\probwith$ has therefore the coordinates to $\selindexof{[\atomorder]}\in\atomstates$ by
    \begin{align*}
        \meanparamat{\indexedselvariableof{[\atomorder]}}
        = \contractionof{\identityat{\shortcatvariables,\selvariableof{[\atomorder]}}}{\indexedselvariableof{[\atomorder]}}
        = \probat{\shortcatvariables=\selindexof{[\atomorder]}} \, .
    \end{align*}
    For the maxterm statistic $\maxtermformulaset$ we analogously have
    \begin{align*}
        \sencodingofat{\maxtermformulaset}{\shortcatvariables,\selvariableof{[\atomorder]}}
        = \onesat{\shortcatvariables,\selvariableof{[\atomorder]}}-\identityat{\shortcatvariables,\selvariableof{[\atomorder]}}
    \end{align*}
    and thus the mean parameter to any distribution $\probwith$ has therefore the coordinates to $\selindexof{[\atomorder]}\in\atomstates$ by
    \begin{align*}
        \meanparamat{\indexedselvariableof{[\atomorder]}}
        = 1-\probat{\shortcatvariables=\selindexof{[\atomorder]}} \, .
    \end{align*}
    The claim on the forward maps follows for
    \begin{align*}
        \probwith=\normalizationof{\bencodingofat{\maxtermformulaset}{\headvariableof{\mlnmaxtermsymbol},\shortcatvariables},
            \tnetofat{\elgraph}{\headvariableof{\mlnmaxtermsymbol}}}{\shortcatvariables} \, . & \qedhere
    \end{align*}
\end{proof}

% Fitting arbitrary distributions
Any probability distribution can thus be represented by a \HybridLogicNetwork{} in the minterm statistic, as well as in the maxterm statistic.
Thus, we have identified sets of $2^{\atomorder}$ formulas, which is rich enough to fit any distribution.

\subsubsect{Atomic formulas}

% Repeat atomic formulas
Let us now derive a closed form backward mapping for the statistic
\begin{align*}
    \atomformulaset \coloneqq \{\atomicformulaof{\atomenumerator} \wcols \atomenumeratorin\}
\end{align*}
of atomic formulas, which coincides with a variable selection map.
The selection encoding of this statistic is the tensor
\begin{align*}
    \sencodingof{\atomformulaset}{\shortcatvariables,\selvariable}
    = \sum_{\atomenumeratorin} \onehotmapofat{1}{\catvariableof{\atomenumerator}} \otimes \onesat{\catvariableof{[\atomorder]/\{\atomenumerator\}}} \otimes \onehotmapofat{\atomenumerator}{\selvariable} \, .
\end{align*}
For each probability distribution $\probwith$ we the corresponding mean parameter has coordinates at $\atomenumeratorin$ by
\begin{align*}
    \meanparamat{\selvariable=\atomenumerator} = \probat{\catvariableof{\atomenumerator}=1}  \, .
\end{align*}

\begin{theorem}
    For the \HybridLogicNetworks{} to the statistic of atomic formulas $\atomformulaset$ we have the forward map
    \begin{align*}
        \forwardmapwrtof{\mlnatomsymbol}{\hybridparam}[\selvariable=\atomenumerator]
        = \begin{cases}
              \headindexof{\atomenumerator} & \text{if} \quad \atomenumerator\in\variableset \\
              \frac{\expof{\canparamat{\selvariable=\atomenumerator}}}{1+\expof{\canparamat{\selvariable=\atomenumerator}}} & \text{if} \quad \atomenumerator\notin\variableset % sigmoid
        \end{cases}
    \end{align*}
\end{theorem}
\begin{proof}
    Let $\hybridparam\in\hybridparamsetofdim{\atomorder}$ and denote the
    \begin{align*}
        \meanparamat{\selvariable} = \forwardmapwrtof{\mlnatomsymbol}{\hybridparam} \, .
    \end{align*}
    By definition we have for any $\atomenumeratorin$
    \begin{align*}
        \meanparamat{\selvariable=\atomenumerator}
        &= \contractionof{\sencodingof{\atomformulaset}{\shortcatvariables,\selvariable},
            \normalizationof{\bencodingofat{\atomformulaset}{\headvariableof{[\atomorder]},\shortcatvariables},\hypercoreofat{\hybridparam}{\headvariableof{[\atomorder]}}}{\shortcatvariables}
        }{\selvariable=\atomenumerator} \\
        &= \contraction{\onehotmapofat{1}{\catvariableof{\atomenumerator}},
            \normalizationof{
                \bencodingofat{\atomicformulaof{\atomenumerator}}{\headindexof{\atomenumerator},\catvariableof{\atomenumerator}},
                \hypercoreof{\selindex}{\headvariableof{\atomenumerator}}}{\catvariableof{\atomenumerator}}
        }
    \end{align*}
    Now, if $\atomenumerator\in\variableset$ we have $\hypercoreof{\selindex}{\headvariableof{\atomenumerator}}=\onehotmapofat{\headindexof{\atomenumerator}}{\headvariableof{\atomenumerator}}$ and $\meanparamat{\selvariable=\atomenumerator}=\headindexof{\atomenumerator}$.
    If $\atomenumerator\notin\variableset$ then $\hypercoreof{\selindex}{\headvariableof{\atomenumerator}}=\actcoreofat{\atomenumerator,\canparamat{\selvariable=\atomenumerator}}{\headvariableof{\atomenumerator}}$ and
    \begin{align*}
        \meanparamat{\selvariable=\atomenumerator} = \frac{\expof{\canparamat{\selvariable=\atomenumerator}}}{1+\expof{\canparamat{\selvariable=\atomenumerator}}} \, . & \qedhere
    \end{align*}
\end{proof}

A backward map to the atomic statistic is given by
\begin{align*}
    \backwardmapwrtof{\mlnatomsymbol}{\meanparamat{\selvariable}}
    =\hybridparam
    = \Big(\{\atomenumerator\wcols\atomenumeratorin\meanparamat{\selvariable=\atomenumerator}\in\{0,1\}\}, [\meanparamat{\selvariable=\atomenumerator}\wcols ] \Big)
\end{align*}
where
\begin{itemize}
    \item $\variableset = \{\atomenumerator\wcols\atomenumeratorin\meanparamat{\selvariable=\atomenumerator}\in\{0,1\}\}$
    \item For $\atomenumerator\in\variableset$ $\headindexof{\atomenumerator}=\meanparamat{\selvariable=\atomenumerator}$
    \item For $\atomenumerator\in\variableset$ we have $\canparamat{\selvariable=\atomenumerator}=0$ and for $\atomenumerator\notin\variableset$
    \begin{align*} % logit
        \canparamat{\selvariable=\atomenumerator}
        = \lnof{\frac{\meanparamat{\selvariable=\atomenumerator}}{1-\meanparamat{\selvariable=\atomenumerator}}}
    \end{align*}
\end{itemize}

The forward and backward map on the soft atomic formulas are the coordinatewise sigmoid and logit, respectively.

%\begin{theorem}
%    Given a \MarkovLogicNetwork{} with the statistic $\atomformulaset$ of atomic formulas, the forward mapping from canonical parameters to mean parameters is the coordinatewise sigmoid, that is
%    \[ \forwardmapwrtof{\mlnatomsymbol}{\canparamat{\selvariable}} = \frac{\expof{\canparamat{\selvariable}}}{\onesat{\selvariable}+\expof{\canparamat{\selvariable}}}   \]
%    where the quotient is performed coordinatewise.
%
%    A backward mapping is the coordinatewise logit, that is
%    \[ \backwardmapwrt{\mlnatomsymbol}(\meanparamwith)
%    = \lnof{\frac{
%        \meanparamwith
%    }{
%        \onesat{\selvariable}-\meanparamwith
%    }}  \, . \]
%\end{theorem}
%\begin{proof}
%    We have for any $\canparamat{\selvariable}\in\rr^{\atomorder}$
%    \[ \probofat{(\atomformulaset,\canparam)}{\shortcatvariables}
%    = \bigotimes_{\atomenumeratorin} \normalizationof{\expof{\canparamat{\selvariable=\atomenumerator}\cdot \atomicformulaof{\atomenumerator}}}{\catvariableof{\atomenumerator}}  \, . \]
%
%
%    For any $\atomenumeratorin$ it therefore holds, that
%    \begin{align*}
%        \forwardmapwrtof{\mlnatomsymbol}{\canparamat{\selvariable}}[\selvariable=\atomenumerator]
%        &=\contraction{\atomicformulaof{\atomenumerator},  \probofat{(\atomformulaset,\canparam)}{\shortcatvariables}} \\
%        &=\contraction{\atomicformulaof{\atomenumerator},  \normalizationof{\expof{\canparamat{\selvariable=\atomenumerator}\cdot \atomicformulaof{\atomenumerator}}}{\catvariableof{\atomenumerator}}} \\
%        & = \frac{\expof{\canparamat{\selvariable=\atomenumerator}}}{1+\expof{\canparamat{\selvariable=\atomenumerator}}} \, .
%    \end{align*}
%
%    Since the coordinatewise logit is the inverse function of the coordinatewise sigmoid the map
%    \begin{align*}
%        \backwardmapwrtof{\mlnatomsymbol}{\meanparamwith}[\selvariable=\atomenumerator]
%        & = \lnof{\frac{\meanparamat{\selvariable=\atomenumerator}}{1- \meanparamat{\selvariable=\atomenumerator}}}
%    \end{align*}
%    satisfies for any $\meanparam$ in the image of the forward map
%    \begin{align*}
%        \forwardmapwrt{\mlnatomsymbol}(\backwardmapwrt{\mlnatomsymbol}(\meanparam)) = \meanparam
%    \end{align*}
%    and is therefore a backward map.
%\end{proof}
%
%
%% Representation by selection tensor networks
%In a selection tensor networks they are represented by a single neuron with identity connective and variable selection to all atoms.
%We will investigate such examples in more detail in \charef{cha:sparseRepresentation}, where atomic formulas \MarkovLogicNetworks{} are specific cases of monomial decomposition of order 1.

% Interpretation of the result as independence approximation
%The maximum likelihood estimator of a positive probability distribution by the MLN of atomic formulas is therefore the tensor product of the marginal distributions.

%The Kullback-Leibler divergence between the distribution and its projection is the mutual information of the atoms, see for example Chapter~8 in \cite{mackay_information_2003}.

%\begin{remark}[Decomposition into systems of atomic networks]
%    \red{By Independence Decomposition we reduce to a system of atomic MLN.
%    The minterms of such MLNs are the literals.
%    By redundancy (literals sum up to $\ones$), it suffices to take only the positive or the negative literal.
%    }
%%	We set the weights of $\weightof{\lnot\atomicformulaof{\atomenumerator}}=0$ (corresponding with a gauge normalization of the energy offset symmetry). % Not needed!
%\end{remark}





\sect{Alternating Moment Matching}\label{sec:alternatingParEstMLN}

We now derive an algorithm for the Maximum Likelihood Estimation in case of \HybridLogicNetworks{}.
To this end, we first solve local cross-entropy minimization problems, which are then alternated to find global solutions.

%Parameter estimation is the Maximum Likelihood Problem for exponential families.
%It is solved by the backward map, when the mean parameter is in the interior of the mean parameter polytope.

\subsect{Local updates}

Let us now varying a distribution $\secprobwith$ by adding an additional feature $\formula$
\begin{align*}
    \probofat{\hypercore}{\shortcatvariables}
    \coloneqq \normalizationof{\secprobwith,\formulaccwith,\hypercoreat{\formulavar}}{\shortcatvariables} \, .
\end{align*}
Note, that the normalization exists for positive $\hypercoreat{\formulavar}$ and if $\contraction{\secprobtensor,\exformula}\notin\{0,1\}$ then also for any non-vanishing $\hypercoreat{\formulavar}$.
If $\contraction{\secprobtensor,\exformula}\in\{0,1\}$, then the $\probof{\hypercore}$ is constant among $\hypercoreat{\formulavar}$, when it exists.

We want to solve the local cross-entropy minimization problem
\begin{align*}
    \min_{\hypercore} \centropyof{\empdistribution}{\probof{\hypercore}} \,.
\end{align*}
If $\contraction{\secprobtensor,\exformula}\in\{0,1\}$ then the minimum is taken at any $\hypercoreat{\formulavar}$ such that $\probof{\hypercore}$ exists.

\begin{lemma}
    \label{lem:localHybridParamUpdate}
    Let $\empdistribution$, $\secprobtensor$ be distributions and $\exformula$ a formula such that $\contraction{\secprobtensor,\exformula}\notin\{0,1\}$.
    If $\contraction{\empdistribution,\exformula}\in\{0,1\}$ the solution of the local cross-entropy minimization is
    \begin{align*}
        \hypercoreat{\formulavar} = \onehotmapofat{\contraction{\empdistribution,\exformula}}{\formulavar}
    \end{align*}
    If $\contraction{\empdistribution,\exformula}\notin\{0,1\}$ the solution is
    \begin{align*}
        \hypercoreat{\formulavar} = \actcoreofat{\canparam}{\formulavar}
    \end{align*}
    where
    \begin{align*}
        \canparam = \lnof{
            \frac{\contraction{\empdistribution,\exformula}}{(1-\contraction{\empdistribution,\exformula})}
            \cdot \frac{1-\contraction{\secprobtensor,\exformula}}{\contraction{\secprobtensor,\exformula}}
        } \, .
    \end{align*}
\end{lemma}
\begin{proof}
    The cross entropy is decomposed into
    \begin{align*}
        \centropyof{\empdistribution}{\probof{\hypercore}}
        &= \centropyof{\empdistribution}{\secprobtensor} \\
        &\quad + \contraction{\empdistributionwith,\formulaccwith,-\lnof{\hypercoreat{\formulavar}}} \\
        &\quad + \lnof{\contraction{\secprobwith,\formulaccwith,\hypercoreat{\formulavar}}}\, .
    \end{align*}
    Since the first term is constant among $\hypercore$, we focus on the minimization of the second and third term.
    For each $\hypercoreat{\formulavar}$ and its boolean support $\kcoreat{\formulavar}\coloneqq\greaterzeroof{\hypercoreat{\formulavar}}$ we find $\lambda>0$ and $\canparam\in\rr$ such that
    \begin{align*}
        \hypercoreat{\formulavar}
        = \lambda\cdot\contractionof{\kcoreat{\formulavar},\actcoreof{\canparam}}{\formulavar} \, .
    \end{align*}
    Given this parametrization we have
    \begin{align*}
        & \contraction{\empdistributionwith,\formulaccwith,-\lnof{\hypercoreat{\formulavar}}}
        + \lnof{\contraction{\secprobwith,\formulaccwith,\hypercoreat{\formulavar}}} \\
        & \quad  = \contraction{\empdistributionwith,\formulaccwith,-\lnof{\kcoreat{\formulavar}}} - \canparam\cdot\contraction{\empdistributionwith,\formulawith} \\
        & \quad \quad  + \lnof{\contraction{\secprobwith,\formulaccwith,\kcoreat{\formulavar},\actcoreofat{\canparam}{\formulavar}}} \, .
    \end{align*}
    The minimum over $\kcore$ is taken at
    \begin{align*}
        \kcoreat{\formulavar} =
        \begin{cases}
            \tbasisat{\formulavar} & \ifspace \contraction{\empdistribution,\exformula}=1 \\
            \fbasisat{\formulavar} & \ifspace \contraction{\empdistribution,\exformula}=0 \\
            \onesat{\formulavar} & \text{else}
        \end{cases} \, .
    \end{align*}
    If the optimal $\kcore$ is not the trivial vector $\onesat{\formulavar}$, the parameter $\canparam\in\rr$ does not influence the distribution and we can arrive at the claim when choosing $\canparam=0$.
    If the optimal $\kcore$ is trivial, we optimize further over $\actcoreof{\canparam}$
    \begin{align*}
        \min_{\canparam\in\rr} \contraction{\empdistributionwith,\formulaccwith,-\lnof{\hypercoreat{\formulavar}}}
        + \lnof{\contraction{\secprobwith,\formulaccwith,\actcoreofat{\canparam}{\formulavar}}} \, .
        \, .
    \end{align*}
    The derivation of the objective is
    \begin{align*}
        & \difofwrt{\contraction{\empdistributionwith,\formulaccwith,-\lnof{\hypercoreat{\formulavar}}}
        + \lnof{\contraction{\secprobwith,\formulaccwith,\actcoreofat{\canparam}{\formulavar}}}}{\canparam} \\
        & \quad = \contraction{\empdistributionwith,\formulawith}
        - \contraction{\probofat{\actcoreof{\canparam}}{\shortcatvariables},\formulawith} \\
        & \quad = \contraction{\empdistribution,\exformula}
        - \frac{\expof{\canparam} \cdot \contraction{\secprobwith,\formulawith}}{
            \expof{\canparam} \cdot \contraction{\secprobwith,\formulawith} + (1-\contraction{\secprobwith,\formulawith})
        } \, .
    \end{align*}
    The derivative vanished thus at the unique minimum of the cross entropy at
    \begin{align*}
        \canparam = \lnof{
            \frac{\contraction{\empdistribution,\exformula}}{(1-\contraction{\empdistribution,\exformula})}
            \cdot \frac{(1-\contraction{\secprobtensor,\exformula})}{\contraction{\secprobtensor,\exformula}}
        } \, . & \qedhere
    \end{align*}
\end{proof}


\subsubsect{\MarkovLogicNetworks{}}

In case of boolean statistics, we can provide a particular simple implementation of the Alternating Moment Matching Algorithm~\ref{alg:AMM}.
In the following section we will then generalize to \HybridLogicNetworks{} by allowing hard cores.
%The moment matching condition of \lemref{lem:mmContractionEquation} simplifies as follows.

Iteratively we update leg vectors of the activation tensor.
This is the above local variation with
\begin{align*}
    \secprobwith = \normalizationof{\{\bencodingof{\enumformula} \wcols \selindexin\}
    \cup\{\actcoreof{\tilde{\selindex},\canparamat{\selvariable=\tilde{\selindex}}} \wcols \tilde{\selindex} \in [\seldim], \tilde{\selindex}\neq\selindex\}
    \cup\{\basemeasure\}}{\headvariableof{\selindex}}
\end{align*}

To solve the moment matching condition at a formula $\enumformula$ we do not have to compute the normalization constant, since we only require the quotient
\begin{align*}
    \frac{1-\contraction{\secprobtensor,\exformula}}{\contraction{\secprobtensor,\exformula}} \, .
\end{align*}
The updated canonical coordinate is thus computed as
%$\secprobtensor$
\begin{align}
    \label{sol:momentMatchingExformula}
    \indexedcanparam = \lnof{
        \frac{\datameanat{\indexedselvariable}}{(1-\datameanat{\indexedselvariable})}
        \cdot \frac{\hypercoreat{\headvariableof{\selindex}=0}}{\hypercoreat{\headvariableof{\selindex}=1}}
    }
\end{align}
where by $\hypercoreat{\headvariableof{\selindex}}$ we denote the contraction
\begin{align*}
    \hypercoreat{\headvariableof{\selindex}}
    = \contractionof{\{\bencodingof{\enumformula} \wcols \selindexin\}
    \cup\{\actcoreof{\tilde{\selindex},\canparamat{\selvariable=\tilde{\selindex}}} \wcols \tilde{\selindex} \in [\seldim], \tilde{\selindex}\neq\selindex\}
    \cup\{\basemeasure\}}{\headvariableof{\selindex}} \, .
\end{align*}

%%refine \lemref{lem:mmContractionEquation} in the following.
%\begin{lemma}
%    \label{ref:lemMMinMLN}
%    Let there be a base measure $\basemeasure$, a formula selecting map $\formulaset=\{\enumformula \wcols \selindexin\}$ and a canonical parameter $\canparam$, and choose $\selindexin$ such that $\enumformula  \notin \{\onesat{\shortcatvariables},\zerosat{\shortcatvariables}\}$.
%    The moment matching condition relative to $\canparamwith$, and $\datameanat{\indexedselvariable}\in(0,1)$ is then satisfied, if for all $\selindexin$
%    \begin{align}
%        \label{sol:momentMatchingExformula}
%        \indexedcanparam = \lnof{
%            \frac{\datameanat{\indexedselvariable}}{(1-\datameanat{\indexedselvariable})}
%            \cdot \frac{\hypercoreat{\headvariableof{\selindex}=0}}{\hypercoreat{\headvariableof{\selindex}=1}}
%        }
%    \end{align}
%    where by $\hypercoreat{\headvariableof{\selindex}}$ we denote the contraction
%    \begin{align*}
%        \hypercoreat{\headvariableof{\selindex}}
%        = \contractionof{\{\bencodingof{\enumformula} \wcols \selindexin\}
%        \cup\{\actcoreof{\tilde{\selindex},\canparamat{\selvariable=\tilde{\selindex}}} \wcols \tilde{\selindex} \in [\seldim], \tilde{\selindex}\neq\selindex\}
%        \cup\{\basemeasure\}}{\headvariableof{\selindex}} \, .
%    \end{align*}
%\end{lemma}
%\begin{proof}
%    Since $\imageof{\enumformula}\subset[2]$ we have
%    \begin{align*}
%        \idrestrictedto{\imageof{\enumformula}} = \onehotmapofat{1}{\headvariableof{\selindex}}
%    \end{align*}
%    and the moment matching condition is by \lemref{lem:mmContractionEquation} satisfied if
%    \begin{align*}
%        \contraction{\actcorewith, \onehotmapofat{1}{\headvariableof{\selindex}}, \hypercoreat{\headvariableof{\selindex}}}
%        = \contraction{\actcorewith,\hypercoreat{\headvariableof{\selindex}}} \cdot \datameanat{\indexedselvariable} \, .
%    \end{align*}
%    This is equal to
%    \begin{align*}
%        \expof{\canparamat{\indexedselvariable}} \cdot \hypercoreat{\headvariableof{\selindex}=1}
%        = \left( \expof{\canparamat{\indexedselvariable}} \cdot \hypercoreat{\headvariableof{\selindex}=1} + \hypercoreat{\headvariableof{\selindex}=0} \right) \cdot \datameanat{\indexedselvariable} \, .
%    \end{align*}
%    Rearranging the equations this is equal to
%    \begin{align*}
%        \hypercoreat{\headvariableof{\selindex}}
%        = \contractionof{\{\bencodingof{\enumformula}\}
%        \cup\{\actcoreof{\tilde{\selindex}} : \tilde{\selindex} \in [\seldim], \tilde{\selindex}\neq\selindex\}
%        \cup\{\basemeasure\}}{\selvariable} \, .
%    \end{align*}
%    We notice that the right side is well defined, since we have by assumption $\datameanat{\indexedselvariable}, (1- \datameanat{\indexedselvariable}) \neq 0$ and $\hypercoreat{\headvariableof{\selindex}=0}, \hypercoreat{\headvariableof{\selindex}=1} \neq 0$ since \MarkovLogicNetworks{} are positive distributions and $\enumformula \notin \{\onesat{\shortcatvariables},\zerosat{\shortcatvariables}\}$.
%\end{proof}

\begin{algorithm}[hbt!]
    \caption{Alternating Moment Matching for Markov Logic Networks}\label{alg:AMM_MLN}
    \begin{algorithmic}
        \Require Mean parameter $\meanparamwith$ with $\uniquantwrtof{\selindexin}{\meanparamat{\indexedselvariable}\notin\{0,1\}}$, boolean statistic $\sstat$, base measure $\basemeasure$
        \Ensure Canonical parameter $\canparamwith$, such that $\expdist$ is the (approximative) moment projection of $\empdistribution$ onto $\expfamily$
        \iosepline
        \State Set $\canparamwith=\zerosat{\selvariable}$
        \While{Convergence criterion is not met}
            \ForAll{$\selindex\in\secnodes$}
                \State Compute
                \begin{align*}
                    \hypercoreat{\headvariableof{\selindex}}
                    = \contractionof{\{\bencodingof{\enumformula} \wcols \selindexin\}
                    \cup\{\actcoreof{\tilde{\selindex},\canparamat{\selvariable=\tilde{\selindex}}} \wcols \tilde{\selindex}\in[\seldim]\ncond\tilde{\selindex}\neq\selindex\}
                    \cup\{\basemeasure\}}{\headvariableof{\selindex}}
                \end{align*}
                \State Set
                \begin{align*}
                    \canparamat{\indexedselvariable} = \lnof{
                        \frac{\meanparamat{\indexedselvariable}}{(1-\meanparamat{\indexedselvariable})}
                        \cdot \frac{\hypercoreat{\headvariableof{\selindex}=0}}{\hypercoreat{\headvariableof{\selindex}=1}}
                    }
                \end{align*}
            \EndFor
        \EndWhile
        \State \Return $\canparamwith$
    \end{algorithmic}
\end{algorithm}

Note that while $\uniquantwrtof{\selindexin}{\meanparamat{\indexedselvariable}\notin\{0,1\}}$ ensures the well-definedness of the update equations in \algoref{alg:AMM_MLN} (otherwise the update could not be computed), it is only a necessary but not always sufficient criterion for the existence of a solution.
The moment matching conditions are simultaneously satisfiable, if and only if $\meanparamwith\in\sbinteriorof{\meansetof{\mlnstat,\basemeasure}}$.


The algorithm would be finished after a single pass through the loop, if the variables $\catvariableof{\exformula}$ are independent.
This would be the case, if the \MarkovLogicNetwork{} consists of atomic formulas only.
When they fail to be independent, the adjustment of the weights influence the marginal distribution of other formulas and we need an alternating optimization.
%
This situation corresponds with couplings of the weights by a partition contraction, which does not factorize into terms to each formula.

% Concave likelihood
Since the likelihood is concave (see \cite{koller_probabilistic_2009}), there are not local maxima the coordinate descent could run into and coordinate descent will give a monotonic improvement of the likelihood.

% Inference in inner loop
Solving Equation~\ref{sol:momentMatchingExformula} requires inference of a current model by answering a query.
This can be a bottleneck and circumvented by approximative inference, see e.g. CAMEL \cite{ganapathi_constrained_2008}.

\subsubsect{\HybridLogicNetworks{}}

We now extend the alternating parameter estimation algorithm to \HybridLogicNetworks{}, by allowing for mean parameters on the interior of cube faces.
To this end, we first find the smallest cube face containing the mean parameter $\meanparamwith$ (see \lemref{lem:minimalContainingFace}) and then run the algorithm on the exponential family on that face.
In an alternative perspective, we first optimize the leg vectors to features with $\meanparamat{\indexedselvariable}\in\{0,1\}$, which are then constant and left out in the further update sweeps.
The local moment matching conditions of are satisfied simultanously for some $\canparamwith$, if and only if $\meanparamwith$ is elementarily realizable (see \defref{def:elementaryRealizableMeanParams}), that is $\meanparamwith$ is on the interior of a cube face of $\hlnmeanset$.

\begin{algorithm}[hbt!]
    \caption{Alternating Moment Matching for Hybrid Logic Networks}\label{alg:AMM_HLN}
    \begin{algorithmic}
        \Require Mean parameter $\meanparamwith$
        \Ensure Canonical parameter $\canparamwith$, such that $\expdist$ is the (approximative) moment projection of $\empdistribution$ onto $\hlnsetof{\formulaset}$
        \iosepline
        \State Set
        \begin{align*}
            \variableset = \Big\{ \selindex \wcols \selindexin \ncond \meanparamat{\indexedselvariable}\in\{0,1\} \Big\}
        \end{align*}
        and a tuple $\headindexof{\variableset}$ with $\headindexof{\selindex}=\meanparamat{\indexedselvariable}$ for $\selindex\in\variableset$ .
        \State Run \algoref{alg:AMM_MLN} (Alternating Moment Matching for Markov Logic Networks) with base measure $\exformulaof{\variableset,\headindexof{\variableset}}$ and statistic $\mlnstat=\{\enumformula\wcols\selindex\in\hardlegset\}$ to get $\canparamwith$
        \State \Return $(\variableset,\headindexof{\variableset},\canparamwith)$
    \end{algorithmic}
\end{algorithm}


\subsect{Iterative Proportional Fitting} % TO BE FORMULATED

In a special case of partition statistics we can find simultaneous updates to the canonical parameters.
Partition statistics are boolean statistics $\mlnstat$ such that
\begin{align*}
    \contractionof{\sencodingofat{\mlnstat}{\shortcatvariables,\selvariable}}{\shortcatvariables} \, .
\end{align*}
We can thus understand them as a disjoint partition of the state set $\facstates$ into sets
\begin{align*}
    \arbsetof{\selindex} \coloneqq \{\shortcatindices\wcols\shortcatindicesin\ncond\enumformulaat{\indexedshortcatvariables}=1\} \, .
\end{align*}
These statistics will be investigated in more detail in \secref{sec:partitionStatistics}.

\begin{lemma}
    Given distributions $\empdistributionwith,\secprobwith$ we build
    \begin{align*}
        \datameanat{\selvariable} = \contractionof{\empdistributionwith,\sencmlnstatwith}{\selvariable}
        \andspace
        \currentmeanat{\selvariable} = \contractionof{\secprobwith,\sencmlnstatwith}{\selvariable}
    \end{align*}
    and assume that $\greaterzeroof{\datamean}\models\greaterzeroof{\currentmean}$. % and $\equaloneof{\currentmean}\models\equaloneof{\datamean}$.
    Let us vary the distribution $\secprobwith$ by elementary tensors $\hypercoreat{\headvariables}$ as
    \begin{align*} % NEED TO RESTRICT TO ELEMENTARY HYPERCORE?
        \probofat{\hypercore}{\shortcatvariables}
        \coloneqq \normalizationof{\secprobwith,\mlnstatccwith,\hypercoreat{\headvariables}}{\shortcatvariables} \, .
    \end{align*}

    Then
    \begin{align*}
        \argmin_{\hypercore} \centropyof{\empdistribution}{\probof{\hypercore}}
    \end{align*}
    is solved at
    \begin{align*}
        \hypercoreat{\headvariables}
        = \contractionof{\kcoreofat{(\hardlegset,0_{\hardlegset})}{\headvariables},\actcoreofat{\canparam}{\headvariables}}{\headvariables}
    \end{align*}
    where $\hardlegset=\{\selindex\wcols\datameanat{\indexedselvariable}=0\}$ and for $\selindex\in[\seldim]/\hardlegset$
    \begin{align*}
        \canparamat{\indexedselvariable} = \lnof{\frac{\datameanat{\indexedselvariable}}{\currentmeanat{\indexedselvariable}}} \, .
    \end{align*}
\end{lemma}
\begin{proof}
    We use \theref{the:selectionRepresentationPartitionStatistics} to compute
    \begin{align*}
        \contractionof{\secprobwith,\mlnstatccwith,\hypercoreat{\headvariables}}{\selvariable}
        = \contractionof{\secprobwith,\sencmlnstatwith,\expof{\canparamwith},\onehotmapofat{[\seldim]/\hardlegset}{\selvariable}}{\selvariable} \\
        = \contractionof{\currentmeanat{\selvariable},\expof{\canparamwith},\onehotmapofat{[\seldim]/\hardlegset}{\selvariable}}{\selvariable}
        = \datameanat{\selvariable}
    \end{align*}
    Here $\onehotmapofat{[\seldim]/\hardlegset}{\selvariable}$ is the indicator vector, whether $\selindex\notin\hardlegset$ (see for more details \defref{def:subsetEncoding}).

    Using that $\mlnstat$ is a partition statistic we get
    \begin{align*}
        \contraction{\datameanat{\selvariable}}
        &= \contraction{\empdistribution,\sencmlnstatwith} \\
        &= \contraction{\empdistribution} \\
        &= 1 \, .
    \end{align*}
    It follows that
    \begin{align*}
        \contraction{\secprobwith,\mlnstatccwith,\hypercoreat{\headvariables}} = 1
    \end{align*}
    and due to trivial partition function term that
    \begin{align*}
        \probofat{\hypercore}{\shortcatvariables} = \contractionof{\secprobwith,\mlnstatccwith,\hypercoreat{\headvariables}}{\shortcatvariables} \, .
    \end{align*}
    We conclude
    \begin{align*}
        \contractionof{\probofat{\hypercore}{\shortcatvariables},\sencmlnstatwith}{\selvariable}
        = \datameanat{\selvariable} \, .
    \end{align*}
    Thus, the moment matching condition is satisfied for each $\selindexin$ and the cross-entropy is minimized.
\end{proof}

If the $\greaterzeroof{\datamean}\models\greaterzeroof{\currentmean}$ is violated, then there is no solution to the moment matching condition, since the support of the mean parameter cannot be increased by an activation tensor.

The assumptions of a partition statistic are met when taking all features to any hyperedge in a Markov Network seen as an exponential family.
In that case, the update algorithm is refered to as Iterative Proportional Fitting \cite{wainwright_graphical_2008}.

%Further, when activating both $\exformula$ and $\lnot\exformula$.
%\begin{remark}[Grouping of partition statistics]
%    When having a set of coordinates, such that the coordinate functions are boolean and sum to the trivial tensor, one can find simultaneous updates to the canonical parameters, such that the partition function is staying invariant.
%    Given a parameter $\canparam^t$ we compute
%    \[ \meanparam^t = \contractionof{\expdistof{(\sstat,\canparam^t)}, \sstat}{\selvariable} \]
%    and build the update
%    \[ \canparam^{t+1} = \canparam^t + \lnof{\meanparam^{\datamap}}{\meanparam^t} \, . \]
%    Then, $\canparam^{t+1}$ satisfies the moment matching equations for all coordinates in the set.
%\end{remark}


\sect{Structure Learning}

Structure learning refers to the learning of the statistic $\mlnstat$ itself.
Let there be a set $\formulasuperset$ of statistics we build the set of parametrized distributions
\begin{align*}
    \probtensor = \bigcup_{\mlnstat} \elrealizabledistsof{\mlnstat}
\end{align*}
and pose the structure learning problem as the minimization of the cross entropy
\begin{align*}
    \min_{\mlnstat\in\formulasuperset} \min_{\hybridparamin} \centropyof{\empdistribution}{\probof{\hlnparameters}} \, .
\end{align*}
It can be impracticle to learn all formulas at once, since the set $\formulasuperset$ often grows combinatorically, for example when choosing as a powerset of formulas.
To avoid intractabilities, one can choose a greedy approach and learn in addition formulas $\exformula$ when already having learned a set $\formulaset$ of formulas.
%We in this section assume a current model $\currentdistribution$, which is a generic positive distribution not necessarily a \MarkovLogicNetwork{}. % or Hybrid Logic Network.

\subsect{Greedy formula inclusions}

Having a current set of formulas $\formulaset$ we want to choose the best $\formula\in\greedyhypothesis$ to extend the set of formulas to $\formulaset\cup\{\formula\}$ in a way minimizing the cross entropy.
Given this, add each step we solve the greedy cross entropy minimization
\begin{align}
    \label{prob:perfectGreedy}\tag{$\probtagtypeinst{\mathrm{greedy}}{\datamap,\mlnstat,\greedyhypothesis}$}
    \min_{\formula\in\greedyhypothesis} \min_{\hybridparam\in\hybridparamsetofdim{\cardof{\mlnstat}+1}}
    \centropyof{\empdistribution}{\probof{\mlnstat\cup\{\formula\},\hybridparam}} \, .
\end{align}
A brute force solution of Problem~\eqref{prob:perfectGreedy} would require parameter estimation for each candidate in $\greedyhypothesis$.
We provide two more efficient approximative heuristics in the following (see Chapter~20 in \cite{koller_probabilistic_2009}), which are faster to compute estimates of the cross entropy improvement from adding a formula to an existing statistic.

\subsect{Gain Heuristic}

In the gain heuristic, only the parameters of the new formula are optimized and the others left unchanged.
Let $\sechybridparam$ be the canonical parameter of the reference distribution on the statistic $\mlnstat$.
When adding a feature $\formula\in\greedyhypothesis$ we extend $\mlnstat$ by $\formula$ and restrict the parameters to coincide with $\sechybridparam$ on the first $\cardof{\mlnstat}$ coordinates.
We denote this constraint by
\begin{align*}
    \restrictionofto{\hybridparam}{[\mlnstat]} = \sechybridparam \, .
\end{align*}
The greedy gain heuristic is then the problem
\begin{align}
    \label{prob:greedyGain}\tag{$\probtagtypeinst{\gainsymbol}{\datamap,\formulaset,\sechybridparam,\greedyhypothesis}$}
    \min_{\formula\in\greedyhypothesis} \min_{\hybridparamsetofdim{\cardof{\mlnstat}+1} \wcols \restrictionofto{\hybridparam}{[\mlnstat]} = \sechybridparam }
    \centropyof{\empdistribution}{\probof{\hybridparam}} \, .
\end{align}
%Here we denote by $\canparam$ the first $\cardof{\formulaset}$ coordinates of the M-projection $\currentdistribution$  of $\empdistribution$ onto $\formulaset$ and the variable new coordinate at position $\canparamat{\cardof{\formulaset}}$.
To provide further insight into the solution of the gain heuristic, let us quantify the improvement of the cross entropy when addign a feature $\exformula$.

%% FOR GAIN HEURISTIC
\begin{lemma}
    Let $\empdistribution$, $\secprobtensor$ be distributions and $\exformula$ a formula such that $\contraction{\secprobtensor,\exformula}\notin\{0,1\}$.
    Then we have
    \begin{align*}
        \centropyof{\empdistribution}{\probof{\mlnstat,\sechybridparam}}
        - \min_{\hypercore} \centropyof{\empdistribution}{\probof{\hypercore}}
        = \kldivof{\bernoulliof{\contraction{\empdistribution,\exformula}}}{\bernoulliof{\contraction{\secprobtensor,\exformula}}}
    \end{align*}
    where by $\bernoulliof{p}$ we denote the Bernoulli distribution with parameter $p\in[0,1]$.
\end{lemma}
\begin{proof}
    We use the characterization of the local update by \lemref{lem:localHybridParamUpdate} for $\secprobtensor=\probof{\mlnstat,\sechybridparam}$.
    The update on the added feature is then parametrized by the two-dimensional vector $\hypercoreat{\formulavar}$ and we have
    \begin{align*}
        \probof{\mlnstat,\sechybridparam} = \probof{\ones} \, .
    \end{align*}
    Let us abbreviate $\datamean\coloneqq\datamean$ and $\currentmean\coloneqq\contraction{\probof{\mlnstat,\sechybridparam},\exformula}$.
    We distinguish the cases $\datamean\in(0,1)$ and $\datamean\in\{0,1\}$.

    In the case $\datamean\in(0,1)$, we have $\hardlegset=\sechardlegset$, $\headindexof{\hardlegset}=\secheadindexof{\hardlegset}$ and $\canparamat{\indexedselvariable}=\seccanparamat{\indexedselvariable}$ for $\selindex\in[\cardof{\mlnstat}]$.
    The additional coordinate of the canonical parameter is then
    \begin{align*}
        \canparamat{\selvariable=\cardof{\mlnstat}} = \lnof{
            \frac{\datamean}{(1-\datamean)}
            \cdot \frac{1-\currentmean}{\currentmean}
        } \, .
    \end{align*}
    Based on this characterization, the cross entropy difference is
    \begin{align*}
        \centropyof{\empdistribution}{\probof{\ones}} - \min_{\hypercore} \centropyof{\empdistribution}{\probof{\hypercore}} \\
        &= \datamean \cdot \canparamat{\selvariable=\cardof{\mlnstat}}
        - \lnof{\contraction{\secprobtensor,\formulaccwith,\actcoreof{\canparamat{\selvariable=\cardof{\mlnstat}}}}} \, .
    \end{align*}
    We simplify
    \begin{align*}
        \contraction{\secprobwith,\formulaccwith,\actcoreofat{\canparamat{\selvariable=\cardof{\mlnstat}}}{\headvariableof{\formulavar}}}
        = (1-\currentmean) + \currentmean \cdot \expof{\canparamat{\selvariable=\cardof{\mlnstat}}}
        = \frac{(1-\currentmean)}{(1-\datamean)} \, .
    \end{align*}
    We further have
    \begin{align*}
        \datamean \cdot \canparamat{\selvariable=\cardof{\mlnstat}}
        = \datamean \cdot \left[ \lnof{\frac{\datamean}{(1-\datamean)}\cdot \frac{(1-\currentmean)}{\currentmean}}  \right]
        = \datamean \lnof{\datamean} - \datamean \lnof{1-\datamean} + \datamean \lnof{1-\currentmean} - \datamean \lnof{\currentmean}
    \end{align*}
    and arrive at
    \begin{align*}
        &  \centropyof{\empdistribution}{\probof{\ones}} - \min_{\hypercore} \centropyof{\empdistribution}{\probof{\hypercore}} \\
        & \quad =  \datamean \lnof{\datamean} - \datamean \lnof{1-\datamean} + \datamean \lnof{1-\currentmean} - \datamean \lnof{\currentmean}
        -  \lnof{1-\currentmean} - \lnof{1-\datamean} \\
        & \quad = \left( -\datamean \lnof{\currentmean} - (1-\datamean) \lnof{1-\currentmean} \right)  - \left( -\datamean \lnof{\datamean} - (1-\datamean) \lnof{1-\datamean} \right) \\
        & \quad = \kldivof{\bernoulliof{\datamean}}{\bernoulliof{\currentmean}} \, .
    \end{align*}

    In the case $\datamean\in\{0,1\}$, the optimal $\hypercore$ is boolean and only the partition function term is changed by the update.
    We then have
    \begin{align*}
        \centropyof{\empdistribution}{\probof{\ones}} - \min_{\hypercore} \centropyof{\empdistribution}{\probof{\hypercore}}
        %&= \lnof{\contraction{\secprobwith}} - \lnof{\contraction{\secprobwith,\formulaccwith,\hypercoreat{\formulavar}}} \\
        &= - \lnof{\contraction{\secprobwith,\formulaccwith,\hypercoreat{\formulavar}}} \\
        &= \begin{cases}
               \lnof{\currentmean} & \ifspace \datamean=0 \\
               \lnof{1-\currentmean} & \ifspace \datamean=1 \\
        \end{cases} \\
        &= \kldivof{\bernoulliof{\datamean}}{\bernoulliof{\currentmean}} \, . \qedhere
    \end{align*}
\end{proof}

Problem \eqref{prob:greedyGain} is thus solved for
\begin{align*}
    \hat{\formula} \in \argmax_{\formula\in\formulaset} \kldivof{\bernoulliof{\contraction{\empdistribution,\formula}}}{\bernoulliof{\contraction{\currentdistribution,\formula}}}
\end{align*}
and $\hybridparam$ characterized in \lemref{lem:localHybridParamUpdate}.
The minimum is taken at
\begin{align*}
    \centropyof{\empdistribution}{\probof{\mlnstat,\sechybridparam}}
    - \kldivof{\bernoulliof{\contraction{\empdistribution,\hat{\formula}}}}{\bernoulliof{\contraction{\currentdistribution,\hat{\formula}}}} \, .
\end{align*}

% Algorithmic
The gain heuristic thus searches for the mode of a coordinatewise transform of the mean parameter tensors to $\empdistribution$ and $\currentdistribution$, using the Bernoulli Kullback-Leibler divergence as transform function.

% Interpretation
One therefore takes the formula, which marginal distribution in the current model and the targeted distribution are differing at most, measured in the KL divergence.

% Optimization method
%One optimization method would thus be the computation of the mean parameters to both distribution, building the coordinatewise KL divergence and choosing the maximum.
%Since we need to evaluate each coordinate, this can be intractable for large sets of formulas.


% Further weight optimization
Further improvement of the model can be achieved by iteratively optimizing the other weights as well, since their corresponding moment matching conditions might be violated after the integration of a new formula.
Unfortunately, backward maps cannot be expressed in closed form in general.
%This would require the computation of backward mappings for each candidate formula, for which we only have an alternating approach in general.



\subsect{Gradient heuristic}

Gradient heuristic is another approach to select a feature.
We show here how the effective selection tensor network representation of exponentially many formulas described in \charef{cha:formulaSelection} can be utilized. % Or later in proposal

In the gradient heuristic, we estimate the cross entropy improvement by the gradient of the cross entropy with respect to varying with another feature.
Given a distribution $\secprobwith$, we vary % more general than
\begin{align*}
    \probofat{\actcoreof{\canparam}}{\shortcatvariables} \coloneqq \normalizationof{\secprobwith,\formulaccwith,\actcoreofat{\canparam}{\headvariableof{\formulavar}}}{\shortcatvariables} \, .
\end{align*}
In the gradient heuristic we choose the steepest gradient direction
\begin{align}
    \label{prob:greedyGrad}\tag{$\probtagtypeinst{\gradientsymbol}{\datamap,\secprobtensor,\greedyhypothesis}$}
    \min_{\formula\in\greedyhypothesis} \difofwrt{\centropyof{\empdistribution}{\probofat{\actcoreof{\canparam}}{\shortcatvariables}}}{\canparam}   \, .
\end{align}

%% FOR GRADIENT HEURISTIC
\begin{lemma}
    \begin{align*}
        \difofwrt{\centropyof{\empdistribution}{\probof{\actcoreof{\canparam}}}}{\canparam} =
        -\contraction{\empdistribution,\formula} + \contraction{\probof{\actcoreof{\canparam}},\formula} \, .
    \end{align*}
\end{lemma}
\begin{proof}
    We have
    \begin{align*}
        \centropyof{\empdistribution}{\probof{\actcoreof{\canparam}}}
        & = \contraction{\empdistribution,\formulaccwith,-\lnof{\actcoreofat{\canparam}{\headvariableof{\formulavar}}}}
        + \lnof{\contraction{\secprobwith,\formulaccwith,\actcoreofat{\canparam}{\headvariableof{\formulavar}}}}
    \end{align*}
    The derivation of the first term at $\canparam=0$ is $-\contraction{\empdistribution,\formula}$ and of the second $\contraction{\probof{\actcoreof{\canparam}},\formula}$.
\end{proof}

Problem~\eqref{prob:greedyGrad} is thus
\begin{align}
    %\label{prob:greedyGrad}\tag{$\probtagtypeinst{\gradsymbol}{\datamap,\secprobtensor,\greedyhypothesis}$}
    \min_{\formula\in\greedyhypothesis} \contraction{\secprobtensor,\formula} - \contraction{\empdistribution,\formula}   \, .
\end{align}
%% Positive and Negative Search
The gradient shows the typical decomposition into a positive and a negative phase.
While the positive phase comes from the data term and prefers directions of large data support, the negative phase originates in the partition function and draws the gradient away from directions already supported by the current model $\expdistof{(\naivestat, \naivecanparam)}$.
%% Regularization functionality
The negative phase is a regularization, by comparing with what has already been learned.
When nothing has been learned so far, we can take the current model to be the uniform distribution, which is the naive exponential family with vanishing canonical parameters.






\subsect{Proposal distributions}

We now frame the selection of a formula as a sampling problem of proposal distributions.
%All the costs and the exact greedy objective
Each of the scores to candidate formulas are understood as coordinates of an energy tensor.
Among the heuristics, the gradient heuristic has the most accessible form, since the energy tensor is available as a tensor network of the selection encoding of the formulas.

%We can choose selection architectures to efficiently parametrize the formulas in the hypothesis $\greedyhypothesis$ and rewrite the problem as
%\begin{align*}
%	\argmax_{\selindexin} \contractionof{ \gradwrtat{\canparam}{\canparam=0} \lossof{\expdist}}{\indexedselvariable}
%\end{align*}
%This is thus equivalent to the problem \ref{prob:greedyGrad}, when taking all formulas selectable by $\formulaset$ as the hypothesis $\Gamma$.

% Proposal distribution
%Let us now understand the likelihood gradient as the energy tensor of a probability distribution, which we call the proposal distribution.

\begin{definition}[Gradient Heuristic Proposal Distribution]
    Let there be a base distribution $\currentdistribution$, a targeted distribution $\empdistribution$ and a formula selecting map $\greedyhypothesis$.
    The proposal distribution at inverse temperature $\invtemp>0$ is the distribution of $\selvariable$ defined by
    \begin{align*}
        \normalizationof{\expof{\contractionof{\invtemp\cdot(\empdistribution[\shortcatvariables]-\currentdistribution[\shortcatvariables]),\greedyhypothesis\left[\shortcatvariables,\selvariable\right]}{\selvariable}} }{\selvariable} \, .
    \end{align*}
    The proposal distribution is the member of the exponential family with statistics $\greedyhypothesis$ and canonical parameter $\invtemp\cdot(\empdistribution-\currentdistribution)$.
\end{definition}


%. Exponential family
The proposal distribution is in the exponential family with sufficient statistic by the formula selecting map $\greedyhypothesis$, namely the member with the canonical parameters $\canparam=\empdistribution-\currentdistribution$.
Of further interest are tempered proposal distributions, which are in the same exponential family with canonical parameters $\invtemp\cdot(\empdistribution-\currentdistribution)$ where $\invtemp>0$ is the inverse temperature parameter.

% MLN
As \MarkovLogicNetworks{}, the proposal distributions are in exponential families with the sufficient statistic defined in terms of formula selecting maps.
While \MarkovLogicNetworks{} contract the maps on the selection variables $\selvariable$, the proposal distributions contract them along the categorical variables $\catvariable$ to define energy tensors.

% Methods to solve mode search
The gradient heuristic optimization Problem~\eqref{prob:greedyGrad} is the search for the mode of the proposal distribution.
To solve the gradient heuristic, we thus need to answer a mode query, for which we can apply the methods introduced in \charef{cha:probReasoning}, such as Gibbs Sampling or Mean Field Approximations in combination with annealing.


%\subsect{Mean parameter polytope}
The mean parameter polytope of any proposal distribution to statistic $\proposalstat$ is the convex hull of the formulas in $\formulaset$, that is
\begin{align*}
    \meansetof{\proposalstat}
    = \convhullof{\sencodingof{\proposalstat}{\indexedselvariable,\shortcatvariables} \wcols \selindexin}
    = \convhullof{\formulaat{\shortcatvariables} \wcols \formula\in\greedyhypothesis}
\end{align*}
% 0/1
As it was the case for \MarkovLogicNetworks{}, the mean parameter polytopes are instances of a $0/1$-polytopes \cite{ziegler_lectures_2000,gillmann_01-polytopes_2007}.
% Interpretation as formulas
The vertices are the formulas selectable by the formula selecting map $\greedyhypothesis$.

\subsect{Iterations}

Let us now iterate the search for a best formula at a current model with the optimization of weights after each step.
The result is Algorithm~\ref{alg:greedyStructureLearning}, which is a greedy algorithm adding iteratively the currently best feature.

\begin{algorithm}[hbt!]
    \caption{Greedy Structure Learning}\label{alg:greedyStructureLearning}
    \begin{algorithmic}
        \Require Empirical distribution $\empdistribution$, hypothesis $\greedyhypothesis$ of formulas
        \Ensure Distribution $\expdist$ approximating $\empdistribution$
        \iosepline
        \State Initialize
        \[ \currentdistribution \algdefsymbol \frac{1}{\prod_{\catenumeratorin}\catdimof{\atomenumerator}} \cdot \onesat{\shortcatvariables} \quad, \quad \formulaset = \varnothing \]
        \While{Stopping criterion is not met}
        % REFINE! Work in data
            \State
            \begin{itemize}
                \item \textbf{Structure Learning:} Compute a (approximative) solution $\hat{\formula}$ to Problem~\eqref{prob:perfectGreedy} and add the formula to $\formulaset$, i.e.
                \[ \formulaset \algdefsymbol \formulaset \cup\{\hat{\formula}\} \]
                Extend dimension of $\selvariable$ by one, by $\formulaof{\seldim}=\hat{\formula}$ and $\canparamat{\selindex=\seldim}=0$
                \item \textbf{Weight Estimation:} Estimate the best weights for the added formula and recalibrate the weights of the previous formulas, by calling Algorithm~\ref{alg:AMM_HLN}.
                \[ \currentdistribution \algdefsymbol \probof{\formulaset,\hybridparam} \]
            \end{itemize}
        \EndWhile
        \State \Return $\formulaset$, $\hybridparam$ %, $\kb$
    \end{algorithmic}
\end{algorithm}


%% Energy Storage -> Useful after learning for energy-based inference
When having used the same learning architecture multiple times, the energy of the corresponding formulas are all representable by a formula selecting architecture.
Their energy term is therefore a contraction of the selecting tensor with a parameter tensor $\canparam$ in a basis $\cpformat$ decomposition with rank by the number of learned formulas.
When mutiple selection architectures have been used, the energy is a sum of such contractions.
% 
Let us note, that this representation is useful after learning, when performing energy-based inference algorithms on the result.
During learning, one needs to instantiate the proposal distribution, which requires instantiation of the probability tensor.
\red{However, one could alternate data energy-based and use this as a particle-based proxy for the probability tensor.}


\begin{remark}[Sparsification by Thresholding]
    To maintain a small set of active formulas, one could combine greedy learning approaches with thresholding on the coordinates of $\canparam$.
    This is a standard procedure in Iterative Hard Thresholding algorithms of Compressed Sensing, but note that here we do not have a linear in $\canparam$ objective.
\end{remark}









\sect{Discussion}

\begin{remark}[Bayesian approach]
    We only treated the estimation of a single resulting distribution by the data, while in a Bayesian approach one typically considers an uncertainty over possible distributions.
    % MAP
    \red{When treating $\canparam$ as a random tensor, which prior distribution is given and posteriori distribution wanted, we have a more involved Bayesian approach.}
    When having a prior $\probat{\mlnparameters}$ over the \MarkovLogicNetworks{} we alternatively want to find the parameters $\mlnparameters$ solving the maximum a posteriori problem
    \begin{align}
        \argmax_{\mlnparameters} \mlnprobat{\data}\cdot \probat{\mlnparameters}\, .
    \end{align}
\end{remark}

% Polytopes - MLN 
The polytopes of mean parameters to \HybridLogicNetworks{} and proposal distributions are an interesting connection between the fields of combinatorical optimization and the study of expressivity of tensor networks.
% Minimal Connectivity: Local consitency - Hierarchical Tucker
This is of special interest, when the computation cores of a hybrid logic network are minimally connected, the mean parameters are captured by local consistencies.
Similar investigations have been made in the field of tensor networks, where minimal connected tensor networks are refered to by Hierarchical Tucker formats (HT).
Minimal connection is exploited in the tensor network community to show numerical properties of the format, such as closedness and existence of best approximators.
















    \section{\chatextconcentration}\label{cha:concentration}

When drawing data independently from a random distribution, we are limited by random effects.
We in this chapter derive guarantees, that the learning methods introduced in \charef{cha:probReasoning} and \charef{cha:networkReasoning} are robust against such effects.

\subsection{Fluctuations of random data}

A random tensor is a random element of a tensor space $\facspace$, drawn from a probability distribution on $\facspace.$
In contrast to the discrete distributions investigated previously in this work, the random tensors are in most generality continuous distributions. % However, when drawing data they are 

\subsubsection{Fluctuation of the empirical distribution}

% Random one hot encodings
When drawing random states $\datapoint\in\facstates$ by a distribution $\gendistribution$, we use the one-hot encoding to forward each random state to the random tensor
\[ \onehotmapofat{\datapoint}{\shortcatvariables} \, . \]
The expectation of this random tensor is
\begin{align*}
    \expectationof{\onehotmapof{\datapoint}}
    = \sum_{\shortcatindices\in\facstates} \gendistributionat{\indexedshortcatvariables} \onehotmapofat{\shortcatindices}{\shortcatvariables}
    = \gendistributionat{\shortcatvariables} \, .
\end{align*}

The empirical distribution is then the average of independent random one-hot encodings, namely the random tensor
\[ \empdistribution = \frac{1}{\datanum} \sum_{\datindexin}  \onehotmapofat{\datapoint}{\shortcatvariables} \, . \]
To avoid confusion let us strengthen, that in this chapter we interpret $\empdistribution$ as a random tensor taking values in $\facspace$, whereas each supported value of $\empdistribution$ is an empirical distribution taking values in $\facstates$.
The forwarding of $\facstates$ under the one-hot encoding is a multinomial random variable, see \defref{def:mulinomialVariable}.


% Expectation -> Does not make use of independence here!
When the marginal of each datapoint is $\gendistribution$, the expectation of the empirical distribution is
\begin{align*}
    \expectationof{\empdistribution}
    = \frac{1}{\datanum} \sum_{\datindexin}  \expectationof{\onehotmapof{\datapoint}}
    = \gendistribution \, .
\end{align*}

% Law of large numbers
From the law of large numbers it follows, that in the limit of $\datanum\rightarrow\infty$ at any coordinate $\catindex\in\facstates$ almost everywhere
\[ \empdistributionat{\indexedshortcatvariables} \rightarrow \expectationof{\empdistributionat{\indexedshortcatvariables}} =  \gendistributionat{\indexedshortcatvariables} \, . \]

% Fluctuation
At finite $\datanum$ the empirical distribution differs from the by the difference
\[ \empdistribution - \gendistribution \]
which we call a fluctuation tensor.

\subsubsection{Mean parameter of the empirical distribution}

We now investigate the empirical mean parameter
\[
    \datameanat{\selvariable} = \contractionof{\sencsstatwith,\empdistributionat{\shortcatvariables}}{\selvariable} \, .
\]

Each coordinate of $\datamean$ is decomposed as
\[ \datameanat{\indexedselvariable} = \frac{1}{\datanum}\sum_{\datindexin} \sstatcoordinateofat{\selindex}{\datapointof{\datindex}} \]
and thus stores the empirical average of the feature $\sstatcoordinateof{\selindex}$ on the dataset $\data$.

% Expectation of the empirical mean
Since the mean parameter depends linearly on the corresponding distribution, we can show the following correspondence between the empirical and the expected mean parameter.

\begin{theorem}
    \label{the:expectedMeanParameter}
    When drawing data independently from $\gendistribution$, we have $\expectationof{\datameanat{\selvariable}}=\genmeanat{\selvariable}$, where we call
    \[
        \genmeanat{\selvariable} = \contractionof{\sencsstatwith,\empdistributionat{\shortcatvariables}}{\selvariable} \,
    \]
    the expected mean parameter.
\end{theorem}
\begin{proof}
    Since the expectation commutes with linear functions.
%    Since the mean parameter of a distribution depends linearly on the distribution.
\end{proof}


% Convergence by Law of Large Numbers and issues
For each $\selindexin$ the law of large numbers guarantees that $\genmeanat{\indexedselvariable}$ converges almost surely against $\genmeanat{\indexedselvariable}$ when $\datanum\rightarrow\infty$.
To utilize these we need to approach the following issues:
\begin{itemize}
    \item We need non-asymptotic convergence bounds, since one has access to finite data when learning
    \item The convergence has to happen uniformly for all $\selindexin$
    \item Guarantees on the result of an estimated model are more accessible when provided for quantities like the canonical parameter and KL-divergences of the learning result.
    Those, however, depend nonlinearly on $\datameanat{\selvariable}$ and therefore require further investigation.
\end{itemize}

\subsubsection{Noise tensor and its width}

% Definition of noise tensors
Motivated by \theref{the:expectedMeanParameter}, we build our derivation of probabilistic guarantees on non-asymptotic and uniform convergence bounds for $\datameanat{\selvariable}$.
Let us first define the fluctuations of the empirical mean parameter, when drawing the data independently from a random distribution, as the noise tensor.

\begin{definition}
    Given a statistic $\sstat$, $\datanum\in\nn$ and a distribution $\gendistribution$, we call
    \[ \sstatnoise = \sbcontractionof{(\empdistribution-\gendistribution),\sencsstat}{\selvariable} \]
    the \emph{noise tensor}, where $\datamap$ is a collection of $\datanum$ independent samples of $\gendistribution$.
\end{definition}

% Naive Ex
The fluctuation of the empirical distribution around the generating distribution corresponds in this notation with the minterm exponential family, taking the identity as statistics.
% Appearances
Besides this, fluctuation tensors appears in Markov Logic Networks as fluctuations of random mean parameters and in proposal distributions as fluctuation of random energy tensor.
We will discuss these examples in the following sections.


% Fluctuation of mean parameter
We notice, that the fluctuation tensor $\sstatnoise$ is the centered mean parameter to the empirical distribution, that is
\begin{align*}
    \datamean - \expectationof{\datamean} =  \sbcontractionof{\sencsstat,\empdistribution-\gendistribution}{\selvariable} \, .
\end{align*}

% Widths
In the following we will use the supremum of contractions with random tensors in the derivation of success guarantees for learning problems.
Such quantities are called widths.

\begin{definition}
    Given a set $\canparamhypothesis\subset\facspace$ and $\noisetensor$ a random tensor taking values in $\facspace$ we define the width as the random variable
    \[ \widthwrtof{\canparamhypothesis}{\noisetensor} = \sup_{\canparamin} \absof{\sbcontraction{\canparam,\noisetensor}} \, . \]
\end{definition}

% Uniform concentration events
Bounds on the widths are also called uniform concentration bounds \cite{goessmann_uniform_2021} and generic probabilistic bounds will be provided in \secref{sec:directWidthBounds} and  \secref{sec:chainingWidthBounds}.

\subsection{Error bounds based on the noise width}

We now derive error bounds for parameter estimation and structure learning, as introduced in \charef{cha:networkReasoning}.
When combined with probabilistic bounds on the noise width, they are probabilistic success guarantees.

\subsubsection{Parameter Estimation}

\red{We in this section always assume, that $\empdistribution$ is representable by the base measure $\basemeasure$ of the respective exponential families.}

Parameter Estimation is the M-projection of the empirical distribution onto an exponential family.
In \charef{cha:probReasoning} we have characterized those by the backward map acting on the mean parameter.
Thus, while we are interested in the expected canonical parameter
\[
    \gencanparamat{\selvariable} = \backwardmapof{\genmeanat{\selvariable}}
\]
we get an estimation by the empirical canonical parameter
\[
    \datacanparamat{\selvariable}  = \backwardmapof{\datameanat{\selvariable}} \, .
\]

% Nonlinearity
Unfortunately, since the backward map is not linear, we in general do not have that $\expectationof{\backwardmapof{\datamean}}$ coincides with $\backwardmapof{\genmean}$.
To build intuition on the concentration we recall the expression of the backward map as
% Concentration
\begin{align*}
    \backwardmapof{\meanparam}
    = \argmax_{\canparam} -\centropyof{\meanrepprob}{\stanexpdistof{\canparam}}
\end{align*}
where $\meanrepprob$ is any distribution reproducing the mean parameter.
We want to compare the solutions $\backwardmapof{\datamean}$ and $\backwardmapof{\genmean}$, in which case $\meanrepprob$ can be chosen as $\empdistribution$ and $\gendistribution$.
It is common to call the objectives $\centropyof{\empdistribution}{\stanexpdistof{\canparam}}$ and $\centropyof{\gendistribution}{\stanexpdistof{\canparam}}$ empirical and expected risk \cite{shalev-schwartz_understanding_2014,goesmann_uniform_2021}
Since the empirical risk has a linear dependence on $\datamean$, we have at each $\canparam$
\begin{align*}
    \expectationof{\centropyof{\empdistribution}{\stanexpdistof{\canparam}}}
    &= \expectationof{\contraction{\datamean,\canparam} - \cumfunctionof{\canparam}} \\
    &= \contraction{\expectationof{\datamean},\canparam} - \cumfunctionof{\canparam} \\
    &= \centropyof{\gendistribution}{\stanexpdistof{\canparam}}
\end{align*}
By the law of large numbers, in the limit $\datanum\rightarrow\infty$ we thus have at each $\canparam$ a convergence of the empirical risk to the expected risk.
However, since the backward map is defined by the minima of these risks, we need a uniform and non-asymptotical concentration guarantee to get more useful bounds.
To this end, we now relate the supremum on the differences between expected and empirical risks with the width of the noise tensor.

\begin{lemma}
    \label{lem:centropyWidthCharacterization}
    For any $\canparamhypothesis$ and $\datamap$ we have
    \begin{align*}
        \widthwrtof{\canparamhypothesis}{\noisetensor}
        = \sup_{\canparamin} \absof{\centropyof{\empdistribution}{\stanexpdistof{\canparam}} - \centropyof{\gendistribution}{\stanexpdistof{\canparam}}}
    \end{align*}
\end{lemma}
\begin{proof}
    For any $\canparam\in\canparamhypothesis$ and by $\meanrepprob$ realizable mean parameter $\meanparam$ we have
    \begin{align*}
        \centropyof{\meanrepprob}{\stanexpdistof{\canparam}}
        = - \contraction{\meanparam,\canparam} -
    \end{align*}

    Using the decomposition of cross entropy in the exponential family
    \[ \centropyof{\empdistribution}{\stanexpdistof{\canparam}}
    =\contraction{\probtensor,\lnof{\gendistribution}} - \cumfunctionof{\lnof{\gendistribution}} \, . \]
\end{proof}

As a direct consequence, we have at any $\canparam\in\canparamhypothesis$
\begin{align*}
    \absof{\centropyof{\empdistribution}{\stanexpdistof{\canparam}} - \centropyof{\gendistribution}{\stanexpdistof{\canparam}}}
    \leq \widthwrtof{\canparamhypothesis}{\noisetensor} \, .
\end{align*}
Thus, the absolute difference of the expected risk and the empirical risk is bounded by the width of the noise tensor.
This is especially useful for the solution $\datamean$ of the empirical risk minimization, where we can state
\begin{align*}
    \centropyof{\gendistribution}{\stanexpdistof{\datacanparam}}
    \leq \centropyof{\empdistribution}{\stanexpdistof{\datacanparam}} + \widthwrtof{\canparamhypothesis}{\noisetensor} \, .
\end{align*}
At the solution of a empirical risk minimization problem over $\canparamhypothesis$, the expected risk exceeds the empirical risk at most by the noise tensor width.

% Further KL divergence bound when assuming gendistribution in the hypothesis
When the generating distribution is in the hypothesis, we can further show the following KL-divergence bound for the estimated distribution.

\begin{theorem}
    Let us assume that for $\gencanparam\in\canparamhypothesis$ we have $\gendistribution=\stanexpdistof{\gencanparam}$. %and that $\partitionfunctionof{\canparam}$ is constant among $\canparamin$.
    Then for any solution $\datacanparam$ of the empirical problem we have
    \begin{align}
        \kldivof{\stanexpdistof{\gencanparam}}{\stanexpdistof{\datacanparam}} \leq 2\widthwrtof{\canparamhypothesis}{\noisetensor} \, .
    \end{align}
\end{theorem}
\begin{proof}
    For the solution $\datacanparam$ of the empirical risk minimization on $\canparamhypothesis$ we have since $\gencanparam\in\canparamhypothesis$ that
    \begin{align*}
        \centropyof{\empdistribution}{\stanexpdistof{\datacanparam}}
        \leq \centropyof{\empdistribution}{\stanexpdistof{\gencanparam}} \, .
    \end{align*}
    It follows that
    \begin{align*}
        \kldivof{\stanexpdistof{\gencanparam}}{\stanexpdistof{\datacanparam}}
        & \leq \kldivof{\stanexpdistof{\gencanparam}}{\stanexpdistof{\datacanparam}}
        + \centropyof{\empdistribution}{\stanexpdistof{\gencanparam}}
        - \centropyof{\empdistribution}{\stanexpdistof{\datacanparam}} \\
        & = \left(\centropyof{\stanexpdistof{\gencanparam}}{\stanexpdistof{\datacanparam}} - \centropyof{\empdistribution}{\stanexpdistof{\datacanparam}}\right) \\
        & \quad - \left(\centropyof{\stanexpdistof{\gencanparam}}{\stanexpdistof{\gencanparam}} - \centropyof{\empdistribution}{\stanexpdistof{\gencanparam}}\right) \, ,
    \end{align*}
    where we expanded the KL-divergence as a difference of cross entropies.
    We apply \lemref{lem:centropyWidthCharacterization} to estimate the terms in brackets and get
    \begin{align*}
        \left(\centropyof{\stanexpdistof{\gencanparam}}{\stanexpdistof{\datacanparam}} - \centropyof{\empdistribution}{\stanexpdistof{\datacanparam}}\right)
        - \left(\centropyof{\stanexpdistof{\gencanparam}}{\stanexpdistof{\gencanparam}} - \centropyof{\empdistribution}{\stanexpdistof{\gencanparam}}\right)
        \leq 2 \widthwrtof{\canparamhypothesis}{\noisetensor} \, .
    \end{align*}
    Combined with the above inequality we arrive at
    \begin{align}
        \kldivof{\stanexpdistof{\gencanparam}}{\stanexpdistof{\datacanparam}} \leq 2\widthwrtof{\canparamhypothesis}{\noisetensor} \, . \qedhere
    \end{align}
\end{proof}

\subsubsection{Structure Learning}

In the gradient heuristic of structure learning, one selects the statistic to the maximal coordinate of the energy tensor of the proposal distribution.
This tensor coincides with the mean parameter of a markov logic network and has thus a fluctuation by the noise tensor.
We now use these insights to show a guarantee, that the formula chosen by grafting with respect to the empirical proposal distribution coincides with the formula chosen with respect to the expected proposal distribution.
To this end, we need to define the max gap, which is the difference between the maximal coordinate of a tensor to the second maximal coordinate.

\begin{definition}
    The max gap of a tensor $\hypercoreat{\shortcatvariables}$ is the quantity
    \begin{align*}
        \maxgapof{\hypercore} =
        \left(\max_{\shortcatindices} \hypercoreat{\indexedshortcatvariables}\right) -
        \left(\max_{\shortcatindices\notin\argmax_{\shortcatindices}\hypercoreat{\indexedshortcatvariables}}
        \hypercoreat{\indexedshortcatvariables}\right) \, .
    \end{align*}
\end{definition}

When comparing the gap with the noise width, we get the following guarantee.

\begin{theorem}
    \label{the:detGuaranteeProposalDist}
    Whenever
    \begin{align*}
        \maxgapof{\genmean}
        > 2 \cdot \widthwrtof{\{\onehotmapof{\shortcatindices}:\shortcatindices\in\facstates\}}{\sstatnoise} \, ,
    \end{align*}
    then any mode $\shortcatindices$ of the empirical proposal distribution is a mode of the expected proposal distribution.
\end{theorem}
\begin{proof}
    Let us assume that for a mode $\selindex^{\datamap}\in\argmax_{\selindexin}\datameanat{\indexedselvariable}$ of the empirical mean parameter we have
    \begin{align*}
        \selindex^{\datamap}\notin\argmax_{\selindexin}\genmeanat{\indexedselvariable} \, .
    \end{align*}
    For a mode $\selindex^{*}\in\argmax_{\selindexin}\genmeanat{\indexedselvariable}$ of the expected mean parameter we then have
    \begin{align*}
        \genmeanat{\selvariable=\selindex^{\datamap}} \leq \genmeanat{\selvariable=\selindex^{*}} - \maxgapof{\genmean}
    \end{align*}
    and
    \begin{align*}
        \datameanat{\selvariable=\selindex^{\datamap}} \geq \datameanat{\selvariable=\selindex^{*}} \, .
    \end{align*}
    Comparing both intequalities we get
    \begin{align*}
        \left(\datameanat{\selvariable=\selindex^{\datamap}} - \genmeanat{\selvariable=\selindex^{\datamap}}\right)
        + \left( - \datameanat{\selvariable=\selindex^{*}} + \genmeanat{\selvariable=\selindex^{*}} \right)
        \geq \maxgapof{\genmean} \, .
    \end{align*}
    Estimating the terms in the bracket by the width of the noise tensor with respect to basis vectors, we get
    \begin{align*}
        2 \cdot  \widthwrtof{\{\onehotmapof{\shortcatindices}:\shortcatindices\in\facstates\}}{\sstatnoise}
        \geq \maxgapof{\genmean} \, ,
    \end{align*}
    which is a contradiction to the assumption.
    Thus, any mode of the empirical mean parameter is also a model of the expected mean parameter.
\end{proof}


\subsection{Fluctuations in Logic Networks}

\red{In case of logical formulas being statistics, the coordiantes of the mean parameter are satisfaction rates to the formulas.}

For Logic Networks we have statistics consistent of boolean statistics $\enumformula$, which are logical formulas.
In this case the marginal distributions of the coordinates of $\sstatnoise$ are scaled and centered binomials, which we show now.

\begin{lemma}
    Let $\sstat$ be a statistic of boolean features $\sstatcoordinate$ for all $\selindexin$, i.e. let $\imageof{\sstatcoordinate}\subset\ozset$.
    Then, the marginal distribution of the coordinate $\sstatnoise[\indexedselvariable]$ is
    \[\frac{1}{\datanum}\left(\bidistof{\fprobof{\selindex},\datanum}- \fprobof{\selindex}\right)  \, , \]
    where by $\bidistof{\fprobof{\selindex},\datanum}$ we denote the binomial distribution with mean parameter
    \[ \fprobof{\selindex} = \sbcontraction{\sstatcoordinate,\gendistribution} \, . \]
\end{lemma}
\begin{proof}
    We notice that when forwarding a random sample $\datapoint$ of $\gendistribution$ is the random tensor
    \[ \onehotmapofat{\datapoint}{\shortcatvariables} \, \]
    and since $\imageof{\sstatcoordinate}\subset \{0,1\}$ the contraction
    \[ \sbcontraction{\sstatcoordinate, \onehotmapofat{\datapoint}{\shortcatvariables}} \]
    is a random variable taking values in $\{0,1\}$.
    The variable therefore follows a Bernoulli distribution with mean parameter
    \[ \fprobof{\selindex}
    = \expectationof{\sbcontraction{\sstatcoordinate, \onehotmapofat{\datapoint}{\shortcatvariables}}}
    = \sbcontraction{\sstatcoordinate, \gendistribution}  \, \qedhere\]
\end{proof}

%\subsubsection{Mean parameter in Markov Logic Networks}

The mean parameter of the M-projection of the empirical distribution on the family of Markov Logic Networks with statistic $\fselectionmap$ is the random tensor
\begin{align*}
    \datameanat{\selvariable}
    = \sbcontractionof{\sencmlnstat,\empdistribution}{\selvariable} \, .
\end{align*}

The expectation of this random tensor is
\begin{align*}
    \expectationof{\datamean}
    =  \sbcontractionof{\sencmlnstat,\expectationof{\empdistribution}}{\selvariable}
    =  \sbcontractionof{\sencmlnstat,\gendistribution}{\selvariable}
    =  \genmean \, ,
\end{align*}
where we used that the expectation and contraction operation can be commuted due to the multilinearity of contractions.

\subsubsection{Energy tensor in proposal distributions}

The fluctuation tensor appears as a fluctuation of the energy of the proposal distribution.
The expectation of the energy of the proposal distribution is
\begin{align*}
    \expectationof{\energytensorof{\proposalstat,\empdistribution-\currentdistribution}}
    = \expectationof{\sbcontractionof{\sencproposalstat,\empdistribution-\currentdistribution}{\selvariable}}
    = \sbcontractionof{\sencproposalstat,\expectationof{\empdistribution-\currentdistribution}}{\selvariable}
    = \sbcontractionof{\sencproposalstat,\gendistribution-\currentdistribution}{\selvariable}
    = \expectationof{\energytensorof{\proposalstat,\gendistribution-\currentdistribution}} \, .
\end{align*}

% Fluctuation
The fluctuation of this random tensor is
\begin{align*}
    \expectationof{\energytensorof{\proposalstat,\empdistribution-\currentdistribution}}  - \expectationof{\energytensorof{\proposalstat,\gendistribution-\currentdistribution}}
    = \expectationof{\energytensorof{\proposalstat,\empdistribution-\gendistribution}}
\end{align*}
and coincides with $\mlnnoise$.

\subsubsection{Minterm Exponential Family} % Interesting, since here is the connection with probability tensors: Forwarding of each random datapoint by the one hot encoding to get a multinomial random tensor.

In case of the minterm exponential family, we have $\sstat=\identityat{\shortcatvariables,\selvariable}$ and the fluctuation tensor is
\[ \mintermnoise = \empdistribution - \gendistribution \, .  \]

% Multinomial
This fluctuation tensor is related to tensor encodings of multinomial distributions, which we now define as multinomial random tensors.

\begin{definition}
    \label{def:mulinomialVariable}
    A multinomial random tensor is the sum of the one-hot encodings of independent identically distributed random states $x^\datindex$, drawn from a distribution $\probtensor$, that is
    \[ Z^{\probtensor, \datanum} = \sum_{\datindexin} \onehotmapofat{x^\datindex}{\shortcatvariables} \, . \]
\end{definition}

% Multinomial as a more general characterization
In the case of minterm exponential families, the fluctuation tensor is a multinomial, as we show next.
This characterization goes beyond the characterization of the marginal distributions as centered binomial variables, which holds for generic Markov Logic Networks.

\begin{lemma}
    \label{lem:multinomialEmpdistFluctuation}
    The fluctuation $\empdistribution - \gendistribution$ is a by $\frac{1}{\datanum}$ rescaled centered multinomial random tensor with parameters $\gendistribution$ and $\datanum$. % Needs some more explanation based on one-hot encodings?
\end{lemma}
\begin{proof}
    By the above construction we have
    \[  \empdistribution - \gendistribution
    = \frac{1}{\datanum}\sum_{\datindexin} \left( \onehotmapofat{\datapoint}{\shortcatvariables} - \expectationof{\onehotmapofat{\datapoint}{\shortcatvariables}} \right) \, .  \]
\end{proof}

\subsubsection{Guarantees for Mode of the Proposal Distribution}

Let us now derive probabilistic guarantees, that the mode of the proposal distribution at the empirical and the generating distribution are equal.

\begin{theorem}
    \label{the:probGuaranteeProposalDist}
    Whenever the energy tensor of the expected proposal distribution has a gap of $\maxgap$, then for every $\failprob>0$ any mode of the empirical proposal distribution coincides is also a mode of the expected proposal distribution with probability at least $1-\expof{-\frac{1}{\failprob^2}}$, provided that
    \[ \datanum > C\frac{\left(\sum_{\atomenumeratorin}\lnof{\catdimof{\atomenumerator}}\right)}{\maxgap^2} \]
    where $C$ is a universal constant.
\end{theorem}
\begin{proof}
    To proof the theorem we combine the deterministic guarantee \theref{the:detGuaranteeProposalDist} with the width bound of \theref{the:basisTensorWidthBound}.
    Given the assumed bound, the sub-gaussian norm of the width is upper bounded by $C_2\cdot \maxgap$, thus for any $\failprob>0$ we have
    \[  \widthwrtof{\{\onehotmapof{\shortcatindices} :\shortcatindices\in\facstates\}}{\mlnnoise}  < 2 \maxgap \]
    with probability at least $1-\expof{-\frac{1}{\failprob^2}}$.
    The claim thus follows with \theref{the:detGuaranteeProposalDist}.
\end{proof}


\begin{example}[Gap of a MLNs with single formulas]
    Let there be the MLN of a maxterm $\formula$ with $\atomorder$ variables, and let $\formulaset$ be the maxterm selecting tensor, then
    \[ \maxgap(
    \energytensorof{(\formulaset, \expdistof{(\{\formula\},\weightof{\formula})} - \normationof{\ones}{\shortcatvariables} )}
    ) = \frac{1}{2^{\atomorder}-1 + \expof{-\weightof{\formula}}}  \]
    If $\weightof{\formula}>0$ we have an exponentially small gap.
    Thus, for the above Lemma to apply, the width needs to be exponentially in $\atomorder$ small.


    Let there be the MLN of a minterm $\formula$ with $\atomorder$ variables, then
    \[ \maxgap(
    \energytensorof{(\formulaset, \expdistof{(\{\formula\},\weightof{\formula})} - \normationof{\ones}{\shortcatvariables} )}
    ) = \frac{1}{1+(2^{\atomorder}-1)\cdot\expof{-\weightof{\formula}}}  \]
    For large $\weightof{\formula}$ and $\atomorder$, the gap tends to $1$.
\end{example}

\subsubsection{Guarantees for Parameter Estimation}

\red{This is mean parameter fluctuation interpretation of the random tensor.}

\begin{lemma}
    \label{lem:meanParamDistance}
    For any $\mlnstat$ and $\datamap$ drawn from $\gendistribution$ we have
    \begin{align*}
        \normof{\datamean - \genmean}
        = \widthwrtof{\subsphere}{\mlnnoise} \, ,
    \end{align*}
    where $\datamean=\sbcontractionof{\sencmlnstat,\empdistribution}{\selvariable}$ and $\genmean=\sbcontractionof{\sencmlnstat,\gendistribution}{\selvariable}$.
\end{lemma}

%
We can thus apply the sphere bounds.


\begin{theorem}
    For any $\failprob\in(0,1)$ we have the following with probability at least $1-\failprob$.
    Let $\hat{\canparam}$ and $\precision>0$, then
    \[ \absof{\centropyof{\gendistribution}{\mlnexpdistof{\datacanparam}} - \centropyof{\empdistribution}{\mlnexpdistof{\datacanparam}}} \leq \tau \cdot \normof{\datacanparam} \]
    provided that
    \[ \datanum \geq \frac{\sbcontraction{\genmean}-\sbcontraction{(\genmean)^2}}{\failprob \precision^2} \, . \]
\end{theorem}
\begin{proof}
    We have by Cauchy Schwartz
    \[ \absof{\sbcontraction{\datamean - \genmean,\datacanparam}} \leq \normof{\datamean - \genmean} \cdot \normof{\datacanparam}\]
    and with \lemref{lem:meanParamDistance}
    \[ \absof{\sbcontraction{\datamean - \genmean,\datacanparam}} \leq \widthwrtof{\subsphere}{\mlnnoise} \cdot \normof{\datacanparam} \, . \]
    We show in Part III that in \theref{the:sphereBoundVariance} that
    \[  \widthwrtof{\subsphere}{\mlnnoise} \leq \tau \]
    when $\datanum$ satisfies the assumed lower bound, from which the claim follows.
\end{proof}


















    %\section{Concentration of the Expected Sufficient Statistics}
\section{Uniform Concentration of Random Contractions}\label{cha:widthBounds}

We here derive bounds on the uniform concentration of contractions with random tensors.




	
	
	

The width of a vector $\noisetensor$ is the supremum of contractions with respect to a set $\Gamma$ is
\begin{align}
	\widthwrtof{\Gamma}{\noisetensor}
	= \sup_{\theta\in\Gamma} \sbcontraction{\theta,\noisetensor} \, . 
	%= \sup_{\theta\in\Gamma} \contractionof{\{\theta, \noisetensor\}}{\varnothing} \, . 
\end{align}

We are interested in the random vector
	\[ \noisetensor^{\formulaset,\gendistribution,\datamap} = \sbcontractionof{\empdistribution,\formulaset}{\selvariable} -  \sbcontractionof{\gendistribution,\formulaset}{\selvariable}  \,  \]
	%\essdistof{\empdistribution}-\expectationof{\essdistof{\empdistribution}} \]
which is the difference between the mean parameters given the empirical distribution and the underlying generating distribution.

We derive bounds for the hypothesis $\Gamma$ being the set of basis vectors (i.e. for feature search) and being the set of normed vectors (i.e. for feature calibration).





%\subsection{Binomials}
%\subsubsection{Concentration Bounds for Binomials}





\subsection{Naive bounds given binomial coordinates}


We here investigate width bounds on random tensors $\noisetensor$, which coordinates have marginal distributions by Binomials.

% MLN
This is the case for $\noisetensor^{\fselectionmap,\gendistribution,\datanum}$.

Naive means, that we do not exploit the dependencies of the coordinates on each other, as would be the case in more sophisticated chaining approaches.

\subsubsection{Basis Vectors}

We exploit the sub-Gaussian Norm (Def 2.5.6 in CITE Vershynin Book) to state concentration inequalities.

\begin{definition}[Sub-Gaussian Norm]
	The sub-Gaussian norm of a random variable $X$ is defined as
		\[ \sgnormof{X} = \inf \left\{ C > 0 \, : \expectationof{\frac{X^2}{C^2} } \leq 2 \right\} \, .  \]
\end{definition}

\begin{theorem}
	Any coordinate of $\noisetensor$ is sub-Gaussian with 
		\[ \sgnormof{\noisetensor_i} \leq \frac{C}{\sqrt{\ln2}} \frac{1}{\sqrt{m}} \]
\end{theorem}
\begin{proof}
	Centered Bernoulli is bounded and therefore sub-Gaussian.
	Binomial is a sum and we apply Proposition 2.6.1 in CITE Vershynin Book.
\end{proof}


\begin{theorem}\label{the:basisTensorWidthBound}
	Let 
		\[ \Gamma = \{\onehotmapof{i} : i \in [p] \}\]
	and $\noisetensor$ be a random vector in $\rr^p$, which coordinates have a marginal distribution being centered Binomials with a fixed $\datanum\in\nn$.
	Then
		\[ \sgnormof{\widthatof{\Gamma}{\noisetensor}} \leq C \sqrt{\frac{\ln p}{m}} \, . \]
	where $C>0$ is a universal constant.
\end{theorem}
\begin{proof}
	Supremum of Sub-Gaussian variables.
\end{proof}


\subsubsection{Sphere}

We first provide a Chebyshev bound on the width of the sphere.

\begin{theorem}\label{the:sphereBoundVariance}
	Let $\noisetensor$ be a random tensor with marginal coordinate distributions by binomials with parameters $(\fprobof{\catindex},\datanum)$.

	For any $\failprob>0$, $\precision>0$ and $\datanum\in\nn$ with probability at least $1-\failprob$ we have
		\[ \normof{\frac{\noisetensor-\expectationof{\noisetensor}}{\datanum}} \leq   \precision \, \]
	provided that
		\[ \datanum \geq  \frac{ \sum_{\catindex\in\facstates} \fprobof{\catindex}(1-\fprobof{\catindex})}{\precision^2 \failprob} \, . \]
\end{theorem}
\begin{proof}
	%We estimate with the Cauchy Shwartz inequality
	%	\[ \braket{\theta,\noisetensor} \leq \|\noisetensor\| \|\theta\| \, . \]
	Since the squared norm of the noise is the sum of squared centered and averaged Binomials, we have
		\[  \expectationof{\normof{\noisetensor-\expectationof{\noisetensor}}^2}  
		= \sum_{\catindex\in\facstates} \frac{\fprobof{\catindex}(1-\fprobof{\catindex})}{\datanum} \]
	Here we used that the variance of Binomials with parameters $(\fprobof{\catindex},\datanum)$ is $\fprobof{\catindex}(1-\fprobof{\catindex}) \datanum$.
	
	If follows, that 
		\[ \expectationof{\left(\normof{\frac{\noisetensor-\expectationof{\noisetensor}}{\datanum}}\right)^2} =  \frac{ \sum_{\catindex\in\facstates} \fprobof{\catindex}(1-\fprobof{\catindex})}{\datanum} \, . \]
	
	Then we apply a Chebyshev Bound to get for any $\precision>0$
	\begin{align}
		\probof{\normof{\frac{\noisetensor-\expectationof{\noisetensor}}{\datanum}} > \precision} 
		= \probof{\left(\normof{\frac{\noisetensor-\expectationof{\noisetensor}}{\datanum}}\right)^2 > \precision^2} 
		\leq \frac{ \sum_{\catindex\in\facstates} \fprobof{\catindex}(1-\fprobof{\catindex})}{\datanum \cdot \precision^2}
	\end{align} 
	For a $\failprob>0$ we choose any $\datanum$ with
		\[ \datanum \geq  \frac{ \sum_{\catindex\in\facstates} \fprobof{\catindex}(1-\fprobof{\catindex})}{\precision^2 \failprob} \, \]
	and get 
	\begin{align}
		\probof{\normof{\frac{\noisetensor-\expectationof{\noisetensor}}{\datanum}} > \precision} \leq \failprob \, . 
	\end{align} 
	Thus, we have 
	\begin{align}
		\probof{\normof{\frac{\noisetensor-\expectationof{\noisetensor}}{\datanum}} \leq \precision} = 1 - \probof{\normof{\frac{\noisetensor-\expectationof{\noisetensor}}{\datanum}} > \precision}  \geq 1-\failprob \, . 
	\end{align} 
\end{proof}


% Multinomial
For $\sstat=\identity$ the noise tensor is a rescaled and centered multinomial.

\begin{corollary}
	Let there be multinomial variable with parameters $(\fprob,\datanum)$ where $\fprob\in\facspace$ a positive and normed tensor.
	Let $\datamap$ be a set of independent samples
	For any $\failprob>0$, $\precision>0$ and $\datanum\in\nn$ with probability at least $1-\failprob$ we have
		\[ \normof{\frac{\noisetensor-\expectationof{\noisetensor}}{\datanum}} \leq   \precision \, \]
	provided that
		\[ \datanum \geq  \frac{(1-\sbcontraction{(\fprob)^2})}{\precision^2 \failprob} \, . \]
\end{corollary}
\begin{proof}
	Theorem~\ref{the:sphereBoundVariance} with 
		\[ \sum_{\catindex\in\facstates} \fprobof{\catindex}(1-\fprobof{\catindex}) = \sum_{\catindex\in\facstates} \fprobof{\catindex} - \sum_{\catindex\in\facstates} (\fprobof{\catindex})^2 = 1-\sbcontraction{(\fprob)^2} \, . \]
\end{proof}



\subsubsection{Bounds based on the sub-gaussian norm}

\red{Unclear whether this is needed.}

A faster tail decay can be achieved, when bounding sub-gaussian norms.


\begin{theorem}
	Let $\noisetensor$ be a random vector in $\rr^p$, which coordinates have a marginal distribution being centered Binomials with a fixed $\datanum\in\nn$.
	Then
		\[ \sgnormof{\widthatof{\subsphere}{\noisetensor}} \leq C \sqrt{\frac{p}{m}} \, . \]
	where $C>0$ is a universal constant.
\end{theorem}
\begin{proof}
	Using that each coordinate has Sub-gaussian norm of at most $1$.
	\red{Asymptotically, the binomial tends to a gaussian, which has a smaller sg norm. 
	But the binomial has a sub-exponential regime preventing tighter sg bounds.}

	
	Norm of a Sub-gaussian vector, another application of Proposition 2.6.1 in CITE Vershynin Book.
\end{proof}




\subsection{Chaining bounds given binomial coordinates}

To proceed with the uniform concentration investigation, we need a concentration bound on Binomials.

\begin{theorem}
	For any $p\in[0,1]$ and $\datanum\in\mathbb{N}$ any $X \sim \bidistof{p,\datanum}$ satisfies for any $t>0$
		\[ \probof{X-\expectationof{X} > t}  \leq \expof{- \frac{t^2}{2\datanum p + \frac{2t}{3}} } \]
	and
		\[ \probof{\expectationof{X} - X > t}  \leq \expof{- \frac{t^2}{2\datanum p}} \]
	Thus
		\[ \probof{|\expectationof{X} - X| > t} \leq  2 \expof{- \frac{t^2}{2\datanum p + \frac{2t}{3}} }  \]
\end{theorem}
\begin{proof}
	See e.g.
	\href{https://mathweb.ucsd.edu/~fan/wp/concen.pdf}{https://mathweb.ucsd.edu/~fan/wp/concen.pdf}
%	The proof uses the Chernoff bound applying the moment generating function, which is for Binomial variables X and $\lambda\geq0$
%	\begin{align}
%	 	\expectationof{\expof{\lambda (X - \expectationof{X})}} 
%		= & \left( (1-p) \cdot \expof{\lambda -p} + p \cdot \expof{\lambda (1-p)}\right)^\datanum \\
%		= & \expof{-\lambda p \datanum} \left(1 + p(\expof{\lambda}-1) \right)^\datanum \, .
%	\end{align}
%	The Chernoff bound used for any $\lambda>0,t>0$ the Markov inequality
%	\begin{align}
%		\probof{X-\expectationof{X} > t} 
%		= \probof{\expof{\lambda(X-\expectationof{X})} > \expof{\lambda t} }
%		\leq \frac{\expectationof{\expof{\lambda(X-\expectationof{X})}}}{\expof{\lambda t} } 
%	\end{align}
\end{proof}

The binomial thus has a sub-gaussian and a sub-exponential regime.

\begin{theorem}
	For any $p\in[0,1]$ and $\datanum\in\mathbb{N}$ any $X \sim \bidistof{p,\datanum}$ satisfies for any $t>0$
		\[ \probof{|\expectationof{X} - X| > \sqrt{4\datanum p t} + 2 t} \leq  2 \expof{- t }  \]
\end{theorem}
\begin{proof}
	For any s>0 we choose t>0 such that 
		\[ s = - \frac{t^2}{2\datanum p + \frac{t}{3}}  \]
	and observe 
		\[ \min\left( \frac{t^2}{4\datanum p},\frac{t^2}{2t} \right) \]
	and
		\[ t \leq \max(\sqrt{4\datanum p s},2s) \leq  \sqrt{4\datanum p s} + 2s \, . \]
	With the above bound it holds, that
		\[  \probof{|\expectationof{X} - X| >  \sqrt{4\datanum p s} + 2s}
		\leq \probof{|\expectationof{X} - X| > t}
		\leq 2 \expof{- \frac{t^2}{2\datanum p + \frac{t}{3}} } 
		\leq 2 \expof{-s} \, . \]
\end{proof}

We apply this on the variable  $\braket{\ftensor,\noisetensor}$.

\begin{theorem}
	Let $\ftensor = \sum_{\mlnformulain} \weightof{\exformula}\exformula$, then
	\[ \probof{\datanum |\braket{\ftensor,\noisetensor}|\geq \sqrt{4 \datanum} \cdot \left( \sum_{\mlnformulain} |\weightof{\exformula}| \sqrt{\fprobof{\exformula}} \right) \sqrt{t}  + 2 \cdot \left( \sum_{\mlnformulain} |\weightof{\exformula}|  \right) t } \leq 2|\mlnformulaset | \cdot \expof{- t} \]
\end{theorem}
\begin{proof}
	We can not assume independence of the $\braket{\exformula,\noisetensor}$ (in that case we could use a Bernstein inequality) and instead take the naive bound over all formulas in $\mlnformulaset$ 
	\begin{align}
		& \probof{|\braket{\ftensor,\noisetensor}|\geq
		 \sqrt{4 \datanum} \cdot \sum_{\mlnformulain} |\weightof{\exformula}| \sqrt{\fprobof{\exformula}}\sqrt{t}  
		 + 2 \cdot \sum_{\mlnformulain} |\weightof{\exformula}|  t } \\
		& \quad \quad \leq \probof{\exists \mlnformulain \, : \, |\braket{\exformula,\noisetensor}|\geq  \sqrt{4 \datanum}  |\weightof{\exformula}| \sqrt{\fprobof{\exformula}}\sqrt{t}  + 2 |\weightof{\exformula}|  t  }\\
		& \quad \quad \leq \sum_{\mlnformulain} \probof{ |\braket{\exformula,\noisetensor}|\geq  \sqrt{4 \datanum}  |\weightof{\exformula}| \sqrt{\fprobof{\exformula}}\sqrt{t}  + 2 |\weightof{\exformula}|  t  }\\
		& \quad \quad  \leq 2|\mlnformulaset| \cdot  \expof{-t} \, .
	\end{align}
\end{proof}

We can thus proof uniform concentration bounds given covering bounds of a hypothesis set in $\ell_1$ norm (and the reweighted one).

\begin{remark}[Small Formula Probabilities]	
	Directions with small $\fprobof{\exformula}$ will require larger covering sets and thus have large contributions to the bounds.
	Intuitively, they correspond with exceptional special cases, which need many samples to be observed. 
\red{Strange: Should also $1-\fprobof{\exformula}$ small intuitely be an issue?}
\end{remark}


\subsubsection{Generic Width Bounds}

We define the maps
	\[ \nu^p_2(\ftensor) =  \inf_{\mlnparameters : \sum_{\mlnformulain}\weightof{\exformula}\exformula = \ftensor}  \left( \sum_{\mlnformulain} |\weightof{\exformula}| \sqrt{\fprobof{\exformula}} \right) \]
and
	\[ \nu_1(\ftensor) =  \inf_{\mlnparameters : \sum_{\mlnformulain}\weightof{\exformula}\exformula = \ftensor} \left( \sum_{\mlnformulain} |\weightof{\exformula}| \sqrt{\fprobof{\exformula}} \right)  \]
corresponding to the sub-gaussian and sub-exponential regimes.

%This is almost the mixed tail definition of the thesis, just a factor on the probability

\begin{theorem}
	Let $\mlnformulaset$ be a set of formulas and $W\subset\rr^{|\mlnformulaset|}$ a set of weight vectors.
	Then with probability at least $1- |\mlnformulaset| C_1 \expof{-\frac{u^2}{2}} $ we have for the set
		\[ \Gamma = \big\{ \sum_{\mlnformulain}\weightof{\exformula}\exformula \, : \, (\weightof{\exformula})_{\mlnformulain} \in W \big\} \]
	the bound
		\[ \omega_\Gamma(\noisetensor)  \leq \frac{C_2 \gamma_{2}(\Gamma,\nu^p_2)}{\sqrt{\datanum}} + \frac{C_3 \gamma_{1}(\Gamma,\nu_1)}{\datanum} \, . \]
\end{theorem}



%% OLD: These situations are addressed more directly

%\subsubsection{Recovery of atom assignment in skeleton}
%
%We select for each placeholder in the skeleton an atom of $\variableorder$ possible choices.
%The set of formula tensors resulting from these choices is
%	\[ \Gamma^{\skeleton} = \left\{ \exformula \, : \, \exformula = \skeleton(\atomindices), \atomindices \in [\variableorder] \right\} \, .\]
%We can estimate the cardinality by
%	\[ \cardof{\Gamma^{\skeleton}} \leq \variableorder^\atomorder \, . \]
%This is just an inequality, since assignments of atoms to placeholders of the skeleton can result in identical formulas.
%
%When restricting choices by a candidatesdict, the bound can be sharpened by the product of the cardinality at each placeholder.
%
%\begin{theorem}
%	Let $\skeleton$ be a skeleton formula with $\atomorder$ placeholders and $\variableorder$ atoms, which can be selected at each position.
%	Then we have with probability at least $1-C_1\expof{-\frac{u^2}{2}}$
%		\[ \omega_\skeleton(\noisetensor)  \leq 2C_2 \sqrt{\frac{ \atomorder \ln\variableorder}{\datanum}} + 2C_3\frac{\atomorder\ln\variableorder}{\datanum}  \]
%\end{theorem}
%\begin{proof}
%	Application of the generic Dudleys entropy bound.
%\end{proof}
%
%Thus
%	\[ \datanum \sim \atomorder\ln(\variableorder) \]
%is enough for a sharp Kullback Leibler bound of the solution.
%
%
%\subsection{Recovery of weight parameters }
 % To be integrated in Concentration!

    \chapter{\chatextfolModels}\label{cha:folModels}

We now extend the tensor representation from to structured representations, whereas we previously focused on factored representation of systems.
%The models/events in this situation are precise relations between objects.


\red{We observe that the more expressive first-order logic bears another tensor structure:
The representation of each world is a boolean tensor.
}


% Formulas in FOL
%Formulas in first order logic can contain variables, which are placeholder for specific individuals.
%Given a model and an assignment of objects to the arguments of a formula, the truth of the formula can be interpreted.
%This truth interpretation defines thus for any model a tensor, which we call the grounding tensor.

%
\sect{World Tensors}

% Index interpretation of world domain
Since first-order logic follows structured representations of a system, a first-order logic world consists in objects and relations between them.
To each world there is a world domain $\worlddomain$ of objects, which we assume to be finite (this is a restrictive assumption).
We exploit the set-encoding formalism discussed in more detail in \charef{cha:basisCalculus} and use bijective index interpretation maps
\begin{align*}
    \indexinterpretation : [\inddim] \rightarrow \worlddomain \, .
\end{align*}
A so-called term variable $\indvariable$ takes states $\indindexin$, which represent objects
\begin{align*}
    \indexinterpretationat{\indindex} \in \worlddomain \, .
\end{align*}

%
The relations between objects are described by $\indorder$-ary predicates $\folpredicate$.
Given a specific world $\dataworld$ the truth of relations is represented by boolean tensors
\begin{align*}
    \groundingof{\folpredicate} : \symindstates\rightarrow\ozset \, .
\end{align*}
Given a tuple $\indindexlist\in\symindstates$ the boolean
\begin{align*}
    \groundingofat{\folpredicate}{\indexedindvariableof{0},\ldots,\indexedindvariableof{\indorder-1}} \in\ozset
\end{align*}
is called a grounding and encodes, whether the relation $\folpredicate$ is satisfied in the world $\dataworld$ for the objects $\invindexinterpretationat{\indindexof{0}},\ldots,\invindexinterpretationat{\indindexof{\indorder-1}}$.

% Assumptions
Let us assume, that we have a function-free theory with $\folpredicateorder$ predicates, where are predicates all of the same arity $\variableorder$.
We then formalize a world in the following based on a selection variable $\selvariable$ selecting a specific predicate and term variables $\shortindvariablelist=\indvariablelist$ representing choices of objects from a given set $\worlddomain$.

\begin{definition}[FOL World]
    \label{def:folWorld}
    Given a set of objects $\worlddomain$ enumerated by an index interpretation function $\indexinterpretation:[\inddim]\rightarrow \worlddomain$ and a finite set $\{\folpredicates\}$ of $\variableorder$-ary predicates a world is a boolean tensor
    \begin{align}
        \dataworldwith : [\catorder] \times \left( \symindstates\right) \rightarrow [2] \, . % ! Selvariable gets catorder !
    \end{align}
    We interpret the world tensor as encoding in the coordinate $\dataworldat{\selvariable=\catenumerator,\indexedshortindvariables}$, whether the $\catenumerator$-th predicate is satisfied on the object tuple $\invindexinterpretationat{\indindexof{0}},\ldots,\invindexinterpretationat{\indindexof{\indorder-1}}$.
\end{definition}


% Inclusion of functions, predicates of differing order
When the assumptions of function-free and constant variable order are not met, we can do the following tricks.
Functions are turned to predicates by their relation interpretation.
If there are predicates of different arity in the theory, we can trivially extend them to $\variableorder$ary predicates by tensor products with the trivial tensor $\ones$.
This can be done by a tensor product with $\onehotmapofat{\inddim}{\indvariable}$, where we add an auxiliary object $\indexinterpretationof{\inddim}$ as a placeholder for predicates with smaller arity.

% Finite worlds -> By database semantics?
While in first order logics, depending on the chosen semantics, worlds can have infinite sets of objects, we here only treat worlds with finite objects.

\subsect{Case of Propositional Logics}

%
Before continuing with the one-hot encoding of first-order logic worlds, let us show that the previously discussed formalism of propositional logics (see \charef{cha:logicalRepresentation}) is a special case of first-order logics, namely when demanding $\indorder=0$.
Consistent with \defref{def:folWorld} we have a propositional logic world by
\begin{align*}
    \dataworld: [\catorder] \rightarrow [2] \, ,
\end{align*}
which we have in \charef{cha:logicalRepresentation} represented by the assignments $\catindexof{\atomenumerator} = \dataworldat{\selvariable=\atomenumerator}$ to the categorical variables $\catvariableof{\atomenumerator}$.

% Comparison with PL
%Compared with propositional formulas, the grounding tensor does not take as input a specific world, but is defined on a given world.
%We show in this chapter, how both tensor interpretations can be transformed, i.e. by extracting samples from a FOL world $\dataworld$ interpretated as an empirical distribution over PL worlds, and by generating FOL worlds by a set of samples generated from a PL distribution.

% One-hot maps
To represent logical formulas as sets of possible worlds, and distributions of worlds, we applied in \parref{par:one} one-hot encodings of possible worlds.
For the case of propositional logics, this is
\begin{align*}
    \onehotmapofat{\dataworld}{\shortcatvariables} = \bigotimes_{\catenumeratorin} \onehotmapofat{\dataworldat{\selvariable=\atomenumerator}}{\catvariableof{\catenumerator}} \, .
\end{align*}

\subsect{One-hot encoding of worlds}

Let us now generalize the one-hot encodings of propositional logic worlds to worlds of first-order logic.
To encode the boolean tensors $\dataworld$ describing first order logics as basis elements of a tensor space, we take the one-hot encoding
\begin{align*}
    \onehotmap :
    \bigtimes_{\atomenumeratorin}\bigtimes_{\indindexofin{0}}\cdots\bigtimes_{\indindexofin{\indorder-1}} [2]
    \rightarrow \bigotimes_{\catenumeratorin}\bigotimes_{\indindexofin{0}}\cdots\bigotimes_{\indindexofin{\indorder-1}} \rr^2
\end{align*}
defined by
\begin{align*}
    \onehotmapofat{\dataworld}{\catvariableof{[\catorder]\times[\inddim]^{\indorder}}}
    = \bigotimes_{\catenumeratorin}\bigotimes_{\indindexofin{0}}\cdots\bigotimes_{\indindexofin{\indorder-1}}
    \onehotmapofat{\dataworldat{\selvariable=\atomenumerator,\indexedshortindvariables}}{\catvariableof{\catenumerator,\shortindindices}} \, .
\end{align*}
This is a tensor of order $\catorder\cdot\inddim^{\indorder}$, in a tensor space of dimension $2^{\left(\catorder\cdot\inddim^{\indorder}\right)}$.
Storage of such tensors in naive formats would not be possible.
However, the basis CP format discussed in \charef{cha:sparseCalculus} still provides storage with demand linear in the order $\catorder\cdot\inddim^{\indorder}$.

% Domain
Another issue when comparing different first-order logic worlds arises in potentially different world domains.
As we have explored, the cardinality of the domain influences the order of the one-hot encoding tensors.
To avoid such issues we here enumerate worlds coinciding in their domains.
This restriction is called database semantics (see e.g. Section 8.2.8 in \cite{russell_artificial_2021}), where only those worlds are considered, which domains have a one-to-one map to the constant symbols appearing in a respective knowledge base. % Unique name assumption + Domain closure!
% Factored systems
When restricting to worlds coinciding in their domain, we still have a factored representation of the system, since we can enumerate the possible worlds by a cartesian product.
However, the number of categorical variables representing the world is $\atomorder\cdot \inddim^{\indorder}$ and tensor representations, even in sparse formats, are not feasible due to the large order required.
These techniques to restrict to comparable factored representations are often refered to propositionalization of a first-order logic knowledge base.

% Propositional 
%Propositional worlds have been enumerated by indices of $\atomorder$ Booleans, that is for a world $\dataworld: [\folpredicateorder] \rightarrow [2]$ we take the index
%	\[ \atomindices \quad \text{where} \quad \atomlegindexof{\atomenumerator} = \dataworld(\atomenumerator) \, .  \]

% FOL
%When we want to enumerate the first order logic worlds to a fixed set of objects $\worlddomain$, we flatten the tensor $\dataworld$ and get indices
%	\[ \{\atomlegindexof{\atomenumerator,\indindexlist} \, : \, \atomenumeratorin, \indindexlist \in[\inddim] \} \quad
%	\text{where} \quad \atomlegindexof{\atomenumerator,\indindexlist} = \dataworld(\atomenumerator,\indindexlist) \, .  \]





\subsect{Probability distributions}

Having established the formalism of one-hot encodings also in the case of first-order logic worlds, we can now proceed with the definition of distributions and formulas, analogously to the development in \parref{par:one}.
Probability distributions over worlds coinciding on their domain are then non-negative and normed tensors
\begin{align*}
    \probat{\catvariableof{[\catorder]\times[\inddim]^{\indorder}}} \in \bigotimes_{\atomenumeratorin,\shortindindices\in[\inddim]^{\indorder}} \rr^2 \, .
\end{align*}
where each coordinate of a world $\dataworld$ is captured by a boolean random variable $\catvariableof{\atomenumerator,\shortindindices}$, indicating whether the $\atomenumerator$-th predicate holds on the object tuple indexed by $\shortindindices$.

% High-dimensional - watch out for repetitions!
We notice, that by definition these probability distributions are distributions of $\atomorder\cdot\inddim^{\indorder}$ Booleans with $2^{\left(\atomorder\cdot\inddim^{\indorder}\right)}$ many states.
% One-hot encodings minimal
Unfortunately, it is not possible to design encoding spaces of smaller dimension, when our aim is to get any distribution over possible worlds by an element in the encoding space.
This is due to the fact, that one-hot encodings provide a basis in the tensor space, as will be shown in \charef{cha:coordinateCalculus}.
The reason for the large encoding space dimension is therefore rooted in the equal number of possible worlds and not in an overhead in the dimension of the one-hot encoding space.
We will later in this chapter investigate methods to handle such high-dimensional distributions in the formalism of exponential families.

\subsect{Semantics of formulas}

Following the development of \charef{cha:logicalRepresentation}, we can choose a semantic approach to the definition of formulas, under the assumption of database semantics.
Since the semantic of a logical formula is the set of its models, we again have a one-to-one correspondence between logical formulas and the boolean tensors in the one-hot encoding space
\begin{align*}
    \bigotimes_{\atomenumeratorin,\shortindindices\in[\inddim]^{\indorder}} \rr^2 \, .
\end{align*}
This correspondence between the semantics and boolean tensor is through a subset encoding (see \defref{def:subsetEncoding}) of the respective formulas.
However, due to the large state dimensions, we will in the following sections choose a syntactical approach to the construction of formulas, which will naturally provide efficient tensor network decompositions.

\subsect{Two levels of tensor representation}

In comparison with propositional logics, first-order logic bears two levels of natural tensor representations.
In the first level, which we call the structured level, each world (see \defref{def:folWorld}) has a natural structure by a tensor, since it encodes relations between objects chosen by assignments to term variables.
This is different to the worlds of a propositional logic theory, which are represented by a boolean vector instead of a tensor.
The second level arises as in propositional logics, by understanding each world as a uncertain state and studying distributions over states, which are understood themself as a tensor (see \defref{def:probabilityDistribution}).
We call this the factored level, since it arises in general in the discussion of factored representations.
As argued above, the assumption of database semantics is central to exploit the tensor structure of the substitution level.
Under this assumption, representation of an uncertain state, or a collection of possible states, is done in the tensor space
\begin{align*}
    \bigotimes_{\atomenumeratorin,\shortindindices\in[\inddim]^{\indorder}} \rr^2 \,
\end{align*}
where the enumeration of the $2$-dimensional axes contains the tensor structure of the substitution level.



\sect{Formulas in a fixed first-order logic world}

Following the argumentation above, we in this section restrict to the exploitation of tensors in the structured level, namely a fixed world represented as a tensor $\dataworldwith$, see \defref{def:folWorld}.
We are specifically interested in the tensor network decomposition of first order formulas, which contain in full generality variables and therefore also have a tensor.
The evaluation of a first-order formula on a specific world is therefore different to the case in propositional logics, where the evaluation was a boolean in $\ozset$ indicating whether the world is a model.
%\red{Here we investigate grounding tensors to formulas with variables, and calculate them in a fixed world.}
% Arbitrary formulas

\subsect{Grounding tensors}

Given a first-order logic world $\dataworldwith$, arbitrary formulas are interpreted in terms of the satisfactions of their groundings.
We define their semantic first, and then relate their syntactical decomposition to tensor networks, similar to our approach to propositional logics in \charef{cha:logicalRepresentation}.

\begin{definition}[Grounding of a first-order formula given a world]
    Given a specific world $\dataworld$, with an domain $\worlddomain$ enumerated by $[\inddim]$, the grounding of a formula $\folexformula$ with variables $\indvariableof{\folexformula}$  is the tensor
    \begin{align*}
        \groundingofat{\folexformula}{\indvariableof{\folexformula}} :
        \bigtimes_{\indenumerator\in[\indvariableof{\folexformula}]} [\inddim] \rightarrow \ozset \, .
    \end{align*}
    Each coordinate represents thereby the boolean, whether the substitution of the variables in the formula is satisfied given a world $\dataworld$, that is
    \begin{align*}
        \groundingofat{\folexformula}{\indexedindvariableof{\folexformula}} = 1
    \end{align*}
    if and only if the substitution of $\folexformula$ with the variables $\indvariableof{\folexformula}$ replaced by the objects $\indexinterpretationat{\indindexof{\indenumerator}}$ is satisfied on the world $\dataworld$.
\end{definition}

% Comment: Formulas as maps to
The grounding tensor formalism can be used to define formulas as a map
\begin{align*}
    \folexformula : \left(\bigotimes_{\atomenumeratorin,\shortindindices\in[\inddim]^{\indorder}}\rr^{2}\right)
    \rightarrow \left(\bigotimes_{\atomenumeratorin,\indindexof{\folexformula}\in[\inddim]^{\cardof{\indvariableof{\folexformula}}}}\rr^{2}\right)
\end{align*}
where each world $\dataworld$ is mapped to a grounding tensor
\begin{align*}
    \folexformula(\dataworld) = \groundingof{\folexformula} \, .
\end{align*}
This would involve the factored level of tensor interpretation, namely representation of all possible worlds.

%% Basis encoding
%When interpreting this map as a basis encoding, formulas are tensors in the tensor space
%\begin{align*}
% 	\left(\bigotimes_{\atomenumeratorin, \indindexlist\in[\inddim]} \rr^{2} \right) \otimes
%	\left(  \bigotimes_{\atomenumeratorin, \indindexof{0},\ldots,\indindexof{\individualorder_{\folexformula}}\in[\inddim]} \rr^{2} \right) \, .
%\end{align*}

\subsect{Atomic Formulas}

Atomic formulas in first-order logic are predicates, which are applied on terms.%, that is constants or variables.
We restrict in this chapter to function-free logic, therefore terms are either constants or variables.
%The predicates itself are the simplest cases of first-order formulas with term variables.
% Atomic
If all arguments of a predicate are assigned by free variables, the corresponding grounding tensor is stored in the slices to the first axis of $\dataworld$ and we have
\begin{align}
    \groundingof{\folpredicateof{\folpredicateenumerator}} =
    \contractionof{\dataworldat{\selvariable,\shortindvariablelist},\onehotmapofat{\folpredicateenumerator}{\selvariable}}{\shortindvariablelist} \, .
\end{align}
In contrast, when a constant object $\indexinterpretationof{\indindex}$ is assigned to an argument of a predicate, the grounding tensor reduced to a slice of the grounding with exclusively free variables.
We capture such slicings by contractions with one-hot encodings of the corresponding constant.

We formalize this approach by atom creating tensors, which contraction with the world tensor results in the grounding of the corresponding atomic formula.

\begin{definition}\label{def:atomCreatingTensor}
    Let there be an atomic formula $\folexformula$, which is constructed using the $\selindex$-th predicate and has constants assigned on the arguments $\arbsetof{C}\subset[\indorder]$ and free variables to the arguments $\arbsetof{V}=[\indorder]/\arbsetof{C}$.
    Let the constant map $C: \arbsetof{C}\subset[\indorder] \rightarrow [\inddim]$ map to the specific objects represented by the constant and $V: \arbsetof{V}\subset[\indorder] \rightarrow \nodes$ to free variables labeled by a set $\nodes$.
    Then the atom creating tensor to $\folexformula$ is
    \begin{align*}
        \atomcreatorofat{\folexformula}{\indvariableof{V(\arbsetof{V})}}
        = \onehotmapofat{\selindex}{\selvariable} \otimes
        \left( \bigotimes_{\indenumerator\in\arbsetof{C}} \onehotmapofat{C(\indenumerator)}{\indvariableof{\indenumerator}} \right) \otimes
        \left( \bigotimes_{\indenumerator\in\arbsetof{V}} \identityat{\indvariableof{V(\indenumerator)},\indvariableof{\indenumerator}} \right) \, .
    \end{align*}
\end{definition}

The ground of the atom is then the contraction of the atom creating tensor with the world tensor, that is
\begin{align*}
    \groundingofat{\folexformula}{\indvariableof{V(\arbsetof{V})}}
    = \contractionof{\dataworldwith, \atomcreatorofat{\folexformula}{\indvariableof{\nodes}}}{\indvariableof{V(\arbsetof{V})}} \, .
\end{align*}


% Predicates as objects
What is more abstract, we can understand the predicate itself as an object, then take the first-order world as a grounding tensor of a more abstract formula.
We will follow this thought in the ternary representation of Knowledge Graphs in \secref{subsec:knowledgeGraphTernaryRep}.

%\subsect{Substitution by slicing}

% Slicing interpretation
%Slicing the grounding tensor of a formula a first-order formula amounts to substitution of the respective variable by the constant at the enumeration index.

%\subsect{Syntactical Decomposition of quantifier-free formulas}

\subsect{Formula synthesis by connectives}\label{sec:folConnectiveRepresentation}

In order to have a sound semantic, the grounding of FOL formulas is determined by the syntax of the formula, i.e. a decomposition of the formula into connectives and quantifiers acting on atomic formulas.

% Formulas as maps from worlds to groundings
Quantifier-free formulas are connectives acting on atomic formulas.
We can describe them as in the case of propositional logics in the $\rencodingof{}$-formalism.
While the atomic formulas where delta tensors copying states, they are more involved here.



\begin{theorem}
    For any connective $\exconnective$ and formulas $\folexformula_1$ and $\folexformula_2$ we have
    \begin{align}
        &\groundingofat{(\folexformula_1\exconnective\folexformula_2)}{\indvariableof{\folexformula_1}\cup\indvariableof{\folexformula_2}} \\
        &\quad=
        \contractionof{
            \rencodingofat{\groundingof{\folexformula_1}}{\headvariableof{\folexformula_1},\indvariableof{\folexformula_1}},
            \rencodingofat{\groundingof{\folexformula_2}}{\headvariableof{\folexformula_2},\indvariableof{\folexformula_2}},
            \rencodingofat{\exconnective}{\headvariableof{\folexformula_1\exconnective\folexformula_2}, \headvariableof{\folexformula_1}, \headvariableof{\folexformula_2}},
            \tbasisat{\headvariableof{\folexformula_1\exconnective\folexformula_2}}
        }
        {\shortindvariablelist} \, .
    \end{align}
\end{theorem}
\begin{proof}
    This directly follows from \theref{the:compositionByContraction}.
%	By the semantic interpretation of the groundings, which has to be sound.
\end{proof}

% Shared variables
Here, variables can be shared by the connected formulas, therefore the variables in the combined formula are unions of the possible not disjoint variables of the connected formulas.

%% Propositional interpretation
%When we understand the head variables in the basis encoding of atoms as the categorical variables, and get a similar interpretation of the tensor network decomposition as in the propositional case.
%\subsect{Propositionalization}

When interpreting the head variables of relational encoded atomic formulas as the atoms of a propositional theory, we find a propositional formula $\exformula$ associated with any decomposable first order logic formula.

\begin{definition}
    \label{def:propositionalEquivalent}
    Given a formula $\folexformula$ in first order logic, we say that a propositional formula $\formulaat{\shortcatvariables}$ is the propositional equivalent to $\folexformula$ given atomic formulas $\extformulaof{\atomenumerator}$ in first order logic, when for any world $\dataworld$ we have
    \begin{align*}
        \groundingofat{\folexformula}{\indvariableof{\folexformula}}
        = \contractionof{
            \{\rencodingofat{\groundingof{\extformulaof{\atomenumerator}}}{\catvariableof{\atomenumerator},\indvariableof{\extformulaof{\atomenumerator}}} : \atomenumeratorin\}
            \cup \{\formulaat{\shortcatvariables}\}
        }{\indvariableof{\folexformula}} \, .
    \end{align*}
    We here denote the head variables of the basis encoding to $\rencodingof{\groundingof{\extformulaof{\atomenumerator}}}$ by $\catvariableof{\atomenumerator}$ to highlight their interpretation as propositional atoms.
\end{definition}

We depict the relation of a grounding tensor to a propositional formula as:
\begin{center}
    \input{./PartII/tikz_pics/fol_models/propositionalization.tex}
\end{center}


\subsect{Quantifiers}

Existential and universal quantifiers appear in generic first order logic and are besides substitutions further means to reduce the number of variables in a formula.
%They are not representable as linear transform of the respective quantifier-free formula.


% Definition of existential and universal quantifiction needed!
The semantics of existential quantification consists in a formula being true, if at least one state of the quantified variable is true, as we define next.

\begin{definition}
    Given a grounding tensor
    \begin{align*}
        \groundingofat{\folexformula}{\indvariableof{0},\ldots,\indvariableof{\indorder-1}} \,
    \end{align*}
    the existential and universal quantification with respect to the first variable are the tensors
    \begin{align*}
        \groundingofat{\left(\exists_{\indindexof{0}}\folexformula\right)}{\indvariableof{1},\ldots,\indvariableof{\indorder-1}} \quad \text{and} \quad
        \groundingofat{\left(\forall_{\indindexof{0}}\folexformula\right)}{\indvariableof{1},\ldots,\indvariableof{\indorder-1}} \,
    \end{align*}
    with coordinates as follows.
    For an assignment $\indindexof{1},\ldots,\indindex$ to the non-quantified variables we have
    \begin{align*}
        \groundingofat{\left(\exists_{\indindexof{0}}\folexformula\right)}{\indexedindvariableof{1},\ldots,\indexedindvariableof{\indorder-1}} = 1
    \end{align*}
    if and only if there is an assignment $\indindexofin{0}$ such that
    \begin{align*}
        \groundingofat{\folexformula}{\indexedindvariableof{0},\indexedindvariableof{1},\ldots,\indexedindvariableof{\indorder-1}} = 1 \, .
    \end{align*}
    Conversely, we have for the universal quantification that
    \begin{align*}
        \groundingofat{\left(\forall_{\indindexof{0}}\folexformula\right)}{\indexedindvariableof{1},\ldots,\indexedindvariableof{\indorder-1}} = 1
    \end{align*}
    if and only if for any assignment $\indindexofin{0}$ we have
    \begin{align*}
        \groundingofat{\folexformula}{\indexedindvariableof{0},\indexedindvariableof{1},\ldots,\indexedindvariableof{\indorder-1}} = 1 \, .
    \end{align*}
\end{definition}


Let us now show, that existential and universal quantification are coordinatewise transforms (see \defref{def:coordinatewiseTransform}) of contracted grounding tensors.
To this end, let us introduce the greater-$z$ indicator $\greaterthanfunction{z}$, where $z\in\rr$, as the function
\begin{align*}
    \greaterthanfunction : \rr \rightarrow \ozset
    \quad, \quad \greaterthanfunctionof{z}{x} =
    \begin{cases}
        1 & \quad  \text{if} \quad x > z\\
        0 & else
    \end{cases} \, .
\end{align*}

\begin{theorem}
    For any formula $\folexformula$ with variables $\shortindvariablelist$ we have
    \begin{align*}
        \groundingofat{\left(\exists{\indindexof{0}}\folexformula\right)}{\indvariableof{1},\ldots,\indvariableof{\indorder-1}} =
        \coordinatetrafowrtofat{\existquanttrafo}{\contractionof{\groundingof{\folexformula}}{\indvariableof{1},\ldots,\indvariableof{\indorder-1}}}{\indvariableof{1},\ldots,\indvariableof{\indorder-1}}
    \end{align*}
    and
    \begin{align*}
        \groundingofat{\left(\forall{{\indindexof{0}}} \folexformula\right)}{\indvariableof{1},\ldots,\indvariableof{\indorder-1}}=
        \coordinatetrafowrtofat{\universalquanttrafo}{\contractionof{\groundingof{\folexformula}}{\indvariableof{1},\ldots,\indvariableof{\indorder-1}}}{\indvariableof{1},\ldots,\indvariableof{\indorder-1}}
    \end{align*}
\end{theorem}
\begin{proof}
    We proof the claimed equalities to arbitrary slices of the remaining variables, which amount to arbitrary substitutions of the formulas.
    For any indices $\indindexofin{1},\ldots,\indindexofin{\indorder-1}$ we notice, that
    \begin{align*}
        \sbcontractionof{\groundingof{\folexformula}}{\indexedindvariableof{1},\ldots,\indexedindvariableof{\indorder-1}}
        &= \sum_{\indindexofin{0}} \groundingofat{\folexformula}{\indexedindvariableof{0},\ldots,\indexedindvariableof{\indorder-1}} \\
        &= \cardof{\indindexofin{0} \, : \, \groundingofat{\folexformula}{\indexedindvariableof{0},\ldots,\indexedindvariableof{\indorder-1}}=1} \, .
    \end{align*}
    We can thus understand the contracted grounding tensor as storing in its coordinates the count of the coordinate extensions to the zeroth variable, such that the grounding tensor is satisfied.
    This is analogous to our interpretation of contracted propositional formulas as world counts.
    From this it is obvious, that the existential quantification is satisfied, if the count is different from zero, which is captured by the coordinatewise transform with $\existquanttrafo$.
    We therefore arrive at
    \begin{align*}
        \groundingofat{\left(\exists_{\indindexof{0}}\folexformula\right)}{\indexedindvariableof{1},\ldots,\indexedindvariableof{\indorder-1}} =
        \coordinatetrafowrtofat{\existquanttrafo}{\contractionof{\groundingof{\folexformula}}{\indvariableof{1},\ldots,\indvariableof{\indorder-1}}}{\indexedindvariableof{1},\ldots,\indexedindvariableof{\indorder-1}} \, .
    \end{align*}
    The first claim follows, since the assignment to the non-quantified variables was arbitrary.
    The universal quantification is satisfied, when all extensions are satisfied, and the count is $\inddim$.
    Since $\inddim$ is the maximal count, this is captured by the coordinatewise transform with $\universalquanttrafo$ and we get
    \begin{align*}
        \groundingofat{\left(\forall{\indindexof{0}}\folexformula\right)}{\indexedindvariableof{1},\ldots,\indexedindvariableof{\indorder-1}} =
        \coordinatetrafowrtofat{\universalquanttrafo}{\contractionof{\groundingof{\folexformula}}{\indvariableof{1},\ldots,\indvariableof{\indorder-1}}}{\indexedindvariableof{1},\ldots,\indexedindvariableof{\indorder-1}} \, .
    \end{align*}
    With the same argument, the second claim is established.
\end{proof}

% Customized quantifiers
We can extend this discussion towards more generic counting quantifiers, of which the existential and the universal quantifier are extreme cases.
One can define quantifiers by demanding that at least $z\in\nn$ compatible groundings are satisfied, and show that they amount to coordinatewise transforms with $\greaterthanfunction{z}$.
What is more, quantifiers demanding that at most $z\in\nn$ are satisfied would be representable by transforms with an analogously defined function $\ones_{\leq z}$.
Such customized quantifiers appear for example in the $\mathrm{OWL\,2}$ standard of description logics (see \cite{rudolph_foundations_2011} and \secref{sec:kgRepresentation}).

% basis encodings
As will be discussed in \charef{cha:basisCalculus}, any coordinatewise transform can be performed by a contraction of a basis encoding of the tensor with a head vector prepared by the transform function (see \theref{the:tensorFunctionComposition}).
In the case here, a direct implementation would require a dimension of these head variables by $\inddim$, which can be infeasible when having large object sets.

% Prenex
To summarize, let us assume a formula is in its prenex normal form, that is a collection of quantifiers are acting on a qantifier free part.
We can represent its grounding tensor by
\begin{itemize}
    \item Instantiations of the tom groundings with the assigned variables, as contractions of the basis encoding of the world tensor with atom selecting tensors.
    \item Propositional formula acting on the head variables of the predicate instantiations, representing the connectives combining the formula.
    \item Quantifiers as a composition of contractions closing the quantified variable and coordinatewise transforms with the respective greater-than indicators.
\end{itemize}



\subsect{Storage in basis CP decomposition}\label{sec:basisCPgrounding}

In many situations, grounding cores are sparse and representations as single tensor cores comes with a drastic overhead.
We often encounter sparse grounding tensors, where the number of non-zero coordinates (to be investigated by basis CP ranks in \charef{cha:sparseCalculus}) satisfies
\begin{align*}
    \sparsityof{\groundingof{\folexformula}} << \inddim^{\cardof{\indvariableof{\folexformula}}} \, .
\end{align*}
In this case, since most coordinates vanish, the basis CP decomposition (see \secref{sec:basisCP}) enables a representation of the grounding with significantly lower storage demand, see \theref{the:sparseBasisCP}.
This is particularly useful for representing large relational databases, where each object has only a few relations with others, while the majority of possible relations remains unsatisfied.
We depict such CP decomposition of a formula grounding in \theref{fig:groundingCP}.

% Standard KB Encoding and Assumptions
Most logical syntaxes exploit $\ell_0$-sparsity, explicitly storing only known assertions.
The interpretation of unspecified assertions depends on the underlying assumptions.
Under the Closed World Assumption, for example, all unspecified assertions are assumed to be false.

\begin{figure}[h]
    \begin{center}
        \input{./PartII/tikz_pics/fol_models/grounding_decomposition.tex}
    \end{center}
    \caption{Basis CP Decomposition of the grounding of $\folexformula$, following the scheme of \theref{the:sparseBasisCP}.
    Instead of direct storage of the grounding tensor $\groundingof{\folexformula}$, the non-zero coordinates are enumerated by a variable $\datvariable$ and the corresponding coordinates stored in leg-matrices $\legcoreof{\folexformula,\indenumerator}$.}
    \label{fig:groundingCP}
\end{figure}

\subsect{Queries}

A database is understood as a specific fist order logic world, and are operations on such a single world.
Queries are described by a formula $\impformula$, which are asked against a specific world $\dataworld$ to retrieve the grounding $\groundingof{\impformula}$.
The variables of such formulas are called projection variables.
The answer $\groundingof{\impformula}$ of a query is most conveniently represented as a list of solution mappings from the projection variables to objects in the world, such that the query formula is satisfied.
Answering a query by solution mappings corresponds with finding the basis CP Decomposition (see \secref{sec:basisCP}) of $\groundingof{\impformula}$.
We can understand these solution mappings as stored in the leg-matrices $\legcoreof{\folexformula,\indenumerator}$ (see \figref{fig:gorundingCP}).

Let us give with the outer join an example of a popular operation to define queries, which efficient execution and storage can be improved based on considerations in the tensor network formalism.

\begin{definition}[Outer join]
    Let there be a world $\dataworld$ and formulas $\extformulaof{\selindex}$ depending on variables $\indvariableof{\nodesof{\selindex}}$, which have grounding tensors by
    \begin{align*}
        \groundingofat{\extformulaof{\selindex}}{\indvariableof{\node}} \, : \,  \bigtimes_{\node\in\nodesof{\selindex}}[\inddimof{\node}] \rightarrow \ozset \, .
    \end{align*}
    Then their (outer) $\joinsymbol$ is defined as the grounding of their conjunctions, as
    \begin{align*}
        \groundingofat{\joinsymbol\left(\extformulaof{0},\ldots,\extformulaof{\seldim-1}\right)}{\bigcup_{\selindexin}\indvariableof{\nodesof{\selindex}}}
        = \contractionof{\groundingofat{\extformulaof{\selindex}}{\indvariableof{\nodesof{\selindex}}}\,:\,\selindexin}{\bigcup_{l\in[p]}\indvariableof{\nodesof{\selindex}}} \, .
    \end{align*}
\end{definition}

%Visualization and efficiency
We can understand the $\joinsymbol$ of groundings by a factor graph, where each grounding tensor decorates the hyperedge to the node set $\nodesof{\selindex}$.
The projection variable assignment to each formula combined in a $\joinsymbol$ operation provide a basic tensor network format to store the output of the operation.
There are thus situations, in which the solution map storage corresponding with a CP Decomposition comes with unnecessary overheads compared with other formats.

% Coordinatewise transform
We can also understand the $\joinsymbol$ operation as a coordinatewise transform (see \defref{def:coordinatewiseTransform}) with the product as transform function.
To make this connection solid, one would need to extend each joined formula trivially to the variables appearing in other formulas.

% Evaluation similar constraint propagation
The efficiency of evaluating the contraction to a $\joinsymbol$ operation might be improved by understanding it as an Constraint Satisfaction Problem (see \charef{cha:logicalReasoning}).
When applying efficient Message Passing algorithms such as Knowledge Propagation (see \algoref{alg:knowledgePropagation}), the groundings can be sparsified by local constraint propagation operations before turning to more global and more demanding contraction operations.
Here the groundings $\groundingof{\extformulaof{\selindex}}$ would be used to initialize Knowledge Cores $\kcoreof{\edge}$ and sequentially sparsified during the algorithm.

%\begin{example} % WOULD NEED OVERWORK: DRAW!
%	For example take a query with many basic graph patterns with pairwise different projection variables.
%	The global CP Decomposition would come here with an exponential storage overhead compared with storage as a tensor product of CP Decompositions to each Basic graph pattern.
%\end{example}

%% CONFUSING?
%\begin{remark}[Distinguishing from probabilistic queries]
%	Let us distinguish the discussion here from those of queries in probabilistic reasoning, which have two main differences.
%	First, we ask queries against all possible pairs of variables, instead of asking the probability of satisfaction of a specific formula.
%	Second, since we made the epistemologic assumption of knowing possibilities and not probabilities in logics, a query is answered by a truth value.
%	We then only output in the shape of solution mappings the variable assignments where the query formula is true.
% 	Thus, the queries here can be thought of as a batch of probabilistic queries with Boolean answers.
%	% Alternative -> Later?
%	Probabilistic queries can furthermore be understood in terms of the data extraction process described in this section.
%	We can ask the query in probabilistic form (decomposed into atomic formulas) on the resulting empirical distribution.
%	This results in the ratio of the worlds satisfying the query among those worlds satisfying the extraction query $\impformula$.
%\end{remark}


\sect{Representation of Knowledge Graphs}\label{sec:kgRepresentation}

Let us now represent a specific fragment of first-order logic, namely Description Logics which Knowledge Bases are often refered to as Knowledge Graphs.
We here use the $\mathrm{OWL\,2}$ standard, which encodes the syntax of the description logic $\mathcal{SROIQ(D)}$ \cite{rudolph_foundations_2011}.

\subsect{Representation as unary and binary predicates}

% Reduction to binary
Predicates in knowledge graphs are binary (owl:ObjectProperties) and unary (owl:Class).
%Larger formulas are created by logical connections of these atomic formulas using disjunctions, conjunctions etc.
We enumerate the predicates by $[\folpredicateorder]$, the objects in the domain $\worlddomain$ by $[\inddim]$, and extend the unary predicates to binaries by tensor product with $\onehotmapofat{0}{\indvariableof{1}}$.
A Knowledge Graph on the set $\worlddomain$ of constants (owl:NamedIndividuals) is then the tensor
\begin{align*}
    \kgat{\selvariable,\indvariableof{0},\indvariableof{1}} : [\folpredicateorder] \times [\inddim] \times [\inddim] \rightarrow \ozset \, .
\end{align*}


\subsect{Representation as ternary predicate}\label{subsec:knowledgeGraphTernaryRep}

It has been particulary convenient to represent a Knowledge Graph instead as a grounding of a single ternary predicate $\rdf$.
To this end, the predicates $\folpredicateof{\catenumerator}$ and another object $\mathrdftype$ are added to a domain $\worlddomain$, by extending the $\inddim$ and the index interpretation function accordingly.


% RDF triple: Alternative viewpoint to collection of unary and binary predicates!
Following our notation we understand a Knowledge Graph as a grounding of the rdf triple relation $\rdf$ (being a formula of order 3) on a specific world $\kg$ with individuals $\worlddomain$

We then construct a grounding tensor $\kggroundingof{\rdf}$ out of the world $\kgat{\selvariable,\indvariableof{0},\indvariableof{1}}$ by
\begin{align*}
    \kggroundingof{\rdf} : [\inddim] \times [\inddim] \times [\inddim] \rightarrow \ozset
\end{align*}
where
\begin{align*}
    &\kggroundingofat{\rdf}{\indexedindvariableof{s}, \indexedindvariableof{p}, \indexedindvariableof{o}} \\
    &\quad =
    \begin{cases}
        \kgat{\selvariable=\indindexof{s},\indvariableof{0}=\indindexof{o},\indvariableof{1}=0}
        & \text{if} \quad \indindexof{p} = \invindexinterpretationat{\mathrdftype} \\
        \kgat{\selvariable=\indindexof{p},\indvariableof{0}=\indindexof{s},\indvariableof{1}=\indindexof{o}}
        & \text{if} \quad \indindexof{p} = \invindexinterpretationat{\folpredicateof{\catenumerator}} \quad \text{for some} \quad \catenumerator \\
        0  \quad & \text{else}
    \end{cases} \, .
\end{align*}


Slicing the tensor $\kggroundingof{\rdf}$ along the predicate axis retrieves specific information about roles and can be efficiently be performed on these formats.
The role $\mathrdftype$ has a specific meaning, since it contains from a DL perspective classifications (memberships of named concepts).
Further slicing the tensor along object axis therefore results in membership lists for specific classes (concepts).
One can thus regard $\mathrdftype$ as a placeholder for unitary formulas in a space of binary formulas.

% Triple Stores, sparsity
Exploiting the $\ell_0$-sparsity now leads to a so-called triple store, where $\kggroundingof{\rdf}$ is stored by a listing of those triples $\indindexof{\subsymbol},\indindexof{\predsymbol},\indindexof{\objsymbol}$ such that $\kggroundingofat{\rdf}{\indexedindvariableof{s}, \indexedindvariableof{p}, \indexedindvariableof{o}}=1$
A recent implementation of a triple store exploiting these intuitions is $\mathrm{TENTRIS}$, see \cite{pan_tentris_2020}.
In this work, such decompositions are generalized into more generic CP formats, see \charef{cha:sparseCalculus}.
% Approximation of KG Groundings
Approximations of grounding tensors by decompositions leads to embeddings of the individuals such as $\mathrm{Tucker}$, $\mathrm{ComplEx}$ and $\mathrm{RESCAL}$ (see \cite{nickel_review_2016}).

% Sparse representation
%Sparse representation of the grounding tensor to a knowledge graph is of central importance, as investigated in \cite{pan_tentris_2020}.
%We here do basis CP for sparse representation.


% basis encoding
For our purposes of evaluating logical formulas such as $\sparql$ queries we use the basis encoding of the groundings, which are depicted by
\begin{center}
    \begin{tikzpicture}[scale=0.3, thick] % , baseline = -3.5pt

    \draw[->] (0,1)--(0,3) node[midway,left] {\tiny $\headvariable$};
    \draw (-3,1) rectangle (3,-1);
    \node[anchor=center] (text) at (0,0) {\small $\rencodingof{\kggroundingof{\rdf}}$};
    \draw[<-] (-2,-1)--(-2,-3) node[midway,left] {\tiny $\sindvariable$};
    \draw[<-] (0,-1)--(0,-3) node[midway,left] {\tiny $\pindvariable$};
    \draw[<-] (2,-1)--(2,-3) node[midway,left] {\tiny $\oindvariable$};

\end{tikzpicture}
\end{center}




\subsect{$\sparql$ Queries}

The $\sparql$ query language is a syntax to express first-order logic formulas $\folexformula$ and intended to be evaluated given a Knowledge Graph.
We here consider tensor network representations of the $\mathrm{WHERE}{\cdot}$ block.
Given a specific knowledge graph $\kggroundingof{\rdf}$, the execution of query is the interpretation $\groundingof{\folexformula}$, typical represented in a sparse basis CP format where each slice represents a solution mapping.

\subsubsect{Triple Patterns}

\red{Central to $\sparql$ queries are triple patterns, which we understand as slicings of the tensor $\kggroundingof{\rdf}$.}
To each so-called triple pattern we build a corresponding atom creating tensor (see \defref{def:atomCreatingTensor}).
The triple pattern is then evaluated by contraction of the atom creating tensor with $\kggroundingof{\rdf}$.

Let us now provide examples of such pattern tensors.
A unary triple patterns contains a single projection variable, typically related with the subject variable $\sindvariable$ of $\kggroundingof{\rdf}$.
The corresponding pattern tensor is then
\begin{align*}
    \atomcreatorofat{\kgtriple{\provariable}{\mathrdftype}{\folpredicateof{\catenumerator}}}{
        \sindvariable, \pindvariable, \oindvariable, \provariable
    }
    = \identityat{\sindvariable,\provariable}
    \otimes \onehotmapofat{\invindexinterpretationat{\mathrdftype}}{\pindvariable}
    \otimes \onehotmapofat{\invindexinterpretationat{\folpredicateof{\atomenumerator}}}{\oindvariable} \, .
\end{align*}

Binary triple patterns come with two projection variables, typically related with the subject and the object variables $\sindvariable$ and $\oindvariable$.
The pattern tensor to the $\catenumerator$-th predicate is then
\begin{align*}
    \atomcreatorofat{\kgtriple{\provariableof{0}}{\folpredicateof{\catenumerator}}{\provariableof{1}}}{
        \sindvariable, \pindvariable, \oindvariable, \provariableof{0}, \provariableof{1}
    }
    = \identityat{\sindvariable,\provariableof{0}}
    \otimes \onehotmapofat{\invindexinterpretationat{\folpredicateof{\atomenumerator}}}{\pindvariable}
    \otimes \identityat{\oindvariable,\provariableof{1}} \, .
\end{align*}

Contraction with these pattern tensor evaluated the specific triple pattern, and outputs in a boolean tensor the indicator, which objects are members of a specific class (for unary patterns) or which pair of objects are related by a specific relation.
Again, the output of such contractions is a subset encodings of the set of solutions (see \defref{def:subsetEncoding}).

%%%%%%%%%%%% END OF FRIDAY 14.3.
%%%%%%%%%%%%

% Examples
Examples of triple patterns, drawn in \figref{fig:triplePatterns} are
\begin{itemize}
    \item Unary triple pattern with one variable, representing a formula with a single projection variable.
    For the example $\exunarytriple$ see Figure~\ref{fig:triplePatterns}a.
    \begin{align*}
        \atomcreatorofat{\kgtriple{\provariable}{\mathrdftype}{\folpredicateof{\catenumerator}}}{
            \sindvariable, \pindvariable, \oindvariable, \provariable
        }
        = \identityat{\sindvariable,\provariable}
        \otimes \onehotmapofat{\invindexinterpretationat{\mathrdftype}}{\pindvariable}
        \otimes \onehotmapofat{\invindexinterpretationat{\exaunaryrelation}}{\oindvariable}
    \end{align*}
    If and only if the output slice is $\tbasis$, then the corresponding object encoded by the input indices is of class $\exaunaryrelation$.
    \item Binary triple pattern with two variables, representing a formula with two projection variables.
    For the example  $\exbinarytriple$ see Figure~\ref{fig:triplePatterns}b.
    If and only if the output slice is $\tbasis$, then the corresponding object tuple encoded by the input indices has a relation $\exabinaryrelation$.
\end{itemize}

% Projection picture
The composition $\psi (\psi^T)$ of the matrification of the tensor $\psi$ is an orthogonal projection.
That means that applying $\psi (\psi^T)$ is the same map as applying once.


\begin{figure}[h]
    \begin{center}
        \begin{tikzpicture}[scale=0.3,thick] % , baseline = -3.5pt

    \begin{scope}
        [shift={(0,0)}]

        \node[anchor=center] (text) at (-12,2) {$a)$};

        \begin{scope}
            [shift={(-7,2)}]

            \draw (0,-3) rectangle (-6,-5);
            \draw[<-] (-3,-1)--(-3,-3) node[midway,right] {\tiny $\headvariable$};
            \node[anchor=center] (text) at (-3,-4) {$\rencodingof{\kggroundingof{\exunarytriple}}$};
            \draw[<-] (-3,-5)--(-3,-7) node[midway,left] {\tiny $\provariableof{0}$};

        \end{scope}

        \node[anchor=center] (text) at (-5.5,-2) {${=}$};

        \draw[->] (0,1)--(0,3) node[midway,left] {\tiny $\headvariable$};
        \draw (-4,1) rectangle (4,-1);
        \node[anchor=center] (text) at (0,0) {\small $\rencodingof{\kggroundingof{\rdf}}$};

        \draw (-2,-3) rectangle (-4,-5);
        \draw[<-] (-3,-1)--(-3,-3) node[midway,left] {\tiny $\sindvariable$};
        \node[anchor=center] (text) at (-3,-4) {$\delta$};
        \draw[<-] (-3,-5)--(-3,-7) node[midway,left] {\tiny $\provariableof{0}$};

        \draw (-1,-3) rectangle (1,-5);
        \draw[<-] (0,-1)--(0,-3) node[midway,left] {\tiny $\pindvariable$};
        \node[anchor=center] (text) at (0,-4) {$\onehotmapof{\invrdftypesymbol}$};

        \draw (2,-3) rectangle (4,-5);
        \draw[<-] (3,-1)--(3,-3) node[midway,left] {\tiny $\oindvariable$};
        \node[anchor=center] (text) at (3,-4) {$\onehotmapof{\exaunaryrelation}$};

    \end{scope}


    \begin{scope}
        [shift={(24,0)}]

        \node[anchor=center] (text) at (-13,2) {$b)$};

        \begin{scope}
            [shift={(-8,2)}]

            \draw (0.5,-3) rectangle (-6.5,-5);
            \draw[<-] (-3,-1)--(-3,-3) node[midway,right] {\tiny $\headvariable$};
            \node[anchor=center] (text) at (-3,-4) {$\rencodingof{\kggroundingof{\exbinarytriple}}$};

            \draw[<-] (-2,-5)--(-2,-7) node[midway,right] {\tiny $\provariableof{0}$};
            \draw[<-] (-4,-5)--(-4,-7) node[midway,left] {\tiny $\provariableof{1}$};

        \end{scope}

        \node[anchor=center] (text) at (-5.5,-2) {${=}$};

        \draw[->] (0,1)--(0,3) node[midway,left] {\tiny $\headvariable$};
        \draw (-4,1) rectangle (4,-1);
        \node[anchor=center] (text) at (0,0) {\small $\rencodingof{\kggroundingof{\rdf}}$};

        \draw (-2,-3) rectangle (-4,-5);
        \draw[<-] (-3,-1)--(-3,-3) node[midway,left] {\tiny $\sindvariable$};
        \node[anchor=center] (text) at (-3,-4) {$\delta$};
        \draw[<-] (-3,-5)--(-3,-7) node[midway,left] {\tiny $\provariableof{1}$};

        \draw (-1,-3) rectangle (1,-5);
        \draw[<-] (0,-1)--(0,-3) node[midway,left] {\tiny $\pindvariable$};
        \node[anchor=center] (text) at (0,-4) {$\onehotmapof{\exabinaryrelation}$};

        \draw (2,-3) rectangle (4,-5);
        \draw[<-] (3,-1)--(3,-3) node[midway,left] {\tiny $\oindvariable$};
        \node[anchor=center] (text) at (3,-4) {$\delta$};
        \draw[<-] (3,-5)--(3,-7) node[midway,right] {\tiny $\provariableof{0}$};

    \end{scope}

\end{tikzpicture}
    \end{center}
    \caption{Triple patterns of $\sparql$ as tensor networks.
    a) Example of unary triple pattern $\exunarytriple$ specifying whether an individual $\indexinterpretationof{\indindexof{1}}$ is a member of class $C$.
    %Here by $0$ we denote the element $\invindexinterpretationat{\mathrdftype}$
        b) Example of a binary triple pattern $\exbinarytriple$ specifying whether individuals $\indexinterpretationof{\indindexof{1}}$ and $\indexinterpretationof{\indindexof{2}}$ have a relation $R$.
        By $\onehotmapof{\invrdftypesymbol},\onehotmapof{\exaunaryrelation},\onehotmapof{\exabinaryrelation}$ we denote the one-hot encodings of the enumeration of the resources $rdf:type, C$ and $R$.
    }
    \label{fig:triplePatterns}
\end{figure}




\subsubsect{Basic Graph Patterns}

Generic $\sparql$ queries are compositions of triple patterns by logical connectives. % Except for some stuff like regex
These triple patterns possibly share projection variables.
Statements in $\sparql$ can be translated into Propositional Logics combining the triple patterns:
\begin{center}
    \begin{tabular}{|c|c|c|}
        \hline
        \textbf{$\sparql$}                & \textbf{Propositional Logics} & \textbf{Tensor Representation}                                                                   \\
        \hline
        $\{f_1, f_2\}$                    & $f_1\land f_2$                & $\rencodingofat{\land}{\headvariableof{f_1\land f_2},\headvariableof{f_1},\headvariableof{f_2}}$ \\
        \hline
        $\mathrm{UNION}\{f_1, f_2\} $     & $f_1\lor f_2$                 & $\rencodingofat{\lor}{\headvariableof{f_1\lor f_2},\headvariableof{f_1},\headvariableof{f_2}}$   \\
        \hline
        $\mathrm{FILTER}\,\,\mathrm{NOT}\,\,\mathrm{EXISTS}\{f\}$ & $\lnot f$                     & $\rencodingofat{\lnot}{\headvariableof{\lnot f},\headvariableof{f}}$                             \\
        \hline
    \end{tabular}
\end{center}

If a $\sparql$ query consists of these keywords, we find a straight forward corresponding network of triple patterns and encoded logical connectives, by applying our findings of \secref{sec:folConnectiveRepresentation}.
To this end, we prepare for each appearing triple pattern the corresponding pattern tensor, and a copy of $\kggroundingof{\rdf}$.
Here we also copy the term variables $\sindvariable,\pindvariable$ and $\oindvariable$, to ensure that each copy of $\kggroundingof{\rdf}$ shares variables with a single pattern tensor.
Projection variables are not copied, since we need to keep track of them shared among triple patterns.
Then we prepare the basis encoding of logical connectives according to the hierarchy specified in the $\sparql$ query.
Finally we add a $\tbasis$-vector to the final head variable representing the complete $\sparql$ query, to restrict the support to coordiantes corresponding with solution mappings.
We then contract the resulting tensor network, leaving all projection variables open.

If a projection variable is not appearing in the $\mathrm{SELECT}$ statement in front of the $\mathrm{WHERE}\{\cdot\}$-block, we simply exclude it from the open variables of the described contraction.
Note that in that case, the coordinates contain solution counts, i.e. how many assignments to the dropped variable have been a $1$ coordinate.
We can drop this additional information simply by performing a coordinatewise transform with the greater zero indicator $\existquanttrafo$.

% Effective calculus alternative
Here we represented a $\sparql$ query $\impformula$ consistent of multiple triple pattern by instantiating a head variables to each triple pattern.
Alternatively, the more direct hybrid calculus developed in \secref{sec:hybridCalculus} can be applied and the additional head variables avoided.
This is especially compelling, when the $\mathrm{WHERE}\{\cdot\}$-block does not contain further keywords, i.e. it is the conjunction of all triple patterns.
In that case, we avoid the instantiation of head variables (i.e. close the head variables separately by $\tbasis$-vectors) and represent the query by a contraction of all triple pattern tensors.

% Expressivity
We further notice, that any propositional formula acting on the head variables of the triple patterns can be expressed by a hierarchical combination of the key words in the above table.
To find the expression, one can transform a given formula into its conjunctive or disjunctive normal form and apply the statements according to the apperaing operations $\land,\lor$ and $\lnot$.


%% Further $\sparql$ features
%Further $\sparql$ features, which cannot be expressed by a tensor network are:
%\begin{itemize}
%    \item $\mathrm{FILTER}\{\cdot\}$ does not depend on triple patterns (e.g. numeric inequalities, regex functions on strings).
%    We can regard it as another basic formula, which does not result from a slicing of the $\rdf$ grounding tensor.
%    Besides that, we can understand it as formulas and include it in compositions.
%    \item $\mathrm{OPTIONAL}\{\cdot\}$ would result in $\ones$ leg vectors, when there is a missing variable assignment resulting.
%\end{itemize}



\sect{Probabilistic Relational Models}

% MLN in FOL and PL
So far we have studied Markov Logic Networks in Propositional Logics as probability distributions over worlds.
In FOL they define probability distributions over relations in worlds with a fixed set of objects.
More generally, such models are probabilistic relational models (see for an overview \cite{getoor_introduction_2019}.


We in this section treat random worlds in first-order logics with fixed domains $\worlddomain$.

%
We in this section show, when and how we can interpret likelihoods of Markov Logic Networks in First Order Logic in terms of samples of a Markov Logic Network in Propositional Logics.

\subsect{Hybrid First-Order Logic Networks}

% Templates
Following \cite{richardson_markov_2006} Markov Logic Networks in first-order logics are templates for distributions, which instantiate random worlds when choosing a set of objects $\worlddomain$.
Given a fixed set of constants, they then define a distribution over the worlds, which objects correspond with the constants. % this is database semantics!
This applies database semantics, where only those worlds are considered, where the unique name and domain closure assumptions given a set of constants are satisfied.
\red{Here we directly define them as exponential families distributing $\randworld$ for a given set of objects $\worlddomain$.}
\red{To avoid a similar discussion as in \charef{cha:networkRepresentation} we directly allow for boolean base measures and call the distributions Hybrid First-Order Logic Networks.}

\begin{definition}[Hybrid First-Order Logic Networks (HFLN)]
%    A Markov Logic Network is a template of probability distributions defined
    Let there be a set $\folformulaset$ of first-order logic formulas with maximal arity $\individualorder$, which is enumerated by a selection variable $\selvariable$ of dimension $\seldim$.
    Further, let there be a set of objects $\worlddomain$ and a boolean base measure $\basemeasureat{\shortindvariables}$.
    The family of Hybrid First-Order Logic Networks $\expfamilyof{\restfolformulaset,\basemeasure}$ defined by the tuple $(\folformulaset,\worlddomain,\basemeasure)$ is the exponential family of joint distributions to the variables $\randworld$ with the statistics
    \begin{align*}
        \sstat^{\restfolformulaset}_{\selindex}\left[\indexedrandworld\right]
        = \sbcontraction{\groundingof{\enumfolformula}}
    \end{align*}
    and the base measure $\basemeasure$.
%    The Markov Logic Network instantiated for a given set of objects $\worlddomain$ and a base measure $\basemeasure$ is the random world, which is a member of the exponential family with sufficient statistics
%    \begin{align*}
%        \sstatcoordinateofat{\selindex}{\indexedrandworld} = \sbcontraction{\groundingof{\enumfolformula}}
%        %\sstat_{\selindex}(\dataworld)  = \sbcontraction{\groundingof{\folexformula_\selindex}} % Formulas can have different
%    \end{align*}
%    and canonical parameters $\weight$.
\end{definition}

Each element of the family $\expfamilyof{\restfolformulaset,\basemeasure}$ is represented by a canonical parameter $\canparamat{\selvariable}$.

The mean parameter polytope is the convex hull of the vectors
\begin{align*}
    \sencodingofat{\folformulaset}{\indexedrandworld,\selvariable}
\end{align*}
to the worlds $\dataworld$ with $\basemeasureat{\indexedrandworld}=1$.
These vectors store are the counts of satisfied groundings to each formula, that is
\begin{align*}
    \sencodingofat{\folformulaset}{\indexedrandworld,\selvariable} = \cardof{
        \indindexof{\enumfolformula} \, : \, \groundingofat{\enumfolformula}{\indexedindvariableof{\enumfolformula}} = 1
    } \, .
\end{align*}
Each substitution of the variables in $\enumfolformula$ by objects in $\worlddomain$, which satisfies the formula in the world $\dataworld$, therefore provides a factor of $\expof{\canparamat{\indexedselvariable}}$ to the probability of $\dataworld$.

Let us notice, that different to the case of Hybrid Logic Networks treated in \charef{cha:networkRepresentation}, the statistic does not consist of boolean features, when formulas contain variables and we have multiple objects.
One could, however, replace each $\enumfolformula$ by the set of the possible groundings, i.e. substitutions of the formulas variables by any tuple of objects in $\worlddomain$.
The resulting distribution would be an Hybrid Logic Network with boolean statistic, which coincides with the HFLN when posing certain weight sharing conditions on $\canparam$.
The downside of this construction is the increase in the number of features from $\seldim$ to $\sum_{\selindexin} \cardof{\worlddomain}^{\cardof{\indvariableof{\enumfolformula}}}$.
This polynomial in the cardinality of the domain set increase poses significant computational challenges, see \cite{richardson_markov_2006}.
We will in the next sections explore an alternative way to apply the theory of \charef{cha:networkRepresentation} and \charef{cha:networkReasoning}, namely based on importance formulas.


%% Interpretation
%The statistics
%\begin{align*}
%    \sbcontraction{\groundingof{\folexformula_\selindex}} % Formulas can have different
%\end{align*}
%can be interpreted as the number of substitutions to a formula, such that the formula ist satisfied.
%Each substitution satisfying a formula adds a factor of $\expof{\canparam_\selindex}$ to the probability of the respective world before normalization.


%
%When constructing a world tensor to a theory with predicates of different order, we already argued that we extend the arity of predicates by tensor products with $\onehotmapof{0}$.
%To define random world tensors, we then restrict the corresponding base measure to be supported only on those worlds where the extended predicates hold only at the individual $\exindividualof{0}$ at the extended axis.


% Comparison with PL MLN
%We choose extraction formulas $\extformulaof{\atomenumerator}$ such that any formula in the FOL MLN has a propositional equivalent (see \defref{def:propositionalEquivalent}).
%The statistic map is then a formula selecting tensor as in the propositional logic case contracted with the groundings of $\extformulaof{\atomenumerator}$.






\subsect{Base measures by importance formulas}

%\red{Analogous to a guard formula in \cite[Definition 6.11]{koller_probabilistic_2009}!}

The boolean base measure $\basemeasure$ of a Hybrid First-Order Logic Network is the subset encoding of the possible worlds which have a non-vanishing probability with respect to any member of the family.
We now construct specific base measures based on a fixed grounding tensor of an importance formula.
This will reduce the number of object tuples influencing the probability distribution in order to arrive at an interpretation of FOL MLNs as likelihoods to datasets of propositional MLNs.

To this end, we mark pairs of term indices relevant to the distributions by an auxiliary index $\datindexin$.
Given a set $\{\indindexof{[\indorder]}^{\datindex} \, : \, \datindexin \}$ of indices to the important tuples we build a set encoding (see \defref{def:subsetEncoding})
\begin{align*}
    \fixedimpformula = \sum_{\datindexin} \left(
    \bigotimes_{\indenumeratorin} \onehotmapofat{\indindexof{\indenumerator}^{\datindex}}{\indvariableof{\indenumerator}}
    \right) \, .
\end{align*}

% Interpretation as grounding
We interpret the tensor $\fixedimpformula$ as the grounding of a formula, which we call the importance formula.

% Restricting to worlds with identical grounding
To have a constant importance formula we define a syntactic representation and restrict the support of the HFLN to those world coinciding with groundings of the importance formula coinciding with $\fixedimpformula$ by designing a base measure
\begin{align*}
    \fixedimpbm
    = \begin{cases}
          1 & \text{if} \quad \groundingofat{\impformula}{\indvariableof{\impformula}} = \fixedimpformula \\
          0 & \text{else}
    \end{cases} \, .
\end{align*}

% Conditioning on exquery
The base measure restricts the HFLN to be those worlds, where $\groundingof{\impformula}$ is coincides with the fixed tensor $\fixedimpformula$.
Intuitively, $\groundingof{\impformula}$ represents certain evidence about a first-order logic world, whereas other formulas are uncertain.


\begin{assumption}
    \label{ass:importanceBasemeasure}
    Given a base measure $\fixedimpbm$, we assume that there is an importance formula $\impformulaat{\shortindvariables}$ such that
    \begin{align*}
        \fixedimpbm
        = \begin{cases}
              1 & \text{if} \quad \groundingofat{\impformula}{\indvariableof{\impformula}} = \fixedimpformula \\
              0 & \text{else}
        \end{cases} \, .
    \end{align*}
\end{assumption}


\subsect{Decomposition of the log likelihood}


% Extraction query
To reduce the likelihood of a world to we make the assumption that all formulas in a HFLN are of the form
\begin{align}
    \label{eq:folImplicationForm}
%    \folexformula_{\selindex}(\individuals) =
    \enumfolformulaat{\indvariableof{\enumfolformula}}
    = \left( \impformulaat{\shortindvariables} \Rightarrow \headfolformulaofat{\selindex}{\indvariableof{\enumfolformula}} \right)
\end{align}
that is a rule with the importance formula being the premise.
In particular, we assume, that they depend on all term variables $\shortindvariables$.
If this is not the case, we extend the formula trivially on the missing term variables.
When this assumption holds, we can think of the importance formula as a conditions on individuals to satisfy a statistical relation given by $\headfolexformula$.

Towards connecting with propositional logics, we further make the assumption, that we can decompose the formula $\headfolformulaof{\selindex}$ in what we will call extraction formulas.

\begin{assumption}
    \label{ass:propositionalHeads}
    We assume that there exist formulas $\{\extformulaofat{\catenumerator}{\shortindvariables} \, : \, \catenumeratorin\}$, which we refer to as atom extraction formulas, and an importance formula $\impformulaat{\shortindvariables}$ such that the following holds.
    To each first-order logic formula $\enumfolformula$ there is another first-order logic formula $\headfolformulaofat{\selindex}{\indvariableof{\enumfolformula}}$ and a propositional formula $\enumformulaat{\shortcatvariables}$ such that
    \begin{align*}
        \enumfolformulaat{\indvariableof{\enumfolformula}}
        = \left( \impformulaat{\shortindvariables} \Rightarrow \headfolformulaofat{\selindex}{\indvariableof{\enumfolformula}} \right)
    \end{align*}
    and
    \begin{align*}
        \headfolformulaofat{\selindex}{\indvariableof{\enumfolformula}} =
        \contractionof{
            \{\enumformulaat{\shortcatvariables}\} \cup \{\rencodingofat{\extformulaof{\catenumerator}}{\catvariableof{\catenumerator},\shortindvariables} \, : \, \catenumeratorin\}
        }{\indvariableof{\enumformula}} \, .
    \end{align*}
\end{assumption}

We depict the assumption, that any formula is of the form \eqref{eq:folImplicationForm} in the diagram
\begin{center}
    \begin{tikzpicture}[scale=0.35, yscale=1, thick] % , baseline = -3.5pt




\draw (1,-1) rectangle (7,-3);
\node[anchor=center] (text) at (4,-2) {$\groundingof{\left(\impformula\Rightarrow\folexformula\right)}$};

\draw[] (2,-3) -- (2,-5) node[midway,left] {\tiny $\individualvariableof{0}$};
\node[anchor=center] (text) at (4,-4) {$\cdots$};
\draw[] (6,-3) -- (6,-5) node[midway,right] {\tiny $\individualvariableof{\individualorder-1}$};


\node[anchor=center] (text) at (10,-2) {${=}$};



%\draw (1,-1) rectangle (7,-3);
%\node[anchor=center] (text) at (4,-2) {$\rencodingof{\impformula}$};

\begin{scope}[shift={(12,-2)}]

\draw (1,3) rectangle (12,5);
\node[anchor=center] (text) at (6.5,4) {$\exformula$};

\draw[->-] (2.5,1) -- (2.5,3) node[midway,right] {\tiny $\atomicformulaof{0}$};
\draw (1,-1) rectangle (4,1);
\node[anchor=center] (text) at (2.5,0) {$\rencodingof{\groundingof{\extformulaof{0}}}$};
\node[anchor=center] (text) at (2.5,-2) {$\cdots$};

\node[anchor=center] (text) at (6.5,0) {$\cdots$};

\draw[->-] (10.5,1) -- (10.5,3) node[midway,right] {\tiny $\atomicformulaof{\atomorder-1}$};
\draw (8.75,-1) rectangle (12.25,1);
\node[anchor=center] (text) at (10.5,0) {$\rencodingof{\groundingof{\extformulaof{\atomorder\shortminus1}}}$};
\node[anchor=center] (text) at (10.5,-2) {$\cdots$};

\draw[<-] (13,-3) -- (3.5,-3) ;
\draw[<-] (13,-5) -- (1.5,-5) ;

\draw[fill] (11.5,-3) circle (0.15cm);
\draw[->-] (11.5,-3) -- (11.5,-1);

\draw[fill] (9.5,-5) circle (0.15cm);
\draw[->-] (9.5,-5) -- (9.5,-1);

\draw[fill] (3.5,-3) circle (0.15cm);
\draw[->-] (3.5,-3) -- (3.5,-1);

\draw[fill] (1.5,-5) circle (0.15cm);
\draw[->-] (1.5,-5) -- (1.5,-1);


\draw[fill] (7.5,-3) circle (0.15cm);
\draw[<-] (7.5,-3) -- (7.5,-7) node[right] {\tiny $\individualvariableof{\individualorder-1}$} ;

\node[anchor=center] (text) at (6.5,-6) {$\cdots$};

\draw[fill] (5.5,-5) circle (0.15cm);
\draw[<-] (5.5,-5) -- (5.5,-7) node[left] {\tiny $\individualvariableof{0}$} ;


\draw (13,-2) rectangle (15,-6);
\node[anchor=center] (text) at (14,-4) {$\rencodingof{\impformula}$};
\node[anchor=center] (text) at (12,-3.75) {$\vdots$};
\draw[->-] (15,-4) -- (16,-4);
\draw[] (17,-4) -- (16,-4);
\draw (17,-3) rectangle (19,-5);
\node[anchor=center] (text) at (18,-4) {$\tbasis$};

\end{scope}




\node[anchor=center] (text) at (30,-2) {${+}$};





\begin{scope}[shift={(22,1)}]

\draw (13,0) rectangle (15,2);
\node[anchor=center] (text) at (14,1) {$\fbasis$};

\draw[] (14,-1) --(14,0);
\draw[->-] (14,-2) --(14,-1);

\draw (12,-2) rectangle (16,-4);
\node[anchor=center] (text) at (14,-3) {$\rencodingof{\impformula}$};

\draw[<-] (12.5,-4) -- (12.5,-6) node[left] {\tiny $\individualvariableof{0}$} ;
\node[anchor=center] (text) at (14,-5) {$\cdots$};
\draw[<-] (15.5,-4) -- (15.5,-6) node[right] {\tiny $\individualvariableof{\individualorder\shortminus1}$} ;


		
\end{scope}

\end{tikzpicture}
\end{center}
where the second summand depends only on the query $\impformula$ and therefore does not appear in the likelihood.


%\begin{example}[Trivial importance formula]
%	When the importance formula is always satisfied, any tuple of objects contributes to the likelihood. 
%	This original approach to Markov Logic Networks \cite{richardson_markov_2006} however leads to many datapoints which are also dependent on each other.
%\end{example}


% Define probability
Let us now show, how to decompose the probability of a first-order logic world to a HFLN under the above assumptions.
Given a HFLN $\probof{\folmlnparameters}$, the probability of a world $\dataworld$ with $\groundingof{\impformula}=\fixedimpformula$ is % and $\groundingof{\folpredicateof{\folpredicateenumerator}} \prec \fixedimpformula$ as
\begin{align*}
    \probofat{\folmlnparameters}{\indexedrandworld}
    = \frac{1}{\partitionfunctionof{\folmlnparameters}}
    %\expof{\sum_{\folexformulain}\weightof{\folexformula}\contraction{\groundingof{(\impformula\Rightarrow\headfolexformula)}}}
    \expof{\sum_{\selindexin}\canparamat{\indexedselvariable}\contraction{\groundingof{(\impformula\Rightarrow\headfolexformula)}}}
\end{align*}
where the partition function is
\begin{align*}
    \partitionfunctionof{\folmlnparameters} =
    \sum_{\supportedworlds}
    \expof{\sum_{\selindexin}\canparamat{\indexedselvariable}\contraction{\groundingof{(\impformula\Rightarrow\headfolexformula)}}} \, .
    %\expof{\sum_{\folexformulain}  \weightof{\folexformula}  \sum_{\indindexlist\in[\inddim]} \groundingofat{(\impformula\Rightarrow\headfolexformula)}{\indexedshortindvariables} } }
    %\prod_{\individuals\in\worlddomain} \left(\prod_{\folexformulain} \expof{\weightof{\folexformula}(\impformula\Rightarrow\folexformula)(\individuals)} \right)\, .
\end{align*}


Let us now decompose the statistics into constant and varying terms.
We have
\begin{align*}
    \contraction{\groundingof{(\impformula\Rightarrow\headfolexformula)}} =
    \contraction{\groundingof{\impformula\land\headfolexformula}} + \contraction{\groundingof{\lnot\impformula}} \, ,
\end{align*}
where the the second term is constant among the supported worlds and the first can be enumerated by the satisfied substitutions of $\impformula$, that is
\begin{align*}
    \contraction{\groundingof{\impformula\land\headfolexformula}}
    = \sum_{\datindexin}\groundingofat{\headfolexformula}{\indvariableof{[\indorder]} = \indindexof{[\indorder]}^{\datindex}} \, .
\end{align*}


Using these insights we decompose a normalized log likelihood as
\begin{align}
    \label{eq:dataworldLogProb}
    \frac{1}{\datanum} \lnof{\probofat{\folmlnparameters}{\indexedrandworld}}
    = & \frac{1}{\datanum} \sum_{\datindexin} \sum_{\selindexin} \canparamat{\indexedselvariable}
    \groundingofat{\headfolexformula}{\shortindindices = \indindexof{[\indorder]}^{\datindex}} \\
    & - \frac{1}{\datanum} \lnof{
        \frac{\partitionfunctionof{\folmlnparameters}}{
            \expof{\contraction{\weight} \cdot \contraction{\groundingof{\lnot\impformula}}}
        }
    }
%	\sum_{\individuals\in\worlddomain \, : \, \impformula(\individuals)=1} \left(\sum_{\exformula\in\formulaset} \weightof{\exformula} \exformula(\individuals)\right) -  \frac{1}{\datanum}\lnof{\secpartitionfunctionof{\folformulaset,\weight,\worlddomain,\fixedimpformula}} \, . 
\end{align}

% Data identification
We notice a similarity with the likelihood in the case of MLNs in propositional logic.
When we interpret each tuple $\shortindindices\in(\worlddomain)^{\indorder}$ satisfying $\impformulaat{\indexedshortindvariables}=1$ as a datapoint, and choose the formulas
\begin{align*}
    \formulaset = \{\headfolformulaof{\selindex} \, : \, \selindexin \}
\end{align*}
from the propositional equivalents to the formulas $\folformulaset$, the fist term in \eqref{eq:dataworldLogProb} coincides with the first term of the likelihood
\begin{align*}
    \centropyof{\empdistribution}{\expdistof{(\formulaset,\canparam,\ones)}}
    = \dataaverage \sum_{\selindexin} \canparamat{\indexedselvariable} \enumformulaat{\datapoint} - \lnof{\partitionfunctionof{\mlnparameters}}
\end{align*}

However, the partition function couples multiple samples, with possible couplings, and prevents a straight forward interpretation as an empirical dataset.
We in the next section present assumptions on the tuples satisfying $\impformula$, which lead to a factorization of the partition function.

% Partition function simplification
%\subsect{Interpretation as Likelihood of Propositional Dataset}
\subsect{Decomposition of the Partition function}


We now make additional assumptions to decompose the partition function of an HFLN as a product of HLN partition functions.

\begin{assumption}
    \label{ass:independentTuples}
    Let $\fixedimpbm$ be a base measure of worlds such that the vectors
    \begin{align}
        \label{eq:data}
        \left(  \groundingofatwrt{\extformulaof{0}}{\indvariableof{0}=\indindexof{0}^\datindex,\ldots,\indvariableof{\indorder-1}=\indindexof{\indorder-1}^\datindex}{\tilde{\dataworld}}, \ldots,
        \groundingofatwrt{\extformulaof{\atomorder-1}}{\indvariableof{0}=\indindexof{0}^\datindex,\ldots,\indvariableof{\indorder-1}=\indindexof{\indorder-1}^\datindex}{\tilde{\dataworld}}
        \right)
    \end{align}
    for $\datindexin$ are independent and identical distributed by the normation of a boolean base measure $\atombasemeasure$, when drawing $\randworld$ by $\fixedimpbm$.
\end{assumption}

When these assumption holds, we now show that the probability of a first-order logic world coincides with the likelihood of a propositional logic dataset.

\begin{theorem}
    \label{the:FOLworldToPLdataset}
    Let there be a set of formulas $\folformulaset$ and a base measure $\fixedimpbm$ such that \assref{ass:importanceBasemeasure}, \assref{ass:propositionalHeads} and \assref{ass:independentTuples} hold.
    We then have for the likelihood of any by $\fixedimpbm$ supported world $\dataworld$ that
    \begin{align*}
        \frac{1}{\datanum} \lnof{\probofat{\folmlnparameters}{\indexedrandworld}}
        = \centropyof{\empdistribution}{\expdistof{\mlnparameters}}
    \end{align*}
    where $\formulaset$ is the set of propositional equivalents to $\folformulaset$ (see \assref{ass:propositionalHeads}) and $\datamap$ the data map with evaluation at $\datindexin$ by the enumerated non-vanishing coordinates of $\fixedimpformulawith$
    \begin{align*}
        \datapoint
        = \big( \groundingofat{\extformulaof{0}}{\indvariableof{0}=\indindexof{0}^\datindex,\ldots,\indvariableof{\indorder-1}=\indindexof{\indorder-1}^\datindex}, \ldots ,
        \groundingofat{\extformulaof{\atomorder-1}}{\indvariableof{0}=\indindexof{0}^\datindex,\ldots,\indvariableof{\indorder-1}=\indindexof{\indorder-1}^\datindex} \big) \, .
    \end{align*}
\end{theorem}

To show the theorem, we show first in the following lemma the factorization of the partition function of the HFLN.

\begin{lemma}
    \label{lem:FOLpartitionfunctionfactorization}
    Given the assumptions of \theref{the:FOLworldToPLdataset}, we have
    \begin{align*}
        \frac{\partitionfunctionof{\folmlnparameters}}{\expof{\contraction{\canparam} \cdot \contraction{\groundingof{\lnot\impformula}}}}
        = \left(\partitionfunctionof{\mlnparameters,\atombasemeasure}\right)^\datanum \, .
    \end{align*}
\end{lemma}
\begin{proof}
    We have
    \begin{align*}
        \partitionfunctionof{\folmlnparameters}
        &= \expectationofwrt{
            \expof{\sum_{\folexformulain}\weightof{\folexformula}\contraction{\groundingof{(\impformula\Rightarrow\headfolexformula)}}}
        }{\dataworld\sim\fixedimpbm} \\
        &= \expof{\contraction{\weight} \cdot \contraction{\groundingof{\lnot\impformula}}} \cdot
        \expectationofwrt{
            \expof{\sum_{\folexformulain}\weightof{\folexformula}  \sum_{\datindexin} \groundingofat{\headfolexformula}{\datshortindvariables} }
        }{\dataworld\sim\fixedimpbm} \\
        &= \expof{\contraction{\weight} \cdot \contraction{\groundingof{\lnot\impformula}}} \cdot
        \expectationofwrt{
            \prod_{\datindexin} \expof{ \sum_{\folexformulain} \weightof{\folexformula} \cdot \groundingofat{\headfolexformula}{\datshortindvariables} }
        }{\dataworld\sim\fixedimpbm}
    \end{align*}
    Since the substitutions of the atom formulas at the respective object tuples are independent, also the variables
    \begin{align*}
        \expof{\weightof{\folexformula}  \cdot \groundingofat{\headfolexformula}{\datshortindvariables}  }
    \end{align*}
    for $\datindexin$ are independent.
    We therefore get
    \begin{align}
        \label{eq:independentSamplesFOL}
        \partitionfunctionof{\folmlnparameters}
        &= \expof{\contraction{\weight} \cdot \contraction{\groundingof{\lnot\impformula}}} \cdot
        \prod_{\datindexin}
        \expectationofwrt{
            \expof{ \sum_{\folexformulain} \weightof{\folexformula} \cdot \groundingofat{\headfolexformula}{\datshortindvariables} }
        }{\dataworld\sim\fixedimpbm}
    \end{align}
    Each $\groundingofat{\headfolexformula}{\datshortindvariables}$ depends by \assref{ass:propositionalHeads} only on the random tuple $\{\extformulaof{\atomenumerator}[\datshortindvariables] \, : \, \catenumeratorin\}$.
    We build the expectation over all possible values $\shortcatindices$ of this tuple at any $\datindexin$ and get
    \begin{align*}
        & \expectationofwrt{
            \expof{\sum_{\selindexin} \canparamat{\indexedselvariable} \cdot \groundingofat{\headfolformulaof{\selindex}}{\datshortindvariables}}
        }{\dataworld\sim\fixedimpbm} \\
        & \quad = \sum_{\shortcatindices\in\atomstates}
        \probofwrt{\forall{\catenumeratorin} \, : \, \extformulaof{\atomenumerator}[\datshortindvariables] =\catindexof{\atomenumerator}}{\dataworld\sim\fixedimpbm}
        \cdot \expof{\sum_{\selindexin}\canparamat{\indexedselvariable}\cdot\enumformulaat{\indexedshortcatvariables}} \\
        & = \sum_{\shortcatindices\in\atomstates} \atombasemeasure[\indexedshortcatvariables] \cdot
        \expof{\sum_{\selindexin} \canparamat{\indexedselvariable} \cdot \enumformulaat{\indexedshortcatvariables}}
        \\
        & = \partitionfunctionof{\mlnparameters,\atombasemeasure} \, .
    \end{align*}
    We arrive at the claim, when combining this equation with \eqref{eq:independentSamplesFOL}.
\end{proof}

With this lemma, we are now show \theref{the:FOLworldToPLdataset}.

\begin{proof}[Proof of \theref{the:FOLworldToPLdataset}]
    %Use Equation \ref{eq:dataworldLogProb} and the \lemref{lem:FOLpartitionfunctionfactorization}.
    We have for the logarithm of the probability of a world $\dataworld$ given the distribution $\probof{\folmlnparameters}$, that
    \begin{align*}
        \lnof{\probofat{\folmlnparameters}{\indexedrandworld}}
        =  \sum_{\selindexin} \canparamat{\indexedselvariable} \contraction{\groundingof{\enumfolformula}} - \lnof{\partitionfunctionof{\folmlnparameters}}
    \end{align*}
    The first term obeys with \assref{ass:propositionalHeads}
    \begin{align*}
        \sum_{\selindexin} \canparamat{\indexedselvariable} \contraction{\groundingof{\enumfolformula}}
        &=  \contraction{\canparam} \cdot \contraction{\groundingof{\lnot\impformula}}
        + \sum_{\selindexin}\sum_{\datindexin} \canparamat{\indexedselvariable} \cdot \headfolformulaofat{\selindex}{\datshortindvariables} \\
        &= \contraction{\canparam} \cdot \contraction{\groundingof{\lnot\impformula}}
        + \sum_{\selindexin}\sum_{\datindexin} \canparamat{\indexedselvariable} \cdot \enumformulaat{\datshortcatvariables} \, .
    \end{align*}
    With \lemref{lem:FOLpartitionfunctionfactorization} we have under the given assumptions for the second term
    \begin{align*}
        \lnof{\partitionfunctionof{\folmlnparameters}} = \datanum \cdot \lnof{\partitionfunctionof{\mlnparameters,\atombasemeasure}}  + \contraction{\canparam} \cdot \contraction{\groundingof{\lnot\impformula}} \, .
    \end{align*}
    Combining both, we have
    \begin{align*}
        \frac{1}{\datanum} \lnof{\probofat{\folmlnparameters}{\indexedrandworld}}
        =  \dataaverage\sum_{\selindexin} \canparamat{\indexedselvariable} \cdot \enumformulaat{\datshortcatvariables}  - \lnof{\partitionfunctionof{\mlnparameters,\atombasemeasure}}
    \end{align*}
    which coincides with $\centropyof{\empdistribution}{\expdistof{\mlnparameters}}$.
%    Given \assref{ass:propositionalHeads} we can
%    Note that we need to correct the likelihood by the averalge log basemeasure on the data, since that term is appearing in the likelihood of a MLN.
\end{proof}

%%%%%%%%%%%%%%%%%%%%%%%
%%%% END of Monday 17.3.
%%%%%%%%%%%%%%%%%%%%%%

% Independent data investigation
Let us now investigate, in which cases the \assref{ass:independentTuples} of independent data can be matched.

\begin{lemma}
    Let $\impformula$ and $\extformulas$ be quantor and constant free and let the index tuples of the support of $\fixedimpformula$ be pairwise disjoint.
    Then the vectors \eqref{eq:data} are pairwise independent.
\end{lemma}
\begin{proof}
    Then we can reduce each sample as dependent only on an independent random world with domain by the respective objects.
    Quantor and constant-free is needed that this reductions is possible.
\end{proof}


There are situations, where \assref{ass:independentTuples} is violated.
For example, when two object tuples are not disjoint, then some formulas might always coincide on both datapoints, which would violate independence.

In further situations the atom base $\atombasemeasure$ are not the uniform $\ones$:
\begin{itemize}
    \item extraction formula being a) conjunctions of predicates: Probability that they are satisfied decreases
    b) disjunctions of predicates: Probability that they are satisfied increases
    \item extraction formula coinciding with importance formula: Always satisfied, in this case still boolean
    \item extraction formulas contradicting each other, more general not independent from each other
\end{itemize}

Let us notice, that non-boolean base measures could be treated in a same manner, but several developments in this work, such as cross-entropy decompositions in \charef{cha:probReasoning} would receive further terms.



\begin{remark}[Approximation by Independent Samples]
    As observed above, we do not have independent samples in general.
    As a consequence, we cannot apply \lemref{lem:FOLpartitionfunctionfactorization} to decompose the partition function term of the log-probability into factors to each solution map of $\impformula$.
    In this case, it might be still benefitial to use the reduction to the likelihood of a HLN, but needs to understand it as a approximation to the true world probability.

    %
    If the expectations of each sample with respect to the marginalized distributions coincide, the average of empirical distribution also coincides with these (by linearity).
    When the creation of samples has sufficient mixing properties, the empirical distribution converges to this expectation in the asymptotic case of large numbers of samples.

\end{remark}



\sect{Sample extraction from first-order logic worlds}

The decomposition of the likelihood suggests the following approach to generate samples from groundings:
%We propose the following approach to generate datacores from groundings:
\begin{itemize}
    \item Define a query formula $\impformula$, which we decompose in the basis CP decomposition and interpret each slice as the one-hot encoding of the datapoint.
    \item Define for $\atomenumeratorin$ queries $\extformulaof{\atomenumerator}$ generating the the atoms $\atomicformulaof{\atomenumerator}$:
    Predicates along with assignment of variables / constants to its positions.
    \item Contract the groundings of each formula $\extformulaof{\atomenumerator}$ with the grounding of $\impformula$ to build a data core
\end{itemize}


\subsect{Representation by Tensor Networks}

We model the extraction process as a relation between a tuple of individuals and the extracted world in the factored system of atoms $\catvariableof{\atomenumerator}$.

\begin{definition}
    \label{def:extractionRelation}
    Given a first-order logic world $\dataworld$, an importance formula $\impformula$ and extraction formulas $\extformulaof{\catenumerator}$ for $\catenumeratorin$, we define the extraction relation
    \begin{align*}
        \extractionrelation \subset \left(\symindstates\right) \otimes \left(\atomstates\right)
    \end{align*}
    by
    \begin{align*}
        \extractionrelation
        = \{ (\shortindindices, \shortcatindices)
        \, : \,  \groundingofat{\impformula}{\indexedshortindvariables} = 1 \, , \, \forall {\catenumeratorin} : \,  \catindexof{\atomenumerator} = \extformulaofat{\atomenumerator}{\indexedshortindvariables} \} \, .
    \end{align*}
\end{definition}

The encoding of an extraction relation is
\begin{align*}
    \rencodingofat{\extractionrelation}{\shortindvariables,\shortcatvariables} \subset \left(\indspace\right) \otimes \left(\atomspace\right) \,
\end{align*}
and drawn in a contraction diagram by
\begin{center}
    \input{./PartII/tikz_pics/fol_models/extraction_relation.tex}
\end{center}
Here the contraction of $\rencodingof{\impformula}$ with the truth vector $\tbasis$ represents the matching condition posed by $\impformula$ when extracting pairs of individuals.


%% Empirical Distribution
The empirical distribution is then the normalized contraction leaving only the legs to the extracted atomic formulas open, that is
\begin{align*}
    \empdistribution
    = \frac{
        \sbcontractionof{\rencodingof{\extractionrelation}}{\shortcatvariables}
    }{
        \sbcontraction{\rencodingof{\extractionrelation}}
    }  \, .
\end{align*}
Here the number of extracted data is the denominator
\begin{align*}
    \datanum
    = \contraction{\rencodingof{\extractionrelation}}
    = \contraction{\rencodingofat{\impformula}{\headvariableof{\impformula},\shortindvariables},\tbasisat{\headvariableof{\impformula}}}\, .
\end{align*}

We depict this by
\begin{center}
    \input{./PartII/tikz_pics/fol_models/empirical_generation.tex}
\end{center}




\subsect{Basis CP Decomposition of extracted data}

To connect with the empirical distribution introduced in \secref{sec:empDistribution} we now show how the empirical distribution extracted from the interpretations of the formulas $\impformula,\extformulas$ on a first-order logic world $\dataworld$ can be represented by tensor networks.

First of all, we decompose the importance formula into a basis CP format (see \charef{cha:sparseCalculus}), that is a decomposition
\begin{align*}
    \groundingofat{\impformula}{\shortindvariables}
    = \contractionof{
        \{\legcoreofat{\indenumerator}{\indvariableof{\indenumerator},\datvariable} \, : \, \indenumeratorin \}
    }{\shortindvariables}
\end{align*}
such that all $\legcoreofat{\indenumerator}{\indvariableof{\indenumerator},\datvariable}$ are directed and boolean tensors.
Here an auxiliary variables $\datvariable$ taking values in $[\datanum]$ is introduced, which we call the data variable, which enumerates the non-vanishing coordinates of $\groundingof{\impformula}$.
With this decomposition, we can understand the decomposition of $\groundingofat{\impformula}{\shortindvariables}$ as a basis encoding of an term selection map $\secdatamap$ with coordinate maps defined such that
\begin{align*}
    \rencodingofat{\secdatamap_{\indenumerator}}{\indvariableof{\indenumerator},\datvariable}
    = \legcoreofat{\indenumerator}{\indvariableof{\indenumerator},\datvariable} \, .
\end{align*}
We depict this decomposition by:
\begin{center}
    \input{./PartII/tikz_pics/fol_models/impformula_cp.tex}
\end{center}

Based on these construction, we now provide a tensor network decomposition of the extracted empirical distribution.

\begin{theorem}
    Given a first-order logic world $\dataworld$, an importance formula $\impformula$ and extraction formulas $\extformulaof{\catenumerator}$ for $\catenumeratorin$, we have
    \begin{align*}
        \rencodingofat{\extractionrelation}{\shortindvariables,\shortcatvariables} =
        \contractionof{
            \{\rencodingofat{\groundingof{\extformulaof{\atomenumerator}}}{\catvariableof{\catenumerator},\shortindvariables} \, : \, \catenumeratorin\}
            \cup \{\rencodingofat{\secdatamap_{\indenumerator}}{\indvariableof{\indenumerator},\datvariable} \, : \, \indenumeratorin\}
        }{\shortindvariables,\shortcatvariables}
    \end{align*}
    and thus
    \begin{align*}
        \empdistributionat{\shortcatvariables} =
        \frac{1}{\datanum}  \contractionof{
            \{\rencodingofat{\groundingof{\extformulaof{\atomenumerator}}}{\catvariableof{\catenumerator},\shortindvariables} \, : \, \catenumeratorin\}
            \cup \{\rencodingofat{\secdatamap_{\indenumerator}}{\indvariableof{\indenumerator},\datvariable} \, : \, \indenumeratorin\}
        }{\shortcatvariables} \, .
    \end{align*}
\end{theorem}
\begin{proof}
    To show the first claim, let us choose arbitrary state tuples $\shortindindices$ and $\shortcatindices$.
    We then have
    \begin{align*}
        &\contractionof{
            \{\rencodingofat{\groundingof{\extformulaof{\atomenumerator}}}{\catvariableof{\catenumerator},\shortindvariables} \, : \, \catenumeratorin\}
            \cup \{\rencodingofat{\secdatamap_{\indenumerator}}{\indvariableof{\indenumerator},\datvariable} \, : \, \indenumeratorin\}
        }{\indexedshortindvariables,\indexedshortcatvariables} \\
        & \quad  =  \contraction{
            \{\rencodingofat{\groundingof{\extformulaof{\atomenumerator}}}{\indexedcatvariableof{\catenumerator},\indexedshortindvariables} \, : \, \catenumeratorin\}
            \cup \{\rencodingofat{\secdatamap_{\indenumerator}}{\indexedindvariableof{\indenumerator},\datvariable} \, : \, \indenumeratorin\}
        } \, .
    \end{align*}
    This contraction evaluates to $1$, if and only if for all $\catenumeratorin$ we have $\rencodingofat{\groundingof{\extformulaof{\atomenumerator}}}{\catvariableof{\catenumerator},\shortindvariables}=1$ and
    \begin{align*}
        \contraction{\{\rencodingofat{\secdatamap_{\indenumerator}}{\indexedindvariableof{\indenumerator},\datvariable} \, : \, \indenumeratorin\}}  = 1 \, .
    \end{align*}
    The first condition is equal to $\catindexof{\atomenumerator} = \extformulaofat{\atomenumerator}{\indexedshortindvariables}$ for all $\catenumeratorin$ and the second to
    \begin{align*}
        \groundingofat{\impformula}{\indexedshortindvariables} = 1 \, .
    \end{align*}
    Comparing with the definition of the extraction relation (see \defref{def:extractionRelation}), we notice that these conditions are equal to $(\shortindindices,\shortcatindices)\in\extractionrelation$ and therefore to
    \begin{align*}
        \rencodingofat{\extractionrelation}{\indexedshortindvariables,\indexedshortcatvariables} \, .
    \end{align*}
    The first claim follows, since $\rencodingof{\extractionrelation}$ is boolean, as is the contraction of the cores $\rencodingof{\groundingof{\extformulaof{\atomenumerator}}}$ with the cores $\rencodingof{\secdatamap_{\indenumerator}}$, which leaves the outgoing variables $\shortcatvariables$ open.
    The second claim follows from the first using that $\empdistributionat{\shortcatvariables}=\frac{1}{\datanum}\contractionof{\rencodingof{\extractionrelation}}{\shortcatvariables}$.
\end{proof}

To connect with the representation of empirical distributions based on data cores (see \secref{sec:empDistribution}), we now form data cores by contractions with the grounding of extraction formulas with the cores $\rencodingof{\secdatamap_{\indenumerator}}$ (see \figref{fig:datacoreGeneration}),
\begin{align*}
    \datacoreofat{\atomenumerator}{\catvariableof{\catenumerator},\datvariable}
    = \sbcontractionof{
        \{\rencodingofat{\groundingof{\extformulaof{\atomenumerator}}}{\catvariableof{\catenumerator},\shortindvariables}\}
        \cup \{ \legcoreofat{\indenumerator}{\indvariableof{\indenumerator},\datvariable} \, : \, \indenumeratorin\}
    }{\catvariableof{\atomenumerator},\datvariable} \, .
\end{align*}

% Empirical distribution
The empirical distribution is then a tensor network of these tensors, as we show next.

\begin{theorem}
    \label{the:extractionDataCores}
    We have
    \begin{align*}
        \sbcontractionof{\rencodingof{\extractionrelation}}{\shortcatvariables}
        = \contractionof{\{\datacoreofat{\atomenumerator}{\datvariable,\catvariableof{\atomenumerator}} \, : \, \atomenumeratorin\}}{\shortcatvariables}
    \end{align*}
    and thus
    \begin{align*}
        \empdistributionat{\shortcatvariables}
        = \frac{1}{\datanum} \contractionof{\{\datacoreofat{\atomenumerator}{\datvariable,\catvariableof{\atomenumerator}}  \, : \, \atomenumeratorin\}}{\shortcatvariables} \, .
    \end{align*}
\end{theorem}
\begin{proof}
    By \theref{the:extractionrelationDecomposition} we have
    \begin{align*}
        \rencodingofat{\extractionrelation}{\shortindvariables,\shortcatvariables} =
        \contractionof{
            \{\rencodingofat{\groundingof{\extformulaof{\atomenumerator}}}{\catvariableof{\catenumerator},\shortindvariables} \, : \, \catenumeratorin\}
            \cup \{\rencodingofat{\secdatamap_{\indenumerator}}{\indvariableof{\indenumerator},\datvariable} \, : \, \indenumeratorin\}
        }{\shortindvariables,\shortcatvariables} \, .
    \end{align*}
    Since $\rencodingofat{\secdatamap_{\indenumerator}}{\indvariableof{\indenumerator},\datvariable}$ are directed and boolean, they can be copied and separately contracted with each $\groundingof{\extformulaof{\atomenumerator}}$, without changing the contraction.
    We arrive at
    \begin{align*}
        &\rencodingofat{\extractionrelation}{\shortindvariables,\shortcatvariables} \\
        &\quad = \contractionof{
            \big\{\contractionof{
                \{\rencodingofat{\groundingof{\extformulaof{\atomenumerator}}}{\catvariableof{\catenumerator},\shortindvariables}\}
                \cup \{\rencodingofat{\secdatamap_{\indenumerator}}{\indvariableof{\indenumerator},\datvariable} \, : \, \indenumeratorin\}
            }{\catvariableof{\catenumerator},\datvariable} \, : \, \catenumeratorin \big\}
        }{\shortindvariables,\shortcatvariables} \\
        & \quad =  \contractionof{\{\datacoreofat{\atomenumerator}{\datvariable,\catvariableof{\atomenumerator}}  \, : \, \atomenumeratorin\}}{\shortcatvariables} \, ,
    \end{align*}
    which established the claim.
\end{proof}

% Efficient contraction: Do also basis decomposition of the extraction query and use efficient contraction!
%Towards efficient calculation of the data cores, we build a basis CP decomposition of $\groundingof{\impformula}$, where we further demand $\scalarcore=\ones$.
%This is a collection of basis leg cores $\legcoreof{\fixedimpformula,\indenumerator}$ such that
%\begin{align*}
%    \fixedimpformula[\shortindvariablelist]
%    = \contractionof{ \left\{ \legcoreofat{\fixedimpformula,\indenumerator}{\datvariable,\indvariableof{\indenumerator}} \, : \, \indenumeratorin \right\} }{\shortindvariablelist} \, .
%\end{align*}

% Data enumeration -> To representation
%We can further utilize any decomposition of $\impformula$ into a directed and binary CP Format to enumerate the datapoints by the slice index $\datindex$. % Approaches like SPARQL directly give us these by solution mappings.
%Understanding $\impformula$ as a query on the world being the database, such decomposition is given by the set of solution mappings.


\begin{figure}[h]
    \begin{center}
        \begin{tikzpicture}[scale=0.35, yscale=1, thick] % , baseline = -3.5pt


    \draw[->] (4,-1) -- (4,1) node[midway, right] {\tiny $\catvariableof{\atomenumerator}$};
    \draw (3,-1) rectangle (5,-3);
    \node[anchor=center] (text) at (4,-2) {$\datacoreof{\atomenumerator}$};
    \draw[<-] (4,-3) -- (4,-5) node[midway, right] {\tiny $\datvariable$};

    \node[anchor=center] (text) at (7,-2) {${=}$};

    \begin{scope}
        [shift={(10,0)}]

        \draw[->] (3,1) -- (3,3) node[midway, right] {\tiny $\catvariableof{\atomenumerator}$};
        \draw (-1,1) rectangle (7,-1);
        \node[anchor=center] (text) at (3,0) {$\rencodingof{\groundingof{\extformulaof{\atomenumerator}}}$};

        \draw[->] (0,-3) -- (0,-1) node[midway,left] {\tiny $\indvariableof{0}$};
        \draw[->] (3,-3) -- (3,-1) node[midway,left] {\tiny $\indvariableof{1}$};
        \draw[->] (6,-3) -- (6,-1) node[midway,left] {\tiny $\indvariableof{2}$};


    \end{scope}

    \begin{scope}
        [shift={(10,-2)}]

        \coordinate (conposseldec) at (4.5,-5.5);
        \drawvariabledot{4.5}{-5.5}
        \draw[<-] (conposseldec) -- (4.5,-7.5) node[midway, right] {\tiny $\indexvariable$};

        \draw (-1,-1) rectangle (1, -3);
        \node[anchor=center] (text) at (0,-2) {\small $\rencodingof{\secdatamap_0}$};%{\small $\legcoreof{\fixedimpformula,0}$};
        \draw[<-] (0,-3) to[bend right=20] (conposseldec);

        \draw (2,-1) rectangle (4, -3);
        \node[anchor=center] (text) at (3,-2) {\small $\rencodingof{\secdatamap_1}$};%{\small $\legcoreof{\fixedimpformula,1}$};
        \draw[<-] (3,-3) to[bend right=20]  (conposseldec);

        \draw (5,-1) rectangle (7, -3);
        \node[anchor=center] (text) at (6,-2) {\small $\rencodingof{\secdatamap_2}$};%{\small $\legcoreof{\fixedimpformula,2}$};
        \draw[<-] (6,-3) to[bend right=-20]  (conposseldec);

        \draw[<-] (9,1) -- (9,-1) node[midway,left] {\tiny $\indvariableof{3}$};
        \draw (8,-1) rectangle (10, -3);
        \node[anchor=center] (text) at (9,-2) {\small $\rencodingof{\secdatamap_3}$};%{\small $\legcoreof{\fixedimpformula,3}$};
        \draw[<-] (9,-3) to[bend right=-20]  (conposseldec);


        \node[anchor=center] (text) at (12,-2) {$\cdots$};

        \draw[<-] (15,1) -- (15,-1) node[midway,left] {\tiny $\indvariableof{\indorder-1}$};
        \draw (13.5,-1) rectangle (16.5, -3);
        \node[anchor=center] (text) at (15,-2) {\small $\rencodingof{\secdatamap_{\indorder-1}}$};%{\small $\legcoreof{\fixedimpformula,\variableorder-1}$};
        \draw[<-] (15,-3) to[bend left=20]  (conposseldec);


        \draw (8,1) rectangle (16, 3);
        \node[anchor=center] (text) at (12,2) {\small $\ones$};


    \end{scope}


    \node[anchor=center] (text) at (29,-2) {${=}$};


    \begin{scope}
        [shift={(32,0)}]

        \draw[->] (3,1) -- (3,3) node[midway, right] {\tiny $\catvariableof{\atomenumerator}$};
        \draw (-1,1) rectangle (7,-1);
        \node[anchor=center] (text) at (3,0) {$\rencodingof{\groundingof{\extformulaof{\atomenumerator}}}$};

        \draw[->] (0,-3) -- (0,-1) node[midway,left] {\tiny $\indvariableof{0}$};
        \draw[->] (3,-3) -- (3,-1) node[midway,left] {\tiny $\indvariableof{1}$};
        \draw[->] (6,-3) -- (6,-1) node[midway,left] {\tiny $\indvariableof{2}$};


    \end{scope}

    \begin{scope}
        [shift={(32,-2)}]


        \coordinate (conposseldec) at (3,-5.5);
        \drawvariabledot{3}{-5.5}
        \draw[<-] (conposseldec) -- (3,-7.5) node[midway, right] {\tiny $\datvariable$};

        \draw (-1,-1) rectangle (1, -3);
        \node[anchor=center] (text) at (0,-2){\small $\rencodingof{\secdatamap_0}$};%{\small $\legcoreof{\fixedimpformula,0}$};
        \draw[<-] (0,-3) to[bend right=20] (conposseldec);

        \draw (2,-1) rectangle (4, -3);
        \node[anchor=center] (text) at (3,-2) {\small $\rencodingof{\secdatamap_1}$};%{\small $\legcoreof{\fixedimpformula,1}$};
        \draw[<-] (3,-3) to[bend right=0]  (conposseldec);

        \draw (5,-1) rectangle (7, -3);
        \node[anchor=center] (text) at (6,-2) {\small $\rencodingof{\secdatamap_2}$};%{\small $\legcoreof{\fixedimpformula,2}$};
        \draw[<-] (6,-3) to[bend right=-20]  (conposseldec);


    \end{scope}


\end{tikzpicture}
    \end{center}
    \caption{Generation of a data core for the variable $\catvariableof{\catenumerator}$ given an extraction formula $\extformulaof{\catenumerator}$ and an importance formula, which grounding is decomposed into a basis CP format with leg vectors $\rencodingofat{\secdatamap_{\indenumerator}}{\indvariableof{\indenumerator},\datvariable}$.
    Term variables, which are appearing in the importance formula, but not in the extraction formula $\extformulaof{\catenumerator}$ can be treated trivally by contraction with the trivial tensor (here $\indvariableof{4},\ldots,\indvariableof{\indorder-1})$.
    }
    \label{fig:datacoreGeneration}
\end{figure}


% Comment: Exploitation of common structure
When many atom extraction formulas differ only by a constant, we can replace the constant by an auxiliary term variable.
The atoms are then the atomizations of this variable (see \secref{sec:categoricalTN}), treated as a categorical variable, with respect to the constant in the extraction query.
The advantages are that we can avoid the $\rencodingof{}$-formalism and directly model the categorical distributions.

This also enables a batchwise computation of multiple $\sparql$ queries, which differ only in one constant.


%\subsect{Design of the Formulas}
%
%Most intuitive when labeling individuals by classes.
%Extraction formulas $\extformulas$ can then be defined by subclasses of the member of a class and relations between objects of different classes. % Koller calls atomic formulas the template attributes
%We then choose $\formulaset$ as more involved formulas decomposed into connectives acting on these atoms.
%The importance formula $\impformula$ is then designed based on class memberships to ensure, that the arguments of the formulas are always of specific classes. % Koller specifies to each argument of the attributes a class
%
%% Approach
%We propose to
%\begin{itemize}
%    \item Execute an extraction query to get pairs of individuals (the pairDf).
%    \item Propositionalize the FOL Formulas independently on each tuple taking the individuals as a set of constant and filtering on the possible properties of each individuals.
%    (Can understand as adding knowledge that most of the relations do not hold)
%    \item Understand each such generated knowledge base as datapoint and average over them to get the empirical distribution to be fit.
%    \item Fit a MLN describing the statistical relations of unseen results of the extraction query, based on likelihood maximation.
%\end{itemize}




\sect{Generation of first-order logic worlds}

\red{
    So far we have discussed, how MLNs for FOL Knowledge Bases such as Knowledge Graphs can be built by extracting data.
    Conversely, any binary tensor can be interpreted as a Knowledge Graph.
    To be more precise, we follow the intuition that the ones coordinates mark possible worlds compatible with the knowledge about a factored system.
    Each possible world can then be encoded in a subgraph of the Knowledge Graph representing the world.
%
    This amounts to an "inversion" of the data generation process described in the subsection above.
}

In the previous section we have described a way to extract an effective empirical distribution for the likelihood of a first-order logic world given a HFLN.
We now want to investigate methods to reproduce an empirical distribution based on a constructed first-order logic world.

\begin{definition}[Reproduction of Empirical Distributions]
    Given an empirical distribution $\empdistribution\in\atomspace$, we say that a triple $(\dataworld,\impformula,\shortextformulas)$ of a FOL world $\dataworld$ an importance formula $\impformula$ and extraction formulas $\shortextformulas=\{\extformulaof{\atomenumerator}\,:\,\atomenumeratorin\}$ reproduces $\empdistribution$, when
    \begin{align*}
        \empdistribution
        = \normationof{\{\groundingof{\impformula}\}\cup\{\rencodingof{\kggroundingof{\extformulaof{\atomenumerator}}\, : \, \atomenumeratorin}\}}{\shortcatvariables} \, .
    \end{align*}
\end{definition}

% If \datamap is not known
Note that for distribution $\probtensor$ to be reproducable, it needs to have rational coordinates. %, since each coordinate can be interpreted as the frequency of the respective world in the data $\datamap$.
If any only if all coordinates are rational, we find a $\datanum\in\nn$ such that $\imageof{\datanum\cdot\probtensor}\subset\nn$.
We can then interpret $\datanum$ as the number of samples, and construct a sample selector map by understanding each coordinate of $\datanum\cdot\probtensor$ as the number of appearances of the respective world in the samples.

We show different schemes and give examples on Knowledge Graphs, where we provide examples for importance and extraction formulas by $\sparql$ queries.


%\subsect{Example: Generation of Knowledge Graphs} % To generation of FOL worlds?
%
% Having a directed and binary CP decomposition of $\exformula$, each possible world is encoded by a slice.


% Formalization
%\begin{definition}[Reproduction of Empirical Distributions]
%    Given an empirical distribution $\empdistribution\in\bigotimes_{\atomenumeratorin}\rr^2$, we say that a tuple $(\kg,\impformula,\{\extformulas\})$ of a Knowledge Graph $\kg$ and queries $\impformula,\extformulaof{\atomenumerator}$ reproduces $\empdistribution$, when
%    \[\empdistribution = \normationof{\{\kggroundingof{\impformula}\}\cup\{\rencodingof{\kggroundingof{\extformulaof{\atomenumerator}}\, : \, \atomenumeratorin}\}}{\shortcatvariables} \, .  \]
%\end{definition}

%

%In a frequentist interpretation we instantiate each world according to the rate $\probtensor(\atomindices)$.
%This interpretation requires a rounding of the real probabilities by rational numbers.


\subsect{Samples by single objects}

%\subsect{Samples by single objects}

In the first reproduction scheme we construct datapoints by dedicated objects, which represent a sample, that is we choose a domain $\worlddomain=[\datdim]$.

\begin{theorem}
    \label{the:reproducingSingleObjects}
    Let there be an empirical distribution $\empdistribution$ to a sample selector map $\datamap$ (see \defref{def:dataMap}), we construct a world $\dataworld[\selvariable,\indvariable]$ with $\atomorder$ unary predicates by
    \begin{align*}
        \dataworldat{\selvariable,\indvariable}
        = \sum_{\atomenumeratorin} \sum_{\datindexin \, : \datamap_{\atomenumerator}(\datindex)=1} \onehotmapofat{\atomenumerator}{\selvariable} \otimes \onehotmapofat{\datindex}{\indvariable} \, .
    \end{align*}
    We further choose a trivial importance query, that is
    \begin{align*}
        \groundingofat{\impformula}{\indvariable} = \onesat{\indvariable} \, ,
    \end{align*}
    and extraction queries coinciding with the unary predicates, that is for $\atomenumeratorin$
    \begin{align*}
        \extformulaof{\atomenumerator} = \folpredicateof{\atomenumerator} \, .
    \end{align*}
    Then, the triple $(\dataworld,\impformula,\shortextformulas)$ reproduces $\empdistribution$.
%    reproduces with the trivial importance query and extraction queries coinciding with the predicates the dataset $\datamap$.
\end{theorem}
\begin{proof}
    By \theref{the:extractionDataCores} it is enough to show, that the data cores constructed from the data extraction process coincide with those of $\empdistribution$.
    We enumerate to this end the non-vanishing coordinates of $\groundingof{\impformula}$ by the data variable $\datvariable$ taking values $\datindexin$, as
    \begin{align*}
        \groundingofat{\impformula}{\indvariable=\datindex} = 1 \,
    \end{align*}
    and choose
    \begin{align*}
        \secdatamap = \identity \, .
    \end{align*}
    For arbitrary $\atomenumeratorin$ and $\datindexin$ we now have
    \begin{align*}
        \datacoreofat{\atomenumerator}{\catvariableof{\catenumerator},\indexeddatvariable}
        &= \contractionof{
            \rencodingofat{\groundingof{\extformulaof{\atomenumerator}}}{\catvariableof{\catenumerator},\indvariable},
            \legcoreofat{0}{\indvariable,\indexeddatvariable}
        }{\catvariableof{\atomenumerator},\datvariable} \\
        &= \contractionof{
            \rencodingofat{\groundingof{\extformulaof{\atomenumerator}}}{\catvariableof{\catenumerator},\indvariable},
            \onehotmapofat{\secdatamap(\datindex)}{\indvariable}
        }{\catvariableof{\atomenumerator},\indexeddatvariable} \\
        &= \onehotmapofat{\datamap_\atomenumerator(\datindex)}{\catvariableof{\catenumerator}} \, .
    \end{align*}
    This coincides with the slice of the data core of the CP representation of empirical distributions used in \theref{the:empCPRep}.
    Since the slice and the core was arbitrary, the tensor network representations in \theref{the:empCPRep} and \theref{the:extractionDataCores} are equal and thus the triple $(\dataworld,\impformula,\shortextformulas)$ reproduces $\empdistribution$.
\end{proof}


We now give by the next theorem an example of a Knowledge Graph with $\sparql$ queries reproducing and arbitrary empirical distribution.

\begin{theorem}
    \label{the:reproducingKGSingelObjects}
    Let $\empdistribution$ be an empirical distribution to the sample selector $\datamap$.
    We construct a Knowledge Graph of the resources $\worlddomain = \{s_\datindex \, : \, \datindexin\} \cup \{C\} \cup \{C_\atomenumerator \, : \, \atomenumeratorin\}$, where $s_{\datindex}$ represent samples and $C_\atomenumerator$ unary predicates, by
    \begin{align*}
        \kggroundingof{\rdf}
        =
        \sum_{\datindexin}
        \onehotmapof{\indexinterpretationof{s_\datindex}}{\sindvariable}
        \otimes \onehotmapof{\indexinterpretationof{\mathrdftype}}{\pindvariable}
        \otimes \onehotmapof{\indexinterpretationof{C}}{\oindvariable}
        +
        \sum_{\datindexin} \sum_{\atomenumeratorin \, : \, \datamap_{\atomenumerator}(\datindex)=1}
        \onehotmapof{\indexinterpretationof{s_\datindex}}{\sindvariable}
        \otimes \onehotmapof{\indexinterpretationof{\mathrdftype}}{\pindvariable}
        \otimes \onehotmapof{\indexinterpretationof{C_\atomenumerator}}{\oindvariable} \, .
    \end{align*}
    We further define an importance formula by the $\sparql$ query
    \begin{centeredcode}
        \impformula = SELECT \{ ?x \} WHERE \{ ?x \quad \rdftype\quad C \, .\}
    \end{centeredcode}
    and for each $\atomenumeratorin$ an extraction formula by the query
    \begin{centeredcode}
        $\extformulaof{\atomenumerator}$ = SELECT \{ ?x \} WHERE \{ ?x \quad \rdftype \quad $C_\atomenumerator$ \, .\} \, .
    \end{centeredcode}
    Then the triple $(\kg,\impformula,\shortextformulas)$ reproduces $\empdistribution$.
\end{theorem}
\begin{proof}
    We show the theorem analogously to \theref{the:reproducingSingleObjects}, with the slide difference in the importance formula.
    We have for the grounding of $\impformula$ on $\kg$ that
    \begin{align*}
        \kggroundingofat{\impformula}{\indvariable} = \sum_{\datindexin}  \onehotmapof{\indexinterpretationof{s_\datindex}}{\indvariable}
    \end{align*}
    and enumerate the non-vanishing coordinates by $\datvariable$.

    For each extraction formula we have
    \begin{align*}
        \kggroundingofat{\extformulaof{\atomenumerator}}{\indvariable} = \sum_{\datindexin \, : \, \datamap_{\atomenumerator}(\datindex)=1} \onehotmapof{\indexinterpretationof{s_\datindex}}{\indvariable} \,.
    \end{align*}
    It follows that the data cores used in \theref{the:extractionDataCores} are
    \begin{align*}
        \rencodingofat{\datamap_\atomenumerator}{\catvariableof{\atomenumerator},\datindex}
        = \onehotmapofat{0}{\catvariableof{\atomenumerator}} \otimes \left(\sum_{\datindexin \, : \, \datamap_{\atomenumerator}(\datindex)=0} \onehotmapofat{\datindex}{\datvariable}\right)
        +\onehotmapofat{1}{\catvariableof{\atomenumerator}} \otimes \left(\sum_{\datindexin \, : \, \datamap_{\atomenumerator}(\datindex)=1} \onehotmapofat{\datindex}{\datvariable}\right)
    \end{align*}
    and they thus coincide with those in the decomposition in \theref{the:empCPRep}.
    The claim follows therefore with the same argumentation as in the proof of \theref{the:reproducingSingleObjects}.
\end{proof}

%
Let us provide some more insights on the construction of the reproducing Knowledge Graph in \theref{the:reproducingKGSingelObject}.
By the insertions to the one-hot encodings $\onehotmapof{\indexinterpretationof{s_\datindex}}{\sindvariable} \otimes \onehotmapof{\indexinterpretationof{\mathrdftype}}{\pindvariable} \otimes \onehotmapof{\indexinterpretationof{C}}{\oindvariable}$ we mark each sample representing resource by a class and ensure its appearance as a $\mathrm{owl:NamedIndividual}$ in the graph.
The insertions $\onehotmapof{\indexinterpretationof{s_\datindex}}{\sindvariable}\otimes \onehotmapof{\indexinterpretationof{\mathrdftype}}{\pindvariable} \otimes \onehotmapof{\indexinterpretationof{C_\atomenumerator}}{\oindvariable}$ on the other side encode the sample selecting map, by inserting exactly the assertions corresponding with the respective sample.
% 
In this simple Knowledge Graph, Description Logic is expressive enough to represent any formula $\folexformula$ composed of the formulas $\extformulas$.

%
%\begin{theorem}
%    Let there any empirical distribution $\empdistribution\in\bigotimes_{\atomenumeratorin}\rr^2$ and $\datanum\in\nn$ such that $\imageof{\datanum\cdot\empdistribution}\subset\nn$.
%    Then the tuple $(\kg,\impformula,\{\extformulas\})$ defined by a Knowledge Graph
%    \begin{align}
%        \kg =
%        & \bigcup_{\atomindicesin}  \{(
%        s_{j, \atomindices} \quad \mathrm{rdf:type} \quad C ) : j \in [\datanum\cdot\empdistribution(\atomindices)] \}  \\
%        &\bigcup_{\atomindicesin}  \{(
%        s_{j, \atomindices} \quad \mathrm{rdf:type} \quad C_\atomenumerator
%        ) : j \in [\datanum\cdot\empdistribution(\atomindices)], \atomenumeratorin , \atomlegindexof{\atomenumerator}=1\}
%    \end{align}
%    further an importance formula by the query
%    \begin{centeredcode}
%        \impformula = SELECT \{ ?x \} WHERE \{ ?x \quad \rdftype\quad C \, .\}
%    \end{centeredcode}
%    and extraction formulas for each $\atomenumeratorin$ by the query
%    \begin{centeredcode}
%        $\extformulaof{\atomenumerator}$ = SELECT \{ ?x \} WHERE \{ ?x \quad \rdftype \quad $C_\atomenumerator$ \, .\}
%    \end{centeredcode}
%    reproduces $\empdistribution$.
%\end{theorem}
%\begin{proof}
%    With respect to any enumeration of the resources of $\kg$ we have
%    \begin{align}
%        \kggroundingof{\impformula}
%        = \sum_{\atomindicesin} \sum_{j \in [\datanum\cdot\empdistribution(\atomindices)]} \onehotmapof{s_{j, \atomindices} }
%    \end{align}
%    and
%    \begin{align}
%        \kggroundingof{\extformulaof{\atomenumerator}}
%        = \sum_{\atomindicesin \, : \, \atomlegindexof{\atomenumerator} = 1} \sum_{j \in [\datanum\cdot\empdistribution(\atomindices)]} \onehotmapof{s_{j, \atomindices} } \, .
%    \end{align}
%    Summing over the resource variables of these tensors in a contraction we get
%    \begin{align}
%        \contractionof{\{\kggroundingof{\impformula}\}\cup\{\rencodingof{\kggroundingof{\extformulaof{\atomenumerator}}\, : \, \atomenumeratorin}\}}{\shortcatvariables}
%        & = \sum_{\atomenumeratorin}  \datanum\cdot\empdistribution(\atomindices) \cdot \onehotmapof{\atomindices} = \datanum \cdot \empdistribution
%    \end{align}
%    and therefore
%    \begin{align}
%        \normationof{\{\kggroundingof{\impformula}\}\cup\{\rencodingof{\kggroundingof{\extformulaof{\atomenumerator}}}\, : \, \atomenumeratorin\}}{\shortcatvariables} = \empdistribution \, .
%    \end{align}
%\end{proof}







\subsect{Samples by pairs of objects}

%\paragraph{TBox:} The categorical variables of the factored system are the classes.
%We define atomic formulas by the state indicators of each categorical variable as in \secref{sec:categoricalTN}.
%Each such atomic formula corresponds with a sub-class of the classes.
%By definition, each collection of state indicators define thus pairwise disjoint subclasses.
%
%\paragraph{ABox:} The samples are represented by single individuals in the Knowledge Graph.
%Their sub-class memberships corresponding with the categorical variables of the system are instantiated whenever the atom is true in the sample.
%%\subsubsect{Samples by pairs of resources}
%
%\begin{remark}[Refinement of the Samples]
%    We can split each sample node into a pair of individuals.
%    For this we need to specify, which each class membership will be encoded in a unary or binary attribute of the splitted individuals.
%    This specification is possible based on the extraction query and the atomic formulas.
%\end{remark}
%
%%
%Taking any importance query $\impformula$, which has no permutation symmetries, we can instantiate each projection variable for each sample and prepare the links according to the triple patterns.
%When the atom queries $\extformulas$ have different triple patterns compared with $\impformula$, we instantiate those in cases where $\atomlegindexof{\atomenumerator}=1$.


%
We now instantiate multiple objects for each datapoint, one for each variable of the importance formula, i.e. $\worlddomain=[\datdim]\times[\indorder]$
Label individuals $s_{\datindex,\indenumerator}$ by data index and variable index.

\begin{lemma}
    Let there a data map $\datamap$, queries $\impformula,\shortextformulas$ and a first-order logic world containing objects $s_{\datindex,\indenumerator}$ for $\datindexin$ and $\indenumeratorin$
    If
    \begin{align*}
        \kggroundingof{\impformula}
        = \sum_{\datindexin} \bigotimes_{\indenumeratorin} \onehotmapofat{\indexinterpretationof{s_{\datindex,\indenumerator}}}{\indvariableof{\indenumerator}}
    \end{align*}
    and for any $\atomenumeratorin$
    \begin{align*}
        \kggroundingof{\extformulaof{\atomenumerator}}
        = \sum_{\datindex : \datamap_{\atomenumerator}(\datindex)=1} \bigotimes_{\indvariableof{\indenumerator} \in \indvariableof{\extformulaof{\atomenumerator}}}
        \onehotmapofat{\indexinterpretationof{s_{\datindex,\indenumerator}}}{\indvariableof{\indenumerator}} \, .
    \end{align*}
%    \[ \kggroundingof{\extformulaof{\atomenumerator}}
%    = \sum_{\datindex : \datamap^{\atomenumerator}(\datindex)=1} \bigotimes_{\indenumerator \in \extformulaof{\atomenumerator}} \onehotmapof{\datindex,\indenumerator} \, . \]
    Then the tuple $(\kg,\impformula,\{\extformulas\})$ reproduces $\empdistribution$.
\end{lemma}
\begin{proof}
    We notice, that the grounding of the importance formula is in a basis CP format, since by assumption
    \begin{align*}
        \kggroundingof{\impformula}
        = \sum_{\datindexin} \bigotimes_{\indenumeratorin} \onehotmapofat{\indexinterpretationof{s_{\datindex,\indenumerator}}}{\indvariableof{\indenumerator}} \, .
    \end{align*}
    We choose $\datvariable$ to enumerate the non-vanishing entries and get a term selecting map
    \begin{align*}
        \secdatamap_{\indenumerator}(\datindex) = \indexinterpretationof{s_{\datindex,\indenumerator}} \, .
    \end{align*}
    From this we have
    \begin{align*}
        \contractionof{
            \{\rencodingofat{\kggroundingof{\extformulaof{\atomenumerator}}}{\catvariableof{\atomenumerator},\indvariableof{\extformulaof{\atomenumerator}}}\} \cup
            \{\rencodingofat{\secdatamap_{\indenumerator}}{\indvariableof{\indenumerator},\datvariable} \, : \, \indenumeratorin\}
        }{\catvariableof{\catenumerator},\datvariable}
        = \rencodingofat{\datamap_{\atomenumerator}}{\catvariableof{\catenumerator},\datvariable}
    \end{align*}
    and the claim follows with the same argumentation as in the proof of \theref{the:reproducingSingleObjects}.
\end{proof}


%Let us construct a Knowledge Graph
%\begin{align*}
%        \kggroundingof{\rdf}
%        =
%        \sum_{\datindexin}\sum_{\indenumeratorin}
%        \onehotmapof{\indexinterpretationof{s_{\datindex,\indenumerator}}}{\sindvariable}
%        \otimes \onehotmapof{\indexinterpretationof{\mathrdftype}}{\pindvariable}
%        \otimes \onehotmapof{\indexinterpretationof{C}}{\oindvariable}
%        +
%        \sum_{\datindexin} \sum_{\atomenumeratorin \, : \, \datamap_{\atomenumerator}(\datindex)=1} \sum_{\indvariableof{\indenumerator}\in\indvariableof{}}
%        \onehotmapof{\indexinterpretationof{s_{\datindex,\indenumerator}}}{\sindvariable}
%        \otimes \onehotmapof{\indexinterpretationof{\mathrdftype}}{\pindvariable}
%        \otimes \onehotmapof{\indexinterpretationof{C_\atomenumerator}}{\oindvariable} \, .
%\end{align*}
%    We further define an importance formula by the $\sparql$ query
%\begin{centeredcode}
%        \impformula = SELECT \{ ?x_0 \cdots ?x_{\indorder-1} \} WHERE \{ ?x_0 \quad \rdftype\quad C \, .\}
%\end{centeredcode}
%    and for each $\atomenumeratorin$ an extraction formula by the query
%\begin{centeredcode}
%        $\extformulaof{\atomenumerator}$ = SELECT \{ ?x \} WHERE \{ ?x \quad \rdftype \quad $C_\atomenumerator$ \, .\} \, .
%\end{centeredcode}
%    Then the triple $(\kg,\impformula,\shortextformulas)$ reproduces $\empdistribution$.



\sect{Discussion}


% Probabilistic Relational Models
Statistical Models are called Probabilistic Relational Models. % (RUSSELL - Chapter Probabilistic Programming).
Extensions are models that also handle structural uncertainty, i.e. distributions of worlds with varying $\worlddomain$.

% Comparison with network science
In the emerging area of network science \cite{barabasi_network_2016, giovanni_russo_vito_latora_complex_2017}, statistical models for random graphs are investigated.
Statistical Models of first-order logic go beyond the typical single edge type perspective of network science.


%
\begin{remark}[Alternative Representation of empirical distributions]
    So far, we have motivated the representation of empirical distributions based on basis CP decompositions based on data maps.
    In this section, based on the extraction queries, we have observed that empirical distributions might have more efficient representation formats.
    In many applications such as the computation of log-likelihoods we can use any representation of the empirical distribution by tensor networks.
    It is thus not necessary to compute the data cores as above, unless one requires a list of the extracted samples.
\end{remark}


    \part{\partthreetext}\label{par:three}

    Based on the logical interpretation we often handle tensor calculus with specific tensors.
    Often, they are binary (that is their coordinates are in $\{0,1\}$ corresponding with a Boolean), and sparse (that is having a decomposition with less storage demand).
    We investigate it in this part in more depth the properties of such tensors, which where exploited in the previous parts.

    \section{Coordinate Calculus} 

In the previous chapters, information to states has been stored in coordinates of a tensor.
To distinguish from other schemes of calculus, we call this scheme of storing and retrieving information the coordinate calculus.
We in this chapter investigate in more depth, which operations can be performed based on such tensors and proof the applied properties.

\red{Add summations and further operations?}

\red{Discuss things like $\expof{\hypercore}$ as coordinatewise operations, by circles in the factor graph diagrams.}

\subsection{One-hot encodings as basis}

\begin{lemma}[Basis of tensor spaces]\label{lem:tensorBasisDecomposition}
	The image of the one-hot encoding map is a linear basis of the tensor space $\facspace$.
	Any element $\exformula\in\facspace$ has a decomposition 
		\[ \exformula[\catvariables] = \sum_{\catindices} \exformula[\indexedcatvariables] \cdot \onehotmapof{\catindices}[\catvariables] \, . \]
	We notice that the coordinates are the weights to the basis elements in the one-hot decomposition.
\end{lemma}
\begin{proof}
	For any $\tildecatindices\in\facstates$ we have
		\[ \formulaat{\shortcatvariables=\tilde{\catindex}_{[\catorder]}}
		= \sum_{\catindices} \exformula[\indexedcatvariables] \cdot \onehotmapofat{\catindices}{\shortcatvariables=\tilde{\catindex}_{[\catorder]}} 
		= \formulaat{\shortcatvariables=\tilde{\catindex}_{[\catorder]}} \cdot \onehotmapofat{\catindices}{\shortcatvariables=\tilde{\catindex}_{[\catorder]}}   \]
\end{proof}


\begin{definition}
	Given any real-valued function 
		\[ \exfunction : \facstates \rightarrow \rr \]
	we define the coordinate encoding by
		\[ \hypercoreof{\exfunction} = \sum_{\catindices\in\facstates} \exfunction(\catindices) \cdot \onehotmapof{\catindices} \, . \]
\end{definition}


\begin{theorem}[Coordinate Calculus]\label{the:coordinateCalculus}
	Given any tensor $\hypercoreat{\catvariables}$ can retrieve its coordinate indexed by $\catindices$ as
		\[ \hypercoreat{\indexedcatvariables} = \sbcontraction{\hypercore, \onehotmapof{\catindices}} \, . \]
\end{theorem}
\begin{proof}
	We use the decomposition in Lemma~\ref{lem:tensorBasisDecomposition} and have
	\begin{align*}
		\contractionof{\{\hypercore, \onehotmapof{\catindices}\}}{\varnothing} 
		& = \sum_{\tildecatindices} \hypercoreat{\tildeindexedcatvariables} \cdot \contractionof{\{\onehotmapof{\tildecatindices} ,\onehotmapof{\catindices} \}}{\catvariables} \\
		& =  \sum_{\tildecatindices} \hypercoreat{\tildeindexedcatvariables} \cdot \delta_{(\tildecatindices),(\catindices)} \\
		& = \hypercoreat{\indexedcatvariables} \, ,
	\end{align*}
	where we used that one-hot encodings are orthonormal.
\end{proof}

% Coordinate Calculus
Coordinate calculus is the representation of real-valued functions as tensors, from which its evaluations can be retrieved by the scheme of Theorem~\ref{the:coordinateCalculus}.


%\red{Incorporate theorem: Contracting tensor network conditioned on indices can be done by contracting the conditioned.}


% Retrieval of Coorinates from tensor networks
Tensors of large orders often admit a decomposition by tensor networks.
We in the next theorem show, how such a decomposition can be exploited for efficient contraction and in particular coordinate retrieval.


\begin{theorem}\label{the:slicedContractionToCores}
	Given a tensor network $\tnetof{\graph}$ on a hypergraph $\graph=(\nodes,\edges)$, disjoint subsets $\nodesa,\nodesb\subset\nodes$ and $\catindexofin{\nodesb}$, we have
		\[ \contractionof{\tnetof{\graph}}{\catvariableof{\nodesa},\indexedcatvariableof{\nodesb}} 
		=  \contractionof{\{
			\sbcontractionof{\hypercoreof{\edge}}{\catvariableof{\edge/\nodesb},\indexedcatvariableof{\nodesb}} \, : \, \edge\in\edges
		\}}{\catvariableof{\nodesa}} \, .
		\]
\end{theorem}
\begin{proof}
	Using a delta tensor, which copies basis vector when contracting with one.
\end{proof}

% Special case of retrieving single coordinates
If we retrieve a single coordinate of a tensor, we have the situation $\nodesa=\varnothing$, $\nodesb=\nodes$.
In that case, Theorem~\ref{the:slicedContractionToCores} shows, that the coordinate is the product of the coordinates of the cores. % Thus no contraction required!



\subsection{Differentiation of Contraction}

We add adiitional variables $\seccatvariable$ selecting a coordinate of a tensor, which is varied in a differentiation.

\begin{lemma}\label{lem:difMNprob}
	For any tensor network $\extnet$ with positive $\hypercoreof{\edge}$ we have
	\begin{align*}
		\difwrt{\hypercoreofat{\edge}{\seccatvariableof{\edge}}} \extnetdist
		& = \sbcontractionof{
	 	\identityat{\seccatvariableof{\edge},\edgevariables}, 
		\frac{\contractionof{\extnet}{\edgevariables}}{\hypercoreofat{\edge}{\edgevariables}}, 
		\normationofwrt{\extnet}{\catvariableof{\nodes/\edge}}{\edgevariables} }{\seccatvariableof{\edge},\nodevariables} \\
		& \quad -  \extnetdist \otimes \sbcontractionof{\frac{\contractionof{\extnet}{\seccatvariableof{\edge}}}{\hypercoreofat{\edge}{\seccatvariableof{\edge}}}
		}{\seccatvariableof{\edge}} \, .
	\end{align*}
\end{lemma}
\begin{proof}
	By multilinearity of tensor network contractions we have
	\begin{align*}
		\difwrt{\hypercoreofat{\edge}{\seccatvariableof{\edge}}} \contractionof{\extnet}{\nodevariables}
		& = \contractionof{\{\identityat{\seccatvariableof{\edge},\edgevariables}\}\cup\{\hypercoreofat{\secedge}{\catvariableof{\secedge}} \, : \, \secedge\neq\edge \}}{\seccatvariableof{\edge},\nodevariables}
	\end{align*}
	and thus	
	\begin{align*}
		\difwrt{\hypercoreofat{\edge}{\seccatvariableof{\edge}}} \contraction{\extnet}
		& = \contractionof{\{\identityat{\seccatvariableof{\edge},\edgevariables}\}\cup\{\hypercoreofat{\secedge}{\catvariableof{\secedge}} \, : \, \secedge\neq\edge \}}{\seccatvariableof{\edge}} \, . 
	\end{align*}
		
	Using both we get
	\begin{align}
		\difwrt{\hypercoreofat{\edge}{\seccatvariableof{\edge}}} \extnetdist 
		& = \difwrt{\hypercoreofat{\edge}{\seccatvariableof{\edge}}}  \frac{\contractionof{\extnet}{\nodevariables}}{\contraction{\extnet}} \nonumber \\
		& = \frac{ \difwrt{\hypercoreofat{\edge}{\seccatvariableof{\edge}}} \contractionof{\extnet}{\nodevariables}}{\contraction{\extnet}} 
		- \frac{ \contractionof{\extnet}{\nodevariables} \difwrt{\hypercoreofat{\edge}{\seccatvariableof{\edge}}} \contraction{\extnet} }{(\contraction{\extnet})^2} \nonumber \\
		& = \frac{ \contractionof{\{\identityat{\seccatvariableof{\edge},\edgevariables}\}\cup\{\hypercoreofat{\secedge}{\catvariableof{\secedge}} \, : \, \secedge\neq\edge \}}{\seccatvariableof{\edge},\nodevariables}}{\contraction{\extnet}} \nonumber \\
		& \quad\quad - \extnetdist \cdot  \frac{\contractionof{\{\identityat{\seccatvariableof{\edge},\edgevariables}\}\cup\{\hypercoreofat{\secedge}{\catvariableof{\secedge}} \, : \, \secedge\neq\edge \}}{\seccatvariableof{\edge}}}{\contraction{\extnet}} \label{eq:differentiatingMNpreresult}
		% = \contractionof{\{\identityat{\seccatvariableof{\edge},\edgevariables}\}\cup\{\hypercoreofat{\secedge}{\catvariableof{\secedge}} \, : \, \secedge\neq\edge \}}{\seccatvariableof{\edge},\nodevariables}
	\end{align}
		
	For the first term we get with a normation equation (see Theorem~\ref{the:normationContractionEQ}) that
	\begin{align*}
		\frac{ \contractionof{\{\identityat{\seccatvariableof{\edge},\edgevariables}\}\cup\{\hypercoreofat{\secedge}{\catvariableof{\secedge}} \, : \, \secedge\neq\edge \}}{\seccatvariableof{\edge},\nodevariables}}{\contraction{\extnet}} 	
		&= \frac{\contractionof{\{\identityat{\seccatvariableof{\edge},\edgevariables}\}\cup\{\hypercoreofat{\secedge}{\catvariableof{\secedge}} \, : \, \secedge\in\edges \}}{\seccatvariableof{\edge},\nodevariables}}{\hypercoreofat{\edge}{\edgevariables}  \cdot \contraction{\extnet}} \\
		&= \frac{
		\sbcontractionof{\identityat{\seccatvariableof{\edge},\edgevariables},\extnetdist}{\seccatvariableof{\edge},\nodevariables}
		}{\hypercoreofat{\edge}{\edgevariables}}  \\ 
		&= \frac{\sbcontractionof{\identityat{\seccatvariableof{\edge},\edgevariables},
			\normationof{\extnet}{\edgevariables},
			\normationofwrt{\extnet}{\catvariableof{\nodes/\edge}}{\edgevariables}
			}{\seccatvariableof{\edge},\nodevariables}
		}{\hypercoreofat{\edge}{\edgevariables}}  \, . 
	\end{align*}
	
	Analogously, we have 
	\begin{align*}
		\frac{ \contractionof{\{\identityat{\seccatvariableof{\edge},\edgevariables}\}\cup\{\hypercoreofat{\secedge}{\catvariableof{\secedge}} \, : \, \secedge\neq\edge \}}{\seccatvariableof{\edge}}}{\contraction{\extnet}} 	
		&= \frac{\sbcontractionof{\identityat{\seccatvariableof{\edge},\edgevariables},
			\normationof{\extnet}{\edgevariables}%,
			%\normationofwrt{\extnet}{\catvariableof{\nodes/\edge}}{\edgevariables}
			}{\seccatvariableof{\edge}}
		}{\hypercoreofat{\edge}{\edgevariables}}  \, . 
	\end{align*}
		
	With \eqref{eq:differentiatingMNpreresult}, we arrive at the claim 
	\begin{align*}
		\difwrt{\hypercoreofat{\edge}{\seccatvariableof{\edge}}} \extnetdist
		& = \sbcontractionof{
	 	\identityat{\seccatvariableof{\edge},\edgevariables}, 
		\frac{\contractionof{\extnet}{\edgevariables}}{\hypercoreofat{\edge}{\edgevariables}}, 
		\normationofwrt{\extnet}{\catvariableof{\nodes/\edge}}{\edgevariables} }{\seccatvariableof{\edge},\nodevariables} \\
		& \quad -  \extnetdist \otimes \sbcontractionof{\frac{\contractionof{\extnet}{\seccatvariableof{\edge}}}{\hypercoreofat{\edge}{\seccatvariableof{\edge}}}
		}{\seccatvariableof{\edge}} \, .
	\end{align*}
	
\end{proof}


% Could put it into contraction equations?
\begin{lemma}\label{lem:difMNExpectation}
	%See Proposition 11.9 in Koller Book.
	For any function $\exfunction(\hypercoreof{\edge})[\nodevariables]$ we have
	\begin{align*}
		 \difwrt{\hypercoreofat{\edge}{\seccatvariableof{\edge}}} &
		\sbcontraction{\extnetdist,\exfunction(\hypercoreof{\edge})[\nodevariables]} \\
		= & 
		\frac{\normationof{\extnet}{\indexedcatvariableof{\edge}}}{\hypercoreofat{\edge}{\indexedcatvariableof{\edge}}} 
		\Big( \sbcontraction{\normationofwrt{\extnet}{\catvariableof{\nodes/\edge}}{\indexedcatvariableof{\edge}}, \exfunction(\hypercoreof{\edge})[\nodevariables,\seccatvariableof{\edge}]} \\
		& \quad \quad \quad \quad \quad - \sbcontraction{\extnetdist, \exfunction(\hypercoreof{\edge})[\nodevariables]}
		\Big) \\
		& + \contraction{ \extnetdist
		\difofwrt{\exfunction(\hypercoreof{\edge})[\nodevariables]}{\hypercoreofat{\edge}{\seccatvariableof{\edge}}}
		}
	\end{align*}
\end{lemma}
\begin{proof}
	By product rule of differentiation we have
	\begin{align*}
		\difwrt{\hypercoreofat{\edge}{\indexedcatvariableof{\edge}}} \sbcontraction{\extnetdist,\exfunction(\hypercoreof{\edge})[\nodevariables]} 
		& =  \sbcontraction{\difwrt{\hypercoreofat{\edge}{\indexedcatvariableof{\edge}}}\extnetdist,\exfunction(\hypercoreof{\edge})[\nodevariables]} \\
		& \quad +  \sbcontraction{\extnetdist,\difwrt{\hypercoreofat{\edge}{\indexedcatvariableof{\edge}}}\exfunction(\hypercoreof{\edge})[\nodevariables]}  \, . 
	\end{align*}
	The claim now follows with the application of Lemma~\ref{lem:difMNprob} on the first term.
\end{proof}






\subsection{Selection Encodings}

Selection encodings as introduced in Definition~\ref{def:selectionEncoding} are best understood in terms of linear mapping interpretations of tensors.
We will first provide by basis encodings a generic relation between the coordinatewise tensor definitions in this work and linear maps.

We then show the utility of this perspective in the representation of composed linear functions.
The results are applicable in the exponential family theory, in the tensor representation of energies and means.

\subsubsection{Tensors as linear maps}

The state sets $\facstates$ can be interpreted as an enumeration of basis elements $\onehotmapof{\catindex}$ of the tensor space $\facspace$.

Along this interpretation, tensors have an interpretation as maps between tensor spaces.

\red{Any tensor and any partition of its variables into two sets can be interpreted as the basis elements of a linear map between the tensor spaces of the respective variables.}

Tensor valued functions on state sets $\facstates$ are an intermediate representation.

\begin{definition}
	Let there be two tensor spaces $V_1$ and $V_2$ with basis by sets $\mathcal{M}_1\subset V_1$ and $\mathcal{M}_2\subset V_2$ of cardinality $\catdimof{1}$ and $\catdimof{2}$, which are enumerated by variables $\individualvariableof{1}$ and $\individualvariableof{2}$.
	The basis encoding of a linear map $\exfunction\in\linmapspace(V_1,V_2)$ is the tensor
		\[ \bencodingof{\exfunction}[\individualvariableof{1},\individualvariableof{2}] \in \rr^{\catdimof{1}} \otimes \rr^{\catdimof{2}} \, . \] 
\end{definition}

% Matrices
Basis encodings are standard linear algebra tools, where matrices are understood as linear maps between vector spaces.

\begin{theorem}\label{the:linearCompositionBasisEncoding}
	If $\linmapof{1}$ is a linear function between $V_1$ and $V_2$  and $\linmapof{2}$ between $V_2$ and $V_3$, and let $\individualvariableof{1},\,\individualvariableof{2}$ and $\individualvariableof{3}$ be enumerations of chosen bases in the spaces.
	We have
	\begin{align*}
		\bencodingofat{\linmapof{2}\circ\linmapof{1}}{\individualvariableof{1},\individualvariableof{3}} 
		= \sbcontractionof{
		\bencodingofat{\linmapof{2}}{\individualvariableof{2},\individualvariableof{3}}, \bencodingofat{\linmapof{1}}{\individualvariableof{1},\individualvariableof{2}}
		}{\individualvariableof{1},\individualvariableof{3}}  \, . 
	\end{align*}
\end{theorem}
\begin{proof}
	By basis decompositions and representations of linear maps as contractions.
\end{proof}

% Matrix Multiplication
A typical instance is matrix multiplication, where matrices understood as representations of linear maps.


\subsubsection{Selection encodings}

Selection encodings (see Definition~\ref{def:selectionEncoding}) are related to basis encodings of linear maps as we show in the next theorem.

\begin{theorem}\label{the:selectionToBasisEncoding}
	Given a function 
		\[ \exfunction : \facstates \rightarrow \parspace \]
	we define a linear map $\linmapof{\exfunction}\in\linmapspace(\facspace,\parspace)$ by the basis elements to $\catindexof{1} \in\facstates$ and $\catindexof{2}\in\parstates$ by
	\begin{align*}
	 	\linmapof{\exfunction}(\onehotmapof{\catindexof{1}},\onehotmapof{\catindexof{2}}) 
		= \sbcontraction{\exfunction(\onehotmapof{\catindexof{1}}),\onehotmapof{\catindexof{2}}} \, .  
	\end{align*}
	We then have
	\begin{align*}
		\sencodingof{\exfunction} = \bencodingof{\linmapof{\exfunction}} \, . 
	\end{align*}
\end{theorem}



%\red{Selection encodings are interpretations of matrifications.
%Along that, maps between factored systems are understood as basis decompositions of linear maps between the tensor spaces.}


% Comparison with relational encodings - definition
While relational encoding works for maps from $\facstates$ to arbitrary sets (which are enumerated), selection encodings as introduced in Definition~\ref{def:selectionEncoding} require and exploit that their image is embedded in a tensor space.

% Slicing
Given a selection encoding of a function, the function is retrieved by slicing with respect to the 
	\[ \exfunction(\catindex) = \sencodingofat{\exfunction}{\indexedcatvariableof{},\selvariable} \, . \]
More generally, we show in the next Lemma how to construct to any tensor and any partition of its variables functions by slicing operations, such that the tensor is the selection encoding of the function.

\begin{lemma}\label{lem:inverseSelectionEncoding} % To be used for MLN - proposal distribution
	Let $\hypercoreat{\nodevariables}$ be a tensor in $\bigotimes_{\nodein}\rr^{\catdimof{\node}}$ and let $\nodesa$, $\nodesb$ be a disjoint partition of $\nodes$, that is $\nodesa\dot{\cup}\nodesb=\nodes$.
	Then the function
		\[ \exfunction : \bigtimes_{\node\in\nodesa}[\catdimof{\node}] \rightarrow \bigotimes_{\node\in\nodesb} \rr^{\catdimof{\node}}  \]
	with coordinates
		\[ \exfunction(\catindexof{\nodesa}) = \hypercoreat{\indexedcatvariableof{\nodesa},\catvariableof{\nodesb}}  \]
	obeys
		\[ \sencodingofat{\exfunction}{\nodevariables} = \hypercoreat{\nodevariables} \, . \]
	Here we have renamed the selection variables $\selvariableof{\nodesa}$ by the categorical variables $\catvariableof{\nodesa}$. 
\end{lemma}
\begin{proof}
	From Theorem~\ref{the:linearCompositionBasisEncoding} using the basis encoding equivalence of Theorem~\ref{the:selectionToBasisEncoding}.
\end{proof}


\begin{example}[Markov Logic Networks and Proposal Distributions]
	% Via inverse selection encodings
	While the statistic of MLN (namely $\fselectionmap$) and the proposal distribution (namely $\tranfselectionmap$) have a common selection encoding, both result from the inverse selection encoding described in Lemma~\ref{lem:inverseSelectionEncoding}.
	We can construct $\tranfselectionmap$ by first building the selection encoding to $\fselectionmap$ and then applying the construction of Lemma~\ref{lem:inverseSelectionEncoding} with $\nodesa=\selvariable$ and $\nodesb=\shortcatvariables$.
\end{example}


% Composition
We use selection encodings to express compositions of functions, based on the next theorem.

\begin{theorem}[Selection Encoding for Linear Compositions]\label{the:linCompSelEncoding}
	Let $\sstat$ be a tensor valued function from $\facstates$ to $\simpleparspace$ with image coordinates $\sstat_\statenumerator$ and let $\exfunction$ be a tensor 
	Then
		\[ \left(\sum_{\statenumeratorin}\exfunction[\selvariableof{\sstat}=\statenumerator]\cdot \sstat_\statenumerator \right) [\shortcatvariables] : \facstates \rightarrow \rr \]
	is represented as
		\[ \left(\sum_{\statenumeratorin}\exfunction[\selvariableof{\sstat}=\statenumerator]\cdot \sstat_\statenumerator \right) [\shortcatvariables] 
		 = \sbcontractionof{\sencodingofat{\sstat}{\shortcatvariables,\selvariableof{\sstat}} , \exfunction[\selvariableof{\sstat}]}{\shortcatvariables} \, . \]
\end{theorem}
\begin{proof}
	The representation holds, since for any $\catindexof{[\atomorder]}\in\facstates$ we have
	\begin{align*}
		\sbcontractionof{\sencodingofat{\sstat}{\shortcatvariables,\selvariableof{\sstat}} , \exfunction[\selvariableof{\sstat}]}{\indexedcatvariableof{[\atomorder]}}  
		= \sum_{\statenumeratorin}\exfunction[\selvariableof{\sstat}=\statenumerator]\cdot\sstat_\statenumerator(\catindexof{[\atomorder]}) \, . 
	\end{align*} 
\end{proof}

% Linear 
This theorem shows, that while relation encodings can represent any composition with another function by a contractions, selection encodings can be used to represent linear transforms.
To see this, we interpret $\sstat$ and $\exfunction$ in Theorem~\ref{the:linCompSelEncoding} as basis decompositions of linear maps.


\subsection{Discussion}

Representations of linear maps is the typical application of tensors, reason for refering to tensor networks as multilinear algebra.

    \section{Contraction Equations}

We have observed, that many concepts and theorems in probability theory and logics can be understood as contraction equations.
We first provide a summary of the used contraction equations.

\subsection{Contraction equations in logical and probabilistic reasoning}

Let us summarize the application of contractions and normation in the definition of
\begin{itemize}
	\item Marginal probabilities (\defref{def:marginalProbability}, \theref{the:marginalContraction})
		\[ \probat{\exrandom} = \sbcontractionof{\probtensor}{\exrandom} \]
	\item Conditional probabilities (\defref{def:conditionalProbability}, \theref{the:conditionalContraction})
		\[ \condprobof{\exrandom}{\secexrandom} = \sbnormationofwrt{\probtensor}{\exrandom}{\secexrandom} \]
	\item The partition function of a Markov Networks 
		\begin{align*}
			\partitionfunctionof{\extnet} = \contractionof{\extnet}{\varnothing}
		\end{align*}
	\item The probability distribution of a Markov Network is (\defref{def:markovNetwork})
		\begin{align*}
			\probtensor^{\extnet} = \normationofwrt{\extnet}{\nodes}{\varnothing}
		\end{align*}
\end{itemize}


Further the following properties have been defined by contraction equations:
\begin{itemize}
	\item $\exrandom$ and $\secexrandom$ are independent when (\defref{def:independence}, \theref{the:independenceProductCriterion})
		\[  \sbcontractionof{\probtensor}{\exrandom,\secexrandom} 
		=  \sbcontractionof{\probtensor}{\exrandom} 
			\otimes  \sbcontractionof{\probtensor}{\secexrandom} \]
	\item $\exrandom$ and $\secexrandom$ are called independent conditioned on $\thirdexrandom$ when (\defref{def:condIndependence}, \theref{the:condIndependenceProductCriterion})
		\[ \sbnormationofwrt{\probtensor}{\exrandom,\secexrandom}{\thirdexrandom} 
		= \sbnormationofwrt{\probtensor}{\exrandom}{\thirdexrandom} 
		\otimes \sbnormationofwrt{\probtensor}{\secexrandom}{\thirdexrandom} \]
\end{itemize}





\subsection{Normation Equations}

\begin{theorem}[Normation as a Contraction Equation]\label{the:normationContractionEQ}
	For any on $\innodes$ normable tensor $\hypercoreat{\catvariableof{\nodes}}$, where $\innodes\dot{\cup}\outnodes=\nodes$, we have
	\begin{align*}
		\sbcontractionof{\hypercore}{\catvariableof{\nodes}} 
		= \sbcontractionof{\sbnormationofwrt{\hypercore}{\catvariableof{\outnodes}}{\catvariableof{\innodes}},\sbcontractionof{\hypercore}{\catvariableof{\innodes}}}{\catvariableof{\nodes}} \, . 
	\end{align*}
\end{theorem}
\begin{proof}
	Let us choose indices $\catindexof{\innodes}$ and $\catindexof{\outnodes}$.
	We have that
	\begin{align*}
		%\sbcontractionof{
		\sbnormationofwrt{\hypercore}{\indexedcatvariableof{\innodes}}{\indexedcatvariableof{\outnodes}}
		%}{\indexedcatvariableof{\innodes},\indexedcatvariableof{\outnodes}} 
		= \frac{
			\sbcontractionof{\hypercore}{\indexedcatvariableof{\innodes},\indexedcatvariableof{\outnodes}} 	
		}{
			\sbcontractionof{\hypercore}{\indexedcatvariableof{\innodes}} 	
		} 
	\end{align*}
	and therefor
	\begin{align*}
		\sbcontractionof{\hypercore}{\indexedcatvariableof{\innodes},\indexedcatvariableof{\outnodes}} = 
		\sbnormationofwrt{\hypercore}{\indexedcatvariableof{\innodes}}{\indexedcatvariableof{\outnodes}}
		\cdot 
		\sbcontractionof{\hypercore}{\indexedcatvariableof{\innodes}} 	
	\end{align*}
	Since the equation holds for arbitrary indices, the claim is established.
\end{proof}


\begin{theorem}[Generic Chain Rule]\label{the:genericChainRule}
	For any Tensor $\hypercoreat{\catvariableof{\nodes}}$ and any total order $\prec$ on the nodes $\nodes$ we have % ! CAN DIRECTLY USE [d] when having the order !
	\begin{align*}
		\hypercoreat{\catvariableof{\nodes}} = 
		\contractionof{
			\{ \sbnormationofwrt{\hypercore}{\catvariableof{\node}}{\catvariableof{\prenodes}}  \, : \nodein \}
		}{\catvariableof{\nodes}}
	\end{align*}
\end{theorem}
\begin{proof}
	We apply \theref{the:normationContractionEQ} on the tensor
	\begin{align*}
		\sbnormationofwrt{\hypercore}{
			\catvariableof{\node},\catvariableof{\afternodes}
		}{
			\indexedcatvariableof{\prenodes}
		} \, ,
	\end{align*}
	where $\nodein$ and $\catindexof{\nodes}$ are chosen arbitrarly.
	For any $\nodein$ we get
	\begin{align*}
		\sbnormationofwrt{\hypercore}{
			\catvariableof{\node},\catvariableof{\afternodes}
			}{
			\catvariableof{\prenodes}
		} 
		= \sbcontractionof{
			\normationofwrt{\hypercore}{
				\catvariableof{\afternodes}
				}{
				\catvariableof{\node},\catvariableof{\prenodes}
				},
			\normationofwrt{\hypercore}{
				\catvariableof{\node}
				}{
				\catvariableof{\prenodes}
				}
		}{
			\catvariableof{\nodes} 
		} \, .
	\end{align*}
	Applying this equation iteratively and making use of the commutation of contractions we get for any $\nodein$
	\begin{align*}
		\sbnormationofwrt{\hypercore}{
			\catvariableof{\node},\catvariableof{\afternodes}
		}{
			\catvariableof{\prenodes}
		}
		= \sbcontractionof{
			\normationofwrt{\hypercore}{
				\catvariableof{\secnode}
			}{
				\catvariableof{\{\thirdnode : \thirdnode \prec \secnode, \thirdnode\neq\secnode\}}
			} 
			\, : \node \prec \secnode
		}{
			\catvariableof{\nodes} 
		} \, .
	\end{align*}
	With the maximal node $\node$, that is the $\node$, such that no $\secnode\in\nodes$ with $\node\prec\secnode$ and $\node\neq\secnode$ exists, this is the claim.
\end{proof}




\subsection{Proof of Hammersley-Clifford Theorem}\label{sec:proofHCTheorem}

\red{Different to the standard case, we show a version of Hammersley-Clifford for hypergraphs.}


\begin{lemma}\label{the:contractionFactorization}
	Let $\hypercoreat{\catvariableof{\nodes}}$ be a positive tensor and $\seccatindexof{\nodes}$ an arbitrary index.
	Then we have 
	\begin{align*}
		\hypercoreat{\catvariableof{\nodes}}
		= \sbcontractionof{
			\big(\sbcontractionof{\hypercore}{\catvariableof{\nodes/\thirdnodes}, \catvariableof{\thirdnodes} = \seccatindexof{\thirdnodes}}\big)^{(-1)^{\cardof{\secnodes}-\cardof{\thirdnodes}}} \, : \, \thirdnodes \subset \secnodes \subset \nodes
		}{\catvariableof{\nodes}} \, ,
	\end{align*}
	where the exponentiation is performed coordinatewise and positivity of $\hypercore$ ensures the well-definedness.
\end{lemma}
\begin{proof}
	It suffices to show, that for an arbitrary index $\catindexof{\nodes}$ be an arbitrary index we have
	\begin{align*}
		\hypercoreat{\indexedcatvariableof{\nodes}}
		= \prod_{\secnodes\subset\nodes} \prod_{\thirdnodes\subset\secnodes}
			\big(\sbcontractionof{\hypercore}{\indexedcatvariableof{\nodes/\thirdnodes}, \catvariableof{\thirdnodes} = \seccatindexof{\thirdnodes}}\big)^{(-1)^{\cardof{\secnodes}-\cardof{\thirdnodes}}} \, .
	\end{align*}
	We do this by applying a logarithm on the right hand side and grouping the terms by $\thirdnodes$ as
	\begin{align*}
		%\lnof{\hypercoreat{\indexedcatvariableof{\nodes}}} 
		& \lnof{\prod_{\secnodes\subset\nodes} \prod_{\thirdnodes\subset\secnodes}
			\sbcontractionof{\hypercore}{\indexedcatvariableof{\nodes/\thirdnodes}, \catvariableof{\thirdnodes} = \seccatindexof{\thirdnodes}}\big)^{(-1)^{\cardof{\secnodes}-\cardof{\thirdnodes}}}} \\
		& = \sum_{\thirdnodes\subset\nodes} \lnof{\sbcontractionof{\hypercore}{\indexedcatvariableof{\nodes/\thirdnodes}, \catvariableof{\thirdnodes} = \seccatindexof{\thirdnodes}}} 
		\left( \sum_{\secnodes\subset\nodes \, : \, \thirdnodes\subset \secnodes} (-1)^{\cardof{\secnodes}-\cardof{\thirdnodes}} \right) \\
		& =  \sum_{\thirdnodes\subset\nodes} \lnof{\sbcontractionof{\hypercore}{\indexedcatvariableof{\nodes/\thirdnodes}, \catvariableof{\thirdnodes} = \seccatindexof{\thirdnodes}}} 
		\left( \sum_{i \in [\cardof{\nodes}-\cardof{\thirdnodes}]}  (-1)^{i}  \binom{\cardof{\nodes}-\cardof{\thirdnodes}}{i}  \right) 
	\end{align*}
	Now, by the generic binomial theorem we have that for $n\in\nn, n \neq 0$
		\[ 0 = (1-1)^n = \sum_{i \in [n]}  (-1)^{i}  \binom{n}{i}   \, . \]
	Therefore, the summands for $\thirdnodes\neq\nodes$ vanish and we have 
	\begin{align*}
			& \lnof{ \prod_{\secnodes\subset\nodes} \prod_{\thirdnodes\subset\secnodes}
			\big(\sbcontractionof{\hypercore}{\indexedcatvariableof{\nodes/\thirdnodes}, \catvariableof{\thirdnodes} = \seccatindexof{\thirdnodes}}\big)^{(-1)^{\cardof{\secnodes}-\cardof{\thirdnodes}}} } \\
			& = \lnof{\hypercoreat{\indexedcatvariableof{\nodes}}}
			\left( \sum_{i \in [0]}  (-1)^{i}  \binom{0}{i}  \right) \\
			& = \lnof{\hypercoreat{\indexedcatvariableof{\nodes}}} \, . 
	\end{align*}
	Applying the exponential function on both sides establishes the claim.
\end{proof}


%\begin{lemma}\label{the:condIndMN}
%	Let $\extnet$ be a tensor network of positive cores and $\catindexof{\nodes}$ an arbitrary index.
%	Then we have
%	\begin{align*}
%		\contractionof{\extnet}{\catvariableof{\nodes}} =
%		\prod_{\secnodes\subset\nodes} \prod_{\thirdnodes\subset\secnodes} 
%		\left(\contractionof{\extnet\cup\{\onehotmapof{\catindexof{\nodes/\thirdnodes}}\}}{\catvariableof{\thirdnodes}}\right)^{(-1)^{\cardof{\secnodes}-\cardof{\thirdnodes}}}
%	\end{align*}
%\end{lemma}
%\begin{proof}
%	Show, that any factor $\contractionof{\extnet\cup\{\onehotmapof{\catindexof{\nodes/\thirdnodes}}\}}{\catvariableof{\thirdnodes}}$ with $\thirdnodes\neq\nodes$ is cancelled (by the $(1-1)^n$ binominal theorem).
%\end{proof}


\begin{lemma}\label{lem:independentContractionFactorization}
	Let $\hypercore$ be a positive tensor, $\secnodes\subset\nodes$ and arbitrary subset and $\catindexof{\secnodes}$ an arbitrary index.
	When there are $\nodesa,\nodesb \in\secnodes$, such that
	\begin{align*}
		\sbnormationofwrt{\hypercore}{\catvariableof{\nodesa,\nodesb}}{\catvariableof{\nodes/\{\nodesa,\nodesb\}}}
		= \sbcontractionof{ 
		\sbnormationofwrt{\hypercore}{\catvariableof{\nodesa}}{\catvariableof{\nodes/\{\nodesa,\nodesb\}}},
		\sbnormationofwrt{\hypercore}{\catvariableof{\nodesb}}{\catvariableof{\nodes/\{\nodesa,\nodesb\}}}
		}{\catvariableof{\secnodes}}
	\end{align*}
	then
	\begin{align*}
 \prod_{\thirdnodes\subset\secnodes} 
		\left(\sbcontractionof{\hypercore}{\indexedcatvariableof{\nodes/\thirdnodes}, \catvariableof{\thirdnodes} = \seccatindexof{\thirdnodes}}\right)^{(-1)^{\cardof{\secnodes}-\cardof{\thirdnodes}}} = 1 \, .
	\end{align*}
\end{lemma}
\begin{proof}
	We abbreviate
		\[ Z_{\thirdnodes} = \sbcontractionof{\hypercore}{\indexedcatvariableof{\nodes/\thirdnodes}, \catvariableof{\thirdnodes} = \seccatindexof{\thirdnodes}} \, . 
 \]
	By reorganizing the sum over $\thirdnodes\subset\secnodes$ into  $\thirdnodes\subset\secnodes/\nodesa\cup\nodesb$ we have
	\begin{align}\label{eq:indContFacProof}
	 	\prod_{\thirdnodes\subset\secnodes} 
		\left(
			Z_{\thirdnodes}
		\right)^{(-1)^{\cardof{\secnodes}-\cardof{\thirdnodes}}} = 
		 \prod_{\thirdnodes\subset\secnodes/\{\nodesa,\nodesb\}} 
		 \left( 
		 	\frac{
				Z_{\thirdnodes} \cdot Z_{\thirdnodes\cup\{\nodesa,\nodesb\}}
			}{
				Z_{\thirdnodes\cup\{\nodesa\}} \cdot Z_{\thirdnodes\cup\{\nodesb\}}
			}
		 \right)^{(-1)^{\cardof{\secnodes}-\cardof{\thirdnodes}}} \, . 
	\end{align}
	From the independence assumption it follows that for any index $\catindex$
	\begin{align*}
		& \sbnormationofwrt{\hypercore}{
			 	\indexedcatvariableof{\nodesa} 
			 }{\indexedcatvariableof{\nodes/\thirdnodes\cup\{\nodesa,\nodesb\}},\catvariableof{\thirdnodes}=\seccatindexof{\thirdnodes},  \indexedcatvariableof{\nodesb} } 
			 \\
		& \quad =  
		\sbnormationofwrt{\hypercore}{
			 	\indexedcatvariableof{\nodesa} 
			 }{\indexedcatvariableof{\nodes/\thirdnodes\cup\{\nodesa,\nodesb\}}, \catvariableof{\thirdnodes}=\seccatindexof{\thirdnodes}} \\
		 & \quad  = 
		\sbnormationofwrt{\hypercore}{
			 	\indexedcatvariableof{\nodesa} 
			 }{\indexedcatvariableof{\nodes/\thirdnodes\cup\{\nodesa,\nodesb\}},\catvariableof{\thirdnodes}=\seccatindexof{\thirdnodes},  \catvariableof{\nodesb} = \seccatindexof{\nodesb}} 
	\end{align*}
	Applying this in each squares bracket term of \eqref{eq:indContFacProof} we get
	\begin{align*}
		\frac{
			Z_{\thirdnodes}
		}{
			Z_{\thirdnodes\cup\{\nodesa\}}
		} 
		& = 
		\frac{
			 \sbnormationofwrt{\hypercore}{
			 	\indexedcatvariableof{\nodesa} 
			 }{\indexedcatvariableof{\nodes/\thirdnodes\cup\{\nodesa,\nodesb\}}, \catvariableof{\thirdnodes}=\seccatindexof{\thirdnodes}, \indexedcatvariableof{\nodesb} } 
		}{
			 \sbnormationofwrt{\hypercore}{
			 	\catvariableof{\nodesa} = \seccatindexof{\nodesa}
			 }{\indexedcatvariableof{\nodes/\thirdnodes\cup\{\nodesa,\nodesb\}} , \catvariableof{\thirdnodes}=\seccatindexof{\thirdnodes}, \indexedcatvariableof{\nodesb}} 
		} \\
		& = 
		\frac{
			 \sbnormationofwrt{\hypercore}{
			 	\indexedcatvariableof{\nodesa} 
			 }{\indexedcatvariableof{\nodes/\thirdnodes\cup\{\nodesa,\nodesb\}}, \catvariableof{\thirdnodes}=\seccatindexof{\thirdnodes}, \catvariableof{\nodesb} = \seccatindexof{\nodesb}} 
		}{
			 \sbnormationofwrt{\hypercore}{
			 	\catvariableof{\nodesa} = \seccatindexof{\nodesa}
			 }{\indexedcatvariableof{\nodes/\thirdnodes\cup\{\nodesa,\nodesb\}}, \catvariableof{\thirdnodes}=\seccatindexof{\thirdnodes},\catvariableof{\nodesb} = \seccatindexof{\nodesb}} 
		} \\
		& = 
		\frac{
			Z_{\thirdnodes\cup\{\nodesb\}}
		}{
			Z_{\thirdnodes\cup\{\nodesa,\nodesb\}}
		} \, . 
	\end{align*}
	Thus, each factor in \eqref{eq:indContFacProof} is trivial, which establishes the claim.
\end{proof}

We are finally ready to proof \theref{the:condIndMN} based on the Lemmata above.

\begin{theorem}[\theref{the:condIndMN}]
	Let $\probat{\catvariableof{\nodes}}$ be a probability distribution and $\graph$ a clique-capturing hypergraph, such that for $\nodesa$, $\nodesb$, $\nodesc$ we have that $\catvariableof{\nodesa}$ is independent of $\catvariableof{\nodesb}$ conditioned on $\catvariableof{\nodesc}$, when $\nodesc$ separates $\nodesa$ and $\nodesb$ in the hypergraph.
	Then there is a Markov Network on $\graph$, which distribution is equal to $\probat{\catvariableof{\nodes}}$.
\end{theorem}
\begin{proof}[Proof of \theref{the:condIndMN}]
	By \lemref{the:contractionFactorization} we have for any index $\catindexof{\nodes}$
	\begin{align*}
		\probat{\indexedcatvariableof{\nodes}} =
		\prod_{\secnodes\subset\nodes} \prod_{\thirdnodes\subset\secnodes} 
		\left(
			\probat{\indexedcatvariableof{\thirdnodes},\catvariableof{\nodes/\thirdnodes}=\seccatindexof{\nodes/\thirdnodes}}
		%	\contractionof{\extnet\cup\{\onehotmapof{\catindexof{\nodes/\thirdnodes}}\}}{\catvariableof{\thirdnodes}}
		\right)^{(-1)^{\cardof{\secnodes}-\cardof{\thirdnodes}}}
	\end{align*}
	For any subset $\secnodes\subset\nodes$, which is not contained in a hyperedge, we find $\nodesa,\nodesb \in\secnodes$ such that $\catvariableof{\nodesa}$ is independendent on $\catvariableof{\nodesb}$ conditioned on $\catvariableof{\secnodes/\{\nodesa,\nodesb\}}$.
	If no such nodes $\nodesa,\nodesb \in\secnodes$ exists, $\secnodes$ would be contained in a hyperedge, since the hypergraph is assumed to be clique-capturing.
	By \lemref{lem:independentContractionFactorization} we then have
	\begin{align*}
	 \prod_{\thirdnodes\subset\secnodes} 
		\left(
			\probat{\indexedcatvariableof{\thirdnodes},\catvariableof{\nodes/\thirdnodes}=\seccatindexof{\nodes/\thirdnodes}}
		\right)^{(-1)^{\cardof{\secnodes}-\cardof{\thirdnodes}}} = 1 \, .
	\end{align*}
	We label by a function 
	\begin{align*}
		\alpha: \{\secnodes : \exists\edge\in\edges: \secnodes \subset \edge \} \rightarrow \edges
	\end{align*}	
	the remaining node subsets by a hyperedge containing the subset.
	We build the tensor
	\begin{align*}
		\hypercoreofat{\edge}{\catvariableof{\edge}} = \prod_{\secnodes \, : \, \alpha(\secnodes) = \edge} \prod_{\thirdnodes\subset\secnodes} 
		\left(
			\probat{\indexedcatvariableof{\thirdnodes},\catvariableof{\nodes/\thirdnodes}=\seccatindexof{\nodes/\thirdnodes}}
		\right)^{(-1)^{\cardof{\secnodes}-\cardof{\thirdnodes}}} \, . 
	\end{align*}
	and get, that 
	\begin{align*}
		\probat{\catvariableof{\nodes}} & = \contractionof{\extnetasset}{\catvariableof{\nodes}} \\
		& = \normationofwrt{\extnetasset}{\catvariableof{\node}}{\varnothing} \, .
	\end{align*}
	We have thus constructed a Markov Network with trivial partition function, which contraction coincides with the probability distribution.
\end{proof}





\subsection{Commutation of Contractions}

We show in the next theorem, that a contractions can be performed by contracting a subnetwork first and then further contracting the result with the rest. 

%% OLD Statement
%\begin{theorem}\label{the:splittingContractions}
%	Let $\tnetof{\graph}$ be a tensor network on a hypergraph $\graph=(\nodes,\edges)$.
%	Let us now split the $\graph$ into two graphs $\graph_1=(\nodes_1,\edges_1)$ and $\graph_2=(\nodes_2,\edges_2)$, such that $\edges_1\dot{\cup}\edges_2=\edges$, and let $\secnodes,\secnodes_2\subset\nodes$ be such that $\nodes_2\cap(\nodes_1\cup\secnodes_2) \subset \secnodes$.
%	%$\nodes_1\cup\nodes_2=\nodes$.
%	If $\nodes_2\cup\secnodes \subset \secnodes_2$ then
%		\[ \contractionof{\tnetof{\graph}}{\secnodes} = \contractionof{\tnetof{\graph_1} \cup \{
%			\contractionof{\tnetof{\graph_2}}{\secnodes_2}
%		\}}{\nodes}   \, . \]
%\end{theorem}
%\begin{proof}
%	By an exchange of summations.
%\end{proof}



\begin{theorem}\label{the:splittingContractions}
	Let $\tnetof{\graph}$ be a tensor network on a hypergraph $\graph=(\nodes,\edges)$.
	Let us now split the $\graph$ into two graphs $\graph_1=(\nodes_1,\edges_1)$ and $\graph_2=(\nodes_2,\edges_2)$, such that $\edges_1\dot{\cup}\edges_2=\edges$, $\nodes_1\cup\nodes_2=\nodes$ and all nodes in $\nodes_2$ are contained in an hyperedge of $\edges_2$.
	We then have
		\[ \contractionof{\tnetof{\graph}}{\catvariableof{\secnodes}} 
		= 
		\contractionof{\tnetof{\graph_1} \cup \{
			\contractionof{\tnetof{\graph_2}}{\catvariableof{\nodes_2\cap(\nodes_1\cup\secnodes)}}
		\}}{\catvariableof{\secnodes}}   \, . \]
\end{theorem}
\begin{proof}
	For any index $\catindexof{\secnodes}$ we show that 
			\[ \contractionof{\tnetof{\graph}}{\indexedcatvariableof{\secnodes}} 
		= 
		\contractionof{\tnetof{\graph_1} \cup \{
			\contractionof{\tnetof{\graph_2}}{\catvariableof{\nodes_2\cap(\nodes_1\cup\secnodes)}}
		\}}{\indexedcatvariableof{\secnodes}}   \, . \]
	By definition we have
	\begin{align*}
		\contractionof{\tnetof{\graph}}{\indexedcatvariableof{\secnodes}} 
		& = \sum_{\catindexof{\nodes/\secnodes}} \prod_{\edge\in\edges} \hypercoreofat{\edge}{\indexedcatvariableof{\edge}} \\
		& = \sum_{\catindexof{\nodes/\secnodes}} 
		 	\left( \prod_{\edge\in\edges_1} \hypercoreofat{\edge}{\indexedcatvariableof{\edge}} \right) 
		 	\cdot \left( \prod_{\edge\in\edges_2} \hypercoreofat{\edge}{\indexedcatvariableof{\edge}}  \right) \\
		& =  \sum_{\catindexof{\nodes_1/\secnodes}} \sum_{\catindexof{\nodes_2/(\secnodes\cup\nodes_1)}} 
			\left( \prod_{\edge\in\edges_1} \hypercoreofat{\edge}{\indexedcatvariableof{\edge}} \right) 
		 	\cdot \left( \prod_{\edge\in\edges_2} \hypercoreofat{\edge}{\indexedcatvariableof{\edge}}  \right) \\
		& =  \sum_{\catindexof{\nodes_1/\secnodes}}  
			\left( \prod_{\edge\in\edges_1} \hypercoreofat{\edge}{\indexedcatvariableof{\edge}} \right) 
		 	\cdot \left( \sum_{\catindexof{\nodes_2/(\secnodes\cup\nodes_1)}}  \prod_{\edge\in\edges_2} \hypercoreofat{\edge}{\indexedcatvariableof{\edge}}  \right) \, .
	\end{align*}
	When contracting the variables $\catvariableof{\nodes_2/(\secnodes\cup\nodes_1)}$ on $\tnetof{\graph_2}$, the variables $\catvariableof{\nodes_2\cap(\secnodes\cup\nodes_1)}$ are left open. 
	We therefore have for any $\catindexof{\nodes_2\cap(\secnodes\cup\nodes_1)}$ 
	\begin{align*}
		\sbcontractionof{\tnetof{\graph_2}}{\indexedcatvariableof{\nodes_2\cap(\secnodes\cup\nodes_1)}} =
		 \left( \sum_{\catindexof{\nodes_2/(\secnodes\cup\nodes_1)}}  \prod_{\edge\in\edges_2} \hypercoreofat{\edge}{\indexedcatvariableof{\edge}}  \right) \, . 
	\end{align*}
	It follows with the above, that 
	\begin{align*}
		\contractionof{\tnetof{\graph}}{\indexedcatvariableof{\secnodes}} 
		& =  \sum_{\catindexof{\nodes_1/\secnodes}}  \left( \prod_{\edge\in\edges_1} \hypercoreofat{\edge}{\indexedcatvariableof{\edge}} \right) \cdot \sbcontractionof{\tnetof{\graph_2}}{\indexedcatvariableof{\nodes_2\cap(\secnodes\cup\nodes_1)}} \\
		& = \contractionof{\tnetof{\graph_1} \cup \{
			\contractionof{\tnetof{\graph_2}}{\catvariableof{\nodes_2\cap(\nodes_1\cup\secnodes)}}
		\}}{\indexedcatvariableof{\secnodes}}   \, . 
	\end{align*}
\end{proof}







\subsection{Support of Contractions}\label{sec:supportContractionEquations}



To state the next theorem we introduce the nonzero function $\nonzerofunction: \rr \rightarrow [2]$ by
\begin{align}
	\nonzeroof{x} = \begin{cases}
	0 & \text{if }x=0 \\
	1 & \text{else}
	\end{cases}
\end{align}
Applied coordinatewise on tensors it marks the nonzero coordinates by $1$.

We show that adding binary tensor cores to an contraction orders the results by the partial ordering introduced in \defref{def:partialFTOrder}

\begin{theorem}[Monotonicity of Tensor Contractions]\label{the:monotonicityBinaryContractions}
	Let $\extnet, \secextnet$ be tensor network of non-negative tensors and $\catvariableof{\secnodes}$ an arbitrary set of random variables. %, and $\tilde{\theta}$ another binary tensor. 
	Then we have
		\[ \nonzeroof{\contractionof{\extnet\cup\secextnet}{\catvariableof{\secnodes}}} \prec
		\nonzeroof{\contractionof{\extnet}{\catvariableof{\secnodes}}} \, .  \]
\end{theorem}
\begin{proof}
	It suffices to show that for any $\catindexof{\secnodes}$ with 
		\[ \nonzeroof{\contractionof{\extnet\cup\secextnet}{\indexedcatvariableof{\secnodes}}}=1 \]
	we also have 
		\[ \nonzeroof{\contractionof{\extnet}{\indexedcatvariableof{\secnodes}}}=1 \, . \]
	For any $\catindexof{\secnodes}$ satisfying the first equation we find an extension $\catindexof{\nodes}$ to all variables of the tensor networks such that
		\[ \contractionof{\extnet\cup\secextnet}{\indexedcatvariableof{\nodes}} > 0 \]
	and it follows that
		\[ \contractionof{\extnet}{\indexedcatvariableof{\nodes}} > 0 \quad\text{and}\quad  \contractionof{\secextnet}{\indexedcatvariableof{\nodes}} > 0  \, . \]
	But this already implies, that 
		\[ \nonzeroof{\contractionof{\extnet}{\indexedcatvariableof{\secnodes}}}=1 \, . \]
\end{proof}

Let us now state an equivalence of the contraction, when we add the result of the same contraction 
\begin{theorem}[Invariance under adding subcontractions]\label{the:invarianceAddingSubcontractions}
	Let $\extnet$ be a tensor network of non-negative tensors with variables $\catvariableof{\nodes}$ and let $\secextnet$ be a subset.
	Then we have for any subset $\catvariableof{\secnodes}$ of $\catvariableof{\nodes}$
		\[ \contractionof{\extnet \cup\{
			\nonzeroof{
			\contractionof{\secextnet}{\catvariableof{\secnodes}}
			}
		\}}{\catvariableof{\nodes}} 
		= \contractionof{\extnet}{\catvariableof{\nodes}}
		\, . \]
	
	%For any sets of leg variables $\randomxof{V},\randomxof{\tilde{V}}$ appearing in $\theta$  we have
	%Then we have
	%	\[ \contractionof{\theta\cup
	%	\nonzeroof{\contractionof{\tilde{\theta}}{\randomxof{\tilde{V}}}}
	%			}{\randomxof{V}} = \contractionof{\theta}{\randomxof{V}} \, . \]
\end{theorem}
\begin{proof}
	For any $\catindexof{\nodes}$ with 
		\[ \contractionof{\extnet}{\indexedcatvariableof{\nodes}} = 0 \]
	we also have 
		\[ \contractionof{\extnet \cup\{
			\nonzeroof{
			\contractionof{\secextnet}{\catvariableof{\secnodes}}
			}
		\}}{\indexedcatvariableof{\nodes}} = 0 \, . \]
	For any $\catindexof{\nodes}$ with 
		\[ \contractionof{\extnet}{\indexedcatvariableof{\nodes}} \neq 0 \]
	we have for the reduction $\catindexof{\secnodes}$ of the index $\catindexof{\nodes}$ that
		\[  \contractionof{\secextnet}{\indexedcatvariableof{\secnodes}} \neq 0 \]
	and thus
	\begin{align*}
		\contractionof{\extnet \cup\{
			\nonzeroof{
			\contractionof{\secextnet}{\catvariableof{\secnodes}}
			}
		\}}{\indexedcatvariableof{\nodes}} 
		= \contractionof{\extnet}{\indexedcatvariableof{\nodes}} \cdot \nonzeroof{
			\contractionof{\secextnet}{\catvariableof{\secnodes}}
			}[\indexedcatvariableof{\secnodes}]
		= \contractionof{\extnet}{\indexedcatvariableof{\nodes}} \, . 
	\end{align*}
%	When the subcore transformed by $\nonzeroof{\cdot}$ contains a zero slice, then this
%	 zero slice is also appearing in the rest contraction.
%	Multiplying a zero slice with zero does not affect the contraction, neither does multiplication with one on any slice.
\end{proof}





\begin{remark}
	Similar statements hold, when dropping the non-negativity assumption on the, but demanding that all variables are left open.
\end{remark}




















 % To be integrated in Coordinate Calculus!
    \chapter{\chatextbasisCalculus}\label{cha:basisCalculus}

\red{
Basis Calculus stores informations in the selection of basis elements, while coordinate calculus uses the coordinates to each index for storage.
While coordinate calculus is more expressive, basis calculus can be exploited in sparse representations of composed functions.
}

\sect{Classification of tensors}

\red{
    We in this chapter investigate the properties of tensors, which where arising in probabilistic and logical reasoning.
    We observed already before, that
}
\begin{itemize}
    \item Conditional probability tensors are directed tensors.
    \item Logical formulas are boolean tensors.
\end{itemize}

\red{
    Thus, the set of tensors, which are both directed and boolean is of much interest.
    We will show in this chapter, that they are equal to the set of relational encodings of functions.
}

\begin{figure}[h]
	\begin{center}
		\begin{tikzpicture}[yscale=0.6]
	\draw[dashed] (-10.5,12) rectangle (5.5,2);
	\node[anchor=center] (text) at (-2.5,11) {Tensors with non-negative coordinates};
	
	\draw[red] (-10,10) rectangle (2.5,5); 
	\node[anchor=center,red] (text) at (-5,9) {Directed Tensors: Conditional probability distributions};
	\draw[blue] (-7.5,7.5) rectangle (5,2.5); 
	\node[anchor=center,blue] (text) at (0,3.5) {Boolean Tensors: Encoding of subsets (see \defref{def:subsetEncoding}) and relations (see \defref{def:daryRelation})};

	\node[anchor=center] (text) at (-2.5,6.5) {Directed and Boolean Tensors: Encoding of functions (see \theref{the:rencodingDirected})};
\end{tikzpicture}
	\end{center}
	\caption{Sketch of the tensors with non-negative coordinates.
	We investigate in this chapter tensors, which are directed and boolean.}
\end{figure}

\sect{Encoding of Subsets and Relations}

Based on the concept of one-hot encodings of states we in this chapter develop the construction of encodings to sets, relations and functions.
We start with the definition of subset encodings, which represent set memberships in their boolean coordinates.

\begin{definition}[Subset Encoding]\label{def:subsetEncoding}
	We say that an arbitrary set $\arbset$ is enumerated by an enumeration variable $\indvariableof{\arbset}$ taking values in $[\inddimof{\arbset}]$, when $\inddimof{\arbset}=\absof{\arbset}$ and there is a bijective function
	\begin{align*}
		\indexinterpretation : [\inddimof{\arbset}] \rightarrow \arbset \, .
	\end{align*}
	Given an set $\arbset$ enumerated by the variable $\indvariableof{\arbset}$, any subset $\arbsubset\subset\arbset$ is encoded by the tensor $\onehotmapto{\arbsubset}[\indvariable]$ defined for $\indindex\in[\absof{\arbset}]$ as
	\begin{align*}
	 	\onehotmapofat{\arbsubset}{\indexedindvariable}
		= \begin{cases}
		1 & \text{if} \indexinterpretationat{\indindex} \in \arbsubset \\
		0 & \text{else}
		\end{cases} \, . 
	\end{align*}
\end{definition}

% Decomposition
In a one-hot basis decomposition we have
\begin{align*}
	\onehotmapofat{\arbsubset}{\indvariable}
	\coloneqq \sum_{\indindex\in[\cardof{\arbset}]\,:\,\indexinterpretationat{\indindex}\in\arbsubset}\onehotmapofat{\indindex}{\indvariable} \, .
\end{align*}

% Explanation
%Encoding of subsets as vectors: Each coordinate associated with a possible element, $\{0,1\}$ encoding whether in subset.
%The encodings is thus a boolean tensor.
%Any subset encoding is a boolean tensor.

% Relation
Since relations are subsets of cartesian products between two sets, their encoding is a straightforward generalization of \defref{def:subsetEncoding}.

\begin{definition}[Relation Encoding]
	A relation between two finite sets $\inset$ and $\outset$ is a subset of their cartesian product
	\begin{align*}
		 \exrelation \subset \inset \times \outset \, .
	\end{align*}
	Given an enumeration of $\inset$ and $\outset$ by the categorical variables $\indvariableof{\insymbol}$ and $\indvariableof{\outsymbol}$ and interpretation maps $\indexinterpretationof{\insymbol}$, $\indexinterpretationof{\outsymbol}$, we define the encoding of this subset as the tensor $\onehotmapto{\exrelation}[\indvariableof{\insymbol},\indvariableof{\outsymbol}]$ with the coordinates
	\begin{align*}
		\onehotmapofat{\exrelation}{\indexedindvariableof{\insymbol},\indexedindvariableof{\outsymbol}}
		= \begin{cases}
		1 & \text{if } (\indexinterpretationofat{\insymbol}{\indindexof{\insymbol}},\indexinterpretationofat{\outsymbol}{\indindexof{\outsymbol}}) \in \exrelation \\
		0 & \text{else}
		\end{cases} \, . 
	\end{align*}
\end{definition}

% Decomposition
The relation encoding has a decomposition into one-hot encodings as
\begin{align*}
	\onehotmapofat{\exrelation}{\indvariableof{\insymbol},\indvariableof{\outsymbol}}
	= \sum_{\indindexof{\insymbol},\indindexof{\outsymbol} \, : \, (\indexinterpretationofat{\insymbol}{\indindexof{\insymbol}},\indexinterpretationofat{\outsymbol}{\indindexof{\outsymbol}}) \in \exrelation}
	\onehotmapofat{\indindexof{\insymbol}}{\indvariableof{\insymbol}}  \otimes \onehotmapofat{\indindexof{\outsymbol}}{\indvariableof{\outsymbol}}  \, .
\end{align*}

Relational encodings have a matrix structure by the cartesian product, which can be further folded to tensors, when the sets itself are cartesian products.
The relational encoding is a bijection between the relations of two sets and the boolean tensors with their enumeration variables.

%They provide representations of generic relations by boolean tensors, in the sense that each relation between two sets is represented
%\begin{theorem}
%	The relational encoding is a bijection between the set of relations and the set of boolean tensors.
%\end{theorem}
%\begin{proof}
%	% =>
%	By definition, a relational encoding is the encoding of a subset and thus a boolean tensor.
%	% <=
%	Any matrification of a boolean tensor marks by its $1$ coordinates the elements of a relation.
%\end{proof}
%
%% Significance
%We can thus understand any matrification of a boolean tensor as the encoding of a relation and vice versa.



\subsect{Higher order relations}

We can extend this contraction to relations of higher order, and arrive at encoding schemes usable for relational databases.

\begin{definition}\label{def:daryRelation}
	Given sets $\arbsetof{\atomenumerator}$ for $\atomenumeratorin$, a $\atomorder$-ary relation is a subset of a their cartesian product, that is
	\begin{align*}
		\exrelation \subset\bigtimes_{\atomenumeratorin} \arbsetof{\atomenumerator} \, .
	\end{align*}
	Given an enumeration of each set $\arbsetof{\atomenumerator}$ by a variable $\indvariableof{\atomenumerator}$ and an interpretation map $\indexinterpretationof{\atomenumerator}$, we define the encoding of the relation as the tensor $\onehotmapto{\exrelation}[\indvariableof{[\atomorder]}]$ with coordinates
	\begin{align*}
		\onehotmapofat{\exrelation}{\indexedindvariableof{[\catorder]}}
		= \begin{cases}
		1 & \text{if} \quad (\indexinterpretationofat{0}{\indindexof{0}},\ldots,\indexinterpretationofat{\atomorder-1}{\indindexof{\atomorder-1}}) \in \exrelation \\
		0 & \text{else}
		\end{cases} \, .
	\end{align*}
\end{definition}

\begin{example}[Propositional Formulas]
	Let there be for $\atomenumeratorin$ sets $\arbsetof{\atomenumerator}$ of truth assignments to the $\atomenumerator$-th atom, which are all enumerated by $[2]$.
	A propositional formula then corresponds with a $\atomorder$-ary relation and we directly defined them in \defref{def:formulas} by their relational encoding.
\end{example}

\begin{example}[Relational Databases]
	Relational Databases can be encoded as tensors using the relation encoding scheme.
	Each column is thereby understood as an eunumeration variable, which values form the sets $\arbsetof{\catenumerator}$.
\end{example}

% Sparse Representations
Let us notice, that the dimensionality of the tensor space used for representing a relation is
\begin{align*}
	\prod_{\catenumeratorin} \cardof{\arbsetof{\catenumerator}}
\end{align*}
and therefore growing exponentially with the number of variables.
Relations are however often sparse, in the sense that
\begin{align*}
	 \cardof{\exrelation} << \prod_{\catenumeratorin} \cardof{\arbsetof{\catenumerator}} \, .
\end{align*}
It is therefore often benefitially to choose sparse encoding schemes, for example by restricted CP formats (see \charef{cha:sparseCalculus}) to represent $\onehotmapof{\exrelation}$.


\sect{Encoding of Functions}

Let us now restrict to relations, which have an expression by functions.
We in this section then show, how contractions of their encodings can be exploited in function evaluation.

\subsect{Relational Encoding of Functions}

%We now generalize the representation scheme towards maps between arbitrary unstructured sets.

\begin{definition}[Relational Encoding of Maps]\label{def:functionRelationEncoding}
	Any map
	\begin{align*}
		\exfunction : \inset \rightarrow \outset
	\end{align*}
	can be represented by a relation
	\begin{align*}
		\exrelationof{\exfunction} \coloneqq \left\{ (x,\exfunction(x) \, : \, x \in\inset )\right\} \subset \inset \times \outset \, .
	\end{align*}
	Given a enumeration of the sets by $\indvariableof{\insymbol}$ and $\indvariableof{\outsymbol}$ we define the relational encoding of $\exfunction$ as the tensor
	\begin{align*}
		\rencodingofat{\exfunction}{\indvariableof{\insymbol},\indvariableof{\outsymbol}}
		= \onehotmapofat{\exrelationof{\exfunction}}{\indvariableof{\insymbol},\indvariableof{\outsymbol}}  \, .
	\end{align*}
\end{definition}

\begin{remark}[Reduction to images]
	% Image enumeration
	When $\exfunction$ maps into a set of infinite cardinality, we restrict $\outset$ to the image of $\exfunction$ and enumerate the image by a variable $\indvariableof{\exfunction}$.
	This scheme is applied, when $\exfunction$ is itself a tensor, i.e. $\outset=\rr$.
	While the variable $\indvariableof{\exfunction}$ can in general be of the same cardinality as the domain set $\inset$, it will be valued in $[2]$ when considering boolean tensors.
\end{remark}

% Characterization of the directed and boolean tensors
We notice, that any relational representation of a function is also a directed tensor with incoming variables to the domain and outgoing variables to the image.
It furthermore holds, that the set of directed and boolean tensors is characterized by the relational encoding of functions.
This is shown in the next theorem, by the claim that any boolean tensor which is directed is the relational representation of a function.

\begin{theorem}\label{the:rencodingDirected}
	Let $\inset,\outset$ be sets and $\exrelation\subset\inset\times\outset$ a relation.
	If and only if there exists a map $\exfunction:\inset\rightarrow\outset$ such that $\exrelation=\exrelationof{\exfunction}$, the relational encoding $\rencodingof{\exfunction}$ is a directed tensor with $\indvariableof{\insymbol}$ incoming and $\indvariableof{\outsymbol}$ outgoing.
\end{theorem}
\begin{proof}
	\proofrightsymbol:
	When $\exfunction$ is a function, we have for any $\indindexofin{\insymbol}$
	\begin{align*}
		\sum_{\indindexofin{\outsymbol}} \rencodingofat{\exfunction}{\indexedindvariableof{\insymbol},\indexedindvariableof{\outsymbol}}
		=  \rencodingofat{\exfunction}{\indexedindvariableof{\insymbol},\indvariableof{\outsymbol}=\invindexinterpretationofat{\outsymbol}{\exfunctionat{\indexinterpretationofat{\insymbol}{\indindexof{\insymbol}}}}}
		= 1 \, .
	\end{align*}
	Thus, $\rencodingofat{\exfunction}{\indvariableof{\outsymbol},\indvariableof{\insymbol}}$ is a directed tensor with variables $\indvariableof{\insymbol}$ incoming and $\indvariableof{\outsymbol}$ outgoing.

	\proofleftsymbol:
	Conversely let there be a relation $\exrelation$, such that $\rencodingof{\exrelation}$ is directed.
	To this end, we observe that for any $\indindexofin{\insymbol}$ the tensor
	\begin{align*}
		\onehotmapofat{\exrelation}{\indexedindvariableof{\insymbol},\indvariableof{\outsymbol}}
	\end{align*}
	is a boolean tensor with coordinate sum one and therefore a basis vector.
	It follows that the function $\exfunction : \inset \rightarrow \outset $ defined for $x\in\inset$ as
	\begin{align*}
		\exfunctionat{x}
		= \indexinterpretationofat{\outsymbol}{\invonehotmapof{\onehotmapofat{\exrelation}{\indvariableof{\insymbol}=\indexinterpretationofat{\insymbol}{x},\indvariableof{\outsymbol}}}}
	\end{align*}
	is well-defined.
	We then have by construction
	\begin{align*}
		\rencodingofat{\exfunction}{\indvariableof{\outsymbol},\indvariableof{\insymbol}}
		& = \sum_{\indindexofin{\insymbol}}
		\onehotmapofat{\exfunction(\indindexof{\insymbol})}{\indvariableof{\outsymbol}} \otimes
		\onehotmapofat{\indindexof{\insymbol}}{\indvariableof{\insymbol}} \\
		& =  \sum_{\indindexofin{\insymbol}} \onehotmapofat{\exrelation}{\indexedindvariableof{\insymbol},\indvariableof{\outsymbol}} \otimes
		\onehotmapofat{\indindexof{\insymbol}}{\indvariableof{\insymbol}} \\
		& = \onehotmapofat{\exrelation}{\indvariableof{\outsymbol},\indvariableof{\insymbol}}
	\end{align*}
	and therefore by \defref{def:functionRelationEncoding} $\exrelation=\exrelationof{\exfunction}$.
\end{proof}

% Grid sets
We are specially interested in sets of states of a factored system, which amounts to the case in \defref{def:functionRepresentation}.
Those state sets have a decomposition into a cartesian product of $\atomorder$ sets
	\[ \arbset = \facstates \, . \]
The most obvious enumeration of the set $\arbset$ is therefore by the collection of state variables $\{\catvariableof{\atomenumerator} \, : \, \atomenumeratorin \}$.
Functions between states of factored systems with $\atomorder_{\insymbol}$ and $\atomorder_{\outsymbol}$ state variables can be represented by $\atomorder_{\insymbol}+\atomorder_{\outsymbol}$-ary relations and \defref{def:functionRelationEncoding} has an obvious generalization to this case with multiple enumeration variables.

%% NOT NEEDED -> Done in propositional logics
%% Conditional
%Since the relational encoding of any map between factored systems is directed, it can be interpreted by a conditional probability tensor, as we state next.
%
%%% Maps
%\begin{corollary}%\label{the:condProbFunctionRepresentation}
%	The relational encoding $\rencodingof{\exfunction}$ (see \defref{def:functionRepresentation}) of a function $\exfunction$ between factored systems is a conditional probability tensor, where the legs to the image system are the conditions and the legs to the target system the distribution legs.
%\end{corollary}
%
%%% Deterministic by construction
%These are deterministic conditional probability tensors, in the sense that any slice with respect to the input variables is a basis tensor.
%Through contractions with distribution tensors (e.g. distributions in domain systems) they get stochastic.
%This is for example the case in the empirical distribution, which can be understood as the forwarding of the uniform distribution on the sample enumeration.

\subsect{Function Evaluation}

We now justify the nomenclature of basis calculus, by showing that contraction with basis elements produce the one-hot encoded function evaluation.

\begin{theorem}[Basis Calculus]\label{the:basisCalculus}
	Retrieving the value of the function $\exfunction$ at a specific state is then the contraction of the tensor representation with the one-hot encoded state.
	For any $\arbelement\in\inset$ we have
	\begin{align*}
		\onehotmapofat{\invindexinterpretationofat{\outsymbol}{\exfunctionat{\arbelement}}}{\indvariableof{\outsymbol}}
		= \contractionof{
			\rencodingofat{\exformula}{\indvariableof{\outsymbol},\indvariableof{\insymbol}},
			\onehotmapofat{\indexinterpretationofat{\insymbol}{\arbelement}}{\indvariableof{\insymbol}}
		}{\indvariableof{\outsymbol}} \, .
	\end{align*}
	Thus, we can retrieve the function evaluation by the inverse one-hot mapping as
	\begin{align*}
		\exfunctionat{\arbelement} = \invonehotmapof{\contractionof{
			\rencodingofat{\exformula}{\indvariableof{\outsymbol},\indvariableof{\insymbol}},
			\onehotmapofat{\indexinterpretationofat{\insymbol}{\arbelement}}{\indvariableof{\insymbol}}
		}{\indvariableof{\outsymbol}}} \, .
\end{align*}
\end{theorem}
\begin{proof}
	From the representation
	\begin{align*}
		\rencodingofat{\exfunction}{\indvariableof{\outsymbol},\indvariableof{\insymbol}}
		& =  \sum_{\indindexofin{\insymbol}}
			\onehotmapofat{(\invindexinterpretationof{\outsymbol} \circ \exfunction \circ \indexinterpretationof{\insymbol}) \indindexof{\insymbol}
				}{\indvariableof{\insymbol}}
			\otimes
			\onehotmapofat{\indindexof{\insymbol}}{\indvariableof{\insymbol}}
	\end{align*}
	and the orthonormality of the one-hot encodings of the input enumeration we get
	\begin{align*}
		 \contractionof{
			\rencodingofat{\exformula}{\indvariableof{\outsymbol},\indvariableof{\insymbol}},
			\onehotmapofat{\indexinterpretationofat{\insymbol}{\arbelement}}{\indvariableof{\insymbol}}
		}{\indvariableof{\outsymbol}}
		= \onehotmapofat{\invindexinterpretationofat{\outsymbol}{\exfunctionat{\arbelement}}}{\indvariableof{\outsymbol}} \, .
	\end{align*}
\end{proof}

%% Usage: Basis Calculus
We can thus use tensor contractions to calculate the values of functions.
Since basis vectors being the one-hot encoding of the domain system are mapped to basis vectors being the encoding of the image system, we call these contraction basis calculus.



\sect{Calculus of relational encodings}

We now show the utility of relational encodings for functions, by developing tensor network representation to composed functions.
\red{We in this section use the notation of factored system representation, as developed in \parref{par:one} and enumerate states of factored systems by variables $\catvariable$ with states in $[\catdim]$, instead of combinations of variables $\indvariable$ with index interpretation functions $\indexinterpretation$ enumerating arbitrary sets.}

\subsect{Composition of function}

We have already used (see \theref{the:formulaDecomposition}), that combination of propositional formulas by connectives can be represented by contractions.
We now show in a more general perspective, that in basis calculus, any composition of functions in its relational encoding the contraction of the encoded functions.

\begin{theorem}[Composition of Functions]\label{the:compositionByContraction}
	Let there be two maps between factored systems
	\begin{align*}
		\exfunction : \nodestatesof{\nodesone} \rightarrow \nodestatesof{\nodestwo}
	\end{align*}
	and
	\begin{align*}
		\secexfunction : \nodestatesof{\nodestwo} \rightarrow \nodestatesof{\nodesthree}
	\end{align*}
	with the image system of $\exfunction$ is the domain system of $\secexfunction$.
	Then the relational encoding of the composition
	\begin{align*}
		\compositionof{\secexfunction}{\exfunction} : \nodestatesof{\nodesone} \rightarrow \nodestatesof{\nodesthree}
	\end{align*}
	is the contraction
	\begin{align*}
		\rencodingofat{\compositionof{\secexfunction}{\exfunction}}{\catvariableof{\nodesthree},\catvariableof{\nodesone}}
		= \contractionof{
			\rencodingofat{\secexfunction}{\catvariableof{\nodesthree},\catvariableof{\nodestwo}},
			\rencodingofat{\exfunction}{\catvariableof{\nodestwo},\catvariableof{\nodesone}},
		}{\catvariableof{\nodesthree},\catvariableof{\nodesone}} \, .
	\end{align*}
\end{theorem}
\begin{proof}
	By definition we have the relational encoding of the composition as
	\begin{align*}
		\rencodingofat{\compositionof{\secexfunction}{\exfunction}}{\catvariableof{\nodesthree},\catvariableof{\nodesone}}
		= \sum_{\catindexof{\nodesone}\in\nodestatesof{\nodesone}}
		\onehotmapofat{\compositionofat{\secexfunction}{\exfunction}{\catindexof{\nodesone}}}{\catvariableof{\nodesthree}} \otimes
		\onehotmapofat{\catindexof{\nodesone}}{\catvariableof{\nodesone}}  \, .
	\end{align*}
	By using a similar representation for $\rencodingof{\secexfunction}$ and $\rencodingof{\exfunction}$ we now show, that this coincides with the contraction of these relational encodings with closed variables $\catvariableof{\nodestwo}$.
	By the linearity of the contraction operation we get
	\begin{align*}
		\contractionof{\rencodingof{\exfunction},\rencodingof{\secexfunction}}{\catvariableof{\nodesthree},\catvariableof{\nodesone}}
		& = \sum_{\catindexof{\nodesone}\in\bigtimes_{\node\in\nodesone}[\catdimof{\node}]}
			\sum_{\catindexof{\nodestwo} \in \bigtimes_{\node\in\nodestwo}[\catdimof{\node}]}
			\breakablecontractionof{
				\left( \onehotmapofat{\secexfunctionat{\catindexof{\nodestwo}}}{\catvariableof{\nodesthree}} \otimes
				\onehotmapofat{\catindexof{\nodestwo}}{\catvariableof{\nodestwo}} \right), \\
				& \hspace{4.5cm} \left( \onehotmapofat{\exfunctionat{\catindexof{\nodesone}}}{\catvariableof{\nodestwo}} \otimes
				\onehotmapofat{\catindexof{\nodesone}}{\catvariableof{\nodesone}} \right)
			}{\catvariableof{\nodesthree},\catvariableof{\nodesone}} \\
		& = \sum_{\catindexof{\nodesone}\in\bigtimes_{\node\in\nodesone}[\catdimof{\node}]}
			\delta_{\catindexof{\nodestwo},\catindexof{\nodesone}} \, \cdot \,
			\onehotmapofat{\secexfunctionat{\catindexof{\nodestwo}}}{\catvariableof{\nodesthree}} \otimes
			\onehotmapofat{\catindexof{\nodesone}}{\catvariableof{\nodesone}} \\
		& = \sum_{\catindexof{\nodesone}\in\nodestatesof{\nodesone}}
		\onehotmapofat{\compositionofat{\secexfunction}{\exfunction}{\catindexof{\nodesone}}}{\catvariableof{\nodesthree}} \otimes
		\onehotmapofat{\catindexof{\nodesone}}{\catvariableof{\nodesone}} \\
		& = \rencodingofat{\compositionof{\secexfunction}{\exfunction}}{\catvariableof{\nodesthree},\catvariableof{\nodesone}} \, ,
	\end{align*}
	where we exploited the orthonormality of the one-hot encodings to the states of $\catvariableof{\nodestwo}$, which contraction thus results in the delta symbol $\delta$ applied on the respective states.
\end{proof}

% Iterative usage
We can use \theref{the:compositionByContraction} iteratively to further decompose the function $\secexfunction$.
In this way, the relational encoding of a function consistent of multiple compositions can be represented as the contractions of all the functions.
This has been applied in \theref{the:formulaDecomposition} to efficiently represent propositional formulas, for which syntactical expressions are given.

\subsect{Compositions with real functions}

We here investigate how the composition of a tensor
\begin{align*}
	\hypercore : \facstates \rightarrow \rr
\end{align*}
with arbitrary functions
\begin{align*}
	\chainingfunction: \rr \rightarrow \rr
\end{align*}
can be represented.
This is for example relevant, when representing coordinatewise tensor transforms (see \secref{sec:coordinatewiseTransforms}) based on tensor network contractions.
% Strategy
%Our main strategy is in understanding the tensor $\hypercore$ as a map to its finite image, seen as the enumerated states of a categorical variable building a factored system.
To this end we understand the tensor $\hypercoreat{\shortcatvariables}$ as a map of the states $\facstates$ onto its by a variable $\indvariableof{\hypercore}$ and index interpretation $\indexinterpretation$ enumerated image $\imageof{\hypercore}$.
We then define the restriction of $\chainingfunction$ onto $\imageof{\hypercore}$ as the tensor $\restrictionofto{\chainingfunction}{\imageof{\hypercore}}\left[\indvariableof{\hypercore}\right]$ with coordinates $\indindexof{\hypercore}$
\begin{align*}
	\restrictionofto{\chainingfunction}{\imageof{\hypercore}}\left[\indexedindvariableof{\hypercore}\right]
	= \compositionofat{\chainingfunction}{\indexinterpretation}{\indindexof{\hypercore}} \, .
\end{align*}

%By $\restrictionofto{\chainingfunction}{\mathcal{M}}$ we further denote the restriction of a real function $\chainingfunction$ to an enumerated set $\mathcal{M} =\{x_i \, : \, i \in [\cardof{\imageof{\hypercore}}]\} \subset \rr$, i.e. the vector
%	\[ \restrictionofto{\chainingfunction}{\mathcal{M}} : [\cardof{\mathcal{M}}] \rightarrow \rr \]
%defined for $i \in [\cardof{\mathcal{M}}]$ as
%	\[ \restrictionofto{\chainingfunction}{\mathcal{M}}(i) = \chainingfunction(x_i) \, . \]


\begin{theorem}\label{the:tensorFunctionComposition}
	The coordinatewise transform of any tensor $\hypercore$ (see \defref{def:coordinatewiseTransform}) by a real function $\chainingfunction$ is the contraction (see \figref{fig:tensorFunctionComposition})
	\begin{align*}
		\chainingfunction(\hypercore)[\shortcatvariables]
		= \contractionof{\rencodingofat{\hypercore}{\indvariableof{\hypercore},\shortcatvariables},\restrictionofto{\chainingfunction}{\imageof{\hypercore}}\left[\indvariableof{\hypercore}\right] }{\shortcatvariables} \, .
	\end{align*}
\end{theorem}
\begin{proof}
	By the basis calculus \theref{the:basisCalculus} we have for any state $\shortcatindices\in\facstates$, that
	\begin{align*}
		\contractionof{\rencodingofat{\hypercore}{\indvariableof{\hypercore},\shortcatvariables},\restrictionofto{\chainingfunction}{\imageof{\hypercore}}\left[\indvariableof{\hypercore}\right]}{\indexedshortcatvariables}
		&= \contraction{\rencodingofat{\hypercore}{\indvariableof{\hypercore},\indexedshortcatvariables},\restrictionofto{\chainingfunction}{\imageof{\hypercore}}\left[\indvariableof{\hypercore}\right]} \\
		& = \contraction{\onehotmapofat{\indexinterpretationof{\hypercoreat{\indexedshortcatvariables}}}{\indvariableof{\hypercore}},\restrictionofto{\chainingfunction}{\imageof{\hypercore}}\left[\indvariableof{\hypercore}\right]} \\
		& = \chainingfunction(\hypercore)[\indexedshortcatvariables] \, .
	\end{align*}
	Since both tensors coincide on all coordinates, they are equal.
\end{proof}

\begin{figure}[h]
\begin{center}
	\begin{tikzpicture}[scale=0.35] % , baseline = -3.5pt




\begin{scope}[shift={(-15,0)}]

\drawatomcore{-6}{-8}{$\chainingfunction\circ\hypercore$}


	\begin{scope}[shift={(-6,-12)}]
		\draw[] (0,1)--(0,-1) node[midway,left] {\tiny $\catvariableof{0}$}; 
		\draw[] (1.5,1)--(1.5,-1) node[midway,left] {\tiny $\catvariableof{1}$}; 
		\node[anchor=center] (text) at (3,0) {$\cdots$};
		\draw[] (4,1)--(4,-1) node[midway,right] {\tiny $\catvariableof{\atomorder\shortminus1}$}; 
	\end{scope}

\end{scope}



%\node[anchor=center] (text) at (-14.25,-10) {${=}$};



%\begin{scope}[shift={(-12.5,0)}]
%
%\node[anchor=center] (text) at (1,-7.25) {\small $\chainingfunction$};
%\draw (5.5,-7.25) ellipse (6 and 4.5);
%
%
%\drawatomcore{3.5}{-8}{$\rencodingof{\exformula}$}
%\drawatomindices{3.5}{-12}	
%\draw[] (5.5,-9)--(5.5,-7) node[midway,right] {\tiny $\randomxof{\hypercore}$};
%
%
%\draw (4.75,-3.5) rectangle (6.25,-7);
%\node[anchor=center] (text) at (5.5,-5.25) {$\begin{bmatrix} 
%0 \\
%1
%\end{bmatrix}$};
%
%\end{scope}


\begin{scope}[shift={(-12.5,0)}]

\node[anchor=center] (text) at (-0.5,-10) {${=}$};

%\node[anchor=center] (text) at (0.5,-8) {$\mathrm{log}$};

\drawatomcore{3.5}{-8}{$\rencodingof{\hypercore}$}
\drawatomindices{3.5}{-12}	
\draw (5.5,-9)--(5.5,-6) node[midway,right] {\tiny $\randomxof{\hypercore}$};
\draw[->] (5.5,-9) -- (5.5,-7.5);

\draw (3.25,-4) rectangle (7.5,-6);
\node[anchor=center] (text) at (5.5,-5) {$\restrictionofto{\chainingfunction}{\imageof{\hypercore}}$
%\begin{bmatrix} 
%	\chainingfunction(0) \\
%	\chainingfunction(1)
%\end{bmatrix}
};

\end{scope}

\end{tikzpicture}
\end{center}
\caption{Representation of the composition of a tensor $\hypercore$ with a real function $\chainingfunction$.}
\label{fig:tensorFunctionComposition} 
\end{figure}


\begin{corollary}\label{cor:rhoToNormal}
	For any tensor $\hypercoreat{\shortcatvariables}$ we have
	\begin{align*}
		\hypercoreat{\shortcatvariables}
		= \contractionof{\rencodingofat{\hypercore}{\indvariableof{\hypercore},\shortcatvariables},\idrestrictedto{\imageof{\hypercore}}\left[\indvariableof{\hypercore}\right]}{\shortcatvariables} \, .
	\end{align*}
\end{corollary}
\begin{proof}
	This follows from \theref{the:tensorFunctionComposition} using $\chainingfunction=\idsymbol$ and by noticing that
	\begin{align*}
		\hypercoreat{\shortcatvariables} = \idsymbol(\hypercore)[\shortcatvariables] \, .
	\end{align*}
\end{proof}

\begin{corollary}\label{cor:onesHead}
	For any tensor $\hypercore$, which is directed with $\shortcatvariables$ incoming, we have
		\[ \onesat{\shortcatvariables} = \contractionof{\rencodingof{\hypercore}}{\shortcatvariables} \, . \]
\end{corollary}
\begin{proof}
	This follows from \theref{the:tensorFunctionComposition} using $\chainingfunction=\ones$ and by noticing that
	\begin{align*}
		\onesat{\shortcatvariables} = \ones(\hypercore)[\shortcatvariables] \, .
	\end{align*}
\end{proof}


%% COULD STATE SLICING THEOREM AS A COMPOSITION OF CHAININGS! But unclear, wheter needed
%\begin{theorem}
%	\begin{align*}
%		\coordinatetrafowrtofat{\chainingfunction}{\contractionof{\exvector[\indvariableof{\exfunction}],\rencodingofat{\hypercore}{\indvariableof{\exfunction},\shortcatvariables}}{\shortcatvariables}}{\shortcatvariables}
%		= \coordinatetrafowrtofat{\left(\coordinatetrafowrtofat{\chainingfunction}{\exvector}{\indvariableof{\exfunction}}\right)}{\hypercore}{\shortcatvariables}
%	\end{align*}
%\end{theorem}
%\begin{proof}
%	Simply by compositions of transforms.
%\end{proof}
%
%
%% Replacement of Slicing Theorem
%\begin{corollary}\label{cor:directedTrafo}
%	Let $\basisslices$ be a directed and boolean tensor with incoming variables being $\shortcatvariables$, and $\gentensor$ a tensor, which variables are the outgoing variables of $\basisslices$.
%	Let further $\chainingfunction:\rr\rightarrow\rr$ be any real function.
%	Then
%		\[ \chainingfunction \circ \contractionof{\basisslices,\gentensor}{\shortcatvariables}
%		= \contractionof{\basisslices,\chainingfunction\circ\gentensor}{\shortcatvariables} \, . \]
%\end{corollary}
%\begin{proof}
%	Since $\basisslices$ is a directed and boolean tensor, we find a map
%		\[ \exfunction : \facstates \rightarrow \secfacstates \]
%	such that $\basisslices=\rencodingof{\exfunction}$ and a map $V$ such that $\gentensor=\restrictionofto{V}{\imageof{\exfunction}}$.
%	Then \theref{the:tensorFunctionComposition} implies that
%		\[ \contractionof{\basisslices,\gentensor}{\shortcatvariables} = V \circ \exfunction \, . \]
%	It follows that
%	\begin{align*}
%		\chainingfunction \circ \contractionof{\basisslices,\gentensor}{\shortcatvariables} = \chainingfunction \circ V \circ \exfunction
%	\end{align*}
%	and by another application of Theorem~\ref{the:tensorFunctionComposition} that
%	\begin{align*}
%		\chainingfunction \circ V \circ \exfunction
%		& = \contractionof{\rencodingof{\exfunction}, \restrictionofto{\chainingfunction \circ V}{\imageof{\exfunction}}}{\shortcatvariables} \\
%		& = \contractionof{\basisslices,\chainingfunction\circ\gentensor}{\shortcatvariables} \, .
%	\end{align*}
%	The claim follows as a combination of both equations.
%\end{proof}





\subsect{Decomposition in case of structured images}

When a set is structured as the cartesian product of other sets, that is
\begin{align*}
	\outset = \bigtimes_{\catenumeratorin} \arbsetof{\catenumerator} \, ,
\end{align*}
we can enumerate it by a collection $\{\indvariableof{\catenumerator} \, : \, \catenumeratorin\}$ of enumeration variables, each with respective index interpretation maps.
When the image of a function admits such a cartesian representation, we now show that the relational encoding can be represented by a contraction of relational encodings to each image coordinate.

\begin{theorem}\label{the:functionImageDecompositionContraction}
	Let $\exfunction$ be a function between factored systems
	\begin{align*}
		\exfunction : [\catdim] \rightarrow  \facstates
	\end{align*}
	and denote by
	\begin{align*}
		\exfunctionof{\catenumerator} : [\catdim] \rightarrow [\catdimof{\catenumerator}]
	\end{align*}
	the image coordinate restrictions of $\exfunction$, that is we have $\exfunction=(\exfunctionof{0},\ldots,\exfunctionof{\catorder-1})$.
	Let us assign the variable $\catvariable$ to the factored system in the domain system of $\exfunction$ and the variables $\catvariableof{\atomenumerator}$ for $\atomenumeratorin$ to the image system of $\exfunction$.
	We can then decompose the relational encoding of $\exfunction$ into the relational encodings of its image coordinate restrictions, that is
	\begin{align*}
		\rencodingofat{\exfunction}{\shortcatvariables,\catvariable}
		= \contractionof{
		\{\rencodingofat{\exfunctionof{\atomenumerator}}{\catvariableof{\atomenumerator},\catvariable} : \atomenumeratorin \} 
		}{\shortcatvariables,\catvariable} \, .
	\end{align*}
\end{theorem}
\begin{proof}
	For any $\catindexin$ we have
	\begin{align*}
		\rencodingofat{\exfunction}{\shortcatvariables,\indexedcatvariable}
		&= \onehotmapofat{\exfunctionat{\catindex}}{\shortcatvariables} \\
		&= \bigotimes_{\atomenumeratorin} \rencodingofat{\exfunctionof{\atomenumerator}}{\catvariableof{\atomenumerator},\indexedcatvariable} \\
		&= \contractionof{
		\{\rencodingofat{\exfunctionof{\atomenumerator}}{\catvariableof{\atomenumerator},\indexedcatvariable} : \atomenumeratorin\}
		}{\shortcatvariables} \\
		&= \contractionof{
		\{\rencodingofat{\exfunctionof{\atomenumerator}}{\catvariableof{\atomenumerator},\catvariable} : \atomenumeratorin\}
		}{\shortcatvariables,\indexedcatvariable}
	\end{align*}
	and therefore equality of both tensors.
\end{proof}

% Continue discussion in Sparse TC
In \charef{cha:sparseTC} we will apply \theref{the:functionImageDecompositionContraction} in \theref{the:functionDecompositionBasisCP} to show sparse basis CP decompositions to $\rencodingof{\exfunction}$.
These decompositions are then applied for efficient the representation of empirical distribution, which involve the relational encoding of data maps (see \exaref{exa:empDistCP}), and for exponential families, which statistics have images, which are included in cartesian products of the images to each coordinate (see \exaref{exa:expFamCP}).


\sect{Effective Coordinate Calculus}\label{sec:effectiveCalculus} % -> Part III

% Motivation: Effective Coordinate Calculus
In some situations, we can perform basis calculus more effectively by avoiding image enumeration variables, and instead apply coordinatewise transforms on tensors (see \defref{def:coordinatewiseTransform}).
As we show here, these include conjunctions, which correspond with coordinatewise multiplication, and negation, which correspond with coordinatewise substraction from 1.
Such schemes are applied for example in \cite{tsilionis_tensor-based_2024} in batchwise logical inference.

\begin{figure}
\begin{center}
	\input{./PartIII/tikz_pics/basis_calculus/skeleton_elements.tex}
\end{center}
\caption{Decomposition schemes by effective calculus, using coordinatewise transforms of tensors (see \defref{def:coordinatewiseTransform}).
	a) Conjunction performed by coordinatewise multiplications, b) Negations performed by coordinatewise substraction from one.}\label{fig:ConNegDecomposition}
\end{figure}

\begin{theorem}\label{the:effectiveConjunction}
	For any formulas $\exformula,\secexformula$ we have
	\begin{align*}
		\sbcontractionof{
			\rencodingofat{\land}{\headvariableof{\exformula\land\secexformula},\catvariableof{\exformula},\catvariableof{\secexformula}},\tbasisat{\headvariableof{\exformula\land\secexformula}}
		}{\catvariableof{\exformula},\catvariableof{\secexformula}}
		= \tbasisat{\catvariableof{\exformula}} \otimes \tbasisat{\catvariableof{\secexformula}} \, . 
	\end{align*}
	In particular, it holds that (see Figure~\ref{fig:ConNegDecomposition}a)
	\begin{align*}
		(\exformula\land\secexformula)[\shortcatvariables] = \sbcontractionof{\exformula,\secexformula}{\shortcatvariables} \, . 
	\end{align*}
\end{theorem}
\begin{proof}
	We decompose 
	\begin{align*}
		\rencodingofat{\land}{\headvariableof{\exformula\land\secexformula},\catvariableof{\exformula},\catvariableof{\secexformula}}
		= \tbasisat{\headvariableof{\exformula\land\secexformula}} \otimes \tbasisat{\catvariableof{\exformula}} \otimes \tbasisat{\catvariableof{\secexformula}}
		+ \fbasisat{\headvariableof{\exformula\land\secexformula}} \left( \onesat{\catvariableof{\exformula},\catvariableof{\secexformula}} -  \tbasisat{\catvariableof{\exformula}} \otimes \tbasisat{\catvariableof{\secexformula}} \right)
	\end{align*}
	and get the first claim as
	\begin{align*}
		\sbcontractionof{
			\rencodingofat{\land}{\headvariableof{\exformula\land\secexformula},\catvariableof{\exformula},\catvariableof{\secexformula}},\tbasisat{\headvariableof{\exformula\land\secexformula}}
		}{\catvariableof{\exformula},\catvariableof{\secexformula}}
		& = \sbcontractionof{
			\tbasisat{\headvariableof{\exformula\land\secexformula}} \otimes \tbasisat{\catvariableof{\exformula}} \otimes \tbasisat{\catvariableof{\secexformula}},\tbasisat{\headvariableof{\exformula\land\secexformula}}
		}{\catvariableof{\exformula},\catvariableof{\secexformula}} \\
		& = \tbasisat{\catvariableof{\exformula}} \otimes \tbasisat{\catvariableof{\secexformula}} \, . 
	\end{align*}
	To show the second claim we use
	\begin{align*}
		(\exformula\land\secexformula)[\shortcatvariables] 
		&= \sbcontractionof{
			\rencodingofat{\exformula}{\catvariableof{\exformula},\shortcatvariables},
			\rencodingofat{\secexformula}{\catvariableof{\secexformula},\shortcatvariables},
			\rencodingofat{\land}{\headvariableof{\exformula\land\secexformula},\catvariableof{\exformula},\catvariableof{\secexformula}},
			\tbasisat{\headvariableof{\exformula\land\secexformula}}
			}{\shortcatvariables} \\
		&  = \sbcontractionof{
			\rencodingofat{\exformula}{\catvariableof{\exformula},\shortcatvariables},
			\rencodingofat{\secexformula}{\catvariableof{\secexformula},\shortcatvariables},
			(\tbasisat{\catvariableof{\exformula}}\otimes \tbasisat{\catvariableof{\secexformula}})
			%\rencodingofat{\land}{\catvariableof{\exformula},\catvariableof{\secexformula},\headvariableof{\exformula\land\secexformula}}
			}{\shortcatvariables} \\
		&= \sbcontractionof{\exformula,\secexformula}{\shortcatvariables} \, . 
	\end{align*}
\end{proof}

A similar decomposition holds for negations, as we show next.

\begin{theorem}
	For any formula $\exformula$ we have
	\begin{align*}
		\sbcontractionof{
			\rencodingofat{\lnot}{\headvariableof{\lnot\exformula},\catvariableof{\exformula}},\tbasisat{\headvariableof{\lnot\exformula}}
		}{\catvariableof{\exformula}}
		= \fbasisat{\catvariableof{\exformula}} =  \onesat{\catvariableof{\exformula}} - \tbasisat{\catvariableof{\exformula}} \, .
	\end{align*}
	and
	\begin{align*}
		\sbcontractionof{
			\rencodingofat{\lnot}{\catvariableof{\exformula},\headvariableof{\lnot\exformula}},\fbasisat{\headvariableof{\lnot\exformula}}
		}{\catvariableof{\exformula}}
		= \tbasisat{\catvariableof{\exformula}} \, . 
	\end{align*}
	In particular, it holds that (see Figure~\ref{fig:ConNegDecomposition}b)
	\begin{align*}
		(\lnot\exformula)[\shortcatvariables] = \onesat{\shortcatvariables} - \formulaat{\shortcatvariables}  \, . 
	\end{align*}
\end{theorem}
\begin{proof}
	Using that for two dimensional variables we have $\onesat{\catvariable}=\fbasisat{\catvariable}+\tbasisat{\catvariable}\, .$
\end{proof}

% Usage
These theorems provide a mean to represent logical formulas by sums of one-hot encodings.
Since any propositional formula can be represented by compositions of negations and conjunctions, they are universal.
We further notice, that the resulting decomposition is a basis+ CP format, as further discussed in \charef{cha:sparseCalculus}.
In Figure~\ref{fig:DecompositionExample} we provide an example of this decomposition.


\begin{figure}
\begin{center}
	\input{./PartIII/tikz_pics/basis_calculus/skeleton_example.tex}
\end{center}
\caption{
	Example of a decomposition by effective calculus of a formula $\exformula(\catvariableof{1},\catvariableof{2}) = \textcolor{blue}{\lnot} \secexformula^{(1)}(\catvariableof{1},\catvariableof{2}) \textcolor{red}{\land}  \secexformula^{(2)}(\catvariableof{1},\catvariableof{2})$ into a sum of contractions.}
	\label{fig:DecompositionExample}
\end{figure}


% Calculus against the direction
\red{In an alternative perspective, effective calculus amounts to an contraction against the directionality of the relational encodings.}
For specific functions, slices of the relational encodings with respect to head variables are basis vectors.
In that case, we can perform basis calculus in the inverse direction than suggested by the directions of the tensors.
We examplify this situation in the following theorem for relational encodings of logical conjunctions and negations.




\sect{Applications in Machine Learning}

The neural paradigm of Machine Learning describes the relevance of sparse function to be effective models in the sense of learning and approximation.

% Neural Paradigm by Tensor Network Decompositions
Our model of the neural paradigm are tensor network decompositions, seen as decomposition of functions into smaller functions, which take each other as input.
Summations along input axis are avoided, when having directed and boolean tensor networks with basis calculus interpretation.

% Basis Calculus
We have already observed in Theorem~\ref{the:basisCalculus}, that the value of discrete maps can be calculated by contractions of the directed boolean relation encodings.
This has been framed as Basis Calculus.
What is more, tensor network decompositions into directed boolean tensors correspond with representation of functions as compositions of smaller functions.
We can understand each composition as marking a neuron in an architecture and thus have established a neural perspective on boolean directed tensor networks.


    \input{PartIII/sparse_calculus.tex}

    \section{Contraction Message Passing}\label{cha:messagePassing}

In this chapter we introduce local contraction passed along tensor clusters to approximatively calculate global contractions.
These message passing schemes provide tradeoffs between efficiency increases and exactness of the global contraction.
We use the CP Decompositions to investigate the asymptotic behavior of the message passing algorithms.



\subsection{Exact Contractions}

%\red{This is the junction tree algorithm!}

We apply Theorem~\ref{the:splittingContractions} to split a contraction into subcontractions, which are consecutively performed.

% Message Passing
Contractions can be performed partially, and the result passed to the rest of the network as a message.

\subsubsection{Construction of Cluster Graphs}

% Cluster Graphs
\begin{definition}[Cluster Graph]
	Given a tensor network $\extnet$ a cluster partition is a partition of the tensor network into $n$ clusters, by a function
		\[ \alpha : \edges \rightarrow [n] \, . \]
	The clusters are with tensors decorated edge sets $\enc = \{\edge \, : \, \alpha(\edge) = i\}$ with variables $\nodes_i = \bigcup_{\edge \in \enc} \edge$.
	The clusters form a graph where edges between $\enc$ and $C_j$ exist, when the node sets $\nodes_i$ and $\nodes_j$ are not disjoint.
	In this case, we define separation sets $S_{i,j}=\enc\cup C_j$
\end{definition}

\begin{theorem}
	Given a tensor network $\extnet$ and a cluster graph.
	We then define for each cluster the node set
		\[ \tilde{\nodes}_i = \bigcup_{j\neq i} \nodes_j \]
	and have
		\[ \contractionof{\extnet}{\catvariableof{\secnodes}} = 
		\contractionof{
			\{ \contractionof{ \tnetof{\enc} }{\nodes_i \cap (\tilde{\nodes}_i\cup\secnodes)}  : i \in [n]\}
		}{\catvariableof{\secnodes}}  \, . \]
\end{theorem}
\begin{proof}
	By Theorem~\ref{the:splittingContractions} applied for each cluster seen as a subgraph.
\end{proof}



\subsubsection{Message Passing to calculate contractions}

% Cluster Graphs
Having a hypergraph $\graph$, we iteratively apply Theorem~\ref{the:splittingContractions} and call the $\graph_2$ a cluster.
When iterating until $\graph$ is empty, we get a cluster graph, where all tensors are assigned to a cluster.


% Cluster Trees -> Clique Trees in Koller Book
When the cluster are a polytree, that is a union of disjoint trees, we define messages between neighbored clusters $\enc$ and $\secenc$ with $\secenc\prec\enc$ by the contractions

\begin{align}
	\upmes{j}{i} = \contractionof{\{\upmes{\tilde{j}}{j} \, : \,  \thirdenc \prec \secenc\} \cup \tnetof{\secenc}}{\catvariableof{\nodes_i\cap \nodes_j}} \, .
\end{align}
and
\begin{align}
	\downmes{i}{j}  = \contractionof{\{\downmes{\tilde{j}}{i} \, : \,  \enc \prec  \thirdenc\} \cup \tnetof{\enc}}{\catvariableof{\nodes_i\cap \nodes_j}} \, .
\end{align}


We note, that the messages are well defined by these recursive equations, exactly when the cluster graph is a polytree.
%Since messages are recursively defined, we need the tree structure to ensure well-definedness.


\begin{lemma}
	When the cluster graph is a tree, we have for neighbored clusters $\enc$ and $\secenc$ with $\secenc\prec\enc$
		\[ \upmes{\secclusterenumerator}{\clusterenumerator} 
		= \contractionof{\{\tnetof{\thirdenc} \, : \, \thirdenc \prec \secenc \}}{\catvariableof{\nodes_i\cap\nodes_j}}   \]
	and
		\[ \downmes{\clusterenumerator}{\secclusterenumerator}
		= \contractionof{\{\tnetof{\thirdenc} \, : \, \enc \prec \thirdenc \}}{\catvariableof{\nodes_i\cap\nodes_j}}  \, . \]
\end{lemma}
\begin{proof}
	By induction over the cardinality of the preceding clusters.
	\paragraph{$n=1$}: Only a single cluster before, therefore trivial.
	\paragraph{$n+1\rightarrow n$}: Assuming the statement holds for up to $n$ preceding clusters, let there be $n+1$ preceding clusters.
	Then Theorem~\ref{the:splittingContractions} splits contractions into terms, which are by inductive assumption the messages.
\end{proof}


\begin{theorem}
	When the cluster graph is a tree, then we have for each cluster $\enc$ with neighbors $N(\clusterenumerator)$
%	Then for each clique we have the conditional probability of its variables being the contraction of the messages with the cliques cores, that is
	\begin{align}
		\contractionof{\extnet}{\nodes_i} = 
		\contractionof{
			\{ \upmes{\secclusterenumerator}{\clusterenumerator}  \, : \, j \in N(i) , \, \secenc\prec\enc \}  \cup 
			\{ \downmes{\secclusterenumerator}{\clusterenumerator}  \, : \, j \in N(i),  \, \enc\prec\secenc \} \cup \tnetof{\enc}
		}{\nodes_i} \, .
	\end{align}
\end{theorem}
\begin{proof}
	By Theorem~\ref{the:splittingContractions} we split into contractions of the clusters up and down of the respective neighbors and apply the above lemma.
\end{proof}





\subsubsection{Variable Elimination Cluster Graphs}


\begin{remark}[Construction of Cluster Graphs by Variable Elimination]
	% Build a cluster graph
	Following an elimination order of the colors, mark those tensors containing the colors, which have not been marked before, as the cluster.
	% Extension to clique tree
	A clique tree can be constructed by these cluster, when iterating through the clusters and either connect them to previous disconnected clusters or leave the current cluster disconnected.
	Add the disconnected clusters with the current cluster in case there are overlaps of their open colors.
	If the disconnected cluster added has more open colors, 
\end{remark}


\subsubsection{Bethe Cluster Graphs}


\begin{figure}[h]
\begin{center}
	\begin{tikzpicture}[scale=0.35, thick] % , baseline = -3.5pt




\begin{scope}[shift={(23,0)}]

\node[anchor=center] (text) at (-6,8) {$b)$};

\draw (2,8) rectangle (4,6);
\node[anchor=center] (text) at (3,7) {\small $\bencodingof{\lor}$};

\draw (2,5) rectangle (4,3);
\node[anchor=center] (text) at (3,4) {\small $\bencodingof{\land}$};

\draw (2,2) rectangle (4,0);
\node[anchor=center] (text) at (3,1) {\small $\datacoreof{c}$};

\draw (2,-1) rectangle (4,-3);
\node[anchor=center] (text) at (3,-2) {\small $\datacoreof{b}$};

\draw (2,-4) rectangle (4,-6);
\node[anchor=center] (text) at (3,-5) {\small $\datacoreof{a}$};

\draw (2,-7) rectangle (4,-9);
\node[anchor=center] (text) at (3,-8) {\small $\lambda$};


\draw[fill] (-3,-8.5) circle (0.15cm);
\node[anchor=center] (text) at (-4,-8.5) {\tiny $\indexset$};

\draw[] (-3,-8.5) to[bend left=-10]  (2,-8);
\draw[] (-3,-8.5) to[bend left=0]  (2,-5.5);
\draw[] (-3,-8.5) to[bend left=10]  (2,-2.5);
\draw[] (-3,-8.5) to[bend left=20]  (2,0.5);


\draw[fill] (-3,-5.5) circle (0.15cm);
\node[anchor=center] (text) at (-4,-5.5) {\tiny $a$};

\draw[] (-3,-5.5) to[bend left=-10]  (2,-4.5);
\draw[] (-3,-5.5) to[bend left=10]  (2,3.5);

\draw[fill] (-3,-2.5) circle (0.15cm);
\node[anchor=center] (text) at (-4,-2.5) {\tiny $b$};

\draw[] (-3,-2.5) to[bend left=-10]  (2,-1.5);
\draw[] (-3,-2.5) to[bend left=10]  (2,4);

\draw[fill] (-3,0.5) circle (0.15cm);
\node[anchor=center] (text) at (-4,0.5) {\tiny $c$};

\draw[] (-3,0.5) to[bend left=-10]  (2,1.5);
\draw[] (-3,0.5) to[bend left=10]  (2,6.5);

\draw[fill] (-3,3.5) circle (0.15cm);
\node[anchor=center] (text) at (-4.5,3.5) {\tiny $a\land b$};

\draw[] (-3,3.5) to[bend left=-10]  (2,4.5);
\draw[] (-3,3.5) to[bend left=10]  (2,7);

\draw[fill] (-3,6.5) circle (0.15cm);
\node[anchor=center] (text) at (-5.5,6.5) {\tiny $(a\land b)\lor c$};

\draw[] (-3,6.5) to[bend left=10]  (2,7.5);


\draw[dashed] (-0.5,-12) -- (-0.5,8);

\node[right] (text) at (0.5,-11) {$\tilde{\edges}$};
\node[left] (text) at (-1.5,-11) {$\Delta$};

\end{scope}


\node[anchor=center] (text) at (-2,8) {$a)$};

\newcommand{\conposseldec}{3,-5.5}

\draw[fill] (\conposseldec) circle (0.15cm);
\draw (\conposseldec) -- (3,-7.5) node[midway, right] {\tiny ${\indexset}$}; % Unclear, whether this is the best notation!
\draw[] (2,-7.5) rectangle (4, -9.5);
\node[anchor=center] (text) at (3,-8.5) {\small $\lambda$};

\draw[] (0,1) -- (0,-1) node[midway,left] {\tiny $a$};
\draw (-1,-1) rectangle (1, -3);
\node[anchor=center] (text) at (0,-2) {\small $\datacoreof{a}$};
\draw[] (0,-3) to[bend right=20] (\conposseldec);


\draw[] (3,1) -- (3,-1) node[midway,left] {\tiny $b$};
\draw (2,-1) rectangle (4, -3);
\node[anchor=center] (text) at (3,-2) {\small $\datacoreof{b}$};
\draw[] (3,-3) to[bend right=0]  (\conposseldec);


\draw[] (6,5) -- (6,-1) node[midway,left] {\tiny $c$};
\draw (5,-1) rectangle (7, -3);
\node[anchor=center] (text) at (6,-2) {\small $\datacoreof{c}$};
\draw[] (6,-3) to[bend left=20]  (\conposseldec);


\draw[] (1.5,5) -- (1.5,3) node[midway,left] {\tiny $a \land b $};
\draw (-1,3) rectangle (4, 1);
\node[anchor=center] (text) at (1.5,2) {\small $\bencodingof{\land}$};


\draw[] (3.5,9) -- (3.5,7) node[midway,left] {\tiny $(a \land b) \lor c $};
\draw (0,7) rectangle (7, 5);
\node[anchor=center] (text) at (3.5,6) {\small $\bencodingof{\lor}$};

%\draw[] (6,1) to[bend left=20]  (\conposseldec);


		


\end{tikzpicture}
\end{center}
\caption{Example of a Bethe Cluster Graph.
	a) Example of a Tensor Network $\tnetof{\graph}$, which represents the by $\lambda$ averaged evaluation of the formula $(a\land b)\lor c$ on data $\datamap$.
	b) Corresponding Bethe Cluster Hypergraph, which dual is bipartite by the sets $\Delta$ and $\tilde{\edges}$.
	}
\label{fig:betheDataExample} 
\end{figure}

By adding delta tensors to each node $\node\in\nodes$ and defining its leg variables by $\node^{\edge}$ for $\edge\in\edges$.
We mark each such delta tensor by a cluster in $\Delta^{\graph}$, as defined in the following (see also Figure~\ref{fig:betheDataExample}).

\begin{definition}
	Given a tensor network $\tnetof{\graph}$ on a decorated hypergraph $\graph$, we define the Bethe Cluster Hypergraph $\secgraph$ as
	$(\secnodes, \secedges \cup \Delta^{\graph})$ where we have
	\begin{itemize}
		\item Recolored Edges $\secedges = \{\tilde{\edge} \, : \, \edge\in \edges\}$ where $\tilde{\edge} = \{\node^{\edge} \, : \, \node\in\edge\}$, which decoration tensor has same coordinates as $\hypercoreof{\edge}$
		\item Nodes $\secnodes = \bigcup_{\edge\in\edges}\tilde{\edge}$ %$\secnodes = \bigcup_{\edge\in\edges}\{\node^{\edge} \, : \, \node\in\edge \}$ 
		\item Delta Edges $\Delta^{\graph} =  \big\{ \{\node^{\edge} \, : \, \edge\ni\node \} \, : \, \node\in\nodes \big\} $, each of which decorated by a delta tensor $\delta^{\{\node^{\edge} \, : \, \edge\ni\node \}}$
	\end{itemize}
\end{definition}

By \lemref{lem:deltification} this construction does not change contractions.

% Dual graph
The dual is bipartite, since any variable appears exactly in one cluster in $\secedges$ and in one cluster of $\Delta^{\graph}$.
This further makes the dual of the Bethe Cluster Hypergraph a proper graph (i.e. edges consistent of node pairs). 





\subsubsection{Computational Complexity}

\red{Tree-width here.}

Naive execution of $\contractionof{\tnetof{\graph}}{\secnodes}$: $\prod_{\node\in\nodes} \catdimof{\node}$ many products are built and summed up.
When splitting contractions into local subcontractions, the product can be turned into sums with tremendous decrease in complexity.









\subsection{Approximate Contractions}

We ignore that cluster graphs are not trees and perform contraction message passing along neighbored clusters.
For the contraction of basis tensor networks, this scheme still provides the exact contraction.

\subsubsection{Exact Message Passing for Directed and Binary Contractions}


%% Function Composition Perspective
A Tensor Network of directed and binary cores represents the evaluation of composed functions.
In a Message Passing Perspective each component (let us call them neurons) can be evaluated, when the evaluation of the ancestor neurons are known.

\begin{lemma}\label{lem:diracConBasis}
	\red{Required? Basis vector factorization suffices?}
	For any collection of categorical variables $\shortcatvariables$ with identical dimension and any $\catindexofin{0}$ we have
		\[ \sbcontractionof{\delta[\catvariableof{0},\ldots,\catvariableof{\catorder-1}]}{\indexedcatvariableof{0},\catvariableof{1},\ldots,\catvariableof{\catorder-1}} 
		= \bigotimes_{\catenumerator\in\{1,\ldots,\catorder-1\}} \onehotmapofat{\catindexof{0}}{\catvariableof{\catenumerator}} \, . \]
\end{lemma}
\begin{proof}
	Directly by sum decomposition of delta tensors.
\end{proof}

\begin{theorem}
	Let $\tnetof{\graph}$ be a tensor network on a directed acyclic hypergraph $\graph$, such that each tensor is Boolean and directed, and such that each variable is appearing only once as an outgoing variable of a hyperedge.
	We build a cluster graph by storing each edge as a cluster and use the topological order on $\graph$.
	%We denote the scalar messages on the edges with no outgoing variables a single propagation of messages along the direction of the Bethe Cluster Graph by $\delta^{\edge}$ 
	%Then they coincide with the exact contractions leaving the variables of the edge open.	
	Then
		\[ \upmes{j}{i} = \contractionof{\tnetof{\graph}}{\catvariableof{\nodes_i\cap\nodes_j}} \, . \]
\end{theorem}
\begin{proof}
\red{Lemma above needed?}
	By using that each message is a basis vector (Using Theorem~\ref{the:conditionalContractionPreservation} in an induction argument) and can thus be splitted into the product of multiple copies.

	% Reducing to downcore
	Any hypercore, which is not a precessor to $\hypercoreof{\edge_{\clusterenumerator}}$ can be omitted from the contraction by a root-stripping argument using its directionality.
	Therefore we have
	\begin{align*}
		\contractionof{\tnetof{\graph}}{\catvariableof{\nodes_i\cap\nodes_j}}
		= \contractionof{\{\hypercoreof{\edge_{\thirdclusterenumerator}} \, : \, \edge_{\thirdclusterenumerator} \prec \edge_{i} \}}{\catvariableof{\nodes_i\cap\nodes_j}} \, . 
	\end{align*}

	We then replace each variable which is appearing more than once in outgoing legs by a delta tensor, which does not change the contraction by \lemref{lem:deltification}.
	%When there are undirected loops in the network beyond the cluster, we apply \lemref{lem:diracConBasis} to replace the variable by its copies and a delta tensor.
	
	We then follow a leaf stripping argument and apply Theorem~\ref{the:splittingContractions} iteratively on the remaining leaves. 
	Along that line, the leave and its successors are contracted.
	The contraction is a basis vector and can therefore be represented as an outer product of basis vectors.

\end{proof}



When replacing $\onehotmapof{\catindexof{0}}$ by an arbitrary vector in \lemref{lem:diracConBasis} we have
	\[ \sbcontractionof{\delta[\catvariableof{0},\ldots,\catvariableof{\catorder-1}], V[\catvariableof{0}]}{\catvariableof{1},\ldots,\catvariableof{\catorder-1}} 
		\neq \bigotimes_{\catenumerator\in\{1,\ldots,\catorder-1\}} V[{\catvariableof{\catenumerator}}]  \, . \]
Therefore, the messages will in general differ from the exact contractions.
To provide intuition of what happens in this case, let us take the following cases into account:
\begin{itemize}
	\item $\lambda\cdot\onehotmapof{\catindex}$ sent multiple times: Result gets a factor of $\lambda^{\# \text{copies}}$ compared with the exact contraction.
	\item $\onehotmapof{\catindex}+\onehotmapof{\tilde{\catindex}}$: Result is the exact contraction added by the crossterms of sending $\onehotmapof{\catindex}$ in one and $\onehotmapof{\tilde{\catindex}}$ in the other direction.
\end{itemize}	



\subsubsection{Case of Matrices}

\red{Here a toy example of cycling messages starting with $\ones$.
When normating the messages, the maximal singular vectors will be dominant.}

We investigate the Bethe message passing for a tensor network consisting of a single matrix.

\begin{theorem}
	The stable messages are the linear subspace of the maximal singular values of the fixed core $\hypercore$.
\end{theorem}
\begin{proof}
	Having a Singular Value Decomposition of $\hypercore$ and decompose the messages in the orthonormal system of the respective singular vectors.
\end{proof}


For the propagation of $a$ and $b$ on binary $\exformula(a, b)$ starting with trivial messages of $a$ and $b$ the above theorem implies:
\begin{itemize}
	\item Case of single possible world: Exact 
		$\exformula(a, b) \in \{ a \land b, a \land \lnot b, \lnot a \land b, \lnot a \land \lnot b \}$
		Messages are after first iteration exact
	\item Case of two possible worlds and $\exformula(a, b) \in \{ a \Leftrightarrow b, a \Leftrightarrow \lnot b, \lnot a \Leftrightarrow b, \lnot a \Leftrightarrow \lnot b \}$. 
		In this situation any start message is stable and determines the other.
	\item Case of two possible worlds and $\exformula(a, b) \in \{ a, b, \lnot a, \lnot b\}$.
		In this situation one stable message is determined by the specified atom and the other is always stable.
	\item Case of  three possible worlds: Approximative (exact: $1/3, 2/3$, approximative: $1-golden ratio, golden ratio$
	\item Case of four possible worlds: Exact ($\ones$) 
\end{itemize}


\subsubsection{Case of Tensors}

Let there now be a single tensor of arbitrary order.
When the tensor is not ODECO, we cannot find a CP-Decomposition with leg vectors building an orthonormal system in the respective leg spaces.
This prohibits direct application of the same techniques in the case of a matrix, which is always ODECO.



\subsection{Basis Calculus}

Message Passing of directed and binary message by relational encoding of functions can be interpreted as function evaluation.
This is because any relational encoding of a function, the decomposition
\begin{align*}
	\rencodingof{\exfunction} = \sum_{y \in \imageof{\exfunction}} ( \sum_{i: \exfunction(i)=y}\onehotmapof{i} )  \otimes \onehotmapof{y}
\end{align*}
is a SVD of the matrification of $\rencodingof{\exfunction}$ with respect to incoming and outgoing legs.


Passing a message $\onehotmapof{i}$ in direction thus gives the message $\onehotmapof{\exfunction(i)}$.


%After having established a one-to-one connection between the directed and binary tensors with the encoding of functions, we now interpret contractions as evaluations of the respective functions.
%Applying this insight iteratively on composed functions we show the following theorem.

\begin{remark}[Basis Calculus as Message Passing]
	Given a tensor network of directed and binary tensor cores $\hypercoreof{\edge}$, each representing a function $\exfunction^{\edge}$.
	When there are not directed cycles, we define the compositions of $\exfunction^{\edge}$ to be the function $\exfunction$ from the nodes $\nodes_1$ not appearing as incoming nodes to the nodes $\nodes_2$ not appearing as outgoing nodes in an edge.
	Choosing arbitrary $\catindexof{\node}\in[\catdimof{\node}]$ for $\node\in\nodes_1$ we have
	\begin{align*}
		\contractionof{\{\hypercoreof{\edge} \, : \edge\in\edges\}}{\nodes_2} = \onehotmapof{\exfunction(\catindexof{\node} \, : \, \node\in\nodes_1)}\, . 
	\end{align*}
\end{remark}
%\begin{proof}
%	Use a message passing argument for each function $\exfunction^{\edge}$.
%\end{proof}




\subsection{Applications}

% Application: Dynamic programming
When queries share same parts, can perform their contraction using dynamic programming.
For conditional probability queries, which variables are the clusters of a cluster tree, this results in belief propagation.




    \section{Reasoning by Tensor Approximation}\label{cha:tensorApproximation}

Often reasoning requires the execution of demanding contractions of tensors networks, or combinatorical search of maximum coordinates.
We in this chapter investigate methods, to replace hard to be sampled tensor networks by approximating tensor networks, which then serve as a proxy in inference tasks.


\subsection{Approximation of Energy tensors}

\subsubsection{Direct Approximation}

Direct approximation is the problem
	\[ \argmin_{\canparam\in\Gamma^{\graph}} \|\energytensorat{\shortcatvariables} - \canparamat{\shortcatvariables}\|^2 \, . \]


\subsubsection{Approximation involving Selection Architectures}

Approximation involving a selection architecture $\fselectionmap$ is the problem
	\[ \argmin_{\canparam\in\Gamma^{\graph}} \|\energytensor - \sbcontractionof{\sencodingof{\fselectionmap},\canparam}{\shortcatvariables}\|^2 \, . \]

In a tensor network diagram we depict this as
\begin{center}
    \input{PartIII/tikz_pics/approximation/least_squares.tex}
\end{center}


\begin{example}[Approximate based on a slice sparsity selecting architecture]
	Use a term selecting neural network (conjunction neuron on $\atomorder$ unary neurons selecting a variable and $\mathrm{Id},\lnot,\mathrm{True}$ as connective selector.
	Demand the parameter tensor $\canparam$ to be in a basis CP format, then each slice of the parameter tensor corresponds with the slice of the energy.
	The use the approximation for MAP search.
	Same construction possible for probability tensors, but often more involved to instantiate them as tensor network.
\end{example}



\subsection{Transformation of Maximum Search to Risk Minimization}

By the squares risk trick, maximum coordinate searches involving contractions with boolean tensors can be turned into squares risk minimization problems.
This trick can be applied in MAP inference of MLN and the proposal distribution.

\subsubsection{Weighted Squares Loss Trick}

\begin{lemma}
	Let $\hypercore$ be a Boolean tensor, that is $\imageof{\hypercore}\subset\{0,1\}$.
	Then
		\[ \hypercoreat{\shortcatvariables} = \onesat{\shortcatvariables} - \left( \hypercoreat{\shortcatvariables} - \onesat{\shortcatvariables} \right)^2  \]
	where $\ones$ is a tensor with same shape as $\hypercore$ and all coordinates being $1$.
\end{lemma}
\begin{proof}
	Since for each $\shortcatindices\in\facstates$ we have $\hypercore[\shortcatvariables=\shortcatindices]\in\{0,1\}$, it holds that
		\[ \hypercoreat{\shortcatvariables=\shortcatindices} = 1 - (\hypercoreat{\shortcatvariables=\shortcatindices}-1)^2 \]
	and thus in coordinatewise calculus
		\[ \hypercoreat{\shortcatvariables} = \onesat{\shortcatvariables} - \left( \hypercoreat{\shortcatvariables} - \onesat{\shortcatvariables} \right)^2 \, .   \]
\end{proof}

We apply this property to reformulate optimization problems over boolean tensors into weighted least squares problems.

\begin{theorem}[Weighted Squares Loss Trick]\label{the:reweightedLeastSquares}
	Let $\Gamma$ be a set of boolean tensors in $\facspace$ and $\importancetensor\in\facspace$ arbitrary.
	Then we have
	\begin{align}
		\argmax_{\hypercore\in\Gamma} \contraction{\importancetensor,\hypercore} 
		= \argmin_{\hypercore\in\Gamma} \contraction{\importancetensor, (\hypercoreat{\shortcatvariables}-\onesat{\shortcatvariables})^2}
	\end{align} 
\end{theorem}
\begin{proof}
	Using the Lemma above, $\hypercore$ is identical to $\onesat{\shortcatvariables}-(\hypercoreat{\shortcatvariables}-\onesat{\shortcatvariables})^2$ and we get
	\begin{align*}
		 \contraction{\importancetensor,\hypercore} 
		 &=  \contraction{\importancetensor,\onesat{\shortcatvariables}}-\contraction{\importancetensor,(\hypercoreat{\shortcatvariables}-\onesat{\shortcatvariables})^2} 
	\end{align*}
	Since the first term does not depend on $\hypercore$, it can be dropped in the maximization problem.
	The $(-1)$ factor then turns the maximization into a minimization problem.
\end{proof}

% Interpretation and Importance Tensor
\theref{the:reweightedLeastSquares} reformulates maximation of binary tensors with respect to an angle to another tensor into minimization of a squares risk.
This squares risk trick is especially useful when combining it with a relaxation of $\Gamma$ to differentiably parametrizable sets, since then common squares risk solvers can be applied.
We will call $\importancetensor$ in the \theref{the:reweightedLeastSquares} importance tensor, since it manipulates the relevance of each coordinate in the squares loss.

%
As a result, we interpret the objective
	\[ \contraction{\importancetensor, (\hypercoreat{\shortcatvariables}-\onesat{\shortcatvariables})^2} \]
as a weighted squares loss.

\begin{example}[Proposal distribution maxima]
	The Problem~\ref{prob:steepestAscent} of finding the maximal coordinate can thus be turned into
	\begin{align*}
		\argmax_{\shortselindices} \contractionof{(\empdistribution-\currentdistribution),\fselectionmap}{\shortselvariables=\shortselindices}  
		= \argmin_{\shortselindices} \sbcontraction{(\empdistribution-\currentdistribution),
		\left(\contractionof{\fselectionmap,\onehotmapofat{\shortselindices}{\shortselvariables}}{\shortcatvariables}-\onesat{\shortcatvariables}\right)^2} \, . 
	\end{align*}
\end{example}


\subsubsection{Problem of the trivial tensor}

By the above we motivated least squares problems on the set of one-hot encoded states.
One is tempted to extend this set to $\mnexpfamily$ for efficient solutions by alternating algorithms.

However, for any hypergraph $\graph$ we have $\onesat{\shortcatvariables}\in\mnexpfamily$.
In many situations (e.g. disjoint model sets supported at positive data) the objective is more in favor at the trivial tensor than at the one-hot encoding.
As a result, we do not solve the previously posed one-hot encoding problem, when allowing such an hypothesis embedding.


\begin{example}[Fitting a tensor by a formula tensor]\label{exa:formulaFitting}
	Task: Given a tensor $\hypercore$, find a formula $\exformula\in\formulaset$ such that it coincides with $\hypercore$.

	If $\hypercore$ is a binary tensor, we understand it as a formula and want to find an $\exformula$ such that its number of worlds is maximal, that is solve the problem
		\[ \argmax_{\exformula\in\formulaset}\sbcontraction{\exformula\Leftrightarrow\hypercore}  \, . \]

	We can use the squares risk trick and get an equivalent problem
		\[ \argmin_{\exformula\in\formulaset} \| \sbcontractionof{\exformula\Leftrightarrow\hypercore}{\shortcatvariables}  - \onesat{\shortcatvariables} \|^2 \, . \]
\end{example}

%\begin{remark}{Least Squares Loss by Tensor Fitting}
%	\red{Alternative approach to least squares problems: Tensor Fitting}
%	And, if the target is another formula $y$, such that $\exformula$ conincides with $\tilde{f} \iff y $ we have
%		\[ \left(\polynomialof{\exformula}(\datamap)-1\right)^2 = \left(  \polynomialof{\tilde{f}}(\datamap) - y(\datamap) \right)^2  \]
%	This is exactly the least squares loss, which would appear in a supervised interpretation of the learning.
%\end{remark}




\subsection{Alternating Solution of Least Squares Problems}

When the parameter tensor $\canparam$ is only restricted to have a decomposition as a tensor network on $\graph$, we can iteratively update each core.
The resulting algorithm is called Alternating Least Squares (ALS) (see Algorithm \ref{alg:ALS}).

\begin{algorithm}[hbt!]
\caption{Alternating Least Squares (ALS)}\label{alg:ALS}
\begin{algorithmic}
\For{$\edgein$}
	\State Set $\hypercoreofat{\edge}{\catvariableof{\edge}}$ to a random element in $\bigotimes_{\atomenumerator\in\edge}\rr^{\catdimof{\atomenumerator}}$
\EndFor
%\For{$\atomenumeratorin$}
%	\State Set $\varcore{\atomenumerator}$ to a random element in $\rr^{\atomlegdimof{\atomenumerator}}$ 
%\EndFor
\While{Stopping criterion is not met}
\For{$\edgein$}
	\State Set $\hypercoreofat{\edge}{\catvariableof{\edge}}$ to a to a solution of the local problem, that is
	\[ 
	\hypercoreofat{\edge}{\catvariableof{\edge}}
	 \algdefsymbol 
	 \argmin_{\hypercoreofat{\edge}{\catvariableof{\edge}}} 
	 \contraction{\importancetensor, (\contractionof{\fselectionmap,\canparam}{\shortcatvariables} - \targettensor[\shortcatvariables])^2}
	 \]
\EndFor
\EndWhile
\end{algorithmic}
\end{algorithm}


\subsubsection{Choice of Representation Format}

\red{The choice of the hypergraph $\graph$ used for approximation bears a tradeoff between expressivity and complexity in sampling.
Hidden variables, that is variables only present in $\graph$, but not in the sensing matrix, increase the expressivity, especially when assigning large dimensions to them.
When there are no hidden variables, the maximum of $\canparam$ can be found by maximum calibration through a message passing algorithm, since no hidden variable has to marginalized.}


In case of skeleton expressions with many placeholders further decomposition for algorithmic efficiency are required.
\begin{itemize}
	\item Elementary Format ($\elformat$-Format): 
	\item $\cpformat$-Format: Closest to sum of formula tensors (when all vectors are basis, then have a sum).
	\item $\ttformat$-Format: Showed better heuristic performance in optimization
\end{itemize}

For any tensor network decomposition into cores $\canparamof{\selenumerator}$ have the derivative $\frac{\partial}{\partial \canparamof{\selenumerator}} \canparam$ as the tensor network with out the core $\canparamof{\selenumerator}$.

%\begin{remark}[Parametercores being basis tensors]
%	When the parameter core is a basis tensor, the contraction with the parametercore coincides with the respective formula tensor.
%	Thus, we will search for basis tensors optimizing in contractions objectives to specific reasoning tasks, and add them iteratively to the network at hand.
%\end{remark}


%\subsection{Projection onto Basis Tensors}
%\red{This is sampling!}
%We project onto basis tensors to achieve single formulas.


\subsection{Regularization and Compressed Sensing}


When regularizing the least squares problem by enforcing the sparsity of $\canparam$, we arrive at the compressed sensing problem
\begin{align}
	\argmin_{\canparamat{\selvariable}} \sparsityof{\canparam} 
	\quad \text{subject to } \quad 
	\left\| \contractionof{\sencsstat,\canparam}{\shortcatvariables} - \energytensorat{\shortcatvariables} \right\|_2 \leq \eta
\end{align}
Here, the sensing matrix is the selection tensor.


\begin{example}[Formula fitting to an example]
	Choosing the best formula fitting data (see Example~\ref{exa:formulaFitting}) is the problem
	\begin{align}
	\argmin_{\canparamat{\selvariable}\, : \,  \sparsityof{\canparam}=1} \left\| \contractionof{\importancetensor,\sencsstat,\canparam}{\shortcatvariables} - \targettensor \right\|_2 
	\end{align}
	where $\importancetensor$ has nonzero entries at marked coordinates and $\targettensor$ stores in Boolean coordinates whether the marked coordinates are positive or negative examples.
	\red{When the number of positive and negative examples are identical, we can linearly transform the objective to that of a grafting instance, where the current model is the empirical distribution of negative examples and the data consists of the positive examples.}
\end{example}

% Usage as sparse tensor
The sparse tensor solving the problem then has a small number of nonzero coordinates and the selection tensor can be restricted to those.
As a consequence, inference can be performed more efficiently.

% Algorithmic solution
The algorithmic solution of these problems can be done by greedy algorithms, thresholding based algorithms or optimization based algorithms \cite{foucart_mathematical_2013}.

% Guarantees
Guarantees for the success of the algorithms depend on the properties of the sensing matrices.
Here the sensing matrices are deterministic, since constructed as selection tensors, and concentration based approaches towards probabilistic bounds on these properties (see \cite{goesmann_uniform_2021}) are not applicable.





\begin{example}[Propositional Formulas]
	Let there be a set $\formulaset$ of formulas, then we have
	\begin{align*}
		\sbcontractionof{\sencodingofat{\formulaset}{\shortcatvariables,\selvariableof{\insymbol}},\sencodingofat{\formulaset}{\shortcatvariables,\selvariableof{\outsymbol}}}{\indexedselvariableof{\insymbol},\indexedselvariableof{\outsymbol}}
		= \sbcontraction{\formulaof{\selindexof{\insymbol}}, \formulaof{\selindexof{\outsymbol}}} \, . 
	\end{align*}
	If the formulas have disjoint model sets then 
	\begin{align*}
		\sbcontractionof{\sencodingofat{\formulaset}{\shortcatvariables,\selvariableof{\insymbol}},\sencodingofat{\formulaset}{\shortcatvariables,\selvariableof{\insymbol}}}{\indexedselvariableof{\insymbol},\indexedselvariableof{\outsymbol}}
		= \begin{cases}
			\sbcontraction{\formulaof{\insymbol}} & \text{if } \selindexof{\insymbol}=\selindexof{\outsymbol} \\
			0 & \text{else} 
		\end{cases} \, . 
	\end{align*}
\end{example}


\begin{example}[Slice selection networks]
	
	For the slice selection network
	\begin{align*}
		\sbcontractionof{\sliceselectionmapat{\shortcatvariables,\selvariableof{\insymbol}},\sliceselectionmapat{\shortcatvariables,\selvariableof{\outsymbol}}}{\indexedselvariableof{\insymbol},\indexedselvariableof{\outsymbol}}
		= \begin{cases}
			0 & \text{if for a }\seccatenumerator\in\variablesetof{\selindexof{\insymbol}}\cap\variablesetof{\selindexof{\outsymbol}}\text{ we have }\catindexof{\seccatenumerator}^{\selindexof{\insymbol}}\neq\catindexof{\seccatenumerator}^{\selindexof{\outsymbol}} \\
			\prod_{\seccatenumerator\notin\variablesetof{\selindexof{\insymbol}}\cup\variablesetof{\selindexof{\outsymbol}}} \catdimof{\seccatenumerator}& \text{else} 
		\end{cases} \, . 
	\end{align*}

	Given a fixed $\selindexof{\insymbol}$, the maximum value in the respective slice is thus taken at $\selindexof{\insymbol}=\selindexof{\outsymbol}$
\end{example}












    \appendix
    \chapter{Implementation in the \tnreason package}\label{cha:implementation}

We here document the implementation of the discussed concepts in the \python package \tnreason, in the version \curvertnreason
 
 % Name
\tnreason is an abbreviation of \textbf{t}ensor \textbf{n}etwork \textbf{reason}ing, by which we emphasize the capabilities of this package to represent and answer reasoning tasks by tensor network contractions. 

% Installation
The package can be installed either by cloning the repository
\begin{center}
	\href{https://github.com/EnexaProject/enexa-tensor-reasoning}{https://github.com/EnexaProject/enexa-tensor-reasoning}
\end{center}
or by
\begin{lstlisting}
	!pip install tnreason==2.0.0
\end{lstlisting}

\sect{Architecture}

\tnreason is structured in four subpackages and three layers
\begin{itemize}
	\item Layer 1: Storage and numerical manipulations, by subpackage \spengine, "Tensor Networks" -> building "tn" of \tnreason
	\item Layer 2: Specification of workload, subpackage \sprepresentation specific for storage, subpackage \spreasoning specific for manipulations
	\item Layer 3: Applications in reasoning, by subpackage \spapplication, "Reasoning" -> building "reason" of \tnreason
\end{itemize}

We sketch this structure by
\begin{center}
\input{./OtherContent/tikz_pics/implementation/architecture_sketch.tex}
\end{center}


\sect{Implementation of basic notation}

First of all, we explain how the basis notation explained in \charef{cha:notation} is reflected in the implementation.

\subsect{\bncategoricals}
Categorical Variables are identified by strings, which then appear as colors of the corresponding tensor axes.
Their dimension is stored in shapeDicts, but most practically these shapes are stored in the tensors in which variables appear.
Suffixes in the color string (defined in \inlinecode{representation.suffixes}) denote the type of the variable:
\begin{itemize}
	\item Distributed variables with color suffix \disVarSuf: $\catvariableof{\cdot}$
	\item Computed variables with color suffix \comVarSuf: $\headvariableof{\cdot}$
	\item Selection variables with color suffix \selVarSuf: $\selvariableof{\cdot}$
	\item Term variables with color suffix \terVarSuf: $\indvariableof{\cdot}$
\end{itemize}

\subsect{\bntensors}
\paragraph{Tensors} are objects of classes inheriting \inlinecode{engine.TensorCore} with main attributes
\begin{itemize}
	\item \inlinecode{values}: Storing the coordinates of the tensors (individual realization for different cores)
	\item \inlinecode{colors}: List of the variables $[\headvariableof{\formula},\catvariableof{0},\catvariableof{1}]$
	\item \inlinecode{name}: Reflecting the notation such as $\rencodingof{\formula}$
	\item \inlinecode{shape}: Storing the dimension of each appearing variable, as a list of integers with the same length as colors.
\end{itemize}

Suffixes in the name string (defined in \inlinecode{representation.suffixes}) highlight the origin and purpose of the tensor.
Cores are named with suffixes based on their functionality
\begin{itemize}
	\item Computation core with name suffix \comCoreSuf: They represent the computation of a function in basis calculus, and are directed cores.
		Their colors are \inlinecode{[headColors] + [inputColors]}, where \inlinecode{[inputColors]} are either distributed variables or, if having a composition of formulas.
		When the function is a selection augmentation of other functions, selection colors are listed in the end of \inlinecode{[inputColors]}.
	\item Activation core with name suffix \actCoreSuf: two-dimensional vectors representing of the activation core to a formula
\end{itemize}

Exploiting efficient representation tricks we further have the tensor name suffices:
\begin{itemize}
	\item \atoCoreSuf: Atomization core, for sparse representation of categorical constraints
	\item \vselCoreSuf: Variable selection core: For sparse representation of variable selectors
\end{itemize}

Tensors are instantiated by
\begin{lstlisting}
	engine.getCore(coreType)(values, colors, name, shape)
\end{lstlisting}
where \inlinecode{coreType} is a string further specifying a specific implementation of tensors (see for more detail \secref{sec:implementationEngine}).
The default tensor implementation \defaultCoreType is chosen, when \inlinecode{coreType} is not specified.



One-hot encodings are specific tensors created in \sprepresentation.

\subsect{\bncontractions}
\paragraph{Tensor networks} are stored as dictionaries of tensors, where the keys coincide with the names of the corresponding tensors.

\paragraph{Contractions} are implemented in the subpackage \spengine, orienting on \defref{def:contraction}.
Reflected in the notation
\begin{align*}
	\contractionof{\tnetof{\graph}}{\secnodevariables}
\end{align*}
a contraction is defined by
\begin{itemize}
	\item Tensor Network $\tnetof{\graph}$, specified by a dictionary of tensor names as keys and valued by tensor cores.
	\item Open Variables $\secnodes$, specified by a list of colors to the variables.
\end{itemize}
Contraction calls are implemented as
\begin{lstlisting}
	engine.contract(contractionMethod, coreDict, openColors, dimensionDict, evidenceColorDict)
\end{lstlisting}
where the arguments are
\begin{itemize}
	\item \inlinecode{contractionMethod}: str, chooses one of the contraction providers. The default contraction method \defaultContractionMethod is chosen, when
	\item \inlinecode{coreDict}: Dictionary of TensorCores (of the above formats), representing the Tensor Network $\tnetof{\graph}$
	\item \inlinecode{openColors}: List of str, each str identifying a color, that is a variable to be left open in the contraction
	\item \inlinecode{dimensionDict}: Dict valued by int and keys by str, storing dimensions to each variable. This is of optional usage, when a color in openColors does not appear in the coreDict.
	\item \inlinecode{evidenceColorDict}: Dict valued by int and keys by str, indicating sliced variables
\end{itemize}

Coordinates of tensors can be retrieved by
\begin{align*}
	\contractionof{\hypercoreat{\nodevariables}}{\secnodevariables=\catindexof{\secnodes}} \, .
\end{align*}
We implement this by leaving \inlinecode{openColors} empty and passing $\catindexof{\secnodes}$ as the \inlinecode{evidenceColorDict}, as a dictionary with keys by the \inlinecode{str} colors to the variables and values by the corresponding \inlinecode{int} indices.

Graphical illustrations can be generated by
\begin{lstlisting}
	engine.draw_factor_graph(coreDict)
\end{lstlisting}
where \inlinecode{coreDict} is a tensor network to be visualized.


\subsect{\bnencoding}
Encoding schemes are implemented in the subpackage \sprepresentation.



\sect{Subpackage \spengine}\label{sec:implementationEngine}

The \spengine subpackage is for the storage and numerical manipulation of tensors and tensor networks.
We organize the subpackage as the lowest layer of \tnreason, specializing in storage of Tensor Networks and performing the contractions.

\subsect{Cores}

\paragraph{Iterator based Core Initialization}
We orient on basis+ sparse tensor decomposition in the initialization of tensor cores, as discussed in detail in \charef{cha:sparseCalculus}.
An elementary basis+ tensor is specified by tuples
\begin{lstlisting}
	(value, posDict)
\end{lstlisting}
where \inlinecode{posDict} specifies the values to the variables, which do not have a trivial leg vector, and \inlinecode{value} a scalar scaling the basis vector.
Comparing with the notation of \charef{cha:sparseCalculus}, the keys of \inlinecode{posDict} correspond with $\variableset$, the values of \inlinecode{posDict} with $\catindexof{\variableset}$ and \inlinecode{value} corresponds with $\slicescalar$.

A basis+ $\cpformat$ tensor is specified by an iterator \inlinecode{sliceIterator} over elementary basis+ tensors, where the $\cpformat$ rank is the length of the iterator
Given such a representation a tensor is instantiated by
\begin{lstlisting}
	engine.create_from_slice_iterator(shape, colors, sliceIterator, coreType, name)
\end{lstlisting}
where \inlinecode{shape, colors, coreType, name} are used in the call of an empty core by \inlinecode{engine.get_core} and \inlinecode{sliceIterator} used to iterative add the basis+ elementary tensors to create the tensor.

\paragraph{Core Arithmetics}
When executing
\begin{lstlisting}
	exampleCore[posDict] = value
\end{lstlisting}
we add a basis+ tensor specified by \inlinecode{posDict}, that is
\begin{align*}
	\hypercoreat{\shortcatvariables} \algdefsymbol \hypercoreat{\shortcatvariables} + \contractionof{\onehotmapofat{\catindexof{\variableset}}{\catvariableof{\variableset}}}{\shortcatvariables} \, .
\end{align*}
The linear structure of tensors spaces are reflected in sums of tensors implemented with the same \inlinecode{coreType}, as
\begin{lstlisting}
	summed = exampleCore1 + exampleCore2
\end{lstlisting}
and scalar multiplication, where a scalar \inlinecode{value} of type \inlinecode{int} or \inlinecode{float}
\begin{lstlisting}
	multiplied = value * exampleCore
\end{lstlisting}
Both operations are performed as manipulations of the tensors \inlinecode{values}.
Contraction of two cores
\begin{lstlisting}
	contracted = exampleCore1.contract_with(exampleCore2)
\end{lstlisting}
these are especially used in corewise contraction, where \inlinecode{contractionMethod="CoreWiseContractor"}.


\subsect{Contractions}

\textbf{Cores}

%Each Tensor core has attributes
%\begin{itemize}
%	\item values (array-like): storing the value of the coordinates
%	\item colors (list of str): specifying the name of the variables represented by its axes
%	\item name (str): to distinguish from other cores
%\end{itemize}
%The implemented core types differ in the values argument.


\textbf{Polynomial Cores}
Polynomial Cores are implementations of the monomial decomposition or basis+ (see \defref{def:polynomialSparsity}).
Here the each tuple $(\lambda,\variableset,\catvariableof{\variableset})$ is stored as a tuple of the scalar $\lambda$ and a dictionary with $\variableset$ as keys and $\catvariableof{\variableset}$ as values.

\red{The spare cores (Polynomial and Pandas Core) exploit the matrix representation of \remref{rem:matSotrageBasPlus}.}

% Contraction Method List
The supported cores are
\begin{center}
\begin{tabular}{|c|c|c|}
  	\hline
 	\textbf{coreType} & \textbf{Package} & \text{Explanation}  \\
  	\hline
 	\stringof{NumpyTensorCore} 	&  $\mathrm{numpy}$  & Numpy array storing the values\\
  	\hline
 	\stringof{PolynomialCore} 	&  $\mathrm{numpy}$  & Storing the values in a basis+ $\cpformat$ Decomposition\\
  	\hline
\end{tabular}
\end{center}


\textbf{Binary CP Decomposition}

Based on the monomial decomposition $\slicesparsityof{\cdot}$ as specified in \defref{def:polynomialSparsity}.
To store the values of a tensor we store the slices of tensors by the indices $\catindexof{\variableset}$. 

% Trick -> To BinaryCP
Contractions can be performed by partially contracting the cores of the decomposition.
In this way, one can avoids coordinatewise storages of high-order tensors, which can be intractable.

\textbf{Tensor Networks}

Tensor networks $\tnetof{\graph}$ are defined by hypergraphs with hyperedges decorated by tensor cores. 
We store them by dictionaries with values being tensor cores and keys coinciding with the name of each tensor core.


% Contraction Method List
The supported contraction methods are
\begin{center}
\begin{tabular}{|c|c|c|}
  	\hline
 	\textbf{contractionMethod} (str) & \textbf{Package} & \text{Explanation}  \\
  	\hline
 	\stringof{NumpyEinsum} 	&  $\mathrm{numpy}$  & Einstein summation of $\mathrm{numpy}$ arrays\\
  	\hline
 	\stringof{TensorFlowEinsum} 	&  $\mathrm{tensorflow}$  & Einstein summation of $\mathrm{tensorflow}$ tensors\\
  	\hline
	\stringof{TorchEinsum} 	&  $\mathrm{torch}$  & Einstein summation of $\mathrm{torch}$ tensors\\
  	\hline
	\stringof{TentrisEinsum} 	&  $\mathrm{tentris}$  & Einstein summation of $\mathrm{tentris}$ hypertries\\
  	\hline
	\stringof{PgmpyVariableEliminator} 	&  $\mathrm{pgmpy}$  & Variable Elimination of DiscreteFactors in $\mathrm{pgmpy}$\\
  	\hline
	\stringof{CorewiseContractor} 	&  $\mathrm{numpy}$  & Contraction of CP Decompositions stored in $\mathrm{numpy}$ arrays\\
  	\hline	
\end{tabular}
\end{center}


%%\textbf{Einstein Summation}
%Contractions represented as Einstein summation, as implemented in:
%\begin{itemize}
%	\item numpy
%	\item tensorflow
%	\item pytorch
%	\item tentris
%\end{itemize}

%\textbf{Variable Elimination}
%Contractions can be executed by variable elimination as implemented in:
%\begin{itemize}
%	\item pgmpy
%\end{itemize}

%\textbf{Manipulation of Binary CP Decomposition}
%Contraction of tensors in Binary CP Decomposition as in \secref{sec:BinaryCPManipulation}.







\sect{Subpackage \sprepresentation}\label{sec:implementationRepresentation}

The \sprepresentation subpackage consists in a collection of core creation methods.

Here the relational encodings $\rencodingof{\exfunction}$ of various maps $\exfunction$ are created.


We arrange the \sprepresentation subpackage into the second layer of the \tnreason architecture, since it specifies tensor cores which formats are specified in \spengine.




\textbf{Coordinate Calculus}

Main function
\begin{lstlisting}
	engine.coordinatewise_transform(coresList, transformFunction)
\end{lstlisting}

\textbf{Basis Calculus}

Main function
\begin{lstlisting}
	engine.relational_encoding()
\end{lstlisting}
basis calculus then based on contractions.







\subsect{Refinement by infixes}

Both the cores and the colors are further refined by infixes before the suffices to denote specific instantiations.

\begin{itemize}
	\item \selCoreIn: Involving a selection variable
	\item \eviCoreIn: Storing evidence about a variable
	\item \heaIn: Head of a function, typically the variable computed at a activation selector
	\item \funIn: Function selection variables
	\item \posIn+\stringof{i}: Variable selection for argument at position $i$
	\item \datIn: Involving data (data cores and colors)
\end{itemize}

Further infixes are strings denoting atom names and neuron neames.


\subsect{Relational encoding of formulas} % -> To application!

Propositional formulas $\exformula$ are represented in three schemes:
\begin{itemize}
	\item Script language $\synencodingof{\exformula}$ by nested lists (see \secref{subsec:scriptLanguage}).
		Most practical to choose a formula from a neuro-symbolic architecture.
	\item Strings specifying the categorical variables $\catvariableof{\exformula}$.
	\item Representation of formulas by tensor networks being contracted to $\rencodingof{\exformula}$
\end{itemize}

Conversions of the formats:
\begin{itemize}
	\item $\synencodingof{\exformula}$ to color by
		\begin{lstlisting}
			representation.get_formula_color($\synencodingof{\exformula}$)
		\end{lstlisting}
		Here the nested lists are turned in a string by concatenating all elements of a list with \stringof{\_} and adding \stringof{[} and \stringof{]} at the beginning and end of each list.
	\item  $\synencodingof{\exformula}$ to tensor network
		\begin{lstlisting}
			representation.create_raw_cores($\synencodingof{\exformula}$)
		\end{lstlisting}
		This creates the connective cores for the semantic representation of $\rencodingof{\exformula}$.
We encode them by
\end{itemize}

When encoding formulas with hard interpretation, we furthermore add a head core of type \stringof{truthEvaluation} since we have
 	\[ {\exformula} = \sbcontractionof{\rencodingof{\exformula},\tbasis}{\catvariableof{\exformula}} \, . \]



\subsect{Representation of MLNs}

\textbf{Computation Cores} are binary cores relating the variables in a predefined way, which is not changing during reasoning.
\begin{itemize}
	\item Logical interpretation: Cores $\rencodingof{\exconnective}$ \red{Structure Cores are those of the Bayesian Propositional Network}
	\item Categorical constraints: Cores $\categoricalcore$
	%\item Data: Cores $\datacore$
\end{itemize}

\textbf{Activation Cores} encode the weights of the formulas in a Markov Logic Network.
%For proper MLN only have unary cores, which we call headCores.
%Head cores with suffix "headCore" in name.

They are modified during reasoning: Selection of activation cores in structure learning, assigning a weight in parameter estimation.



\subsect{Formula Selecting Networks}

Encoding of Neurons according to \defref{def:fsNeuron}:
\begin{itemize}
	\item Activation selection core with suffix \stringof{actCore} in name.
		 Selection by variable with suffix \stringof{actVar}
	\item Selection of neurons as arguments with suffix \stringof{selCore} in name.
		Each argument of each neuron comes with a control variable with suffix \stringof{selVar}.
\end{itemize}

Encoding of Formula Selecting Neural Networks (\defref{def:fsNeuron}) by creating all formula selecting neurons.

Skeleton expression (\defref{def:skeleton}) are stored with placeholderkeys and the candidatelists by dictionaries with the placeholderkeys and values being the possible symbols.



\sect{Subpackage \spreasoning}\label{sec:implementationReasoning}


The \spreasoning subpackage implements contraction-based reasoning algorithm on representation.ComputationActivationNetworks.
%basic tensor network algorithms with calls of specific execution in \spengine.
As the \sprepresentation subpackage it is arranged in the second layer of the \tnreason architecture, since it specifies the manipulation of tensor networks in the \spengine subpackage.

\subsect{Sampling}

Sampling is performed by MCMC methods calling local sampling methods.

\begin{itemize}
	\item Tensor Network of Structure Cores
	\item Parameter cores: Variable tensor network cores representing basis vectors.
	\item List of importance cores
\end{itemize}

\begin{centeredcode}
	reasoning.Gibbs
\end{centeredcode}

\subsect{Energy-based Algorithms}

These algorithms execute reasoning tasks solely on energy dictionaries, which are created by \inlinecode{representation.ComputationActivationNetwork.get_energy_dict()}.

\begin{centeredcode}
	reasoning.NaiveMeanField
\end{centeredcode}

\begin{centeredcode}
	reasoning.GenericMeanField
\end{centeredcode}

\begin{centeredcode}
	reasoning.EnergyBasedGibbs
\end{centeredcode}




\sect{Subpackage \spapplication}\label{sec:implementationApplication}

With the \spapplication subpackage we provide an interface for reasoning workload.
It builds a third layer, since it used \sprepresentation to represent knowledge by tensor networks and \spreasoning in the execution of reasoning tasks.
%
To have a user-friendly high-level syntax of tensor-network creation a the script language (logical formulas or neuro-symbolic architectures), categorical constraints or data, is introduced.
Given a specification of a formula $\exformula$ in script language $\synencodingof{\cdot}$, the task amounts to building a semantic representation based on the syntactic specification.

\subsect{Script Language}\label{subsec:scriptLanguage}

To specify propositional sentences, neuro-symbolic architectures and Markov Logic Networks, we developed a script language.

\textbf{Propositional Sentences by Nested Lists}

%\textbf{Production Rules}
Are those of Propositional Logics, but instead of brackets we nest the symbols into lists.

% Connectives
\textbf{Connectives} are represented by strings, where the following are supported (see \defref{def:connectives}):
\begin{center}
\begin{tikzpicture}
\node [anchor=center] at (0,0) {
	\begin{tabular}{|c|c|}
  	\hline
 	\textbf{Unary connective $\exconnective$} & \textbf{$\synencodingof{\exconnective}$} \\
  	\hline
 	$\lnot$ 	&  \stringof{not} \\
  	\hline
 	$()$		&  \stringof{id} \\
  	\hline
	\end{tabular}};
\node [anchor=center] at (7,0) {
	\begin{tabular}{|c|c|}
  	\hline
 	\textbf{Binary connective $\exconnective$} & \textbf{$\synencodingof{\exconnective}$} \\
  	\hline
 	$\land$ 		&  \stringof{and} \\
  	\hline
 	$\lor$ 		&  \stringof{or} \\
  	\hline
 	$\Rightarrow$ 	&  \stringof{imp} \\
  	\hline
	 $\oplus$ 		&  \stringof{xor} \\
  	\hline
	 $\Leftrightarrow$ &  \stringof{eq} \\
  	\hline
	\end{tabular}};
\end{tikzpicture}
\end{center}

% WOLFRAM Numbers
Besides these specific connectives we exploit a generic representation scheme of propositional formulas by the so-called Wolfram code orginially designed for the classification of cellular automaton rules \cite{wolfram_statistical_1983} and popularized in the book \cite{wolfram_new_2002}.
Along this, the coordinate encodings of connectives $\exconnective$ with differing arity are flattened and interpreted as a binary number, which is transformed into a decimal number and represented as a string $\synencodingof{\exconnective}$.
We then choose a prefix to encode the arity by
\begin{itemize}
	\item \stringof{u} for unary
	\item \stringof{b} for binary
	\item \stringof{t} for ternary
	\item \stringof{q} for quarternary
\end{itemize}
connectives.
Together, the connective is represented by the string concatenation
	\[  \synencodingof{\exconnective} = \synencodingof{\catorder} + \synencodingof{\exconnective} \, . \]


% Atoms
\textbf{Atomic Formulas} are represented by arbitrary strings, which are not used for the representation of connectives.
We further avoid the symbols \{\stringof{(}, \stringof{)}, \stringof{\_}\} in the names of atoms, to not confuse them with colors of categorical variables.

% Composed Formulas
\textbf{Composed Formulas} $\exformula_1\exconnective,\exformula_2$ are represented by
\begin{centeredcode}
	$\synencodingof{\exformula_1\exconnective,\exformula_2}$ = [$\synencodingof{\exconnective}$, $\synencodingof{\exformula_1}$, $\synencodingof{\exformula_2}$]
\end{centeredcode}
where we apply the conventions
\begin{itemize}
	\item Connectives are at the 0th position in each list
	\item Further entries are either atoms as strings or encoded formulas itself
\end{itemize}

% Backus-Naur
The applied grammar in Backus-Naur form is \\
\begin{tabular}{|l|l|}
  	\hline
 	Unary Connective & \stringof{not} | \stringof{id}\\
  	\hline
 	Binary Connective & \stringof{and} | \stringof{or} | \stringof{imp} | \stringof{xor}  | \stringof{eq} \\
  	\hline
 	Atomic Formula & Set of strings not in Connectives\\
  	\hline
	Complex Formula & Atomic Formula | [Unary Connective, Complex Formula] | \\
	&  [Binary Connective, Complex Formula, Complex Formula] \\
	\hline
\end{tabular}


\begin{example}[Encoding of the Wet Street example]
For example we have
\begin{itemize}
	\item
	Atomic variable $\var{Rained}$ by
		\begin{centeredcode}
			$\synencodingof{\var{Rained}}$
			= \stringof{Rained}
		\end{centeredcode}
	\item
	Negative literal $\lnot\var{Rained}$ by
		\begin{centeredcode}
			$\synencodingof{\lnot\var{Rained}}$
			= [\stringof{not},\stringof{Rained}]
		\end{centeredcode}
	\item
	Horn clause $\left(\var{Rained}\Rightarrow\var{Wet}\right)$ by
		\begin{centeredcode}
			$\synencodingof{\var{Rained}\Rightarrow\var{Wet}}$
			= [\stringof{imp},\stringof{Rained},\stringof{Wet}]
		\end{centeredcode}
	\item
	Knowledge Base
	$(\lnot\var{Rained})\land(\var{Rained}\Rightarrow\var{Wet})$ by
		\begin{centeredcode}
			$\synencodingof{\lnot\var{Rained})\land(\var{Rained}\Rightarrow\var{Wet}}$
			=  [\stringof{and}, [\stringof{not}, \stringof{Rained}], [\stringof{imp}, \stringof{Rained}, \stringof{Wet}]]
		\end{centeredcode}
\end{itemize}
\end{example}




\textbf{Knowledge Bases}

% Should distinguish these in knowledge?
We distinguish here formulas, with propositional logic interpretation and formulas which have a soft logic interpretation.
%\textbf{Facts.}
The formulas with hard interpretation are called facts in a knowledge base $\kb$ and encoded by dictionaries
\begin{centeredcode}
	\{key($\exformula$) : $\synencodingof{\exformula}$ for $\exformula\in\kb$ \}
\end{centeredcode}

\textbf{Markov Logic Networks}

%\textbf{Weighted formulas.}
The formulas with soft interpretation are called weighted formulas and encoded by $\expof{\weightof{\exformula}\cdot\exformula}$.
We thus require a specification of the weights, which we do by adding $\weightof{\exformula}$ as a $\mathrm{float}$ or an $\mathrm{int}$ to the list $\synencodingof{\exformula}$.
We then store Markov Logic Networks by dictionaries
\begin{centeredcode}
	\{key($\exformula$) : $\synencodingof{\exformula}$ + [$\weightof{\exformula}$] for $\exformula\in\formulaset$\}
\end{centeredcode}

\textbf{Neuro-Symbolic Architecture by Nested Lists}

% Generalizing the script language to specify architectures
To specify neuro-symbolic architectures in terms of formula selecting maps, as has been the subject of \charef{cha:formulaSelection} we further exploit the nested list structure of encoding propositional logics.
We replace, in each hierarchy of the nested structure each entry by a list of possible choices.
In this way, we reinterpret the list index as the choice indices $\selindex$ introduced for connective and formula selections (see \defref{def:connectiveSelector} and \ref{def:formulaSelector}).

% Neurons
A connective selector (see \defref{def:connectiveSelector}) is encoded by the list
	\begin{centeredcode}
			$\synencodingof{\exconnective}$
			= [$\synencodingof{\exconnective_{0}}$, $\ldots$, $\synencodingof{\exconnective_{\seldim\shortminus1}}$]
	\end{centeredcode}
and a formula selector (see \defref{def:formulaSelector}) by
	\begin{centeredcode}
			$\synencodingof{\fselectionmap}$
			= [$\synencodingof{\exconnective_{0}}$, $\ldots$, $\synencodingof{\exconnective_{\seldim\shortminus1}}$]
	\end{centeredcode}
A logical neuron of order $\selorder$ (see \defref{def:fsNeuron}), defined by a connective selector $\exconnective$, and a formula selector $\fselectionmap_\atomenumerator$ on each argument $\atomenumerator\in[\selorder]$, is encoded by
		\begin{centeredcode}
			$\synencodingof{\lneuron}$
			= [$\synencodingof{\exconnective}$, $\synencodingof{\fselectionmap_0}$, $\ldots$,  $\synencodingof{\fselectionmap_{\selorder-1}}$]
		\end{centeredcode}
Only the unary $\selorder=1$ and the $\selorder=2$ cases are supported.


% Confusing?
The resulting nested lists indices have an alternating interpretation at each level compared with the elements of each list.
That is, when $\synencodingof{\lneuron}$ is the encoding of a neuron, then any element $x\in\synencodingof{\lneuron}$ represents a list of choices.
When $x$ is not the first element, then each choice is either the encoding $\synencodingof{\catvariable}$ of an atomic formula, or another neuron.

% Find another symbol?
A neural architecture $\larchitecture$ is then represented in the dictionary
\begin{centeredcode}
	$\synencodingof{\larchitecture}$ = \{key($\lneuron$) : $\synencodingof{\lneuron}$ for $\exformula\in\larchitecture$\}
\end{centeredcode}
%To store this structure, we choose dictionaries of neuron spe
%\begin{centeredcode}
%	\{key($\lneuron$) : $\synencodingof{\lneuron}$ for $\exformula\in\formulaset$\}
%\end{centeredcode}
where key($\lneuron$) is a string, which can be used in the formula selections of other neurons.

% Important for well-definedness
It is important that the directed graph of neurons induced by the choice possibilities is acyclic, to ensure well-definedness of the architecture.


% Backus-Naur
In order to represent neuro-symbolic architectures, the grammar of $\synencodingof{\cdot}$ in Backus-Naur Form is extended by the production rules \\
\begin{tabular}{|l|l|}
  	\hline
 	Unary Connectives & [Unary Connective] | [Unary Connective] + Unary Connectives \\
  	\hline
 	Binary Connectives & [Unary Connective] | [Binary Connective] + Binary Connectives \\
	%\hline
	%Neuron Name & Any set of strings not used for atoms or connectives \\
  	\hline
 	Dependency Choice & Atomic Formula | Neuron \\
  	\hline
	Dependency Choices & [Dependency Choice] | [Dependency Choice] + Dependency Choices \\
	\hline
	Neuron & [Unary Connectives, Dependency Choices] | \\
	&  [Binary Connectives, Dependency Choices, Dependency Choices] \\
	\hline
\end{tabular}


\begin{example}[Neuro-Symbolic Architecture for the Wet Street]
	Following the wet street example, we can define a neuron by
	\begin{centeredcode}
		$\synencodingof{\lneuron}$ = [[\stringof{imp},\stringof{eq}],[\stringof{Wet},\stringof{Sprinkler}],[\stringof{Street}]]
	\end{centeredcode}
	from which the formulas
	\begin{centeredcode}
		[\stringof{imp}, \stringof{Wet}, \stringof{Street}] \\
		\hspace{0.25cm} [\stringof{eq}, \stringof{Wet}, \stringof{Street}] \\
		\hspace{1cm}[\stringof{imp}, \stringof{Sprinkler}, \stringof{Street}] \\
		\hspace{1cm}[\stringof{eq}, \stringof{Sprinkler}, \stringof{Street}]
	\end{centeredcode}
	can be chosen.
	Combining this neuron with further neurons, e.g. by the architecture
	\begin{centeredcode}
		$\synencodingof{\larchitecture}$ = \{ \stringof{neur1}: [[\stringof{imp},\stringof{eq}],[\stringof{neur2}],[\stringof{Street}]] , \\
		\hspace{1.8cm}\stringof{neur2}: [[\stringof{lnot},\stringof{id}],[\stringof{Wet},\stringof{Sprinkler}],[\stringof{Street}]] \}
	\end{centeredcode}
	the expressitivity increases.
	In this case, the further neuron provides the flexibility of the first atoms to be replaced by its negation.
\end{example}



\subsect{Distributions}

We encode Markov Networks by specifying a set of tensor cores.
Each distribution needs to have a routine
\begin{lstlisting}
	.create_cores()
\end{lstlisting}
creating the factor cores and 
\begin{lstlisting}
	.get_partition_function()
\end{lstlisting}
calculating the partition function.
Although the partition function can be calculated by the contraction of all cores, we separate the method since there are situations where a faster calculation can be performed.


\textbf{Empirical Distributions} are distributions of sample data.
We represent the values as a CP Format of data cores as specified in \secref{sec:empDistribution}
\begin{lstlisting}
	knowledge.EmpiricalDistribution
\end{lstlisting}
Here the partition function is the number of samples used in the creation of the empirical distribution.


\textbf{HybridKnowledgeBases} are probability distributions, which are specified by propositional formulas in the script language.
\begin{lstlisting}
	knowledge.HybridKnowledgeBase
\end{lstlisting}
They are initialized with arguments
\begin{itemize}
	\item facts: Dictionary of propositional formulas stored as $\synencodingof{\exformula}$ representing hard logical constraints
	\item weightedFormulas: Dictionary of propositional formulas stored as $\synencodingof{\exformula}$+$[\weightof{\exformula}]$ representing soft logical constraints
	\item evidence: Dictionary of atomic formulas, where key are the formulas in string representation and values the certainty in $[0,1]$ (float or int) of the atom being true
	\item categoricalConstraints: Dictionary of categorical constrained, which values are lists of atomic formulas stored as strings $\synencodingof{\atomicformula}$
\end{itemize}


\subsect{Inference}

By
\begin{lstlisting}
	knowledge.InferenceProvider
\end{lstlisting}
taking a distribution from the above as argument.

% Probabilistic Queries
Probabilistic queries as specified \defref{def:queries})  by
\begin{lstlisting}
	.query(variableList, evidenceDict)
\end{lstlisting}

MAP queries by
\begin{lstlisting}
	.exact_map_query()
\end{lstlisting}
or by
\begin{lstlisting}
	.annealed_sample()
\end{lstlisting}
using Simulated Annealing (see Remark~\ref{rem:simulatedAnnealing}) to find an approximate maximum.
The second method circumvents the creation of the coordinatewise representation of the distribution and circumvents therefore, at the expense of potentially approximative solutions, a bottleneck in case of many query variables.

% Entailment Queries
Entailment from the distribution (\defref{def:entailment}) is be decided by
\begin{lstlisting}
	.ask(queryFormula, evidenceDict)
\end{lstlisting}
where queryFormula is the formula $\exformula$ to be tested for entailment in the representation $\synencodingof{\exformula}$.

% Sampling
Samples can be drawn by
\begin{lstlisting}
	.draw_samples(sampleNum, variableList, annealingPattern)
\end{lstlisting}
based on Gibbs sampling, where
\begin{itemize}
	\item sampleNum (int) gives the number of samples to be drawn
	\item variableList (list of str) defines the variables to be represented by the samples (default: all atoms in the distribution)
	\item annealingPattern specifies an annealing pattern 
\end{itemize}


\subsect{Parameter Estimation}

\textbf{EntropyMaximizer} implements Algorithm~\ref{alg:AWO}, which is motivated by the maximum entropy principle (see \secref{sec:maxEntDuality}) to optimize Markov Logic Networks.
The class  
\begin{lstlisting}
	knowledge.EntropyMaximizer
\end{lstlisting}
is initialized with the arguments
\begin{itemize}
	\item expressionsDict: Dictionary of formulas in the format $\synencodingof{\exformula}$ 
	\item satisfactionDict: Dictionary of the satisfaction rates (mean parameters) to be matched by the optimal distribution
\end{itemize}
The optimization is then performed by
\begin{lstlisting}
	.alternating_optimization(sweepNum, updateKeys)
\end{lstlisting}
method, where the iteration in Algorithm~\ref{alg:AWO} through the updateKeys is performed sweepNum times.

\subsect{Structure Learning}

\textbf{Formula Booster} chooses a formula given a formula selecting map.
\begin{lstlisting}
	knowledge.FormulaBooster
\end{lstlisting}
is initialized with the arguments
\begin{itemize}
	\item knowledgeBase: Distribution representing a current model to be improved
	\item specDict: A neuro-symbolic architecture encoded in a dictionary of neurons
\end{itemize}




    \bibliographystyle{plainnat}
    \bibliography{literature/tensor_logic}


\end{document}