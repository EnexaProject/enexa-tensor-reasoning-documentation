\documentclass[aps,onecolumn,nofootinbib,pra]{article}

\usepackage{arxiv}
\usepackage{amsmath,amsfonts,amssymb,amsthm,bbm,graphicx,enumerate,times}
\usepackage{mathtools}
\usepackage[usenames,dvipsnames]{color}
\usepackage{hyperref}
\hypersetup{
	breaklinks,
	colorlinks,
	linkcolor=gray,
	citecolor=gray,
	urlcolor=gray,
	pdftitle={The Tensor Network Approach to Efficient and Explainable AI},
	pdfauthor={Alex Goessmann}
}

\usepackage{tikz}
\usepackage{graphicx}
\usepackage{float}
\usepackage{comment}
\usepackage{csquotes}

\usepackage{listings}
\usepackage{verbatim}
\usepackage{etoolbox}
\usepackage{braket}
\usepackage[utf8]{inputenc}
\usepackage[english]{babel}
\usepackage[T1]{fontenc}
\usepackage{amsmath}
\usepackage{amsfonts}
\usepackage{amssymb}
\usepackage{amsthm}
\usepackage{titlesec}
\usepackage{tikz}
\usepackage{mathtools}
\usepackage{fancyhdr}
\usepackage{bbm}
\usepackage{bm}
\usepackage{algpseudocode}
\usepackage{algorithm}
\usepackage{lipsum}

\newtheorem{remark}{Remark}
\newtheorem{theorem}{Theorem}
\newtheorem{lemma}{Lemma}
\newtheorem{corollary}{Corollary}
\newtheorem{definition}{Definition}
\newtheorem{example}{Example}

\newcommand{\var}[1]{\text{\emph{#1}}}


\newcommand{\synencodingof}[1]{S\left(#1\right)} % Syntax encoding!
\newcommand{\stringof}[1]{"#1"}

% Text Macros
\newcommand{\python}{$\mathrm{python}$ }
\newcommand{\tnreason}{$\mathrm{tnreason}$ }
\newcommand{\spengine}{$\mathrm{engine}$ }
\newcommand{\spencoding}{$\mathrm{encoding}$ }
\newcommand{\spalgorithms}{$\mathrm{algorithms}$ }
\newcommand{\spknowledge}{$\mathrm{knowledge}$ }


\newcommand{\layeronespec}{\textbf{Layer 1}: Storage and manipulations}
\newcommand{\layertwospec}{\textbf{Layer 2}: Specification of workload}
\newcommand{\layerthreespec}{\textbf{Layer 3}: Applications in reasoning}

\newcommand{\rdf}{\mathrm{RDF}}
\newcommand{\rdftype}{$\mathrm{rdf}$:$\mathrm{type}$}

\newcommand{\true}{$\mathrm{true}$}
\newcommand{\truesymbol}{\mathrm{True}}
\newcommand{\false}{$\mathrm{false}$}
\newcommand{\falsesymbol}{\mathrm{False}}

\newenvironment{centeredcode}
    {\begin{center}\begin{algorithmic}\hspace{1cm}}
    {\end{algorithmic}\end{center}}

\newcommand{\algdefsymbol}{\leftarrow}

\newcommand{\probcolor}{red}
\newcommand{\concolor}{blue}
\newcommand{\inactivecolor}{gray}

\newcommand{\conjunctioncolor}{red}
\newcommand{\negationcolor}{blue}
\newcommand{\nodeminsize}{0.8cm}
\newcommand{\nodegrayscale}{gray!50}

% Basic Symbols
\newcommand{\sentropyof}[1]{\mathbb{H}\left[#1\right]}
\newcommand{\centropyof}[2]{\mathbb{H}\left[#1,#2\right]}

\newcommand{\subsphere}{\mathcal{S}}
\newcommand{\rr}{\mathbb{R}}
\newcommand{\nn}{\mathbb{N}}

\newcommand{\closureof}[1]{\overline{#1}}
\newcommand{\interiorof}[1]{{#1}^{\circ}}

\newcommand{\difofwrt}[2]{\frac{\partial #1}{\partial #2}}
\newcommand{\difwrt}[1]{\difofwrt{}{#1}}
\newcommand{\gradwrt}[1]{\nabla_{#1}}
\newcommand{\gradwrtat}[2]{\nabla_{#1}|_{#2}}

\newcommand{\cardof}[1]{\left|#1\right|}
\newcommand{\absof}[1]{\left|#1\right|}

\newcommand{\imageof}[1]{\mathrm{im}\left(#1\right)}

\newcommand{\argmin}{\mathrm{argmin}}
\newcommand{\argmax}{\mathrm{argmax}}


% Help functions
\newcommand{\chainingfunction}{h}
\newcommand{\longchainingfunction}{h}

\newcommand{\greaterzerofunction}{\psi}
\newcommand{\greaterzeroof}[1]{\greaterzerofunction\left(#1\right)}

\newcommand{\nonzerofunction}{\chi}
\newcommand{\nonzeroof}[1]{\nonzerofunction\left(#1\right)}


% Probability distributions
\newcommand{\expof}[1]{\mathrm{exp}\left[#1\right]}
\newcommand{\probtensor}{\mathbb{P}}
\newcommand{\probtensorof}[1]{\probtensor^{#1}}
\newcommand{\secprobtensor}{\tilde{\mathbb{P}}}
\newcommand{\secprobat}[2]{\secprobtensor[#1]}

\newcommand{\probtensorset}{\Gamma}

\newcommand{\gendistribution}{\probtensor^*}
\newcommand{\currentdistribution}{\tilde{\probtensor}}

\newcommand{\probof}[1]{\probat{#1}} % Bad usage!
\newcommand{\probat}[1]{\probtensor\left[#1\right]} 
\newcommand{\probofat}[2]{\probtensor^{#1}\left[#2\right]} 

\newcommand{\condprobof}[2]{\mathbb{P}\left[#1|#2\right]}
\newcommand{\margprobof}[2]{\probof{#1}}%{\mathbb{P}^{#2}\left[#1\right]} % OLD -> Use probof
\newcommand{\expectationof}[1]{\mathbb{E}\left[#1\right]}
\newcommand{\expectationofwrt}[2]{\mathbb{E}_{#2}\left[#1\right]}
\newcommand{\lnof}[1]{\ln \left[ #1 \right] }
\newcommand{\sgnormof}[1]{\left\|#1\right\|_{\psi_2}} % subgaussian
\newcommand{\normof}[1]{\left\|#1\right\|_{2}}


\newcommand{\distof}[1]{\mathbb{P}^{#1}}

\newcommand{\ones}{\mathbb{I}}
\newcommand{\onesof}[1]{\ones^{#1}}
\newcommand{\onesat}[1]{\ones\left[#1\right]}
\newcommand{\zerosat}[1]{0\left[#1\right]}
\newcommand{\identity}{\delta}
\newcommand{\identityat}[1]{\identity\left[#1\right]}
\newcommand{\dirdeltaof}[1]{\delta^{#1}}

\newcommand{\exmatrix}{M}
\newcommand{\matrixat}[1]{\exmatrix[#1]}

\newcommand{\idrestrictedto}[1]{\restrictionofto{\mathrm{Id}}{#1}}



%% KNOWLEDGE GRAPH
\newcommand{\kg}{kg}
\newcommand{\kgreptensor}{\rencodingof{\kg}}

\newcommand{\kgtriple}[3]{\braket{#1,#2,#3}}
\newcommand{\exunarytriple}{\kgtriple{?\variableof{0}}{a}{C}}
\newcommand{\exbinarytriple}{\kgtriple{?\variableof{0}}{R}{?\variableof{1}}}

\newcommand{\sparql}{\mathrm{SPARQL}}

\newcommand{\sindex}{s}
\newcommand{\pindex}{p}
\newcommand{\oindex}{o}

\newcommand{\cindex}{c}
\newcommand{\rindex}{r}

\newcommand{\resourcenumber}{\variablelegdim}
\newcommand{\indexedkgreptensor}{\kgreptensor_{:\sindex\pindex\oindex}}

\newcommand{\exreptensor}{\ftensor}


% Propositional Logics: New square bracket notation
\newcommand{\formula}{f}
\newcommand{\formulaat}[1]{\formula\left[#1\right]}
\newcommand{\formulaof}[1]{\formula_{#1}}
\newcommand{\formulaofat}[2]{\formulaof{#1}\left[#2\right]}
\newcommand{\enumformula}{\formulaof{\selindex}}
\newcommand{\enumformulaat}[1]{\formulaof{\selindex}\left[#1\right]}

\newcommand{\exformula}{\formula}
\newcommand{\formulavar}{\catvariableof{\exformula}}

\newcommand{\texformula}{\tilde{\exformula}}
\newcommand{\secexformula}{h} % Since g is atom
\newcommand{\exformulain}{\exformula\in\formulaset}
\newcommand{\exformulaof}[1]{\exformula\left(#1\right)}

\newcommand{\formulasuperset}{\mathcal{H}}


\newcommand{\exindividual}{a}
\newcommand{\secindividual}{b}
\newcommand{\exindividualof}[1]{\exindividual_{#1}}

% First order Logics
\newcommand{\folexformula}{q}
\newcommand{\headfolexformula}{\tilde{\folexformula}} % When representing \folexformula as \importancequery \rightarrow \headfolexformula
\newcommand{\folformulaset}{\mathcal{Q}}
\newcommand{\folexformulain}{\folexformula\in\folformulaset}

\newcommand{\folpredicate}{g}
\newcommand{\folpredicateof}[1]{\folpredicate_{#1}}
\newcommand{\folpredicates}{\folpredicateof{0},\ldots,\folpredicateof{\folpredicateorder-1}}
\newcommand{\folpredicateenumerator}{\atomenumerator} % Due to PL being a special case
\newcommand{\folpredicateorder}{\atomorder}

\newcommand{\worlddomain}{A}

\newcommand{\atombasemeasure}{\mu}

%\newcommand{\individualset}{\variableset}

\newcommand{\individuals}{\exindividualof{\indindexof{0}},\ldots,\exindividualof{\indindexof{\individualorder-1}}}


\newcommand{\individualsof}[1]{\exindividualof{0}^{#1},\ldots,\exindividualof{\individualorder-1}^{#1}} % Do not use, index already in individuals

%\newcommand{\individualsspace}{\bigotimes_{\individualenumeratorin}\rr^{\cardof{\individualset}}}







%% Redundant to individual variables
\newcommand{\individualvariable}{\indvariable}
\newcommand{\individualvariableof}[1]{\indvariableof{#1}}
\newcommand{\individualvariables}{\indvariablelist}


\newcommand{\individualorder}{\indorder}
\newcommand{\individualenumerator}{\indenumerator}
\newcommand{\individualenumeratorin}{\indenumeratorin}

\newcommand{\variableindex}{\indindex}
\newcommand{\variableindexof}[1]{\indindexof{#1}}
\newcommand{\variableenumerator}{\indenumerator}
\newcommand{\variableorder}{\indorder}
\newcommand{\variableenumeratorin}{\indenumeratorin}
\newcommand{\variableindices}{\indindexof{0}\ldots\indindexof{\indorder-1}}



% Tensor Representaion of functions: Outdated, since functions by definition tensors
\newcommand{\ftensor}{\theta}
\newcommand{\ftensorof}[1]{\rencodingof{#1}}

% Vector Representation of functions
%\newcommand{\vtensor}{\phi}
%\newcommand{\vtensorof}[1]{\vtensor^{#1}}



%\newcommand{\concore}{\theta} -> Relational Encoding in macros_tc
\newcommand{\concoreof}[1]{\concore^{#1}}

\newcommand{\exconnective}{\circ}
\newcommand{\connectiveof}[1]{\exconnective_{#1}}
\newcommand{\connectiveofat}[2]{\connectiveof{#1}\left[#2\right]}


\newcommand{\chadamard}{\circ}

%\newcommand{\gtensorspace}{ \rr^{\cardof{\indsetof{1}} \times \ldots \times \cardof{\indsetof{\variableorder}}}}

%\newcommand{\grounding}{\rho}

\newcommand{\dataworld}{W}
\newcommand{\dataworldat}[1]{\dataworld[#1]}

\newcommand{\groundingofatwrt}[3]{{#1}|_{#3} \left[#2\right]}
\newcommand{\groundingofat}[2]{{#1}|_{\dataworld} \left[#2\right]}
\newcommand{\groundingof}[1]{{#1}|_{\dataworld}}
\newcommand{\kggroundingof}[1]{{#1}|_{\kg}}

\newcommand{\restrictionofto}[2]{{#1}|_{#2}}

\newcommand{\gtensor}{\rho} % For decompositions
\newcommand{\gtensorof}[1]{\gtensor^{#1}}


\newcommand{\parametrization}{\rho}

%% For the TCalculus Theorem
\newcommand{\coordinatetrafo}{h}
\newcommand{\gentensor}{T}
\newcommand{\basisslices}{U}




% Parameters 


\newcommand{\candidatelist}{\mathcal{M}}
\newcommand{\candidatelistof}[1]{\candidatelist^{#1}}

\newcommand{\aselectionvariable}{L}
\newcommand{\aselectionvariableof}[1]{\aselectionvariable_{#1}}


%% OUTDATED IN MAINLINE
\newcommand{\atomlegdim}{p} %% Not needed: This is 2
\newcommand{\atomlegdimof}[1]{\atomlegdim_{#1}}
\newcommand{\atomlegset}{\mathcal{M}}
\newcommand{\atomlegsetof}[1]{\atomlegset^{#1}}



% Data Extraction Spec
\newcommand{\impformula}{p}
\newcommand{\fixedimpformula}{\underline{\impformula}}
\newcommand{\fixedimpbm}{\basemeasure_{\fixedimpformula}}
\newcommand{\supportedworlds}{\dataworld \, : \, \groundingof{\impformula} = \fixedimpformula}

\newcommand{\extformula}{q}
\newcommand{\extformulaof}[1]{\extformula_{#1}}
\newcommand{\extformulas}{\extformulaof{0},\ldots,\extformulaof{\atomorder-1}}



% Old, redundant
\newcommand{\exquery}{\impformula}
\newcommand{\fixedexquery}{\fixedimpformula}
\newcommand{\atomicqueryof}[1]{\extformulaof{#1}}
\newcommand{\atomicqueries}{\extformulas}



\newcommand{\extractionrelation}{\exrelation}

\newcommand{\variableset}{A} % Still used in monomial decomposition, NOT for object sets!
\newcommand{\variablesetof}[1]{\variableset_{#1}}


\newcommand{\variable}{\exindividual}
\newcommand{\variableof}[1]{\exindividualof{#1}}



%\newcommand{\variablelegdim}{m} % \inddim?
%\newcommand{\variablelegdimof}[1]{\variablelegdim_{#1}}
%\newcommand{\variablespace}{\rr^{\variablelegdimof{1}\times\ldots\times\variablelegdimof{\variableorder}}}

%\newcommand{\selectororder}{\atomorder}
%\newcommand{\selectorenumerator}{\atomenumerator}
%\newcommand{\selectorenumeratorin}{\selectorenumerator\in\selectororder}

%\newcommand{\datapoint}{\datamap}
%\newcommand{\atomicdatapoint}{\datapoint^{(\secexformula)}}

%\newcommand{\tensordataof}[1]{\datapointof{#1}}
%\newcommand{\evidencecore}{E}
%\newcommand{\evidencecoreof}[1]{\evidencecore^{#1}}

\newcommand{\formulaset}{\mathcal{F}}
\newcommand{\formulasetof}[1]{\formulaset_{#1}}
\newcommand{\hardformulaset}{\kb}
\newcommand{\hfbasemeasure}{\basemeasureof{\hardformulaset}}
\newcommand{\hfbasemeasureat}[1]{\hfbasemeasure\left[#1\right]}
\newcommand{\softformulaset}{\formulaset}


% Formula Selecting
\newcommand{\fselector}{\fselectionmap} % OLD, use \fselectionmap

\newcommand{\larchitecture}{\mathcal{A}}
\newcommand{\larchitectureat}[1]{\larchitecture\left[#1\right]}

\newcommand{\inneuronset}{\mathcal{A}^{\mathrm{in}}}
\newcommand{\outneuronset}{\mathcal{A}^{\mathrm{out}}}

\newcommand{\lneuron}{\sigma}
\newcommand{\lneuronof}[1]{\lneuron_{#1}}
\newcommand{\lneuronat}[1]{\lneuron\left[#1\right]}
\newcommand{\lneuractivation}{\lneuron^{\larchitecture}}
\newcommand{\lneuractivationat}[1]{\lneuractivation\left[#1\right]}

\newcommand{\fsnn}{\fselectionmapof{\larchitecture}}
\newcommand{\fsnnat}[1]{\fsnn\left[#1\right]}


\newcommand{\skeleton}{S}
\newcommand{\skeletonof}[1]{\skeleton\left(#1\right)}
\newcommand{\skeletontensor}{\ftensorof{\skeleton}} %OLD! Use skeleton

%\newcommand{\parameterspace}{\bigotimes_{\parenumeratorin} \rr^{\parlegdimof{\parenumerator}}}
\newcommand{\modelspace}{\bigotimes_{\atomenumeratorin} \rr^2}

\newcommand{\skeletoncore}{S}
\newcommand{\skeletoncoreof}[1]{\skeletoncore^{#1}}

\newcommand{\cselectionsymbol}{C}
\newcommand{\vselectionsymbol}{V}
\newcommand{\sselectionsymbol}{S}

\newcommand{\selinputvariable}{\selvariable} 
\newcommand{\cselinputvariable}{\selvariableof{\cselectionsymbol}}
\newcommand{\vselinputvariable}{\selvariableof{\vselectionsymbol}}

\newcommand{\fselectionmap}{\mathcal{H}}
\newcommand{\fselectionmapof}[1]{\fselectionmap_{#1}}
\newcommand{\fselectionmapat}[1]{\fselectionmap\left[#1\right]}
\newcommand{\fselectionmapofat}[2]{\fselectionmap_{#1}\left[#2\right]}

\newcommand{\cselectionmap}{\fselectionmapof{\cselectionsymbol}}
\newcommand{\cselectionmapat}[1]{\fselectionmapofat{\cselectionsymbol}{#1}}

\newcommand{\vselectionmap}{\fselectionmapof{\vselectionsymbol}}
\newcommand{\vselectionmapat}[1]{\fselectionmapofat{\vselectionsymbol}{#1}}

\newcommand{\sselectionmap}{\fselectionmapof{\sselectionsymbol}}
\newcommand{\sselectionmapat}[1]{\fselectionmapofat{\sselectionsymbol}{#1}}


\newcommand{\vselectionmapof}[1]{\fselectionmapof{\vselectionsymbol,#1}} % tb deleted!


\newcommand{\tranfselectionmap}{\fselectionmap^T}


% Input variables -> Use generic for target coordinates of sufficient statistics!
\newcommand{\selvariables}{\selvariableof{0},\ldots,\selvariableof{\selorder-1}}
\newcommand{\shortselvariables}{\selvariableof{[\selorder]}}

% OLD: To selection variables
\newcommand{\parorder}{\selorder} 
\newcommand{\parenumerator}{\selenumerator}
\newcommand{\parenumeratorin}{\selenumeratorin}
\newcommand{\parindex}{\selindex}
\newcommand{\parindexof}[1]{\selindexof{#1}}
\newcommand{\parindexin}{\selindex\in[\seldim]}
\newcommand{\parlegdim}{\seldim} 
\newcommand{\parlegdimof}[1]{\seldimof{#1}}



% Output variables - Following the catvariable convention
\newcommand{\seloutputvariable}{\randomx}
\newcommand{\cseloutputvariable}{\catvariableof{\cselectionsymbol}}
\newcommand{\vseloutputvariable}{\catvariableof{\vselectionsymbol}}

% Tensor Core Representation
\newcommand{\selectorcore}{{\rencodingof{\vselectionsymbol}}}
%\newcommand{\selectorcoreof}[1]{\rencodingof{\vselectionsymbol_{#1}}} 
\newcommand{\selectorcoreof}[1]{\rencodingof{\vselectionmapof{#1}}} 

\newcommand{\selectorcomponentof}[1]{\hypercoreof{\vselectionsymbol_{#1}}} % Since not an relational encoding!
\newcommand{\selectorcomponentofat}[2]{\selectorcomponentof{#1}\left[#2\right]} 

%% OLD
\newcommand{\fselectionvariable}{\selvariable}
\newcommand{\vselectionvariable}{\fselectionvariable} % For variable selection
\newcommand{\vselectionvariables}{\fselectionvariable_0,\ldots,\fselectionvariable_{\parorder-1}}
\newcommand{\cselectionvariable}{\fselectionvariable_{\exconnective}} % For connective selection
\newcommand{\selectionvariables}{\{\fselectionvariable_0,\ldots,\fselectionvariable_{\parorder-1}\}}

\newcommand{\parametercore}{\canparam}
\newcommand{\parametertensor}{\parametercore}
\newcommand{\parametertensorof}[1]{\parametertensor^{#1}}
\newcommand{\parametercoreof}[1]{\parametertensor^{#1}}

\newcommand{\parindices}{\parindexof{0}\ldots\parindexof{\parorder-1}}
\newcommand{\parindicesin}{\{\parindexof{\parenumerator} \in [\parlegdimof{\parenumerator}] \, : \, \parenumeratorin\}}
\newcommand{\parstates}{\bigtimes_{\parenumeratorin}[\parlegdimof{\parenumerator}]}
\newcommand{\parspace}{\bigotimes_{\parenumeratorin}\rr^{\parlegdimof{\parenumerator}}}
\newcommand{\simpleparspace}{\rr^{\parlegdim}}


\newcommand{\parametrizingtensor}{P^{\skeleton}}
\newcommand{\parametrizingspace}{\rr^{\atomlegdimof{1}\times\ldots\times\atomlegdimof{\atomorder}}}

\newcommand{\unitvectoratof}[2]{e^{(#1)}_{#2}}
\newcommand{\parametrizingunittensor}{e_{\atomindices}} % Not required?


\newcommand{\fulltensor}{T}
\newcommand{\fullspace}{\parametrizingspace \otimes \variablespace}


\newcommand{\placeholder}{Z} %% When not used in formulas, take the set for it
\newcommand{\placeholderof}[1]{\placeholder^{#1}}

\newcommand{\atomicformula}{\catvariable}
\newcommand{\atomicformulaof}[1]{\catvariableof{#1}}
\newcommand{\atomicformulas}{\catvariableof{[\atomorder]}} %{\{\atomicformulaof{\atomenumerator} :  \atomenumeratorin \}}
\newcommand{\enumeratedatoms}{\atomicformulaof{0},\ldots,\atomicformulaof{\atomorder-1}}
\newcommand{\atomformulaset}{\formulasetof{\mlnatomsymbol}}


\newcommand{\clause}{Z^{\lor}}
\newcommand{\clauseof}[2]{\clause_{#1,#2}}
\newcommand{\maxtermof}[1]{\clause_{#1}}
\newcommand{\maxtermformulaset}{\formulasetof{\mlnmaxtermsymbol}}

\newcommand{\term}{Z^{\land}}
\newcommand{\termof}[2]{\term_{#1,#2}}
\newcommand{\mintermof}[1]{\term_{#1}}
\newcommand{\mintermofat}[2]{\mintermof{#1}\left[#2\right]}
\newcommand{\mintermformulaset}{\formulasetof{\mlnmintermsymbol}}

\newcommand{\atomstates}{\bigtimes_{\atomenumeratorin}[2]}
\newcommand{\atomspace}{\bigotimes_{\atomenumeratorin}\rr^2}
%\newcommand{\atomsspace}{\atomspace}

\newcommand{\indexedplaceholderof}[1]{\placeholderof{#1}_{\atomlegindexof{#1}}}
\newcommand{\indexedplaceholders}{\indexedplaceholderof{1},\ldots,\indexedplaceholderof{\atomorder}}


\newcommand{\atomorder}{d}
\newcommand{\secatomorder}{r}
\newcommand{\atomenumerator}{k}
\newcommand{\secatomenumerator}{l}

\newcommand{\atomenumeratorin}{\atomenumerator\in[\atomorder]}
\newcommand{\secatomenumeratorin}{\secatomenumerator\in[\secatomorder]}
\newcommand{\atomlegindex}{\catindex}
\newcommand{\tatomlegindex}{\tilde{\atomlegindex}}
\newcommand{\atomlegindexof}[1]{\atomlegindex_{#1}}
\newcommand{\tatomlegindexof}[1]{\tatomlegindex_{#1}}
\newcommand{\atomindices}{{\atomlegindexof{0},\ldots,\atomlegindexof{\atomorder-1}}}
\newcommand{\atomindicesin}{\atomindices\in\atomstates}

\newcommand{\atombasisvector}{e}
\newcommand{\indexedatombasisvectorof}[1]{\atombasisvector^{(#1)}_{\atomlegindexof{#1}}}

\newcommand{\truevectorat}[1]{\atombasisvector_1^{#1}}
\newcommand{\falsevectorat}[1]{\atombasisvector_0^{#1}}



%% OPTIMIZATION

\newcommand{\targettensor}{Y}
\newcommand{\importancetensor}{I}




\newcommand{\affineoperator}{X}
\newcommand{\affineoperatorof}[1]{\affineoperator^{(#1)}}

%% MARKOV LOGIC NETWORK
\newcommand{\loss}{\mathcal{L}}
\newcommand{\lossof}[1]{\loss_{\datamap}\left(#1\right)}
\newcommand{\mlnformulaset}{\mathcal{F}}
\newcommand{\mlnformulain}{\exformula\in\mlnformulaset}
\newcommand{\weight}{\theta}
\newcommand{\weightof}[1]{\weight_{#1}}
\newcommand{\weightat}[1]{\weight[#1]}
\newcommand{\polynomial}{p}
\newcommand{\polynomialof}[1]{\polynomial^{#1}}

\newcommand{\mlnparameters}{(\formulaset,\canparam)}
\newcommand{\mlnparameterswithout}{(\tilde{\formulaset},\canparamt)}
\newcommand{\mlntrueparameters}{(\formulaset^*,\weight^*)}


% Examples
\newcommand{\mlnatomsymbol}{[\catorder]}
\newcommand{\mlnmintermsymbol}{\land}
\newcommand{\mlnmaxtermsymbol}{\lor}




\newcommand{\mlntensor}{\theta^*} % Should be dropped?

\newcommand{\partitionfunction}{\mathcal{Z}}
\newcommand{\secpartitionfunction}{\tilde{\mathcal{Z}}}
\newcommand{\partitionfunctionof}[1]{\partitionfunction{\left(#1\right)}}
\newcommand{\secpartitionfunctionof}[1]{\secpartitionfunction{\left(#1\right)}}


\newcommand{\mlnprob}{\probtensorof{\mlnparameters}}
\newcommand{\mlnprobat}[1]{\expdistofat{\mlnparameters}{#1}}
\newcommand{\mlnenergy}{\energytensorof{\mlnparameters}}

\newcommand{\folmlnparameters}{\folformulaset |_{\worlddomain},\weight , \basemeasure_{\fixedexquery}}

%% OLD
\newcommand{\mlnprobof}[2]{\probtensor^{(#2)}\left[#1\right]}
%\newcommand{\mlnprobabilityof}[1]{\mlnprobat{#1}}
%\newcommand{\mlnprobabilityexpof}[1]{\frac{1}{\partitionfunctionof{\mlnparameters}} \expof{\sum_{\mlnformulain}\exformula(#1)\cdot \weightof{\exformula}}}



\newcommand{\activationof}[1]{A^{#1}}




\newcommand{\variablesum}{\frac{1}{\datanum}\sum_{\variableindex=1}^\datanum}
\newcommand{\formulasum}{\sum_{\mlnformulain}}


%\newcommand{\acore}{A}
%\newcommand{\acoreof}[1]{\acore^{#1}}
%\newcommand{\aleg}{a}
%\newcommand{\alegof}[1]{\aleg_{#1}}

\newcommand{\mlntn}{\Theta}




% For Probabilistic Analysis

\newcommand{\kldivof}[2]{\mathrm{D}_{\mathrm{KL}}\left[ #1 || #2 \right]}
\newcommand{\kllossof}[1]{\kldivof{\expdistof{\ftensor^*}}{\expdistof{#1}}}

\newcommand{\expsolution}{\theta^*}
\newcommand{\empsolution}{\hat{\theta}}

\newcommand{\noisetensor}{\eta}
\newcommand{\fluctuationtensor}{\eta}
\newcommand{\expfamfluctuation}{\fluctuationtensor^{\sstat, \gendistribution, \datamap}}
\newcommand{\naivefluctuation}{\fluctuationtensor^{\identity, \gendistribution, \datamap}}
\newcommand{\proposalfluctuation}{\fluctuationtensor^{\proposalstat, \gendistribution, \datamap}}
\newcommand{\mlnfluctuation}{\fluctuationtensor^{\mlnstat, \gendistribution, \datamap}}

\newcommand{\fprob}{p}
\newcommand{\fprobof}[1]{\fprob^{(#1)}}
\newcommand{\bidistof}[1]{B\left(#1\right)}
\newcommand{\widthwrtof}[2]{\omega_{#1}(#2)}
\newcommand{\widthatof}[2]{\widthwrtof{#1}{#2}}

\newcommand{\failprob}{\epsilon}
\newcommand{\precision}{\tau}
\newcommand{\maxgap}{\delta}

% For Basis Calculus
%\newcommand{\booleanvariable}{X}
%\newcommand{\booleanvariableof}[1]{\booleanvariable^{(#1)}}

%\newcommand{\legspace}{\mathcal{H}}
%\newcommand{\legspaceof}[1]{\legspace^{(#1)}}


%% CONTRACTION 

\newcommand{\cost}{c}
\newcommand{\costof}[1]{\cost\left( #1 \right)}

\newcommand{\temperature}{t}
\newcommand{\invtemp}{\beta}
\newcommand{\currentstate}{\Theta}
\newcommand{\newstate}{\Theta^{\rm new}}


%% Hard Logic

\newcommand{\kb}{\mathcal{KB}}
\newcommand{\kbvar}{\catvariableof{\kb}}
\newcommand{\kbat}[1]{\kb\left[#1\right]}

\newcommand{\seckb}{\tilde{\kb}}
%\newcommand{\qvariable}{q} % query variable
\newcommand{\thing}{\mathrm{Thing}}
\newcommand{\nothing}{\mathrm{Nothing}}

%% TIKZ




%% Tensor Network Formats

\newcommand{\elformat}{\mathrm{EL}}
\newcommand{\cpformat}{\mathrm{CP}}

\newcommand{\htformat}{\mathrm{HT}}
\newcommand{\ttformat}{\mathrm{TT}}

\newcommand{\subspaceof}[1]{V^{#1}}

%\newcommand{\contractionof}[2]{\mathcal{C}\left(#1,#2\right) }
%\newcommand{\sbcontractionof}[2]{\contractionof{\{#1\}}{\{#2\}}}
\newcommand{\contractionof}[2]{\left\langle #1\right\rangle \left[ #2 \right]}
\newcommand{\sbcontractionof}[2]{\contractionof{#1}{#2}}
%\newcommand{\sbcontractionof}[2]{\contractionof{\left\{#1\right\}}{#2}}

\newcommand{\contraction}[1]{\contractionof{#1}{\varnothing}}
\newcommand{\sbcontraction}[1]{\contraction{#1}}
%\newcommand{\sbcontraction}[1]{\contraction{\left\{#1\right\}}}

%\newcommand{\normationofwrt}[3]{\mathcal{N}\left(#1,#2,#3\right)}
%\newcommand{\sbnormationofwrt}[3]{\normationofwrt{\{#1\}}{\{#2\}}{\{#3\}}}
\newcommand{\normationofwrt}[3]{\left\langle #1\right\rangle \left[ #2 | #3 \right]}
\newcommand{\sbnormationofwrt}[3]{\normationofwrt{\left\{#1\right\}}{#2}{#3}}

\newcommand{\normationof}[2]{\normationofwrt{#1}{#2}{\varnothing}}
\newcommand{\sbnormationof}[2]{\sbnormationofwrt{#1}{#2}{\varnothing}}

\newcommand{\extnet}{\mathcal{T}^{\graph}} % [\catvariableof{\nodes}]
\newcommand{\secextnet}{\mathcal{T}^{\tilde{\graph}}} %[\catvariableof{\nodes}]
\newcommand{\extnetat}[1]{\extnet[#1]}

\newcommand{\objof}[1]{O\left(#1\right)}

\newcommand{\nodevariables}{\catvariableof{\nodes}}
\newcommand{\edgevariables}{\catvariableof{\edge}}
\newcommand{\extnetdist}{\normationof{\extnet}{\nodevariables}}


\newcommand{\extnetasset}{\{\hypercoreof{\edge}\, : \, \edge\in\edges\}}
\newcommand{\tnetof}[1]{\mathcal{T}^{#1}}
\newcommand{\tnetofat}[2]{\tnetof{#1}\left[#2\right]}
\newcommand{\exvariable}{\randomx}

%% Probability Representation
\newcommand{\randomx}{\catvariable}

\newcommand{\exrandom}{\catvariableof{0}}
\newcommand{\secexrandom}{\catvariableof{1}}
\newcommand{\thirdexrandom}{\catvariableof{2}}

\newcommand{\exrandind}{\catindexof{0}}
\newcommand{\exranddim}{\catdimof{0}}

\newcommand{\secexrandind}{\catindexof{1}}
\newcommand{\secexranddim}{\catdimof{1}}

\newcommand{\thirdexrandind}{\catindexof{2}}

\newcommand{\randomxof}[1]{\randomx_{#1}} % In combination with atomenumerator or tenumerator
\newcommand{\randome}{E}
\newcommand{\randomeof}[1]{\randome_{#1}}
\newcommand{\tenumerator}{t}
\newcommand{\tdim}{T}
\newcommand{\tenumeratorin}{\tenumerator\in[\tdim]}

%% Exponential families
\newcommand{\expdistof}[1]{\probtensorof{#1}}
\newcommand{\expdistofat}[2]{\expdistof{#1}[#2]}
\newcommand{\expdist}{\probtensorof{(\sstat,\canparam,\basemeasure)}}
\newcommand{\expfamily}{\Gamma^{\sstat,\basemeasure}}

\newcommand{\mnexpfamily}{\Gamma^{\graph}} % The exponential family of Markov Networks on \graph
\newcommand{\mlnexpfamily}{\Gamma^{\formulaset}}

\newcommand{\basemeasure}{\nu}
\newcommand{\basemeasureof}[1]{\basemeasure^{#1}}
\newcommand{\basemeasureofat}[2]{\basemeasure^{#1}\left[#1\right]}

%\newcommand{\sstat}{\tau} % sufficent statistics
\newcommand{\sstat}{\phi}
\newcommand{\proposalstat}{\fselectionmap^T}
\newcommand{\mlnstat}{\fselectionmap}
\newcommand{\naivestat}{\mathrm{Id}}

\newcommand{\sstatcoordinateof}[1]{\sstat_{#1}}
\newcommand{\sstatcoordinateofat}[2]{\sstat_{#1}{#2}}

\newcommand{\sstatcatof}[1]{\catvariableof{\sstatcoordinateof{#1}}}
\newcommand{\sstatindof}[1]{\catindexof{\sstatcoordinateof{#1}}}
\newcommand{\sencsstat}{\sencodingof{\sstat}}
\newcommand{\sencsstatat}[1]{\sencodingof{\sstat}\left[#1\right]}

\newcommand{\canparam}{\theta}
\newcommand{\canparamat}[1]{\canparam\left[#1\right]}
\newcommand{\canparamwrtat}[2]{\canparam^{#1}\left[#2\right]}
\newcommand{\estcanparam}{\hat{\canparam}}
\newcommand{\naivecanparam}{\tilde{\canparam}}
\newcommand{\naivecanparamat}[1]{\naivecanparam\left[#1\right]}

\newcommand{\meanparam}{\mu}
\newcommand{\meanparamof}[1]{\meanparam_{#1}}
\newcommand{\meanparamat}[1]{\meanparam\left[#1\right]}
\newcommand{\meanparamofat}[2]{\meanparamof{#1}\left[#2\right]}


\newcommand{\meanrepprob}{\probtensor^{\meanparam}}

\newcommand{\meanset}{\mathcal{M}}
\newcommand{\meansetof}[1]{\meanset_{#1}}

\newcommand{\datamean}{\meanparamof{\datamap}}
\newcommand{\datameanat}[1]{\datamean\left[#1\right]}

\newcommand{\genmean}{\meanparam^*}
\newcommand{\genmeanat}[1]{\genmean[#1]}

\newcommand{\currentmean}{\tilde{\meanparam}}


\newcommand{\cumfunctionwrt}[1]{A^{#1}}
\newcommand{\cumfunctionwrtof}[2]{\cumfunctionwrt{#1}(#2)}
\newcommand{\cumfunction}{\cumfunctionwrt{(\sstat,\basemeasure)}}
\newcommand{\cumfunctionof}[1]{\cumfunction(#1)}
\newcommand{\dualcumfunction}{\big(\cumfunction\big)^*}

\newcommand{\forwardmapwrt}[1]{F^{#1}}
\newcommand{\forwardmap}{\forwardmapwrt{(\sstat,\basemeasure)}}
\newcommand{\forwardmapwrtof}[2]{\forwardmapwrt{#1}(#2)}
\newcommand{\forwardmapof}[1]{\forwardmapwrtof{(\sstat,\basemeasure)}{#1}}

\newcommand{\backwardmapwrt}[1]{B^{#1}}
\newcommand{\backwardmap}{\backwardmapwrt{(\sstat,\basemeasure)}}
\newcommand{\backwardmapwrtof}[2]{\backwardmapwrt{#1}(#2)}
\newcommand{\backwardmapof}[1]{\backwardmapwrtof{(\sstat,\basemeasure)}{#1}}


\newcommand{\energytensor}{E}
\newcommand{\energytensorofat}[2]{\energytensor^{#1}[#2]}
\newcommand{\energytensorof}[1]{\energytensor^{#1}}
\newcommand{\energytensorat}[1]{\energytensor\left[#1\right]}
\newcommand{\expenergy}{\energytensorofat{(\sstat,\canparam,\basemeasure)}{\shortcatvariables}}
\newcommand{\expenergyat}[1]{\energytensorofat{(\sstat,\canparam,\basemeasure)}{#1}}

\newcommand{\energyhypothesis}{\Theta}
%\newcommand{\energyhypothesisat}[1]{\energyhypothesis^{#1}}
\newcommand{\energyhypothesisof}[1]{\energyhypothesis^{#1}}

\newcommand{\statenumerator}{\selindex} % Since statistics are selected by a single selection variable!
\newcommand{\statorder}{\seldim}  
\newcommand{\statenumeratorin}{\selindexin}
\newcommand{\parameterspace}{\rr^\seldim}

\newcommand{\essdistof}[1]{\mathbb{E}\sstat_{#1}} % expected sufficient statistics on distributions
\newcommand{\esspar}{\mathrm{ess}}
\newcommand{\essparof}[1]{\esspar\left(#1\right)} % expected sufficient statistics on distributions


%% Logical Reasoning
\newcommand{\kcore}{K}
\newcommand{\kcoreof}[1]{\kcore^{#1}}

\newcommand{\tbasis}{e_1}
\newcommand{\fbasis}{e_0}
\newcommand{\nbasis}{\ones}




\newcommand{\graph}{\mathcal{G}}
\newcommand{\graphof}[1]{\graph^{#1}}
\newcommand{\secgraph}{\tilde{\graph}}
\newcommand{\nodes}{\mathcal{V}}
\newcommand{\nodesof}[1]{\nodes^{#1}}
\newcommand{\innodes}{\nodesof{\mathrm{in}}}
\newcommand{\outnodes}{\nodesof{\mathrm{out}}}

\newcommand{\prenodes}{\{\secnode \, : \, \secnode \prec \node, \secnode\neq\node\}}
\newcommand{\afternodes}{\{\secnode \, : \, \node \prec \secnode, \secnode\neq\node\}}

\newcommand{\incomingnodes}{\edge^{\mathrm{in}}}
\newcommand{\outgoingnodes}{\edge^{\mathrm{out}}}

\newcommand{\nodesa}{A}
\newcommand{\nodesb}{B}
\newcommand{\nodesc}{C}

\newcommand{\secnodes}{\tilde{\nodes}}
\newcommand{\thirdnodes}{\bar{\nodes}}
\newcommand{\node}{v}
\newcommand{\nodein}{\node\in\nodes}
\newcommand{\secnode}{\tilde{\node}}
\newcommand{\thirdnode}{\bar{\node}}

\newcommand{\edges}{\mathcal{E}}
\newcommand{\edgesof}[1]{\edges^{#1}}
\newcommand{\secedges}{\tilde{\edges}}
\newcommand{\edge}{e}
\newcommand{\secedge}{\tilde{\edge}}
\newcommand{\thirdedge}{\hat{\edge}}
\newcommand{\edgein}{\edge\in\edges}

\newcommand{\parentsof}[1]{\mathrm{Pa}(#1)}
\newcommand{\nondescendantsof}[1]{\mathrm{NonDes}(#1)}

\newcommand{\extensor}{T}
\newcommand{\extensorspace}{\bigotimes_{\node\in\nodes}\rr^{\catdimof{\node}}}
\newcommand{\hypercore}{\extensor}
\newcommand{\hypercoreat}[1]{\extensor\left[#1\right]}

\newcommand{\hypercoreof}[1]{\hypercore^{#1}}
\newcommand{\hypercoreofat}[2]{\hypercoreof{#1}\left[#2\right]}
\newcommand{\sechypercore}{\tilde{\extensor}}
\newcommand{\sechypercoreat}[1]{\sechypercore\left[#1\right]}

%% Factored System

\newcommand{\onehotmap}{e}
\newcommand{\onehotmapof}[1]{\onehotmap_{#1}}
\newcommand{\onehotmapofat}[2]{\onehotmap_{#1}\left[#2\right]}
\newcommand{\onehotmapto}[1]{\onehotmapof{#1}} % For encoding of sets, relations
\newcommand{\invonehotmapof}[1]{\onehotmap^{-1}(#1)}

\newcommand{\sembedding}{\phi} % Selection Embedding
\newcommand{\sembeddingof}[1]{\sembedding^{#1}}

\newcommand{\facsystem}{\mathcal{F}}





\newcommand{\exfunction}{f}
\newcommand{\exfunctiontargetspace}{\bigotimes_{l\in[p]}\rr^{\catdimof{l}}}
\newcommand{\exfunctiontargetvariables}{Y_0,\ldots,Y_{p-1}}

\newcommand{\secexfunction}{g}

\newcommand{\catleg}{\randomx}
\newcommand{\catlegof}[1]{\catleg_{#1}}


%% Message Passing
\newcommand{\cluster}{C}
\newcommand{\clusterof}[1]{\cluster_{#1}}
\newcommand{\clusterenumerator}{i}
\newcommand{\secclusterenumerator}{j}
\newcommand{\thirdclusterenumerator}{\tilde{j}}

\newcommand{\enc}{\clusterof{\clusterenumerator}}
\newcommand{\secenc}{\clusterof{\secclusterenumerator}}
\newcommand{\thirdenc}{\clusterof{\thirdclusterenumerator}}

\newcommand{\clusterorder}{n}
\newcommand{\clusterenumeratorin}{\clusterenumerator\in[\clusterorder]}

\newcommand{\upmes}[2]{\delta_{#1 \rightarrow #2}}
\newcommand{\downmes}[2]{\delta_{#2 \leftarrow #1}}

% OLD and not used in main document
%\newcommand{\resources}{R}
%\newcommand{\stortensor}{T}
%\newcommand{\cmmatrix}{C}
%\newcommand{\emb}{\tau}

% Binary connective symbols
\newcommand{\eqbincon}{\Leftrightarrow}
\newcommand{\lpasbincon}{\triangleleft}

\newcommand{\indexinterpretationof}[1]{I^{#1}}
\newcommand{\indexinterpretationofat}[2]{\indexinterpretationof{#1}(#2)}

%% CONTRACTIONS
\newcommand{\contractionof}[2]{\left\langle #1\right\rangle \left[ #2 \right]}

\newcommand{\breakablecontractionof}[2]{\big\langle #1 \big\rangle \big[ #2 \big]}
\newcommand{\sbcontractionof}[2]{\contractionof{#1}{#2}}
\newcommand{\contraction}[1]{\contractionof{#1}{\varnothing}}
\newcommand{\sbcontraction}[1]{\contraction{#1}}
\newcommand{\normationofwrt}[3]{\left\langle #1\right\rangle \left[ #2 | #3 \right]}
\newcommand{\sbnormationofwrt}[3]{\normationofwrt{#1}{#2}{#3}}
\newcommand{\normationof}[2]{\normationofwrt{#1}{#2}{\varnothing}}
\newcommand{\sbnormationof}[2]{\sbnormationofwrt{#1}{#2}{\varnothing}}

\newcommand{\nzcontractionof}[2]{\nonzerocirc\contractionof{#1}{#2}}

%% ENCODING SCHEMES
\newcommand{\rencodingof}[1]{\rho^{#1}}
\newcommand{\rencodingofat}[2]{\rencodingof{#1}\left[#2\right]}

\newcommand{\sencodingof}[1]{\gamma^{#1}}
\newcommand{\sencodingofat}[2]{\sencodingof{#1}\left[#2\right]}  

\newcommand{\bencodingof}[1]{\beta^{#1}}
\newcommand{\bencodingofat}[2]{\bencodingof{#1}\left[#2\right]}  

\newcommand{\linmap}{F}
\newcommand{\linmapof}[1]{\linmap^{#1}}
\newcommand{\linmapofat}[2]{\linmapof{#1}\left(#2\right)}
\newcommand{\linmapspace}{\mathbb{L}}

%% Directed Tensor Calculus
\newcommand{\exrelation}{\mathcal{R}}
\newcommand{\exrelationof}[1]{\exrelation^{#1}}
\newcommand{\arbset}{\mathcal{U}}
\newcommand{\arbsetof}[1]{\arbset^{#1}}
\newcommand{\arbsubset}{\mathcal{V}} % Conflicts with nodes!
\newcommand{\arbelement}{u}
\newcommand{\arbelementin}{\arbelement\in\arbset}

\newcommand{\insymbol}{\mathrm{in}}
\newcommand{\outsymbol}{\mathrm{out}}
\newcommand{\inset}{\arbsetof{\insymbol}}
\newcommand{\outset}{\arbsetof{\outsymbol}}

\newcommand{\incatindex}{\catindexof{\insymbol}}
\newcommand{\outcatindex}{\catindexof{\outsymbol}}

%% Sparse Tensor Calculus
\newcommand{\sparsityof}[1]{\ell_0\left(#1\right)}


%% MAIN VARIABLES
\newcommand{\indvariable}{O} 
\newcommand{\inddim}{r}
\newcommand{\indindex}{o} % was s
\newcommand{\indenumerator}{l}
\newcommand{\indorder}{n} % number of variables 

\newcommand{\selvariable}{L} 
\newcommand{\seldim}{p}
\newcommand{\selindex}{l}
\newcommand{\selenumerator}{s}
\newcommand{\selorder}{n}

\newcommand{\catvariable}{X} 
\newcommand{\catdim}{m}
\newcommand{\catindex}{x} % was i
\newcommand{\catenumerator}{\atomenumerator}
\newcommand{\catorder}{\atomorder}

\newcommand{\headvariable}{Y} % head of a relational encoding
\newcommand{\headdim}{n}
\newcommand{\headindex}{y}

\newcommand{\datvariable}{J} % Can be understood as selvariable selecting a datapoint, catvariable as a random datapoint, indvariable as a abstract object representing the sample, also used at indexvariable!
\newcommand{\datdim}{m}
\newcommand{\datindex}{j}

%% Syntactic Sugar
\newcommand{\indvariableof}[1]{\indvariable_{#1}}
\newcommand{\selvariableof}[1]{\selvariable_{#1}}
\newcommand{\catvariableof}[1]{\catvariable_{#1}}
\newcommand{\headvariableof}[1]{\headvariable_{#1}}

\newcommand{\indvariablelist}{\indvariableof{0},\ldots,\indvariableof{\individualorder-1}}
\newcommand{\catvariablelist}{\catvariableof{0},\ldots,\catvariableof{\atomorder-1}}
\newcommand{\selvariablelist}{\selvariableof{0},\ldots,\selvariableof{\selorder-1}}

\newcommand{\shortindvariablelist}{\indvariableof{[\individualorder]}}
\newcommand{\shortcatvariablelist}{\catvariableof{[\atomorder]}}
\newcommand{\shortselvariablelist}{\selvariableof{[\selorder]}}

\newcommand{\selindices}{\selindexof{0},\ldots,\selindexof{\selorder-1}}

\newcommand{\shortindindices}{\indindexof{[\indorder]}}
\newcommand{\shortcatindices}{\catindexof{[\catorder]}}
\newcommand{\shortselindices}{\selindexof{[\selorder]}}

\newcommand{\inddimof}[1]{\inddim_{#1}}
\newcommand{\seldimof}[1]{\seldim_{#1}}
\newcommand{\catdimof}[1]{\catdim_{#1}}
\newcommand{\headdimof}[1]{\headdim_{#1}}

\newcommand{\indindexof}[1]{\indindex_{#1}}
\newcommand{\selindexof}[1]{\selindex_{#1}}
\newcommand{\catindexof}[1]{\catindex_{#1}} 
\newcommand{\headindexof}[1]{\headindex_{#1}}

\newcommand{\indindexin}{\indindex\in[\inddim]}
\newcommand{\selindexin}{\selindex\in[\seldim]}
\newcommand{\catindexin}{\catindex\in[\catdim]}
\newcommand{\datindexin}{\datindex\in[\datdim]}

\newcommand{\indindexofin}[1]{\indindexof{#1}\in[\inddimof{#1}]}
\newcommand{\catindexofin}[1]{\catindexof{#1}\in[\catdimof{#1}]}
\newcommand{\selindexofin}[1]{\selindexof{#1}\in[\seldimof{#1}]}

\newcommand{\indindexlist}{\indindexof{0},\ldots,\indindexof{\indorder-1}}
\newcommand{\catindexlist}{\catindexof{0},\ldots,\catindexof{\atomorder-1}}
\newcommand{\selindexlist}{\selindexof{0},\ldots,\selindexof{\selorder-1}}

\newcommand{\indenumeratorin}{\indenumerator\in[\indorder]}
\newcommand{\selenumeratorin}{\selenumerator\in[\selorder]}
\newcommand{\catenumeratorin}{\catenumerator\in[\catorder]}

\newcommand{\indexedindvariableof}[1]{\indvariableof{#1}=\indindexof{#1}}
\newcommand{\indexedcatvariableof}[1]{\catvariableof{#1}=\catindexof{#1}}
\newcommand{\indexedselvariableof}[1]{\selvariableof{#1}=\selindexof{#1}}
\newcommand{\indexedheadvariableof}[1]{\headvariableof{#1}=\headindexof{#1}}

\newcommand{\indexedindvariable}{\indexedindvariableof{}}
\newcommand{\indexedcatvariable}{\indexedcatvariableof{}}
\newcommand{\indexedselvariable}{\indexedselvariableof{}}

\newcommand{\catstatesof}[1]{[\catdimof{#1}]}

\newcommand{\catspaceof}[1]{\rr^{\catdimof{#1}}}

\newcommand{\indspace}{\bigotimes_{\indenumeratorin}\rr^{\inddim}}
\newcommand{\catspace}{\bigotimes_{\atomenumeratorin} \rr^{\catdimof{\atomenumerator}}}

\newcommand{\selstates}{\bigtimes_{\selenumeratorin}[\seldimof{\selenumerator}]}
\newcommand{\selvectorspace}{\rr^{\seldim}}
\newcommand{\selspace}{\bigotimes_{\selenumeratorin}\rr^{\seldimof{\selenumerator}}}
%%

\newcommand{\datanum}{\datdim}

\newcommand{\datain}{\datindex\in[\datanum]}
\newcommand{\data}{\{\datapointof{\datindex}\}_{\datindexin}}
\newcommand{\dataaverage}{\frac{1}{\datanum}\sum_{\datindexin}}

\newcommand{\catvariables}{\catvariablelist}
\newcommand{\shortcatvariables}{\shortcatvariablelist}
\newcommand{\indexedshortcatvariables}{\shortcatvariables=\shortcatindices}
\newcommand{\shortcatindicesin}{\shortcatindices\in\facstates}
\newcommand{\shortatomindicesin}{\shortcatindices\in\atomstates}
\newcommand{\datshortcatvariables}{\shortcatvariables=\shortcatindices^{\datindex}}

\newcommand{\shortindvariables}{\shortindvariablelist}
\newcommand{\indexedshortindvariables}{\shortindvariables=\shortindindices}
\newcommand{\datshortindvariables}{\shortindvariables=\shortindindices^{\datindex}}

\newcommand{\selvariables}{\selvariableof{0},\ldots,\selvariableof{\selorder-1}}
\newcommand{\shortselvariables}{\selvariableof{[\selorder]}}
\newcommand{\indexedshortselvariables}{\shortselvariables=\shortselindices}
\newcommand{\secselenumerator}{\tilde{\selenumerator}}

\newcommand{\nodestatesof}[1]{\bigtimes_{\node\in#1}\catstatesof{\node}}
\newcommand{\atomstates}{\bigtimes_{\atomenumeratorin}[2]}


\newcommand{\symindstates}{\bigtimes_{\indenumeratorin}[\inddim]}

\newcommand{\facstates}{\bigtimes_{\atomenumeratorin}\catstatesof{\atomenumerator}}
\newcommand{\facdim}{\prod_{\atomenumeratorin}\catdimof{\atomenumerator}}
\newcommand{\secfacstates}{\bigtimes_{\secatomenumerator\in[\secatomorder]}\catstatesof{\secatomenumerator}}

\newcommand{\atomspace}{\bigotimes_{\atomenumeratorin}\rr^2}
\newcommand{\facspace}{\catspace}
\newcommand{\secfacspace}{\bigotimes_{\secatomenumerator\in[\seccatorder]} \rr^{\catdimof{\secatomenumerator}}}

\newcommand{\indexedcatvariables}{\indexedcatvariableof{0},\ldots,\indexedcatvariableof{\atomorder-1}} 
\newcommand{\tildeindexedcatvariables}{\catvariableof{0}=\tilde{\catindex}_0,\ldots,\catvariableof{\atomorder-1}=\tilde{\catindex}_{\atomorder-1}} 

\newcommand{\seccatenumerator}{\tilde{\catenumerator}}
\newcommand{\seccatenumeratorin}{\seccatenumerator\in[\catorder]}

\newcommand{\seccatvariable}{Y} % used as differentiation variable
\newcommand{\seccatindex}{y}
\newcommand{\seccatorder}{p} % Has to coincide with seldim for relational encoding def

\newcommand{\seccatvariableof}[1]{\seccatvariable_{#1}}
\newcommand{\indexedseccatvariableof}[1]{\seccatvariableof{#1}=\seccatindexof{#1}}
\newcommand{\seccatvariables}{\seccatvariableof{0},\ldots,\seccatvariableof{\seccatorder\shortminus1}}
\newcommand{\secshortcatvariables}{\seccatvariableof{[\seccatorder]}}
\newcommand{\indexedseccatvariables}{\indexedseccatvariableof{0}\ldots,\indexedseccatvariableof{\seccatorder-1}} 
\newcommand{\indexedsecshortcatvariables}{\indexedseccatvariableof{[\seccatorder]}}

\newcommand{\catvariablesinset}[1]{\catvariableof{#1}}%{\catvariableof{\node} \, : \, \node \in #1}
\newcommand{\seccatindexof}[1]{\seccatindex_{#1}}

\newcommand{\catindices}{\catindexlist}
\newcommand{\tildecatindexof}[1]{\tilde{\catindex}_{#1}}
\newcommand{\tildecatindices}{\tildecatindexof{0},\ldots,\tildecatindexof{\atomorder-1}}
\newcommand{\seccatindices}{{\seccatindexof{0},\ldots,\seccatindexof{\secatomorder-1}}}
\newcommand{\tildeshortcatindices}{\tildecatindexof{[\catorder]}}

\newcommand{\catindicesof}[1]{{\catindexof{0}^{#1},\ldots,\catindexof{\atomorder-1}^{#1}}}

\newcommand{\catzeropositions}{\{\atomenumerator : \catindexof{\atomenumerator}=0\}}
\newcommand{\catonepositions}{\{\atomenumerator : \catindexof{\atomenumerator}=0\}}

%% Cores
\newcommand{\categoricalmap}{Z}
\newcommand{\categoricalmapat}[1]{\categoricalmap\left[#1\right]}
\newcommand{\categoricalmapof}[1]{\categoricalmap^{#1}}
\newcommand{\categoricalmapofat}[2]{\categoricalmap^{#1}\left[#2\right]}

\newcommand{\categoricalcore}{\rencodingof{\categoricalmap}}
\newcommand{\categoricalcoreof}[1]{\rencodingof{\categoricalmapof{#1}}}
\newcommand{\categoricalcoreofat}[2]{\rencodingof{\categoricalmapof{#1}}\left[#2\right]}

\newcommand{\datamap}{D}
\newcommand{\datamapat}[1]{\datamap(#1)}
\newcommand{\datamapof}[1]{\datamap_{#1}}
\newcommand{\datapointof}[1]{\datamapat{#1}}
\newcommand{\datapoint}{\datapointof{\datindex}}
\newcommand{\dataset}{\left((\catindicesof{\datindex})\,:\,\datindexin\right)}

\newcommand{\secdatamap}{\tilde{\datamap}}
\newcommand{\datacore}{\rencodingof{\datamap}}
\newcommand{\datacoreat}[1]{\datacore\left[#1\right]}
\newcommand{\datacoreof}[1]{\rencodingof{\datamap_{#1}}}
\newcommand{\datacoreofat}[2]{\rencodingof{\datamap_{#1}}[#2]}

\newcommand{\secdatacoreof}[1]{\rencodingof{\secdatamap_{#1}}}
\newcommand{\empdistribution}{\probtensor^{\datamap}}
\newcommand{\empdistributionat}[1]{\empdistribution\left[#1\right]}
\newcommand{\empdistributionwith}{\empdistributionat{\shortcatvariables}}

\newcommand{\varcore}[1]{U^{#1}} % For optimization of tensor network
\newcommand{\varspace}[1]{\rr^{p_{#1}}}
\newcommand{\varcollection}{\big\{\varcore{\atomenumerator}\, :\, \atomenumeratorin \big\}}

\newcommand{\headcore}{W} % activation of a formula, typical exp
\newcommand{\headcoreof}[1]{\headcore^{#1}}
\newcommand{\headcoreat}[1]{\headcore\left[#1\right]}
\newcommand{\headcoreofat}[2]{\headcore^{#1}[#2]}

\newcommand{\actcore}{W} % activation of a formula, typical exp
\newcommand{\actcoreof}[1]{\actcore^{#1}}
\newcommand{\actcoreofat}[2]{\actcore^{#1}[#2]}

%% CP Decomposition
\newcommand{\legcore}{V}
\newcommand{\legcoreof}[1]{\legcore^{#1}}
\newcommand{\legcoreofat}[2]{\legcoreof{#1}\left[#2\right]}

\newcommand{\scalarcore}{\sigma}
\newcommand{\scalarcoreof}[1]{\scalarcore[#1]}
\newcommand{\scalarcoreat}[1]{\scalarcore[#1]}
\newcommand{\scalarcoreofat}[2]{\scalarcore^{#1}[#2]}

% DecompositionIndex 
\newcommand{\decvariable}{I}
\newcommand{\decvariableof}[1]{\decvariable_{#1}}
\newcommand{\decindex}{i} % Needs to be different to datindex!
\newcommand{\decindexof}[1]{\decindex_{#1}}
\newcommand{\indexeddecvariableof}[1]{\decvariableof{#1}=\decindexof{#1}}
\newcommand{\decdim}{n}
\newcommand{\decdimof}[1]{\decdim_{#1}}
\newcommand{\decindexin}{\decindex\in[\decdim]}
\newcommand{\indexeddecvariable}{\decvariable=\decindex}
\newcommand{\inddecvar}{\indexeddecvariable}

\newcommand{\indexeddatvariable}{\datvariable=\datindex}

% Used in poly sparsity
\newcommand{\indexvariable}{\datvariable} % for datacores used
\newcommand{\indexset}{J}
\newcommand{\indexsetof}[1]{\indexset^{#1}}

\newcommand{\slackvariable}{z}
\newcommand{\slackindex}{z}
\newcommand{\slackindexof}[1]{\slackindex_{#1}}

\newcommand{\rankofat}[2]{\mathrm{rank}^{#1}\left(#2\right)}
\newcommand{\cprankof}[1]{\mathrm{rank}\left(#1\right)}
\newcommand{\bincprankof}[1]{\mathrm{rank}^{\mathrm{bin}}\left(#1\right)}
\newcommand{\slicesparsityof}[1]{\slicerankwrtof{\catorder}{#1}} % former {\tilde{\ell} \left(#1\right)}


\newcommand{\dircprankof}[1]{\mathrm{rank}^{\mathrm{dir}}\left(#1\right)}
\newcommand{\bascprankof}[1]{\mathrm{rank}^{\mathrm{bas}}\left(#1\right)}
\newcommand{\baspluscprankof}[1]{\mathrm{rank}^{\mathrm{bas+}}\left(#1\right)}
\newcommand{\quacprankof}[1]{\mathrm{rank}^{\mathrm{qua}}\left(#1\right)}

\newcommand{\sliceset}{\mathcal{M}}
\newcommand{\slicescalar}{\lambda}
\newcommand{\slicescalarof}[1]{\slicescalar^{#1}}
\newcommand{\slicetupleof}[1]{(\slicescalar^{#1}, \variablesetof{#1}, \catindexof{\variablesetof{#1}}^{#1})}
\newcommand{\enumeratedslices}{\{\slicetupleof{\decindex} \, : \, \decindexin\}}

\newcommand{\sliceorder}{r}
\newcommand{\slicerankwrtof}[2]{\mathrm{rank}^{#1}\left(#2\right)}

\usetikzlibrary {arrows.meta} 
\usetikzlibrary{shapes,positioning}
\usetikzlibrary{decorations.markings}
\tikzset{
    midarrow/.style={
        postaction={decorate},
        decoration={markings, mark=at position 0.5 with {\arrow{>}}}
    },
    midbackarrow/.style={
        postaction={decorate},
        decoration={markings, mark=at position 0.5 with {\arrow{<}}}
    },
     ->/.style={midarrow},
     <-/.style={midbackarrow}
}

\newcommand{\skeletoncolor}{blue}
\newcommand{\arrowstyle}{->}

\newcommand{\shortminus}{\scalebox{0.4}[1.0]{$-$}}

\newcommand{\drawvariabledot}[2]{
	\draw[fill] (#1,#2) circle (0.15cm);
}

% Draws indices and below the indices the core
\newcommand{\drawatomindices}[2]{
	\begin{scope}[shift={(#1,#2)}]
		\draw[<-] (0,1)--(0,-1) node[midway,left] {\tiny $\catvariableof{0}$}; 
		\draw[<-] (1.5,1)--(1.5,-1) node[midway,left] {\tiny $\catvariableof{1}$}; 
		\node[anchor=center] (text) at (3,0) {$\cdots$};
		\draw[<-] (4,1)--(4,-1) node[midway,right] {\tiny $\catvariableof{\atomorder\shortminus1}$}; 
	\end{scope}
}
\newcommand{\drawundiratomindices}[2]{
	\begin{scope}[shift={(#1,#2)}]
		\draw[] (0,1)--(0,-1) node[midway,left] {\tiny $\catvariableof{0}$}; 
		\draw[] (1.5,1)--(1.5,-1) node[midway,left] {\tiny $\catvariableof{1}$}; 
		\node[anchor=center] (text) at (3,0) {$\cdots$};
		\draw[] (4,1)--(4,-1) node[midway,right] {\tiny $\catvariableof{\atomorder\shortminus1}$}; 
	\end{scope}
}
\newcommand{\drawparindices}[2]{
	\begin{scope}[shift={(#1,#2)}]
		\draw (0,1)--(0,-1) node[midway,left] {\tiny $\parindexof{1}$}; 
		\draw (1.5,1)--(1.5,-1) node[midway,left] {\tiny $\parindexof{2}$}; 
		\node[anchor=center] (text) at (3,0) {$\cdots$};
		\draw (4,1)--(4,-1) node[midway,right] {\tiny $\parindexof{\parorder}$}; 
	\end{scope}
}
\newcommand{\drawatomcore}[3]{
	\begin{scope}[shift={(#1,#2)}]
		\draw (-1,-1) rectangle (5,-3);
		\node[anchor=center] (text) at (2,-2) {#3};
	\end{scope}
}
\newcommand{\drawsmallcore}[3]{
	\begin{scope}[shift={(#1,#2)}]
		\draw (-1,1) rectangle (1,-1);
		\node[anchor=center] (text) at (0,0) {#3};
	\end{scope}
}


\newcommand{\drawformulatensor}{
    \drawformulatensorof{\ftensorof{\exformula}}
    }
    
\newcommand{\drawformulatensorof}[1]{
    \draw (-5,1) rectangle (5,-1);
    \node at (0,-1) [above] {$#1$} ;
    \draw (-4,-1)--(-4,-3) node[midway,left] {$\variableindexof{1}$};
    \draw (4,-1)--(4,-3) node[midway,left] {$\variableindexof{\variableorder}$};
    \node at (0,-3.1) [above] {$\cdots$};
    }
       
    
\newcommand{\drawskeleton}{
	\renewcommand{\skeletoncolor}{}
    	\draw[\skeletoncolor,\arrowstyle] (-4,-1)--(-4,-3) node[midway,left] {$\variableindexof{1}$};
	\draw[\skeletoncolor,\arrowstyle] (-2,-1)--(-2,-3) node[midway,right] {$\variableindexof{2}$};
    	\draw[\skeletoncolor,\arrowstyle] (4,-1)--(4,-3) node[midway,left] {$\variableindexof{\variableorder}$};

	\draw[\skeletoncolor]  (-5,-3) rectangle (5,-5);
	\node[\skeletoncolor]  at (0,-5.1) [above] {$\gtensorof{\skeleton}$} ;
    	\draw[\skeletoncolor,\arrowstyle] (-4,-5)--(-4,-7) node[midway,left] {$\variableindexof{1}$};
	\draw[\skeletoncolor,\arrowstyle]  (-2,-5)--(-2,-7) node[midway,right] {$\variableindexof{2}$};
   	\draw[\skeletoncolor,\arrowstyle]  (4,-5)--(4,-7) node[midway,left] {$\variableindexof{\variableorder}$};
	\node[\skeletoncolor] at (1,-7) [above] {$\ldots$};
}
    
\newcommand{\drawatomdecomposition}{
	\draw (-5,1) rectangle (-1,-1);
    	\node at (-3,-1.1) [above] {$\gtensorof{\atomicformulaof{1}}$} ;
	\node at (0.65,-1.1) [above] {$\cdots$};
	\draw (2,1) rectangle (6,-1);
    	\node at (4,-1.1) [above] {$\gtensorof{\atomicformulaof{\atomorder}}$} ;
	\drawskeleton
  }
  


\newcommand{\drawchadamardcore}{
	\node at (0,0) [left,\skeletoncolor] {$\delta$};
	\draw[\skeletoncolor]  (0,2) -- (0,0);% node[midway,left] {$\placeholderof{1}$};
	\draw[fill,\skeletoncolor] (0,0) circle (0.25cm);
	\draw[\skeletoncolor,\arrowstyle] (0,0) -- (0,-2) node[midway,right] {$\variableindexof{2}$};
	\draw[\skeletoncolor] (4,2) to[bend left=20] (0,0);
	\draw[\skeletoncolor,\arrowstyle] (-2,2) -- (-2,-2) node[midway,left] {$\variableindexof{1}$};
}


\newcommand{\drawchadamardcoretwocontractions}{
	\node at (0,0) [left,\skeletoncolor] {$\delta$};
	\draw[\skeletoncolor]  (0,2) -- (0,0);% node[midway,left] {$\placeholderof{1}$};
	\draw[fill,\skeletoncolor] (0,0) circle (0.25cm);
	\draw[\skeletoncolor,\arrowstyle] (0,0) -- (0,-2) node[midway,right] {$\variableindexof{2}$};
	\draw[\skeletoncolor] (5,2) to[bend left=20] (0,0);


	\draw[fill,\skeletoncolor] (-2,0.5) circle (0.25cm);
	\draw[\skeletoncolor] (3,2) to[bend left=20] (-2,0.5);
	\draw[\skeletoncolor,\arrowstyle] (-2,2) -- (-2,-2) node[midway,left] {$\variableindexof{1}$};
}

  
\newcommand{\drawchadamard}{
	\begin{scope}[shift={(-7,0)}]
	  	\draw[dashed]  (-3,2) rectangle (1,4);
		\node at (-1,1.9) [above] {$\gtensorof{\exformula\land\secexformula}$};
		\draw (-2,2) -- (-2,-2) node[midway,left] {$\variableindexof{1}$};
		\draw (0,2) -- (0,-2) node[midway,right] {$\variableindexof{2}$};	
	\end{scope}
	\node at (-4.5,1.9) [above] {$=$};	
  	\draw[dashed] (-3,2) rectangle (1,4);
	\node at (-1,1.9) [above] {$\gtensorof{\exformula}$};
  	\draw[dashed] (3,2) rectangle (5,4);
	\node at (4,1.9) [above] {$\gtensorof{\secexformula}$};
  	\drawchadamardcore
  }
  
  
  \newcommand{\drawanegationcore}{
    	\begin{scope}[shift={(-2,-2)}]
  		\draw[\skeletoncolor] (-8,2) rectangle (-4,4);
		\node at (-6,2) [above,\skeletoncolor] {$\ones$};
		\draw[\skeletoncolor,\arrowstyle] (-7,2) -- (-7,0) node[midway,left] {$\variableindexof{1}$};
		\draw[\skeletoncolor,\arrowstyle] (-5,2) -- (-5,0) node[midway,right] {$\variableindexof{2}$};
	\end{scope}
	
	\draw[\skeletoncolor,\arrowstyle] (-2,2) -- (-2,-2) node[midway,left] {$\variableindexof{1}$};
	\draw[\skeletoncolor,\arrowstyle](0,2) -- (0,-2) node[midway,right] {$\variableindexof{2}$};
	\node[\skeletoncolor] at (-4.5,0) [above] {$-$};
  }
  
  \newcommand{\drawanegation}{

  	\begin{scope}[shift={(-14,0)}]
	  	\draw[dashed]  (-3,2) rectangle (1,4);
		\node at (-1,1.9) [above] {$\gtensorof{\lnot\exformula}$};
		\draw (-2,2) -- (-2,-2) node[midway,left] {$\variableindexof{1}$};
		\draw (0,2) -- (0,-2) node[midway,right] {$\variableindexof{2}$};	
	\end{scope}

	\drawanegationcore



	\node at (-11.5,1.9) [above] {$=$};	
	
  	\draw[dashed]  (-3,2) rectangle (1,4);
	\node at (-1,1.9) [above] {$\gtensorof{\exformula}$};	
  
  	  }

\pretolerance=500
\tolerance=100 
\emergencystretch=10pt

% Bibliography
\DeclareUnicodeCharacter{FB01}{fi}
\usepackage[round]{natbib}


\begin{document}



\title{The Tensor Network Approach to Efficient and Explainable AI}
\author{Alex Goessmann, DATEV eG}


\maketitle
\date{\today}

\begin{abstract}
	Tensor spaces appear naturally in factored representations of systems, when storing information about a systems state.
	Since the curse of dimensionality prevents feasible generic representations and reasoning, logical and probabilistic reasoning focuses on tradeoffs between the sparsity and the generality of reasoning.
	In this work we present these tradeoffs based on the tensor network formalism and formulate feasible reasoning algorithms based on tensor network contractions.
	We review the classical logical and probabilistic approaches to reasoning using and show applications in neuro-symbolic AI.
\end{abstract}	

\tableofcontents


\newcommand{\red}[1]{\textcolor{red}{#1}}

\section{Introduction}

% Explaining the title
Artificial intelligence is a long-standing dream of humanity, which has in recent years received enourmous attention, driven by breakthroughs in large language models.
Among the key priorities towards an economic and trustworthy usage remain the creation of efficient and the explainable models.

% Explainability
Instead of post-hoc explainability of a models inference given specific data, our aim in this work is the intrinsic human understandability of a model.
We are motivated by the theory of logic, which formalization of human thoughts serves as an interface between mechanized reasoning on a machine and human understandability.
Having established this advanced form of explainability enables novel forms of human interactions with a model based on verbalizations, manipulations and guarantees on the models inference output.

% Efficiency
The desire of an efficient model originates more from an economic perspective on the realizability of a model and its power consumption.
Tensors appear naturally as representations of a system with multiple variables, both in logical and probabilistic approaches towards artificial intelligence. % avoid factored at this point!
However, already for moderate numbers of variables, the curse of dimensionality prevents a typical machines memory to store a generic representation.
The careful design of representation formats is therefore a necessary task to avoid the exponential increase of storage demands and balance the expressivity and the efficiency of representation formats.

% Tensor Networks
We in this work exploit the formalism of tensor networks in the creation of efficient representation schemes.
The chosen tensor network formats are motivated from explainable learning architectures and provide a synergy between the aims of efficiency and explainability.
Tensor networks appear as the natural numerical structures in probabilistic graphical models and logical knowledge bases.
After presenting the probabilistic and logical approaches based on the tensor network formalism we develop novel applications schemes towards neuro-symbolic artificial intelligence.

\subsection{Background}

Before presenting an overview over the contents, we further motivate this work based on the broach approaches towards artificial intelligence and more recent developments.

\subsubsection{Classical Approached towards AI}

We start with ontological commitments in the description of a system and follow the book \cite{russell_artificial_2021} distinguishing atomic, factored and structured representations.
While in atomic representation, the states of a systems are enumerated and represented in a single variable, factored representations describe a systems state based on a collection of variables.
In the tensor formalism, each state of a system corresponds with a coordinate of a representing tensor.
The order of the tensor coincides therefore with the number of variables in a system.
In an atomic representation, where there is a single coordinate, each state corresponds with a coordinate of the representing vector being a tensor of order one.
Having a factored representation with two variables requires order two tensors or matrices, where a coordinate is specified by a row and a column index.
Given larger numbers of coordinates now extends this representation picture to tensors of larger orders, which have more abstract axes besides rows and columns.
The generalization of the atomic representation to a factored system thus corresponds with the generalization of vectors towards matrices and tensors of larger orders.
Along this line, we can always transform a factored representation of a system to an atomic one, just by enumerating the states of the factored system and interpreting them by a single variable.
This amounts to the flattening of a representing tensor to a vector.
However, by doing so, we would loose much of the structure of the representation, which we would like to exploit in reasoning processes.

% Structured Representations
A more generic representation of systems are structured representation.
Structured representations involve objects of differing numbers and relations between them.
As a consequence the numbers of variables can differ depending on the state of a system.
This poses a challenge to the tensor representation, since a fixed number of variables is required to motivate a tensor space of representations.
There are approaches to circumvent these difficulty by the development of template models such as Markov Logic Networks \cite{richardson_markov_2006}, which are instantiated on systems with differing number of objects.
We will discuss those in \charef{cha:folModels}.

% Continous vs discrete
In this work we treat discrete systems, where the number of states is finite.
One can understand them as a discretization of continuous variables and many results will generalize by the converse limit to the situation of continuous variables.

% Epistemologic
Besides ontological commitments in the choice of a representation scheme, modelling a system also requires epistemologic commitments, by defining what properties are to be reasoned about.
In logical approaches the properties of states are boolean values representing whether a state is consistent with known constraints.
Probabilistic approaches assign to the coordinates of the tensors numbers in $[0,1]$ encoding the probability of a state.
Compared with logical approached to reasoning, probabilistic approaches thus bear a more expressive modelling.

\subsubsection{Logic and Explainability in AI}

\textbf{Inductive Logic Programming:}
\begin{itemize}
    \item ILP is a classical task \cite{muggleton_inductive_1994}
    \item Amie \cite{galarraga_amie_2013} is a method of learning Horn clauses using a refinement operator.
    \item Class Expression Learning \cite{lehmann_class_2011} is a more recent approach to assist in the design of reasoning capabilities in Knowledge Graphs.
        However, problems arise from the expressivity of description logics and the efficient choice of formulas from exponentially large hypothesis sets.
    \item CEL has therefore recently received further popularity in combination with reinforcement learning \cite{demir_drill-_2021} and neural networks \cite{kouagou_neural_2022, pesquita_neural_2023}, which are methods searching efficienctly in exponentially large spaces of formulas.
\end{itemize}

\textbf{Statistical Relational AI:} \cite{getoor_introduction_2019}
\begin{itemize}
    \item Classical combination of logical and probabilistic approaches to reasoning
\end{itemize}

\textbf{Knowledge Graphs}
\cite{hogan_knowledge_2021}
\begin{itemize}
    \item The advent of large Knowledge Graphs enables explainable reasoning methods on structured data.
    \item Knowledge Graphs are stored in a sparse format, i.e. only true atoms instead of all + truth label.
\end{itemize}


\subsubsection{Tensor Networks in AI}

\textbf{Tensor Network formats}
\begin{itemize}
    \item HT Format \cite{hackbusch_new_2009}
    \item CP Format
\end{itemize}

\textbf{Tensor Networks as Regressors}
\begin{itemize}
    \item Dynamical Systems learning \cite{gels_multidimensional_2019, goesmann_tensor_2020}
    \item Supervised learning CITE: Stoudenmire etc.
\end{itemize}

\textbf{Tensor Representation of Logics}
\begin{itemize}
    \item Tensor Networks have been applied in the automatization of logic reasoning \cite{li_linear_2017, sato_linear_2017} apply Matrix multiplication in reasoning.
    \item \cite{nickel_review_2016} review over relational machine learning and latent features via matrix embeddings.
\end{itemize}

\textbf{Tensor Representation of Knowledge Graphs}
\begin{itemize}
    \item Effective representation of queries
    \item Usage of tensor networks in embeddings \cite{yang_embedding_2015} and using complex extensions \cite{trouillon_complex_2017, trouillon_knowledge_2017}
\end{itemize}

\textbf{Tensor Representation of Graphical Models}
\begin{itemize}
    \item Duality of Graphical Models and Tensor Networks:
    \cite{robeva_duality_2019}
    \item Expressivity studies \cite{glasser_expressive_2019}
\end{itemize}

\subsubsection{Infrastructure of AI}

The formalism of tensors and their network decompositions and contractions bears the potential of parallel computations exploited in the AI-dedicated soft and hardware.
\begin{itemize}
    \item Hardware: TPUs beyond GPUs
    \item Software: Tensors as basic data structure in TensorFlow, pyTorch etc., storing neural activations and model weights.
\end{itemize}



\subsection{Structure of the work}

The chapters are structured into three parts, and two focuses, as sketched by:
\newcommand{\horDistChapter}{6}
\newcommand{\verDistChapter}{2.25}
\newcommand{\parBlockDistance}{0.25}
\newcommand{\drawchapter}[4]{
    \coordinate (logRepStart) at (#1, #2);
    \draw (logRepStart) rectangle ($(logRepStart) + (blockDiagonal)$);
    \node[anchor=center] (text) at ($(logRepStart) + (toTop) +0.5*(blockDiagonal)$) {\small #3};
    \node[anchor=center] (text) at ($(logRepStart) + (toBottom) +0.5*(blockDiagonal)$) {\small #4};
}

\begin{tikzpicture}[scale=1]
    \coordinate (blockDiagonal) at (5,1.75);
    \coordinate (toTop) at (0,0.35);
    \coordinate (toBottom) at (0,-0.35);

    % Representation
    \node[anchor=center] (text) at (-0.5, 2*\verDistChapter+0.35) {Focus~I: Representation};
    \draw[dashed]  (-0.25-1*\horDistChapter, 2*\verDistChapter+1) -- (1*\horDistChapter-0.75, 2*\verDistChapter+1) -- (1*\horDistChapter-0.75, -4*\verDistChapter-1)
    -- (-0.25-1*\horDistChapter, -4*\verDistChapter-1) -- (-0.25-1*\horDistChapter, 2*\verDistChapter+1);

    % Reasoning
    \node[anchor=center] (text) at (-0.5+1.5*\horDistChapter, 2*\verDistChapter+0.35) {Focus~II: Reasoning};
    \draw[dashed]  (-0.25+1*\horDistChapter, 2*\verDistChapter+1) -- (2*\horDistChapter-0.75, 2*\verDistChapter+1) -- (2*\horDistChapter-0.75, -4*\verDistChapter-1)
    -- (-0.25+1*\horDistChapter, -4*\verDistChapter-1) -- (-0.25+1*\horDistChapter, 2*\verDistChapter+1);

    % Part I
    \node[anchor=center] (text) at (-0.5-0.5*\horDistChapter, 1*\verDistChapter-0.25) {\parref{par:one}: Factored Systems};
    \draw (-0.5-1*\horDistChapter,-0.25) -- (2*\horDistChapter-0.5, -0.25) -- (2*\horDistChapter-0.5, 2*\verDistChapter-0.25) --
    (-0.5-1*\horDistChapter, 2*\verDistChapter-0.25) -- (-0.5-1*\horDistChapter,-0.25);
    \foreach \x/\y/\key/\name in {
        0/1/cha:probRepresentation/Probability Representation,
        0/0/cha:logicalRepresentation/Logical Representation,
        1/1/cha:probReasoning/Probabilistic Reasoning,
        1/0/cha:logicalReasoning/Logical Reasoning
    } {
        \drawchapter{\x * \horDistChapter}{\y *\verDistChapter}{\charef{\key}}{\name}
    }

    % Part II
    \node[anchor=center] (text) at (-0.5-0.5*\horDistChapter, -1.5*\verDistChapter-0.5) {\parref{par:two}: Neuro-Symbolic AI};
    \draw (-0.5-1*\horDistChapter,-0.5) -- (2*\horDistChapter-0.5, -0.5) -- (2*\horDistChapter-0.5, -2*\verDistChapter-0.5) --
    (-0.5-1*\horDistChapter, -2*\verDistChapter-0.5) -- (-0.5-1*\horDistChapter,-0.5);
    \foreach \x/\y/\key/\name in {
        -1/-1/cha:formulaSelection/Formula Selecting Networks,
        0/-1/cha:networkRepresentation/Logic Network Representation,
        0/-2/cha:folModels/Statistical Models of FOL,
        1/-1/cha:networkReasoning/Logic Network Reasoning,
        1/-2/cha:mlnConcentration/Probabilistic Success Guarantees
    } {
        \drawchapter{\x * \horDistChapter}{\y *\verDistChapter - \parBlockDistance}{\charef{\key}}{\name}
    }

    % Part III
    \node[anchor=center] (text) at (-0.5-0.5*\horDistChapter, -3.5*\verDistChapter-0.5) {\parref{par:three}: Contraction Calculus};
    \draw (-0.5 - 1*\horDistChapter, -2*\verDistChapter-0.75) -- (2*\horDistChapter-0.5, -2*\verDistChapter-0.75) -- (2*\horDistChapter-0.5, -4*\verDistChapter-0.75) --
    (-0.5-1*\horDistChapter, -4*\verDistChapter-0.75) -- (-0.5-1*\horDistChapter,-2*\verDistChapter-0.75);
    \foreach \x/\y/\key/\name in {
        -1/-3/cha:coordinateCalculus/Coordinate Calculus,
        0/-3/cha:basisCalculus/Basis Calculus,
        0/-4/cha:sparseCalculus/Sparse Calculus,
        1/-3/cha:approximation/Approximation,
        1/-4/cha:messagePassing/Message Passing
    } {
        \drawchapter{\x * \horDistChapter}{\y *\verDistChapter - 2 * \parBlockDistance}{\charef{\key}}{\name}
    }

\end{tikzpicture}

\textbf{\parref{par:one}: \partonetext} \\
\ \\
The probabilistic and logical approaches towards artificial intelligence are reviewed in the tensor network formalism. \\

Tensors appear naturally in
\begin{itemize}
    \item Logics: Boolean tensors indicating models (propositional case) and interpretation tensors in first order logics
    \item Probability theory: Truth tables, which are tensors of probabilities for joint distibutions of categorical variables.
\end{itemize}

% Classical usage of tensor network decompositions
Tensor network decompositions as representation schemes appear in
\begin{itemize}
    \item Logics: Conjunctions of formulas are Hadamard products of the tensor representation of formulas (Coordinate Calculus/ Effective Calculus)
    \item Probability theory: Graphical models are tensor networks of the factors. Further sparsity schemes apply, when placing restrictions on the structure of each factor.
    \item Data bases: Relations encoded by lists as storage of nonvanishing coordinates of a relation encoding
\end{itemize}

% Classical usage of tensor network contractions
Tensor network contractions as reasoning schemes appear in
\begin{itemize}
    \item Logics: Model counts, used for satisfiablility decisions and entailment
    \item Probability theory: Marginal probability distributions, extended to conditional probability distributions through normations
\end{itemize}

\ \\
\textbf{\parref{par:two}: \parttwotext} \\
\ \\
Motivated by the classical approaches we apply the tensor network formalism towards learning and infering neuro-symbolic models. \\

\textbf{Neurosymbolic AI}
\begin{itemize}
    \item Required for more advanced AI \cite{hochreiter_toward_2022}
    \item Add the paradigm of neural computing to logical reasoning
    \item Potential benefits from Statistical Relational AI \cite{marra_statistical_2024}
    \item Tensor based approaches \cite{cohen_tensorlog_2020}
    \item \cite{badreddine_logic_2022} representation of logic using tensor networks and automated differentiation to optimize.
\end{itemize}

%\subsubsection{Neuro-Symbolic AI}

\textbf{Tensor Approaches to Neuro-Symbolic AI}
\begin{itemize}
    \item TensorLog \cite{cohen_tensorlog_2020}
    \item \cite{badreddine_logic_2022} representation of logic using tensor networks and automated differentiation to optimize.
\end{itemize}

%% Decomposition of Neural Networks
In Deep Neural Networks, functions between the input layer and the output layer are decomposed into neurons.
Typical neurons are linear transforms with an activation function.

%% Sparsity by fixed architecture
Sparsity means restriction to functions, which are decomposable into a small number of neurons.
Approximations of generic functions (see the universal approximation theorems) would require large amounts of neurons. % CITE!
When restricting to functions based on a fixed architecture, we restrict to a certain set of functions called the inductive bias of the architecture.

\ \\
\textbf{\parref{par:three}: \partthreetext}\\
\ \\
The applied schemes of calculus using tensor network contractions are investigated in more detail.
\ \\
% Representation
\textbf{\focusonespec}\\
\\\
Here we motivate and investigate the efficient representation of tensors based on tensor network decompositions. \\
\ \\
% Reasoning
\textbf{\focustwospec}\\
\ \\
We develop schemes to efficiently perform inductive and deductive reasoning based on information stored in decomposed tensor.


%\subsubsection{\focusonespec}
%
%\subsubsection{\focustwospec}
%
%\subsubsection{\parref{par:one}: \partonetext}
%
%\subsubsection{\parref{par:two}: \parttwotext}
%
%\subsubsection{\parref{par:three}: \partthreetext}


% Tensor-Network Based Reasoning
\part{Decomposition and Inference of Factored Representations}

The computational automation of reasoning is rooted both in the probabilistic and the logical reasoning tradition.
Both draw on the same ontological commitment that systems have a factored structure, that is their states are described by assignments to a set of variables.
Based on this commitment both approaches bear a natural tensor representation of their states and a formalism of the respective reasoning algorithms based on multilinear methods.
%We discuss them in this part separated from each other, and unify them in the next part by Markov Logic Networks.

\section{Notation}\label{cha:TensorNetworks}

We here provide the fundamental definitions of tensors, which are essentiell for the content in Part~I and Part~II.
In Part~III we will further investigate the properties of tensors focusing on their contractions.

\subsection{Categorical Variables and Representations}

We will in this work investigate systems, which are described by a set of properties, each called a categorical variable. 
This is called an ontological commitment, since it defines what properties a system has.

\begin{definition}
	An atomic representation of a system is described by a categorical variables $\catvariable$ taking values $\catindex$ in a finite set 
		\[  [\catdim]\coloneqq \{0,\ldots, \catdim-1\} \]
	of cardinality $\catdim$.
\end{definition}

% Notation: Large and small literals
We notate variables by large literals and indices by small literals.


%The state of an atomic system is determined by the value of the associated categorical variable.

\begin{definition}
	A factored representation of a system is a set of categorical variables $\catvariableof{\atomenumerator}$, where $\atomenumeratorin$, taking values in $[\catdimof{\atomenumerator}]$.
\end{definition}

\subsection{Tensors}

\red{
We define Tensors here in an non-canonical way based on categorical variables assigned to its axis.
This allows us to define contractions without further specification of axes, based on comparisons of shared categorical variables.}

%% Index Set Abbreviation
In the following we will denote index sets as
	\[ [\catdimof{\atomenumerator}] \coloneqq \{0,\ldots,\catdimof{\atomenumerator-1} \} \, . \]

Tensors are multiway arrays and a generalization of vectors and matrices to higher orders.
We will first define Tensors by their coordinates.

\begin{definition}[Tensor]\label{def:tensor}
	Let there be numbers $\catdimof{\atomenumerator}\in\mathbb{N}$ for $\atomenumeratorin$ and categorical variables $\catvariableof{\atomenumerator}$ taking their values in $[\catdimof{\atomenumerator}]$.
	We call maps
	\begin{align*}
		\hypercoreat{\catvariables} : \bigtimes_{\atomenumeratorin} [\catdimof{\atomenumerator}] \rightarrow \rr
	\end{align*}
	tensor of order $\atomorder$ and leg dimensions $\catdimof{0},\ldots,\catdimof{\atomorder-1}$.
	Evaluations of these maps at indices $\catindices$ are denoted by
	\begin{align*}
		\hypercoreat{\indexedcatvariables} = \hypercoreat{\catvariables}(\catindices) \, .
	\end{align*}	
%	with coordinates denoted by $\hypercore_{\catindices}$ is called a tensor of order $\atomorder$ and legs with the dimensions $\catdimof{0},\ldots,\catdimof{\atomorder-1}$.
	Tensors $\hypercoreat{\catvariables}$ are elements of the space
	\begin{align*}
		\bigotimes_{\atomenumeratorin} \rr^{\catdimof{\atomenumerator}} \,  
	\end{align*}
	which is, with the operations of coordinatewise summation and scalar multiplication, a linear space called a tensor space.
\end{definition} 

We abbreviate lists $\catvariables$ of categorical variables by $\shortcatvariables$, that is denote $\hypercoreat{\catvariables}$ by $\hypercoreat{\shortcatvariables}$.
Occasionally, when the categorical variables of a tensor are clear from the context, we will omit the notation of the variables. %further abbreviate $\hypercoreat{\catvariables}$ by $\hypercore$.


\begin{example}[Trivial Tensor]\label{exa:trivialTensor}
	The trivial tensor is defined as the map 
		\[ \onesat{\shortcatvariables} : \facstates \rightarrow \{1\} \subset \rr \]
	with all coordinates being $1$, that is for all $\catindices\in\facstates$
		\[ \onesat{\indexedcatvariables} = 1 \, . \]
\end{example}


\subsection{One-hot encodings}

We are now ready to provide the link between tensors and states of systems with factored representations.
To this end, we define the one-hot encoding of a state, which is a bijection between the states and the basis elements of a tensor space.

\begin{definition}[One-hot representations of Atomic Systems]
	Given an atomic system described by the categorical variable $\catvariable$, we define for each $\catindex\in[\catdim]$ the basis vector $\onehotmapofat{\catindex}{\catvariable}$ by
	\begin{align}
		\onehotmapofat{\catindex}{\catvariable=\tilde{\catindex}} = \begin{cases}
			1 & \text{if} \quad \catindex=\tilde{\catindex} \\
			0 & \text{else} \, .
		\end{cases} 
	\end{align}
	The one-hot encoding of states $\catindex\in[\catdim]$ of the atomic system described by the categorical variable $\catvariable$ is the map
		\[ \onehotmap: [\catdim] \rightarrow \rr^\catdim \]
	which maps $\catindex \in [\catdim]$ to the basis vectors $\onehotmapofat{\catindex}{\catvariable}$.
\end{definition}

% Coordinatewise representation
The basis vectors $\onehotmapofat{\catindex}{\catvariable}$ are tensors of order $1$ and leg dimension $\catdim$ of the structure
\begin{align}
	\onehotmapofat{\catindex}{\catvariable} = \begin{bmatrix}
	0 & \cdots & 0 & 1 &  0 & \cdots & 0
	\end{bmatrix} \, ,
\end{align}
where the $1$ is at the $\catindex$th coordinate of the vector.

% Atomic -> Factored system
We have so far described one-hot representations of the states of a single categorical variable, which would suffice to encode the state of an atomic system.
In a factored system on the other side, we are dealing with multiple categorical variables.

\begin{definition}[One-hot representations of Factored Systems]
	Let there be a factored system defined by a tuple $(\catvariables)$ of variables taking values in $\facstates$.
	The one-hot encoding of its states is the tensor product of the one-hot encoding to each categorical variables, that is the map
		\[ \onehotmap : \facstates \rightarrow  \facspace \]
	defined by mapping $\catindices$ to
		\[ % \onehotmap(\catindices) =
		 \onehotmapofat{\catindices}{\catvariables}
		=: \bigotimes_{\atomenumeratorin} \onehotmapofat{\catindexof{\atomenumerator}}{\catvariableof{\atomenumerator}} \, . \]
	We will call one-hot representations \emph{tensor representations} and depict them as
	\begin{center}
		\input{PartI/tikz_pics/overview/one_hot_tensorproduct.tex}
	\end{center}
\end{definition}


\begin{remark}[Flattening of Tensors]
	The use the tensor product to represent states of factored systems can be motivated by the reduction to atomic systems by enumeration of the states.
	We have this property reflected in the state encoding of factored systems, since the tensor space $\bigotimes_{\atomenumeratorin}\rr^{\catdimof{\atomenumerator}}$ is isomorphic to the vector spaces $\rr^{\prod_{\atomenumeratorin}\catdimof{\atomenumerator}}$.
	This operation is called flattening (or unfolding) of tensors with many axes to tensors of less axes.
\end{remark}



\subsection{Contractions}



\subsubsection{Tensor Product}



\begin{definition}[Tensor Product]\label{def:tensorProduct}
	Let there be two tensor
	\begin{align*}
		\hypercoreat{\catvariables} : \facstates \rightarrow \rr \quad \text{and} \quad  \sechypercoreat{\seccatvariables} : \secfacstates \rightarrow \rr \, 
	\end{align*}
	with different categorical variables assigned to its axes.
	Then there tensor product is the map
	\begin{align*}
		\left(\hypercore \otimes \sechypercore\right)\left[\catvariables,\seccatvariables \right] :  \left(\facstates\right) \times \left(\secfacstates\right) \rightarrow \rr
	\end{align*}
	defined for $\catindices\in\facstates$ and $\seccatindices\in\secfacstates$ as
	\begin{align*}
		\left(\hypercore \otimes \sechypercore\right)&\left[\indexedcatvariables,\indexedseccatvariables \right] \\
		&:=  \hypercoreat{\indexedcatvariables}\cdot \sechypercoreat{\indexedseccatvariables} \, .
	\end{align*}
\end{definition}


\subsection{Illustrations by Hypergraphs}

\begin{definition}\label{def:hypergraphs}
	A hypergraph is a pair $\graph=(\nodes,\edges)$ of a set of nodes $\nodes$ and a set of edges $\edges$, where each hyperedge $\edge\in\edges$ is a subset of the nodes $\nodes$.
	A directed hypergraph is a pair $\graph=(\nodes,\edges)$, such that each hyperedge $\edge\in\edges$ is the tuple of two disjoint sets $\incomingnodes,\outgoingnodes\subset\nodes$, that is
		\[ \edge = (\incomingnodes,\outgoingnodes)  \, . \]
\end{definition}


%% Diagrammatic representation
We will use a diagrammatic illustration of tensors, where tensors are represented by block nodes and each axis as a leg decorated with a categorical variable $\catvariableof{\atomenumerator}$ representing the choice of an element in the set $[\catdimof{\atomenumerator}]$ as shown in Figure~\ref{fig:tensors}b).

%% Hyperedge view
Another useful depiction is by hypergraphs, which are generalizations of graphs allowing that edges being arbitrary nonempty subsets of the nodes, whereas canonical graphs assume pairs.
Along this depiction (see Figure~\ref{fig:tensors}a) each axis of the tensor is represented by a node representing the variable $\catvariableof{\atomenumerator}$ and the tensor $\hypercore$ is associated with the hyperedge $\edge$ connecting all variables.

The two representations are dual to each other in the sense that the nodes of the one are the hyperedges of the other graph.

\begin{figure}[h!]
	\begin{center}
		\input{PartI/tikz_pics/tensor_networks/hypercore.tex}
	\end{center}
	\caption{Depiction of Tensors 
	a) Decoration of hyperedges connecting the variables $\catvariableof{\atomenumerator}$ to each axis $\atomenumeratorin$.
	b) Blockwise notation with axis denoted by open legs represented by the variables $\catvariableof{\atomenumerator}$.
	}\label{fig:tensors}
\end{figure}


% Diagramatic representation of vectors
To depict vector calculus and its generalizations, we will apply the graphical notation (mainly version b) introduced in Chapter~\ref{cha:TensorNetworks}. 
Along this line, we represent vectors and their generalization to tensors by blocks with legs representing its indices.
The basis vectors being one-hot encodings of states are in this scheme represented by
	\begin{center}
		\input{PartI/tikz_pics/overview/one_hot_atomic.tex}
	\end{center}
where $\tilde{\catindex}$ is an indexed represented by an open leg. 
Assigning $\catindex$ to this index will retrieve the $\catindex$th coordinate (with value $1$), whereas all other assignments will retrieve the coordinate values $0$. 

Drawing on the interpretation of tensors by hyeredges we can continue with the definition of tensor networks.

\begin{definition}\label{def:tensorNetwork}
	Let $\graph=(\nodes,\edges)$ be a hypergraph with nodes decorated by categorical variables $\catvariableof{\node}$ with dimensions
		\[ \catdimof{\node} \in \nn \]	
	and hyperedges $\edge\in\edges$ decorated by core tensors
		\[ \hypercoreofat{\edge}{\catvariableof{\edge}} \in \bigotimes_{\node\in\edge}\rr^{\catdimof{\node}} \, , \]
	where we denote by $\catvariableof{\edge}$ the set of categorical variables $\catvariableof{\node}$ with $\node\in\edge$.
	Then we call the set 
		\[ \tnetofat{\graph}{\catvariableof{\nodes}} = \{\hypercoreofat{\edge}{\catvariableof{\edge}}  \, : \, \edge\in\edges\} \]
	the Tensor Network of the decorated hypergraph $\graph$.
\end{definition}


\begin{figure}
	\begin{center}
		\input{PartI/tikz_pics/tensor_networks/network.tex}
	\end{center}
	\caption{
	Example of a tensor network.
	a) Hypergraph with edges $\edge_0=\{\catvariableof{0},\catvariableof{1},\catvariableof{2}\}$, $\edge_1=\{\catvariableof{1},\catvariableof{2}\}$ and $\edge_2=\{\catvariableof{2},\catvariableof{3}\}$ decorated by tensor cores.
	b) Dual tensor network, depicting a contraction with leaving all variables open.
	}\label{fig:network}
\end{figure}

%%
Diagrammatic notation: Best to do version a) as used in the definition, highlighting that tensors have shared categorical variables with fixed dimensions.


\subsection{Generic Contractions}


Contractions of Tensor Networks $\extnet$ are operations to retrieve single tensors by summing products of tensors in a network over common indices.
We will define contractions formally by specifying just the indices not to be summed over.


%need specification of Tensor Network at hand and edges to be left over (as in numpy.einsum)
%	\[ \contractionof{\extnet}{[\randomxof{\atomenumerator_1},...\randomxof{\atomenumerator_d}]} \, . \]
When some of the variables are not appearing as leg variables, we define the contraction as being a tensor product with the trivial tensor $\ones$ carrying the legs of the missing variables.


\begin{definition}\label{def:contraction}
	Let $\tnetof{\graph}$ be a tensor network on a decorated hypergraph $\graph=(\nodes,\edges)$.
	For any subset $\secnodes\subset\nodes$ we define the contraction  to be the tensor 
	\begin{align}
		\contractionof{\tnetof{\graph}}{\catvariableof{\secnodes}} \in \bigotimes_{\node\in\secnodes} \rr^{\catdimof{\node}}
	\end{align}
	defined coordinatewise by the sum	
	\begin{align}
		\contractionof{\tnetof{\graph}}{\indexedcatvariableof{\node} \, : \, \node\in\secnodes} =
%		\contractionof{\tnetof{\graph}}{\secnodes}\left(\{\catindexof{\node} \, : \, \node\in\secnodes\}\right) = 
		\sum_{\{\catindexof{\node} \in [\catdimof{\node}] \, : \, \node \in \nodes/\secnodes\}}
		\left( \prod_{\edge\in\edges}\hypercoreofat{\edge}{\indexedcatvariableof{\node} \, : \, \node\in\edge} \right) \, .
		 %\left(\{\catindexof{\node} : \node\in\edge\}\right) \, . 
	\end{align}
\end{definition}

\begin{figure}
	\begin{center}
		\input{PartI/tikz_pics/tensor_networks/contraction.tex}
	\end{center}
	\caption{
		Example of a tensor network contraction of all but the variables $\catvariableof{1},\catvariableof{3}$.
		Contraction of variables can always be depicted by closing the open legs with trivial tensors $\ones$ performing index sums.
	}\label{fig:contraction}
\end{figure}



%%
Diagrammatic notation: Best to do version b), since this is easiest to see how tensors combine to new tensors by contractions.

\begin{remark}[Alternative Notations]
	Contractions can also denoted by the Einstein summations of the indices along connected edges, understood as scalar product in each subspace.
%	This notation is used in $\mathrm{numpy.einsum}$.
	This is as in Definition~\ref{def:contraction}, just omitting the sums.
	We found it useful in this work to do the diagrammatic representation instead, since it offers a better possibility to depict hierarchical arrangements of shared variables.
\end{remark}


% Example
\begin{example}
	Tensor products are examples of contractions, where the hypergraph consists of two disjoint edges, which union is left open.
\end{example}


%% Diagrammatic representation of Matrix Vector
\begin{example}
	Matrix vector contractions are special cases of Tensor Contractions.
	We use the visualization as in Figure~\ref{fig:tensors}b where vectors are blocks with single categorical variables and matrices with two indices and matrix vector products by
	\begin{center}
		\begin{tikzpicture}[scale=0.3,thick,xscale=-1] % , baseline = -3.5pt

\draw (-9,2)--(-7,2) node[midway,above] {\tiny $\exrandom$};
\draw (-21,1) rectangle (-9,3);
\node[anchor=center] (text) at (-15,2) {\small $\contractionof{\matrixat{\exrandom,\secexrandom},\vectorat{\secexrandom}}{\exrandom}$};

\node[anchor=center] (text) at (-5,2) {\small ${=}$};

\draw (3,2)--(5,2) node[midway,above] {\tiny $\exrandom$};
\draw (1,1) rectangle (3,3);
\node[anchor=center] (text) at (2,2) {\small $\exmatrix$};
\draw (1,2)--(-1,2) node[midway,above] {\tiny $\secexrandom$};
\draw (-1,1) rectangle (-3,3);
\node[anchor=center] (text) at (-2,2) {\small $\exvector$};

%\node[anchor=center] (text) at (7,1) {$\cdot$};


\end{tikzpicture}
	\end{center}
	Here the index $j$ is represented by a closed edge, which means that it is eliminated by a sum.
\end{example}


%% Hadamard Product 
\begin{example}
	A node appearing in arbitrary many hyperedges denotes a Hadamard product of the axis of the respective decorating tensors.
	To give an example, let $V^k\in\rr^p$ be vectors for $\atomenumeratorin$. Their hadamard product is the vector
		\[ V^1\circ V^2 \circ \ldots \circ V^{\atomorder-1} \in \rr^p \]
	defined by
		\[ \left( V^1\circ V^2 \circ \ldots \circ V^{\atomorder-1} \right)_\catindex = \prod_{\atomenumeratorin} V^\atomenumerator_\catindex \, . \]
	In a contraction diagram the Hadamard product is depicted by 
	\begin{center}
		\begin{tikzpicture}[scale=0.3,thick] % , baseline = -3.5pt


\begin{scope}[shift={(-10,0)}]

\draw (-5,1) rectangle (7,3);
\node[anchor=center] (text) at (1,2) {\small $\contractionof{\{\vectorofat{\catenumerator}{\catvariable} \, : \, \catenumeratorin\}}{\catvariable}$};
\draw (1,-1)--(1,1) node[midway,right] {\tiny $\catleg$};

\node[anchor=center] (text) at (9,2) {${=}$};

\end{scope}



\draw (1,1) rectangle (3,3);
\node[anchor=center] (text) at (2,2) {\small $\vectorof{0}$};
\draw (2,-1)--(2,1) node[midway,right] {\tiny $\catvariable$};


\begin{scope}[shift={(5,0)}]

\draw (1,1) rectangle (3,3);
\node[anchor=center] (text) at (2,2) {\small $\vectorof{1}$};
\draw (2,-1)--(2,1) node[midway,right] {\tiny $\catvariable$};

\end{scope}

\node[anchor=center] (text) at (11.5,2) {\small $\cdots$};


\begin{scope}[shift={(15,0)}]

\draw (0.75,1) rectangle (3.25,3);
\node[anchor=center] (text) at (2,2) {\small $\vectorof{\atomorder\shortminus1}$};
\draw (2,-1)--(2,1) node[midway,right] {\tiny $\catvariable$};

\end{scope}


\draw[fill] (9.125,-4.5) circle (0.25cm);

\draw (9.125,-4.5) to[bend right=-20] (2,-1); 
\draw (9.125,-4.5) to[bend right=-20] (7,-1); 
\draw (9.125,-4.5) to[bend right=20] (17,-1); 

\draw (9.125,-4.5) -- (9.125,-6.5) node[midway,right] {\tiny $\catvariable$};; 

\end{tikzpicture}
	\end{center}
\end{example}



\subsubsection{Decompositions}

\begin{definition}\label{def:tnDecomposition}
	Let $\extensor$ be a tensor in $\extensorspace$.
	A Tensor Network Decomposition of a tensor $\hypercoreat{\catvariablesinset{\nodes}}$ is a Tensor Network $\tnetof{\graph}$ to a hypergraph containing  such that
		\[ \hypercoreat{\catvariablesinset{\nodes}}= \contractionof{\tnetof{\graph}}{\catvariablesinset{\nodes}} \, . \]
\end{definition}



\subsection{Properties of Tensors}

%% Binary
We will often encounter situations, where the coordinates of tensors are binary valued, i.e. the tensor is a map to the set $[2]$.

\begin{definition}\label{def:binaryTensor} % CALL BOOLEAN INSTEAD?
	We call a tensor binary, when $\imageof{T}\subset[2]$, i.e. all coordinates are either $0$ or $1$.
\end{definition}

%% Directionality
Directionality represents constraints on the structure of tensors:
Summing over outgoing trivializes the tensor.

\begin{definition}\label{def:directedTensor}
	A Tensor 
		\[ \hypercoreat{\catvariablesinset{\nodes}} \in \bigotimes_{\nodein}\rr^{\catdimof{\node}} \]
	is said to be directed with incoming variables $\innodes$ and outgoing variables $\outnodes$, where $\nodes=\innodes\dot{\cup}\outnodes$, when
		\[ \contractionof{\{\hypercore\}}{\catvariablesinset{\outnodes}} =  \onesat{\catvariablesinset{\innodes}} \]
	where $\onesat{\catvariablesinset{\innodes}}$ denoted the trivial tensor in  $\bigotimes_{\node\in\innodes}\rr^{\catdimof{\node}}$ which coordinates are all $1$.
\end{definition}

While by default all legs are outgoing, we can change the direction by normation.

\begin{definition}\label{def:normation}
	A tensor $\hypercoreat{\catvariablesinset{\nodes}}$ is said to be normable on $\innodes\subset\nodes$, if for any $\atomlegindexof{\innodes}\in\bigtimes_{\node\in\innodes}[\catdimof{\node}]$ we have
		\[ \contractionof{\{\hypercore\} \cup\{\onehotmapof{\atomlegindexof{\innodes}}\}}{\varnothing} > 0 \, . \]
	The normation of a on $\innodes\subset\nodes$ normable tensor is the tensor
	\begin{align*}
		\normationofwrt{\{\hypercore\}}{\outnodes}{\innodes} = 
		\sum_{\atomlegindexof{\innodes}\in\bigtimes_{\node\in\innodes}[\catdimof{\node}]} 
		\onehotmapof{\atomlegindexof{\innodes}} \otimes \frac{
		\contractionof{\{\hypercore\}\cup\{\onehotmapof{\atomlegindexof{\innodes}}\}}{\outnodes}
		}{
		\contractionof{\{\hypercore\}\cup\{\onehotmapof{\atomlegindexof{\innodes}}\}}{\varnothing}
		} 
	\end{align*}
	where $\outnodes = \nodes / \innodes$.
%	
%	A tensor network $\extnet$ on variables $\nodes$ can be normed on $\secnodes$, if the coordinates of no slice with respect to $\secnodes$ sum to $0$.
%	Then we define the normed tensor
%		\[ \normationofwrt{\extnet}{\outnodes}{\innodes} 
%		\in \left( \bigotimes_{\node\in\innodes} \rr^{\catdimof{\node}} \right) \otimes \left( \bigotimes_{\node\in\outnodes} \rr^{\catdimof{\node}} \right) \]
%	by
%	 \begin{align*}
%	 	\normationofwrt{\extnet}{\outnodes}{\innodes} 
%		= \sum_{\atomlegindexof{\innodes}\in\bigtimes_{\node\in\innodes}[\catdimof{\node}]} 
%		\onehotmapof{\atomlegindexof{\innodes}} \otimes \frac{
%		\contractionof{\extnet\cup\{\onehotmapof{\atomlegindexof{\innodes}}\}}{\outnodes}
%		}{
%		\contractionof{\extnet\cup\{\onehotmapof{\atomlegindexof{\innodes}}\}}{\varnothing}
%		} \, . 
%	 \end{align*}
\end{definition}

We will investigate the contractions of directed tensors in Part~III, where we show in Theorem~\ref{the:normationDirected} that normations are directed tensors.


%% Diagrammatic notation
In our graphical tensor notation, we depict directed tensors by directed hyperedges (a), which are decorated by directed tensors (b), for example:
%\red{Draw incoming and outgoing example.}
	\begin{center}
		\input{PartI/tikz_pics/tensor_networks/directed_core.tex}
	\end{center}



\subsection{Encoding schemes for functions}

Tensors are defined here as real-valued functions on the state set of a system described by categorical variables.
We provide further schemes to represent functions in order to perform sparse calculus and to handle more generic functions.


\subsubsection{Real-valued functions}


\begin{example}[Uncertainty about States]\label{exa:onehotUncertainty}
	The uncertainty about the state of a categorical variable $\catvariable$ can be expressed in vectors.
	For example let there be real numbers $\probof{\catvariable=\catindex} \in [0,1]$ for $\catindex\in[\catdim]$ with $\sum_{\catindex\in[\catdim]}\probof{\catvariable=\catindex}=1$ with the interpretation that $\probof{\catvariable=\catindex}$ is the probability of a system being in state $\catindex$. 
	We can represent this uncertain state simply by a vector 
		\[ \probof{\catvariable}\in\rr^{\catdim} \]
	defined as the sum of one-hot representations weighted by $\probof{\catvariable=\catindex}$
	\[ \sum_{\catindex\in[\catdim]} \probof{\catvariable=\catindex} \cdot \onehotmapofat{\catindex}{\catvariable} =
		\begin{bmatrix}
		\probof{\catvariable=0} & \probof{\catvariable=1} & \cdots & \probof{\catvariable=\catdim-1}
		\end{bmatrix} \, . 
	\]
\end{example}



\subsubsection{Relational encodings}

We have already observed in Example~\ref{exa:atomicFunction}, that any function of a categorical variable has a representation as a linear function acting on the one-hot encoding of the variable.
Let us now show how we can encode maps between factored systems.
The scheme is described in more generality and detail (encoding of subsets and relations) in Chapter~\ref{cha:tensorEncodings}, see Definition~\ref{def:functionRelationEncoding}.

\begin{definition}[Relation encoding of maps between Factored Systems]\label{def:functionRepresentation}
	Let $\exfunction$ be a function
		\[ \exfunction : \facstates \rightarrow  \secfacstates \]
	which maps the states of a factored system to variables $\catvariables$ to the states of another factored system with variables $\seccatvariables$.
	Then the tensor representation of $\exfunction$ is a tensor
		\[ \rencodingofat{\exformula}{\catvariables,\seccatvariables} \in   \left(\facspace\right) \otimes \left(\secfacspace\right)  \]
	defined by
		\[ \rencodingofat{\exformula}{\catvariables,\seccatvariables}= \sum_{\catindices\in\facstates}  
		  \onehotmapofat{\catindices}{\catvariables} \otimes \onehotmapofat{\exfunction(\catindices)}{\seccatvariables} \, . \]
\end{definition}

% Notation with image categorical variable
When the categorical variables of the image factored system to a map $\exfunction$ are not specified otherwise, we will denote them by $\catvariableof{\exfunction}$.


\subsubsection{Tensor-valued functions}


%% TO DETAILLED HERE -> Part III?
\begin{definition}[Selection encoding of Maps between Factored Systems]\label{def:selectionEncoding}
	Given a tensor space $\parspace$ described by categorical variables $\selvariables$ and a tensor-valued function
		\[ \exfunction : \facstates \rightarrow \parspace \]
	the selection encoding of $\exfunction$ is a tensor
		\[ \sencodingofat{\exfunction}{\shortcatvariables,\shortselvariables} \in \left(\facspace\right) \otimes \left(\parspace\right) \]
	defined by the basis decomposition
		\[ \sencodingofat{\exfunction}{\shortcatvariables,\shortselvariables} = \sum_{\catindices\in\facstates} \onehotmapofat{\catindices}{\shortcatvariables} \otimes \exfunction(\catindices)[\shortselvariables] \, .  \]
\end{definition}

%%
We call these tensor representation of maps selection encodings, since the coordinate of a function $\exfunction$ to be processed is selected by another argument to $\sencodingof{\exfunction}$.

%\begin{example}[Vector valued functions]\label{exa:atomicFunction} %% CONFUSIN, since already needs selection variables?
%	When using a one-hot representation of the state of a categorical variable, any real valued function has a representation by a real valued matrix acting on the one-hot encoding. 
%	Let there be a vector valued function
%		\[ \exformula : [\catdim] \rightarrow \rr^p \]
%	which maps $\catindex\in[\catdim]$ to the vector
%		\[ \exformula(\catindex)[\selvariable] \in \rr^p \, , \]
%	where we introduced the variable $\selvariable\in[p]$ selecting a coordinate of the image vector.
%	The 
%		\[ \exformula(\catindex)[\selvariable] = 
%		\contractionof{\{\onehotmapof{\catindex}[\catvariable] , \,\concore_{\exformula}[\catvariable,\selvariable]\}}{\selvariable}  \]
%	where $\concore_{\exformula} \in \rr^{\catdim \times p} $ is the matrix defined by the function evaluation vectors of $\exformula$ as
%		\[ \concore_{\exformula}[\catvariable,\selvariable] = \begin{bmatrix}
%			-- & \exformula(0) & -- \\
%			-- & \exformula(1) & -- \\
%			& \vdots &  \\
%			-- & \exformula(\catdim-1) & -- 
%		\end{bmatrix} \, . 
%		\]
%	This can easily be verified, since matrix multiplication with basis vectors amounts to selection of rows (when the basis vector is acting from the left) or columns (when the basis vector is acting from the right).
%	Thus, linear transforms (matrices) acting on the one-hot representation are sufficient to represent any vector valued function of the states of a categorical variable.
%\end{example} 


%% 
We will provide more detail to the tensor representation of functions in Chapter~\ref{cha:tensorEncodings}. %where we show that domain encodings coincide with selection encodings.







% Parametrization of Probability Distributions
\section{Probability Distributions}\label{cha:probDecomposition}

In this chapter we will establish relations between the formalism of tensor networks and basic concepts of probability theory.
We will first understand distributions as tensors and connect their marginalizations and conditionings to the tensor operations of contractions and normations.
Then we discuss independence assumptions as examples of contraction equations, which lead to tensor network decompositions known as graphical models.
We then treat more generic exponential families and investigate their representation as tensor networks.

\subsection{Tensor Representation of Distributions}

%% Random Variables: Introduction in Bayesian way by uncertainties
After having discussed how to represent states of factored systems by one-hot encodings, let us now take advantage of these representation by associating properties with these states.
Let there be uncertainties of the assignments $\catindexof{\atomenumerator}$ to the categorical variables $\catvariableof{\atomenumerator}$ of a factored system.
We then understand $\catvariableof{\atomenumerator}$ as random variables, which have a joint distribution defined by the uncertainties of the state assignments.
To capture these uncertainties we now make use of the one-hot representation of factored systems in Chapter~\ref{cha:factoredRepresentation}.

\begin{definition}[Probability Tensor] % From the axioms of Kolmogorov!
	Let there be a factored system defined by a categorical variable $\catvariableof{\atomenumerator}$ for each $\atomenumeratorin$ taking values in $[\catdimof{\atomenumerator}]$. 
	A probability distribution over the states of $\facsystem$ is a tensor
		\[ \probat{\catvariableof{0},\ldots,\catvariableof{\atomorder-1}} : \facstates \rightarrow [0, 1] \subset \rr \]
	such that
		\[ \sum_{\catindices\in\facstates} \probat{\indexedcatvariables} = 1 \, . \]
\end{definition}

We notice that there are two conditions for a tensor to be probability tensor.
First, the tensor needs to have non-negative coordinates and second, the coordinates need to sum to $1$.

%% One-hot Decomposition -> Contraction Equivalences
The probability tensor to the distribution is an object
		\[ \probat{\catvariables} \in \bigotimes_{\atomenumeratorin}\rr^{\catdimof{\atomenumerator}} \]
which is the sum over the one-hot encodings (see Lemma~\ref{lem:tensorBasisDecomposition})
		\[ \probat{\catvariables} = \sum_{\catindices\in\facstates} \probat{\indexedcatvariables} \cdot \onehotmapofat{\catindices}{\catvariables} \, . \]
		
%%
The normation condition of probability tensors can be expressed by the contraction equation $1= \sbcontraction{\probtensor}$ since
\begin{align*}
	1 = \sum_{\catindices}\probat{\indexedcatvariables}
	=  \sum_{\catindices}\sbcontraction{\probtensor, \onehotmapof{\catindices}}
	= \sbcontraction{\probtensor} \, . 
\end{align*}

%% NOT NEEDED
%Using the Coordinate Calculus as described in Theorem~\ref{the:coordinateCalculus} we can retrieve the coordinates of $\probtensor$ storing the probabilities of specific states by the contraction
%\begin{align*}
%	\probat{\indexedcatvariables} = \contractionof{\{\probtensor, \onehotmapof{\catindices}\}}{\varnothing} \, . 
%\end{align*}

%% Coordinates
%The probability tensor stores all probabilities on its coordinates, which are by construction
%	\[ \probtensor_{\catindices} = \probat{\catvariableof{\atomenumerator} = \catindexof{\atomenumerator} \, : \, \atomenumeratorin}  \, . \]
%We here draw on the redundancy of the one-hot encoding of each state of a factored system, which enables us to represent the properties of multiple states in single tensors (see Example~\ref{exa:onehotUncertainty}).

Probability tensors are depicted as
\begin{center}
	\begin{tikzpicture}[scale=0.35,thick] % , baseline = -3.5pt

    \node[anchor=center] (text) at (-2,0) {$a)$};

    \node [circle, draw, thick, fill=gray!50, minimum size = \nodeminsize] (P1) at (0,-3) {\tiny $\catvariableof{0}$};
    \node [circle, draw, thick, fill=gray!50, minimum size = \nodeminsize] (P2) at (3,-3) {\tiny $\catvariableof{1}$};

    \node[anchor=center] (text) at (6,-3) {$\cdots$};

    \node [circle, draw, thick, fill=gray!50, minimum size = \nodeminsize] (P3) at (9,-3) {};

    \node[anchor=center] (text) at (9,-3) {\tiny $\catvariableof{\atomorder-1}$};


    \draw[->]
    (4.5,0) to[bend right=25] (P1);
    \draw[->]
    (4.5,0) to[bend right=10] (P2);
    \draw[->]
    (4.5,0) to[bend right=-25] (P3);

    \node[anchor=center] (text) at (4.5,0.5) {$\edge$};


    \begin{scope}
        [shift={(20,0)}]

        \node[anchor=center] (text) at (-2,0) {$b)$};

        \draw (-1,-1) rectangle (5,-3);
        \node[anchor=center] (text) at (2,-2) {\small $\probtensor$};
%\draw[->] (0,-3)--(0,-5) node[midway,left] {\tiny $\catvariableof{0}$};
%\draw[->] (1.5,-3)--(1.5,-5) node[midway,left] {\tiny $\catvariableof{1}$};
        \node[anchor=center] (text) at (3,-4) {$\cdots$};
%\draw[->] (4,-3)--(4,-5) node[midway,right] {\tiny $\catvariableof{\atomorder-1}$};


        \draw[midarrow]  (0,-3) -- (0,-5) node[midway,left] {\tiny $\catvariableof{0}$};
        \draw[midarrow]
        (1.5,-3)--(1.5,-5) node[midway,left] {\tiny $\catvariableof{1}$};
        \draw[midarrow]
        (4,-3)--(4,-5) node[midway,right] {\tiny $\catvariableof{\atomorder-1}$};
    \end{scope}


\end{tikzpicture}
\end{center}


\subsection{Marginal Distribution}

Contractions of probability distributions are related to marginalizations as we introduce next.

\begin{definition}[Marginal Probability]\label{def:marginalProbability}
	Given a distribution $\probat{\exrandom,\secexrandom}$ of the categorical variables $\exrandom$ and $\secexrandom$ the marginal distribution of the categorical variable $\exrandom$ is defined for each $\exrandind$ as the tensor
	\begin{align*}
		\probat{\exrandom} : [\exranddim] \rightarrow \rr
	\end{align*}
	defined for $\exrandind\in[\exranddim]$ by
	\begin{align*}
		\probat{\indexedexrandom} 
		= \sum_{\secexrandind\in[\secexranddim]} \probat{\indexedexrandom,\indexedsecexrandom} \, .
	\end{align*}
\end{definition}

% Sets of variables
Definition~\ref{def:marginalProbability} generalizes to marginalizations of sets of variables, since we can always group a set of categorical variables and understand them as a single one.

%% Contractions
\begin{theorem}\label{the:marginalContraction}
	%Given a Tensor Network (see Definition~\ref{def:tensorNetwork}) $\{\probtensor\}$ consistent of the variables $\exrandom,\secexrandom$ and hyperedge $\{\exrandom,\secexrandom\}$ decorated with the tensor $\probtensor$.
	For any distribution $\probat{\exrandom,\secexrandom}$ the marginal distribution of the variable $\catvariable$ is the contraction
	\begin{align*}
		\probat{\exrandom} = \sbcontractionof{\probtensor}{\exrandom} \, .
	\end{align*}
	Further, any marginal distribution is a probability distribution.
\end{theorem}
\begin{proof}
	We have $\probat{\exrandom} = \contractionof{\{\probtensor\}}{\exrandom}$ by definition.
	To show that $\probat{\exrandom}$ is a probability distribution, we need to show that $\sbcontraction{\probat{\exrandom}}=1$.
	But this follows from the normation of $\probtensor$ and the commutativity of contractions (see Theorem~\ref{the:splittingContractions} in Chapter~\ref{cha:localContractions}) as
		\[ \sbcontraction{\probat{\exrandom}} = 
		\sbcontraction{
			\sbcontractionof{\probtensor}{\exrandom}
		} =
		 \sbcontraction{\probtensor}
		= 1 \, . 
		\]
\end{proof}

%% Tensor Representation
We depict the sum over the possible values of $\secexrandom$ by contraction of the probability tensor with the trivial tensors $\ones$ as 
\begin{center}
	\input{PartI/tikz_pics/probability_decomposition/marginalized_probability.tex}
\end{center}
Let us notice, that marginal distributions are probability tensors for themself, which we again denote by a directed leg.
%We here omit the denotation of the nodes in the hypergraph of a Tensor Network and represent a Tensor Network just by the appearing Tensor Cores on the hyperedge.


\subsection{Conditional Probabilities}

Normations of probability distributions result in conditional distributions as we define next.

\begin{definition}[Conditional Probability]\label{def:conditionalProbability}
	Let $\probat{\exrandom,\secexrandom}$ be a distribution of the categorical variables $\exrandom$ and $\secexrandom$, such that $\probtensor$ is normable on $\{\secexrandom\}$.
	Then the distribution of $\exrandom$ conditioned on $\secexrandom$ is defined by
		\[ \condprobof{\indexedexrandom}{\indexedsecexrandom}  
		= \frac{\probat{\indexedexrandom,\indexedsecexrandom}}{\probat{\indexedsecexrandom}} \, . \]
\end{definition}

%The conditional probability
%	\[ \condprobof{\exrandom}{\indexedsecexrandom}  
%	= \frac{\probat{\exrandom,\indexedsecexrandom}}{\probat{\indexedsecexrandom}} \]
%is also a tensor with legs to $\exrandom$ and $\secexrandom$.
%For each one-hot encoding $\onehotmapof{\secexrandind}$ of the assignment $\secexrandind$ to the variable $\secexrandom$ we represent the conditional probability by the diagrams
%\begin{center}
%	\begin{tikzpicture}[scale=0.3, thick] % , baseline = -3.5pt

\draw (-21,-1) rectangle (-15,-3);
\node[anchor=center] (text) at (-18,-2) {\small $\condprobof{\exrandom}{\secexrandom}$};
\draw[midarrow]  (-20,-3)--(-20,-5) node[midway,left] {\tiny $\exrandom$}; 

\draw[midarrow]  (-16,-5)--(-16,-3) node[midway,left] {\tiny $\secexrandom$}; 
\draw[dashed] (-15,-5) rectangle (-17,-7); 
\node[anchor=center] (text) at (-16,-6) {\small $\onehotmapof{\secexrandind}$};

\node[anchor=center] (text) at (-13,-2) {${=}$};


\begin{scope}[shift={(0,6)}]

\draw (-11,-1) rectangle (-5,-3);
\node[anchor=center] (text) at (-8,-2) {\small $\probof{\exrandom,\secexrandom}$};
\draw[midarrow]  (-10,-3)--(-10,-5) node[midway,left] {\tiny $\exrandom$}; 
\draw[midarrow]  (-6,-3)--(-6,-5) node[midway,left] {\tiny $\secexrandom$};
\draw[dashed] (-7,-5) rectangle (-5,-7); 
\node[anchor=center] (text) at (-6,-6) {\small $\onehotmapof{\secexrandind}$};

\end{scope}

\draw (-12,-2) -- (-4,-2);

\begin{scope}[shift={(0,-2)}]

\draw (-11,-1) rectangle (-5,-3);
\node[anchor=center] (text) at (-8,-2) {\small $\probof{\exrandom,\secexrandom}$};
\draw[midarrow]  (-10,-3)--(-10,-5) node[midway,left] {\tiny $\exrandom$}; 
\draw (-11,-5) rectangle (-9,-7); 
\node[anchor=center] (text) at (-10,-6) {$\ones$};
\draw[midarrow]  (-6,-3)--(-6,-5) node[midway,left] {\tiny $\secexrandom$};
\draw[dashed] (-7,-5) rectangle (-5,-7); 
\node[anchor=center] (text) at (-6,-6) {\small $\onehotmapof{\secexrandind}$};

\end{scope}

\end{tikzpicture}
%\end{center}
%Here we denote by the quotient a coordinatewise normation, as sketched by the dashed unit vector. % is contracted before each normation, but we will omit it in future diagrams.
%We depict conditional variables by directed edges, where legs to conditions are incoming while the others outgoing.

%% Normation and Directed Notation
We show in the next theorem, that conditional distributions are calculated by normations.
%We will discuss operations on tensors like conditioning more detail in Chapter~\ref{cha:directedTC} as normation operation of Definition~\ref{def:normation}.
%In Theorem~\ref{the:conditionalContraction} we will show that the resulting tensor is directed with incoming variables by the conditions.

\begin{theorem}\label{the:conditionalContraction}
	The tensor $\condprobof{\exrandom}{\secexrandom}$ is the normation of $\probat{\exrandom,\secexrandom}$ on $\secexrandom$  (see Definition~\ref{def:normation}), that is
	\begin{align*}
		\condprobof{\exrandom}{\secexrandom}   
		= \sbnormationofwrt{\probtensor}{\exrandom}{\secexrandom} \, . 
	\end{align*}
	Further, for any $\secexrandind\in[\secexranddim]$ the tensor $\condprobof{\exrandom}{\indexedsecexrandom}$ is a probability tensor.
\end{theorem}
\begin{proof}
	The first claim follows from a comparison of Definition~\ref{def:conditionalProbability} and \ref{def:normation}.
	The second claim follows from the first and Theorem~\ref{the:normationDirected}.
	Alternatively, the second claim can be showed using the diagrammatic notation as
	\begin{center}
		\begin{tikzpicture}[scale=0.3,thick] % , baseline = -3.5pt

\node[anchor=center] (text) at (-30,-2) {\small $\sum_{\atomlegindexof{\exrandom}} \, \condprobof{X=\atomlegindexof{\exrandom}}{Y=\atomlegindexof{\secexrandom}} \quad {=}$};

\draw (-21,-1) rectangle (-15,-3);
\node[anchor=center] (text) at (-18,-2) {\small $\condprobof{X}{Y}$};
\draw[->]  (-20,-3)--(-20,-5) node[midway,left] {\tiny $X$}; 

\draw[<-]  (-16,-3)--(-16,-5) node[midway,left] {\tiny $Y$}; 
\draw[] (-15,-5) rectangle (-17,-7); 
\node[anchor=center] (text) at (-16,-6) {\small $\onehotmapof{\catindexof{Y}}$};

\draw (-21,-5) rectangle (-19,-7); 
\node[anchor=center] (text) at (-20,-6) {$\ones$};

\node[anchor=center] (text) at (-13,-2) {${=}$};


\begin{scope}[shift={(0,6)}]

\draw (-11,-1) rectangle (-5,-3);
\node[anchor=center] (text) at (-8,-2) {\small $\probof{X,Y}$};
\draw[->]  (-10,-3)--(-10,-5) node[midway,left] {\tiny $X$}; 
\draw (-11,-5) rectangle (-9,-7); 
\node[anchor=center] (text) at (-10,-6) {$\ones$};
\draw[->]  (-6,-3)--(-6,-5) node[midway,left] {\tiny $Y$};
\draw[] (-7,-5) rectangle (-5,-7); 
\node[anchor=center] (text) at (-6,-6) {\small $\onehotmapof{\catindexof{Y}}$};

\end{scope}

\draw (-12,-2) -- (-4,-2);

\begin{scope}[shift={(0,-2)}]

\draw (-11,-1) rectangle (-5,-3);
\node[anchor=center] (text) at (-8,-2) {\small $\probof{X,Y}$};
\draw[->]  (-10,-3)--(-10,-5) node[midway,left] {\tiny $X$}; 
\draw (-11,-5) rectangle (-9,-7); 
\node[anchor=center] (text) at (-10,-6) {$\ones$};
\draw[->]  (-6,-3)--(-6,-5) node[midway,left] {\tiny $Y$};
\draw[] (-7,-5) rectangle (-5,-7); 
\node[anchor=center] (text) at (-6,-6) {\small $\onehotmapof{\catindexof{Y}}$};

\end{scope}

%\node[anchor=center] (text) at (-3,-2) {${=}$};
%
%\draw (-1,-3) rectangle (1,-1); 
%\node[anchor=center] (text) at (0,-2) {$\ones$};
%\draw[<-]  (0,-3)--(0,-5) node[midway,left] {\tiny $Y$};
%\draw[] (-1,-5) rectangle (1,-7); 
%\node[anchor=center] (text) at (0,-6) {\small $\onehotmapof{\catindexof{Y}}$};

\node[anchor=center] (text) at (-1,-2) {${=}\quad 1 \, .$};

%\node[anchor=center] (text) at (9,-7) {${.}$};

\end{tikzpicture}
	\end{center}
\end{proof}



% Contraction Formalism
Theorem~\ref{the:marginalContraction} and \ref{the:conditionalContraction} show that the formalism of contractions and normations is applied in basic operations of probabilistic reasoning.

We can further show, that exactly the directed tensors with non-negative coordinates are conditional probability tensors.

\begin{theorem}\label{the:conditionalDirected}
	Any tensor with non-negative coordinates is a conditional distribution tensor, if and only if it is directed with the condition variables ingoing and the other outgoing.
\end{theorem}
\begin{proof}
	\proofrightsymbol:
	By Theorem~\ref{the:conditionalContraction} a conditional probability tensor $\condprobof{\exrandom}{\secexrandom}$ is the normation of a tensor and by Theorem~\ref{the:normationDirected} a directed tensor.
	Since probability tensors have only non-negative coordinates, their contractions with one-hot encodings also have only non-negative coordinates and also their normations. 
	
	\proofleftsymbol:
	Conversely, let $\hypercoreat{\nodevariables}$ be a directed tensor with $\innodes$ incoming and $\outnodes$ outgoing and non-negative coordinates.
	Then
	\begin{align}
		\probat{\nodevariables} = \frac{1}{\prod_{\node\in\innodes}\catdimof{\node}} \cdot \hypercoreat{\nodevariables}
	\end{align}
	is a probability tensor, since 
	\begin{align*}
		\sum_{\atomlegindexof{\innodes}} \sum_{\atomlegindexof{\outnodes}} \probat{\indexedcatvariableof{\nodes}} =
		\sum_{\atomlegindexof{\innodes}} \sum_{\atomlegindexof{\outnodes}} \frac{1}{\prod_{\node\in\innodes}\catdimof{\node}} \cdot \hypercoreat{\indexedcatvariableof{\nodes}} =
		\sum_{\atomlegindexof{\innodes}} \frac{1}{\prod_{\node\in\innodes}\catdimof{\node}} = 1 \, . 
	\end{align*}
	The conditional probability $\condprobof{\catvariableof{\outnodes}}{\catvariableof{\innodes}}$ coincides with $\hypercore$, since
	\begin{align*}
		\condprobof{\catvariableof{\outnodes}}{\indexedcatvariableof{\innodes}} 
		=& \frac{
		\probat{\catvariableof{\outnodes},\indexedcatvariableof{\innodes}}
		}{
		\sum_{\catindexof{\outnodes}} \probat{\indexedcatvariableof{\outnodes},\indexedcatvariableof{\innodes}}
		} \\
		=& \frac{
		\hypercoreat{\catvariableof{\outnodes},\indexedcatvariableof{\innodes}}
		}{
		\sum_{\catindexof{\outnodes}} \hypercoreat{\indexedcatvariableof{\outnodes},\indexedcatvariableof{\innodes}}
		} 
		= \hypercoreat{\catvariableof{\outnodes},\indexedcatvariableof{\innodes}} \, ,
	\end{align*}
	where in the last equation we used that the denominator is by definition trivial since $\hypercore$ is normed.
\end{proof}


Since conditional probabilities are directed tensors we therefore depict them by
\begin{center}
	\begin{tikzpicture}[scale=0.3,thick] % , baseline = -3.5pt


    \node[anchor=center] (text) at (-2,0) {$a)$};

    \node [circle, draw, thick, fill=gray!50, minimum size = \nodeminsize] (P1) at (0,-3) {\tiny $\exrandom$};
    \node [circle, draw, thick, fill=gray!50, minimum size = \nodeminsize] (P3) at (9,-3) {};

    \node[anchor=center] (text) at (9,-3) {\tiny $\secexrandom$};

    \draw[->]
    (4.5,0) to[bend right=25] (P1);
    \draw[-<-]
    (4.5,0) to[bend right=-25] (P3);

    \node[anchor=center] (text) at (4.5,0.5) {$\edge$};

    \begin{scope}
        [shift={(43,0)}]

        \node[anchor=center] (text) at (-23,0) {$b)$};

        \draw (-21,-1) rectangle (-15,-3);
        \node[anchor=center] (text) at (-18,-2) {\small $\condprobof{\exrandom}{\secexrandom}$};
        \draw[->]  (-20,-3)--(-20,-5) node[midway,left] {\tiny $\exrandom$};
        \draw[-<-]  (-16,-3)--(-16,-5) node[midway,left] {\tiny $\secexrandom$};

    \end{scope}

\end{tikzpicture}
\end{center}


%
Theorem~\ref{the:conditionalDirected} specifies a broad class of tensors to represent conditional probabilities.
In combination with Theorem~\ref{the:rencodingDirected}, which states that relational encodings are directed, we get that any relational encoding of a function is a conditional probability tensor.

\subsection{Bayes Theorem and the Chain Rule}

So far, we have connected concepts of probability theory such as marginal and conditional probabilities with contractions and normations of tensors.
We will now proceed to show that basic theorems of probability theory translate into more general contraction equations.

\begin{theorem}[Bayes Theorem]\label{the:bayes}
	For any probability distribution $\probat{\exrandom, \secexrandom}$ with positive $\probat{\secexrandom}$ we have
	\begin{align*}
		\probat{\exrandom,\secexrandom} 
		= \contractionof{\condprobof{\exrandom}{\secexrandom},\probat{\secexrandom}}{\exrandom,\secexrandom} \, . 
	\end{align*}
\end{theorem}
\begin{proof}
	Directly from the more generic contraction equation Theorem~\ref{the:normationContractionEQ}, since by assumption of positivity of $\probat{\secexrandom}$, the tensor network $\probtensor$ is normable with respect to $\secexrandom$.
\end{proof}


Probability distributions can be decomposed into conditional probabilities, as we demonstrate in the next theorem.

\begin{theorem}[Chain Rule]\label{the:chainRule}
	For any joint probability distribution $\probtensor$ of the variables $\probat{\catvariableof{0},\ldots,\catvariableof{\atomorder-1}}$ we have
	\begin{align*}
		\probtensor = \sbcontractionof{\condprobof{\catvariableof{\atomenumerator},\ldots,\catvariableof{\atomorder-1}}{\catvariableof{0},\ldots,\catvariableof{\atomenumerator-1}}\, : \, \atomenumeratorin\}}{\enumeratedatoms} 
	\end{align*}
	where for $\atomenumerator=0$ we denote by $ \condprobof{\catvariableof{0}}{\catvariableof{0},\ldots,\catvariableof{-1}}$ the marginal distribution $\probat{\catvariableof{0}}$.
\end{theorem}
\begin{proof}
	This follows from Theorem~\ref{the:genericChainRule}.
%	We apply Theorem~\ref{the:bayes} on the distribution
%	\begin{align*}
%	\condprobof{
%	\catvariableof{\atomenumerator},\ldots,\catvariableof{\atomorder}
%	}{
%	\indexedcatvariableof{1},\ldots,\indexedcatvariableof{\atomenumerator-1}
%	} \, ,
%	\end{align*}
%	where $\atomenumeratorin$ and $\catindexof{[\atomorder]}$ are chosen arbitrarly.
%	For any $\atomenumeratorin$ we get
%	\begin{align*}
%		%\contractionof{\{
%			\condprobof{\catvariableof{\atomenumerator},\ldots,\catvariableof{\atomorder-1}}{\catvariableof{1},\ldots,\catvariableof{\atomenumerator-1}}
%		%\}}{} 
%		= \contractionof{\{
%			\condprobof{\catvariableof{\atomenumerator+1},\ldots,\catvariableof{\atomorder-1}}{\catvariableof{1},\ldots,\catvariableof{\atomenumerator-1}},
%			\condprobof{\catvariableof{\atomenumerator}}{\catvariableof{1},\ldots,\catvariableof{\atomenumerator-1}}	
%		\}}{
%			\catvariableof{[\atomorder]} 
%		} \, .
%	\end{align*}
%	Applying this equation iteratively and making use of the commutation of contractions we get for any $\atomenumeratorin$
%	\begin{align*}
%		\condprobof{\catvariableof{\atomenumerator},\ldots,\catvariableof{\atomorder-1}}{\catvariableof{1},\ldots,\catvariableof{\atomenumerator-1}}
%		= \contractionof{\{
%			\condprobof{\catvariableof{\secatomenumerator}}{\catvariableof{1},\ldots,\catvariableof{\atomenumerator-1}} \, : \, \secatomenumerator = \atomenumerator, \atomenumerator +1 , \ldots \atomorder-1
%		\}}{
%			\catvariableof{[\atomorder]} 
%		} \, .
%	\end{align*}
%	For $\atomenumerator=0$, this is the claim.
\end{proof}






\subsection{Independent Variables}

Independence leads to severe simplifications of conditional probabilities and is thus the key assumption to gain sparse decompositions.
We will demonstrate this here applying the chain rule.

\begin{definition}[Independence]\label{def:independence}
	Given a joint distribution of variables $\exrandom$ and $\secexrandom$, we say that $\exrandom$ is independent from $\secexrandom$ if for any values $\exrandind,\secexrandind$ we have
		\[ \probat{\indexedexrandom,\indexedsecexrandom} 
		= \margprobof{\indexedexrandom}{\exrandom}
		 \cdot 
		 \margprobof{\indexedsecexrandom}{\secexrandom} \, . \]
\end{definition}

We give a criterion on independence based on a contraction equation of the probability distribution in the next theorem.

\begin{theorem}\label{the:independenceProductCriterion}
	Given a probability distribution $\probtensor$, $\exrandom$ is independent from $\secexrandom$, if and only if 
	\begin{align*}
		\probat{\exrandom,\secexrandom} 
		= \sbcontractionof{\contractionof{\probtensor}{\exrandom},\contractionof{\probtensor}{\secexrandom}}{\exrandom,\secexrandom} \, . 
	\end{align*}
\end{theorem}
\begin{proof}
	By Theorem~\ref{the:marginalContraction} we know that marginal probabilities are equivalent to contracted probability distributions, i.e. $\probat{\exrandom} = \contractionof{\{\probtensor\}}{\exrandom} $.
	By orthogonality of one-hot encodings we have that
	\begin{align*}
		\forall \exrandind, \secexrandind : \quad  \probat{\indexedexrandom,\indexedsecexrandom} 
		= \margprobof{\indexedexrandom}{\exrandom}
		 \cdot 
		 \margprobof{\indexedsecexrandom}{\secexrandom} 
	\end{align*}
	is equivalent to 
	\begin{align*}
		\sum_{\exrandind}\sum_{\secexrandind} \probat{\indexedexrandom,\indexedsecexrandom} \cdot \onehotmapofat{\exrandind}{\exrandom}\onehotmapofat{\secexrandind}{\secexrandom}
		= \sum_{\exrandind}\sum_{\secexrandind} 
		\margprobof{\indexedexrandom}{\exrandom}
		 \cdot 
		 \margprobof{\indexedsecexrandom}{\secexrandom} \cdot \onehotmapofat{\exrandind}{\exrandom}\onehotmapofat{\secexrandind}{\secexrandom} \, .
	\end{align*}
	We reorder the summations and arrive at
	\begin{align*}
		\sum_{\exrandind,\secexrandind} 
		\probat{\indexedexrandom,\indexedsecexrandom} \cdot \onehotmapofat{\exrandind,\secexrandind}{\exrandom, \secexrandom}
		= \left(\sum_{\exrandind}\margprobof{\indexedexrandom}{\exrandom} \onehotmapofat{\exrandind}{\exrandom} \right)
		\cdot 
		\left( \sum_{\secexrandind}  \margprobof{\indexedsecexrandom}{\secexrandom} \cdot \onehotmapofat{\secexrandind}{\secexrandom}  \right) 
	\end{align*}
	which is by Lemma~\ref{lem:tensorBasisDecomposition} equal to the claim
	\begin{align*}
		\probat{\exrandom,\secexrandom} = \sbcontractionof{\contractionof{\probtensor}{\exrandom},\contractionof{\probtensor}{\secexrandom}}{\exrandom,\secexrandom} \, . 
	\end{align*}
\end{proof}


% Usage for tensor decompositions
Independent variables result in decompositions of $\probtensor$ in a tensor product of marginal probability tensors. 
Having pairwise independent variables reduces the degrees of freedom from exponentially many in the number of atoms to linear.

In the tensor network decomposition we depict this by
	\begin{center}
		\input{PartI/tikz_pics/probability_decomposition/independent_decomposition.tex}
	\end{center}

Independence is a very strong assumption, which is often too restrictive.
Conditional independence instead is a less demanding assumption, when certain conditional distribution variables are independent. 
This leads to tensor network decompositions with a more realistic assumption.

\begin{definition}[Conditional Independence]\label{def:condIndependence}
	Given a joint distribution of variables $\exrandom$, $\secexrandom$ and $\thirdexrandom$, we say $\exrandom$ is independent from $\secexrandom$ conditioned on $\thirdexrandom$ if for any incides $\exrandind,\secexrandind$ and $\thirdexrandind$
		\[ \condprobof{\indexedexrandom,\indexedsecexrandom}{\indexedthirdexrandom} 
		= \condprobof{\indexedexrandom}{\indexedthirdexrandom} 
		\cdot \condprobof{\indexedsecexrandom}{\indexedthirdexrandom}   \, . \]
\end{definition}

Conditional independence is a relation between conditional probabilities and is therefore equivalent to a normation equation stated next.

\begin{theorem}[Conditional Independence as a Contraction Equation]\label{the:condIndependenceProductCriterion}
	Given a distribution $\probtensor$ of variables $\exrandom$, $\secexrandom$ and $\thirdexrandom$, the variable $\exrandom$ is independent from $\secexrandom$ if and only if the contraction equation
	\begin{align*}
		 \condprobof{\exrandom,\secexrandom}{\thirdexrandom} 
		 = \sbcontractionof{
		 \condprobof{\exrandom}{\thirdexrandom} ,\condprobof{\secexrandom}{\thirdexrandom} 
		 }{\exrandom,\secexrandom,\thirdexrandom}
	\end{align*}
	holds.
\end{theorem}
\begin{proof}
	Directly by Theorem~\ref{the:conditionalContraction} used on the conditional probabilities in Definition~\ref{def:condIndependence}.
\end{proof}

We can exploit conditional independence to find tensor network decompositions of probability tensors, as we show in the next theorem.

\begin{corollary}\label{cor:secCriterionCondIndepencence}
	If and only if $\exrandom$ is independent from $\secexrandom$ conditioned on $\thirdexrandom$ the probability distribution $\probtensor$ satisfies
		\[ \probat{\exrandom, \secexrandom, \thirdexrandom} 
		= \contractionof{
			\{ \condprobof{\exrandom}{\thirdexrandom}, \condprobof{\secexrandom}{\thirdexrandom}, \margprobof{\thirdexrandom}{\thirdexrandom} \}
		}{
			\exrandom, \secexrandom, \thirdexrandom
		} \, .
		\]
\end{corollary}
\begin{proof}
	Follows from Theorem~\ref{the:condIndependenceProductCriterion} and Theorem~\ref{the:bayes}.
%	We start with the chain rule decomposition of Theorem~\ref{the:chainRule} and have
%		\[ \probat{\exrandom,\secexrandom,\thirdexrandom} = \probat{\thirdexrandom}  \cdot \condprobof{\exrandom,\secexrandom}{\thirdexrandom} \]
%	Since $\exrandom$ is independent from $\secexrandom$ conditioned on $\thirdexrandom$ we have
%		\[ \condprobof{\exrandom,\secexrandom}{\thirdexrandom}  = \condprobof{\exrandom}{\thirdexrandom}  \cdot \condprobof{\secexrandom}{\thirdexrandom}  \, . \]
%	Converse direction similar.
\end{proof}


\begin{corollary}\label{cor:conditionDropping}
	Whenever $\exrandom$ is independent of $\secexrandom$ given $\thirdexrandom$, we have
	\begin{align*}
		\condprobof{\exrandom}{\secexrandom,\thirdexrandom} = \condprobof{\exrandom}{\thirdexrandom} \, .
	\end{align*}
\end{corollary}


\begin{figure}[h]
\begin{center}
	\input{PartI/tikz_pics/probability_decomposition/cond_independence_decomposition.tex}
\end{center}
\caption{Diagrammatic visualization of the contraction equation in Corollary~\ref{cor:secCriterionCondIndepencence}. Conditional independence of $\exrandom$ and $\secexrandom$ given $\thirdexrandom$ holds if the contraction on the right ride is equal to the probability tensor on the left side.}
\end{figure}



% More of an example?
\begin{theorem}[Markov Chain]\label{the:MarkovChain}
	Let there be a set of variables $\catvariableof{\tenumerator}$ where $\tenumeratorin$.
	When $\catvariableof{\tenumerator}$ is independent of $\catvariableof{0:{\tenumerator-2}}$ conditioned on $\catvariableof{\tenumerator-1}$ (the Markov Property), then
	\begin{align*}
		\probtensor = \contractionof{\{ \condprobof{\catvariableof{\tenumerator}}{\catvariableof{\tenumerator-1}}\, : \, \tenumeratorin \}}{
		\catvariableof{0},\ldots,\catvariableof{\tdim-1}
		} 
	\end{align*}	
%		\[ \probat{\catvariableof{0},\ldots,\catvariableof{\tdim-1}} = %\probat{\catvariableof{1}} 
%		\prod_{\tenumeratorin} \condprobof{\catvariableof{\tenumerator}}{\catvariableof{\tenumerator-1}} \, . \] 
	We depict this decomposition in Figure~\ref{fig:MC}.
\end{theorem}
\begin{proof}
	By the chain rule (Theorem~\ref{the:chainRule}) we have
	\begin{align*}
	 	\probat{\catvariableof{0},\ldots,\catvariableof{\tdim-1}}
		= \contractionof{
		\{ \condprobof{\catvariableof{\tenumerator}}{\catvariableof{0:\tenumerator}} : \tenumeratorin \}
		}{\catvariableof{[\tdim]}}
		%= \contractionof{\{\probat{\catvariableof{0}} \prod_{\tenumeratorin, \tenumerator>1} \condprobof{\catvariableof{\tenumerator}}{\catvariableof{0:\tenumerator}}\}{\catvariableof{[\tdim]}} \, . 
	\end{align*}
	Using the conditional independence of $\catvariableof{\tenumerator}$ and $\catvariableof{0:{\tenumerator-2}}$ conditioned on $\catvariableof{\tenumerator-1}$ we further have by Corollary~\ref{cor:conditionDropping}
		\[ \condprobof{\catvariableof{\tenumerator}}{\indexedcatvariableof{0:\tenumerator}}  = \condprobof{\catvariableof{\tenumerator}}{\indexedcatvariableof{\tenumerator-1}} \, .  \]
	Composing both equalities shows the claim.
\end{proof}

Here we denoted by $\catvariableof{0:\tenumerator}$ the tuple $\catvariableof{0},...,\catvariableof{\tenumerator}$.

\begin{remark}
	Let us notice that the dimensionality dropped drastically through applying the independence assumption.
	The tensor space in the naive representation of any probability distribution has
		\[ \prod_{\tenumeratorin} \catdimof{\tenumerator}\]
	coordinates, while the Markov Chain is represented by
		\[ \sum_{\tenumeratorin}  \catdimof{\tenumerator}\cdot \catdimof{\tenumerator-1} \, . \]
	Replacing exponential scaling with the number of variables to linear scaling is the advantage of tensor network decompositions.
\end{remark}

\begin{figure}[h]
\begin{center}
	\begin{tikzpicture}[scale=0.3,thick] % , baseline = -3.5pt

\node[anchor=center] (text) at (-1,3) {${a)}$};

	\node [circle, draw, thick, fill=gray!50] (T1) at (0,0) {\tiny $\randomxof{0}$};
	\node [circle, draw, thick, fill=gray!50] (T2) at (5,0) {\tiny $\randomxof{1}$};
	\draw[->] (T1) -- (T2);
	\node [circle, draw, thick, fill=gray!50] (T3) at (10,0) {\tiny $\randomxof{2}$};
	\draw[->] (T2) -- (T3);
	\node [circle, draw, thick, fill=gray!50] (T4) at (15,0) {\tiny $\randomxof{3}$};
	\draw[->] (T3) -- (T4);
	\draw[->] (T4) -- (18,0);

	\node[anchor=center] (text) at (19,0) {$\cdots$};

	%\node [circle, draw, thick, fill=gray!50] (T4) at (17,0) {\tiny $\randomxof{\atomorder}$};
	%\draw[->] (14,0) -- (T4);	
			

\begin{scope}[shift={(25,0)}]

\node[anchor=center] (text) at (-3,3) {${b)}$};

\draw (-3.5,-1) rectangle (0, 1);
\node[anchor=center] (text) at (-1.75,0) {\small $\probat{\randomxof{0}}$};
\draw[->] (0,0) -- (2,0);
\draw[fill] (1,0) circle (0.25cm);
\draw[->] (1,0) -- (1,2) node[above] {\tiny $\catvariableof{0}$};
\draw (2,-1) rectangle (7, 1);
\node[anchor=center] (text) at (4.5,0) {\small $\condprobof{\randomxof{1}}{\randomxof{0}}$};
\draw[->]  (7,0) -- (9,0);
\draw[fill] (8,0) circle (0.25cm);
\draw[->] (8,0) -- (8,2) node[above] {\tiny $\catvariableof{1}$};
\draw (9,-1) rectangle (14, 1);
\node[anchor=center] (text) at (11.5,0) {\small $\condprobof{\randomxof{2}}{\randomxof{1}}$};
\draw[->]  (14,0) -- (16,0);
\draw[fill] (15,0) circle (0.25cm);
\draw[->] (15,0) -- (15,2) node[above] {\tiny $\catvariableof{2}$};
\draw (16,-1) rectangle (21, 1);
\node[anchor=center] (text) at (18.5,0) {\small $\condprobof{\randomxof{3}}{\randomxof{2}}$};
\draw[->]  (21,0) -- (23,0);
\draw[fill] (22,0) circle (0.25cm);
\draw[->] (22,0) -- (22,2) node[above] {\tiny $\catvariableof{3}$};
\node[anchor=center] (text) at (24,0) {$\cdots$};


\end{scope}

\end{tikzpicture} 
\end{center}
\caption{Depiction of a Markov Chain. 
	a) Dependency Graph (of the corresponding chain Graphical Model).
	b) Dual Tensor Network representing the conditional probability factors.}
\label{fig:MC}
\end{figure}





\subsection{Graphical Models}



We have already depicted conditional dependency assumptions made for Markov Chains in Figure~\ref{fig:MC} and discussed the implied decomposition of the dual tensor networks.
Graphical models provide a more general framework for conditional dependency assumptions and provide a generic approach to exploit independences in finding tensor network decompositions of $\probtensor$.


%Graphical Models are typically depicted by nodes to each variable and edges.
Following the tensor network formalism we in this section introduce graphical models based on hypergraphs.
Whether the hypergraph is directed or not distinguished between Bayesian Networks and Markov Networks.




%\begin{remark}[Further nomenclature]
%	The factors of the graphical models are tensors (since multivariate functions of discrete variables).
%	The edges are associated to each axis of the tensor and carry the variables.
%	Since each edge variable can appear in multiple factors, the Tensor Network is defined on a Hypergraph, where edges are interpreted as Hadamard contractions.
%\end{remark}



\subsubsection{Markov Networks}

While typically Markov Networks are defined on graphs, we define them here on hypergraphs to establish a direct connection to tensor networks defined on the same hypergraph.
Along that line, Markov Networks are tensor networks with non-negative tensors (see Definition~\ref{def:tensorNetwork}), which are interpreted as probability distributions after normation.

\begin{definition}\label{def:markovNetwork}
	Let $\tnetof{\graph}$ be a Tensor Network of non-negative tensors on a hypergraph $\graph$.
	Then the Markov Network to $\tnetof{\graph}$ is the probability distribution of $\catvariableof{\node}$ defined by the tensor
		\[ \probofat{\graph}{\nodevariables} = \frac{
			\contractionof{\{\hypercoreof{\edge} : \edge \in \edges\}}{\nodevariables} 
		}{
			\contraction{\{\hypercoreof{\edge} : \edge \in \edges\}}
		} = \normationof{\tnetof{\graph}}{\nodevariables} \, . \] 
	We call the denominator
		\[\partitionfunctionof{\tnetof{\graph}} = \contraction{\{\hypercoreof{\edge} : \edge \in \edges\}} \]
	the partition function of the Markov Network.
\end{definition}

% Marginalization and Conditioning
Often, we are only interested in the distribution of a subset of variables, which are called the observable variables, and call the other variables hidden variables.
The marginalization of a Markov Network to $\tnetof{\graph}$ on the variables $\catvariableof{\secnodes}$ is
	\[
		\probofat{\graph}{\catvariableof{\secnodes}}
		= \normationof{\tnetof{\graph}}{\catvariableof{\secnodes}} \, . 
	\]
This can be derived from Theorem~\ref{the:splittingContractions}, which established an equivalence of contractions with sequences of consecutive contractions.


Further, the distribution of $\catvariableof{\secnodes}$ conditioned on $\catvariableof{\thirdnodes}$, where $\secnodes,\thirdnodes$ are disjoint subsets of $\nodes$, is
	\[
		\probtensor^{\graph}\left[ \catvariableof{\secnodes} | \catvariableof{\thirdnodes}\right] 
		= \normationofwrt{\tnetof{\graph}}{\catvariableof{\secnodes}}{\catvariableof{\thirdnodes}} \, . 
	\]

\begin{definition}[Separation of Hypergraph]
	A path in a hypergraph is a sequence of nodes $\node_{\atomenumerator}$ for $\atomenumeratorin$, such that for any $\atomenumerator\in[\atomorder-1]$ we find a hyperedge $\edge\in\edges$ such that $(\node_{\atomenumerator}, \node_{\atomenumerator+1})\subset \edge$.
	Given disjoint subsets $\nodesa$, $\nodesb$, $\nodesc$ of nodes in a hypergraph $\graph$ we say that $\nodesc$ separates $\nodesa$ and $\nodesb$ with respect to $\graph$, when any path starting at a node in $\nodesa$ and ending in a node in $\nodesb$ contains a node in $\nodesc$.
	%when removing the hyperedges which are contained in $\nodesc$ leads to a hypergraph with no path of hyperedges between a node in $\nodesa$ to a node in $\nodesb$.
\end{definition}

To characterize Markov Networks in terms of conditional independencies we need to further define the property of clique-capturing.
This property of clique-capturing established a correspondence of hyperedges with maximal cliques in an alternative graph-based definition of Markov Networks \cite{koller_probabilistic_2009}.

\begin{definition}[Clique-Capturing Hypergraph]\label{def:ccHypergraph}
	We call a hypergraph $\graph$ clique-capturing, when each subset $\secnodes\subset\nodes$ is contained in a hyperedge, if for any $a,b\in\secnodes$ there is a hyperedge $\edge\in\edges$ with $a,b\in\secnodes$.
\end{definition}

Let us now show a characterization of Markov Networks in terms of conditional independencies, which is analogous to Theorem~\ref{the:condIndBN}.

% Characterization
\begin{theorem}\label{the:condIndMN}
	Given a clique-capturing hypergraph $\graph$, the set of positive Markov Networks on the hypergraph coincides with the set of positive probability distributions, such that each for each disjoint subsets of variables $\nodesa$, $\nodesb$, $\nodesc$ we have $\catvariableof{\nodesa}$ is independent of $\catvariableof{\nodesb}$ conditioned on $\catvariableof{\nodesc}$, when $\nodesc$ separates $\nodesa$ and $\nodesb$ in the hypergraph. % called d-separation
\end{theorem}
\begin{proof}
	%=>
	%Given any Markov Network, contracting with $\onehotmapof{\atomlegindexof{\nodesc}}$ turns all hyperedges contained in $\nodesc$ to scalar factors (copying possible).
	Let there be a hypergraph $\graph$, a Markov Network $\extnet$ on $\graph$ and nodes $\nodesa,\nodesb,\nodesc \subset \nodes$, such that $\nodesc$ separates $\nodesa$ from $\nodesb$.
	Let us denote by $\nodes_0$ the nodes with paths to $\nodesa$, which do not contain a node in $\nodesc$, and by $\nodes_1$ the nodes with paths to $\nodesb$, which do not contain a node in $\nodesc$.
	Further, we denote by $\edges_0$ the hyperedges which contain a node in $\nodes_0$ and by $\edges_1$ the hyperedges which contain a node in $\nodes_1$.
	By assumption of separability, both sets $\edges_0$ and $\edges_1$ are disjoint and no node in $\nodesa$ is in a hyperedge in $\edges_1$, respectively no node in $\nodesb$ is in a hyperedge in $\edges_0$, .
	We then have
	\begin{align*}
		\normationofwrt{\extnetasset}{\catvariableof{\nodesa},\catvariableof{\nodesb}}{\indexedcatvariableof{\nodesc}} 
		= & \normationof{\extnetasset\cup\{\onehotmapof{\catindexof{\nodesc}}\}}{\catvariableof{\nodesa},\catvariableof{\nodesb}} \\
		= &  \normationof{\{\hypercoreof{\edge}\, : \, \edge\in\edges_0\}\cup\{\onehotmapof{\catindexof{\nodesc}}\}}{\catvariableof{\nodesa}}
		\otimes \normationof{\{\hypercoreof{\edge}\, : \, \edge\in\edges_1\}\cup\{\onehotmapof{\catindexof{\nodesc}}\}}{\catvariableof{\nodesb}} \, .
	\end{align*}
	By Theorem~\ref{the:condIndependenceProductCriterion}, it now follows that $\catvariableof{\nodesa}$ is independent of $\catvariableof{\nodesb}$ conditioned on $\catvariableof{\nodesc}$.
	%<= HARDER! Hammersley Clifford needed
	The converse direction, i.e. that positive distributions respecting the conditional indpendence assumptions are representable as Markov Networks, is known as the Hammersley Clifford Theorem, 
	which we will proof later in Section~\ref{sec:proofHCTheorem}.
	%for which proof we refer to Theorem~4.8 in KOLLER.
\end{proof}

% Positivity
From the proof of Theorem~\ref{the:condIndMN} Markov Networks with zero coordinates still satisfy the conditional independence assumption.
However, the reverse is not true, that is there are distributions with vanishing coordinates, which satisfy the conditional independence assumptions, but cannot be represented as a Markov Network (see Example~4.4 in \cite{koller_probabilistic_2009}).




\subsubsection{Bayesian Networks}

Bayesian networks are described by directed acyclic graphs (DAG).
The probability distribution is a Hadamard product of conditional probabilities, where each variable has a conditional probability factor conditioned on the parents variables in the graph.

We introduce Bayesian Networks based on directed hypergraphs (see Definition~\ref{def:hypergraphs}) and define further properties.

\begin{definition}
%% Already in notation chapter 
%	A directed hypergraph is a tuple $\graph=(\nodes,\edges)$ of nodes $\nodes$ and hyperedges $\edges$, where each hyperedge $\edge\in\edges$ is a tuple
%		\[ \edge = (\incomingnodes,\outgoingnodes) \]
%	of disjoint sets of incoming nodes $\incomingnodes\subset\nodes$ and outgoing nodes $\outgoingnodes\subset\nodes$.
	A directed path is a sequence $\node_{0},\ldots\node_{\secatomorder}$ such that for any $\secatomenumeratorin$ there is an hyperedge $\edge=(\incomingnodes,\outgoingnodes)\in\edges$ such that $\node_{\secatomenumerator}\in\incomingnodes$ and $\node_{\secatomenumerator+1}\in\outgoingnodes$.
	We call the hypergraph $\graph$ acyclic, if there is no path with $\secatomorder>0$ such that $\node_{0}=\node_{\secatomorder}$.
	Given a directed hypergraph $\graph=(\nodes,\edges)$ we define for any node $\nodein$ its parents by
		\[ \parentsof{\node} = \{\secnode \, : \, \exists\edge=(\incomingnodes,\outgoingnodes)\in\edges: \secnode\in\incomingnodes,\node\in\outgoingnodes \} \]
	and its non-descendants $\nondescendantsof{\node}$ as the set of nodes $\secnode$, such that there is no directed path from $\node$ to $\secnode$.
\end{definition}

\begin{definition}
	%Let $\nodes$ be a set of nodes decorated by dimensions and 
	Let $\graph=(\nodes,\edges)$ be a directed acyclic hypergraph with edges of the form 
		\[ \edges = \{(\parentsof{\node},\{\node\}) \, : \, \nodein\} \, . \]
%	and for each node $\node\in\nodes$ a random variable $\catvariableof{\node}$.
	A \emph{Bayesian Network} is a decoration of each edge $(\parentsof{\node},\{\node\})$ by a conditional probability distribution
	%Further let there be for each node $\node\in\nodes$ with parents $\parentsof{\node}$ a conditional probability distribution
		\[ \condprobof{\catvariableof{\node}}{\catvariableof{\parentsof{\node}}} \]
	which represents the probability distribution
%	Then the \emph{Bayesian Network} with respect to $\graph$ and the conditional probability terms is the distribution
	\begin{align*}
		\probat{\nodevariables} = \contractionof{\{\condprobof{\catvariableof{\node}}{\catvariableof{\parentsof{\node}}} \, : \, \nodein\}}{\nodevariables} \, .
	\end{align*}
%		\[ \probat{\catvariableof{\node} \, : \, \node\in\nodes } = \prod_{\node\in\nodes} \condprobof{\catvariableof{\node}}{\catvariableof{\parentsof{\node}}} \, . \]
\end{definition}

%
By definition each tensor decorating a hyperedge is directed with $\catvariableof{\parentsof{\node}}$ incoming and $\catvariableof{\node}$ outgoing.
Thus, the directionality of the hypergraph is reflected in each tensor decorating a directed hyperedge.
This allows us to verify with Theorem~\ref{the:conditionalContractionPreservation} that their contraction defines a probability distribution.

% Contraction -> Now in definition!
%By definition we can represent a Bayesian network by the contraction
%\begin{align*}
%	\probtensorof{\graph} = \sbcontractionof{\{ \condprobof{\catvariableof{\node}}{\catvariableof{\parentsof{\node}}} \, : \, \node\in\nodes\}}{\nodes} \, . 
%\end{align*}

% Dual
%The dual tensor network consists of conditional probability distributions to each node $\node\in\nodes$ (see Figure~\ref{fig:BayesianFactor}b).

\begin{figure}[h]
\begin{center}
	\begin{tikzpicture}[scale=0.35,thick] % , baseline = -3.5pt

\node[anchor=center] (text) at (-1,3) {${a)}$};

	\node [circle, draw, thick, fill=gray!50] (H) at (5,0) {\tiny $\randomxof{\node}$};
	\node [circle, draw, thick, fill=gray!50] (P1) at (0,-5) {\tiny $\randomxof{0}$};	
	\node [circle, draw, thick, fill=gray!50] (P2) at (5,-5) {\tiny $\randomxof{1}$};	
	
	\node[anchor=center] (text) at (10,-5) {$\cdots$};
	\node [circle, draw, thick, fill=gray!50] (Pd) at (15,-5) {\tiny $\randomxof{\atomorder\shortminus1}$};
	
	\node [] (E) at (5,-2) {};	
	
	\draw[midarrow] (P1) -- (5,-2) ;	
	\draw[midarrow] (P2) -- (5,-2) ;	
	\draw[midarrow] (Pd) -- (5,-2) ;	
	\draw[midarrow] (5,-2) -- (H) ;	
			

\begin{scope}[shift={(25,0)}]

\node[anchor=center] (text) at (-3,3) {${b)}$};

\draw[->] (4.5,-1) -- (4.5,1) node[midway, right]{\tiny $\catvariableof{\node}$};
\draw (0,-1) rectangle (9,-4); 
\node[anchor=center] (text) at (4.5,-2.5) {\small $\condprobof{\randomxof{\node}}{\randomxof{[\atomorder]}} $};
\draw[->] (1,-6) -- (1,-4) node[midway, right]{\tiny $\catvariableof{0}$};
\draw[->] (2.5,-6) -- (2.5,-4) node[midway, right]{\tiny $\catvariableof{1}$};

\node[anchor=center] (text) at (5.5,-5) {$\cdots$};
	
\draw[->] (8,-6) -- (8,-4) node[midway, right]{\tiny $\catvariableof{\atomorder\shortminus1}$};

\end{scope}

\end{tikzpicture} 
\end{center}
\caption{Example of a Factor of a Bayesian Network to the node $\catvariableof{\node}$ with parents $\catvariableof{0},\ldots,\catvariableof{\catorder-1}$, as subgraph $a)$ and dual tensor core $b)$.}
\label{fig:BayesianFactor}
\end{figure}


%% Marginalization and Contraction
Marginalization of a Bayesian Network are still Bayesian Networks on a graph where the edges directing to variables, which are not marginalized over, are replaced by directed edges to the children.
Conditioned Bayesian Network do not have a simple Bayesian Network representation, which is why we will treat them as Markov Networks to be introduced next.


\begin{theorem}[Independence Characterization of Bayesian Networks]\label{the:condIndBN}
	A probability distribution $\probat{\nodevariables}$ has a representation by a Bayesian Network on a directed acyclic graph $\graph=(\nodes,\edges)$, if and only if for any $\nodein$ the variables $\catvariableof{\node}$ are independent on $\nondescendantsof{\node}$ conditioned on $\parentsof{\node}$.
\end{theorem}
\begin{proof}
	We choose a topological order $\prec$ on the nodes of $\graph$, which exists since $\graph$ is acyclic.
	% =>
	Let us assume, that the conditional independencies are satisfied and apply the chain rule with respect to that ordering to get
	\begin{align*}
		\probat{\nodevariables} =
		\contractionof{
			\condprobof{\catvariableof{\node}}{\catvariableof{\secnode} : \secnode \prec \node}
		}
		{\nodevariables} \, .
	\end{align*}
	Since $\prec$ is a topological ordering we have
		\[ \parentsof{\node} \subset \{\secnode : \secnode \prec \node\} \]
	We apply the assumed conditional independence with Corollary~\ref{cor:conditionDropping} and get
	\begin{align*}
		\probat{\nodevariables} =
		\contractionof{
			\condprobof{\catvariableof{\node}}{\catvariableof{\parentsof{\node}}}
		}
		{\nodevariables} \, .
	\end{align*}
	% <=
	To show the converse direction, let there be a Bayesian Network $\probat{\nodevariables}$ on $\graph$.
	To show for any node $\node$, that $\catvariableof{\node}$ is independent of $\nondescendantsof{\node}$ conditioned on $\parentsof{\node}$, we reorder the tensors in the contraction
	%with respect to a set $\node_0$ 
	\begin{align*}
		& \condprobof{\catvariableof{\node},\catvariableof{\nondescendantsof{\node}}}{\indexedcatvariableof{\parentsof{\node}}} \\
		& \quad\quad = \normationofwrt{
			\{\condprobof{\catvariableof{\secnode}}{\catvariableof{\parentsof{\secnode}}} \, : \, \secnode\in\nodes\}
		}
		{\catvariableof{\node},\catvariableof{\nondescendantsof{\node}}}
		{\indexedcatvariableof{\parentsof{\node}}} \\
		& \quad\quad  = \normationof{
			\{\condprobof{\catvariableof{\secnode}}{\catvariableof{\parentsof{\secnode}}} \, : \, \secnode\in\nodes\} \cup \{\onehotmapof{\catindexof{\parentsof{\node}}}\}
		}
		{\catvariableof{\node},\catvariableof{\nondescendantsof{\node}}}\\
		&  \quad\quad = \normationof{
			\{\condprobof{\catvariableof{\secnode}}{\catvariableof{\parentsof{\secnode}}} \, : \, \secnode\in\nondescendantsof{\node}\} \cup \{\onehotmapof{\catindexof{\parentsof{\node}}}, \condprobof{\catvariableof{\node}}{\catvariableof{\parentsof{\node}}} \}
		}
		{\catvariableof{\node},\catvariableof{\nondescendantsof{\node}}} \\
		&  \quad\quad =  %\contractionof{
		 \normationof{
			\{\condprobof{\catvariableof{\secnode}}{\catvariableof{\parentsof{\secnode}}} \, : \, \secnode\in\nondescendantsof{\node}\} \cup \{\onehotmapof{\catindexof{\parentsof{\node}}}\}
		}
		{\catvariableof{\nondescendantsof{\node}}} \\
		& \quad\quad  \quad  \cdot \normationof{
			\{\condprobof{\catvariableof{\node}}{\catvariableof{\parentsof{\node}}},\onehotmapof{\catindexof{\parentsof{\node}}}\}
		}
		{\catvariableof{\node}} \\
		& \quad\quad  = \contractionof{\{
		\condprobof{\catvariableof{\nondescendantsof{\node}}}{\indexedcatvariableof{\parentsof{\node}}},
		\condprobof{\catvariableof{\node}}{\indexedcatvariableof{\parentsof{\node}}}
		\}}{\catvariableof{\node},\catvariableof{\nondescendantsof{\node}}}
		%}{\catvariableof{\node},\catvariableof{\nondescendantsof{\node}}}
	\end{align*}
	Here we have dropped in the third equation all tensors to the descendants, since their marginalization is trivial (which can be shown by a leaf-stripping argument).
	In the fourth equation we made use of the fact, that any directed path between the non-descendants and the node is through the parents of the node.
	By Theorem~\ref{the:condIndependenceProductCriterion}, it now follows that $\catvariableof{\node}$ is independent of $\nondescendantsof{\node}$ conditioned on $\parentsof{\node}$.
\end{proof}



\subsubsection{Example of a Bayesian Network: Hidden Markov Models}

We here extend the example of Markov Chains from Theorem \ref{the:MarkovChain} to a limited observation of the variables by observations.
Let there be the variables $\catvariableof{\tenumerator}$ (states) and $\randomeof{\tenumerator}$ (observations) with a discrete and finite time $\tenumeratorin$.

The conditional assumptions are 
\begin{itemize}
	\item $\catvariableof{\tenumerator+1}$ is independent of $\catvariableof{0:\tenumerator-1}$ and $\randomeof{0:\tenumerator}$ conditioned on $\catvariableof{\tenumerator}$
	\item $\randomeof{\tenumerator}$ is independent of all other variables conditioned on $\catvariableof{\tenumerator}$
\end{itemize}

Then the probability tensor has the decomposition 
\begin{align}
	\probat{\catvariableof{0:\tdim},\randomeof{0:\tdim}} 
	& = \prod_{\tenumeratorin}
	 \left( \condprobof{\catvariableof{\tenumerator}}{\catvariableof{0:\tenumerator-1},\randomeof{0:\tenumerator-1}} \cdot \condprobof{\randomeof{\tenumerator}}{\catvariableof{0:\tenumerator},\randomeof{0:\tenumerator-1}} \right) \\
	& = \probat{\catvariableof{0}} \cdot \condprobof{\randomeof{0}}{\catvariableof{0}} \cdot \prod_{\tenumeratorin, \tenumerator>0} 
	\left( \condprobof{\catvariableof{\tenumerator}}{\catvariableof{\tenumerator-1}} \cdot \condprobof{\randomeof{\tenumerator}}{\catvariableof{\tenumerator}} \right)
\end{align}
Here we used the Chain Rule decomposition of Theorem~\ref{the:chainRule} in the first equation and the conditional independence assumptions in the second.

We notice, that this is a Bayesian Netowork on a directed acyclic hypergraph $\graph$ consistent in nodes to each state and each observation and directed hyperedges
\begin{itemize}
	\item $(\{\catvariableof{\tenumerator}\}, \{\catvariableof{\tenumerator+1}\})$ for $\tenumerator\in[\tdim-1]$
	\item $(\{\catvariableof{\tenumerator}\}, \{\randomeof{\tenumerator}\})$ for $\tenumeratorin$
\end{itemize}


\begin{figure}[h]
\begin{center}
	\begin{tikzpicture}[scale=0.3,thick] % , baseline = -3.5pt

\node[anchor=center] (text) at (-1,3) {${a)}$};

	\node [circle, draw, thick, fill=gray!50] (T1) at (0,0) {\tiny $\catvariableof{0}$};
	\node [circle, draw, thick, fill=gray!50] (E1) at (0,-4) {\tiny $\randomeof{0}$};
	\draw[->] (T1) -- (E1);	
	\node [circle, draw, thick, fill=gray!50] (T2) at (4,0) {\tiny $\catvariableof{1}$};
	\node [circle, draw, thick, fill=gray!50] (E2) at (4,-4) {\tiny $\randomeof{1}$};
	\draw[->] (T2) -- (E2);	
	\draw[->] (T1) -- (T2);	
	\node [circle, draw, thick, fill=gray!50] (T3) at (8,0) {\tiny $\catvariableof{2}$};
	\node [circle, draw, thick, fill=gray!50] (E3) at (8,-4) {\tiny $\randomeof{2}$};
	\draw[->] (T3) -- (E3);	
	\draw[->] (T2) -- (T3);
	\node [circle, draw, thick, fill=gray!50] (T4) at (12,0) {\tiny $\catvariableof{3}$};
	\node [circle, draw, thick, fill=gray!50] (E4) at (12,-4) {\tiny $\randomeof{3}$};
	\draw[->] (T4) -- (E4);	
	\draw[->] (T3) -- (T4);
	\draw[->] (T4) -- (15,0);

	\node[anchor=center] (text) at (16,0) {$\cdots$};

	%\node [circle, draw, thick, fill=gray!50] (T4) at (17,0) {\tiny $\catvariableof{\atomorder}$};
	%\draw[->] (14,0) -- (T4);	
			

\begin{scope}[shift={(22,0)}]

\node[anchor=center] (text) at (-3,3) {${b)}$};

\draw (-3.5,-1) rectangle (0, 1);
\node[anchor=center] (text) at (-1.75,0) {\small $\probat{\catvariableof{0}}$};
\draw[->] (0,0) -- (2,0);
\draw[fill] (1,0) circle (0.15cm);
\draw[->] (1,0) -- (1,2) node[above] {\tiny ${\catvariableof{0}}$};
\draw[->] (1,0) -- (1,-2);
\draw (-1.5,-2) rectangle (3.5,-4); 
\node[anchor=center] (text) at (1,-3) {\small $\condprobof{\randomeof{0}}{\catvariableof{0}}$};
\draw[->] (1,-4) -- (1,-6) node[midway, right]{\tiny ${\randomeof{0}}$};

\draw (2,-1) rectangle (7, 1);
\node[anchor=center] (text) at (4.5,0) {\small $\condprobof{\catvariableof{1}}{\catvariableof{0}}$};
\draw[->]  (7,0) -- (9,0);
\draw[fill] (8,0) circle (0.15cm);
\draw[->] (8,0) -- (8,2) node[above] {\tiny ${\catvariableof{1}}$};
\draw[->] (8,0) -- (8,-2);
\draw (5.5,-2) rectangle (10.5,-4); 
\node[anchor=center] (text) at (8,-3) {\small $\condprobof{\randomeof{1}}{\catvariableof{1}}$};
\draw[->] (8,-4) -- (8,-6) node[midway, right]{\tiny ${\randomeof{1}}$};


\draw (9,-1) rectangle (14, 1);
\node[anchor=center] (text) at (11.5,0) {\small $\condprobof{\catvariableof{2}}{\catvariableof{1}}$};
\draw[->]  (14,0) -- (16,0);
\draw[fill] (15,0) circle (0.15cm);
\draw[->] (15,0) -- (15,2) node[above] {\tiny ${\catvariableof{2}}$};
\draw[->] (15,0) -- (15,-2);
\draw (12.5,-2) rectangle (17.5,-4); 
\node[anchor=center] (text) at (15,-3) {\small $\condprobof{\randomeof{2}}{\catvariableof{2}}$};
\draw[->] (15,-4) -- (15,-6) node[midway, right]{\tiny ${\randomeof{2}}$};

\draw (16,-1) rectangle (21, 1);
\node[anchor=center] (text) at (18.5,0) {\small $\condprobof{\catvariableof{3}}{\catvariableof{2}}$};
\draw[->]  (21,0) -- (23,0);
\draw[fill] (22,0) circle (0.15cm);
\draw[->] (22,0) -- (22,2) node[above] {\tiny ${\catvariableof{3}}$};
\draw[->] (22,0) -- (22,-2);
\draw (19.5,-2) rectangle (24.5,-4); 
\node[anchor=center] (text) at (22,-3) {\small $\condprobof{\randomeof{3}}{\catvariableof{3}}$};
\draw[->] (22,-4) -- (22,-6) node[midway, right]{\tiny ${\randomeof{3}}$};


\node[anchor=center] (text) at (24,0) {$\cdots$};


\end{scope}

\end{tikzpicture} 
\end{center}
\caption{Depiction of a Hidden Markov Model. 
	a) Dependency Graph (of the corresponding chain Graphical Model).
	b) Dual Tensor Network representing the conditional probability factors.}
\label{fig:HMM}
\end{figure}



\subsubsection{Bayesian Networks as Markov Networks}

Markov Networks are more flexible compared with Bayesian Networks, since any Bayesian Network is a Markov Network by ignoring the directionality of the hypergraph and understanding the conditional distributions as generic tensor cores.
In the next theorem we provide the conditions for the interpretation of a Markov Network as a Bayesian Network.

\begin{theorem}\label{the:MarkovToBayesian}
	Let $\tnetof{\graph}$ be a tensor network on a directed acyclic hypergraph, such that the edges are of the structure
		\[ \edges = \{ (\parentsof{\node}, \{\node\}) \, : \, \node\in\nodes \} \]
	and each tensor $\hypercoreof{\edge}$ respects the directionality of the graph, that is each $\hypercoreof{(\parentsof{\node}, \{\node\})}$ is directed with the variables to $\parentsof{\node}$ incoming and $\node$ outgoing.
	Then $\partitionfunctionof{\tnetof{\graph}}=1$ and for each $\node\in\nodes$ we have
		\[ \bnnodecore = \normationofwrt{\tnetof{\graph}}{\catvariableof{\node}}{\catvariableof{\parentsof{\node}}} \, . \]
	In particular, $\tnetof{\graph}$ is a Bayesian Network.
\end{theorem}
\begin{proof}
	We show the claim by induction over the cardinality of $\nodes$.
	
	$\cardof{\nodes}=1$: In this case we find a unique node $\node\in\nodes$ and have $\edges=\{(\varnothing,\{\node\})\}$.
		The tensor $\hypercoreof{(\varnothing,\{\node\})}$ is then normed with no incoming variables and we thus have
			\[ \partitionfunctionof{\tnetof{\graph}} = \contraction{\tnetof{\graph}} = \contraction{\hypercoreof{(\varnothing,\{\node\})}} = 1 \]
		and
			\[ \normationof{\tnetof{\graph}}{\catvariableof{\node}} = \hypercoreof{(\varnothing,\{\node\})} \, .  \]
			
	$\cardof{\nodes}-1 \rightarrow \cardof{\nodes}$: Let there now be a directed hypergraph $\graph=(\nodes,\edges)$ and let us now assume, that the theorem holds for any tensor networks with node cardinality $\cardof{\nodes}-1$.
		Since the hypergraph is acyclic, we find a root $\node\in\nodes$ such that $\node\notin\parentsof{\secnode}$ for $\secnode\in\nodes$.
		We denote $\tnetof{\secgraph}$ the tensor network on the hypergraph $\secgraph=\{\nodes/\{\node\},\edges/\{(\parentsof{\node},\{\node\})\}\}$ with decorations inherited from $\tnetof{\graph}$.
		With Theorem~\ref{the:splittingContractions}, the directionality of $\bnnodecore$ and the induction assumption on $\tnetof{\secgraph}$ we have
		\begin{align*}
			\contraction{\tnetof{\secgraph}\cup\left\{\bnnodecore\right\}}
			 = \contraction{\tnetof{\secgraph}\cup\left\{\contractionof{\bnnodecore}{\catvariableof{\parentsof{\node}}}\right\}}
			 = \contraction{\tnetof{\secgraph}\cup\left\{\onesat{\catvariableof{\parentsof{\node}}}\right\}}
			 = 1
		\end{align*}
		and thus a trivial partition function.
		Since $\node$ does not appear in $\secgraph$, we have for any index $\catindexof{\parentsof{\nodes}}$
		\begin{align*}
			\contractionof{\tnetof{\graph}}{\catvariableof{\node},\indexedcatvariableof{\parentsof{\node}}}
			= \contractionof{\bnnodecore}{\catvariableof{\node},\indexedcatvariableof{\parentsof{\node}}}
			\cdot \contractionof{\tnetof{\secgraph}}{\indexedcatvariableof{\parentsof{\node}}}
		\end{align*}
		and thus, since $\bnnodecore$ is directed, that
		\begin{align*}
			\normationofwrt{\tnetof{\graph}}{\catvariableof{\node}}{\catvariableof{\parentsof{\node}}}
			= \bnnodecore \, .
		\end{align*}
\end{proof}


%\begin{theorem}\label{the:BayesianToMarkov}
%	Any Bayesian Network on a directed graph $\graph=(\nodes,\edges)$ is a Markov Network on a hypergraph $\secgraph=(\nodes,\secedges)$ with identical nodes and hyperedges consistent of  a hyeredge to each node with $\node$ being the only outgoing node and
%		\[  \{\tilde{\node} \, : \, (\tilde{\node},\node) \in \edges\} \,  \]
%	being the incoming nodes.
%	Each hyperedge of the Markov Network is decorated with the conditional probability distribution and the partition function is vanishing.
%\end{theorem}
%\begin{proof}
%	Each conditional probability distribution is associated with the hyperedge constructed to the representative node.
%	The contraction of all conditional probability distributions is the Bayesian Network, which corresponds with the constructed Markov Network due to the trivial partition function.
%\end{proof}

%% Bayesian Network richer
Theorem~\ref{the:MarkovToBayesian} states that Bayesian Networks are a subset of Markov Networks.
While Markov Network allow generic tensor cores, Bayesian Networks impose a local directionality condition on each tensor core by demanding it to be a conditional probability tensor.
In our diagrammatic notation, the local normation of Bayesian Networks is highlighted by the directionality of the hypergraph.
Generic Markov Networks are on undirected hypergraphs, where in general no local directionality condition is assumed.
As a consequence, tasks such as the determination of the partition functions or calculation of conditional distributions involve global contractions.


%% Conditioning
%The representation of Bayesian Networks by Markov Networks is of special interest when representing conditional distributions.
%Bayesian Networks conditioned on evidence are no longer Bayesian Networks on the same graph, but Markov Networks on a hypergraph enriched by the evidence conditioned about.








\subsection{Exponential Families}\label{sec:exponentialFamilies}

% Usage of the selection encoding -> Can also make a theorem out of this
Exponential families are collections of probability distributions, where each coordinate is determined by a base measure and a set $\sstat$ of features as
	\[ \probat{\indexedshortcatvariables}  \propto \basemeasure(\catindex) \cdot \expof{\sum_{\statenumeratorin} \sstatcoordinateofat{\selindex}{\indexedshortcatvariables} \cdot \canparamat{\indexedselvariable}} \, . \]
We use the selection encoding to represent the weighted summation over the statistics, that is the tensor
	\[ \sencsstatat{\shortcatvariables,\selvariable}: \facstates \times [\statorder] \rightarrow \rr \]
with
	\[ \sencsstatat{\indexedshortcatvariables,\indexedselvariable} = \sstatcoordinateofat{\selindex}{\indexedshortcatvariables} \, . \]
We then understand $\canparam$ as a vector to the categorical variable $\selvariable$ and use Theorem~\ref{the:linCompSelEncoding} to get
	\[ \sum_{\statenumeratorin}\canparamat{\indexedselvariable}\cdot \sstatcoordinateofat{\selindex}{\shortcatvariables}
		 = \sbcontractionof{\sencsstatat{\shortcatvariables,\selvariable},\canparamat{\selvariable}}{\shortcatvariables} \, . \]

\begin{definition}
	Given a sufficient statistics 
		\[ \sstat : \facstates \rightarrow \parameterspace\]
%		\[ \sstat : \atomstates \times [\statorder] \rightarrow \rr \]
	and a non-negative base measure
		\[ \basemeasure : \facstates \rightarrow \rr \]
	the set $\expfamily=\{\expdist \, : \, \canparam[\selvariable] \in \simpleparspace\}$ of probability distributions 
		\[ \expdist = \normationof{\expof{\sbcontractionof{\sencodingof{\sstat},\canparam}{\shortcatvariables}}}{\shortcatvariables} \]
	is called the exponential family to $\sstat$.
	If the base measure is positive, we define for each member with parameters $\canparam$
		\[ \expenergy = \sbcontractionof{\sencsstat,\canparam,\lnof{\basemeasure}}{\shortcatvariables} \]
	the associated energy tensor.
	We further introduce the cumulant function
		\[ \cumfunctionof{\canparam} = \lnof{\sbcontraction{\basemeasure,\expof{\sbcontractionof{\sencsstat,\canparam}{\shortcatvariables} }} } \, .\]
\end{definition}


%We have
%	\[ \expenergyat{\indexedcatvariables} = \sum_{\statenumeratorin}\sstatcoordinateof{\statenumerator}(\catindices) \cdot \canparam(\statenumerator) \, . \]

% Diverging partition functions avoided here
Since we here restrict the discussion to finite state spaces, the distribution $\expdist$ is well-defined for any $\canparam\in\rr^{\statorder}$.
For infinite state space there are sufficient statistics and parameters, such that the partition function $\sbcontraction{\basemeasure,\expof{\sbcontractionof{\sencsstat,\canparam}{\shortcatvariables}}}$ diverges and the normation $\expdist$ is not defined.
In that cases, the canonical parameters need to be chosen from a subset where the partition function is finite. 

% Cumulant representation
\begin{lemma}\label{lem:energyCumulantRepresentation}
	For any member of an exponential family $\expfamily$ with positive base measure we have
		\[ \expdist = \expof{ \expenergy - \cumfunctionof{\canparam}\cdot \onesat{\shortcatvariables}} \, . \]
\end{lemma}
\begin{proof}
	By definition we have
	\begin{align*}
		\expdist 
		&= \normationof{
		\expof{\sbcontractionof{\sencsstat,\canparam}{\shortcatvariables}}
		}{\shortcatvariables} \\
		&= \frac{\contractionof{\expof{\sbcontractionof{\sencsstat,\canparam}{\shortcatvariables}}}{\shortcatvariables}
			}{\contraction{\expof{\sbcontractionof{\sencodingof{\sstat},\canparam	}{\shortcatvariables}}}} \\
		&=  \frac{
		\contractionof{\expof{\expenergyat{\shortcatvariables}}}{\shortcatvariables}
		}{
		\expof{\cumfunctionof{\canparam}}
		} \\
		& = \expof{ \expenergyat{\shortcatvariables} - \cumfunctionof{\canparam}\cdot \onesat{\shortcatvariables}} \, . 
	\end{align*}
\end{proof}


\subsubsection{Tensor Network Representation} 

We can use the relational encoding formalism to represent members of exponential families by a single contraction, as we show next.
The central insight here is a relational encoding of the sufficient statistics, which enables representation by tensor network decomposition, when the sufficient statistic is decomposable.

\begin{theorem}[Generic Representation of Exponential Families]\label{def:expFamilyTensorRep}
	Given any base measure $\basemeasure$ sufficient statistic $\sstat$ we enumerate for each coordindate $\selindexin$ the image $\imageof{\sstatcoordinateof{\selindex}}$ by a variable $\sstatcatof{\selindex}$ taking values in $[\cardof{\imageof{\sstatcoordinateof{\statenumerator}}}]$, given an interpretation map
		\[ \indexinterpretationof{\selindex} : 
		[\cardof{\imageof{\sstatcoordinateof{\statenumerator}}}] \rightarrow \imageof{\sstatcoordinateof{\statenumerator}} \, . \]
	
	For any parameter vector $\canparamat{\selvariable}:[\seldim]\rightarrow\rr$ we build the activation cores
		\[ \headcoreofat{\statenumerator}{\sstatcatof{\selindex}=\sstatindof{\selindex}} 
		= \expof{\canparamat{\indexedselvariable} \cdot \indexinterpretationofat{\selindex}{\sstatcatof{\selindex}} } \,   \]
	and have
		\[ \expdist = 
		\normationof{\{\basemeasure,\rencodingof{\sstat}\}\cup\{\headcoreof{\statenumerator} \, : \, \statenumeratorin\}}{\shortcatvariables} \, . 
		\]
%	where we use the vectors $\headcoreof{\statenumerator} : \imageof{\sstatcoordinateof{\statenumerator}} \rightarrow \rr $ defined for $y \in \imageof{\sstatcoordinateof{\statenumerator}}$ by
\end{theorem}
\begin{proof}
	We use an extended image of $\sstat$ by  %	which does not modify the statement of Theorem~\ref{the:tensorFunctionComposition} (since extension to cases, which are never met).
		\[ \imageof{\sstat} = \bigtimes_{\statenumeratorin} \imageof{\sstatcoordinateof{\selindex}} \, . \]
	Theorem~\ref{the:tensorFunctionComposition} implies
		\[ \expof{\contractionof{\{\sstat, \canparam\}}{X}}
		= \contractionof{\{\rencodingof{\sstat}, \restrictionofto{\expof{\braket{\cdot, \weight}}}{\imageof{\sstat}} \}}{X} \, . \]
	The claim follows from the equation
		\[ \restrictionofto{\expof{\braket{\cdot, \canparam}}}{\imageof{\sstat}} 
		= \bigotimes_{\selindexin} \restrictionofto{\expof{\cdot \canparamat{\indexedselvariable}}}{\imageof{\sstatcoordinateof{\selindex}}}  
		= \bigotimes_{\selindexin} \headcoreof{\selindex} \, . \]
\end{proof}


We notice, that the relational encoding is the contraction of the relational encoding of its coordinate maps as 
	\[ \rencodingofat{\sstat}{\shortcatvariables,\sstatcatof{[\seldim]}} = \contractionof{\rencodingof{\sstatcoordinateof{0}},\ldots,\rencodingof{\sstatcoordinateof{\seldim-1}}}{\shortcatvariables,\sstatcatof{[\seldim]}} \, .  \]
We will show this property in Theorem~\ref{the:functionDecompositionBasisCP}.
One strategy to create $\rencodingof{\sstat}$ is thus the creation of the encoding of all its coordinate maps.
When the coordinate maps are sharing common components, a sparser representation can be derived through encodings of the components shared among the coordinate map encodings.


% Core types
A tensor network representation of an exponential family is thus a Markov Network consistent of two types of cores.
Computation cores are relational encodings of statistics $\rencodingof{\sstatcoordinateof{\selindex}}$.
Our intuition is that they compute the hidden variable $\catvariableof{\sstatcoordinateof{\selindex}}$, based on Basis Calculus (see Chapter~\ref{cha:basisCalculus}).
Activation cores $\headcoreof{\selindex}$ exploit the computed variable and provide, when contracted with the relational encoding, a factor 
	\[ \sbcontractionof{\rencodingof{\sstatcoordinateof{\selindex}}, \headcoreofat{\statenumerator}{\catvariableof{\statenumerator}}}{\shortcatvariables}  \]
to the Markov Network reduced to the visible coordinates $\shortcatvariables$.
The activation cores are trivial, i.e. $\headcoreofat{\selindex}{\sstatcatof{\selindex}}=\onesat{\sstatcatof{\selindex}}$, when $\canparamat{\selvariable=\selindex}=0$.
In that case 
	\[  \sbcontractionof{\rencodingof{\sstatcoordinateof{\selindex}}, \headcoreofat{\statenumerator}{\catvariableof{\statenumerator}}}{\shortcatvariables} 
	= \onesat{\shortcatvariables} \]
and both the activation core and the corresponding computation core can be dropped from the network without changing its distribution.

%% FALSE STATEMENT? 
%We can sum multiples of the trivial tensor on the head cores without changing the distribution as we show next.
%
%\begin{theorem}
%	For any $\statenumeratorin$, the distribution is invariant under replacing $\headcoreofat{\statenumerator}{\selvariableof{\statenumerator}}$ by $\headcoreofat{\statenumerator}{\catvariableof{\statenumerator}}+\lambda\cdot \onesat{\catvariableof{\statenumerator}}$ where $\lambda\in\rr$
%\end{theorem}
%\begin{proof}
%	Follows from linearity in each head core, trivialization by trivial heads and normation.
%	
%	By linearity we have
%	\begin{align*}
%		\sbcontractionof{\rencodingof{\sstatcoordinateof{\selindex}}, (\headcoreofat{\statenumerator}{\catvariableof{\statenumerator}}+\lambda\cdot \onesat{\catvariableof{\statenumerator}})}{\shortcatvariables}
%		= 
%		\sbcontractionof{\rencodingof{\sstatcoordinateof{\selindex}}, \headcoreofat{\statenumerator}{\catvariableof{\statenumerator}}}{\shortcatvariables}
%		+\lambda\cdot  \sbcontractionof{\rencodingof{\sstatcoordinateof{\selindex}}, \onesat{\catvariableof{\statenumerator}}}{\shortcatvariables}
%		=  \sbcontractionof{\rencodingof{\sstatcoordinateof{\selindex}}, \headcoreofat{\statenumerator}{\catvariableof{\statenumerator}}}{\shortcatvariables}
%		+ \lambda \cdot \onesat{\shortcatvariables} \, .
%	\end{align*}
%\end{proof}


\begin{remark}[Comparison of relation and selection encodings]
	% Relation vs Selection encoding
	Relation encodings are in general of higher dimensions than selection encodings.
	In can thus be intractible to instantiate the probability distribution as a tensor networks, while the energy tensor can still be efficiently represented based on selection encodings.
	\red{In this case, energy-based reasoning algorithms are tractible while more direct methods are intractible.}
\end{remark}





\subsubsection{Mean Parameters}

Mean parameters are an alternative way to represent members of exponential families.

\begin{definition}\label{def:meanForwardBackward}
	Let there be an exponential family defined by $\sstat$.
	We call the tensor
		\[ \meanparam = \sbcontractionof{\expdist,\sencodingof{\sstat}}{\selvariable} \]
	the mean parameter tensor to a member $\expdist$ of an exponential family.
	The set 
		\[ \meansetof{\sstat} = \{\contractionof{\probtensor,\sencodingof{\sstat}}{\selvariable} \, : \, \probtensor\in\probtensorset \} \, , \]
	where $\probtensorset$ denotes the set of all probability distributions,
	is called the convex polytope of realizable mean parameters.
	The map
		\[ \forwardmap :  \simpleparspace\rightarrow\simpleparspace\]
	with $\forwardmapof{\canparam} = \sbcontractionof{\expdist,\sencodingof{\sstat}}{\selvariable}$ is called the forward map of the exponential family and any map
		\[ \backwardmap : \imageof{\forwardmap} \rightarrow \simpleparspace\]
	with $\expdistof{(\sstat,\backwardmapof{\forwardmapof{\canparam}},\basemeasure)} = \expdist$ for any $\canparam\in\rr^{\statorder}$ a backward map.
\end{definition}


% Polytope
The image $\imageof{\forwardmap}$ of the forward map is the interior of the convex polytope $\meansetof{\sstat}$.
It contains any mean parameters $\sbcontractionof{\probtensor,\sstat}{\selvariable}$ realizable by any probability distribution $\probtensor$ \cite{wainwright_graphical_2008}.

\red{Add convex hull interpretation!}

\subsubsection{Examples}

% Naive Exponential Family
\begin{example}[The naive exponential family]\label{exa:naiveExpFamily}
	When taking as sufficient statistic the identity $\identityat{\shortcatvariables,\selvariable}$, we can represent any positive distribution $\probtensor$ as a member of the exponential family, namely when choosing the canonical parameter
		\[ \canparam = \lnof{\probtensor} \, . \]
	The associated mean parameter is then
		\[ \meanparam = \probtensor \, . \]
\end{example}


% Markov Networks
Given a hypergraph with fixed node decoration, the different decorations of the hyperedges by tensors can be represented by an exponential family, as we show next.

\begin{theorem}[Exponential Representation of Markov Networks]
	For any hypergraph $\graph=(\nodes,\edges)$ we define a sufficient statistics 
		\[ \sstat = \bigtimes_{\edge\in\edges}  \sstatcoordinateof{\edge} \]
	where 
		\[ \sstatcoordinateof{\edge}(\catindexof{\nodes}) = \catindexof{\edge} \, . \]
	Given any Markov Network $\{\hypercoreof{\edge}\}$ on $\graph$ with positive tensors $\hypercoreof{\edge}$ we define
		\[ \canparam = \bigtimes_{\edge\in\edges} \canparam_{\edge} \]
	where
		\[ \canparam_{\edge} =  \ln\left[ \hypercoreof{\edge} \right] \]
	and $\ln$ acts coordinatewise.
	Then, the Markov Network is in the member of the exponential family with trivial base measure, sufficient statistic $\sstat$ and parameters $\canparam$.
\end{theorem}
\begin{proof}
	We have for any $\catindexof{\nodes}$
	\begin{align}
	\prod_{\edge\in\edges} \hypercoreofat{\edge}{\indexedcatvariableof{\edge}}
		= \expof{\sum_{\edge\in\edges} \canparamwrtat{\edge}{\indexedcatvariableof{\edge}}}
		= \expof{\sum_{\edge\in\edges} \sbcontraction{\canparamwrtat{\edge}{\catvariableof{\edge}},\sstatcoordinateof{\edge}(\catindexof{\nodes})}}  \, .
	\end{align}
	Using that
		\[ \contractionof{\sstat,\canparam}{\nodevariables} = \sum_{\edge\in\edges} \contractionof{\sstatcoordinateof{\edge},\canparam_{\edge}}{\nodevariables} \]
	we get
	\begin{align}
		\contractionof{\{\hypercoreof{\edge}: \edge\in\edges\}}{\nodevariables} = \expof{\contractionof{\canparam,\sstat}{\nodevariables}} \, .
	\end{align}
	This implies 
	\begin{align}
		\normationof{\{\hypercoreof{\edge}: \edge\in\edges\}}{\nodevariables} = \normationof{\expof{\contractionof{\canparam,\sstat}{\nodevariables}}}{\nodevariables} \, .
	\end{align}
\end{proof}


% Mean parameters
The mean parameter of the Markov Network exponential family is the cartesian product of the marginals $\meanparam_\edge[\catvariableof{\edge}]$ are often refered to as beliefs in the literature.



\subsection{Empirical Distributions}\label{sec:empDistribution}


%The statistic of observed worlds is stored in the data tensor $\datacore$, carrying again indices to each atom.
%It is the probability tensor of the empirical distribution $\probtensorof{\datacore}$. 
%Each coordinate is thus the ratio of the observation of the world with the by the indices specified world.
%
%
%\subsubsection{Representing a single sample}
%
%Single samples are states of the factored systems, which are indexed by $(\catindices)$ and represented by one-hot encodings $\onehotmapof{\catindices}$.
%
%%% Inductive vs Deductive perspective
%Each evidence is a probability distribution
%\begin{itemize}
%	\item Inductive Reasoning: When we interpret evidence as a datapoint, they are typically a basis tensor specifying precisely a world.
%	\item Deductive Reasoning: Evidence is a partial observation of the world, typically basis vectors at each variable, but leaving some unspecified ($\ones$).
%	We then interpret the evidence as being a uniform distribution over the worlds not contradicting with the evidence.
%\end{itemize}
%
%
%\subsubsection{Construction from a list of samples}

\begin{definition}\label{def:dataMap}
	Given a dataset $\{(\catindicesof{\dataindex}) \, : \, \dataindexin \}$ of samples of the factored system we define the sample selector map
		\[ \datamap : [\datanum] \rightarrow \facstates \]
	elementwise by 
		\[ \datamap(\dataindex) = (\catindicesof{\dataindex}) \, . \]
	Its relational encoding (see Definition~\ref{def:functionRepresentation}) is the tensor
		\[ \rencodingofat{\datamap}{\datvariable,\shortcatvariables} = \sum_{\dataindexin} \onehotmapofat{\dataindex}{\datvariable} \otimes \onehotmapofat{\catindicesof{\dataindex}}{\shortcatvariables} \, , \]
	which we call a data tensor.
\end{definition}

%% Basic CP Decomposition of the Data Tensor
The Data Tensor is a conditional probability tensor and has a network decomposition depicted in Figure~\ref{fig:DataDecomposition}.
The cores $\datacoreof{\atomenumerator}$ are matrices storing the value of the categorical variable $\catvariableof{\atomenumerator}$ in the sample world indexed by $\dataindex$.
Whereas the one-hot encoding of single samples is a basis tensor (and therefore a basis elementary decomposition), the data tensor has a basis CP decomposition, see Section~\ref{sec:basisCP}).
This follows from Theorem~\ref{the:functionDecompositionBasisCP}, using the coordinate maps of $\datamap$ by
	\[ \datamap_{\atomenumerator} : [\datanum] \rightarrow [\catdimof{\atomenumerator}] \]
defined by
	\[  \datamap_{\atomenumerator}(\dataindex) = \catindexof{\atomenumerator}^\dataindex \, .  \]

\begin{figure}[h]
\begin{center}
	\input{PartI/tikz_pics/probability_decomposition/data_decomposition.tex}
\end{center}
\caption{
	Representation of a dataset.
	a) As a Bayesian network.
	b) Decomposition of the data tensor into a tensor network in the $\cpformat$ Format.
	Without the contraction with the dashed $\frac{1}{\datanum}\ones$ core, the datacore encodes the distribution conditioned on a datapoint. }
\label{fig:DataDecomposition}
\end{figure}


%% Conditional Probability interpretation
The Data Tensor a conditional probability tensor, which retrieves the respective sample distribution when selecting a sample.
We can represent this probability distribution by a random variable $\catvariableof{\dataindex}$ selecting a specific datapoint on which the atoms depend.

%% Empirical Distribution
We define the empirical distribution by the normation of the relational encoding of the datamap as
\begin{align*}
	\empdistribution 
	\coloneqq \sbnormationof{\datacore}{\shortcatvariables} \, . 
\end{align*}
Each coordinate of the empirical distribution is the frequency of the by the index specified word in the data.

\begin{theorem}\label{the:empCPRep}
	Given a data map $\datamap$ we have
	\begin{align*}
		\rencodingofat{\datamap}{\datvariable,\shortcatvariables}  
		= \contractionof{
		\{\rencodingofat{\datamap^{\atomenumerator}}{\datvariable,\catvariableof{\atomenumerator}} : \atomenumeratorin \} 
		}{\datvariable,\shortcatvariables} 
	\end{align*}
	and
	\begin{align*}
	\empdistribution = \sbcontractionof{\datacore, \frac{1}{\datanum}\ones}{\shortcatvariables} 
	= \sbcontractionof{\datacoreof{0},\ldots,\datacoreof{\atomorder-1}, \frac{1}{\datanum}\onesat{\datvariable}}{\shortcatvariables} \, . 
	\end{align*}
\end{theorem}
\begin{proof}
	The first claim is a special case of Theorem~\ref{the:functionDecompositionBasisCP}, to be shown in Chapter~\ref{cha:tensorEncodings}.
	To show the second claim we notice
		\[ \sbcontraction{\datacore} = \sum_{\datindexin} \sbcontraction{\rencodingofat{\datamap}{\datvariable=\datindex,\shortcatvariables}} = \datanum \,  . \]
	With the first claim it follows that
	\begin{align*}
		\empdistribution 
		 = \sbnormationof{\datacore}{\shortcatvariables}
		 = \frac{\sbcontractionof{\datacore}{\shortcatvariables}}{\sbcontraction{\datacore}} 
		 =  \contractionof{
		\{\rencodingofat{\datamap^{\atomenumerator}}{\datvariable,\catvariableof{\atomenumerator}} : \atomenumeratorin \} \cup \{ \frac{1}{\datanum} \onesat{\datvariable} \}
		}{\datvariable,\shortcatvariables}  \, . 
	\end{align*}
\end{proof}

%The normation can be represented by averaging the data index $\dataindex$ and we have
%\begin{align*}
%	\empdistribution = \contractionof{\{\datacore, \frac{1}{\datanum}\ones \}}{\shortcatvariables} 
%	= \contractionof{\{\datacoreof{0},\ldots,\datacoreof{\atomorder-1}, \frac{1}{\datanum}\ones \}}{\shortcatvariables} \, . 
%\end{align*}
In a contraction diagram we represent the empirical distribution by
\begin{center}
	\input{PartI/tikz_pics/probability_decomposition/empirical_distribution.tex}
\end{center}

%% Perspective of forwarding the uniform distribution of the samples
In another perspective, we can understand $\frac{1}{\datanum}\ones$ as the uniform probability distribution over the samples, which is by the map $\datamap$ forwarded to a distribution over $\facstates$.
%By Theorem~\ref{the:conditionalAverage} also the empirical distribution tensor $\empdistribution$ is a probability tensor.






\section{Discussion and Outlook}

\begin{remark}[Alternative definitions of graphical models]
	In the literature, tensor networks are often called dual to the hypergraphs defining graphical models (see e.g. \cite{robeva_duality_2019}).
	The duality becomes clear, when one interpretes the tensors as cores and their common variables as edges.
	We in this work avoid this ambiguity by directly defining tensor networks as decoration of hyperedges by tensors.
	
	Often, the tensors decorating hyperedges are called factors and their logarithm features \cite{koller_probabilistic_2009}.
	
	Further, we directly use hypergraphs instead of the more canonical association of factors with cliques of a graph.
	This avoids the discussion of non-maximal cliques as decorated with trivial tensors.
	Such hypergraphs follow the same line of though compared with factor graphs, which are bipartite graphs with nodes either corresponding with single variables or with a collection of them affected by a factor.
\end{remark}








\section{Probabilistic Reasoning}\label{cha:probReasoning} 

We have investigated means to store the knowledge about a system and now turn to the retrieval of information, a process called inference.


\begin{remark}[Interpretation of Contractions in Probabilistic Reasoning]
	Contraction compute marginal distributions, where the marginal variables are on the open legs of the contraction.
\end{remark}

% 
Contraction of the relational encoding of a function with a Markov Network gives the statistics over the values of the functions.
When contracting the function directly, we get the expectation.

% Message passing
%Another approximation comes from an approximation of the contractions itself. 
One can increase the efficiency of inference algorithms by using approximative contractions.
Here, message passing schemes can be applied as to be introduced in Chapter~\ref{cha:localContractions}.

\subsection{Queries}

Let us now formalize queries by contractions.

\subsubsection{Querying by functions}

We can formalize queries by retrieving expectations of functions given a distribution specified by probability tensors. 

\begin{definition}\label{def:queries}
	The query of a probability distribution $\probof{\shortcatvariables}$ by a tensor 
		\[ \exfunction : \facstates \rightarrow \rr \]
	is the vector $\probof{\catvariableof{\exfunction}} \in \rr^{\cardof{\imageof{\exfunction}}}$ defined as the contraction
	\begin{align*}
		\probof{\catvariableof{\exfunction}} = \contractionof{\probof{\shortcatvariables},\rencodingofat{\exfunction}{\shortcatvariables,\catvariableof{\exfunction}}}{\catvariableof{\exfunction}} \, . 
	\end{align*}
	Given another tensor $\secexfunction: \facstates \rightarrow \rr $ the conditional query of the probability distribution $\probof{\shortcatvariables}$ by the tensor $\exfunction$ conditioned on the tensor $\secexfunction$ is the matrix $\condprobof{\catvariableof{\exfunction}}{\catvariableof{\secexfunction}}\in\rr^{\cardof{\imageof{\exfunction}}}\otimes \rr^{\cardof{\imageof{\secexfunction}}}$ defined as the normation
	\begin{align*}
		\condprobof{\catvariableof{\exfunction}}{\catvariableof{\secexfunction}} 
		= \normationofwrt{\{
		\probof{\shortcatvariables},\rencodingofat{\exfunction}{\shortcatvariables,\catvariableof{\exfunction}},\rencodingofat{\secexfunction}{\shortcatvariables,\catvariableof{\secexfunction}}
		\}}{
		\catvariableof{\exfunction}}{\catvariableof{\secexfunction}
		} \, . 
	\end{align*}
\end{definition}

%% Conditional Probabilities and conditional queries
Conditional probabilities are queries, where the tensors $\exfunction$ and $\secexfunction$ are identity mappings in the respective variable state spaces.
Conversely, we can understand the conditional query $\condprobof{\exfunction}{\secexfunction}$ as the conditional probability of $\exfunction$ conditioned on $\secexfunction$, of the underlying Markov Network with cores $\{\probtensor, \rencodingof{\exfunction}, \rencodingof{\secexfunction} \}$ and variables $\catvariableof{\exfunction},\catvariableof{\secexfunction}$ besides the variables distributed by $\probtensor$.

%% Expectations as event queries -> Consistency with $\probof{X=i}$?
We further denote event queries by
	\[  \expectationof{\exfunction=z} = \sbcontractionof{\probtensor,\rencodingof{\exfunction},\onehotmapof{z}}{\varnothing} \]
where by $\onehotmapof{z}$ be denote the one hot encoding of the state $z$ with respect to some enumeration.
Let us note that they are further contraction of the queries in Definition~\ref{def:queries} since by Theorem~\ref{the:splittingContractions}
\begin{align*}
	 \expectationof{\exfunction=z} 
	& =  \sbcontractionof{ \sbcontractionof{\probtensor,\rencodingof{\exfunction}}{\catvariableof{\exfunction}} ,\onehotmapof{z}}{\varnothing} \\
	& =  \sbcontractionof{ \probof{\exfunction} ,\onehotmapof{z}}{\varnothing} \, . 
\end{align*}

%% OLD: Defining queries by 
%\begin{definition}
%	The expectation of functions $\exfunction$ given a probability tensor is the contraction
%		\[ \expectationofwrt{\exfunction(\catvariables)}{\catvariables\sim\probtensor} = 
%			\contractionof{\{\probtensor,\rencodingof{\exfunction}\}}{\{\exfunctiontargetvariables \}} \, . 
%		\]
%\end{definition}
%This is the canonical definition of expectations, since summing function values weighted by the probability of the argument.
%When we have an unnormalized probability distribution $\phi$ the expectation is the quotient
%\begin{align*}
%	\expectationofwrt{\exfunction(\catvariables)}{\catvariables\sim\phi}  = \frac{
%		\contractionof{\{\phi,\ftensorof{\exfunction}\}}{\{\exfunctiontargetvariables \}} 
%	}{
%		\contractionof{\{\phi\}}{\varnothing} 
%	} \, . 
%\end{align*}

%\subsubsection{Conditional Probability Queries}
%
%Typical queries are the computation of an a posteriori distribution given evidence.
%This is just the contraction.
%
%%% As expectation
%The query consists of the one-hot encoding of the evidence and Ids elsewhere.
%The result is then interpreted as another probability distribution, defined as a Markov network and the possible need to normalize with the partition function.
%
%Given evidence, condition the probability tensor on that evidence.




\subsubsection{MAP Queries}

Find the maximal variable of the (conditioned) probability tensor.

Given a probability tensor the search for its maximal coordinate is 
\begin{align}
	\argmax_{\catindices} \probof{\indexedcatvariables} \, .
\end{align}

Often, the generation of a full (conditioned) probability tensor can be infeasible, if too many variables are queries.
Having a tensor network decomposition of the probability tensor avoids this generation.

Along this we reformulate the maximal coordinate problem as
\begin{align}
	\argmax_{\catindices} \probof{\indexedcatvariables} 
	= \argmax_{\catindices} \contractionof{\{\probtensor, \onehotmapof{\catindices}\}}{\varnothing}  \, .
\end{align}

We understand it as a Tensor Network approximation problem, where the approximating tensor are the one-hot encodings of states.


\red{MAP queries are approximated by sampling from annealed distributions.}

\begin{remark}\label{rem:simulatedAnnealing}
% Simulated annealing
	\red{Here by the naive exponential family}
	Simulated annealing manipulates the probability used to sample $\atomlegindexof{\atomenumerator}$ in terms of an inverse temperature parameter $\invtemp$, by
		\[ \probtensor \rightarrow \frac{\expof{\invtemp\cdot\lnof{\probtensor}}}{\contractionof{\expof{\invtemp\cdot\lnof{\probtensor}}}{\varnothing} } \, . \]
	When the temperature is larger than $1$, the probability of states with low probability increases while the probability of states with large probability decreases and for low temperatures the opposite.
	Simulated annealing, that is the decrease of the temperature to $0$ during Gibbs sampling biases the algorithm towards states with large probability.
%	Tuning this parameter can improve the convergence of Gibbs Sampling.

	% On exponential families
	For any exponential family the transformation 
		\[ \energytensor \rightarrow \invtemp \cdot \energytensor  \]
	can be performed by rescaling the canonical parameters as
		\[ \canparam \rightarrow \invtemp \cdot \canparam \, . \]
\end{remark}





\subsection{Sampling}

Let us here investigate how to draw samples from distributions $\probtensor$.

%Need to generate the full conditional probability distribution by contraction and then sample from it.
Since there are $\prod_{\node\in\nodes}\catdimof{\node}$ coordinates stored in $\probtensor$, naive methods are often infeasible.
One can instead exploit a representation of $\probtensor$ by a Markov network or the energy term in an exponential family for efficient algorithms.
%\begin{itemize}
%	\item Empirical Distribution: Use the basis CP Decomposition and sample by choice of a random slice
%	\item Bayesian Network: Starting from the leaves sample 
%\end{itemize}

\subsubsection{Exact Methods}

Forward Sampling (see Algorithm~\ref{alg:ForwardSampling}) uses a chain decomposition (see Theorem~\ref{the:chainRule}) of a probability distribution to iteratively sample the variables.

\begin{algorithm}[hbt!]
\caption{Forward Sampling}\label{alg:ForwardSampling}
\begin{algorithmic}
\For{$\atomenumeratorin$}
	\State Draw $\atomlegindexof{\atomenumerator}\in[\catdimof{\atomenumerator}]$ from conditional distribution 
		\[ \condprobof{\catvariableof{\atomenumerator}}{\{\catvariableof{\secatomenumerator} = \atomlegindexof{\secatomenumerator} \, : \, \secatomenumerator < \atomenumerator \}} \]
%		\onehotmapof{\atomlegindexof{\secatomenumerator}} \, : \secatomenumerator < \atomenumerator\}}  \] 
\EndFor
\end{algorithmic}
\end{algorithm}





%% Comment on rejection Sampling - required?
% Conditioned Sampling
When sampling from conditional probability distributions, one can sample from the conditioned distribution instead.
However, the conditioning changes the structure of the distribution, and conditioned Bayesian Networks are not Bayesian Networks on the same graph.
One ways around is rejection sampling, where one samples from the unconditioned distribution and rejects samples not satisfying the event conditioned on.
When the event conditioned on is of small probability, methods like rejection sampling will come with large runtimes.

\subsubsection{Approximate Methods}

% Problem of many variables
When there are many variables to be sample, the computation of the conditional probability to all variables can be infeasible.
One way to overcome this is Gibbs Sampling: Iteratively resemble single variables given the rest as evidence.

%\subsubsection{Gibbs Sampling}

% Still old: Sample from Marginal
Sample each variable independent from the marginal distribution.
Then, alternate through the variables and sample each variable from the conditional distribution taking the others as evidence.

\begin{algorithm}[hbt!]
\caption{Gibbs Sampling}\label{alg:Gibbs}
\begin{algorithmic}
\For{$\atomenumeratorin$}
	\State Draw State for atom $\atomenumerator$ from initialization distributions. % In implementation: Initialize with ones and draw -> Avoids zero probability state
\EndFor
\While{Stopping criterion is not met}
\For{$\atomenumeratorin$}
	\State Draw $\atomlegindexof{\atomenumerator}\in[\catdimof{\atomenumerator}]$ from 
		\[ \condprobof{\catvariableof{\atomenumerator}}{\{\catvariableof{\secatomenumerator}=\atomlegindexof{\secatomenumerator} \, : \secatomenumerator \neq \atomenumerator\}} \]
		%= 
		%\frac{
		%\contractionof{\{\probtensor\} \cup \{\onehotmapof{\atomlegindexof{\secatomenumerator}} \, : \secatomenumerator \neq \atomenumerator \}}{\catvariableof{\atomenumerator}}
		%}{
		%\contractionof{\{\probtensor\} \cup \{\onehotmapof{\atomlegindexof{\secatomenumerator}} \, : \secatomenumerator \neq \atomenumerator \}}{\varnothing}
		%} \]
\EndFor
\EndWhile
\end{algorithmic}
\end{algorithm}


% Energy
Let us now interpret a probability tensor at hand as a member of an exponential family (see Section~\ref{sec:exponentialFamilies}), which is always possible when taking the naive exponential family.
Gibbs sampling can be implemented based on the energy tensor $\energytensor$ of the probability tensor, using
	\[ \condprobof{\catvariableof{\atomenumerator}}{\{\catvariableof{\secatomenumerator}=\atomlegindexof{\secatomenumerator} \, : \secatomenumerator \neq \atomenumerator\}} = \normationofwrt{\expof{\contractionof{\{\energytensor\}\cup\{\onehotmapof{\atomlegindexof{\secatomenumerator}} \, : \secatomenumerator \neq \atomenumerator \}}{\catvariableof{\atomenumerator}}}}{\catvariableof{\atomenumerator}}{\varnothing}  \, .\]
Thus, it suffices to build the selection encoding of the statistics, and we can avoid the usage of the relational encoding.
\red{This is in contrast with forward sampling, where we need to sum over many coordinates of the exponentiated energy tensor, which amounts to the representation of the probability distribution as a tensor network using relational encodings.}
%where the operation with energy tensors and selection encodings is not efficient.}










\subsection{Maximum Likelihood Estimation} % Stuff from Parameter Estimation - Problem that Part I is called inference?

Let us now turn to inductive reasoning tasks, where a probabilistic model is trained on given data.

\subsubsection{Likelihood and Cross Entropy}

Given a datapoint $\datapointof{\dataindex}$ consisting of the images of the data selecting map $\datamap$ (see Definition~\ref{def:dataMap}), the likelihood given a Markov Logic Network is denoted as
	\[ \probof{\shortcatvariables = \datamapof{\dataindex}} \, . \]
	
% Independent assumption
When all $\datamapof{\dataindex}$ are drawn independently from $\mlnprobat{\shortcatvariablelist}$, we can factorize into
	\[ \probof{\data}  = \prod_{\dataindexin} \probof{\shortcatvariables=\datamapof{\dataindex}} \, . \]

% Logarithm
It is convenient to apply a logarithm on the objective, which does not influence the optimum.
This is especially useful, when investigating the convergence of the objective for $\datanum\rightarrow\infty$ (see Chapter~\ref{cha:mlnConcentration}).
We define the loss of a distribution $\probtensor$ as
\begin{align}\label{eq:defLikelihoodLossPL}
	\lossof{\probtensor} 
	= \frac{1}{\datanum} \lnof{\probof{\data}} 
\end{align}

\begin{lemma}
	The loss has the form 
	\begin{align}
		\lossof{\probtensor} = \contractionof{\{\empdistribution,\lnof{\probtensor}\}}{\varnothing}
	\end{align}
\end{lemma}
\begin{proof}
	We have
	\begin{align*}
		\lossof{\probtensor} 
		& = \frac{1}{\datanum} \lnof{\probof{\data}} 
		= \frac{1}{\datanum} \sum_{\datain} \lnof{\probof{\shortcatvariables =\datamap(\dataindex)}} 
		= \frac{1}{\datanum} \sum_{\datain} \contractionof{\{\lnof{\probtensor},\onehotmapof{\datamap(\dataindex)}\}}{\varnothing} \\ 
		& = \contractionof{\{\empdistribution,\lnof{\probtensor}\}}{\varnothing} \, .
	\end{align*}
\end{proof}

We thus state the Maximum Likelihood Problem in the form
\begin{align}\tag{$\mathrm{P}_{\lossof{}, \empdistribution}$}\label{prob:parameterMaxLikelihood}
	\argmin_{\probtensor\in\Gamma} \lossof{\probtensor} \, . % Naive Exponential Family perspective!
\end{align}



% M-Projection -> A projection since P^2 = P, i.e. P applied on the image is id
Most general, the Maximum Likelihood Problem is the M-Projection of a distribution $\gendistribution$ onto a set $\Gamma$ of probability tensors is
\begin{align}
	\argmax_{\probtensor\in\Gamma}  \centropyof{\gendistribution}{\probtensor} 
\end{align}
where the Maximum Likelihood Estimation is the special case $\gendistribution=\empdistribution$.




%% From Probability Representation
\subsubsection{Entropic Interpretation}

\begin{definition}[Shannon entropy]
	The information content or the Shannon entropy of a distribution is defined as
		\[ \sentropyof{\probtensor} := \expectationofwrt{-\lnof{\probof{\randomx}}}{\randomx\sim\probtensor} = - \sum_{\catindices} \probof{\indexedcatvariables} \cdot \lnof{\probof{\indexedcatvariables}} \, . \]
	We depict this in a tensor network diagram with an ellipsis denoting a coordinatewise transform (here the $\ln$) as:
	\begin{center}
		\begin{tikzpicture}[scale=0.3,thick] % , baseline = -3.5pt

\node[anchor=center] (text) at (-8,-5) {\small $\sentropyof{\probtensor}$};

\node[anchor=center] (text) at (-5,-5) {\small ${=}$};

\node[anchor=center] (text) at (-3,-2) {\small $\mathrm{ln}$};
\draw (2,-2) ellipse (6 and 2.75);

\draw (-1,-1) rectangle (5,-3);
\node[anchor=center] (text) at (2,-2) {\small $\probtensor$};
\draw (-1,-7) rectangle (5,-9);
\node[anchor=center] (text) at (2,-8) {\small $\probtensor$};
\draw (0,-5)--(0,-3); 
\draw (0,-5)--(0,-7) node[midway,left] {\tiny $\atomlegindexof{1}$}; 
\draw (1.5,-5)--(1.5,-3); 
\draw (1.5,-5)--(1.5,-7) node[midway,left] {\tiny $\atomlegindexof{2}$}; 
\node[anchor=center] (text) at (3,-4) {$\cdots$};
\draw (4,-5)--(4,-3);
\node[anchor=center] (text) at (3,-6) {$\cdots$};
\draw (4,-5)--(4,-7) node[midway,right] {\tiny $\atomlegindexof{\atomorder}$}; 

%\drawatomcore{3.5}{-8}{$\probtensor$}
%\drawatomindices{3.5}{-12}	
%\draw (5.5,-9)--(5.5,-7) node[midway,right] {\tiny $\atomlegindexof{\exformula}$};

\end{tikzpicture}
	\end{center}
\end{definition}

%We can use the Slicing theorem to ease the computation of the Shannon entropy.
%\red{Can be generalized to any Tensor being a sum of binary tensors. Thus topic better in binary tensor calculus?}

\begin{definition}[Cross entropy]\label{def:crossEntropy}
	The cross entropy between two distributions is defined as 
		\[ \centropyof{\probtensor}{\tilde{\probtensor}} 
		=  \expectationofwrt{-\lnof{\probof{\randomx}}}{\randomx\sim\probtensor} 
		= - \sum_{\catindices}  \probof{\indexedcatvariables} \cdot \lnof{\tilde{\probtensor}[\indexedcatvariables]}  \, . \]
	We depict this in a tensor network diagram with an ellipsis denoting a coordinatewise transform (here the $\ln$) as :
	\begin{center}
		\input{PartI/tikz_pics/cross_entropy.tex}
	\end{center}
\end{definition}


Often we have the situation that some entries of $\probtensor$ will be zero and we can thus reduce the contraction of any coordinatewise tensor to these nonzero entries. 

% KL Divergence
The Gibbs inequality states that
		\[ \centropyof{\probtensor}{\tilde{\probtensor}} \geq \sentropyof{\probtensor} \, . \]
The difference between both sides is called the Kullback Leibler Divergence and a useful metric in reasoning, since it vanishes for $\probtensor=\tilde{\probtensor}$.

\begin{definition}[Kullback Leibler Divergence]\label{def:KLDivergence}
	The KL divergence between two distributions is defined as 
		\[ \kldivof{\probtensor}{\tilde{\probtensor}} = \centropyof{\probtensor}{\tilde{\probtensor}} - \sentropyof{\probtensor}  \, . \]
%	We depict this in a tensor network diagram with an ellipsis denoting a coordinatewise transform (here the $\ln$) as :
%	\begin{center}
%		\input{PartI/tikz_pics/cross_entropy.tex}
%	\end{center}
\end{definition}



% Interpretation of MLE as Cross-Entropy Minimization
Comparing with the negative log likelihood we notice that that loss coincides with the cross-entropy between the empirical distribution $\empdistribution$ and $\probtensor$, i.e.
	\[ \lossof{\probtensor} = \centropyof{\empdistribution}{\probtensor} \, . \]
We can therefore rewrite Problem~\ref{prob:parameterMaxLikelihood} as minimization of cross-entropies and of Kullback Leibler divergences as
\begin{align*}
	\argmin_{\probtensor\in\Gamma} \lossof{\probtensor} 
	= \argmin_{\probtensor\in\Gamma} \centropyof{\empdistribution}{\probtensor} 
	= \argmin_{\probtensor\in\Gamma} \kldivof{\empdistribution}{\probtensor} \, .
\end{align*}
	








\subsection{Forward Mapping in Exponential Families} 

\red{Integrate: Selection encodings suffice for variational methods, relational encodings of statistics are required for markov network instantiations of exponential families.}

Following \cite{wainwright_graphical_2008}, we can characterize the forward mapping as a variational problem.



\subsubsection{Variational Formulation}

Besides the direct computation of the mean parameter tensor we can give a variational characterization of the forward mapping.
This is especially useful, when the contraction is intractable, for example because the tensor $\expdist$ is infeasible to create.

We have
\begin{align*}
	\forwardmap(\canparam)  = \argmax_{\meanparam\in\meanset}  \contractionof{\meanparam,\canparam}{\varnothing} + \sentropyof{\probtensor^{\meanparam}} 
\end{align*}
where 
\begin{align*}
	 \meanset = \{  \contractionof{\probtensor,\sencodingof{\sstat}}{\selvariable}  :  \probtensor \quad \text{a distribution} \}
\end{align*}

% Forward mapping as gradient of A
In \cite{wainwright_graphical_2008}: Forward mapping coincides with gradient, i.e. $\meanparam = \nabla \cumfunction(\canparam)$.


\subsubsection{Mean Field Method}

Restrict the maximum over the mean parameters of efficiently contractable distributions and get a lower bound.

We rewrite 
\begin{align*}
	\max_{\meanparam\in\meanset}  \sbcontractionof{\meanparam, \canparam} + \sentropyof{\probtensor^{\meanparam}} 
	=
	\max_{\probtensor} \sbcontraction{\energytensor, \probtensor} + \sentropyof{\probtensor}
\end{align*}
where
	\[ \energytensor = \sbcontractionof{\sstat,\canparam}{\shortcatvariables} \, . \]

We now restrict the distributions in the maximum.
Typically we use the family of independent distributions, also called naive mean field method.
The naive mean field is the approximation by distributions of independent random variables $\legcoreof{\atomenumerator}$, that is
\begin{align*}
	\argmax_{\legcoreof{\atomenumerator} \, : \, \atomenumeratorin} \contractionof{\{\energytensor\} \cup \{\legcoreof{\atomenumerator} \, : \, \atomenumeratorin\}}{\varnothing}
	+ \sum_{\atomenumeratorin} \sentropyof{\legcoreof{\atomenumerator}} \, . 
\end{align*}

Algorithm~\ref{alg:NMF} is the alternation of legwise updates until a stopping criterion is met.

\begin{algorithm}[h!]
\caption{Naive Mean Field Approximation}\label{alg:NMF}
\begin{algorithmic}
\For{$\atomenumeratorin$}
	\State 
		\[ \legcoreofat{\atomenumerator}{\catvariableof{\atomenumerator}} = \onesat{\catvariableof{\atomenumerator}}  \]
\EndFor
\While{Stopping criterion is not met}
	\For{$\atomenumeratorin$}
		\State 
			\[ \legcoreofat{\atomenumerator}{\catvariableof{\atomenumerator}} 
			= \normationofwrt{ \expof{ \contractionof{ \{\energytensor[\shortcatvariables] \} \cup
				\{\legcoreofat{\secatomenumerator}{\catvariableof{\secatomenumerator}}\} }{\shortcatvariables} }
			}{\catvariableof{\atomenumerator}}{\varnothing} \]
\EndFor
\EndWhile
\end{algorithmic}
\end{algorithm}


%% Structured Variational approximation
More generically, one can use any Markov Network as the approximating family. % Also any exponential family?
Let $\graph$ be any hypergraph, we define the problem
\begin{align}\tag{$\mathrm{P}_{\mnexpfamily, \probtensor}$}\label{prob:structuredApproximation}
	\argmax_{\probtensor\in \mnexpfamily} \contractionof{\{\energytensor, \probtensor\}}{\varnothing} + \sentropyof{\probtensor}
\end{align}

\begin{theorem}
	The Markov Network with hypercores $\{ \hypercoreat{\edge} : \edgein \}$ is a stationary point for Problem~\ref{prob:structuredApproximation}, if for all $\edgein$
	\begin{align*}
	\hypercoreofat{\edge}{\catvariableof{\edge}}
	= \lambda\cdot \expof{
	\frac{
		\contractionof{\{\energytensor\}\cup\{
		\hypercoreat{\secedge} : \secedge\neq\edge
		\}}{\catvariableof{\edge}} 
	}{
		\contractionof{\{
		\hypercoreat{\secedge} : \secedge\neq\edge
		\}}{\catvariableof{\edge}} 
	}
	+ \sum_{\secedge\neq\edge} 
		\frac{
		\contractionof{\{\lnof{\hypercoreat{\secedge}}\}\cup\{
		\hypercoreat{\secedge} : \secedge\neq\edge
		\}}{\catvariableof{\edge}} 
	}{
		\contractionof{\{
		\hypercoreat{\secedge} : \secedge\neq\edge
		\}}{\catvariableof{\edge}} 
	}
	}
	\end{align*}
	for any $\lambda>0$ (e.g. by the norm).
	Here, the quotient denotes the coordinatewise quotient.
\end{theorem}
\begin{proof}
	By first order condition on the objective.
\end{proof}

%% KL Divergence
The mean field method corresponds with minimization of the KL Divergence to the efficiently contractable family, i.e. the I-projection onto the family.

\begin{theorem}
	For any hypergraph $\graph$ and energy tensor $\energytensor$ we have 
	\begin{align*}
		\argmax_{\probtensor\in \mnexpfamily} \contractionof{\{\energytensor, \probtensor\}}{\varnothing} + \sentropyof{\probtensor}
		= \argmax_{\probtensor\in \mnexpfamily} \kldivof{\expdistof{(\graph,\canparam)}}{\normationofwrt{\expof{\energytensor}}{\shortcatvariables}{\varnothing}}
	\end{align*}
	Problem~\ref{prob:structuredApproximation} is thus the I-projection onto the exponential family $\mnexpfamily$.
\end{theorem}
\begin{proof}
%	This follows from the fact, that the objective is the cross-entropy and the position of the maximum is invariant under substracting $\sentropyof{\probtensor}$.
	By rearranging the objective to the KL divergence.
\end{proof}


\subsubsection{Mode Search by annealing}

Finding the mode of a distribution is related to the forward mapping of $\invtemp\cdot\canparam$: $\meanparam$ to a delta distribution (or in the convex hull of multiple maxima) in the limit.

% Annealing effect on the optimization problem
This is because 
\begin{align*}
	\argmax_{\meanparam\in\meanset}  \contractionof{\meanparam,\canparam}{\varnothing} 
\end{align*}
is taken at an extreme point in $\meanset$ (since linear objective over closed convex set), which is a delta distribution of a set and
\begin{align*}
	\argmax_{\meanparam\in\meanset}  \contractionof{\meanparam,\invtemp\cdot\canparam}{\varnothing} + \sentropyof{\probtensor^{\meanparam}} 
	= 
	\argmax_{\meanparam\in\meanset}  \contractionof{\meanparam,\canparam}{\varnothing} + \frac{1}{\invtemp} \cdot \sentropyof{\probtensor^{\meanparam}} 	
\end{align*}
thus the entropy term is neglectible for large $\invtemp$.
A more precise argument is using a limit of the maxima and can be found in Theorem~8.1 in \cite{wainwright_graphical_2008}





\subsection{Backward Mapping in Exponential Families}

We find one backward mapping as the dual problem to the forward mapping.

\subsubsection{Variational Formulation}

The backward mapping to $\datamean = \sbcontractionof{\empdistribution, \sstat}{\selvariable}$ is Maximum Likelihood estimation and the solution of the maximum entropy problem.

\begin{lemma}
	Let there be a sufficient statistic $\sstat$.
	The map $\backwardmap: \rr^{\statorder}\rightarrow \rr^{\statorder}$ defined as
	\begin{align*}
		\backwardmap(\meanparam) = \argmax_{\canparam\in\rr^{\statorder}}  \contractionof{\meanparam,\canparam}{\varnothing} - \cumfunctionof{\canparam} \, . 
	\end{align*}
	is a backward mapping.
\end{lemma}
\begin{proof}
	From duality, see Theorem~3.4 in \cite{wainwright_graphical_2008}.
\end{proof}

% Gradient property
In \cite{wainwright_graphical_2008}: The objective is the conjugate dual $\dualcumfunction$ of $\cumfunction$, and backward mapping has an expression by the gradient, i.e. $\canparam = \nabla \dualcumfunction(\meanparam)$.


\begin{theorem}[Moment Matching Criteria]\label{the:MM}
	We have that $\canparam$ is a solution of the backward problem at $\genmean$, if and only if 
		\[ \sbcontractionof{\expdist,\sstat}{\selvariable} = \genmean[\selvariable] \, . \]
\end{theorem}

The equation in seeTheorem~\ref{the:mm} are called moment matching, since the moment of the empirical distribution is matched by the moment of the fitting distribution.




\subsubsection{Connection with Maximum Likelihood Estimation}

% Backward mapping
Backward mapping coincides with the Maximum Likelihood Estimation as we show next.

% From Parameter Estimation: When taking hypothesis to be exponential family
We now take $\Gamma$ to coincide with an exponential family $\expfamily$ for a sufficient statistic $\sstat$.

% Cross entropy
The objective is the cross entropy between a distribution with $\meanparam$ and the distribution $\expdistof{\sstat,\canparam}$.

\begin{lemma}
	Let $\sstat\in\facspace\otimes\rr^{\statorder}$ be a sufficient statistic and $\gendistribution\in\facspace$ a probability distribution.
	For any member $\expdist\in\expfamily$ we have
		\[ \centropyof{\gendistribution}{\expdist} = \sbcontraction{\canparam,\genmean} - \cumfunctionof{\canparam} \]
	where 
		\[ \genmean = \sbcontractionof{\gendistribution,\sstat}{\catvariableof{\sstat}} \,  \]
	and 
		\[ \cumfunction(\canparam) = \lnof{\contraction{\expof{\expenergy}}} \, . \]
	The M-projection of $\gendistribution$ onto $\expfamily$ is  $\expdistof{(\sstat,\estcanparam)}$ for
		\[ \estcanparam\in \argmax_{\canparam}  \contraction{\canparam, \genmean} - \cumfunctionof{\canparam} \, .  \]
\end{lemma}
\begin{proof}
	By decomposing 
	\begin{align*}
		\expdist 	& = \normationofwrt{\expof{\contractionof{\{\sstat, \canparam\}}{\shortcatvariables}}}{\shortcatvariables}{\varnothing} \\
				& = \frac{\expof{\expenergy}}{\contractionof{\{\expof{\expenergy}\}}{\varnothing}}
	\end{align*}
	we get
	\begin{align*}
		\lnof{\expdist} & = \lnof{\expof{\expenergy}} - \onesat{\shortcatvariables} \cdot \contractionof{\expof{\expenergy}}{\varnothing} \\ 
		& = \expenergy - \cumfunction(\canparam) \cdot \onesat{\shortcatvariables}  \, .
	\end{align*}
	If follows that
	\begin{align*}
		\centropyof{\gendistribution}{\expdist} 
		&=  \sbcontraction{\gendistribution,\lnof{\expdist}} \\
		&=  \sbcontraction{\gendistribution,\expenergy} - \cumfunction(\canparam) \cdot \contractionof{\{\gendistribution\}}{\varnothing}   \\
		&= \sbcontraction{\canparam, \genmean} - \cumfunction(\canparam) \, . 
	\end{align*}
\end{proof}




%\subsection{Maximum Likelihood and Maximum Entropy for Exponential Families}

Parameter Estimation is the M-Projection of a distribution onto the exponential family.

\begin{theorem}[\cite{wainwright_graphical_2008}]\ref{the:parEstToBackwardMap}
	Given any probability distribution $\probof{\shortcatvariables}$ and a exponential family defined by the sufficient statistic $\sstat$, the M-Projection onto the family is the distribution $\probtensorof{(\sstat,\estcanparam)}$ where
	\begin{align*}
		\estcanparam = \backwardmap(\contractionof{\probtensor,\sstat}{\catvariableof{\sstat}}) \, .
	\end{align*}
\end{theorem}
\begin{proof}
	$\contractionof{\probtensor,\sstat}{\catvariableof{\sstat}}$ is in $\imageof{\forwardmap}$ and MLE has a variational characterization with maximum at the dual $\estcanparam$, see \cite{wainwright_graphical_2008}.
\end{proof}





\subsubsection{Connection with Maximum Entropy}\label{sec:maxEntDuality}


%Primal: Maximum likelihood principle.
%\begin{align*}
%	\argmax_{\weight\in\rr^{\cardof{\formulaset}}} \lossof{\formulaset,\weight}
%\end{align*}


The Maximum entropy problem with respect to matching expected statistics is
\begin{align}\tag{$\mathrm{P}_{\sentropyof{}, \genmean}$}\label{prob:maxEntropy}
	\argmax_{\probtensor} \sentropyof{\probtensor} \quad \text{subject to} \quad  \sbcontractionof{\probtensor,\sencodingof{\sstat}}{\selvariableof{\sstat}} =  \genmean
\end{align}
where the optimization is over all probability distributions $\probtensor$.





\begin{theorem}\label{the:maxEntMaxLikeDuality}
	Let $\sstat$ be a map and $\gendistribution$ be any distribution of $\atomstates$ and define
		\[ \genmean = \sbcontractionof{\gendistribution,\sencodingof{\sstat}}{\selvariableof{\sstat}} \, .  \]
	Then the solution of \ref{prob:maxEntropy} coincides with the member $\expdistof{(\sstat,\estcanparam)}$ of the exponential family $\expfamily$ where
		\[ \estcanparam = \backwardmap(\genmean)\]
	for a backward map $\backwardmap$ of $\expfamily$.
\end{theorem}
\begin{proof}
	Classical result based on duality of maximum entropy and maximum likelihood, shown e.g. in Koller Book.
\end{proof}

%
Theorem~\ref{the:maxEntMaxLikeDuality} states, that when the maximum entropy problem has a solution (i.e. $\genmean\in\meanset$), then the solution is in the exponential family to the statistic $\sstat$.



\subsubsection{Alternating Algorithms to Approximate the Backward Map}\label{sec:alternatingBackwardMap}


\red{While the forward map always has a representation in closed form by contraction of the probability tensor, the backward map in general fails to have a closed form representation.
Computation of the Backward map can instead be performed by alternating algorithms.} % Are these fixpoint iterations?


Alternate through the coordinates of the statistics and adjust $\canparamat{\selvariable=\statenumerator}$ to a minimum of the likelihood, i.e. where
\begin{align*}
	0 = \frac{\partial}{\partial \canparamat{\selvariable=\statenumerator}} \lossof{\expdist}
\end{align*}

% Moment matching
This condition is equal to the collection of moment matching equations (see Theorem~\ref{the:mm})
\begin{align*}
	\sbcontractionof{\expdist,\sencodingof{\sstat}}{\selvariable=\statenumerator} = \sbcontraction{\empdistribution,\sencodingof{\sstat}}{\selvariable=\statenumerator} \, . 
\end{align*}


\begin{lemma}\label{lem:mmContractionEquation}
	For any sufficient statistic $\sstat$ a parameter vector $\canparam$ and a $\statenumeratorin$ we define
	\begin{align*}
	 	\hypercoreat{\catvariableof{\sstat_\statenumerator}} 
		= \contractionof{\{\rencodingof{\sstat}\}\cup\{\headcoreof{\tilde{\statenumerator}} : \tilde{\statenumerator} \in [\statorder], \tilde{\statenumerator}\neq\statenumerator\}}{\catvariableof{\sstat_\statenumerator}} \, . 
	\end{align*}
	Then the moment matching condition for $\sstat_\statenumerator$ relative to $\canparam$ and $\meanparam_\sstat$ is satisfied for any $\canparamat{\selvariable=\statenumerator}$ with
	\begin{align*}
		\sbcontraction{\headcoreof{\statenumerator}, \idrestrictedto{\imageof{\sstat_\statenumerator}}, \hypercoreat{\selvariable_\sstat}}
		= \sbcontraction{\headcoreof{\statenumerator}, \hypercoreat{\selvariable_\sstat}} \cdot \meanparam_\statenumerator \, . 
	\end{align*}
\end{lemma}
\begin{proof}
	We have
	\begin{align*}
		\expdist = \frac{
			\contractionof{\{\headcoreof{\statenumerator}, \hypercore \}}{\shortcatvariables}
		}{
			\sbcontraction{\headcoreof{\statenumerator}, \hypercore}
		}
	\end{align*}
	and 
	\begin{align*}
		\contractionof{\{\expdist, \sstat_\statenumerator\}}{\varnothing} 
		= \frac{
			\contractionof{\{\headcoreof{\statenumerator}, \idrestrictedto{\imageof{\sstat_\statenumerator}}, \hypercore \}}{\shortcatvariables}
		}{
			\contractionof{\{\headcoreof{\statenumerator}, \hypercore\}}{\varnothing}
		} \, . 
	\end{align*}
	Here we used
		\[ \sstat_\statenumerator = \contractionof{\{\headcoreof{\statenumerator}, \idrestrictedto{\imageof{\sstat_\statenumerator}}\}}{\shortcatvariables} \]
	and redundancies of copies of relational encodings.
	It follows that 
	\begin{align*}
		\contractionof{\{\expdist,\sstat_\statenumerator\}}{\varnothing} = \contractionof{\{\empdistribution,\sstat_\statenumerator\}}{\varnothing}
	\end{align*}
	is equal to
	\begin{align*}
		\contractionof{\{\headcoreof{\statenumerator}, \idrestrictedto{\imageof{\sstat_\statenumerator}}, \hypercoreat{\selvariable_\sstat}  \}}{\varnothing}
		= \contractionof{\{\headcoreof{\statenumerator}, \hypercoreat{\selvariable_\sstat}  \}}{\varnothing} \cdot \meanparam_\statenumerator \, . 
	\end{align*}	
\end{proof}

% Alternation necessary
The steps have to be alternated until sufficient convergence, since matching the moment to $\statenumerator$ by modifying $\canparamat{\selvariable=\statenumerator}$ will in general change other moments, which will have to be refit.


%Coordinate descent
An alternating optimization is the coordinate descent of the negative likelihood, seen as a function of the coordinates of $\canparam$, see Algorithm~\ref{alg:AMM}.
Since the log likelihood is concave, the algorithm converges to a global minimum.





\begin{algorithm}[h!]
\caption{Alternating Moment Matching}\label{alg:AMM}
\begin{algorithmic}
\State Set $\canparamat{\selvariable}=0$
\State Compute $\datameanat{\selvariable}= \sbcontractionof{\empdistribution,\sencodingof{\sstat}}{\selvariable}$
%\For{$\statenumeratorin$}
%	\State Set $\canparamat{\selvariable=\statenumerator}=0$ 
%	\State Compute $\meanparam_\statenumerator^{\datamap} = \contractionof{\{\empdistribution,\sstat_\statenumerator\}}{\varnothing} $ % Or give those as input!
%\EndFor
\While{Stopping criterion is not met}
\For{$\statenumeratorin$}
	\State Compute 
		\begin{align*}
			\hypercoreofat{\statenumerator}{\catvariableof{\sstat_\statenumerator}} 
			= \contractionof{\{\rencodingof{\sstat}\}\cup\{\headcoreof{\tilde{\statenumerator}} : \tilde{\statenumerator} \in [\statorder], \tilde{\statenumerator}\neq\statenumerator\}}{\catvariableof{\sstat_\statenumerator}} 
		\end{align*}
	\State Set $\canparamat{\selvariable=\statenumerator}$ to a solution of 
	\begin{align*}
		\contraction{\{\headcoreof{\statenumerator}, \idrestrictedto{\imageof{\sstat_\statenumerator}}, \hypercoreof{\statenumerator} \}}
		= \contraction{\{\headcoreof{\statenumerator},\hypercoreof{\statenumerator} \}} \cdot \meanparam_\statenumerator^{\datamap} \, . 
	\end{align*}
\EndFor
\EndWhile
\end{algorithmic}
\end{algorithm}


% 
In general, if $\imageof{\sstat_\statenumerator}$ contains more than two elements, there exists no closed form solutions.
We will investigate the case of binary images, where there are closed form expressions, later in Section~\ref{sec:alternatingParEstMLN}.


%
The computation of $\hypercore_\statenumerator$ in Algorithm~\ref{alg:AMM} can be intractable and be replaced by an approximative procedure based on message passing schemes.




% Parametrization of Hard Logic 
\chapter{\chatextlogicalRepresentation}\label{cha:logicalRepresentation}

Propositional logics describes systems with $\atomorder$ boolean variables, which are called atoms and denoted by $\atomicformulaof{\atomenumerator}$ for $\atomenumeratorin$.
Indices $\catindexof{\atomenumerator}\in[2]$ to the atoms $\atomenumeratorin$ enumerate the $2^\atomorder$ states of these systems, which are called worlds.
In each world indexed by $\shortcatindices=\catindices$ the indices $\atomicformulaof{\atomenumerator}$ encode whether the corresponding variable is $\truesymbol$. 

% Propositional logics
The epistemological commitments of propositional logics are whether the state is $\truesymbol$ or $\falsesymbol$ reflected by the coordinate of the one-hot encoding being $1$ or $0$.
Intuitively this describes, whether a specific world can be the state of a factored system.
Propositional logic amounts to reason about boolean variables, which are categorical variables with $2$ possible values.
Such boolean tensors have already appeared as base measures in the representation of probability distributions in \charef{cha:probRepresentation}.

Before discussing the semantics and syntax of propositional formulas, we first investigate how Boolean can be represented by vectors in order to mechanize their processing based on contractions.



\sect{Encoding of Booleans}\label{sec:booleanEncoding}

Booleans are variables valued by $\truthset$ and consist a basic data structure.

\subsect{Representation by coordinates}

To represent Booleans by categorical variables $\catvariable$ with two states we use the index interpretation function
\begin{align*}
	\indexinterpretation:[2]\rightarrow\truthset%\rightarrow\ozset
\end{align*}
defined as
\begin{align*}
	\indexinterpretationat{1} = \truesymbol%} = 1
	\quad \text{and} \quad \indexinterpretationat{0} = \falsesymbol \, .
\end{align*}

% Outlook: General interpretation maps
In \defref{def:subsetEncoding} in \parref{par:three} will define encodings of arbitrary sets based on index interpretation maps.

% Group homomorphism
One motivation for this particular choice of the interpretation function $\indexinterpretation$ is the effective execution of the conjunction as we show in the next Lemma.

\begin{lemma}
	$\indexinterpretation$ is a homomorphism between the groups
	\begin{align*}
		\big(\ozset,\cdot\big)  \quad \text{and} \quad \big(\truthset,\land\big) \, .
	\end{align*}
\end{lemma}
\begin{proof}
	It suffices to notice, that for arbitrary $\truthstateof{0},\truthstateof{1}\in\ozset$ we have
	\begin{align*}
		\indexinterpretationat{\truthstateof{0} \cdot \truthstateof{1}}
		= \indexinterpretationat{\truthstateof{0}} \land \indexinterpretationat{\truthstateof{1}}  \, .
	\end{align*}
\end{proof}
	
% Interpretation of boolean contraction and type conversion application
Based on this homomorphism, contractions of boolean tensors, in which all variables are kept open, can be regarded as parallel calculations of the conjunction $\land$ encoded by $\indexinterpretation$.
This homomorphism is further applied in type conversion in dynamically-typed languages (e.g. in $\mathrm{python}$ \cite{python_software_foundation_python_2025}).

% Nonlinearity issues
Operations like the negation fail to be linear and are only affine linear, since for $\truthstate\in\truthset$ we have
\begin{align}\label{eq:affineLinearNegation}
 	\invindexinterpretationat{\lnot\truthstate} = 1 - \invindexinterpretationat{\truthstate}  \, .
\end{align}
Since any logical connective can be represented as a composition of conjunctions and negations, any logical connective corresponds with an affine linear function on the interpreted truth values.
Direct applications of this insight to execute logical calculus will be discussed later in \secref{sec:effectiveCalculus}.
For our purposes here, we would like to execute logical connective based on single contractions and avoid summations over them.
This is why we call the negation representation as in \eqref{eq:affineLinearNegation} the affine representation problem, which we in the following want to resolve.

% Disjunction central interpretation
While in this work, we will always encode boolean states by $\indexinterpretation$, other index interpretation functions could be chosen.
For example, the interpretation
	\[ \indexinterpretationof{\lor}:\ozset\rightarrow\truthset \]
defined as
    	\[ \indexinterpretationofat{\lor}{0} = \truesymbol \quad \text{and} \quad \indexinterpretationofat{\lor}{1} = \falsesymbol \, , \]
results is a homomorphism between the groups	
	\[ \big(\ozset,\cdot\big) \quad \text{and} \quad \big(\truthset,\lor\big)  \, . \]
While placing the disjunction $\lor$ as the logical connective effectively executed by contractions, the negation will for arbitrary interpretations mapping onto $\ozset$ remain the function %\eqref{eq:affineLinearNegation}
\begin{align*}
 	\invindexinterpretationofat{\lor}{\lnot\truthstate} = 1 - \invindexinterpretationofat{\lor}{\truthstate}  \, .
\end{align*}
Thus, the problem of affine linear operations cannot be resolved by a clever choice of an interpretation function with image in $\ozset$. 


\subsect{Representation by basis vectors} % This is what Basis Calculus does! Refer to that here?

While contractions can just perform conjunctions, we need a representation trick to extend the contraction expressivity to arbitrary connectives and resolve the affine representation problem.
To this end we now compose $\indexinterpretation$ with the one-hot encoding $\onehotmap$ and get an encoding
\begin{align*}
	\onehotmap\circ\invindexinterpretation : \truthset \rightarrow \ozbasisset \, ,
\end{align*}
where $\catvariable$ is a categorical variable with $\catdim=2$.
For any $\truthstate\in\truthset$ we have
\begin{align*}
	\onehotmap\circ\invindexinterpretationat{\truthstate} =
	\begin{bmatrix}
		\invindexinterpretationat{\lnot\truthstate} \\
		\invindexinterpretationat{\truthstate}
	\end{bmatrix}  \, .
\end{align*}


% Resolving the affine representation problem
Performing the negation now amounts to switching the coordinates of the encoded vector, which can be performed by contraction with a transposition matrix
\begin{align*}
	\rencodingofat{\lnot}{\headvariableof{\lnot},\catvariable} = 
	\begin{bmatrix}
		0 & 1 \\
		1 & 0
	\end{bmatrix} \, ,
\end{align*}
where in this notation we always understand the first variable $\catvariable$ as the row index selector and the second variable $\headvariableof{\notucon}$ as the column index selector.
We then have
\begin{align*}
	\onehotmap\circ\invindexinterpretationat{\lnot\truthstate}[\headvariableof{\lnot}]
	= \contractionof{\rencodingofat{\notucon}{\headvariableof{\notucon},\catvariable},\onehotmap\circ\invindexinterpretationat{\truthstate}[\catvariable]}{\headvariableof{\notucon}} \, .
\end{align*}
We therefore arrived at our aim to resolve the affine representation problem and have found a procedure to represent logical negations by a contraction, which is a linear operation.
Besides negations, we will show in this chapter, that arbitrary logical formulas can be represented by contractions.

\subsect{Coordinate and Basis Calculus}

Our findings on the encoding of booleans hint towards more general schemes to encode information into boolean tensors, which will be explored in more detail in \charef{cha:coordinateCalculus} and \charef{cha:basisCalculus}.
When each coordinate in a boolean tensor represents one in $\ozset$ interpreted boolean we call the scheme coordinate calculus.
In basis calculus on the other hand, booleans are represented by elements of $\ozbasisset$.
In that scheme, there are pairs of two coordinates (building slice vectors of the tensors), which are restricted to be different from each other.
This amounts to posing a global directionality constraint on the boolean tensor, as will be shown in \theref{the:rencodingDirected}.

\sect{Semantics of Propositional Formulas}

% Structure
We now choose a semantic centric approach to propositional logic, by defining formulas as boolean tensors.
Then we investigate the corresponding syntax of formulas as specification of a tensor network decomposition of the relational encoding of formulas.

\subsect{Formulas}

% Intro of formulas
Logics is especially useful in interpreting boolean tensors representing Propositional Knowledge Bases, based on connections with abstract human thinking.
To make this more precise, we associate each such tensor is associated with a formula $\exformula$ being a composition of the atomic variables with logical connectives as we proof next.

\begin{definition}\label{def:formulas}
	A propositional formula $\formulaat{\shortcatvariables}$ depending on $\atomorder$ atoms $\catvariableof{\atomenumerator}$ is a boolean-valued tensor
	\begin{align*}
		\formulaat{\shortcatvariables} : \atomstates \rightarrow \ozset \subset \rr \, .
	\end{align*}
	We call a state $\shortcatindices \in \atomstates$ a model of a propositional formula $\formula$, if
	\begin{align*}
		\formulaat{\indexedshortcatvariables}=1 \, .
	\end{align*}
	If there is a model to a propositional formula, we say the formula is satisfiable.
\end{definition}

% Boolean Tensors
The propositional formulas coincide therefore with the boolean tensors (see \defref{def:booleanTensor}).


% Decomposition into model sums
Since propositional formulas are binary valued tensors, the generic decomposition of \lemref{lem:tensorBasisDecomposition} simplifies to
\begin{align}\label{eq:formulaModelDecomposition}
	\formulaat{\shortcatvariables} = \sum_{\catindices\in\atomstates} \formulaat{\indexedcatvariables} \cdot \onehotmapofat{\shortcatindices}{\shortcatvariables} \\
	= \sum_{\catindices\in\atomstates \, : \, \formulaat{\indexedcatvariables}=1}  \onehotmapofat{\catindices}{\shortcatvariables} \, .
\end{align}
Thus, any propositional formula is the sum over the one-hot encodings of its models.
This is equal to the encoding of the set of models, which will be introduced in \charef{cha:basisCalculus} (see \defref{def:subsetEncoding}).

We depict this decomposition in the diagrammatic notation by
\begin{center}
	\begin{tikzpicture}[scale=0.35, thick] % , baseline = -3.5pt

%\draw[->-] (2,-1)--(2,1) node[midway,right] {\tiny $\catvariableof{\exformula}$};
\draw (-1,-1) rectangle (5,-3);
\node[anchor=center] (text) at (2,-2) {\small ${\exformula}$};
\draw[] (0,-3)--(0,-5) node[midway,left] {\tiny $\randomxof{0}$}; 
\draw[] (1.5,-3)--(1.5,-5) node[midway,left] {\tiny $\randomxof{1}$}; 
\node[anchor=center] (text) at (3,-4) {$\cdots$};
\draw[] (4,-3)--(4,-5) node[midway,right] {\tiny $\randomxof{\atomorder\shortminus1}$}; 


\node[anchor=center] (text) at (7,-2) {${=}$};

\node[anchor=center] (text) at (12,-2.5) {${\sum\limits_{\atomindices\in\atomstates}}$};
\node[anchor=center] (text) at (12,-4) {\tiny $\exformula(\atomindices)=1$};

\begin{scope}[shift={(19.5,1)}]

%\draw (-2,1) rectangle (4,-1);
%\node[anchor=center] (text) at (1,0) {\small $\onehotmapof{\exformula(\atomindices)}$};
%\draw[->-] (1,1)--(1,3) node[midway,right] {\tiny $\catvariableof{\exformula}$};

\draw (-3,-2) rectangle (-1,-4);
\node[anchor=center] (text) at (-2,-3) {\small $\onehotmapof{\atomlegindexof{0}}$};
\draw[->-] (-2,-4)--(-2,-6) node[midway,right] {\tiny $\catvariableof{0}$};

\node[anchor=center] (text) at (1,-3) {\small $\cdots$};

\draw (3,-2) rectangle (5,-4);
\node[anchor=center] (text) at (4,-3) {\small $\onehotmapof{\atomlegindexof{\atomorder\shortminus1}}$};
\draw[->-] (4,-4)--(4,-6) node[midway,right] {\tiny $\catvariableof{\atomorder\shortminus1}$};

\end{scope}

\end{tikzpicture}
\end{center}




% Maps to multiple formulas -> Later?
%We can extend the map to factored systems of multiple formulas, by using \defref{def:formulas} as coordinate maps.
%This is exactly what we will study by Bayesian Propositional Networks.
%We will make use of redundancies in the maps to get an efficient representation based on decompositions.





%% Semantic approach
We here chose a semantic approach to propositional logic in contrary to the standard syntactical approach.
Instead of defining formulas by connectives acting on atomic formulas, we define them here as binary valued functions of the states of a factored system.
They are interpreted by marking possible states as models, given the knowledge of $\exformula$.
The syntactical side will then be introduced later by studying decompositions of formulas.


%\begin{figure}[h]
%\begin{center}
%	\begin{tikzpicture}[scale=0.35, thick] % , baseline = -3.5pt

%\draw[->-] (2,-1)--(2,1) node[midway,right] {\tiny $\catvariableof{\exformula}$};
\draw (-1,-1) rectangle (5,-3);
\node[anchor=center] (text) at (2,-2) {\small ${\exformula}$};
\draw[] (0,-3)--(0,-5) node[midway,left] {\tiny $\randomxof{0}$}; 
\draw[] (1.5,-3)--(1.5,-5) node[midway,left] {\tiny $\randomxof{1}$}; 
\node[anchor=center] (text) at (3,-4) {$\cdots$};
\draw[] (4,-3)--(4,-5) node[midway,right] {\tiny $\randomxof{\atomorder\shortminus1}$}; 


\node[anchor=center] (text) at (7,-2) {${=}$};

\node[anchor=center] (text) at (12,-2.5) {${\sum\limits_{\atomindices\in\atomstates}}$};
\node[anchor=center] (text) at (12,-4) {\tiny $\exformula(\atomindices)=1$};

\begin{scope}[shift={(19.5,1)}]

%\draw (-2,1) rectangle (4,-1);
%\node[anchor=center] (text) at (1,0) {\small $\onehotmapof{\exformula(\atomindices)}$};
%\draw[->-] (1,1)--(1,3) node[midway,right] {\tiny $\catvariableof{\exformula}$};

\draw (-3,-2) rectangle (-1,-4);
\node[anchor=center] (text) at (-2,-3) {\small $\onehotmapof{\atomlegindexof{0}}$};
\draw[->-] (-2,-4)--(-2,-6) node[midway,right] {\tiny $\catvariableof{0}$};

\node[anchor=center] (text) at (1,-3) {\small $\cdots$};

\draw (3,-2) rectangle (5,-4);
\node[anchor=center] (text) at (4,-3) {\small $\onehotmapof{\atomlegindexof{\atomorder\shortminus1}}$};
\draw[->-] (4,-4)--(4,-6) node[midway,right] {\tiny $\catvariableof{\atomorder\shortminus1}$};

\end{scope}

\end{tikzpicture}
%\end{center}
%\caption{Direct interpretation of a propositional formula $\exformula$ as a tensor.
%	The tensor is the sum of the one hot encodings of its models.
%	While the one hot encodings are directed, their sum is not.}
%\label{fig:formulaDirect} 
%\end{figure}

%% Intro of connectives
%Logical connectives are basic building blocks of such formulas and can be understood by simple computations represented in truth tables.
% Here truth tables?
%We call each combination of atomic formulas with connectives a formula.

\subsect{Relational encoding of formulas}


%% Direct and Relational interpretation of $\exformula$
There are two ways to represent formulas by tensors.
One way is to understand $[2]$ as subset of $\rr$ and interpreting the formula directly as a tensor (as in \defref{def:formulas}).
Another way is to understand $[2]$ as the possible values of a categorical variable.
% Maps between factored systems
Following this second perspective, formulas are maps between factored systems, where the image system is the factored systems of atoms and the target system the atomic system defined by a variable $\formulavar$ representing the formula satisfaction.
%Following this perspective, formulas are maps between the factored systems of atoms and the atomic system of the formula.
We can then build the relational encoding (\defref{def:functionRepresentation}) of that map to represent the formula (see Figure~\ref{fig:formulaRencoding}).

%\begin{definition}[Relation Encoding of Formulas] % Own definition, since a reinterpretation of the formula
Given a factored system with $\atomorder$ atoms $\shortcatvariables$ and a propositional formula $\formula$, the relational encoding of $\formula$ (see \defref{def:functionRepresentation}) is the tensor
\begin{align*}
	\rencodingofat{\formula}{\formulavar,\shortcatvariables} \in  \left(\atomspace \right) \otimes \rr^2
\end{align*}
decomposable as
\begin{align} 
	\rencodingofat{\formula}{\formulavar,\shortcatvariables} 
	= & \sum_{\shortcatindices\in\atomstates}  \onehotmapofat{\shortcatindices}{\shortcatvariables} \otimes \onehotmapofat{\formulaat{\indexedshortcatvariables}}{\formulavar} \, . 
\end{align}
%\end{definition}

%% More general relational encodings
We can build relational encodings more generally of any tensors, where we identify the image of the tensor with the states of a categorical variable.
Exactly for propositional formulas, this construction will lead to Boolean image variables.


\begin{lemma}\label{lem:formulaEncodingDecomposition}
	For any formula $\formula$ we have
	\begin{align*}
		\rencodingofat{\formula}{\formulavar,\shortcatvariables}
		= \formulaat{\shortcatvariables} \otimes \tbasisat{\formulavar}
		+ \lnot\formulaat{\shortcatvariables} \otimes  \fbasisat{\formulavar} \, .
	\end{align*}
	In particular
	\begin{align*}
		 \formulaat{\shortcatvariables} = \contractionof{
		\rencodingofat{\formula}{\formulavar,\shortcatvariables},\tbasisat{\formulavar}
		}{\shortcatvariables} \, .
	\end{align*}
\end{lemma}
\begin{proof}
%% Decomposition
	We can decompose relational encodings of formulas into the sum (see Figure~\ref{fig:formulaRencoding}) % ! Not a tensor network decomposition !
	\begin{align} 
		 \rencodingofat{\formula}{\formulavar,\shortcatvariables}  
		 = & \fbasisat{\formulavar} \otimes \left( \sum_{\formulazerocoordinates}  \onehotmapofat{\shortcatindices}{\shortcatvariables} \right) \\
		 + & \tbasisat{\formulavar} \otimes \left( \sum_{\formulaonecoordinates}  \onehotmapofat{\shortcatindices}{\shortcatvariables} \right)
	\end{align}
	where the second term sums up the models of $\formula$ and the first one the models of $\lnot\formula$.
\end{proof}


% Comparison with direct interpretation
Compared with the direct interpretation of a formula as a tensor and the decomposition into models in Equation~\ref{eq:formulaModelDecomposition}, we notice that the relational encoding also represents encoding of worlds where the formula is not satisfied.
This representation is required to represent arbitrary propositional formulas by contracted tensor networks of its components, as will be investigated in the following sections.

%% Coordinatewise 
The relational encoding $\rencodingof{\exformula}$ has slices
\begin{align*}
	\contractionof{\rencodingof{\exformula},\onehotmapof{\shortcatindices}}{\formulavar} 
		\rencodingofat{\exformula}{\indexedshortcatvariables,\formulavar}
	= \begin{cases}
		\tbasis[\formulavar] & \text{if the world $\shortcatindices$ is a model of $\exformula$}  \\
		\fbasis[\formulavar] & \text{else}\, .
		\end{cases}
\end{align*}
The contractions of the relational encoding therefore calculate whether an assignment of atoms is a model of the formula, using basis calculus (see \theref{the:basisCalculus}).

\begin{figure}[h]
\begin{center}
	\input{./PartI/tikz_pics/logic_representation/formula_rencoding.tex}
\end{center}
\caption{Relational encoding of a propositional formula. 
The encoding is a sum of the one hot encodings of all states of the factored system in a tensor product with basis vectors, which encode whether the state is a model of the formula.
The tensor is directed, since any contraction with an encoded state results in the basis vector evaluating the formula, which we called basis calculus.
}
\label{fig:formulaRencoding} 
\end{figure}


\sect{Syntax of Propositional Formulas}

Relational encodings of propositional formulas are especially useful when representing function compositions by the representation of their components (see \theref{the:compositionByContraction}).
In propositional logics, the syntax of defining propositional formulas is oriented on compositions of formulas by connectives. % Quantifications will be studied in the FOL Chapter.
We in this section investigate the decomposition schemes of relational encodings into tensor networks of component encodings for binary tensors following propositional logic syntax.

\subsect{Atomic Formulas}

We call atomic formulas the most granular formulas, which are not splitted into compositions of other formulas.
Our syntactic decomposition of propositional formulas will then investigate, how any propositional formula can be represented by these.

\begin{definition}
	The tensors $\formulaofat{\atomenumerator}{\shortcatvariables}$ defined for $\shortcatindices\in\atomstates$ as
	\begin{align*}
		\formulaofat{\atomenumerator}{\indexedshortcatvariables}
		= \catindexof{\atomenumerator}
	\end{align*}
	are called atomic formulas.
\end{definition}

Atomic formulas and their relational encodings have an especially compelling representation.

\begin{theorem}
	Any atomic formula $\formulaofat{\atomenumerator}{\shortcatvariables}$ is represented as
	\begin{align*}
		\formulaofat{\atomenumerator}{\indexedshortcatvariables}
		= \contractionof{\tbasisat{\catvariableof{\atomenumerator}}}{\shortcatvariables}
		= \tbasisat{\catvariableof{\atomenumerator}} \otimes \onesat{\catvariableof{[\catorder]/\{\atomenumerator\}}}  \, .
	\end{align*}

	The relational encoding of any atomic formula $\atomicformulaofat{\atomenumerator}{\shortcatvariables}$ has a tensor decomposition by
	\begin{align*}
		\rencodingofat{\atomicformulaof{\atomenumerator}}{\headvariableof{\atomorder},\shortcatvariables}
		= \contractionof{\identityat{\catvariableof{\atomenumerator},\headvariableof{\atomenumerator}}}{\shortcatvariables}
		= \identityat{\catvariableof{\atomenumerator},\headvariableof{\atomenumerator}} \otimes \onesat{\catvariableof{[\catorder]/\{\atomenumerator\}}} \, .
	\end{align*}
	The decomposition is depicted in a network diagram as
	\begin{center}
		\begin{tikzpicture}[scale=0.35,thick] % , baseline = -3.5pt

\drawatomcore{3.5}{-8}{$\bencodingof{\formulaof{\atomenumerator}}$}
\drawatomindices{3.5}{-12}	
\draw[->-] (5.5,-9)--(5.5,-7) node[midway,right] {\tiny $\headvariableof{\atomenumerator}$};

\node[anchor=center] (text) at (10,-10) {${=}$};

\draw (12,-9) rectangle (15,-11); 
\node[anchor=center] (text) at (13.5,-10) {\small $\ones$}; 
\draw[-<-] (12.5,-11)--(12.5,-13) node[midway,left] {\tiny $\catvariableof{0}$};
\node[anchor=center] (text) at  (13.5,-12) {$\cdots$};
\draw[-<-] (14.5,-11)--(14.5,-13) node[midway,right] {\tiny $\catvariableof{\atomenumerator\shortminus1}$};

\node[anchor=center] (text) at (16.25,-10) {\small $\otimes$}; 

\draw[->-] (18.5,-9)--(18.5,-7) node[midway,right] {\tiny $\headvariableof{\atomenumerator}$};
\draw (17.5,-9) rectangle (19.5,-11);
\node[anchor=center] (text) at (18.5,-10) {\small $\delta$}; 
\draw[-<-]  (18.5,-11)--(18.5,-13) node[midway,right] {\tiny $\catvariableof{\atomenumerator}$};

\node[anchor=center] (text) at (20.75,-10) {\small $\otimes$}; 

\begin{scope}[shift={(10,0)}]

\draw (12,-9) rectangle (15,-11); 
\node[anchor=center] (text) at (13.5,-10) {\small $\ones$}; 
\draw[-<-]  (12.5,-11)--(12.5,-13) node[midway,left] {\tiny $\catvariableof{\atomenumerator+1}$};
\node[anchor=center] (text) at  (13.5,-12) {$\cdots$};
\draw[-<-]  (14.5,-11)--(14.5,-13) node[midway,right] {\tiny $\catvariableof{\atomorder\shortminus1}$};

\node[anchor=center] (text) at  (16.5,-13) {$.$};

\end{scope}

\end{tikzpicture}
	\end{center}
\end{theorem}
\begin{proof}
	We have by definition
	\begin{align*}
		\rencodingofat{\atomicformulaof{\atomenumerator}}{\headvariableof{\atomenumerator},\shortcatvariables}
		=& \sum_{\catindices\in\atomstates} \onehotmapofat{\catindices}{\shortcatvariables} \otimes \onehotmapofat{\formulaofat{\atomenumerator}{\indexedcatvariables}}{\headvariableof{\atomenumerator}} \\
		=& \left( \onehotmapofat{0,0}{\catvariableof{\atomenumerator},\headvariableof{\atomenumerator}} +
		\onehotmapofat{1,1}{\catvariableof{\atomenumerator},\headvariableof{\atomenumerator}} \right) \otimes \onesat{\catvariableof{\secatomenumerator}\, : \, \secatomenumerator \neq \atomenumerator} \\
		=& \contractionof{\identityat{\catvariableof{\atomenumerator},\headvariableof{\atomenumerator}}}{\shortcatvariables,\headvariableof{\atomenumerator}} \, .
	\end{align*} 
\end{proof}

\subsect{Syntactical combination of formulas}

Propositional formulas are elements of tensor spaces with $\atomorder$ axis. 
The number of coordinates thus grows exponentially with the number of atoms, which is
\begin{align*}
	\dimof{\atomspace} = 2^{\atomorder} \, .
\end{align*}
When the number of atoms is large, the naive representation of formula tensors will be thus intractable.
In contrast, typcial logical formulas appearing in practical knowledge bases are sparse in the sense that they have short representations in a logical syntax.
Motivated by this consideration we now discuss propositional syntax and investigate the sparse decomposition of formula tensors along their formula structure to avoid the curse of dimensionality.

%% Propositional Syntax
In logical syntax formulas are described by atomic formulas recursively connected via connectives. 
We show, that representations of logical connectives can be represented by feasible tensor cores $\rencodingof{\exconnective}$ contracted along a tensor network.
Let us first provide in \exaref{exa:connectives} unary ($\atomorder=1$) and binary ($\atomorder=2$) connectives.

\begin{example}\label{exa:connectives}
	We use the following connectives:
	\begin{itemize}
	\item negation $\notucon: [2]\rightarrow [2]$ by the vector
	\begin{align*}
		\notucon[\exformulavar] = \begin{bmatrix}
		0  \\
		1  
		\end{bmatrix} 
	\end{align*}
	\item conjunctions $\land:  [2]\times[2] \rightarrow[2]$
		\begin{align*}
			\land[\exformulavar,\secexformulavar]
			 = \begin{bmatrix}
			0 & 0 \\
			0 & 1 
			\end{bmatrix}
		\end{align*}
	\item disjunctions $\lor : [2]\times[2] \rightarrow[2]$
		\begin{align*}
			\lor[\exformulavar,\secexformulavar]
			 = \begin{bmatrix}
			0 & 1 \\
			1 & 1 
			\end{bmatrix}
		\end{align*}
	\item exact disjunction $\oplus:  [2]\times[2] \rightarrow[2]$	
		\begin{align*}
			\oplus[\exformulavar,\secexformulavar]
			 = \begin{bmatrix}
			0 & 1 \\
			1 & 0 
			\end{bmatrix}
		\end{align*}
	\item implications $\impbincon:  [2]\times[2] \rightarrow[2]$ 
		\begin{align*}
			\impbincon[\exformulavar,\secexformulavar]
			 = \begin{bmatrix}
			1 & 1 \\
			0 & 1 
			\end{bmatrix}
		\end{align*}
	\item biimplication $\eqbincon:  [2]\times[2] \rightarrow[2]$ 
		\begin{align*}
			\eqbincon[\exformulavar,\secexformulavar]
			 = \begin{bmatrix}
			1 & 0 \\
			0 & 1 
			\end{bmatrix}
		\end{align*}
	\end{itemize}
\end{example}

\begin{figure}[h]
\begin{center}
	\begin{tikzpicture}[scale=0.35, thick] % , baseline = -3.5pt

\node[anchor=center] (text) at (2,-4) {$a)$};

\draw[->] (5.5,-5)--(5.5,-3) node[midway,right] {\tiny $\headvariableof{\neg\exformula}$};

\node[anchor=center] (text) at (5.5,-6) {$\rencodingof{\lnot}$};
\draw (4.5,-7) rectangle (6.5,-5);

\draw[->] (5.5,-9)--(5.5,-7) node[midway,right] {\tiny $\formulavar$};


\drawatomcore{3.5}{-8}{$\rencodingof{\exformula}$}
\drawatomindices{3.5}{-12}	




\begin{scope}[shift={(15,0)}]

\node[anchor=center] (text) at (2,-4) {$b)$};

\draw[->] (9.5,-5)--(9.5,-3) node[midway,right] {\tiny $\headvariableof{\exformula\circ\secexformula}$};

\node[anchor=center] (text) at (9.5,-6) {$\rencodingof{\circ}$};
\draw (4.5,-7) rectangle (14.5,-5);

\draw[->] (5.5,-9)--(5.5,-7) node[midway,right] {\tiny $\formulavar$};

\drawatomcore{3.5}{-8}{$\rencodingof{\exformula}$}
\drawatomindices{3.5}{-12}	

\begin{scope}[shift={(8,0)}]

	\draw[->] (5.5,-9)--(5.5,-7) node[midway,right] {\tiny $\secexformulavar$};

	\drawatomcore{3.5}{-8}{$\rencodingof{\secexformula}$}
	\drawatomindices{3.5}{-12}	

\end{scope}

\draw[fill] (7.5,-15) circle (0.25cm);
\draw[] (7.5,-15) to[bend left=25] (3.5,-13);
\draw[] (7.5,-15) to[bend right=25] (11.5,-13);

\draw[fill] (9,-15.25) circle (0.25cm);
\draw[] (9,-15.25) to[bend left=25] (5,-13);
\draw[] (9,-15.25) to[bend right=25] (13,-13);

\draw[fill] (11.5,-15) circle (0.25cm);
\draw[] (11.5,-15) to[bend left=25] (7.5,-13);
\draw[] (11.5,-15) to[bend right=25] (15.5,-13);



\draw[] (7.5,-15)--(7.5,-17) node[midway,left] {\tiny $\catvariableof{0}$}; 
\draw[] (9,-15.25)--(9,-17) node[midway,left] {\tiny $\catvariableof{1}$}; 
\node[anchor=center] (text) at (10.5,-16.5) {$\cdots$};
\draw[] (11.5,-15)--(11.5,-17) node[midway,right] {\tiny $\catvariableof{\atomorder-1}$}; 

\end{scope}

\end{tikzpicture} 
\end{center}
\caption{a) Relational encoding of a negated formula $\exformula$ as a tensor network of the encoded formula and the encoded connective $\notucon$.
	b) Relational encoding of a composition of formulas $\exformula, \secexformula$ by a connective $\exconnective\in\{\land,\lor,\oplus,\impbincon,\eqbincon\}$. 
	The encoding is a contraction of encodings to  $\exformula, \secexformula$ and $\exconnective$.}
	\label{fig:unaryBinaryComposition} 
\end{figure}

We now show how formulas consisting of connectives acting on other formulas can be represented by basis calculus.
Let there be formulas $\exformula$ and $\secexformula$ depending on categorical variables $\shortcatvariables$ and a binary connective
	\[ \exconnective: [2]\times[2] \rightarrow[2] \, . \]
Then we can show as a special case of the next theorem, that (see \figref{fig:unaryBinaryComposition})
\begin{align*}
	\rencodingofat{\exformula\exconnective\secexformula}{\shortcatvariables,\catvariableof{\exformula\exconnective\secexformula}}
	= \contractionof{
	\rencodingofat{\exconnective}{\formulavar,\secexformulavar,\catvariableof{\exformula\exconnective\secexformula}},
	\rencodingofat{\exformula}{\shortcatvariables,\formulavar},
	\rencodingofat{\secexformula}{\shortcatvariables,\secexformulavar} 
	}{
	\shortcatvariables,\catvariableof{\exformula\exconnective\secexformula}
	} \, . 
\end{align*}

For any unary connective $\exconnective: [2] \rightarrow[2]$ we have
\begin{align*}
	\rencodingofat{\exconnective\exformula}{\shortcatvariables,\catvariableof{\exconnective\exformula}}
	= \contractionof{
	\rencodingofat{\exconnective}{\formulavar,\catvariableof{\exconnective\exformula}},
	\rencodingofat{\exformula}{\shortcatvariables,\formulavar}
	}{
	\shortcatvariables,\catvariableof{\exconnective\exformula}} \, . 
\end{align*}

Let us now generalize this observation to arbitrary arity of connectives and provide a proof of its correctness.

% CLASH OF NOTATION: HEADVARIABLES - CATVARIABLES
\begin{theorem}[Composition of Formulas]\label{the:formulaDecomposition}
	Let there be a formula $\formulaat{\shortcatvariables}$, which has a syntactical decomposition into connectives $\{\connectiveofat{\selindex}{\headvariableof{\nodesof{\selindex}}} : \selindexin\}$ taking their inputs by variables $\headvariableof{\nodesof{\selindex}}\subset \headvariableof{\nodes}$ and output by a variable $\headvariableof{\connectiveof{\selindex}}$
	%Let there be a set of boolean variables $\headvariableof{\formulaset}$, where $\formulaset$ is  and boolean atom variables $\shortcatvariables$.
	We here denote by $\formulaset$ the set of sub-formulas and use a boolean variable $\headvariableof{\secexformula}$ for each $\secexformula\in\formulaset$.
	In particular, we denote for each atom in $\formulaset$ the corresponding boolean variable by $\headvariableof{\atomenumerator}$.
	It then holds
	\begin{align*}
		\rencodingofat{\formula}{\formulavar,\shortcatvariables} =
		\contractionof{\left\{
		\rencodingofat{\connectiveof{\selindex}}{\headvariableof{\connectiveof{\selindex}},\headvariableof{\nodesof{\selindex}}} : \selindexin
		\right\} \cup \{\identityat{\headvariableof{\atomenumerator},\catvariableof{\atomenumerator}} \, : \, \atomenumeratorin\}}
		{\formulavar,\shortcatvariables} \, . 
	\end{align*}
\end{theorem}
\begin{proof}
	When a variable in $\headvariableof{\formulaset}$ appears multiple times as input to connectives, we replace it by a set of copies (which wont change the contraction, since all tensors are binary and \theref{the:invarianceAddingSubcontractions} can be applied).
	This follows from an iterative application of \theref{the:compositionByContraction} to be shown in \charef{cha:basisCalculus}.
\end{proof}

\begin{remark}[$\atomorder$-ary connectives such as $\land$ and $\lor$]\label{rem:naryConnectives}
	Since the decomposition of relational encoding can be applied to generic function compositions (see \theref{the:compositionByContraction}), we can also allow for $\atomorder$-ary connectives
		\[ \exconnective : \bigtimes_{\atomenumeratorin} [2] \rightarrow [2] \, . \]
	The connectives $\land$ and $\lor$ satisfy associativity and have thus straightforward generalizations to the $\atomorder$-ary case.
	This is because associativity can be exploited to represent the relational encoding by any tree-structured composition of binary $\land$ and $\lor$ connectives.
\end{remark}

%Propositional Syntax describes generic formulas $\exformula$ based on the composition of these maps starting with atomic formulas.
%We can thus apply \theref{the:compositionByContraction} recursively to decompose the formula tensors $\rencodingof{\exformula}$ into cores $\rencodingof{\exconnective}$.

%% Construction from atomic formula tensors
Propositional syntax consists in the application of connectives on atomic formulas, and recursively on the results of such constructions.
When passed towards connective cores, atomic formula tensors act trivial on the legs and just identify the corresponding atomic formula index $\catindexof{\atomicformulaof{\atomenumerator}}$ with $\catindexof{\atomenumerator}$.
This is due to the fact, that contractions with the trivial tensor $\ones$ leaves any tensor invariant, and the contraction with the elementary matrix $\identity$ identifies indices with each other.
We can thus savely ignore the atomic formula tensors appearing in the decomposition of formula tensors to non-atomic formulas.
An example of such a decomposition is depicted in \figref{fig:decompositionExample}.

\begin{figure}[h]
\begin{center}
	\begin{tikzpicture}[scale=0.35, yscale=-1, thick] % , baseline = -3.5pt

\begin{scope}[shift={(-15,0)}]

\node[anchor=center] (text) at (-3,-6) {${a)}$};

	\node [circle, draw, thick, fill=gray!50] (T1) at (0,0) {\tiny $\catvariableof{a}$};
	\node [circle, draw, thick, fill=gray!50] (T2) at (3,0) {\tiny $\catvariableof{b}$};
	\node [circle, draw, thick, fill=gray!50] (T3) at (6,0) {\tiny $\catvariableof{c}$};
	
	\node [circle, draw, thick, fill=gray!50] (and) at (1.5,-3) {\tiny $\land$};
	\node [circle, draw, thick, fill=gray!50] (not) at (6,-3) {\tiny $\lnot$};	
	
	\draw [->] (T1) -- (and);
	\draw [->] (T2) -- (and);
	
	\draw [->] (T3) -- (not);	
	
	\node [circle, draw, thick, fill=gray!50] (head) at (3.25,-6) {\tiny $\land$};
	
	\draw [->] (and) -- (head);
	\draw [->] (not) -- (head);			
\end{scope}

\node[anchor=center] (text) at (-3,-6) {${b)}$};

\draw[->] (0,1)--(0,-1) node[midway,left] {\tiny $\catvariableof{a}$}; 
\draw[->] (1.5,1)--(1.5,-1) node[midway,right] {\tiny $\catvariableof{b}$}; 
\draw[->] (3,1)--(3,-1) node[midway,right] {\tiny $\catvariableof{c}$}; 
\draw (-1,-1) rectangle (4, -3);
\node[anchor=center] (text) at (1.5,-2) {\small $\rencodingof{a \land b \land \lnot c}$};
\draw[->] (1.5,-3)--(1.5,-5) node[midway,right] {\tiny $\headvariableof{a \land b \land \lnot c}$}; 

\node[anchor=center] (text) at (5,-2) {${=}$};


\begin{scope}[shift={(7,0)}]

\draw[->] (0,1)--(0,-1) node[midway,left] {\tiny $\catvariableof{a}$}; 
\draw[->] (3,1)--(3,-1) node[midway,right] {\tiny $\catvariableof{b}$}; 
\draw[->] (6,1)--(6,-1) node[midway,right] {\tiny $\catvariableof{c}$}; 
	
\draw (-1,-1) rectangle (4, -3);
\node[anchor=center] (text) at (1.5,-2) {\small $\rencodingof{\land}$};

\draw[->] (1.5,-3) --(1.5,-5) node[midway,right]{\tiny $\headvariableof{a \land b}$};

\draw (5,-1) rectangle (7, -3);
\node[anchor=center] (text) at (6,-2) {\small $\rencodingof{\lnot}$};

\draw[->] (6,-3) --(6,-5) node[midway,right]{\tiny $\headvariableof{\lnot c}$};
	
\draw (0.5,-5) rectangle (6.5,-7);
\node[anchor=center] (text) at (3.5,-6) {\small $\rencodingof{\land}$};
	
\draw[->] (4,-7) -- (4,-9) node[midway,right] {\tiny $\headvariableof{a \land b \land \lnot c}$};

%\draw (3,-9) rectangle (5,-11);
%\node[anchor=center] (text) at (4,-10) {$\truevectorat{}$};

\end{scope}

\end{tikzpicture}
\end{center}
\caption{Decomposition of the formula tensor to $\exformula = a \land b \land \lnot c$ into unary (matrix) and binary (third order tensor) cores.
	a) Visualization of $\exformula$ as a graph.
	b) Tensor Network decomposition of $\exformula$.
	We can make use of the invariance of a Hadamard product with a constant tensor $\ones$ and thus not draw axis to atoms not affected by a formula.}
\label{fig:decompositionExample}
\end{figure}

\begin{remark}[Tensor Network Decomposition of Formulas]
	The decomposition of the propositional into a tensor network is a hierarchical decomposition of the formula tensor, which we will describe in more detail in \secref{sec:HT}.
	Of special interest are tree hypergraphs, where the format is called Hierarchical Tucker.
	At each decomposition of a formula into sub-formulas, two subspaces spanned by the respective atomic spaces are selected. 
\end{remark}


\subsect{Syntactical decomposition of formulas}\label{sec:termClauseDecomposition}

% Decomposition in case of missing 
We have seen how the decomposition of complex formulas into connectives acting on the component formulas can be exploited to find effective representations of the semantics by tensor networks.
Here the question arises here, how to perform such decompositions in case of a missing syntactical representation of a formula.
By \defref{def:formulas} any binary tensor is a formula.
We show in the following, how we can find a syntactic specification of a formula given its tensor.

%
%Let us now show that any formula tensor can be decomposed into a network of these connective symbols and atomic formula tensors.


\begin{definition}[Terms and Clauses]\label{def:clauses}
	Given two disjoint subsets $\nodesof{0}$ and $\nodesof{1}$ of $[\atomorder]$, the corresponding term is the formula defined on the indices $\shortcatindices\in\atomstates$ by
		\[ \termofat{\nodesof{0}}{\nodesof{1}}{\shortcatvariables}
		=\left( \bigwedge_{\atomenumerator\in\nodesof{0}} \lnot\formulaof{\atomenumerator} \right)  \land \left( \bigwedge_{\atomenumerator\in\nodesof{1}} \formulaof{\atomenumerator} \right)  \]
	and the corresponding clause is the formula defined on the indices $\catindices\in\atomstates$ by
		\[ \clauseofat{\nodesof{0}}{\nodesof{1}}{\shortcatvariables}
		=\left( \bigvee_{\atomenumerator\in\nodesof{0}} \formulaof{\atomenumerator} \right)  \lor \left( \bigvee_{\atomenumerator\in\nodesof{1}} \lnot\formulaof{\atomenumerator} \right)  \, , \]
	where by $\land_{\atomenumerator\in\nodes}$ and $\lor_{\atomenumerator\in\nodes}$ we refer to the $n$-ary connectives $\land$ and $\lor$.
	%We call the clause a minterm, if $\nodesof{0}\cup\nodesof{1} = [\atomorder]$.
	We call the term a minterm and the clause a maxterm, if $\nodesof{0}\cup\nodesof{1} = [\atomorder]$.
\end{definition}

%% 
Terms and Clauses have for any index tuple $\shortcatindices$ a polynomial representation by
		\[ \termof{\nodesof{0}}{\nodesof{1}}[\indexedshortcatvariables] 
		= \left( \prod_{\atomenumerator \in \nodesof{0}} (1-\catindexof{\atomenumerator}) \right)
		\left(  \prod_{\atomenumerator \in \nodesof{1}} \catindexof{\atomenumerator} \right) \]
and
		\[ \clauseof{\nodesof{0}}{\nodesof{1}}[\indexedshortcatvariables] 
		= 1 - \left( \prod_{\atomenumerator \in \nodesof{0}} (1-\catindexof{\atomenumerator})\right)
		\left(  \prod_{\atomenumerator \in \nodesof{1}} \catindexof{\atomenumerator} \right) \, . \]


\begin{lemma}\label{lem:termClauseOneHot}
	Terms are contractions of one-hot encodings, that is for any disjoint subsets $\nodesof{0},\nodesof{1}\subset[\atomorder]$ we have
		\[ \termof{\nodesof{0}}{\nodesof{1}}[\shortcatvariables] = \contractionof{\onehotmapof{\{\catindexof{\atomenumerator}=0 : \atomenumerator\in\nodesof{0} \} \cup \{\catindexof{\atomenumerator}=1 : \atomenumerator\in\nodesof{1}\}}}{\shortcatvariables} \, . \]
	Clauses are substractions of one-hot encodings from the trivial tensor, that is for any disjoint subsets $\nodesof{0},\nodesof{1}\subset[\atomorder]$ we have
		\[ \clauseof{\nodesof{0}}{\nodesof{1}}[\shortcatvariables] = 
		\onesat{\shortcatvariables} -
		\contractionof{\onehotmapof{\{\catindexof{\atomenumerator}=0 : \atomenumerator\in\nodesof{0} \} \cup \{\catindexof{\atomenumerator}=1 : \atomenumerator\in\nodesof{1}\}}}{\shortcatvariables} \, . \]
\end{lemma}


	
%
The reference of the formulas in the case $\nodesof{0}\dot{\cup}\nodesof{1} = [\atomorder]$ as minterms and maxterms is due to the fact, that minterms are formulas with unique models and maxterms are formulas with a unique world not satisfying the formula.
% Enumeration by $\atomstates$
We use this insight and enumerate maxterms and minterms by the index $\catindex\in\atomstates$ of the unique world where the minterm is satisfied, respectively the maxterm is not satisfied.
For any $\nodesof{0}\dot{\cup}\nodesof{1} = [\atomorder]$ we take the index tuple $\catindices$ where $\catindexof{\atomenumerator}=0$ if $\atomenumerator\in\nodesof{0}$ and $\catindexof{\atomenumerator}=1$ if $\atomenumerator\in\nodesof{1}$ and define
\begin{align*}
	\maxtermof{\catindices} = \clauseof{\nodesof{0}}{\nodesof{1}} \quad \text{and} \quad \mintermof{\catindices} = \termof{\nodesof{0}}{\nodesof{1}} \, .
\end{align*}


\begin{corollary}
	Minterms are basis elements of the tensor space, that is for any $\shortcatindices\in\atomstates$ we have
	\begin{align*}
		\mintermof{\shortcatindices} = \onehotmapofat{\shortcatindices}{\shortcatvariables}
	\end{align*}
	Maxterms are substraction of basis elements from the trivial tensor, that is for any $\shortcatindices\in\atomstates$ we have
	\begin{align*}
		\maxtermof{\shortcatindices} = \onesat{\shortcatvariables} - \onehotmapofat{\shortcatindices}{\shortcatvariables}  \, .
	\end{align*}
\end{corollary}
\begin{proof}
	Follows from \lemref{lem:termClauseOneHot}, since when $\nodesof{0}\cup\nodesof{1} = [\atomorder]$ the contraction of the one-hot encodings coincides with the one-hot encoding of a fully specified state.
\end{proof}


Based on this insight, we can decompose any propositional formula into a conjunction of maxterms or a disjunction of minterms as we show next.

\begin{theorem}\label{the:tensorToMaxMinTerms}
	For any boolean tensor $\hypercoreat{\shortcatvariables}\in\atomspace$ with leg-dimensions two we have
	\begin{align*}
		\hypercoreat{\shortcatvariables} = \left( \bigvee_{\hyperonecoordinates} 
		\termof{\catzeropositions}{\catonepositions} 
		\right)
		[\shortcatvariables] 
	\end{align*}
	and
	\begin{align*}
		\hypercoreat{\shortcatvariables} = \left( \bigwedge_{\hyperzerocoordinates} 
		\clauseof{\catzeropositions}{\catonepositions} 
		\right)
		[\shortcatvariables] \, .
	\end{align*}
\end{theorem}
\begin{proof}
	To show the representation by minterms we use the decomposition
	\begin{align*}
		\hypercoreat{\shortcatvariables}  = \sum_{\hyperonecoordinates} \onehotmapofat{\shortcatindices}{\shortcatvariables}
	\end{align*}
	and notice that each term in the disjunction modifies the formula by adding respective world $\shortcatindices$ to the models of the formula.
	To show the representation by maxterms we use the decomposition
	\begin{align*}
		\hypercoreat{\shortcatvariables}  = \onesat{\shortcatvariables} \quad - \sum_{\hyperzerocoordinates} \onehotmapofat{\shortcatindices}{\shortcatvariables}
	\end{align*}
	and notice that each term in the conjunction modifies the formula by removing the respective world $\shortcatindices$ from the models of the formula.	
	Thus, both decompositions are propositional formulas with the same set of models as the formula $\hypercore$ and are thus identical to $\hypercore$.
\end{proof}


% Canonical normal forms
The decompositions found in \theref{the:tensorToMaxMinTerms} are also called canonical normal forms to propositional formulas $\hypercoreat{\shortcatvariables}$.

%% Universality of representations
\begin{remark}[Efficient Representation in Propositional Syntax]
	% Relation with binary CP
	The decomposition in \theref{the:tensorToMaxMinTerms} is a basis CP decomposition of the binary tensor and will further be investigated in \charef{cha:sparseCalculus}.
	The formulas constructed in the proof of \theref{the:tensorToMaxMinTerms} are however just one possibility to represent a formula tensor in propositional syntax.
	Typically there are much sparser representations for many formula tensors, in the sense that less connectives and atomic symbols are required.
	Having such a sparser syntactical description of a propositional formula can be exploited to find a shorter conjunctive normal form of the formula and construct a sparse polynomial based on similar ideas as in \theref{the:tensorToMaxMinTerms}.
	%One way to eliminate syntactical redundancies are through schemes for decompositions called normal forms, for example the Conjunctive Normal Form (CNF) or the Disjunctive Normal Form (DNF).
	We will provide such constructions in \charef{cha:sparseCalculus}, where we show that dropping the demand of directionality and investigating binary CP Decompositions will improve the sparsity of the polynomial formula representation.
\end{remark}

\subsect{Comparing with probabilistic approaches }

Both probability and logic provide a human-understandable interface to machine learning.
As we will describe in \parref{par:two}, they can be combined in one formalism providing efficient reasoning.

% Same thesises repeated??
\textbf{Probability} represents the uncertainty of states.
The categorical variables are called random variables and their joint distribution is represented by a probability tensor.
Humans interpret probabilities by Bayesian and frequentist approaches.
Reasoning based on Bayes Theorem has an intuitive interpretation in terms of evidence based update of prior distributions to posterior distributions.
However it is based on interpreting (large amounts) of numbers, which makes it hard for humans to assess the probabilistic reasoning process.

\textbf{Logics} explains relations between sets of worlds in a human understandable way.
Categorical variables have dimension $2$, where the first is interpreted as indicating a $\falsesymbol$ state and the second as a $\truesymbol$ state.
We mainly restrict to propositional logics, where there are finite sets of such variables called atomic formulas.
Using model-theoretic semantics it defines entailment of sets by other sets, which is understandable as a consequence relation.

\textbf{Tensors} unify both approaches since they are natural numerical structures to represent properties of states in factored systems.
The potential is then based in employing scalable multilinear algorithms to solve reasoning problems.
Further, algorithms formulated in tensor networks have a high parallelization potential, which is why they are of central interest in the development of AI-dedicated software and hardware.

The different areas have developed separated languages to describe similar objects.
Here we want to provide a rough comparison of those in a dictionary.

\begin{tabular}{l|l|l|l}
    & \textbf{Probability Theory} & \textbf{Propositional Logic} & \textbf{Tensors}   \\
    \hline
    \textit{Atomic System}        & Random Variable             & Atomic Formula               & Vector             \\
    \textit{Factored System}      & Joint Distribution          & Knowledge Base               & Tensor             \\
    \textit{Categorical Variable} & Random Variable             & Atomic Formula               & Axis of the Tensor
\end{tabular}

While the probability theory lacks to provide an intuition about sets of events, propositional syntax has limited functionality to represent uncertainties.
Tensors on the other side can build a bridge by representing both functionalities and relying on probability theory and logics for respective interpretations.


\sect{Outlook}

While we in this chapter investigated representation schemes for single propositional formulas, we will further study the representation of knowledge bases consisting in multiple formulas in \secref{sec:hardNetworks}.
Further, we will build hybrid models bridging the concepts of probability distributions and propositional logics in \secref{sec:hybridNetworks}.
Propositional formulas will therein serve as features and base measures for exponential families.
%Further study of representing Knowledge Bases based on Tensor Networks of its formulas in \secref{sec:hardNetworks} (see \theref{the:conDecKB}).
\input{PartI/logical_reasoning.tex}



% Reasoning on MLN using Tensor Network Decompositions
\part{Neuro-Symbolic Learning}

We now employ tensor networks to define architectures and algorithms for neuro-symbolic reasoning based on the logical and probabilistic foundations.
Markov Logic Networks will be taken as generative models to be learned from data, using formula selecting tensor networks and likelihood optimization algorithms.

\input{PartII/logic_batch.tex}

% Representation
\section{Representing Logic Networks}

Markov Logic Networks exploit the efficiency and interpretability of logical calculus as well as the expressivity of graphical models. 

%Markov Logic Networks are probability functions of truth assignments to logical functions.
%They respect propositional logic as hard constraints, but have beyond that freedom to shape probability distributions on possible situations.
%To capture these properties, we define them as graphical models with structure cores representing propositional logics and activation cores representing the specification of probability distributions.
% We in this part employ them to combine the probabilistic and the logical paradigm.



\subsection{Markov Logic Networks as Exponential Families}

We introduce Markov Logic Networks in the formalism of exponential families (see Section~\ref{sec:exponentialFamilies}).

\begin{definition}[Markov Logic Networks]
	Markov Logic Networks are exponential families $\mlnexpfamily$ with sufficient statistics by functions
		\[ \formulaset : \atomstates \rightarrow \bigtimes_{\exformulain}[2] \subset \rr^{\cardof{\formulaset}} \]
	defined coordinatewise by propositional formulas $\exformulain$.
\end{definition}

% Binary Statistics as propositional formula
Since the image of each feature is contained in $[2]$, they are propositional formulas (see Def.~\ref{def:formulas}).

% Characterization of MLNs among exponential families: When choosing binary features
Conversely, any binary feature $\sstatcoordinateof{\statenumerator}$ of an exponential family defines a propositional formula (see Definition~\ref{def:formulas}).
Thus, any exponential family of distributions of $\atomstates$, such that $\imageof{\sstatcoordinateof{\statenumerator}}\subset\{0,1\}$ for all $\statenumeratorin$ is a set of Markov Logic Networks with fixed formulas.

% Formula Selecting Networks
%We will further study the sparse representation of formula sets in Chapter~\ref{cha:architectures}.


The sufficient statistics consistent in a map $\formulaset$ of formulas brings the following advantages:
\begin{itemize}
	\item Numerical Advantage: The sufficient statistics is decomposable into logical connectives. 
	If the formulas are sparse (in the sense of limited number of connectives necessary in their representation), this gives rise to efficient tensor network decompositions of the relational encoding.
	\item Statistical Advantage: Since each formula is Boolean valued, the coordinates of the sufficient statistic are Bernoulli variables. 
	Due to their boundedness, they and their averages (by Hoeffdings inequality) are sub-Gaussian variables with favorable concentration properties (absence of heavy tails).
\end{itemize}


\begin{remark}[Alternative Definitions]
	We here defined MLNs on propositional logic, while originally they are defined in FOL.
	The relation of both frameworks will be discussed further in Chapter~\ref{cha:folModels}.
	Besides that we allow for degeneracy and improper MLNs.
\end{remark}


%\subsubsection{Interpretation as Graphical Models}






\subsection{Tensor Network Representation of MLNs}

Based on the previous discussion on the representation of exponential families by tensor networks in Section~\ref{sec:exponentialFamilies} we now derive a representation for Markov Logic Networks.

\begin{theorem}[Relational Encodings for Markov Logic Networks]
	A Markov Logic Network to a set of formulas $\formulaset = \{\formulaof{\selindex} \, : \, \selindexin\}$ is represented as
	\begin{align*}
		\mlnprobat{\shortcatvariables} = 
		\normationof{
			\{\rencodingofat{\formulaof{\selindex}}{\catvariableof{\formulaof{\selindex}},\shortcatvariables} : \selindexin \} 
			\cup \{ 
			\headcoreofat{\formulaof{\selindex},\canparamat{\selvariable=\selindex}}{\catvariableof{\formulaof{\selindex}}} 
			: \selindexin \}
		}{\shortcatvariables}
	\end{align*}
	where we denote for each $\selindexin$ a head core
	\begin{align*}
		\headcoreofat{\formulaof{\selindex},\canparamat{\indexedselvariableof{}}}{\catvariableof{\formulaof{\selindex}}} 
		= \begin{bmatrix} 1 & \expof{\canparamat{\indexedselvariableof{}}} \end{bmatrix}[\catvariableof{\formulaof{\selindex}}] \, .
	\end{align*}
\end{theorem}
\begin{proof}
	The claim follows from Theorem~\ref{def:expFamilyTensorRep} and the following contraction equations.
	We have with the grouped variable $\catvariableof{\formulaset} = \{\catvariableof{\formulaof{\selindex}}\, : \, \selindexin\}$
	\begin{align*}
		\rencodingofat{\formulaset}{\shortcatvariables,\catvariableof{\formulaset}}
		= \contractionof{\{\rencodingofat{\formulaof{\selindex}}{\catvariableof{\formulaof{\selindex}},\shortcatvariables} : \selindexin \}}{\shortcatvariables,\catvariableof{\formulaset}} \, .
	\end{align*}
	Since we have a Markov Logic Network we have $\imageof{\formulaof{\selindex}}\subset [2]$ and thus
	\begin{align*}
	 	\headcoreofat{\formulaof{\selindex},\canparamat{\indexedselvariableof{}}}{\indexedcatvariableof{\formulaof{\selindex}}} 
		= \begin{cases}
			1 & \text{for} \quad \catvariableof{\formulaof{\selindex}} = 0 \\
			\expof{\canparamat{\indexedselvariableof{}}} & \text{for} \quad \catvariableof{\formulaof{\selindex}} = 1
		\end{cases}  
	\end{align*}
	Using these equations, the claim follows from Theorem~\ref{def:expFamilyTensorRep}.
\end{proof}

\begin{figure}[h]
\begin{center}
	\begin{tikzpicture}[thick, scale=0.35] % , baseline = -3.5pt

\drawundiratomindices{0}{-4}
\draw (-2,-1) rectangle (6, -3);
\node[anchor=center] (text) at (2,-2) {$\expof{\canparamat{\selvariable=\selindex}\cdot\formulaof{\selindex}}$};

		
\node[anchor=center] (text) at (10,-2) {${=}$};


\begin{scope}[shift={(15,-2)}]

		\draw (-0.5,3) rectangle (4.5, 5);
		\node[anchor=center] (text) at (2,4) {$\headcoreof{\formulaof{\selindex},\canparamat{\selvariable=\selindex}}$};

		\draw[fill] (2,2.25) circle (0.25cm);
		\draw[] (2,2.25) -- (2,3);
		\draw[->] (2,1) -- (2,2.5) node[midway, left] {\tiny $\catvariableof{\formulaof{\selindex}}$};
		
		\draw (-1,1) rectangle (5, -1);
		\node[anchor=center] (text) at (2,0) {$\rencodingof{\formulaof{\selindex}}$};

		\drawatomindices{0}{-2}

\end{scope}

\end{tikzpicture}
\end{center}
\caption{Factor of a Markov Logic Network to a formula $\formulaof{\selindex}$.}
% Where $\headcoreofat{\formulaof{\selindex}}{\catvariableof{\formulaof{\selindex}}} =\begin{bmatrix} 1 & \expof{\weightof{\exformula}} \end{bmatrix}[\catvariableof{\exformula}] $}
\label{fig:mlnFactor}
\end{figure}

% 
Since any member of an exponential family is a Markov Network with tensors to each coordinate of the statistic, also Markov Logic Networks are Markov Networks.

\begin{corollary}\label{cor:MLNasMN}
	Given a set $\formulaset$ of formulas on atomic variables $\catvariableof{\nodes}$, we construct a $\graph=(\nodes,\edges)$, where $\nodes$ are decorated by the atoms and
		\[ \edges = \{ \nodesof{\formula}: \formula\in\formulaset \} \, , \]
	where by $\nodesof{\formula}$ we denote the minimal set such that there exists a tensor $\hypercoreat{\catvariableof{\nodesof{\formula}}}$ with
		\[ \formulaat{\catvariableof{\nodes}} = \hypercoreat{\catvariableof{\nodesof{\formula}}} \otimes \onesat{\catvariableof{\nodes/\nodesof{\formula}}} \, . \]		
	Any Markov Logic Network $\mlnparameters$ is then a Markov Network given the graph $\graphof{\formulaset}$
	$\{\expof{\canparamat{\selvariable=\selindex}\cdot\formulaof{\selindex}}
\, :\,\selindexin\}$.
\end{corollary}


% MLN as graphical models
Markov Logic Networks are Markov Networks with the factors given in a restricted form from the weighted truth of a formula.
Each formula is seen as a factor of the graphical model.

There are two sparsity mechanisms drastically reducing the number of parameters (and loosing generality):
\begin{itemize}
	\item Factors/Formulas contain only subsets of atoms (already in Corollary~\ref{cor:MLNasMN} exploited):
		The underlying assumptions of conditional independence loss generality.
	\item Structure in the factors: In MLN each factor corresponds with a formula evaluated on possible worlds.
		Again, any possible factor can be represented by a formula, but we concentrate on small formulas (see Theorem \ref{the:FormulaToTensor}).
\end{itemize}


% 
\red{We can extend the set of variables, by including the hidden formulas, and get a Markov Network of the relational encodings of connectives and headcores.
Here hidden variables are additional variables facilitating the decomposition, but not appearing in open variables of contractions when doing reasoning.
One can then exploit redundancies and make sure that every subresult is computed just once, by dropping relational encodings with identical head functions.
}


\begin{figure}[h]
\begin{center}
	\input{PartII/tikz_pics/mln/decomposed_representation.tex}
\end{center}
\caption{Example of a decomposed Markov Network representation of a Markov Logic Network with formulas $\{\formulaof{0} = a\lor b, \formulaof{1} = a \lor b \lor \lnot c\}$.
	Since both formulas share the subformula $a\lor b$, their contracted factors have a representation by a connected tensor network.}
% Where $\headcoreofat{\formulaof{\selindex}}{\catvariableof{\formulaof{\selindex}}} =\begin{bmatrix} 1 & \expof{\weightof{\exformula}} \end{bmatrix}[\catvariableof{\exformula}] $}
\label{fig:mlnDecRep}
\end{figure}


%\begin{theorem}[Selection encodings for Energy representation]
%	\red{More the definition of exponential families.}
%	The energy of Markov Logic Networks is the contraction
%		\[ \mlnenergy = \sbcontractionof{\sencodingof{\formulaset},\canparam}{\shortcatvariables} \, . \]
%\end{theorem}


%When the further factor cores contain only one variable, we can label them by the formula $\exformula$ corresponding with that node $[1,\expof{\weightof{\exformula}}]$.
%\begin{definition}{Markov Logic Networks}
%	Given a set of formulas $\mlnformulaset$ with weights $\weight:\mlnformulaset\rightarrow\rr$ the Markov Logic Network is the distribution
%		\[ \mlnprobat{\indexedcatvariables}= \frac{1}{\partitionfunctionof{\mlnparameters}} \expof{
%			\sum_{\mlnformulain} \exformula(\atomindices) \weightof{\exformula}
%			} \]
%	where the partition function
%		\[ \partitionfunctionof{\mlnparameters} = \sum_{\atomindicesin}  \expof{
%			\sum_{\mlnformulain} \exformula(\atomindices) \weightof{\exformula}
%		} \]
%	ensures the normation.
%\end{definition}

%where the partition function is the contraction 
%\begin{align}
%	\partitionfunctionof{\mlnparameters} = \sbcontraction{\expof{\mlnenergy}} \, . 
%\end{align}



\subsubsection{Energy tensors of Markov Logic Networks}

%% Tensor Representation of MLN
With the energy tensor
\begin{align}
	\mlnenergy 
	= \sum_{\selindexin} \canparamat{\selvariable=\selindex} \cdot \formulaofat{\selindex}{\shortcatvariables} 
	= \sbcontractionof{\sencodingofat{\formulaset}{\shortcatvariables,\selvariable},\canparamat{\selvariable}}{\shortcatvariables} 
\end{align}
the MLN is the distribution
\begin{align}
	\mlnprobat{\shortcatvariables} = \normationofwrt{\expof{\mlnenergy}}{\shortcatvariables}{\varnothing} \, . 
\end{align}

In case of a common structure of the formulas in a Markov Logic Network, Formula selecting networks can be applied to represent their energies.

% Energy representation
%The weighted sum of formulas is then the energy of the Markov Logic Network.
We represent the superposition of formulas as a contraction with s parameter tensor.
Given a factored parametrization of formulas $\exformula_{\parindices}$ with indices $\selindexof{\parenumerator}$ we have the superposition by the network representation:
\begin{center}
	\begin{tikzpicture}[thick, scale=0.35] % , baseline = -3.5pt

\node[anchor=east] (text) at (-3,0) {$\sum_{\parindexof{[\parorder]}\in\parstates} \canparamat{\selvariableof{[\parorder]}=\parindexof{[\parorder]}} {\exformula_{\parindexof{[\parorder]}}} \quad {=}$};

%\node[anchor=center] (text) at (0.5,-8) {$\mathrm{log}$};

%\drawatomcore{3.5}{-8}{$\rencodingof{\fselectionmap}$}
%\drawatomindices{3.5}{-12}	
%
%
%\drawatomcore{3.5}{-4}{$\canparam$}
%\drawparindices{3.5}{-8}	



\drawatomindices{0}{-4}
\draw (-1,3) rectangle (5, -3);
\node[anchor=center] (text) at (2,0) {$\rencodingof{\fselectionmap}$};

\draw[->] (2,3)--(2,5) node[midway,right] {\tiny $\catvariableof{\fselectionmap}$}; 
\draw (1,5) rectangle (3,7);
\node[anchor=center] (text) at (2,6) {$\tbasis$};

\draw[<-] (5,-2)--(7,-2) node[midway,below] {\tiny $\selvariableof{0}$}; 
\draw[<-] (5,-0.5)--(7,-0.5) node[midway,below] {\tiny $\selvariableof{1}$}; 
\node[anchor=center] (text) at (6,0.75) {$\vdots$};
\draw[<-] (5,2)--(7,2) node[midway,above] {\tiny $\selvariableof{\parorder\shortminus1}$}; 

\draw (7,3) rectangle (9,-3);
\node[anchor=center] (text) at (8,0) {$\canparam$};

\end{tikzpicture}
\end{center}


% Representation 
If the number of atoms and parameters gets large, it is important to represent the tensor ${\exformula_{\parindices}}$ efficiently in tensor network format and avoid contractions.
To avoid inefficiency issues, we also have to represent the parameter tensor $\canparam$ in a tensor network format to improve the variance of estimations (see Chapter~\ref{cha:mlnConcentration}) and provide efficient numerical algorithms.

% Fail of full probability representation
However, when required to instantiate the probability distribution of a Markov Logic Network as a tensor network, we need to exponentiate and normate the energy tensor, a task for which relational encodings are required.
For such tasks, contractions of formula selecting networks are not sufficient and each formula with a nonvanishing weight needs to be instantiated as a factor tensor of a Markov Network. 






\subsection{Expressivity}\label{sec:MLNMaxMintermRep}

Based on Markov Logic Networks containing only maxterms and minterms (see Definition~\ref{def:clauses}), we here provide an expressivity study.
There are $2^{\atomorder}$ maxterms and $2^{\atomorder}$ minterms which are enough to represent any probability distribution as we show next.

\begin{theorem}\label{the:maximalClausesRepresentation}\label{the:mintermExpressivityMLN}
	Let there be a positive probability tensor %distribution of the worlds to the atoms $\shortcatvariables$ with probability tensor
		 \[ \probof{\shortcatvariables} \in \bigotimes_{\atomenumeratorin}\rr^2 \, . \] %= \frac{\expof{\mlntensor}}{\partitionfunctionof{\mlntensor}} \, . \]
	Then the Markov Logic Network of minterms (see Definition~\ref{def:clauses})
		\[ \mintermformulaset = \{\mintermof{\atomindices} \, : \, \atomindices\in\atomstates \}\]
	with parameters %with nonzero weights at the maxterms indexed by $\atomindicesin$
		\[ \canparamat{\selvariableof{0}=\catindexof{0},\ldots,\selvariableof{\atomorder-1}=\catindexof{\atomorder-1}}% \weightof{\mintermof{\atomindices}} 
		= \ln \probof{\indexedcatvariables} \]
	coincides with $\probof{\shortcatvariables}$.

	Further, the Markov Logic Network of maxterms
		\[ \maxtermformulaset = \{\maxtermof{\atomindices} \, : \, \atomindices\in\atomstates \}\]
	with wparameters
		\[ \canparamat{\selvariableof{0}=\catindexof{0},\ldots,\selvariableof{\atomorder-1}=\catindexof{\atomorder-1}} %\weightof{\maxtermof{\atomindices}} 
		= - \ln\probof{\indexedcatvariables} \]
	coincides with $\probof{\shortcatvariables}$.
\end{theorem}
\begin{proof}
	It suffices to show, that in both cases of choosing $\formulaset$ by minterms or maxterms with the respective parameters
		\[ \mlnenergy =  \ln\probof{\shortcatvariables} \]
	and therefore
		\[ \mlnprobat{\shortcatvariables} 
		= \sbnormationof{\expof{\mlnenergy}}{\shortcatvariables} 
		=  \sbcontractionof{\expof{\mlnenergy}}{\shortcatvariables} 
		= \probof{\shortcatvariables}\, . \]
	
	In the case of minterms, we notice that for any $\atomindicesin$
		\[ \mintermof{\atomindices}[\shortcatvariables] = \onehotmapofat{\atomindices}{\shortcatvariables} \]
	and thus with the weights in the claim
		\[ \sum_{\atomindicesin} 
		\left( \ln \probof{\indexedcatvariables} \right) \cdot \mintermof{\atomindices}[\shortcatvariables] 
		= \ln\probof{\shortcatvariables} \, .
		 \]

	For the maxterms we have analogously
		\[ \maxtermof{\atomindices}[\shortcatvariables] = \onesat{\shortcatvariables} - \onehotmapofat{\catindices}{\shortcatvariables} \]
	and thus that the maximal clauses coincide with the one-hot encodings of respective states.
	We thus have
	\begin{align*}
		& \sum_{\atomindicesin} 
		\left( - \ln \probof{\indexedcatvariables} \right) \cdot \maxtermof{\atomindices}[\shortcatvariables] \\
		& =
		\left(  \sum_{\nodes_0\subset [\atomorder]} 
		\left( - \ln \probof{\indexedcatvariables} \right) \cdot \onesat{\shortcatvariables} \right) \\
		& \quad + 
		\left(  \sum_{\nodes_0\subset [\atomorder]} 
		\left(  \ln \probof{\indexedcatvariables} \right) \cdot 
		\onehotmapofat{\catindices}{\shortcatvariables} 
		\right) 
		 \\
		 & = \ln\probof{\shortcatvariables} + \lambda \cdot  \onesat{\shortcatvariables}\,,
	\end{align*}
	where $\lambda$ is a constant.
\end{proof}

% Redundant parametrization
In general, this representation is redundant, since any offset of the weight by $\lambda\cdot\ones$ results in the same distribution.
However, the only $\bar{\canparam}$ are multiples of $\onesat{\shortcatvariables}$.

% Comparison with previous schemes
Theorem~\ref{the:maximalClausesRepresentation} is the analogue in Markov Logic to Theorem~\ref{the:tensorToMaxMinTerms}, which shows that any binary tensor has a representation by a logical formula, to probability tensors.
Here we require positive distributions for well-defined energy tensors.


\begin{remark}[Representation of Markov Networks]
% Composition of Markov Networks
	If a probability distribution is representable as a Markov Network, we only need to activate clauses and terms, which variables are contained in factors.
	\red{Make a theorem out of that?}
\end{remark}

%\subsection{Markov Logical Networks as Graphical Models}









\subsection{Examples}


\subsubsection{Distribution of independent variables}

We show next, the independent positive distributions are representable by tuning the $\atomorder$ weights of the atomic formulas and keeping all other weights zero.

\begin{theorem}\label{the:independentAtomicMLN}
	Let $\probat{\shortcatvariables}$ be a positive probability distribution, such that disjoint subsets of atoms are independent from each other.
	Then $\probat{\shortcatvariables}$ is the Markov Logic Network of atomic formulas
		\[ \atomformulaset = \{\atomicformulaof{\catenumerator} \, : \, \catenumeratorin \} \]
	and parameters
		\[ \canparamat{\selvariable=\catenumerator} 
		= \lnof{\frac{
		\contractionof{\probtensor}{\catvariableof{\catenumerator}=1}
		}{
		\contractionof{\probtensor}{\catvariableof{\catenumerator}=0}
		}} \]
%	Any distribution such that the atom satisfaction is independent from each other is reproducable by a MLN with nonzero weights only for the atomic formulas.
\end{theorem}
\begin{proof}
%	Using the independent assumptions, the probability tensor factorizes into normed vectors to each atom, with are transformed atomic formulas (leaving out the neutral ones tensors).
%	We then find a weight to each atom such that the vector is reproduced by the contraction with the activation core.
	
	By Theorem~\ref{the:independenceProductCriterion} we get a decomposition 
		\[ \probat{\shortcatvariables} = \bigotimes_{\catenumeratorin} \probofat{\catenumerator}{\catvariableof{\catenumerator}} \,  \]
	where 
		\[ \probofat{\catenumerator}{\catvariableof{\catenumerator}} = \sbcontractionof{\probtensor}{\catvariableof{\catenumerator}} \, . \]
	
	By assumption of positivity, the vector $\probofat{\catenumerator}{\catvariableof{\catenumerator}}$ is positive for each $\catenumeratorin$ and the parameter
		%\[ \canparam^\catenumerator = \lnof{\frac{\probofat{\catenumerator}{\catvariableof{\catenumerator}=1}}{\probofat{\catenumerator}{\catvariableof{\catenumerator}=0}}} \]
		\[ \canparam^\catenumerator 
		= \lnof{\frac{
		\probofat{\catenumerator}{\catvariableof{\catenumerator}=1}
		}{
		\probofat{\catenumerator}{\catvariableof{\catenumerator}=0}
		}} \]
	well-defined.
	
	We then notice, that 
		\[ \expdistofat{(\{\atomicformulaof{\catenumerator}\},\canparam^{\catenumerator})}{\catvariableof{\catenumerator}} 
		= \probofat{\catenumerator}{\catvariableof{\catenumerator}}\]
	and therefore with the parameter vector of dimension $\seldim=\catorder$ defined as
		\[ \canparamat{\selvariable} = \sum_{\catenumeratorin} \canparam^{\catenumerator} \cdot \onehotmapofat{\catenumerator}{\selvariable}  \]
	we have
	\begin{align*}
	 	 \expdistofat{(\{\atomicformulaof{\catenumerator} \, : \, \catenumeratorin\},\canparam)}{\shortcatvariables} 
		& = \bigotimes_{\catenumeratorin} \expdistofat{(\{\atomicformulaof{\catenumerator}\},\canparam^{\catenumerator})}{\catvariableof{\catenumerator}} \\
		& = \bigotimes_{\catenumeratorin} \probofat{\catenumerator}{\catvariableof{\catenumerator}} \\
		& = \probat{\shortcatvariables} \, . 
	\end{align*}
\end{proof}

%In general, the statistic to an atomic formula measures the marginal distribution. -> To Parameter Estimation

% Failing to be positive -> Hybrid networks
In Theorem~\ref{the:independentAtomicMLN} we made the assumption of positive distributions.
If the distribution fails to be positive, we still get a decomposition into distributions of each variable, but there is at least one factor failing to be positive.
Such factors need to be treated by hybrid logic networks, that is they are base measure for an exponential family coinciding with a logical literal (see Chapter~\ref{cha:hardNetworks}.

% Energy representation
All atomic formulas can be selected by a single variable selecting tensor, that is
	\[ \energytensorofat{(\{\atomicformulaof{\catenumerator} \, : \, \catenumeratorin\},\canparam)}{\shortcatvariables}
	= \sbcontractionof{\vselectionmapat{\shortcatvariables,\selvariable},\canparamat{\selvariable}}{\shortcatvariables} \, . 
	\]
	
% Holds also more generally for any formula! -> Place it earlier?
In case of negative $\canparamat{\catenumerator}$ it is convenient to replace $\atomicformulaof{\catenumerator}$ by $\lnot\atomicformulaof{\catenumerator}$, in order to facilitate the interpretation.
The probability distribution is left invariant, when also replacing $\canparamat{\catenumerator}$ by $-\canparamat{\catenumerator}$.



\subsubsection{Boltzmann machines as MLNs}

%\red{Add sufficient statistics?}

A Boltzmann machine is a member of an exponential family with the energy tensor
	\[ \energytensorofat{W,b}{\indexedcatvariables} = 
	\sum_{\atomenumerator,\secatomenumerator \in [\atomorder]} 
		W[\selvariableof{\vselectionsymbol,0}=\atomenumerator, \selvariableof{\vselectionsymbol,1}=\secatomenumerator] \catindexof{\atomenumerator} \catindexof{\secatomenumerator} 
	+ \sum_{\atomenumerator,\secatomenumerator \in [\atomorder]} b[\selvariableof{\vselectionsymbol,0}=\atomenumerator] \, . \]


%sufficient statistic 
%	\[ \sstat : \atomstates \rightarrow (\rr^{\catorder}\otimes \rr^{\catorder}) \times \rr^{\catorder} \]
%by interaction term
%	\[ \sstat(\shortcatindices) = (\catindexof{\atomenumerator} \Leftrightarrow \catindexof{\secatomenumerator})_{\atomenumerator,\secatomenumerator \in[\atomorder]} \]
%and by potential term
%	\[ \sstat(\shortcatindices) =  (\catvariableof{\atomenumerator})_{\atomenumeratorin} \, . \]

We notice, that this coincides with the energy tensor of a Markov Logic Network with formula set 
	\[ \formulaset = \{ \catvariableof{\atomenumerator} \Leftrightarrow \catvariableof{\secatomenumerator} \, : \, \atomenumerator,\secatomenumerator \in[\atomorder] \} 
	\cup \{ \catvariableof{\atomenumerator}\, : \, \atomenumeratorin \} \, \]
with cardinality $\atomorder^2+\atomorder$.

Each formula is in the expressivity of an architecture consisting of a single binary logical neuron selecting any variable of $\shortcatvariables$ in each argument and selecting connectives $\{\eqbincon,\lpasbincon\}$, where by $\lpasbincon$ we refer to a connective passing the first argument, defined for $\catindexofin{0}, \catindexofin{1}$ as 
	\[ \lpasbincon[\indexedcatvariableof{0},\indexedcatvariableof{1}] = \vselectionmapat{\indexedcatvariableof{0},\catvariableof{1},\selvariableof{\vselectionsymbol}=0} \, . \]

The weight is
	\[ \canparam 
	= \onehotmapofat{0}{\selvariableof{\cselectionsymbol}} \otimes W 
	+ \onehotmapofat{1}{\selvariableof{\cselectionsymbol}} \otimes b[\selvariableof{\vselectionsymbol,0}] \otimes  \onehotmapofat{0}{\selvariableof{\vselectionsymbol,0}} 
	\]
	
And we have
	\[ \energytensorofat{W,b}{\shortcatvariables} = 
	\sbcontractionof{\fsnnat{\shortcatvariables,\selvariableof{\cselectionsymbol},\selvariableof{\vselectionsymbol,0},\selvariableof{\vselectionsymbol,1}}, \canparamat{\selvariableof{\cselectionsymbol},\selvariableof{\vselectionsymbol,0},\selvariableof{\vselectionsymbol,1}}}{\shortcatvariables} \, . \]


\begin{figure}[h]
\begin{center}
	\input{PartII/tikz_pics/mln/boltzmann_energy.tex}
\end{center}
\caption{Tensor network representation of the energy of a Boltzmann machine}
\label{fig:boltzmannEnergy}
\end{figure}


%where by $(\cdot,\cdot)|_{0}$
%To connect with the formalism of Boltzmann machines, let us identify the visible units of a Boltzmann machines with the atoms in a propositional theory.

%Boltzmann machines are then reproduced by taking $\atomorder^2+\atomorder$ many formulas, namely those measuring the correlations and the marginal distributions.
%To be more precise, the correlation between atom $\atomicformulaof{\atomenumerator}$ and $\atomicformulaof{\secatomenumerator}$ is measured by the satisfaction rate of the formula 
%	\[ \exformula_{\atomenumerator,\secatomenumerator} = \atomicformulaof{\atomenumerator} \leftrightarrow \atomicformulaof{\secatomenumerator}\]

%\begin{theorem}
%	Any Boltzmann machine over $\atomorder$ units with interaction matrix $U\in\rr^{\atomorder\times\atomorder}$ and potential term $b\in\rr^{\atomorder}$ (MacKay Book notation) is a MLN where the only nonzero weights are 
%		\[ \weightof{\atomicformulaof{\atomenumerator} } = b_{\atomenumerator} \quad, \quad \atomenumeratorin \]
%	and 
%		\[ \weightof{ \exformula_{\atomenumerator,\secatomenumerator} } = U_{\atomenumerator, \secatomenumerator} \quad , \quad \atomenumerator,\secatomenumerator \in [\atomorder]\] 
%\end{theorem}

\red{
Often Boltzmann machines are formulated with hidden variables.
To average those out, one needs to instantiate the probability distribution instead of the energy tensor and leave only visible variables open in a contraction.
}


Markov Logic Networks go beyond the Boltzmann machines already for binary formulas, by the flexibility to capture further dependencies beyond the correlation.
We can use any binary logical connective and have an associated formula where we can put a weight on.



%\begin{remark}[Hopfield networks]
%	Also interesting for MLNs is a Hopfield perspective.
%	Having an initialization the update can be interpreted as a Gibbs sampling step at temperature $0$ (since deterministic update).
%\end{remark}


%\begin{remark}[Representation by Formula Selecting Networks]
%	When choosing an architecture with a single neurons selecting on both arguments any atom and having the logical biconditional as fixed connective.
%	The sufficient statistics of the empirical distribution is the correlation matrix between the atoms.
%\end{remark}

%
%\subsubsection{MLNs as higher-order deep Boltzmann Machines}
%
%\red{Unclear whether this makes sense: Hidden nodes depend deterministically on the observed!}
%
%We understand 
%\begin{itemize}
%	\item Atoms as observed nodes (might be redundant, since they are formulas themselves)
%	\item Formulas (including the atomic formulas) as hidden nodes
%\end{itemize}
%
%Non-atomic formulas have a hard coded deterministic dependency on the atomic formulas.
%
%The MLN distribution is given by associating a potential to some hidden nodes.








\subsection{Applications}

Markov Logic Networks as neuro-symbolic architectures:
\begin{itemize}
	\item Neural Paradigm here by decompositions of logical formulas into their connectives.
		In more generality by decompositions of sufficient statistics into composed functions, using Basis Calculus.
		Deeper nodes as carrying correlations of lower nodes.
	\item Symbolic Paradigm by interpretability of propositional logics.
\end{itemize}


Markov Logic Networks as trainable Machine Learning models:
\begin{itemize}
	\item Expressivity: Can represent any distributions, as shown by Theorem~\ref{the:maximalClausesRepresentation}, with $2^d$ formulas.
	\item Efficiency: Can only handle small subsets of possible formulas, since their possible number is huge.
		Tensor networks provide means to efficiently represent formulas depending on many variables and reason based on contractions.
	\item Differentiability: Distributions are differentiable functions of their weights, see Parameter Estimation Chapter. 
		The log-likelihood of data is therefore also differentiable function of their weights and we can exploit first-order methods in their optimization.
	\item Structure Learning: We need to find differentiable parametrizations of logical formulas respecting a chosen architecture.
		In Chapter~\ref{cha:formulaBatches} such representations are described based on Selector Tensor Networks.
\end{itemize}





%\subsection{Symbolic Paradigm by Logics}
%Logical formulas (which are maps from $\atomstates$ to $[2]$ as explained before) correspond with a function to be represented by a neural network.

%\subsubsection{Neural Paradigm by Formula Decompositions}

%% Decompositions
%In logics, formula can be decomposed into logical connectives acting on smaller formulas as has been shown in Chapter~\ref{cha:FormulaTensors}.







\section{Hard and Hybrid Logic Networks}\label{cha:hardNetworks}

\red{Hard Logic Networks are Knowledge Bases, Hybrid Logic Networks are exponential families on Knowledge Bases.
This makes it impossible to build energy tensors without basemeasures capturing vanishing coordinates.}



\red{Work in Theorem~\ref{the:factorReduction} to reduce entailment!}

% Hard logic vs markov logic
While exponential families are positive distributions, in logics probability distributions can assign states zero probability.
As a consequence, Markov Logic Networks have a soft logic interpretation in the sense that violation of activated formulas have nonzero probability.
We here discuss their hard logic counterparts, where worlds not satisfying activated formulas have zero probability.

Further we investigate, how both hard and soft logic factors can be combined to hybrid networks.

%The Tensor Network decomposition of formula tensors is analogously constructed to a graphical representation of the formulas.
%We thus develop in this section the interpretation of formula tensor decompositions as Bayesian and Markov Propositional Networks.



\subsection{The limit of hard logic}\label{sec:hardLogicLimit} % To be merged with the above

The probability function of Markov Logic Networks with positive weights mimiks the tensor network representation of the knowledge base, which is the conjunction of the formulas. 
The maxima of the probability function coincide with the models of the corresponding knowledge base, if the latter is satisfiable.
However, since the Markov Logic Network is defined as a normed exponentiation of the weighted formula sum, it is a positive distribution whereas uniform distributions among the models of a knowledge base assign zero probability to world failing to be a model.
Since both distributions are tensors in the same space to a factored system, we can take the limits of large weights and observe, that Markov Logic Networks indeed converge to normed knowledge bases.


% Limit of Activation core
\begin{lemma}
	When taking the limit of large weights $\weightof{\exformula}\rightarrow\infty$ we observe a coordinatewise (in the sense of a convergence of each coordinate of the tensor) convergence 
	\begin{align}
%	\frac{1}{\partitionfunctionof{\weightof{\exformula}\exformula}} \expof{\weightof{\exformula}\exformula} 
		\normationofwrt{\expof{\weightof{\exformula}\cdot \exformula}}{\shortcatvariables}{\varnothing} \rightarrow  \normationofwrt{\exformula}{\shortcatvariables}{\varnothing}
%	\frac{1}{\braket{\exformula,\ones}}\exformula
	\end{align}
\end{lemma}
\begin{proof}
	We have 
	\begin{align*}
		\partitionfunctionof{\mlnparameters} = (\prod_{\atomenumerator} \catdimof{\atomenumerator} - \contractionof{\exformula}{\varnothing}) + \contractionof{\exformula}{\varnothing} \cdot \expof{\weightof{\exformula}}
	\end{align*}
	and therefore for any $\atomindices\in\atomstates$ with $\exformula(\atomindices)=1$
	\begin{align*}
		\normationofwrt{\expof{\weightof{\exformula}\cdot \exformula}}{\indexedcatvariables}{\varnothing} 
		&= \frac{
			\expof{\weightof{\exformula}}
			}{
			(\prod_{\atomenumerator} \catdimof{\atomenumerator} - \contractionof{\exformula}{\varnothing}) + \contractionof{\exformula}{\varnothing} \cdot \expof{\weightof{\exformula}}
			} \\
		& \rightarrow \frac{1}{\contractionof{\exformula}{\varnothing}} 
		= \normationofwrt{\exformula}{\indexedcatvariables}{\varnothing} \, . 
	\end{align*}
	For any $\atomindices\in\atomstates$ with $\exformula(\atomindices)=0$ we have on the other side
	\begin{align*}
		\normationofwrt{\expof{\weightof{\exformula}\cdot \exformula}}{\indexedcatvariables}{\varnothing} 
		&= \frac{
			1
			}{
			(\prod_{\atomenumerator} \catdimof{\atomenumerator} - \contractionof{\exformula}{\varnothing}) + \contractionof{\exformula}{\varnothing} \cdot \expof{\weightof{\exformula}}
			} \\
		& \rightarrow 0
		= \normationofwrt{\exformula}{\indexedcatvariables}{\varnothing} \, . 
	\end{align*}
\end{proof}

\begin{theorem}
	Let $\formulaset$ be a formulaset and $\canparam$ a positive parameter vector.
	If the formula
		\[ \kb = \bigwedge_{\exformulain} \exformula \]
	is satisfiable we have in the limit $\invtemp\rightarrow\infty$ the coordinatewise convergence
		\[ \expdistofat{(\formulaset,\invtemp\cdot\canparam)}{\shortcatvariables} \rightarrow \normationofwrt{\kb}{\shortcatvariables} \, . \]
\end{theorem}
\begin{proof}
	Since $\kb$ is satisfiable we find $\catindices\in\atomstates$ with
		\[  \contractionof{\expof{\sum_{\exformulain}\invtemp\cdot \weightof{\exformula} \cdot \exformula}}{\indexedcatvariables} = \expof{\invtemp \cdot \sum_{\exformulain}\weightof{\exformula}}  \]
	and the partition function obeys
		\[ \contractionof{\expof{\sum_{\exformulain}\invtemp\cdot \weightof{\exformula} \cdot \exformula}}{\varnothing} \geq  \expof{\invtemp \cdot \sum_{\exformulain}\weightof{\exformula}}  \, . \]
	For any state $\catindices\in\atomstates$ with $\kb(\catindices)=0$ we find $\secexformula\in\formulaset$ with $\secexformula(\catindices)=0$ and have
	\begin{align*}
	 	\frac{
		\contractionof{\expof{\sum_{\exformulain}\invtemp\cdot \weightof{\exformula} \cdot \exformula}}{\indexedcatvariables}
		}{
		\contractionof{\expof{\sum_{\exformulain}\invtemp\cdot \weightof{\exformula} \cdot \exformula}}{\varnothing}
		} 
		\leq  
	 	\frac{
		\expof{\invtemp\cdot \sum_{\exformulain : \exformula\neq \secexformula}\weightof{\exformula}}
		}{
		\expof{\invtemp\cdot \sum_{\exformulain}\weightof{\exformula}}
		} 
		= \expof{\invtemp \cdot \weightof{\secexformula}} \rightarrow 0 \, . 
	\end{align*}
	The limit of the distribution has thus support only on the models of $\kb$. 
	Since any model of $\kb$ has same energy at any $\invtemp$ the limit is a uniform distribution and coincides therefor with
		\[ \normationofwrt{\kb}{\shortcatvariables} \, . \]
\end{proof}


\begin{remark}[More generic situation of simulated annealing]
	The process of taking $\invtemp\rightarrow\infty$ is known as simulated annealing, see Chapter~\ref{cha:probReasoning}.
	From the discussion there we have the more general statement, that the limiting distribution is the uniform distribution among the maxima of $\expdistofat{(\formulaset,\canparam)}{\shortcatvariables}$.
	If the formula $\kb$ is not satisfiable the normation $\normationofwrt{\kb}{\shortcatvariables}{\varnothing}$ does not exist and the limit distribution has another syntactical representation, to be gained e.g. by minterm or maxterm representation (see Theorem~\ref{the:tensorToMaxMinTerms}).
\end{remark}





%To make this convergence precise, we define the uniform distribution 
%\begin{align}
%	\expdistofat{\kb}{\datapoint}
%	= \begin{cases} 
%	\frac{1}{\braket{\ftensorof{\kb},\ones}} & \text{if } \braket{\ftensorof{\kb},\datapoint} = 1 \\
%	0 &  \text{if } \braket{\ftensorof{\kb},\datapoint} = 0
%	\end{cases}
%\end{align}

%\begin{theorem}
%	Given a MLN parameterized by $\mlnparameters$, we have for $\lambda\rightarrow\infty$
%		\[ \kldivof{\expdistofat{(\formulaset,\lambda\cdot\weight)}{\datapoint}}{\expdistofat{\kb}{\datapoint}} \rightarrow 0 \, .\]
%\end{theorem}
%\begin{proof}
%	Follows directly from the convergence at each core.
%\end{proof}














\subsection{Hard Logic Networks}

Hard Logic Network coincide with Knowledge Bases.
We use $\land$ symmetry to represent them as a contraction of the formulas building the Knowledge Base as conjunction.

\begin{theorem}[Conjunction Decomposition of Knowledge Bases]\label{the:conDecKB}
	For a Knowledge Base
		\[ \kb = \bigwedge_{\exformula\in\formulaset} \exformula \]
	we have
		\[ \kbat{\shortcatvariables} = \contractionof{\formulaat{\shortcatvariables}}{\shortcatvariables}   \]
	and
		\[ \kbat{\shortcatvariables} = \contractionof{\{\rencodingofat{\exformula}{\catvariableof{\exformula},\shortcatvariables} \, : \, \exformula\in\formulaset\} \cup \{\onehotmapofat{1}{\catvariableof{\exformula}} \, : \, \exformula\in\formulaset\} }{\shortcatvariables} \, .  \]
\end{theorem}
\begin{proof}
	By the $\land$-symmetry, see effective calculus and 
		\[ \formulaat{\shortcatvariables} =  \contractionof{\{\rencodingofat{\exformula}{\catvariableof{\exformula},\shortcatvariables}, \onehotmapofat{1}{\catvariableof{\exformula}}\} }{\shortcatvariables} \]
\end{proof}

\begin{remark}{$\land$ symmetry does not generalize to Markov Logic Networks}
	% Comparison to Markov Logic
	In Markov Logic, similar decompositions are not possible.
	For example, consider a MLN with a single formula $\atomicformulaof{0}\land\atomicformulaof{1}$ and nonvanishing weight $\canparam$.
	This does not coincide with the distribution of a MLN of two formulas $\atomicformulaof{0}$ and $\atomicformulaof{1}$.
	To see this, we notice that with respect to the distribution of the first MLN, both variables are not independent, while for any MLN constructed by the two atomic formulas they are.
\end{remark}


%It is known, that there are symmetries in the syntactical represention of Knowledge Bases.
%
%There is a lot of redundancy in the activation of Knowledge Cores describing exactly the same models.
%
%\begin{theorem}[$\land$-symmetry]\label{the:landSymmetry}
%	We observe that the contraction of an $\land$ core with $\tbasis$  is equivalent with $\tbasis$ cores on all the connected subformulas.
%\end{theorem}
%\begin{proof}
%	By equality of the Knowledge Base contraction in both ways: The missing subformulas behave the same if they are activated, since they then are contrained to the same subnetworks somewhere else. 
%	%\red{Find better arguments for missing subformulas when having the larger core.}
%\end{proof}
%
%\begin{theorem}[$\lnot$-symmetry]
%	Similarly the contraction of an $\lnot$ core with $\tbasis$ or $\tbasis$ has the same result as with $\tbasis$ or $\tbasis$ on the subformula.
%\end{theorem}
%
%We call the application of these in changing the Knowledge Cores without changing the contracted network as the representation symmetry.


\subsubsection{Conjunctive Normal Representation}

\red{Poly reps here?}

We can now apply the representation symmetries to represent a propositional Knowledge Base in conjunctive normal form.
A Knowledge Base in Conjunctive Normal Form is a conjunction of clauses, where clauses are disjunctions of literals being atoms (positive literals) or negated atoms (negative literals).




%One tensor representation of a Knowledge Base is the association of the Knowledge Core $\tbasis$ at the formula being the Knowledge Base itself.
%We can use the $\land$ symmetry (Theorem~\ref{the:landSymmetry}) to propagate $\tbasis$ to all clause cores and get an alternative representation.
%Those are especially interesting when using Modus Ponens/Resolution as local sub-KB reasoners (see Section~\ref{subsec:LocalEntailment}).


\subsubsection{Polynomial Representation of Formulas}

\red{We can further derive representation schemes for Knowledge Bases, which are in conjunctive normal forms.}

Formulas can be represented as sparse polynomials, which will be discussed in more detail in Chapter~\ref{cha:sparseTC} (see Definition~\ref{def:polynomialSparsity}).

\begin{lemma}\label{lem:clauseTermBasisPlus}
	Any term is representable by a single monomial and any clause is representable by at most two monomials. %, any term of basis+ with rank 1. %Use also \baspluscprankof{}
\end{lemma}
\begin{proof}
	Let $\nodes_0$ and $\nodes_1$ be disjoint subsets of $\nodes$, then we have
	\begin{align*}
		\termof{\nodes_0}{\nodes_1} = \onehotmapofat{
			\{\catindexof{\atomenumerator} = 0 : \atomenumerator\in\nodes_0\} \cup \{\catindexof{\atomenumerator} = 1 : \atomenumerator\in\nodes_1\}
		}{\catvariableof{\nodes_0\cup\nodes_1}} \otimes \onesat{\catvariableof{\nodes/(\nodes_0\cup\nodes_1)}}
	\end{align*}
	and
	\begin{align*}
		\clauseof{\nodes_0}{\nodes_1} = \onesat{\catvariableof{\nodes}} - \onehotmapofat{
			\{\catindexof{\atomenumerator} = 0 : \atomenumerator\in\nodes_0\} \cup \{\catindexof{\atomenumerator} = 1 : \atomenumerator\in\nodes_1\}
		}{\catvariableof{\nodes_0\cup\nodes_1}}
		\otimes \onesat{\catvariableof{\nodes/(\nodes_0\cup\nodes_1)}} \, . 
	\end{align*}
	We notice, that any tensors $\ones$ and $\onehotmapof{\catindex}\otimes \ones$ habe basis+-rank of $1$ and therefore $\termof{\nodes_0}{\nodes_1}$ of $1$ and $\clauseof{\nodes_0}{\nodes_1}$ of at most $2$.
\end{proof}


We apply Lemma~\ref{lem:clauseTermBasisPlus} to show the following sparsity bound on the energy tensor of Markov Logic Networks.

\begin{theorem}
	Any formula $\exformula$ with a conjunctive normal form of $n$ clauses satisfies
		\[ \slicesparsityof{\exformula} \leq 2^{n} \, . \]
	For any set $\formulaset$ of formulas each with a conjunctive normal form of $n_{\exformula}$ clauses satisfies for any $\weight$
		\[ \slicesparsityof{\sum_{\exformulain}\weightof{\exformula}\cdot\exformula} \leq \sum_{\exformulain}2^{n_{\exformula}} \, . \]
\end{theorem}
\begin{proof}
	Let $\exformula$ have a CNF with clauses indexed by $l\in[n]$ and each clause represented by subsets $\nodes_0^l, \nodes_1^l$, that is
		\[ \exformula = \bigwedge_{l \in [n]}  \clauseof{\nodes_0^l}{\nodes_1^l} \, . \]
	We now use the rank bound of Theorem~\ref{the:CPrankContractionBound} and Lemma~\ref{lem:clauseTermBasisPlus} to get
	\begin{align*}
		\slicesparsityof{\exformula} \leq \prod_{l \in [n]}  \slicesparsityof{\clauseof{\nodes_0^l}{\nodes_1^l}} \leq 2^n \, . 
	\end{align*}
	
	Given a collection of formulas $\formulaset$, each with a CNF of $n_{\exformula}$ clauses we apply Theorem~\ref{the:CPrankSumBound} and get
	\begin{align*}
		\slicesparsityof{\sum_{\exformulain}\weightof{\exformula}\cdot\exformula} \leq \sum_{\exformulain}\slicesparsityof{\exformula} \leq \sum_{\exformulain}2^{n_{\exformula}} \, . 
	\end{align*}
\end{proof}






\subsection{Hybrid Logic Network}

Markov Logic Networks are by definition positive distributions.
In contrary, Hard Logic Networks model uniform distributions over model sets of the respective Knowledge Base and therefore have vanishing coordinates.
We now show how to combine both approaches by defining Hybrid Logic Networks.
\red{
We orient on Example 3.6 in \cite{wainwright_graphical_2008} and choose hard constraints as a base measure $\hfbasemeasure$, probabilistic soft formulas as sufficient statistics as before.
}

\begin{definition}
	Given a set of formulas $\softformulaset$ with weights $\canparam$ and set $\hardformulaset$ of formulas, which conjunction is satisfiable, the hybrid logic network is the probability distribution
	\begin{align*}
		\probtensorof{(\softformulaset,\canparam,\hfbasemeasure)}[\shortcatvariables] 
		= \normationof{
		\{\exformula : \exformula\in\hardformulaset\} \cup \{\expof{\weightof{\exformula}\cdot\exformula} : \exformula\in\softformulaset\}
		}{\shortcatvariables} \, ,
	\end{align*}
	which is the member of the exponential family with statistic by $\softformulaset$ and the base measure
		\[ \hfbasemeasure[\shortcatvariables] = \contractionof{\{\formula : \formula \in \hardformulaset\}}{\shortcatvariables} \, .\]
\end{definition}

The assumption of a satisfiable set $\hardformulaset$ is necessary, as we show next.

\begin{theorem}
	If any only if $\bigwedge_{\formula\in\hardformulaset}\formula$ is satisfiable, the tensor 
		\[  \contractionof{
		\{\exformula : \exformula\in\hardformulaset\} \cup \{\expof{\weightof{\exformula}\cdot\exformula} : \exformula\in\softformulaset\}
		}{\shortcatvariables} \]
	is normable.
\end{theorem}
\begin{proof}
	We need to show that
	\begin{align}\label{eq:tbsWellDefinedHLN}
		\contraction{\{\exformula : \exformula\in\hardformulaset\} \cup \{\expof{\weightof{\exformula}\cdot\exformula} : \exformula\in\softformulaset\}} > 0 \, . 
	\end{align}
	Since the conjunction of $\hardformulaset$ is satisfiable we find a $\shortcatindices$ with $\formulaat{\indexedcatvariableof{[\catorder]}}=1$ for all $\exformula\in\hardformulaset$.
	Then 
	\begin{align*}
		 \contractionof{\{\exformula : \exformula\in\hardformulaset\} \cup \{\expof{\weightof{\exformula}\cdot\exformula} : \exformula\in\softformulaset\}}{\indexedcatvariableof{[\catorder]}}  
		 & = \left( \prod_{\exformula\in\hardformulaset}\formulaat{\indexedcatvariableof{[\catorder]}} \right) 
		 \cdot \left( \prod_{\exformula\in\softformulaset}\expof{\weightof{\exformula}\cdot\exformula}[\indexedcatvariableof{[\catorder]}] \right) \\
		 & =  \left( \prod_{\exformula\in\softformulaset}\expof{\weightof{\exformula}\cdot\exformula}[\indexedcatvariableof{[\catorder]}] \right) \\
		 & > 0 \, . 
	\end{align*}
	Condition \eqref{eq:tbsWellDefinedHLN} follows from this and the Hybrid Logic Network is well-defined.
\end{proof}


%\subsubsection{Representation as Exponential Families}

%We call graphical models which contain cores from a Markov Logik Network and of a Hard Logic Network a Hybrid Logic Network.

%Hybrid logic networks are exponential family, where the hard logic factors build a base measure and the probabilistic logic components a contracted statistics function.
%\begin{theorem}
%	A hybrid logic network $\probtensorof{(\softformulaset,\canparam,\hardformulaset)}$ is in the exponential family with base measure
%		\[ \hfbasemeasure[\shortcatvariables] = \contractionof{\{\formula : \formula \in \hardformulaset\}}{\shortcatvariables} \] % Do not need a normed base measure!
%	and sufficient statistic $\softformulaset$ the element with parameters $\canparam$.
%\end{theorem}

%
%By a slight abuse of notation, we denote by $\hardformulaset$ both the set of propositional formulas and their contractions.



\subsubsection{Tensor Network Representation}


We can employ the formula decompositions to represent both probabilistic facts of the MLN and hard facts (seen as the limit of large weights).

\begin{theorem}\label{the:hybridNetworkRepresentation}
	For any hybrid logic network we have
	\begin{align*}
		\probtensorof{(\softformulaset,\canparam,\hardformulaset)}[\shortcatvariables] 
		= \normationof{
		\{\rencodingofat{\exformula}{\catvariableof{\exformula},\shortcatvariables} : \exformula\in\softformulaset\cup\hardformulaset \}
		\cup \{\onehotmapofat{1}{\catvariableof{\exformula}} : \exformula\in\hardformulaset \}
		\cup \{\headcoreofat{\exformula}{\catvariableof{\exformula}} : \exformula\in\softformulaset \}
		}{\shortcatvariables} \, . 
	\end{align*}
\end{theorem}
\begin{proof}
	By Lemma~\ref{lem:formulaEncodingDecomposition}.
\end{proof}

%% Overwork: Allow for infinite weights?
%Capturing hard and soft constraints at the same time, we can use a weight to each formula:
%\begin{itemize}
%	\item $\weightof{\exformula}=0$: The formula is neutral and does not influence the probability distribution.
%	Techniacally, the formula tensor is contracted with a $\ones$ head with the result being a $\ones$ world tensor, which leaves other products invariant under Hadamard products.
%	\item $\weightof{\exformula}=\infty$: The formula is a hard constraint. 
%	\item $\weightof{\exformula} \in (0,\infty)$: The fromula is a probabilistic constraint.
%\end{itemize}



%The reason for this is the Slicing Theorem, enabling the operations by both (exponentiation and selection of one slice) by the head cores.
For an example see Figure~\ref{fig:ActivatedHeads}.

\begin{figure}[h]
\begin{center}
	\input{PartII/tikz_pics/hybrid_networks/activated_heads.tex}
\end{center}
\caption{Diagram of a formula tensor with activated heads, containing \textcolor{\concolor}{hard constraint cores} and \textcolor{\probcolor}{probabilistic weight cores} .} %along \textcolor{\inactivecolor}{inactive cores}.}
\label{fig:ActivatedHeads} 
\end{figure}



\begin{remark}{Probability interpretation using the Partition function}
	The tensor networks here represent unnormalized probability distributions.
	The probability distribution can be normed by the quotient with the naive contraction of the network, the partition function.
\end{remark}


\subsubsection{Reasoning Properties}



\begin{theorem}
	Let $(\softformulaset,\canparam,\hardformulaset)$ define a Hybrid Logic Network.
	Given a query formula $\exformula$ we have that 
		\[ \probtensorof{(\softformulaset,\canparam,\hardformulaset)} \models \exformula \]
	if and only if
		\[ \hardformulaset \models \exformula \, . \]
\end{theorem}
\begin{proof}
	Application of Theorem~\ref{the:factorReduction} on the representation of Hybrid Logic Networks as Markov Networks in Theorem~\ref{the:hybridNetworkRepresentation}.
\end{proof}


%% Now in theorems
%\begin{itemize}
%	\item Entailment queries answered on the hard logic parts alone.
%	\item Well defined distributions, when hard logic formulas are satisfiable.
%	\item Redundant hard formulas (Redundancy, whenever the contractions unchanged): If entailled by the rest of the hard logic formulas. 
%	\item Redundant soft formulas (Redundancy, whenever the normations unchanged):  Either if entailled or contradicted by the hard logic formulas
%\end{itemize}



Formulas in $\softformulaset$, which are entailed or contradicted by $\hardformulaset$ are redundant 

\begin{theorem}
	If for a formula $\exformula$ and $\hardformulaset$ we have  %and only if
		\[ \hardformulaset \models \exformula \, \quad \text{or} \quad \hardformulaset \models \lnot\exformula \]
	then for any $(\softformulaset,\canparam,\hardformulaset)$
		\[ \probofat{(\softformulaset/\{\exformula\},\tilde{\canparam},\hardformulaset)}{\shortcatvariables} =  \probofat{(\softformulaset,\canparam,\hardformulaset)}{\shortcatvariables}  \, , \]
	where $\tilde{\canparam}$ denotes the tensor $\canparam$, where the coordinate to $\exformula$ is dropped, if $\exformula\in\softformulaset$.
\end{theorem}
\begin{proof}
	Isolate the factor to the hard formula, which is constant for all situations.
\end{proof}

%% Now in the 
A similar statement holds for the hard formulas itself, as shown in Theorem~\ref{the:ReduncancyOfEntailed}.
However, notice that if $\hardformulaset/\{\exformula\}\models\lnot\exformula$, then $\hardformulaset\cup\{\exformula\}$ is not satisfiable and a hybrid logic network cannot be defined for $\hardformulaset\cup\{\exformula\}$ as hard logic formulas.

%If the conjunction of $\hardformulaset/\{\exformula\}$ entails $\exformula$, we can erase $\exformula$ from $\hardformulaset$ without changing the contraction, therefore without changing the base measure of the Hybrid Logic Network.



\subsubsection{Expressivity}

Hybrid Logic Networks extend the expressivity result of Theorem~\ref{the:mintermExpressivityMLN} to arbitrary probability tensors, dropping the positivity constraints for Markov Logic Networks.

\begin{theorem}\label{the:mintermExpressivityHLN}
	Let $\probat{\shortcatvariables}$ a possibly not positive probability tensor we build a base measure
		\[ \hfbasemeasure = \nonzeroof{\probat{\shortcatvariables}} \]
	and a parameter tensor
	\begin{align*}
		\canparamat{\selvariableof{[\catorder]}=\shortcatindices}
		= \begin{cases}
			0 & \text{if} \quad \probat{\shortcatvariables=\shortcatindices} = 0  \\
			\lnof{\probat{\shortcatvariables=\shortcatindices}} & \text{else} 
		\end{cases} \, . 
	\end{align*}
	Then the probability tensor is the member of the minterm exponential family with base measure $\hardformulaset$ and parameter $\canparam$, that is
		\[ \probof{(\mintermformulaset,\canparam,\hfbasemeasure)}\]
\end{theorem}
\begin{proof}
	It suffices to show that 
		\[ \sbcontractionof{\hfbasemeasure, \expof{\contractionof{
		\sencodingof{\mintermformulaset}\canparam
		}{
		\shortcatvariables
		}}}{\shortcatvariables} = \probat{\shortcatvariables} \, . \]
	For indices $\shortcatindices$ with $\probat{\shortcatvariables=\shortcatindices}=0$ we have $\hfbasemeasureat{\shortcatvariables=\shortcatindices}=0$ and thus also 
		\[ \sbcontractionof{\hfbasemeasure, \expof{\contractionof{
		\sencodingof{\mintermformulaset}\canparam
		}{
		\shortcatvariables
		}}}{\shortcatvariables=\shortcatindices} = 0 \, . \]
	For indices $\shortcatindices$ with $\probat{\shortcatvariables=\shortcatindices}>0$ we have $\hfbasemeasureat{\shortcatvariables=\shortcatindices}=1$ and
	\begin{align*}
		 \sbcontractionof{\hfbasemeasure, \expof{\contractionof{
		\sencodingof{\mintermformulaset}\canparam
		}{
		\shortcatvariables
		}}}{\shortcatvariables=\shortcatindices} 
		&= \prod_{\selindexof{[\catorder]}} \expof{\canparamat{\selvariableof{[\catorder]}=\selindexof{[\catorder]}} \cdot \mintermofat{\selindexof{[\catorder]}}{\shortcatvariables=\shortcatindices}} \\
		&=  \expof{\canparamat{\selvariableof{[\catorder]}=\shortcatindices}} \\
		&=  \probat{\shortcatvariables=\shortcatindices} \, .
	\end{align*}
\end{proof}




\subsection{Categorical Constraints}\label{sec:categoricalTN}

\red{Also called atomization of categorical variables.}

%% Categorical variables with more possibilities
We made the assumption that all categorical variables in factored systems to be represented by propositional logics take binary values (i.e. $\catdim=2$).
In cases where a categorical variable $\catvariable$ takes multiple values we define for each $\catindex$ an atomic formula $\catvariableof{\catindex}$ representing whether $\catvariable$ is assigned by $\catindex$ in a specific state.
	%\[ \catvariableof{\catindex} =  (\catvariable = \catindex \, . \] Confusing notation
Following this construction we have the constraint that exactly one of the atoms $\catvariableof{\catindex}$ is $1$ at each state.

%% Capture constraint
To capture the constraints resulting from this construction we introduce auxiliary parts. % of Bayesian Propositional Networks.
Such constraints can also be expressed by a formula but would result in an unnecessary large tensor network.


%% Categorical selection map
\begin{definition}[Categorical Constraint]
	Given a list $\catvariableof{0},\ldots,\catvariableof{\catdim-1}$ of binary variables and a categorical variable $\catvariable$ with dimension $\catdim$ a categorical constraint is a tensor $\categoricalmap[\catvariable,\catvariableof{[\catdim]}]$ defined as
%	A categorical constraint taking values in $[\catdim]$ maps to its atoms by
%		\[ \categoricalmap : [\catdim] \rightarrow \bigtimes_{\catindex\in[\catdim]}[2]  \]
	\begin{align*}
		 \categoricalmap(\catindex,\catindexof{\variableset}) 
		 = \begin{cases} 
		 	1 & \text{if} \quad \catindexof{[\catdim]} = \onehotmapof{\catindex} \\
			0 & \text{else} \, . 
		 \end{cases}
	\end{align*}
% where the $1$ is at position $\catindex$.
\end{definition}

%% Decomposition
With Theorem~\ref{the:functionDecompositionBasisCP} the tensor representation of $\categoricalcore$ decomposes in a basis CP format (see Figure~\ref{fig:CategoricalDecomposition}b) of if its coordinate maps $\categoricalmap_{\catindex}$, where $\catindex\in[\catdim]$.
For the cores
\begin{align}
	\categoricalcoreof{\catindex} = \onehotmapofat{\catindex}{\catvariable} \otimes \onehotmapofat{1}{\catvariableof{\catindex}} + (\onesat{\catvariable}- \onehotmapofat{\catindex}{\catvariable} ) \otimes \onehotmapofat{0}{\catvariableof{\catindex}} 
\end{align}	
we have by Theorem~\ref{the:functionDecompositionBasisCP}
\begin{align*}
	\rencodingofat{\categoricalmap}{\catvariable, \catvariableof{0}, \ldots, \catvariableof{\catdim-1}} 
	= \contractionof{\{\rencodingof{\categoricalmap(\catindex)} \, : \, \catindex\in[\catdim]\}}{\catvariable, \catvariableof{0}, \ldots, \catvariableof{\catdim-1}} \, . 
\end{align*}


In the next theorem we show how a categorical constraint can be enforced in a tensor network by adding the tensor $\categoricalmap$ to a contraction.

\begin{theorem}
	For any tensor $\hypercoreat{\shortcatvariables}$ and a categorical constraint defined by an ordered subset $\catvariableof{\variableset}\subset\shortcatvariables$, a variable $\catvariable\in\shortcatvariables$ we have
	\begin{align*}
	 	\contractionof{\{\hypercoreat{\shortcatvariables}, \categoricalmap\}}{\indexedcatvariables} 
		= \begin{cases}
			\hypercoreat{\indexedcatvariables} & \text{if} \quad \catindexof{\variableset} = \onehotmapof{\catindex} \\
			0 & \text{else} \, . 
		\end{cases}
	\end{align*}
\end{theorem}
\begin{proof}
	For any $\catindexof{[\atomorder]}$ we have
		\[ \contractionof{\{\hypercoreat{\shortcatvariables}, \categoricalmap\}}{\indexedcatvariables}  = 
			\hypercoreat{\indexedcatvariableof{[\atomorder]}} \cdot \categoricalmap[\indexedcatvariableof{},\indexedcatvariableof{\variableset}] \, . 
		\]
	If $\catindexof{\variableset} = \onehotmapof{\catindex}$ we have $\categoricalmap[\indexedcatvariableof{},\indexedcatvariableof{\variableset}] = 1$ and thus
		\[ \contractionof{\{\hypercoreat{\shortcatvariables}, \categoricalmap\}}{\indexedcatvariables}  =  \hypercoreat{\indexedcatvariableof{[\atomorder]}}  \, . \]
	If $\catindexof{\variableset} \neq \onehotmapof{\catindex}$ then $\categoricalmap[\indexedcatvariableof{},\indexedcatvariableof{\variableset}] = 0$ and  
		\[ \contractionof{\{\hypercoreat{\shortcatvariables}, \categoricalmap\}}{\indexedcatvariables}  = 0 \, . \]
\end{proof}




%We define the corresponding network cores as
%\begin{align}
%	\categoricalcoreof{\catindex} =
%	\begin{cases}
%		\tbasis & \text{ if } \catvariable = \catindex \\
% 		\fbasis & \text{ else }
%	\end{cases} \, . 
%\end{align}	

%Similar to atom selector tensor, but different core definition ($\fbasis$ instead of $\ones$, when $\parindexof{}$ is not matching the core position).

%% CONFUSING!
%We represent the categorical constraint as another variable $\randomxof{\categoricalmap}$, which when known defines the values of the corresponding variables $\randomxof{\atomenumerator}$ (see Figure~\ref{fig:CategoricalDecomposition}a).


\begin{figure}[h]
\begin{center}
	\begin{tikzpicture}[scale=0.35, thick] % , baseline = -3.5pt

\begin{scope}[shift={(-15,2)}]

\node[anchor=center] (text) at (-1,3) {${a)}$};


\node [circle, draw, thick, fill=gray!50] (T1) at (0,0) {\tiny $\randomxof{0}$};	
\node [circle, draw, thick, fill=gray!50] (T2) at (3,0) {\tiny $\randomxof{1}$};	
\node[anchor=center] (text) at (6,0) {\small $\cdots$};
\node [circle, draw, thick, fill=gray!50] (T3) at (9,0) {\tiny $\randomxof{\atomorder-1}$};	

\node [circle, draw, thick, fill=gray!50] (C) at (4.5,-5) {\tiny $\randomxof{\categoricalmap}$};	

\draw[->] (C) -- (T1);
\draw[->] (C) -- (T2);
\draw[->] (C) -- (T3);

\end{scope}

\node[anchor=center] (text) at (-1,5) {${b)}$};


\drawatomindices{0}{2}
\draw (-1,1) rectangle (5,-1);
\node[anchor=center] (text) at (2,0) {\small $\categoricalcore$};
\draw[->] (2,-3) -- (2,-1) node[midway,left] {\tiny $\randomxof{\categoricalmap}$};

\node[anchor=center] (text) at (7,0) {${=}$};


\begin{scope}[shift={(10,2)}]

\newcommand{\conposseldec}{4.5,-5.5}

\draw[fill] (\conposseldec) circle (0.15cm);
\draw (\conposseldec) -- (4.5,-7.5) node[midway, right] {\tiny $\randomxof{\categoricalmap}$};
%!TEX encoding = UTF-8 Unicode\draw[dashed] (3.5,-7.5) rectangle (5.5, -9.5);
%\node[anchor=center] (text) at (4.5,-8.5) {\small $\ones$};

\draw[<-] (0,1) -- (0,-1) node[midway,left] {\tiny $\randomxof{0}$};
\draw (-1,-1) rectangle (1, -3);
\node[anchor=center] (text) at (0,-2) {\small $\categoricalcoreof{0}$};
\draw[<-] (0,-3) to[bend right=20] (\conposseldec);


\draw[<-] (3,1) -- (3,-1) node[midway,left] {\tiny $\randomxof{1}$};
\draw (2,-1) rectangle (4, -3);
\node[anchor=center] (text) at (3,-2) {\small $\categoricalcoreof{1}$};
\draw[<-] (3,-3) to[bend right=20]  (\conposseldec);

\node[anchor=center] (text) at (6,-2) {$\cdots$};

\draw[<-] (9,1) -- (9,-1) node[midway,left] {\tiny $\randomxof{\atomorder-1}$};
\draw (7.75,-1) rectangle (10.25, -3);
\node[anchor=center] (text) at (9,-2) {\small $\categoricalcoreof{\atomorder-1}$};
\draw[<-] (9,-3) to[bend left=20]  (\conposseldec);




\end{scope}

		


\end{tikzpicture}
\end{center}
\caption{Representation of a categorical constraint in a $\cpformat$ Format tensor network.
	a) Representation of the dependency of the graphical model.
	b) Tensor Representation with further network decomposition.
	We average by contraction with the dashed tensor $\ones$, if we do not specify the active atom.
	}
\label{fig:CategoricalDecomposition}
\end{figure}

\begin{remark}[Combination of Constraints]
	We can combine constraint cores by Hadamard products in the dual tensor network representation, as long as they can be satisfied together.
	An example, where this is not the case, are the categorical constraints to the three sets
		\[ \{\randomxof{0},\randomxof{1},\randomxof{2},\randomxof{3}\} \, , \, \{\randomxof{0},\randomxof{1}\}\, ,\,\{\randomxof{2},\randomxof{3}\} \, . \] 
	Besides the categorical cores also the datacores have a similar bayesian network affecting the atoms by another hidden variable.
	Combining both is welldefined, only when all datapoints satisfy the categorical constraints (that is only one of the atoms in each constraint is active).
\end{remark}


\begin{example}[Sudoku]
	An interesting example, where categorical constraints are combined is Sudoku, the game of assigning numbers to a grid.
	For a $n\in\nn$ we define variables $\catvariableof{i,j,k}$ where $i,j,k\in[n^2]$ and $\catdimof{i,j,k}=2$.
	By understanding $i$ as a line index and $j$ as a column index, they are ordered in a grid as sketched in Figure~\ref{fig:sudokuGrid} in the case $n=3$.
	
%	We further define atomization variables
%		\[ \catvariableof{i,j,k} = (\catvariableof{i,j} == k) \, . \]
%	\red{These are also at each $i,j$ categorical constraints!}
	
	
	At each position there is a single number, that is for each $i,j\in[n^2]$ have a constraint
		\[ \{\catvariableof{i,j,k} \, : \, k \in [n^2] \} \]
	
	Here the $n^6$ variables are the random variables whether a specific position has a specific number assigned.
	The $3\cdot n^2$ constraints are 
	\begin{itemize}
		\item Each number $k$ appears exactly once in each row $i\in[n^2]$:
			\[ \{\catvariableof{i,j,k}  \, : \, j \in [n^2] \} \]
		\item Each number $k$ appears exactly once in each column $j\in[n^2]$:
			\[ \{\catvariableof{i,j,k}  \, : \, i \in [n^2] \} \]
		\item Each number appears exactly once in each square $s,r\in[n]$
			\[ \{\catvariableof{i+n\cdot s,j+n\cdot r,k}  \, : \, i,j \in [n] \} \]
	\end{itemize}
	
	In total we have $3\cdot n^2 + n^4$ constraints for $n^6$ variables.

	\begin{figure}\label{fig:sudokuGrid} % ! Still without k index of the variables.
	\begin{center}
		\input{PartII/tikz_pics/hybrid_networks/sudoku_grid.tex}
	\end{center}
	\caption{Sudoku grid of categorical variables, here drawn in the standard case of $n=3$, each with dimension $\catdim=n^2=9$.}
	\end{figure}

	\red{Reasoning by Entailment propagation! 
	Also, probabilistic choices possible when exact (!) contraction at a position not a basis vector, then can choose one possibility.}

\end{example}















% Training
\section{Logic Network Inference}\label{cha:networkLearning}

In this chapter we first investigate unconstrained parameter estimated for Markov Logic Networks and Hybrid Logic Networks, which are special cases of the backward maps introduced in Chapter~\ref{cha:probDecomposition}.
We then motivate structure learning based on sparsity constraints on the parameters on the minterm family and present heuristic strategies leading to efficient structure learning algorithms.

\subsection{Mean parameters of Hybrid Logic Networks}

The convex polytope of realizable mean parameters (see Definition~\ref{def:meanForwardBackward}) is for a statistic $\formulaset$ of propositional formulas
	\[ \meansetof{\formulaset} = \left\{ \sbcontractionof{\probtensor,\sencodingof{\formulaset}}{\selvariable} \, : \, \probtensor\in\probtensorset \right\} \, ,\]
where by $\probtensorset$ we denote the set of all probability distributions.

The convex polytope has a characterization as a convex hull
\begin{align}\label{eq:hlnMeansetConvCharacterization}
	\meansetof{\formulaset} = \convhullof{\sencodingofat{\formulaset}{\indexedshortcatvariables,\selvariable} \, : \, \shortcatindices\in\atomstates} \, . 
\end{align}

We exploit this characterization to show, that the extreme points of $\meansetof{\formulaset}$ are exactly those realizable by Hard Logic Networks.

\begin{theorem}\label{the:extremeCharacterizationHLN}
	Any parameter $\meanparamat{\selvariable}\in\meansetof{\formulaset}$ is an extreme point of $\meansetof{\formulaset}$, if and only $\meanparamat{\selvariable}$ is boolean and the formula
			\[ \basemeasureofat{\formulaset,\meanparam}{\shortcatvariables} \coloneqq \bigwedge_{\selindexin \, : \, \meanparamat{\indexedselvariable}\in\{0,1\}}
		\lnot^{(1-\meanparamat{\indexedselvariable})} \enumformulaat{\shortcatvariables} \]
	is satisfiable.
	In that case, $\meanparam$ is the mean parameter of the Hard Logic Network with 
		\[ \kb=\{\lnot^{(1-\meanparamat{\indexedselvariable})} \enumformula \, : \, \selindexin \} \, . \]
\end{theorem}
\begin{proof}
	\proofrightsymbol: Let $\meanparam$ be an extreme point of $\meansetof{\formulaset}$. 
		Since by \eqref{eq:hlnMeansetConvCharacterization} $\meansetof{\formulaset}$ is a convex hull of vectors, there exists a $\shortcatindices\in\atomstates$ such that
			\[ \meanparamat{\selvariable} = \sencodingofat{\formulaset}{\indexedshortcatvariables,\selvariable}  \, . \]
		By definition of $\sencodingof{\formulaset}$, $\meanparamat{\selvariable}$ is a boolean vector and for any $\selindexin$ we have
			\[ \enumformulaat{\indexedshortcatvariables} = \meanparamat{\indexedselvariable} \]
		and thus
			\[ \lnot^{(1-\meanparamat{\indexedselvariable})} \enumformulaat{\indexedshortcatvariables} = 1 \, .  \]
		It follows that $\shortcatindices$ is also a model of $\basemeasureof{\formulaset,\meanparam}$ and $\basemeasureof{\formulaset,\meanparam}$ is satisfiable.
			
	\proofleftsymbol: To show the converse direction, let $\meanparamat{\selvariable}$ be a boolean vector such that $\basemeasureof{\formulaset,\meanparam}$ is satisfiable.
		Then there exists a model $\shortcatindices$ of $\basemeasureof{\formulaset,\meanparam}$.
		We have for any $\selindexin$ 
			\[ \sencodingofat{\formulaset}{\indexedshortcatvariables,\indexedselvariable} =  \enumformulaat{\indexedshortcatvariables} = \meanparamat{\indexedselvariable} \]
		and thus $\meanparamat{\selvariable}=\sencodingofat{\formulaset}{\indexedshortcatvariables,\selvariable}$.
		With the characterization \eqref{eq:hlnMeansetConvCharacterization} this establishes in particular $\meanparamat{\selvariable}\in\meansetof{\formulaset}$.		
		Since $\meanparamat{\selvariable}$ is boolean and therefore an extreme point of the cube $\cubeof{\seldim}$, it is also an extreme point of the subset $\meansetof{\formulaset}\subset\cubeof{\seldim}$.
\end{proof}


We now show, that any element of the mean parameter set can be realized by a hybrid logic network.

\begin{theorem}
	To any $\meanparamat{\selvariable}\in\meansetof{\formulaset}$ we build the set $\formulaset^{\meanparam}=\{\enumformula\in\formulaset : \meanparamat{\indexedselvariable}\notin\{0,1\}\}$ and the base measure
			\[ \basemeasureofat{\formulaset,\meanparam}{\shortcatvariables} \coloneqq \bigwedge_{\selindexin \, : \, \meanparamat{\indexedselvariable}\in\{0,1\}}
		\lnot^{(1-\meanparamat{\indexedselvariable})} \enumformulaat{\shortcatvariables} \, . \]
	Then, if $\meanparamat{\selvariable}$ is realizable by a positive distribution wrt $\basemeasureofat{\formulaset,\meanparam}{\shortcatvariables}$, it is realizable by an element of $\expfamilyof{\formulaset^{\meanparam},\basemeasureof{\formulaset,\meanparam}}$, that is there is a $\probtensor\in\expfamilyof{\formulaset^{\meanparam},\basemeasureof{\formulaset,\meanparam}}$ with
		\[ \sbcontractionof{\probtensor,\sencodingof{\formulaset}}{\selvariable} = \meanparamat{\selvariable} \, .  \] 
\end{theorem}
\begin{proof}
	Since $\meanparamat{\selvariable}\in\meansetof{\formulaset}$ there is a $\secprobat{\shortcatvariables}$ with 
		\[ \contractionof{\secprobtensor,\sencodingof{\formulaset}}{\selvariable} = \meanparamat{\selvariable} \, . \]
	From this it follows, that $\secprobtensor$ has a representation with respect to the base measure $\basemeasureof{\formulaset,\meanparam}$ since
		\[ \secprobat{\shortcatvariables} = \contractionof{\secprobtensor,\basemeasureof{\formulaset,\meanparam}}{\shortcatvariables}  \, . \]
	Thus, we have for the reduced mean parameter $\tilde{\meanparam}$, which results from dropping the coordinates in $\{0,1\}$, that
		\[ \tilde{\meanparam} \in \meansetof{\formulaset^{\meanparam},\basemeasureof{\formulaset,\meanparam}} \, . \]
	%% Using the positive distribution assumption -> Need to incorporate by a lemma!
	\red{Use here the below Lemma!}
	If $\formulaset^{\meanparam}$ is a minimal representation for the family $\expfamilyof{\formulaset^{\meanparam},\basemeasureof{\formulaset,\meanparam}}$,
	then $\tilde{\meanparam}$ is an interior point of $\meansetof{\formulaset^{\meanparam},\basemeasureof{\formulaset,\meanparam}}$ any by Theorem~3.3 in \cite{wainwright_graphical_2008}, there is a $\probtensor$ with $\sbcontractionof{\probtensor,\sencodingof{\formulaset}}{\selvariable} = \meanparamat{\selvariable}$.
	If $\formulaset^{\meanparam}$ is not a minimal representation, we can choose a subset of formulas providing a minimal representation with same expressivity and find $\probtensor$ with the same argument.
\end{proof}


\begin{lemma}
	If $\meanparamat{\selvariable}$ is in an exponential family with boolean base measure $\basemeasure$ and minimal representation, then it is in the interior of $\meanset$ if and only if it is representable by a positive distribution with respect to $\basemeasure$.
\end{lemma}
\begin{proof} 
	\proofrightsymbol: % Orient on proof of Theorem~3.3 
		We show, that for any test vector $\vectorat{\selvariable}$ different from the origin, there is a $\tilde{\meanparam}[\selvariable]$ with
			\[ \contraction{\vectorat{\selvariable},\meanparamat{\selvariable}} <  \contraction{\vectorat{\selvariable},\tilde{\meanparam}[\selvariable]} \, . \]
		Then, by [Rockafellar], since the exponential family is minimal and thus $\meanset$ full-dimensional, we have $\meanparamat{\selvariable}$ in the interior of $\meanset$.
	
	\proofleftsymbol: Theorem~3.3 in \cite{wainwright_graphical_2008}, we can construct the positive distribution by a member of the exponential family using that the gradient of the cumulant function is onto.
\end{proof}


\begin{lemma}
	If $\meanparamat{\selvariable}$ is in an exponential family with boolean base measure $\basemeasure$ and minimal representation, and not representable by a positive distribution with respect to $\basemeasure$, then there is $\canparamat{\selvariable}$ such that
		\[ \meanparamat{\selvariable} \in \argmax_{\meanparam\in\meanset} \contraction{\canparam,\meanparam} \, . \]
	Moreover, $\meanparamat{\selvariable}$ is representable with respect to the basemeasure
		\[ \basemeasure \cup  \indicatorofat{\arbset}{\shortcatvariables} \, , \]
	where the indicator is on the set
		\[ \arbset = \argmax_{\shortcatindices} \contraction{\canparam,\sstat(\shortcatindices)}  \, . \]
\end{lemma}
\begin{proof}
	We use the halfspace representation of the bounded polyhedron $\meanset$ to show the first claim.
	By the above lemma, $\meanparamat{\selvariable}$ is not in the interior, therefore on a facet and thus the $\canparam$ is found by the normal equation of that facet.
	
	This facet is the convex hull of the extreme points of $\meanset$ corresponding with the elements of $\arbset$.
	There exists hence a convex combination in the form of a distribution $\probtensor$ supported on $\arbset$, such that 
		\[ \meanparamat{\selvariable} = \sum_{\shortcatindices\in\arbset} \probat{\indexedshortcatvariables} \sencsstatat{\indexedshortcatvariables,\selvariable} \]
	and $\meanparamat{\selvariable}$ is reproduced by $\probtensor$.
\end{proof}

Each facet of $\meanset$ thus defines a possible base measure, which is sufficient to reproduce the mean parameters on that facet.

\begin{definition}
	The facet basemeasure to the facet of $\meanset$ with normal $\canparam$ is
		\[ \basemeasureof{\formulaset,\canparam} = \indicatorofat{\argmax_{\shortcatindices} \contraction{\canparam,\sstat(\shortcatindices)}}{\shortcatvariables} \, . \]
\end{definition}


\begin{lemma}
	Let $\canparamat{\indexedselvariable}$ be a normal to a facet of $\meanset$.
	If and only if the formula
		\[ \formulaof{\formulaset,\canparam} = \bigwedge_{\selindexin \, : \, \canparamat{\indexedselvariable}\neq0} \lnot^{(1-\greaterzeroof{\canparamat{\indexedselvariable}})} \enumformula  \]
	is satisfiable, then it is the base measure to the facet with normal $\canparam$.
	Here $\greaterzeroof{z}$ denotes the indicator of $z>0$.
	
	If the above is not satisfiable, then the base measure is
		\[ \bigvee_{v[\selvariable] \, : \, \contraction{\canparam,v} \in \max_{\meanparam} \contraction{\canparam,\meanparam} }  \formulaof{\formulaset,\contractionof{\canparam,v}{\selvariable}}  \]
	where $v$ are boolean vectors.
\end{lemma}


% Reduction to Hybrid Logic Networks
We can thus reduce the set of probability distributions in the definition of the convex polytope of mean parameters to the set $\hlnsetof{\formulaset}$ (see Definition~\ref{def:hln}), that is
\begin{align*}
	\meansetof{\formulaset} = \left\{ \sbcontractionof{\probtensor,\sencodingof{\formulaset}}{\selvariable} \, : \, \probtensor\in\hlnsetof{\formulaset} \right\} \, .
\end{align*}

% Mean parameter characterization
To summarize, for any $\meanparam\in\meansetof{\formulaset}$ we have one of the following (see also Figure~\ref{fig:meansetSketch}):
\begin{itemize}
	\item All $\meanparamat{\indexedselvariable}\notin\{0,1\}$: Then reproducible by a Markov Logic Network, $\meanparam\in\meanset$.
		This follows from the classical result of Theorem~3.3 in \cite{wainwright_graphical_2008}.
	\item At least one $\meanparamat{\indexedselvariable}\notin\{0,1\}$ and one $\meanparamat{\indexedselvariable}\in\{0,1\}$: Then reproducible by a Hybrid Logic Network.
	\item All $\meanparamat{\indexedselvariable}\in\{0,1\}$: Then  an extreme point and by Theorem~\ref{the:extremeCharacterizationHLN} reproducible by a Hard Logic Network.
\end{itemize}



\begin{figure}[h]\label{fig:meansetSketch}
\begin{center}
	\begin{tikzpicture}[scale=0.35]
    % Define points
    
    \node[below] at (-1,7) {$\interiorof{\meanset}_{\formulaset}$};
    
    \coordinate (A) at (0,0);
    
    \node[below] at (A) {$\meanparam_1$};
    \draw[fill] (A) circle (0.15cm);
    
    \coordinate (B) at (12,2.5);
    \path (A) -- (B) coordinate[pos=0.7] (P1);

    \node[below] at (P1) {$\meanparam_2$};
    \draw[fill] (P1) circle (0.15cm);
       
    \coordinate (P2) at (2,10); 
    \node[below] at (P2) {$\meanparam_3$};
    \draw[fill] (P2) circle (0.15cm);
    
    \coordinate (C) at (7.5,12);
    \path (B) -- (C) coordinate[pos=0.5] (P4);
    \node[left] at (P4) {$\closureof{\meanset}_{\formulaset}/\interiorof{\meanset}_{\formulaset}$};
    \coordinate (D) at (-3,12);
    \coordinate (E) at (-10,5);

    	\draw[thick] (A) -- (B) -- (C) -- (D) -- (E) -- cycle;

	\draw[dashed] (-10,0) -- (12,0) -- (12,12) -- (-10,12) -- (-10,0);
	    \node[left] at (-10,10) {$[0,1]^\seldim$};
\end{tikzpicture}


\end{center}
\caption{Sketch of the convex polytope ${\meansetof{\formulaset}}$ as a subset of the $\seldim$-dimensional cube $[0,1]^\seldim$ (here as a 2-dimensional projection) with example mean 	parameters $\meanparam_1,\meanparam_2,\meanparam_3$.
	The boundary points $\meanparam_1,\meanparam_2\in\closureof{\meanset}_{\formulaset}/\interiorof{\meanset}_{\formulaset}$ are examples of mean parameters, which can be realized by a Hard Logic Networks (respectively Hybrid Logic Network). 
	Any extreme point $\meanparam_1\in\meansetof{\formulaset}\cup\{0,1\}^{\seldim}$ is realizable by a Hard Logic Network, while a non-extreme boundary point $\meanparam_2\in\meansetof{\formulaset}/\{0,1\}^{\seldim}$ is realizable by a Hybrid Logic Network.
	Any interior point $\meanparam_3\in\interiorof{\meansetof{\formulaset}}$ is realizable by a Markov Logic Network.
} 
\end{figure}




\subsection{Unconstrained Parameter Estimation} \label{sec:parameterEstimation} % Check for redundancy with the mln introduction chapter!

% Repetition and result transfering
Markov Logic Networks are exponential families with statistics by a set $\formulaset$ of propositional formulas.
We furthermore allow for propositional formulas as base measures, to also include the discussion of Hybrid Logic Networks.
Based on this, we apply the theory of probabilistic inference, developed in Chapter~\ref{cha:probReasoning} for the generic exponential families.

% Special example: MLE
The Maximum Likelihood Problem on Markov Logic Networks is the M-projection
\begin{align*}
	\argmax_{\canparamat{\selvariable}\in\rr^{\seldim}} \quad 
	\centropyof{\probtensor}{\expdistof{(\sstat,\canparam,\basemeasure)}}	
\end{align*}
in the case $\probtensor=\empdistribution$ for a data map $\datamap$.

% Backward map
The M-projection coincides, after dropping constant terms in case of non-trivial base measure, with the backward map
\begin{align*}
	\argmax_{\canparamat{\selvariable}\in\rr^{\seldim}} \quad 
	\sbcontraction{\canparamat{\selvariable},\meanparamat{\selvariable}} - \cumfunctionof{\canparamat{\selvariable}} 
\end{align*}
where
\begin{align*}
	\meanparam{\selvariable} 
	= \sbcontractionof{\sencodingof{\formulaset},\probtensor}{\selvariable} 
	\quad \text{and} \quad
	\cumfunctionof{\canparamat{\selvariable}} 
	= \sbcontraction{\expof{ \sbcontractionof{\sencodingofat{\formulaset}{\shortcatvariables,\selvariable},\canparamat{\selvariable}}{\shortcatvariables} }, \basemeasure} \, . 
\end{align*}

\red{Theory for MLN a special case of the generic case of exponential families, here corollaries?.}

% Extension to HLN
We now extend to Hybrid Logic Networks
\begin{align*}
	\argmax_{\secprobtensor\in\hlnsetof{\formulaset}} \quad 
	\centropyof{\probtensor}{\secprobtensor}	
\end{align*}

\begin{theorem}
	Let $\meanparamat{\selvariable} = \contractionof{\probtensor,\sencfset}{\selvariable}$ and
		\[ \secformulaset = \{\enumformula \, : \, \meanparamat{\indexedselvariable} \in \{0,1\}\} \quad , \quad 
		\basemeasureofat{\secformulaset,\meanparam}{\shortcatvariables} 
		= \bigwedge_{\enumformula\in\secformulaset} \lnot^{(1-\meanparamat{\indexedselvariable})} \enumformulaat{\shortcatvariables}
		\, . \]
	The solution of the M-projection of $\probtensor$ onto the set of hybrid logic networks representable by $\formulaset$ then coincides with the projection of $\probtensor$ onto $\expfamilyof{\formulaset/\secformulaset,\basemeasureof{\secformulaset,\meanparam}}$.
\end{theorem}
\begin{proof}
	Work by weight cutoff, in the limit of hard logics?
\end{proof}

\subsubsection{Maximum Entropy in Hybrid Networks}


% Special example: MaxEnt
The Maximum Entropy Problem for Markov Logic Networks is
\begin{align}
	\argmax_{\probtensor} \quad \sentropyof{\probtensor} 
	\quad \text{subject to} \quad  
	\sbcontractionof{\probtensor,\sencodingof{\formulaset}}{\selvariable}
	 =  \meanparamat{\selvariable} 
\end{align}



\begin{corollary}[of Theorem~\ref{the:maxEntMaxLikeDuality}]
	\red{Works only for $\meanparamat{\indexedselvariable}\in(0,1)$}
	Among all distributions $\probtensor$ of $\atomstates$ satisfying $\sbcontractionof{\probtensor,\sencodingof{\formulaset}}{\selvariable}
	 = \sbcontractionof{\empdistribution,\sencodingof{\formulaset}}{\selvariable}$ the Markov Logic Network with formulas $\formulaset$ and weights $\canparam$ being the solution of the maximum likelihood problem has minimal entropy.
\end{corollary}

% Unique property of MLN
We notice, that the solution of the maximum entropy problem is thus a Markov Logic Network.
This is remarkable, because this motivates our restriction to Markov Logic Networks as those distributions with maximal entropy given satisfaction rates of formulas in $\formulaset$.


% Extension to HLN
When now extend to the situations $\meanparamat{\indexedselvariable}\in\{0,1\}$ can appear.
It that case the formula is entailed or contradicted by the facts, and dropping should be considered in both cases.

The max entropy - max likelihood duality still holds for hybrid logic networks as we show in the next theorem.

\begin{theorem}
	Given a set of formulas $\tilde{\formulaset}$ and $\tilde{\meanparam}$, with coordinates $\tilde{\meanparam}_\selindex\in[0,1]$ in the closed interval $[0,1]$.
	If the corresponding maximum entropy problem is feasible, its solution is a hybrid logic network with 
	\begin{itemize}
		\item $\hardformulaset= \{\enumformula : \selindexin, \meanparamat{\indexedselvariable} = 1\} \cup  \{\lnot\enumformula : \selindexin, \meanparamat{\indexedselvariable} = 0\} $
		\item $\softformulaset = \{\enumformula : \selindexin, \meanparamat{\indexedselvariable} \in (0,1)\}$
		\item $\canparam$ being the backward map evaluated at the vector $\meanparam$ consisting of the coordinates of $\tilde{\meanparam}$ not in $\{0,1\}$
	\end{itemize}
\end{theorem}
\begin{proof}
	Feasible distributions have a density with base measure by $\hardformulaset$, we therefore reduce the set of distributions in the argmax to those with density to the base measure.
	The max entropy is a max entropy problem with respect to that base measure, where we only keep the constraints to the mean parameters different from $\{0,1\}$ (those are trivially satisfied).
	The statement then follows from the generic property (see Sec3.1 in \cite{wainwright_graphical_2008}).
\end{proof}






\subsection{Alternating Algorithms to Approximate the Backward Map}\label{sec:alternatingParEstMLN}

Let us now introduce an implementation of the Alternating Moment Matching Algorithm~\ref{alg:AMM} in case of Markov Logic Networks.
To solve the moment matching condition at a formula $\enumformula$ we refine Lemma~\ref{lem:mmContractionEquation} in the following.

\begin{lemma}\label{ref:lemMMinMLN}
	Let there be a base measure $\basemeasure$, a formula selecting map $\formulaset=\{\enumformula \, : \, \selindexin\}$ and weights $\canparam$, and choose $\selindexin$ such that $\enumformula  \notin \{\onesat{\shortcatvariables},\zerosat{\shortcatvariables}\}$.	
	The moment matching condition relative to $\canparam$, $\selindexin$ and $\datameanat{\indexedselvariable}\in(0,1)$ is then satisfied, if
	\begin{align} \label{sol:momentMatchingExformula}
	 	\weightat{\indexedselvariable} = \lnof{
		\frac{\datameanat{\indexedselvariable}}{(1-\datameanat{\indexedselvariable})} 
		\cdot \frac{\hypercoreat{\catvariableof{\enumformula }=0}}{\hypercoreat{\catvariableof{\enumformula }=1}} 
		} 
	\end{align}
	where by $\hypercoreat{\catvariableof{\enumformula }}$ we denote the contraction 
	\begin{align*}
	 	\hypercoreat{\catvariableof{\enumformula}} 
		= \contractionof{\{\rencodingof{\enumformula} \, : \, \selindexin\}
		\cup\{\headcoreof{\tilde{\selindex}} : \tilde{\selindex} \in [\seldim], \tilde{\selindex}\neq\selindex\}
		\cup\{\basemeasure\}}{\catvariableof{\enumformula}} \, . 
	\end{align*}
\end{lemma}
\begin{proof}
	Since $\imageof{\enumformula}\subset[2]$ we have
	\begin{align*}
		\idrestrictedto{\imageof{\enumformula}} = \onehotmapofat{1}{\catvariableof{\enumformula}}
	\end{align*}
	and the moment matching condition is by Lemma~\ref{lem:mmContractionEquation} satisfied if
	\begin{align*}
		\sbcontraction{\headcoreof{\selindex}, \onehotmapof{1}, \hypercore}
			= \sbcontraction{\headcoreof{\selindex},\hypercore} \cdot \datameanat{\indexedselvariable} \, . 
	\end{align*}
	This is equal to 
	\begin{align*}
		\expof{\canparamat{\indexedselvariable}} \cdot \hypercoreat{\catvariableof{\enumformula}=1}
		= \left( \expof{\canparamat{\indexedselvariable}} \cdot \hypercoreat{\catvariableof{\enumformula}=1} + \hypercoreat{\catvariableof{\enumformula}=0} \right) \cdot \datameanat{\indexedselvariable} \, . 
	\end{align*}
	Rearranging the equations this is equal to 
	\begin{align*}
	 	\hypercoreat{\catvariableof{\enumformula}} 
		= \contractionof{\{\rencodingof{\enumformula}\}
		\cup\{\headcoreof{\tilde{\selindex}} : \tilde{\selindex} \in [\seldim], \tilde{\selindex}\neq\selindex\}
		\cup\{\basemeasure\}}{\selvariable} \, . 
	\end{align*}
	We notice that the right side is well defined, since we have by assumption $\datameanat{\indexedselvariable}, (1- \datameanat{\indexedselvariable}) \neq 0$ and $\hypercoreat{\catvariableof{\enumformula}=0}, \hypercoreat{\catvariableof{\enumformula}=1} \neq 0$ since Markov Logic networks are positive distributions and $\enumformula \notin \{\onesat{\shortcatvariables},\zerosat{\shortcatvariables}\}$.
\end{proof}


%% Hard network reference!
In the case $\datameanat{\indexedselvariable}\in\{0,1\}$ the moment matching conditions are not satisfiable for $\canparamat{\indexedselvariable}\in\rr$.
But, we notice, that in the limit $\canparamat{\indexedselvariable}\rightarrow \infty $ (respectively $-\infty$) we have
	\[ \meanparamat{\indexedselvariable} \rightarrow  1 \quad \text{(respectively $0$)}\, ,  \]
and the moment matching can be satisfied up to arbitrary precision.
In Section~\ref{sec:hardNetworks} we will allow infinite weights and interpret the corresponding factors by logical formulas.
As a consequence, we will able to fit graphical models, which we will call hybrid networks on arbitrary satisfiable mean parameters.

%
The cases $\hypercoreat{\catvariableof{\enumformula}=1}=0$, respectively $\hypercoreat{\catvariableof{\enumformula}=1}=0$ only appear for nontrivial formulas when the distribution is not positive. 
This is not the case for Markov Logic Networks, but will happen when formulas are added as cores of a Markov Network.
This situation will has been investigated in Section~\ref{sec:hardNetworks}.


% Concave likelihood 
Since the likelihood is concave (see \cite{koller_probabilistic_2009}), there are not local maxima the coordinate descent could run into and coordinate descent will give a monotonic improvement of the likelihood. 

We suggest an alternating optimization by Algorithm~\ref{alg:AWO}, solving the moment matching equation iteratively for all formulas $\exformulain$ and repeat the optimization until a convergence criterion is met.
This is an coordinate ascent algorithm, when interpreted the loss $\lossof{\expdist}$ as an objective depending on the vector $\canparam$.

\begin{algorithm}[hbt!]
\caption{Alternating Weight Optimization (AWO)}\label{alg:AWO}
\begin{algorithmic}
%% INPUT: Numerated formula set, mean parameter $\datameanat{\selvariable}$
%\For{$\exformula\in\formulaset$}
\State $\kb = \ones$, $\secnodes=\varnothing$
\For{$\selindexin$}
	\If{$\meanparamat{\indexedselvariable}=1$}
		\[ \hardformulaset \algdefsymbol \hardformulaset \cup \{\enumformula\}\]
	%\EndIf
	\ElsIf{$\meanparamat{\indexedselvariable}=0$}
		\[ \hardformulaset \algdefsymbol \hardformulaset \cup \{\lnot\enumformula\}\]
	\Else
		\State $\secnodes\algdefsymbol \secnodes \cup \{\selindex\}$
	\EndIf
\EndFor
\For{$\selindex\in\secnodes$}	
		\State Compute
		\[ \hypercoreat{\catvariableof{\enumformula}}
		\algdefsymbol \sbcontractionof{\rencodingof{\enumformula}}{\catvariableof{\enumformula}} \]
		\State Set 
		\begin{align*}
	 		\canparamat{\indexedselvariable} 
			\algdefsymbol \lnof{
			\frac{\datameanat{\indexedselvariable}}{(1-\datameanat{\indexedselvariable})} 
			\cdot \frac{\hypercoreat{\catvariableof{\enumformula}=0}}{\hypercoreat{\catvariableof{\enumformula}=1}} 
			} 
		\end{align*}
\EndFor
\If {$\sbcontraction{\kb}=0$}
	 \State \textbf{raise} "Inconsistent Knowledge Base"
\EndIf
\While{Convergence criterion is not met}
\For{$\selindex\in\secnodes$}
	\State Compute
	\begin{align*}
	 	\hypercoreat{\catvariableof{\enumformula}} 
		= \contractionof{\{\rencodingof{\enumformula} \, : \, \selindexin\}
		\cup\{\headcoreof{\tilde{\selindex}} : \tilde{\selindex} \in [\seldim], \tilde{\selindex}\neq\selindex\}
		\cup\{\basemeasure\}}{\catvariableof{\enumformula}}
	\end{align*}
	\State Set 
	\begin{align*}
	 	\canparamat{\indexedselvariable} = \lnof{
		\frac{\datameanat{\indexedselvariable}}{(1-\datameanat{\indexedselvariable})} 
		\cdot \frac{\hypercoreat{\catvariableof{\enumformula}=0}}{\hypercoreat{\catvariableof{\enumformula}=1}} 
		} 
	\end{align*}
\EndFor
\EndWhile
\end{algorithmic}
\end{algorithm}


% Independent formulas
In the initialization phase of Algorithm~\ref{alg:AWO}, each parameters is initialized relative to a uniform distribution. 
The algorithm would be finished, if the variables $\catvariableof{\exformula}$ are independent.
This would be the case, if the Markov Logic Network consists of atomic formulas only.
When they fail to be independent, the adjustment of the weights influence the marginal distribution of other formulas and we need an alternating optimization.
% 
This situation corresponds with couplings of the weights by a partition contraction, which does not factorize into terms to each formula.


% Inference
Solving Equation~\ref{sol:momentMatchingExformula} requires inference of a current model by answering the query to the formula $\texformula$.
This can be a bottleneck and circumvented by approximative inference, see e.g. CAMEL \cite{ganapathi_constrained_2008}.



\begin{remark}[Grouping of coordinates with trivial sum]
	When having a set of coordinates, such that the coordinate functions are binary and sum to the trivial tensor, one can find simultaneous updates to the canonical parameters, such that the partition function is staying invariant.
	Given a parameter $\canparam^t$ we compute
		\[ \meanparam^t = \contractionof{\expdistof{(\sstat,\canparam^t)}, \sstat}{\selvariable} \]
	and build the update
		\[ \canparam^{t+1} = \canparam^t + \lnof{\meanparam^{\datamap}}{\meanparam^t} \, . \]
	Then, $\canparam^{t+1}$ satisfies the moment matching equations for all coordinates in the set.
	
	
	The assumptions are met when taking all features to any hyperedge in a Markov Network seen as an exponential family.
	In that case, the update algorithm is refered to as  Iterative Proportional Fitting \cite{wainwright_graphical_2008}.
	Further, when activating both $\exformula$ and $\lnot\exformula$.
\end{remark}


\subsection{Forward and backward mappings in closed form}

% Closed form availability
We recall from Chapter~\ref{cha:probReasoning}, that while forward mappings are always in closed form by contractions, backward mapping in general do not have a closed form representation.
Instead, the backward map is in general implicitly characterized by a maximum entropy problem constrained to matching expected sufficient statistics.
We investigate in this section specific examples, where closed forms are available for both.
In these cases, parameter estimation can thus be solved by application of the inverse on the expected sufficient statistics with respect to the empirical distribution, and iterative algorithms can be avoided.

% Usage
%When the backward map $\backwardmap$ is available in closed form, we directly get optimal parameters by the inversion acting on the satisfaction rate and can avoid iterative algorithms of parameter estimation.

\subsubsection{Maxterms and Minterms}

Minterms (respectively maxterms) are ways in propositional logics to get a syntactical formula representation based on a formula to each world which is a model (respectively fails to be a model).
We have already studied in Section~\ref{sec:MLNMaxMintermRep} how to represent any distribution as a MLN of maxterms (respectively minterms), see Theorem~\ref{the:maximalClausesRepresentation}.

We use the tuple enumeration of the maxterms and minterms by $\atomstates$ introduced in Section~\ref{sec:termClauseDecomposition}.
With respect to this enumeration the canonical parameters and mean parameters are tensors in $\bigotimes_{\atomenumeratorin}\rr^2$. 
%% Interpretation of the mean parameters
Since the statistic of the minterm family is the identity, the mean parameters for the minterm family are
	\[ \meanparamat{\selvariableof{[\atomorder]}=\catindexof{[\atomorder]}} 
	= \probat{\catindexof{[\atomorder]}} 
	\]
and therefore after a relabeling of categorical variables to selection variables $\meanparam=\probtensor$.
For maxterms we have analogously
	\[ \meanparamat{\selvariableof{[\atomorder]}=\catindexof{[\atomorder]}} 
	= 1-\probat{\catindexof{[\atomorder]}} 
	\]
and $\meanparam = \onesat{}-\probtensor$.
We can use these insights to provide a characterization of the forward and backward maps of the minterm and maxterm family.

\begin{theorem}
	Given the Markov Logic Networks to the formula sets
		\[ \mintermformulaset := \{ \mintermof{\atomindices} \, : \, \atomindicesin\} \quad \text{and} \quad 
		\maxtermformulaset := \{ \maxtermof{\atomindices} \, : \, \atomindicesin\}  \]
	of all minterms, respectively of all mapterms, the forward mapping are
		%\[ \forwardmapwrt{\mintermformulaset}: \bigotimes_{\atomenumeratorin}\rr^{2} \rightarrow \bigotimes_{\atomenumeratorin}\rr^{2} \]
		\[ \forwardmapwrt{\mlnmintermsymbol}(\canparam) = \normationofwrt{\expof{\canparam}}{\shortcatvariables}{\varnothing}  
		\quad \text{and} \quad 
		 \forwardmapwrt{\mlnmaxtermsymbol}(\canparam) = \normationofwrt{\expof{-\canparam}}{\shortcatvariables}{\varnothing} \, , \]
	where in a slight abuse of notation we assigned the variables $\shortcatvariables$ to the canonical parameters $\canparam$.

	Possible choices of the backward mappings are
		%\[ \backwardmapwrt{\mlnmintermsymbol}: \bigotimes_{\atomenumeratorin}\rr^{2} \rightarrow \bigotimes_{\atomenumeratorin}\rr^{2} \]
		\[ \backwardmapwrt{\mlnmintermsymbol}(\meanparam) = \lnof{\meanparam} 
			\quad \text{and} \quad 
			\backwardmapwrt{\maxtermformulaset}(\meanparam) = -\lnof{\meanparam} \, .
		 \]
\end{theorem}
\begin{proof}
	For the minterms we use that
		\[ \mintermformulaset[\shortcatvariables,\catvariableof{\mintermformulaset}]  = \identityat{\shortcatvariables,\catvariableof{\maxtermformulaset}}\] 
	and get
		\[ \forwardmapwrt{\mlnmintermsymbol}(\canparam) 
		= \normationof{
		\expof{\contractionof{\{\mintermformulaset, \canparam\}}{\shortcatvariables}}
		}{\shortcatvariables}
		= 
		\normationof{\expof{\canparam}}{\shortcatvariables} \, . 
		\]
	
	We notice that for any $\meanparam$ in the image of the forward map we have
		\[ \forwardmapwrt{\mlnmintermsymbol}(\backwardmapwrt{\mlnmintermsymbol}(\meanparam)) = \meanparam \]
	Therefore, $\backwardmapwrt{\mintermformulaset}$ is indeed a backward mapping to the exponential family of minterms.
	
	For the maxterms we use that
		\[ \maxtermformulaset[\shortcatvariables,\catvariableof{\maxtermformulaset}] = \onesat{\shortcatvariables,\catvariableof{\maxtermformulaset}}-\identityat{\shortcatvariables,\catvariableof{\maxtermformulaset}} \]
	and get
	\begin{align*}
		\forwardmapwrt{\mlnmaxtermsymbol}(\canparam) 
		& = \normationof{
		\expof{\contractionof{\{\mintermformulaset, \canparam\}}{\shortcatvariables}}
		}{\shortcatvariables} \\
		& = \normationof{\{
		\expof{\contractionof{\{\ones, \canparam\}}{\shortcatvariables}}, 
		\expof{-\contractionof{\canparam}{\shortcatvariables}} \}
		}{\shortcatvariables} \\
		& = \normationof{
		\expof{-\canparam}
		}{\shortcatvariables}
	\end{align*}
	where we used, that $\expof{\contractionof{\{\ones, \canparam\}}{\shortcatvariables}}$ is a multiple of $\onesat{\shortcatvariables}$ and is thus eliminated in the normation.
	For any $\meanparam\in\imageof{\forwardmapwrt{\mlnmaxtermsymbol}}$ we have
		\[ \forwardmapwrt{\mlnmaxtermsymbol}(\backwardmapwrt{\mlnmaxtermsymbol}(\meanparam) ) 
		= \meanparam
		%-\lnof{\expof{-\canparam}} + \contractionof{\expof{-\canparam}}{\varnothing} \cdot \onesat{\shortcatvariables}
		%= \canparam + \contractionof{\expof{-\canparam}}{\varnothing} \cdot \onesat{\shortcatvariables}
		\]
	and $\backwardmapwrt{\mlnmintermsymbol}$ is thus a backward map for the exponential family of maxterms.
\end{proof}

% Fitting arbitrary distributions
Any positive probability distribution can thus be fitted by minterms when we choose $\canparam=\lnof{\probtensor}$, respectively by maxterms when we choose $\canparam=\ones-\lnof{\probtensor}$.
Thus, we have identified a subset of $2^{\atomorder}$ formulas, which is rich enough to fit any distribution.





\subsubsection{Atomic formulas}

% Repeat atomic formulas
Let us now derive a closed form backward mapping for the statistic
	\[ \atomformulaset := \{\atomicformulaof{\atomenumerator}: \atomenumeratorin\} \, . \]

The mean parameters coincide with the queries on the atomic formulas, that is the marginal 
	\[ \meanparamat{\selvariable=\atomenumerator} = \probat{\catvariableof{\atomenumerator}=1}  \, . \]

\begin{theorem}
	Given a Markov Logic Network with the statistic $\atomformulaset$ of atomic formulas, the forward mapping from canonical parameters to mean parameters is the coordinatewise sigmoid, that is
		\[ \forwardmapwrtof{\mlnatomsymbol}{\canparamat{\selvariable}} = \frac{\expof{\canparamat{\selvariable}}}{\onesat{\selvariable}+\expof{\canparamat{\selvariable}}}   \]
	where the quotient is performed coordinatewise.

	A backward mapping is the coordinatewise logit, that is
		\[ \backwardmapwrt{\mlnatomsymbol}(\meanparamat{\selvariable}) 
		= \lnof{\frac{
			\meanparamat{\selvariable}
			}{
			\onesat{\selvariable}-\meanparamat{\selvariable}
			}}  \, . \]
\end{theorem}
\begin{proof}
	We have for any $\canparamat{\selvariable}\in\rr^{\atomorder}$
		\[ \probofat{(\atomformulaset,\canparam)}{\shortcatvariables} 
		= \bigotimes_{\atomenumeratorin} \normationof{\expof{\canparamat{\selvariable=\atomenumerator}\cdot \atomicformulaof{\atomenumerator}}}{\catvariableof{\atomenumerator}}  \, . \]

	
	For any $\atomenumeratorin$ it therefore holds, that
	\begin{align*}
		\forwardmapwrtof{\mlnatomsymbol}{\canparamat{\selvariable}}[\selvariable=\atomenumerator] 
		&=\sbcontraction{\atomicformulaof{\atomenumerator},  \probofat{(\atomformulaset,\canparam)}{\shortcatvariables}} \\
		&=\sbcontraction{\atomicformulaof{\atomenumerator},  \normationof{\expof{\canparamat{\selvariable=\atomenumerator}\cdot \atomicformulaof{\atomenumerator}}}{\catvariableof{\atomenumerator}}} \\
		& = \frac{\expof{\canparamat{\selvariable=\atomenumerator}}}{1+\expof{\canparamat{\selvariable=\atomenumerator}}} \, .
	\end{align*}

	Since the coordinatewise logit is the inverse function of the coordinatewise sigmoid the map
	\begin{align*}
		\backwardmapwrtof{\mlnatomsymbol}{\meanparamat{\selvariable}}[\selvariable=\atomenumerator] 
		& = \lnof{\frac{\meanparamat{\selvariable=\atomenumerator}}{1- \meanparamat{\selvariable=\atomenumerator}}}
	\end{align*}
	satisfies for any $\meanparam$ in the image of the forward map
	\begin{align*}
		\forwardmapwrt{\mlnatomsymbol}(\backwardmapwrt{\mlnatomsymbol}(\meanparam)) = \meanparam 
	\end{align*}
	and is therefore a backward map.
\end{proof}


% Representation by selection tensor networks
In a selection tensor networks they are represented by a single neuron with identity connective and variable selection to all atoms.
We will investigate such examples in more detail in Chapter~\ref{cha:sparseTC}, where atomic formulas Markov Logic Networks are specific cases of monomial decomposition of order 1.
	
% Interpretation of the result as independence approximation
The maximum likelihood estimator of a positive probability distribution by the MLN of atomic formulas is therefore the tensor product of the marginal distributions.
The Kullback-Leibler divergence between the distribution and its projection is the mutual information of the atoms, see for example Chapter~8 in \cite{mackay_information_2003}.

\begin{remark}[Decomposition into systems of atomic networks]
	\red{By Independence Decomposition we reduce to a system of atomic MLN.
	The minterms of such MLNs are the literals.
	By redundancy (literals sum up to $\ones$), it suffices to take only the positive or the negative literal.
	}
%	We set the weights of $\weightof{\lnot\atomicformulaof{\atomenumerator}}=0$ (corresponding with a gauge normation of the energy offset symmetry). % Not needed!
\end{remark}

















\subsection{Constrained parameter estimation in the minterm family}

% Naive exponential family
We approach structure learning as constrained parameter estimation in the naive exponential family (see Example~\ref{exa:naiveExpFamily}), which coincides with the minterm family $\formulasetof{\mlnmintermsymbol}$.
The minterm family is defined by the statistic $\sstat = \identityat{\shortcatvariables, \selvariableof{[\catorder]}}$ and has energy tensors coinciding with the canonical parameters.

% Convex polytope characterization
\red{For the minterm family, we have as mean parameter set the convex hull of one-hot encodings. 
Each basis vector is an extreme point is an extreme point.
}


By Theorem~\ref{the:mintermExpressivityMLN} all positive distributions are member of the minterm markov logic network family.
This expressivity result was generalized to arbitrary distributions, when allowing for formulas as basemeasures by Theorem~\ref{the:mintermExpressivityHLN}.

Finding the distribution maximizing the likelihood of data would then be the empirical distribution.
In this case we would have $\datameanat{\selvariableof{[\catorder]}=\shortcatindices} = \empdistributionat{\shortcatvariables=\shortcatindices}$ and the maximum likelihood distribution is found by the problem
\begin{align*}
	\argmax_{\canparam\in\facspace}  \sbcontraction{\canparam,\empdistribution} - \cumfunctionof{\canparam} \, 
\end{align*}
which is solved at $\canparam=\lnof{\empdistribution}$ with $\probtensorof{(\identity,\lnof{\empdistribution})}= \empdistribution$.
This follows from $\lossof{\probtensorof{(\identity,\canparam)}}=\kldivof{\empdistribution}{\probtensorof{(\identity,\canparam)}}$, which is by Gibbs inequality minimized at $\probtensorof{(\identity,\canparam)}=\empdistribution$, which is the case for $\canparam = \lnof{\empdistribution}$.

We here allow for $\lnof{0}=-\infty$, with the convention of $\expof{-\infty}=0$, to handle datasets where specific worlds are not represented. 
\red{Better: Use Theorem~\ref{the:mintermExpressivityHLN} with basemeasure dropping non appearing data.}


% Regularization
To avoid this overfitting situation, we regularize by restricting the parameter to be a set $\energyhypothesis\subset\facspace$ and state
\begin{align}\tag{$\mathrm{P}_{\energyhypothesis, \empdistribution}$}\label{prob:restrictedNaiveMLE}
	\argmax_{\canparam\in\energyhypothesis}  \sbcontraction{\canparam,\empdistribution} - \cumfunctionof{\canparam} \, . 
\end{align}

Problem~\ref{prob:restricedNaiveMLE} has two important types of instantiation, which we discuss in the next sections.

\subsubsection{Parameter Estimation}

% Parameter Estimation
\red{Projecting onto the markov logic family to the statistic $\formulaset$ is the instance of Problem~\ref{prob:restricedNaiveMLE} with the hypothesis choice}
%When the $\formulaset$ is known we take $\energyhypothesis$ as the linear hull 
	\[ \energyhypothesisof{\formulaset} = \spanof{\{\formula : \formula\in\formulaset \}} \, . \]
Then, the problem is the parameter estimation problem studied in Section~\ref{sec:parameterEstimation}.
To see this, we reparametrize by the coefficient vectors of the elements in the span, which are then understood as the canonical parameter of the respective distribution in the markov logic family to $\formulaset$.


\begin{remark}[Overparametrization]
	Taking $\formulaset$ to consist of all propositional formulas, we get a massive overparametrization: 
	The essential statistics maps to a $2^{\left(2^\atomorder \right)}$ dimensional real vector space.
	All possible distributions of the $\atomorder$ atomic variables are mapped to an $2^\atomorder-1$ dimensional submanifold, where also the essential statistics maps to.

	Thus, to identify probabilistic knowledge bases, we need to drastically restrict the shape of formulas allowed.
	It is in principle impossible to decide which formulas to be activated, based only on statistics and not on prior assumptions.

	%The nodes of a Markov Propositional Network are all formulas in a propositional theory and the hyperedges all possible decompositons.
	When having $\atomorder$ atoms, there are $2^{\atomorder}$ states in the factored system.
	Since each state can either be a model of a formula or not, there are
		\[ \cardof{\formulaset} = 2^{\big(2^\atomorder \big)} \]
	formulas.
	Having, for example, $\atomorder=10$, then $\cardof{\formulaset}>10^{308}$.


	% Regularization by sparsity
	One regularization is by allowing only a small number of formulas to be active.
	This corresponds with regularization with $\sparsityof{\canparam}$.
	The problem is then non-convex.


	% Regularization by formula size
	A further regularization strategy is the restriction of the size of the possible formulas to maintain interpretability. 
	Thus, we choose small formula selection networks.
\end{remark}




\subsubsection{Structure Learning}

% Structure Learning
The problem of structure learning arises, when the set of parameters in Problem~\ref{prob:restricedNaiveMLE} is choosen as 
	\[ \energyhypothesisof{\formulasuperset}= \bigcup_{\formulaset\in\formulasuperset} \spanof{\formulaset} \, .  \] %\energyhypothesisof{\formulaset}\, . 
In this case, the problem in general fails to be convex.

% Subspace instuition
Each formula set $\formulaset$ represents a subspace in the parameters of the minterm family, which is spanned by the propositional formulas $\exformula\in\formulaset$.

%\red{Intuition by subspaces in the minterm parameters, which are selected by a nonlinear objective, to distinguish from compressed sensing.}







\subsection{Greedy Structure Learning}


%Motivation 
It can be impracticle to learn all formulas at once, since the set $\formulasuperset$ often grows combinatorically, for example when choosing as a powerset of formulas.
\red{Further, we need to avoid overfitting and carefully choose a hypothesis.}
To avoid intractabilities and overfitting, one can choose a greedy approach and learn in addition formulas $\exformula$ when already having learned a set $\formulaset$ of formulas.
We in this section assume a current model $\currentdistribution$, which is a generic positive distribution not necessarily a Markov Logic Network. % or Hybrid Logic Network.

% 
We will use the effective selection tensor network representation of exponentially many formulas described in Chapter~\ref{cha:formulaBatches} and select from them a small subset.

%\red{Alternative discussion: Can use current distribution as base measure and apply moment matching as first order condition.}


\subsubsection{Greedy formula inclusions}

Having a current set of formulas $\formulaset$ we want to choose the best $\formula\in\fselectionmap$ to extend the set of formulas to $\formulaset\cup\{\formula\}$ in a way minimizing the cross entropy.
Given this, add each step we solve the greedy cross entropy minimization
\begin{align}\label{prob:perfectGreedy}\tag{$\mathrm{P}_{\datamap,\formulaset,\fselectionmap}$}
	\argmin_{\formula\in\fselectionmap} \argmin_{\canparam\in\rr^{\cardof{\formulaset}+1}} 
	\centropyof{\empdistribution}{\expdistof{(\formulaset\cup\{\formula\},\canparam,\basemeasure)}} \, . 
\end{align}


A brute force solution would require parameter estimation for each candidate in $\fselectionmap$.
We provide two more efficient approximative heuristics in the following (see Chapter~20 in \cite{koller_probabilistic_2009}).



\subsubsection{Gradient heuristic and the proposal distribution}

\red{Advantage: Might avoid formulawise calculus, when sampling from proposal distribution. 
Brute force solution of gain heuristic require formulawise approach.}

We now derive a heuristic of choosing features based on the maximal coordinate of the gradient when differentiating the canonical parameter in the minterm family.
To prepare for this, we build the gradient of the loss
%For the naive exponential family 
\begin{align*}
	\lossof{\expdistof{(\naivestat, \naivecanparam)}} 
	%= \frac{1}{\datanum} \sum_{\dataindexin}\lnof{\expdistofat{(\naivestat, \naivecanparam)}{\shortcatvariables=\datamapof{\dataindex}}}
	= \contraction{\empdistribution, \sencodingof{\naivestat}, \naivecanparam} - \lnof{\contraction{\expof{\contractionof{\sencodingof{\naivestat}, \naivecanparam}{\shortcatvariables}}}} 
\end{align*}
as
\begin{align*}
	\gradwrt{\naivecanparamat{\selvariable}} \lossof{\expdistof{(\naivestat, \naivecanparam)}}
	&= \contractionof{\sencodingof{\naivestat},\empdistribution}{\selvariable} - \contractionof{\sencodingof{\naivestat},\expdistof{(\naivestat, \naivecanparam)}}{\selvariable} \\
	&= \empdistribution - \expdistof{(\naivestat, \naivecanparam)} \, . 
\end{align*}

%% Single feature
%Given a feature $\exfunction[\shortcatvariables]$ we vary the naive parameters by a function on $\canparam\in\rr$ by
%\begin{align*}
%	 \naivestat(\canparam) %=  \mlntensor + \weight_{\parindices} \ftensorof{\exformula_{\parindices}}
%	= \naivestat(0) + \canparam\cdot\exfunction
%\end{align*}
%and get a likelihood gradient of
%\begin{align*}
%	 \frac{\partial \lossof{\expdistof{(\naivestat(\canparam), \naivecanparam)}}}{\partial\canparam} 
%	 &= \sbcontraction{
%	 	\frac{\partial\lossof{\expdistof{(\naivestat, \naivecanparam)}}}{\partial\naivecanparam}|_{\naivecanparam(0)},
%		\frac{\partial\naivecanparam(\canparam)}{\partial\canparam} 
%	 }  \\
%	 &= \contraction{\empdistribution,\exfunction} -   \contraction{\expdistof{(\naivestat, \naivecanparam)},\exfunction} \, .
%\end{align*}


%% Positive and Negative Search
The gradient shows the typical decomposition into a positive and a negative phase.
While the positive phase comes from the data term and prefers directions of large data support, the negative phase originates in the partition function and draws the gradient away from directions already supported by the current model $\expdistof{(\naivestat, \naivecanparam)}$.
%% Regularization functionality
The negative phase is a regularization, by comparing with what has already been learned.
When nothing has been learned so far, we can take the current model to be the uniform distribution, which is the naive exponential family with vanishing canonical parameters. 



%% Collection of features by selection
Given a set $\fselectionmap$ of features we vary $\naivecanparam$ by the function
\begin{align*}
	 \exfunction(\canparam) = \naivecanparam + \sbcontractionof{\canparam,\sencodingof{\fselectionmap}}{\shortcatvariables} \, . 
\end{align*}
At $\canparam=0$ we have the gradient of the loss of the parametrized formula by
\begin{align*}
	 \gradwrtat{\canparam}{0} 
	 \lossof{\expdistof{(\naivestat,\exfunction(\canparam),\basemeasure)}}
	 &= \sbcontraction{
	 	 \gradwrtat{\exfunction(\canparam)}{\naivecanparam}  \lossof{\expdistof{(\naivestat,\exfunction(\canparam),\basemeasure)}},
		 \gradwrtat{\canparam}{0}  \exfunction(\canparam)
	 }  \\
	 &= \sbcontractionof{\empdistribution,\sencodingof{\sstat}}{\selvariable} -   \sbcontractionof{\expdistof{(\naivestat, \naivecanparam, \basemeasure)},\sencodingof{\sstat}}{\selvariable} \, . 
\end{align*}


%% Grafting
We want to choose the formula, which is best aligned with the gradient of the log-likelihood, that is using a formula selecting map $\fselectionmap$
\begin{align} \label{prob:greedyGrad} \tag{$\mathrm{P}^{\mathrm{grad}}_{\datamap,\formulaset,\fselectionmap}$} 
	\argmax_{\selindex\in[\seldim]} \sbcontractionof{\empdistribution,\fselectionmap}{\indexedselvariable} 
	- \sbcontractionof{\expdistof{(\naivestat, \naivecanparam, \basemeasure)},\fselectionmap}{\indexedselvariable} \, . 
\end{align}
This method is known as the gradient heuristic or grafting.
% Mean parameter interpretation
The objective of Problem~\eqref{prob:greedyGrad} has another interpretation by the difference of the mean parameter $\datamean$ and $\currentmean$ of the projections of the empirical and current distributions on the family to $\fselectionmap$. % ! NOT the proposal family, those have transposed statistic

%% Formula alignment perspective
Problem~\eqref{prob:greedyGrad} is further equivalent to the formula alignment
\begin{align*}
	\argmax_{\formula\in\fselectionmap} \sbcontraction{\formula,\empdistribution-\currentdistribution} \, . 
\end{align*}
The objective can be interpreted as the difference of the satisfaction probability of the formula with respect to the empirical distribution and the current distribution.
%We can choose selection architectures to efficiently parametrize the formulas in the hypothesis $\fselectionmap$ and rewrite the problem as
%\begin{align*}
%	\argmax_{\selindexin} \contractionof{ \gradwrtat{\canparam}{\canparam=0} \lossof{\expdist}}{\indexedselvariable}
%\end{align*}
%This is thus equivalent to the problem \ref{prob:greedyGrad}, when taking all formulas selectable by $\formulaset$ as the hypothesis $\Gamma$.


%\subsection{Proposal distribution}

% Proposal distribution
Let us now understand the likelihood gradient as the energy tensor of a probability distribution, which we call the proposal distribution.

\begin{definition}[Proposal Distribution]
	Let there be a base distribution $\currentdistribution$, a targeted distribution $\empdistribution$ and a formula selecting map $\fselectionmap[\shortcatvariables, \selvariable]$.
	The proposal distribution at inverse temperature $\invtemp>0$ is the distribution of $\selvariable$ defined by
	\begin{align*}
		\normationof{\expof{\sbcontractionof{\invtemp\cdot(\empdistribution-\currentdistribution),\fselectionmap}{\selvariable}} }{\selvariable} \, . 
	\end{align*}
	The proposal distribution is the member of the exponential family with statistics $\fselectionmap$ and parameter $\invtemp\cdot(\empdistribution-\currentdistribution)$.
\end{definition}


%. Exponential family
The proposal distribution is in the exponential family with sufficient statistic by the formula selecting map $\fselectionmap$, namely the member with the canonical parameters $\canparam=\empdistribution-\currentdistribution$.
Of further interest are tempered proposal distributions, which are in the same exponential family with canonical parameters $\invtemp\cdot(\empdistribution-\currentdistribution)$ where $\invtemp>0$ is the inverse temperature parameter.

% MLN
As Markov Logic Networks, the proposal distributions are in exponential families with the sufficient statistic defined in terms of formula selecting maps.
While Markov Logic Networks contract the maps on the selection variables $\selvariable$, the proposal distributions contract them along the categorical variables $\catvariable$ to define energy tensors.

% Methods to solve mode search
The grafting Problem~\eqref{prob:greedyGrad} is the search for the mode of the proposal distribution.
To solve grafting, we thus need to answer a MAP query, for which we can apply the methods introduced in Chapter~\ref{cha:probReasoning}, such as Gibbs Sampling or Mean Field Approximations in combination with annealing.





\subsubsection{Gain Heuristic}

In the gain heuristic, only the parameters of the new formula are optimized and the others left unchanged.
This amounts to 
\begin{align}\label{prob:greedyGain}\tag{$\mathrm{P}^{\mathrm{gain}}_{\datamap,\formulaset,\fselectionmap}$}
	\argmin_{\formula\in\fselectionmap} \left ( \min_{\canparamat{\cardof{\formulaset}}\in\rr} 
	\centropyof{\empdistribution}{\expdistof{(\formulaset\cup\{\formula\},\canparam,\basemeasure)}} \right) \, . 
\end{align}
Here we denote by $\canparam$ the first $\cardof{\formulaset}$ coordinates of the M-projection $\currentdistribution$  of $\empdistribution$ onto $\formulaset$ and the variable new coordinate at position $\canparamat{\cardof{\formulaset}}$.

\begin{lemma}
	The gain heuristic objective is an upper bound on the true greedy objective. 
\end{lemma}
\begin{proof}
Since
\begin{align*}
	\argmin_{\formula\in\fselectionmap} \left( \argmin_{\canparam\in\rr^{\cardof{\formulaset}+1}} 
	\centropyof{\empdistribution}{\expdistof{(\formulaset\cup\{\formula\},\canparam,\basemeasure)}} \right)
	\leq 	\argmin_{\formula\in\fselectionmap} \left ( \argmin_{\canparamat{\cardof{\formulaset}}\in\rr} 
	\centropyof{\empdistribution}{\expdistof{(\formulaset\cup\{\formula\},\canparam,\basemeasure)}} \right) \, . 
\end{align*}
\end{proof}


% Minterm family interpretation
Further, this is Problem~\eqref{prob:restrictedNaiveMLE} in the case
\begin{align*}
	\energyhypothesis = \lnof{\currentdistribution} + \cup_{\formula\in\formulaset} \spanof{\formula} \, .
\end{align*}



% For single formula
Let us choose a formula $\formula\in\formulaset$ and consider Problem~\ref{prob:restrictedNaiveMLE}  in the case
\begin{align*}
	\energyhypothesisof{\formula} = \lnof{\currentdistribution} + \spanof{\formula} \, . 
\end{align*}
This is parameter estimation on the exponential family with the single feature $\formula$ and the base measure $\currentdistribution$.
Therefore we can apply the theory of Chapter~\ref{cha:probReasoning} and characterize the solution by the $\weight$ satisfying the moment matching condition
\begin{align*}
	\contraction{\currentdistribution, \normationof{\expof{\weight}}{\shortcatvariables} } = \contraction{\empdistribution, \formula} \, . 
\end{align*}
We state the solution of this condition in the next theorem.

\begin{theorem}
	Problem~\eqref{prob:greedyGain} is solved at any
	\begin{align*}
		\hat{\canparam} = \weightof{\hat{\formula}} \cdot \hat{\formula}
	\end{align*}
	where the formula $\hat{\formula}$ is in
	\begin{align*}
		\hat{\formula} \in \argmax_{\formula\in\formulaset} \kldivof{\sbcontraction{\empdistribution,\formula}}{\sbcontraction{\currentdistribution,\formula}}
	\end{align*}
	and $\weightof{\hat{\formula}}$ is the weight of $\hat{\formula}$ in the solution of Problem~\ref{prob:restrictedNaiveMLE} with $\Gamma = \currentdistribution + \mathrm{span}(\exformula)$.
	Here we denote by $\kldivof{p_1}{p_2}$ the Kullback-Leibler divergence between Bernoulli distributions with parameters $p_1,p_2\in[0,1]$, that is
		\[ \kldivof{p_1}{p_2} = p_1 \cdot \lnof{\frac{p_1}{p_2}} + (1-p_1) \cdot \lnof{\frac{(1-p_1)}{(1-p_2)}}  \]
\end{theorem}	
\begin{proof}
	% Solution of the problem restricted to 
	For any formula $\formula$, the inner minimum of Problem~\eqref{prob:greedyGain} is by Lemma~\ref{ref:lemMMinMLN} taken at 
		\[ \weightof{\formula} = \lnof{\frac{\datamean}{(1-\datamean)}\cdot \frac{(1-\currentmean)}{\currentmean}}  \]
	where
		\[ \currentmean = \sbcontraction{\currentdistribution,\formula} \]
	and
		\[ \datamean = \sbcontraction{\empdistribution,\formula} \, . \]
	
	The difference of the likelihood at the current distribution and the optimum is
	\begin{align*}
		\centropyof{\empdistribution}{\currentdistribution}
		- \centropyof{\empdistribution}{\expdistof{(\extendedformulaset,\extendedcanparam,\basemeasure)}}
		= \datamean \cdot \weightof{\formula} - \cumfunctionwrtof{\extendedformulaset,\basemeasure}{\extendedcanparam} \, .
	\end{align*}
	
	% Loss gain at optimum
	We use the representation scheme of Theorem~\ref{the:hybridNetworkRepresentation} and get
	\begin{align*}
		\sbcontraction{\currentdistribution, \expof{\weightof{\formula} \cdot \formula}}
		& = \sbcontraction{\currentdistribution, \rencodingofat{\formula}{\catvariableof{\formula}}, \headcoreofat{\formula}{\catvariableof{\formula}}} \\
		& = (1-\currentmean) + \currentmean\cdot \expof{\weightof{\formula}} \\
		& = (1 - \currentmean) + \frac{\datamean \cdot (1-\currentmean)}{(1-\datamean)} \\
		& = (1-\currentmean) \cdot \frac{1}{(1-\datamean)} \, . 
	\end{align*}
	% Refining the cumulant term
	It follows, that
	\begin{align*}
		\cumfunctionwrtof{\extendedformulaset,\basemeasure}{\extendedcanparam}
		& = \lnof{\sbcontraction{\currentdistribution, \expof{\weightof{\formula} \cdot \formula}}} \\
		& = \lnof{1-\currentmean} - \lnof{1-\datamean} \, . 
	\end{align*}
	% Refining the mean product term
	We further have
	\begin{align*}
		\datamean \cdot \weightof{\formula}
		= \datamean \cdot \left[ \lnof{\frac{\datamean}{(1-\datamean)}\cdot \frac{(1-\currentmean)}{\currentmean}}  \right]	
		= \datamean \lnof{\datamean} - \datamean \lnof{1-\datamean} + \datamean \lnof{1-\currentmean} - \datamean \lnof{\currentmean}
	\end{align*}
	and arrive at
	\begin{align*}
		\centropyof{\empdistribution}{\currentdistribution}
		- \centropyof{\empdistribution}{\expdistof{(\exformula,\weightof{\formula},\currentdistribution)}}
		& =  \datamean \lnof{\datamean} - \datamean \lnof{1-\datamean} + \datamean \lnof{1-\currentmean} - \datamean \lnof{\currentmean}
		-  \lnof{1-\currentmean} - \lnof{1-\datamean} \\
		& = \left( -\datamean \lnof{\currentmean} - (1-\datamean) \lnof{1-\currentmean} \right)  - \left( -\datamean \lnof{\datamean} - (1-\datamean) \lnof{1-\datamean} \right) \, . 
	\end{align*}
	By definition, this is the Kullback-Leibler divergence between Bernoulli distributions with parameters $\datamean$ and $\currentmean$.
	%
	Since the gain in the likelihood loss when restricting to $\energyhypothesis = \spanof{\formula}$ is thus given by $\kldivof{\sbcontraction{\empdistribution,\formula}}{\sbcontraction{\currentdistribution,\formula}}$, we have that Problem~\ref{prob:restrictedNaiveCE}  in the case $\energyhypothesis = \bigcup_{\formula\in\formulaset}\spanof{\formula}$ is solved at $\estcanparam = \weightof{\hat{\formula}}\cdot \hat{\formula}$ where
		\[ \hat{\formula} = \kldivof{\sbcontraction{\empdistribution,\formula}}{\sbcontraction{\currentdistribution,\formula}} \, . \]
\end{proof}

\red{Thus, we solve the grain heuristic with a coordinatewise transform of the mean parameter tensors to $\empdistribution$ and $\currentdistribution$, using the Bernoulli Kullback-Leibler divergence as transform function.}


% Interpretation
One therefore takes the formula, which marginal distribution in the current model and the targeted distribution are differing at most, measured in the KL divergence.

% Optimization method
One optimization method would thus be the computation of the mean parameters to both distribution, building the coordinatewise KL divergence and choosing the maximum. 
Since we need to evaluate each coordinate, this can be intractable for large sets of formulas.


% Further weight optimization
Further improvement of the model can be achieved by iteratively optimizing the other weights as well, since their corresponding moment matching conditions might be violated after the integration of a new formula.
This would require the computation of backward mappings for each candidate formula, for which we only have an alternating approach in general.


\subsubsection{Iterations}

Let us now iterate the search for a best formula at a current model with the optimization of weights after each step.
The result is Algorithm~\ref{alg:greedyStructureLearning}, which is a greedy algorithm adding iteratively the currently best feature.

\begin{algorithm}[hbt!]
\caption{Greedy Structure Learning}\label{alg:greedyStructureLearning}
\begin{algorithmic}
	\State Initialize
		\[ \currentdistribution \algdefsymbol \frac{1}{\prod_{\catenumeratorin}\catdimof{\atomenumerator}} \cdot \onesat{\shortcatvariables} \quad, \quad \formulaset = \varnothing \]
	\While{Stopping criterion is not met}
		\State Structure Learning: Compute a (approximative) solution $\hat{\formula}$ to Problem~\ref{prob:restrictedNaiveMLE} and add the formula to $\formulaset$, i.e.
				\[ \formulaset \algdefsymbol \formulaset \cup \{\hat{\formula}\} \]
			Extend dimension of $\selvariable$ by one, by $\formulaof{\seldim}=\hat{\formula}$ and $\canparamat{\seldim}=0$
		\State Weight Estimation: Estimate the best weights for the added formula and recalibrate the weights of the previous formulas, by calling Algorithm~\ref{alg:AWO}.
				\[ \currentdistribution \algdefsymbol \expdistof{\formulaset, \canparam} \]
\EndWhile
\end{algorithmic}
\end{algorithm}



%% Energy Storage -> Useful after learning for energy-based inference
When having used the same learning architecture multiple times, the energy of the corresponding formulas are all representable by a formula selecting architecture.
Their energy term is therefore a contraction of the selecting tensor with a parameter tensor $\canparam$ in a basis CP decomposition with rank by the number of learned formulas.
When mutiple selection architectures have been used, the energy is a sum of such contractions.
% 
Let us note, that this representation is useful after learning, when performing energy-based inference algorithms on the result.
During learning, one needs to instantiate the proposal distribution, which requires instantiation of the probability tensor.
\red{However, one could alternate data energy-based and use this as a particle-based proxy for the probability tensor.}


\begin{remark}[Sparsification by Thresholding]
	To maintain a small set of active formulas, one could combine greedy learning approaches with thresholding on the coordinates of $\canparam$.
	This is a standard procedure in Iterative Hard Thresholding algorithms of Compressed Sensing, but note that here we do not have a linear in $\canparam$ objective.
\end{remark}


\subsection{Discussion}

\begin{remark}[Bayesian approach]
	We only treated the estimation of a single resulting distribution by the data, while in a Bayesian approach one typically considers an uncertainty over possible distributions.
	% MAP
	\red{When treating $\canparam$ as a random tensor, which prior distribution is given and posteriori distribution wanted, we have a more involved Bayesian approach.}
	When having a prior $\probof{\mlnparameters}$ over the Markov Logic Networks we alternatively want to find the parameters $\mlnparameters$ solving the maximum a posteriori problem
	\begin{align}
		\argmax_{\mlnparameters} \mlnprobat{\data}\cdot \probof{\mlnparameters}\, . 
	\end{align}
\end{remark}




% Success Guarantees
\section{Probabilistic Success Guarantees}\label{cha:mlnConcentration}

When drawing data independently from a random distribution, we in this chapter derive guarantees, that the 

%Uniform concentration bounds require concentration bounds on 
%	\[ \sbcontraction{\theta,\fluctuationtensor} \]
%for tensors $\theta\in\Gamma$.
%
%To show such bounds we use the formula decomposition of any $\theta$ into a set
%	\[ \sum_{\mlnformulain} \weightof{\exformula}\exformula \, .\]	
	
	
\subsection{Fluctuations}

A random tensor is a random element of a tensor space $\facspace$, drawn from a probability distribution on $\facspace.$
In contrast to the discrete distributions investigated previously in this work, the random tensors are in most generality continuous distributions. % However, when drawing data they are 

\subsubsection{Fluctuation of the empirical distribution}

% Random one hot encodings
When drawing random states $\datamapof{\dataindex}\in\facstates$ by a distribution $\gendistribution$, we use the one-hot encoding to forward each random state to the random tensor
	\[ \onehotmapofat{\datamapof{\dataindex}}{\shortcatvariables} \, . \]
The expectation of this random tensor is
\begin{align*}
	\expectationof{\onehotmapof{\datamapof{\dataindex}}} 
	= \sum_{\catindices\in\facstates} \gendistribution[\indexedcatvariableof{[\atomorder]}] \onehotmapofat{\catindices}{\shortcatvariables} 
	= \gendistribution[\shortcatvariables] \, . 
\end{align*}
	
The empirical distribution is then the average of independent random one-hot encodings, namely the random tensor
	\[ \empdistribution = \frac{1}{\datanum} \sum_{\dataindexin}  \onehotmapof{\datamapof{\dataindex}} \, . \]
To avoid confusion let us strengthen, that in this chapter we interpret $\empdistribution$ as a random tensor taking values in $\facspace$, whereas each supported value of $\empdistribution$ is an empirical distribution taking values in $\facstates$.


% Expectation -> Does not make use of independence here!
When the marginal of each datapoint is $\gendistribution$, the expectation of the empirical distribution is
\begin{align*}
	\expectationof{\empdistribution} 
	= \frac{1}{\datanum} \sum_{\dataindexin}  \expectationof{\onehotmapof{\datamapof{\dataindex}}}
	= \gendistribution \, . 
\end{align*}

% Law of large numbers
From the law of large numbers it follows, that in the limit of $\datanum\rightarrow\infty$ at any coordinate $\catindex\in\facstates$ almost everywhere
	\[ \empdistribution[\indexedcatvariableof{[\atomorder]}] \rightarrow \expectationof{\empdistribution[\indexedcatvariableof{[\atomorder]}]} =  \gendistribution[\indexedcatvariableof{[\atomorder]}] \, . \]

% Fluctuation
At finite $\datanum$ the empirical distribution differs from the by the difference
	\[ \empdistribution - \gendistribution \]
which we call a fluctuation tensor.


\subsubsection{Fluctuation tensors and their widths}

Let us now investigate random tensors, which result from the forwarding of the fluctuation of the empirical distribution by sufficient statistics.

\begin{definition}
	Given a statistic $\sstat$, $\datanum\in\nn$ and and a dataset we define the fluctuation tensor as the random tensor
		\[ \expfamfluctuation = \contractionof{\{\empdistribution-\gendistribution,\sstat\}}{\selvariable} \]
	where $\datamap$ is a collection of $\datanum$ independent samples of $\gendistribution$.
\end{definition}

% Naive Ex
The fluctuation of the empirical distribution around the generating distribution corresponds in this notation with the naive exponential family, taking the identity as statistics.
% Appearances
Besides this, fluctuation tensors appears in Markov Logic Networks as fluctuations of random mean parameters and in proposal distributions as fluctuation of random energy tensor.
We will discuss these examples in the following sections.


\subsubsection{Naive Exponential Family}

\red{This is the minterm exponential family!}

In case of the naive exponential family, we have $\sstat=\identityat{\shortcatvariables,\selvariable}$ and the fluctuation tensor is
	\[ \naivefluctuation = \empdistribution - \gendistribution \, .  \]

% Multinomial
This fluctuation tensor is related to tensor encodings of multinomial distributions, which we now define as multinomial random tensors.

\begin{definition}
	A multinomial random tensor is the sum of the one-hot encodings of independent random variables $Z_\dataindex$ each distributed by $\probtensor$
		\[ Z^{\probtensor, \datanum} = \sum_{\dataindexin} \onehotmapof{Z_\dataindex} \, . \] 
\end{definition}

\begin{lemma}\label{lem:multinomialEmpdistFluctuation}
	The fluctuation $\empdistribution - \gendistribution$ is a by $\frac{1}{\datanum}$ rescaled centered multinomial random tensor with parameters $\gendistribution$ and $\datanum$. % Needs some more explanation based on one-hot encodings?
\end{lemma}
\begin{proof}
	By the above construction we have
		\[  \empdistribution - \gendistribution = \frac{1}{\datanum} \left( \onehotmapofat{\datamapof{\dataindex}}{\shortcatvariables} - \expectationof{\onehotmapofat{\datamapof{\dataindex}}{\shortcatvariables}} \right) \, .  \]
\end{proof}


\subsubsection{Mean parameter in Markov Logic Networks}

The mean parameter of the M-projection of the empirical distribution on the family of Markov Logic Networks with statistic $\fselectionmap$ is the random tensor
\begin{align*}
	\meanparam^\datamap =  \sbcontractionof{\mlnstat,\empdistribution}{\selvariable} \, . 
\end{align*}

The expectation of this random tensor is
\begin{align*}
	\expectationof{\meanparam^\datamap} 
	=  \sbcontractionof{\mlnstat,\expectationof{\empdistribution}}{\selvariable} 
	=  \sbcontractionof{\mlnstat,\gendistribution}{\selvariable} 
	=  \meanparam^* \, ,  
\end{align*}
where we used that the expectation and contraction operation can be commuted due to the multilinearity of contractions.

% Fluctuation of mean parameter
The fluctuation of this mean parameter is
\begin{align*}
	\meanparam^\datamap - \expectationof{\meanparam^\datamap} =  \sbcontractionof{\mlnstat,\empdistribution-\gendistribution}{\selvariable} \, . 
\end{align*}
We notice, that this is the fluctuation tensor $\mlnfluctuation$.


%\begin{example}[Proposal distribution]
\subsubsection{Energy tensor in proposal distributions}

The fluctuation tensor appears as a fluctuation of the energy of the proposal distribution.
For the expected energy it holds that
\begin{align*}
	\expectationof{\energytensorof{\proposalstat,\empdistribution-\currentdistribution}} 
	= \expectationof{\sbcontractionof{\proposalstat,\empdistribution-\currentdistribution}{\selvariable}} 
	= \sbcontractionof{\proposalstat,\expectationof{\empdistribution-\currentdistribution}}{\selvariable}
	= \sbcontractionof{\proposalstat,\gendistribution-\currentdistribution}{\selvariable} 
	= \expectationof{\energytensorof{\proposalstat,\gendistribution-\currentdistribution}} \, . 
\end{align*}

% Fluctuation
The fluctuation of this random tensor is
\begin{align*}
	\expectationof{\energytensorof{\proposalstat,\empdistribution-\currentdistribution}}  - \expectationof{\energytensorof{\proposalstat,\gendistribution-\currentdistribution}} 
	= \expectationof{\energytensorof{\proposalstat,\empdistribution-\gendistribution}}
\end{align*}
and coincides with $\proposalfluctuation$.
	




\subsubsection{Binary Features}

In all the above example we have statistics consistent of binary features.
In this case the marginal distributions of the coordinates of $\expfamfluctuation$ are scaled and centered binomials, which we investigate now.

\begin{lemma}
	Let $\sstat$ be a statistic and let for $\statenumeratorin$ $\sstat_\statenumerator$ be a binary feature, i.e. let $\imageof{\sstat_\statenumerator}\subset \{0,1\}$.
	Then, the marginal distribution of the coordinate $\expfamfluctuation[\selvariable=\statenumerator]$ is 
		\[\frac{1}{\datanum}\left(\bidistof{\fprobof{\statenumerator},\datanum}- \fprobof{\statenumerator}\right)  \, , \] 
	where by $\bidistof{\fprobof{\statenumerator},\datanum}$ we denote the binomial distribution with mean parameter
		\[ \fprobof{\statenumerator} = \sbcontraction{\sstat_\statenumerator, \gendistribution} \, . \]
\end{lemma}
\begin{proof}
	We notice that when forwarding a random sample $\datamapof{\dataindex}$ of $\gendistribution$ is the random tensor
		\[ \onehotmapofat{\datamapof{\dataindex}}{\shortcatvariables} \, \]
	and since $\imageof{\sstat_\statenumerator}\subset \{0,1\}$ the contraction
		\[ \sbcontraction{\sstat_\statenumerator, \onehotmapofat{\datamapof{\dataindex}}{\shortcatvariables}} \]
	is a random variable taking values in $\{0,1\}$.
	The variable therefore follows a Bernoulli distribution with mean parameter
		\[ \fprobof{\statenumerator} = \expectationof{\sbcontraction{\sstat_\statenumerator, \onehotmapofat{\datamapof{\dataindex}}{\shortcatvariables}}} = \sbcontraction{\sstat_\statenumerator, \gendistribution}  \, . \]
\end{proof}


%\begin{lemma}	
%	Let $\exformula$ be a tensor with coordinates in $\{0,1\}$.
%	The random variable $\sbcontraction{\exformula,\fluctuationtensor}$ is distributed by 
%		\[ \frac{1}{\datanum}\left(\bidistof{\fprobof{\exformula},\datanum}- \fprobof{\exformula}\right)  \]
%	where $\bidistof{\fprobof{\exformula},\datanum}$ is the binomial distribution with $\fprobof{\exformula}$ being the probability of $\exformula$ satisfied.
%\end{lemma}
%\begin{proof}
%	For a $\datapoint$ the contraction
%		\[ \braket{\exformula,\datapoint} \]
%	follows a Bernoulli distribution with 
%		\[  \probof{\braket{\exformula,\datapoint}=1} = \probof{\exformula = \mathrm{true}} \, . \]	
%	With the assumption of independent data, these scalar products sum up to a Binomial, which is then centered and averaged.
%\end{proof}
%
%If the data is generated by $\mlnprobof{\cdot}{\expsolution}$ we have
%	\[ \fprobof{\exformula} = \frac{1}{\partitionfunctionof{\expsolution}}  \braket{\exformula,\expof{\expsolution}} \, . \]


\subsubsection{Widths of random tensors}

% Widths
In the following we will use the supremum of contractions with random tensors in the derivation of success guarantees for learning problems.
Such quantities are called widths.

\begin{definition}
	Given a set $\Gamma\subset\facspace$ and $\fluctuationtensor$ a random tensor taking values in $\facspace$ we define the width as the random variable
		\[ \widthwrtof{\Gamma}{\fluctuationtensor} = \sup_{\theta\in\Gamma} \absof{\sbcontraction{\theta,\fluctuationtensor}} \, . \]	
\end{definition}



\subsection{Expected and Empirical Risk Minimization}

\red{We take a frequentist approach here and study the distributions of estimated parameters depending on random data.}

\subsubsection{Maximum Likelihood Estimation on random data}

% Aim: Recovery Guarantees
We here investigate the statistical errors of the maximum likelihood estimators.
The empirical distribution is understood as an estimation of an underlying distribution.
When sampling by independent copies of the underlying distribution, its mean is the underlying distribution.
However, due to fluctuations around this mean the solution of estimation problems with respect to the empirical distribution does in general not coincide with the solution given the underlying distribution.

%
To be more precise, when generating data $\data$ by independent copies of a distribution $\gendistribution$ we define the minimization problems
\begin{align}\tag{$P_{\data,\Gamma}$}\label{prob:empEstimation}
	\empsolution = \argmin_{\theta\in\Gamma} \centropyof{\empdistribution}{\expdistof{\theta}}
\end{align}
and
\begin{align}\tag{$P_{\mathbb{E},\Gamma}$}\label{prob:expEstimation}
	\expsolution = \argmin_{\theta\in\Gamma} \centropyof{\gendistribution}{\expdistof{\theta}}
\end{align}
where by $\expdistof{\theta}$ we denote the element of an exponential family with sufficient statistics $\sstat$ and parameter $\theta$.


%% Examples
%In feature calibration $\sstat$ is the concatenation of all formulas and $\Gamma$ is the vector space $\rr^{\cardof{\formulaset}}$ of possible weigths.
%In feature selection we would choose $\sstat$ by a formula selecting tensor network and $\Gamma$ as the set of basis tensors representing the selection of a specific formula. %% TRUE ?? Doing Gradient 
%We then optimize 


% Concentration
We have at each hypothesis $\theta\in\Gamma$
\begin{align}
	\expectationof{\centropyof{\empdistribution}{\expdistof{\theta}}} = \centropyof{\gendistribution}{\expdistof{\theta}}
\end{align}
and thus, the objective in Problem~\ref{prob:empEstimation} converges in the limit $\datanum\rightarrow\infty$ by the law of large numbers at each hypothesis $\theta\in\Gamma$ to the objective in Problem~\ref{prob:expEstimation}.

To use this insight in the derivation of bounds of the distance of $\empsolution$ and $\expsolution$, we need to quantifying this convergence 
\begin{itemize}
	\item Non-asymptotically: Since we typically have access to limited amounts of data, that is finite $m$, we need to quantify the concentration in non-asymptotic cases.
	\item Uniform: Since the problems are optimized at extreme situations, the convergence of the objective has to happen uniformally at multiply $\theta\in\Gamma$.
\end{itemize}

%% Formalization 
%To be more precise let $\expsolution$ be the tensor representing the MLN $\mlntrueparameters$ and $\Gamma$ a hypothesis tensors.
%We then seek to get the tensor $\mlntrueparameters$ maximizing the likelihood function by solving
%\begin{align}\tag{$P_{\loss,\Gamma}$}\label{prob:empMLNrecovery}
%	\argmin_{\theta\in\Gamma} - \variablesum\log\mlnprobof{\datapointof{\variableindex}}{\theta}
%\end{align}

%This is an empirical risk minimization problem given the loss function 
%	\[ \lossof{\theta} = - \probof{\cdot | \theta}\]
%and the empirical risk
%	\[ \loss_{\data}\left( \theta \right) = - \variablesum\log\mlnprobof{\datapointof{\variableindex}}{\theta} \, . \]
%
%In our probabilistic analysis we assume, that the datapoints are drawn independently from a Markov logic network with parameters $\mlntrueparameters$.
%Then we derive recovery guarantees based on nonasymptotic convergence bounds of $\loss_{\data} $ to its expectation.


\subsubsection{Solution of the Expected Problem}

% Motivation
Let us first investigate the solution of Problem~\ref{prob:expEstimation}, to get a reference for Problem~\ref{prob:empEstimation}.

%The expectation of the loss is then
%\begin{align}
%	\expectationof{\variablesum\log\mlnprobof{\datapointof{\variableindex}}{\mlnparameters}} 
%	= \expectationof{\log\mlnprobof{\datapointof{}}{\mlnparameters}}
%\end{align}
%where the expectation is performed over $\datapointof{}$ distributed by $\mlnprobof{\datapointof{}}{\mlntrueparameters}$.
%This is the cross entropy between the generative distribution by $\mlntrueparameters$ and the hypothesis $\mlnparameters$.
%
%The expected loss analogon to Problem \ref{prob:empMLNrecovery} is
%\begin{align}\tag{$P_{\expectationof{\loss},\Gamma}$}\label{prob:expMLNrecovery}
%	\argmin_{\theta\in\Gamma} \expectationof{\log\mlnprobof{\datapointof{}}{\theta}}
%\end{align}

The solution does not change when manipulating the objective as
\begin{align}
		\argmin_{\theta\in\Gamma} \centropyof{\gendistribution}{\expdistof{\theta}}
		= \argmin_{\theta\in\Gamma} \centropyof{\gendistribution}{\expdistof{\theta}} -  \centropyof{\gendistribution}{\gendistribution}
\end{align}

We notice that the objective of the right side problem is the Kullback-Leibler divergence
	\begin{align}
		 \kldivof{\gendistribution}{\expdistof{\theta}}
%		\kldivof{\expdistof{\expsolution}}{\expdistof{\theta}} = \expectationof{\log
%		\left[
%		\frac{
%		\mlnprobof{\datapointof{}}{\expsolution} 
%		}{
%		\mlnprobof{\datapointof{}}{\theta}
%		}
%		\right]} \, .
	\end{align}
Thus, Problem~\ref{prob:expEstimation} coincides with the approximation of $\gendistribution$ by an element $\expdistof{\theta}$ with $\theta\in\Gamma$ in the Kullback-Leibler divergence.

We decompose:
\begin{align}
	\kldivof{\expdistof{\expsolution}}{\expdistof{\theta}} 
	= \log\left[\frac{\partitionfunctionof{\theta}}{\partitionfunctionof{\expsolution}}\right] 
	+ \frac{1}{\partitionfunctionof{\expsolution}} \braket{\expsolution-\theta,\expof{\expsolution}}
\end{align}
Some insights can be drawn based on this decomposition:
\begin{itemize}
	\item The first term vanishes, when both partition functions are the same.
		We can always adjust $\theta$ by constant offsets on all coordinates (of course without changing the distribution), such that the partition functions are equal.
		This is done by the map
			\[ \theta \rightarrow \theta + \lambda \cdot \ones \]
		where we choose $\lambda\in\rr$ by
			\[ \lambda = \frac{\partitionfunctionof{\expsolution}}{\partitionfunctionof{\theta}} \]
		to ensure a vanishing first term.		
	\item In the second term the differences in coordinates are weighted by exponentiated solution $\expsolution$.
		Where the probability mass is small errors in $\theta$ have a small influence on the Kullback-Leibler divergence.
\end{itemize}


\begin{theorem}
	Let us assume $\gendistribution = \expdistof{\expsolution}$ for some $\expsolution\in\Gamma$. 
	Then one solution of Problem \ref{prob:expEstimation} coincides with $\expsolution$ up to constant offsets.
\end{theorem}
\begin{proof}
	This follows directly from the Gibbs inequality, since 
	\begin{align}
		\kldivof{\expdistof{\expsolution}}{\expdistof{\theta}} \leq 0
	\end{align}
	with equality if only if $\expdistof{\expsolution}$ and $\expdistof{\theta}$ are equal.
	The uniqueness follows from Theorem \ref{the:tensorRepUniqueness}
\end{proof}









\subsubsection{Recovery Guarantee based on Widths}

\begin{theorem}
	Let us assume $\expsolution\in\Gamma$ and that $\partitionfunctionof{\theta}$ is constant among $\theta\in\Gamma$.
%	We define
%	\begin{align}
%		\omega_{\Gamma} = \sup_{\theta\in\Gamma} \braket{\theta-\expsolution, \expectationof{\datapointof{}} - \variablesum\datapointof{\variableindex}}
%	\end{align}
	Then for any solution $\hat{\theta}$ of the empirical problem we have
	\begin{align}
		\kldivof{\expdistof{\theta^*}}{\expdistof{\hat{\theta}}} \leq \widthwrtof{\Gamma}{\fluctuationtensor} \, .
	\end{align}
\end{theorem}
\begin{proof}
	First we notice
	\begin{align}
		\argmin_{\theta\in\Gamma} \loss\theta = \argmin_{\theta\in\Gamma} \loss\theta - \loss\expsolution 
	\end{align}
	When $\expsolution\in\Gamma$ the minimum of the empirical loss with respect to $\expsolution$ is negative since
	\begin{align}
		\loss\empsolution - \loss\expsolution \leq \loss\expsolution-\loss\expsolution \leq 0
	\end{align}
	We separate expectations and fluctuations and get
	\begin{align}
		\kllossof{\empsolution} \leq \kllossof{\empsolution} - (\loss\empsolution - \loss\expsolution) \leq \omega_{\Gamma} \, .
	\end{align}
\end{proof}


%We recognize $\omega_\Gamma$ to be a width of the random tensor
%	\[ \fluctuationtensor  = \expectationof{\datapointof{}}-\variablesum\datapointof{\variableindex}  \, . \]



% Width 
The supremum of the differences between expected and empirical risks is the width of the fluctuation tensor, as we state next.

\begin{lemma}
	For any $\Gamma$ and $\datamap$ we have
	\begin{align*}
		\widthwrtof{\Gamma}{\fluctuationtensor^{\identity, \gendistribution, \datamap}} 
		= \sup_{\theta\in\Gamma} \centropyof{\empdistribution}{\expdistof{\theta}} - \centropyof{\gendistribution}{\expdistof{\theta}} 
	\end{align*}
\end{lemma}
\begin{proof}
	Using the decomposition of cross entropy in the naive exponential family 
	 	\[ \centropyof{\empdistribution}{\expdistof{\theta}}=\contractionof{\probtensor,\lnof{\gendistribution}} - \cumfunctionof{\lnof{\gendistribution}} \, . \]
\end{proof}


\begin{corollary}
	At the solution $\hat{\theta}$ of Problem~\ref{prob:empEstimation} we have
		\[ \centropyof{\empdistribution}{\expdistof{\hat{\theta}}} - \centropyof{\gendistribution}{\expdistof{\hat{\theta}}} \leq  \widthwrtof{\Gamma}{\fluctuationtensor^{\identity, \gendistribution, \datamap}} \, . \] 
\end{corollary}






\subsection{Guarantees for Mode of the Proposal Distribution}

Let us now derive probabilistic guarantees, that the mode of the proposal distribution at the empirical and the generating distribution are equal.

\begin{definition}
	The max gap of a tensor $V$ is the quantity
		\[ \maxgap(V) = \min_{i\neq i^{max}} V[X = i^{max}] - V[X=i] \]
	where
		\[ i^{max} \in \argmax_{i} V[X = i] \, . \]
\end{definition}

If multiple maxima exist, the



\begin{theorem}\label{the:probGuaranteeProposalDist}
	Whenever the energy tensor of the expected proposal distribution has a gap of $\maxgap$, then for every $\failprob>0$ the mode of the expected proposal distribution coincides with the empirical proposal distribution with probability at least $1-\expof{-\frac{1}{\failprob^2}}$, provided that
		\[ \datanum > C\frac{\left(\sum_{\atomenumeratorin}\lnof{\catdimof{\atomenumerator}}\right)}{\maxgap^2} \]
	where $C$ is a universal constant.
\end{theorem}

To proof the theorem we first use a deterministic recovery guarantee involving the width of the fluctuation tensor and then apply the width bound of Theorem~\ref{the:basisTensorWidthBound}.

\begin{lemma}\label{lem:detGuaranteeProposalDist}
	Whenever 
	\begin{align*}
		\maxgap(\contractionof{\{\gendistribution,\fselectionmap\}}{\selvariable}) 
		> 2 \cdot  \widthatof{\{\onehotmapof{\catindices} :\catindices\in\facstates\}}{\fluctuationtensor^{\fselectionmap,\gendistribution,\datamap}} \, , 
	\end{align*}
	then the mode of the proposal distribution to the empirical distribution coincides with the mode of the proposal distribution to the generating distribution.
\end{lemma}
\begin{proof}
	If different, then the expected objective at the solutions of the empirical and expected is at least $\maxgap$.
	But at the empirical, the difference it as most twice the width different, so that is a contradiction to the assumption.
\end{proof}


\begin{proof}[Proof of Theorem~\ref{the:probGuaranteeProposalDist}]
	Given the assumed bound, the sub-gaussian norm of the width is upper bounded by $C_2\cdot \maxgap$, thus for any $\failprob>0$ we have
		\[  \widthatof{\{\onehotmapof{\catindices} :\catindices\in\facstates\}}{\fluctuationtensor^{\fselectionmap,\gendistribution,\datamap}}  < 2 \maxgap \]
	with probability at least $1-\expof{-\frac{1}{\failprob^2}}$.
	The claim thus follows with Lemma~\ref{lem:detGuaranteeProposalDist}.
\end{proof}




\begin{example}[Gap of a MLNs with single formulas]
	Let there be the MLN of a maxterm $\formula$ with $\atomorder$ variables, and let $\formulaset$ be the maxterm selecting tensor, then 
		\[ \maxgap(
		%\energytensorof{(\{\formula\},\weightof{\formula})}
		\energytensorof{(\formulaset, \expdistof{(\{\formula\},\weightof{\formula})} - \normationof{\ones}{\shortcatvariables} )}
		) = \frac{1}{2^{\atomorder}-1 + \expof{-\weightof{\formula}}}  \]
	If $\weightof{\formula}>0$ we have an exponentially small gap.
	Thus, for the above Lemma to apply, the width needs to be exponentially in $\atomorder$ small.
	
	
	Let there be the MLN of a minterm $\formula$ with $\atomorder$ variables, then 
		\[ \maxgap(
%		\energytensorof{(\{\formula\},\weightof{\formula})}
		\energytensorof{(\formulaset, \expdistof{(\{\formula\},\weightof{\formula})} - \normationof{\ones}{\shortcatvariables} )}
		) = \frac{1}{1+(2^{\atomorder}-1)\cdot\expof{-\weightof{\formula}}}  \]
	For large $\weightof{\formula}$ and $\atomorder$, the gap tends to $1$.
\end{example}






%%% OLD
%\begin{theorem}
%	When the expected sufficient statistics of $\gendistribution$ is gapped by $\maxgap>0$ we  have
%		\[ \expsolution = \empsolution \]
%	with probability at least
%		\[ 1- XXXX \, . \]
%\end{theorem}
%\begin{proof}
%	We show the theorem by proofing that within the stated probabilistic bounds we have
%		\[ \maxgap > 2 \widthatof{\Gamma}{\fluctuationtensor} \, . \]
%	In case of this event, it follows that $\expsolution = \empsolution$.
%	The probabilistic bound can be shown based on the sub-Gaussian norm of $\widthatof{\Gamma}{\fluctuationtensor}$ bounded in Chapter~\ref{cha:widthBounds}.
%\end{proof}






\subsection{Guarantees for Parameter Estimation}

\red{This is mean parameter fluctuation interpretation of the random tensor.}


\begin{lemma}\label{lem:meanParamDistance}
	For any $\fselectionmap$ and $\datamap$ drawn from $\gendistribution$ we have
	\begin{align*}
		\normof{\meanparam^\datamap - \meanparam^*} 
		= \widthwrtof{\subsphere}{\fluctuationtensor^{\fselectionmap,\gendistribution,\datamap}} \, ,
	\end{align*}
	where $\meanparam^\datamap=\sbcontractionof{\fselectionmap,\empdistribution}{\selvariable}$ and $\meanparam^*=\sbcontractionof{\fselectionmap,\gendistribution}{\selvariable}$.
\end{lemma}

%
We can thus apply the sphere bounds.


\begin{theorem}
	For any $\failprob\in(0,1)$ we have the following with probability at least $1-\failprob$.
	Let $\hat{\canparam}$ and $\precision>0$, then
		\[ \absof{\centropyof{\gendistribution}{\expdistof{\empsolution}} - \centropyof{\empdistribution}{\expdistof{\empsolution}}} \leq \tau \cdot \normof{\empsolution} \]
	provided that
		\[ \datanum \geq \frac{\sbcontraction{\meanparam^*}-\sbcontraction{(\meanparam^*)^2}}{\failprob \precision^2} \, . \]
\end{theorem}
\begin{proof}
	We have by Cauchy Schwartz 
		\[ \absof{\sbcontraction{\meanparam^\datamap - \meanparam^*,\empsolution}} \leq \normof{\meanparam^\datamap - \meanparam^*} \cdot \normof{\empsolution}\]
	and with Lemma~\ref{lem:meanParamDistance}
		\[ \absof{\sbcontraction{\meanparam^\datamap - \meanparam^*,\empsolution}} \leq \widthwrtof{\subsphere}{\fluctuationtensor^{\fselectionmap,\gendistribution,\datamap}} \cdot \normof{\empsolution} \, . \]
	We show in Part III that in Theorem~\ref{the:sphereBoundVariance} that
		\[  \widthwrtof{\subsphere}{\fluctuationtensor^{\fselectionmap,\gendistribution,\datamap}} \leq \tau \]
	when $\datanum$ satisfies the assumed lower bound, from which the claim follows.
\end{proof}








% Structured Extensions
\chapter{\chatextfolModels}\label{cha:folModels}

We now extend the tensor representation from to structured representations, whereas we previously focused on factored representation of systems.
%The models/events in this situation are precise relations between objects.


\red{We observe that the more expressive first-order logic bears another tensor structure:
The representation of each world is a boolean tensor.
}


% Formulas in FOL
%Formulas in first order logic can contain variables, which are placeholder for specific individuals.
%Given a model and an assignment of objects to the arguments of a formula, the truth of the formula can be interpreted.
%This truth interpretation defines thus for any model a tensor, which we call the grounding tensor.

%
\sect{World Tensors}

% Index interpretation of world domain
Since first-order logic follows structured representations of a system, a first-order logic world consists in objects and relations between them.
To each world there is a world domain $\worlddomain$ of objects, which we assume to be finite (this is a restrictive assumption).
We exploit the set-encoding formalism discussed in more detail in \charef{cha:basisCalculus} and use bijective index interpretation maps
\begin{align*}
    \indexinterpretation : [\inddim] \rightarrow \worlddomain \, .
\end{align*}
A so-called term variable $\indvariable$ takes states $\indindexin$, which represent objects
\begin{align*}
    \indexinterpretationat{\indindex} \in \worlddomain \, .
\end{align*}

%
The relations between objects are described by $\indorder$-ary predicates $\folpredicate$.
Given a specific world $\dataworld$ the truth of relations is represented by boolean tensors
\begin{align*}
    \groundingof{\folpredicate} : \symindstates\rightarrow\ozset \, .
\end{align*}
Given a tuple $\indindexlist\in\symindstates$ the boolean
\begin{align*}
    \groundingofat{\folpredicate}{\indexedindvariableof{0},\ldots,\indexedindvariableof{\indorder-1}} \in\ozset
\end{align*}
is called a grounding and encodes, whether the relation $\folpredicate$ is satisfied in the world $\dataworld$ for the objects $\invindexinterpretationat{\indindexof{0}},\ldots,\invindexinterpretationat{\indindexof{\indorder-1}}$.

% Assumptions
Let us assume, that we have a function-free theory with $\folpredicateorder$ predicates, where are predicates all of the same arity $\variableorder$.
We then formalize a world in the following based on a selection variable $\selvariable$ selecting a specific predicate and term variables $\shortindvariablelist=\indvariablelist$ representing choices of objects from a given set $\worlddomain$.

\begin{definition}[FOL World]
    \label{def:folWorld}
    Given a set of objects $\worlddomain$ enumerated by an index interpretation function $\indexinterpretation:[\inddim]\rightarrow \worlddomain$ and a finite set $\{\folpredicates\}$ of $\variableorder$-ary predicates a world is a boolean tensor
    \begin{align}
        \dataworldwith : [\catorder] \times \left( \symindstates\right) \rightarrow [2] \, . % ! Selvariable gets catorder !
    \end{align}
    We interpret the world tensor as encoding in the coordinate $\dataworldat{\selvariable=\catenumerator,\indexedshortindvariables}$, whether the $\catenumerator$-th predicate is satisfied on the object tuple $\invindexinterpretationat{\indindexof{0}},\ldots,\invindexinterpretationat{\indindexof{\indorder-1}}$.
\end{definition}


% Inclusion of functions, predicates of differing order
When the assumptions of function-free and constant variable order are not met, we can do the following tricks.
Functions are turned to predicates by their relation interpretation.
If there are predicates of different arity in the theory, we can trivially extend them to $\variableorder$ary predicates by tensor products with the trivial tensor $\ones$.
This can be done by a tensor product with $\onehotmapofat{\inddim}{\indvariable}$, where we add an auxiliary object $\indexinterpretationof{\inddim}$ as a placeholder for predicates with smaller arity.

% Finite worlds -> By database semantics?
While in first order logics, depending on the chosen semantics, worlds can have infinite sets of objects, we here only treat worlds with finite objects.

\subsect{Case of Propositional Logics}

%
Before continuing with the one-hot encoding of first-order logic worlds, let us show that the previously discussed formalism of propositional logics (see \charef{cha:logicalRepresentation}) is a special case of first-order logics, namely when demanding $\indorder=0$.
Consistent with \defref{def:folWorld} we have a propositional logic world by
\begin{align*}
    \dataworld: [\catorder] \rightarrow [2] \, ,
\end{align*}
which we have in \charef{cha:logicalRepresentation} represented by the assignments $\catindexof{\atomenumerator} = \dataworldat{\selvariable=\atomenumerator}$ to the categorical variables $\catvariableof{\atomenumerator}$.

% Comparison with PL
%Compared with propositional formulas, the grounding tensor does not take as input a specific world, but is defined on a given world.
%We show in this chapter, how both tensor interpretations can be transformed, i.e. by extracting samples from a FOL world $\dataworld$ interpretated as an empirical distribution over PL worlds, and by generating FOL worlds by a set of samples generated from a PL distribution.

% One-hot maps
To represent logical formulas as sets of possible worlds, and distributions of worlds, we applied in \parref{par:one} one-hot encodings of possible worlds.
For the case of propositional logics, this is
\begin{align*}
    \onehotmapofat{\dataworld}{\shortcatvariables} = \bigotimes_{\catenumeratorin} \onehotmapofat{\dataworldat{\selvariable=\atomenumerator}}{\catvariableof{\catenumerator}} \, .
\end{align*}

\subsect{One-hot encoding of worlds}

Let us now generalize the one-hot encodings of propositional logic worlds to worlds of first-order logic.
To encode the boolean tensors $\dataworld$ describing first order logics as basis elements of a tensor space, we take the one-hot encoding
\begin{align*}
    \onehotmap :
    \bigtimes_{\atomenumeratorin}\bigtimes_{\indindexofin{0}}\cdots\bigtimes_{\indindexofin{\indorder-1}} [2]
    \rightarrow \bigotimes_{\catenumeratorin}\bigotimes_{\indindexofin{0}}\cdots\bigotimes_{\indindexofin{\indorder-1}} \rr^2
\end{align*}
defined by
\begin{align*}
    \onehotmapofat{\dataworld}{\catvariableof{[\catorder]\times[\inddim]^{\indorder}}}
    = \bigotimes_{\catenumeratorin}\bigotimes_{\indindexofin{0}}\cdots\bigotimes_{\indindexofin{\indorder-1}}
    \onehotmapofat{\dataworldat{\selvariable=\atomenumerator,\indexedshortindvariables}}{\catvariableof{\catenumerator,\shortindindices}} \, .
\end{align*}
This is a tensor of order $\catorder\cdot\inddim^{\indorder}$, in a tensor space of dimension $2^{\left(\catorder\cdot\inddim^{\indorder}\right)}$.
Storage of such tensors in naive formats would not be possible.
However, the basis CP format discussed in \charef{cha:sparseCalculus} still provides storage with demand linear in the order $\catorder\cdot\inddim^{\indorder}$.

% Domain
Another issue when comparing different first-order logic worlds arises in potentially different world domains.
As we have explored, the cardinality of the domain influences the order of the one-hot encoding tensors.
To avoid such issues we here enumerate worlds coinciding in their domains.
This restriction is called database semantics (see e.g. Section 8.2.8 in \cite{russell_artificial_2021}), where only those worlds are considered, which domains have a one-to-one map to the constant symbols appearing in a respective knowledge base. % Unique name assumption + Domain closure!
% Factored systems
When restricting to worlds coinciding in their domain, we still have a factored representation of the system, since we can enumerate the possible worlds by a cartesian product.
However, the number of categorical variables representing the world is $\atomorder\cdot \inddim^{\indorder}$ and tensor representations, even in sparse formats, are not feasible due to the large order required.
These techniques to restrict to comparable factored representations are often refered to propositionalization of a first-order logic knowledge base.

% Propositional 
%Propositional worlds have been enumerated by indices of $\atomorder$ Booleans, that is for a world $\dataworld: [\folpredicateorder] \rightarrow [2]$ we take the index
%	\[ \atomindices \quad \text{where} \quad \atomlegindexof{\atomenumerator} = \dataworld(\atomenumerator) \, .  \]

% FOL
%When we want to enumerate the first order logic worlds to a fixed set of objects $\worlddomain$, we flatten the tensor $\dataworld$ and get indices
%	\[ \{\atomlegindexof{\atomenumerator,\indindexlist} \, : \, \atomenumeratorin, \indindexlist \in[\inddim] \} \quad
%	\text{where} \quad \atomlegindexof{\atomenumerator,\indindexlist} = \dataworld(\atomenumerator,\indindexlist) \, .  \]





\subsect{Probability distributions}

Having established the formalism of one-hot encodings also in the case of first-order logic worlds, we can now proceed with the definition of distributions and formulas, analogously to the development in \parref{par:one}.
Probability distributions over worlds coinciding on their domain are then non-negative and normed tensors
\begin{align*}
    \probat{\catvariableof{[\catorder]\times[\inddim]^{\indorder}}} \in \bigotimes_{\atomenumeratorin,\shortindindices\in[\inddim]^{\indorder}} \rr^2 \, .
\end{align*}
where each coordinate of a world $\dataworld$ is captured by a boolean random variable $\catvariableof{\atomenumerator,\shortindindices}$, indicating whether the $\atomenumerator$-th predicate holds on the object tuple indexed by $\shortindindices$.

% High-dimensional - watch out for repetitions!
We notice, that by definition these probability distributions are distributions of $\atomorder\cdot\inddim^{\indorder}$ Booleans with $2^{\left(\atomorder\cdot\inddim^{\indorder}\right)}$ many states.
% One-hot encodings minimal
Unfortunately, it is not possible to design encoding spaces of smaller dimension, when our aim is to get any distribution over possible worlds by an element in the encoding space.
This is due to the fact, that one-hot encodings provide a basis in the tensor space, as will be shown in \charef{cha:coordinateCalculus}.
The reason for the large encoding space dimension is therefore rooted in the equal number of possible worlds and not in an overhead in the dimension of the one-hot encoding space.
We will later in this chapter investigate methods to handle such high-dimensional distributions in the formalism of exponential families.

\subsect{Semantics of formulas}

Following the development of \charef{cha:logicalRepresentation}, we can choose a semantic approach to the definition of formulas, under the assumption of database semantics.
Since the semantic of a logical formula is the set of its models, we again have a one-to-one correspondence between logical formulas and the boolean tensors in the one-hot encoding space
\begin{align*}
    \bigotimes_{\atomenumeratorin,\shortindindices\in[\inddim]^{\indorder}} \rr^2 \, .
\end{align*}
This correspondence between the semantics and boolean tensor is through a subset encoding (see \defref{def:subsetEncoding}) of the respective formulas.
However, due to the large state dimensions, we will in the following sections choose a syntactical approach to the construction of formulas, which will naturally provide efficient tensor network decompositions.

\subsect{Two levels of tensor representation}

In comparison with propositional logics, first-order logic bears two levels of natural tensor representations.
In the first level, which we call the structured level, each world (see \defref{def:folWorld}) has a natural structure by a tensor, since it encodes relations between objects chosen by assignments to term variables.
This is different to the worlds of a propositional logic theory, which are represented by a boolean vector instead of a tensor.
The second level arises as in propositional logics, by understanding each world as a uncertain state and studying distributions over states, which are understood themself as a tensor (see \defref{def:probabilityDistribution}).
We call this the factored level, since it arises in general in the discussion of factored representations.
As argued above, the assumption of database semantics is central to exploit the tensor structure of the substitution level.
Under this assumption, representation of an uncertain state, or a collection of possible states, is done in the tensor space
\begin{align*}
    \bigotimes_{\atomenumeratorin,\shortindindices\in[\inddim]^{\indorder}} \rr^2 \,
\end{align*}
where the enumeration of the $2$-dimensional axes contains the tensor structure of the substitution level.



\sect{Formulas in a fixed first-order logic world}

Following the argumentation above, we in this section restrict to the exploitation of tensors in the structured level, namely a fixed world represented as a tensor $\dataworldwith$, see \defref{def:folWorld}.
We are specifically interested in the tensor network decomposition of first order formulas, which contain in full generality variables and therefore also have a tensor.
The evaluation of a first-order formula on a specific world is therefore different to the case in propositional logics, where the evaluation was a boolean in $\ozset$ indicating whether the world is a model.
%\red{Here we investigate grounding tensors to formulas with variables, and calculate them in a fixed world.}
% Arbitrary formulas

\subsect{Grounding tensors}

Given a first-order logic world $\dataworldwith$, arbitrary formulas are interpreted in terms of the satisfactions of their groundings.
We define their semantic first, and then relate their syntactical decomposition to tensor networks, similar to our approach to propositional logics in \charef{cha:logicalRepresentation}.

\begin{definition}[Grounding of a first-order formula given a world]
    Given a specific world $\dataworld$, with an domain $\worlddomain$ enumerated by $[\inddim]$, the grounding of a formula $\folexformula$ with variables $\indvariableof{\folexformula}$  is the tensor
    \begin{align*}
        \groundingofat{\folexformula}{\indvariableof{\folexformula}} :
        \bigtimes_{\indenumerator\in[\indvariableof{\folexformula}]} [\inddim] \rightarrow \ozset \, .
    \end{align*}
    Each coordinate represents thereby the boolean, whether the substitution of the variables in the formula is satisfied given a world $\dataworld$, that is
    \begin{align*}
        \groundingofat{\folexformula}{\indexedindvariableof{\folexformula}} = 1
    \end{align*}
    if and only if the substitution of $\folexformula$ with the variables $\indvariableof{\folexformula}$ replaced by the objects $\indexinterpretationat{\indindexof{\indenumerator}}$ is satisfied on the world $\dataworld$.
\end{definition}

% Comment: Formulas as maps to
The grounding tensor formalism can be used to define formulas as a map
\begin{align*}
    \folexformula : \left(\bigotimes_{\atomenumeratorin,\shortindindices\in[\inddim]^{\indorder}}\rr^{2}\right)
    \rightarrow \left(\bigotimes_{\atomenumeratorin,\indindexof{\folexformula}\in[\inddim]^{\cardof{\indvariableof{\folexformula}}}}\rr^{2}\right)
\end{align*}
where each world $\dataworld$ is mapped to a grounding tensor
\begin{align*}
    \folexformula(\dataworld) = \groundingof{\folexformula} \, .
\end{align*}
This would involve the factored level of tensor interpretation, namely representation of all possible worlds.

%% Basis encoding
%When interpreting this map as a basis encoding, formulas are tensors in the tensor space
%\begin{align*}
% 	\left(\bigotimes_{\atomenumeratorin, \indindexlist\in[\inddim]} \rr^{2} \right) \otimes
%	\left(  \bigotimes_{\atomenumeratorin, \indindexof{0},\ldots,\indindexof{\individualorder_{\folexformula}}\in[\inddim]} \rr^{2} \right) \, .
%\end{align*}

\subsect{Atomic Formulas}

Atomic formulas in first-order logic are predicates, which are applied on terms.%, that is constants or variables.
We restrict in this chapter to function-free logic, therefore terms are either constants or variables.
%The predicates itself are the simplest cases of first-order formulas with term variables.
% Atomic
If all arguments of a predicate are assigned by free variables, the corresponding grounding tensor is stored in the slices to the first axis of $\dataworld$ and we have
\begin{align}
    \groundingof{\folpredicateof{\folpredicateenumerator}} =
    \contractionof{\dataworldat{\selvariable,\shortindvariablelist},\onehotmapofat{\folpredicateenumerator}{\selvariable}}{\shortindvariablelist} \, .
\end{align}
In contrast, when a constant object $\indexinterpretationof{\indindex}$ is assigned to an argument of a predicate, the grounding tensor reduced to a slice of the grounding with exclusively free variables.
We capture such slicings by contractions with one-hot encodings of the corresponding constant.

We formalize this approach by atom creating tensors, which contraction with the world tensor results in the grounding of the corresponding atomic formula.

\begin{definition}\label{def:atomCreatingTensor}
    Let there be an atomic formula $\folexformula$, which is constructed using the $\selindex$-th predicate and has constants assigned on the arguments $\arbsetof{C}\subset[\indorder]$ and free variables to the arguments $\arbsetof{V}=[\indorder]/\arbsetof{C}$.
    Let the constant map $C: \arbsetof{C}\subset[\indorder] \rightarrow [\inddim]$ map to the specific objects represented by the constant and $V: \arbsetof{V}\subset[\indorder] \rightarrow \nodes$ to free variables labeled by a set $\nodes$.
    Then the atom creating tensor to $\folexformula$ is
    \begin{align*}
        \atomcreatorofat{\folexformula}{\indvariableof{V(\arbsetof{V})}}
        = \onehotmapofat{\selindex}{\selvariable} \otimes
        \left( \bigotimes_{\indenumerator\in\arbsetof{C}} \onehotmapofat{C(\indenumerator)}{\indvariableof{\indenumerator}} \right) \otimes
        \left( \bigotimes_{\indenumerator\in\arbsetof{V}} \identityat{\indvariableof{V(\indenumerator)},\indvariableof{\indenumerator}} \right) \, .
    \end{align*}
\end{definition}

The ground of the atom is then the contraction of the atom creating tensor with the world tensor, that is
\begin{align*}
    \groundingofat{\folexformula}{\indvariableof{V(\arbsetof{V})}}
    = \contractionof{\dataworldwith, \atomcreatorofat{\folexformula}{\indvariableof{\nodes}}}{\indvariableof{V(\arbsetof{V})}} \, .
\end{align*}


% Predicates as objects
What is more abstract, we can understand the predicate itself as an object, then take the first-order world as a grounding tensor of a more abstract formula.
We will follow this thought in the ternary representation of Knowledge Graphs in \secref{subsec:knowledgeGraphTernaryRep}.

%\subsect{Substitution by slicing}

% Slicing interpretation
%Slicing the grounding tensor of a formula a first-order formula amounts to substitution of the respective variable by the constant at the enumeration index.

%\subsect{Syntactical Decomposition of quantifier-free formulas}

\subsect{Formula synthesis by connectives}\label{sec:folConnectiveRepresentation}

In order to have a sound semantic, the grounding of FOL formulas is determined by the syntax of the formula, i.e. a decomposition of the formula into connectives and quantifiers acting on atomic formulas.

% Formulas as maps from worlds to groundings
Quantifier-free formulas are connectives acting on atomic formulas.
We can describe them as in the case of propositional logics in the $\rencodingof{}$-formalism.
While the atomic formulas where delta tensors copying states, they are more involved here.



\begin{theorem}
    For any connective $\exconnective$ and formulas $\folexformula_1$ and $\folexformula_2$ we have
    \begin{align}
        &\groundingofat{(\folexformula_1\exconnective\folexformula_2)}{\indvariableof{\folexformula_1}\cup\indvariableof{\folexformula_2}} \\
        &\quad=
        \contractionof{
            \rencodingofat{\groundingof{\folexformula_1}}{\headvariableof{\folexformula_1},\indvariableof{\folexformula_1}},
            \rencodingofat{\groundingof{\folexformula_2}}{\headvariableof{\folexformula_2},\indvariableof{\folexformula_2}},
            \rencodingofat{\exconnective}{\headvariableof{\folexformula_1\exconnective\folexformula_2}, \headvariableof{\folexformula_1}, \headvariableof{\folexformula_2}},
            \tbasisat{\headvariableof{\folexformula_1\exconnective\folexformula_2}}
        }
        {\shortindvariablelist} \, .
    \end{align}
\end{theorem}
\begin{proof}
    This directly follows from \theref{the:compositionByContraction}.
%	By the semantic interpretation of the groundings, which has to be sound.
\end{proof}

% Shared variables
Here, variables can be shared by the connected formulas, therefore the variables in the combined formula are unions of the possible not disjoint variables of the connected formulas.

%% Propositional interpretation
%When we understand the head variables in the basis encoding of atoms as the categorical variables, and get a similar interpretation of the tensor network decomposition as in the propositional case.
%\subsect{Propositionalization}

When interpreting the head variables of relational encoded atomic formulas as the atoms of a propositional theory, we find a propositional formula $\exformula$ associated with any decomposable first order logic formula.

\begin{definition}
    \label{def:propositionalEquivalent}
    Given a formula $\folexformula$ in first order logic, we say that a propositional formula $\formulaat{\shortcatvariables}$ is the propositional equivalent to $\folexformula$ given atomic formulas $\extformulaof{\atomenumerator}$ in first order logic, when for any world $\dataworld$ we have
    \begin{align*}
        \groundingofat{\folexformula}{\indvariableof{\folexformula}}
        = \contractionof{
            \{\rencodingofat{\groundingof{\extformulaof{\atomenumerator}}}{\catvariableof{\atomenumerator},\indvariableof{\extformulaof{\atomenumerator}}} : \atomenumeratorin\}
            \cup \{\formulaat{\shortcatvariables}\}
        }{\indvariableof{\folexformula}} \, .
    \end{align*}
    We here denote the head variables of the basis encoding to $\rencodingof{\groundingof{\extformulaof{\atomenumerator}}}$ by $\catvariableof{\atomenumerator}$ to highlight their interpretation as propositional atoms.
\end{definition}

We depict the relation of a grounding tensor to a propositional formula as:
\begin{center}
    \input{./PartII/tikz_pics/fol_models/propositionalization.tex}
\end{center}


\subsect{Quantifiers}

Existential and universal quantifiers appear in generic first order logic and are besides substitutions further means to reduce the number of variables in a formula.
%They are not representable as linear transform of the respective quantifier-free formula.


% Definition of existential and universal quantifiction needed!
The semantics of existential quantification consists in a formula being true, if at least one state of the quantified variable is true, as we define next.

\begin{definition}
    Given a grounding tensor
    \begin{align*}
        \groundingofat{\folexformula}{\indvariableof{0},\ldots,\indvariableof{\indorder-1}} \,
    \end{align*}
    the existential and universal quantification with respect to the first variable are the tensors
    \begin{align*}
        \groundingofat{\left(\exists_{\indindexof{0}}\folexformula\right)}{\indvariableof{1},\ldots,\indvariableof{\indorder-1}} \quad \text{and} \quad
        \groundingofat{\left(\forall_{\indindexof{0}}\folexformula\right)}{\indvariableof{1},\ldots,\indvariableof{\indorder-1}} \,
    \end{align*}
    with coordinates as follows.
    For an assignment $\indindexof{1},\ldots,\indindex$ to the non-quantified variables we have
    \begin{align*}
        \groundingofat{\left(\exists_{\indindexof{0}}\folexformula\right)}{\indexedindvariableof{1},\ldots,\indexedindvariableof{\indorder-1}} = 1
    \end{align*}
    if and only if there is an assignment $\indindexofin{0}$ such that
    \begin{align*}
        \groundingofat{\folexformula}{\indexedindvariableof{0},\indexedindvariableof{1},\ldots,\indexedindvariableof{\indorder-1}} = 1 \, .
    \end{align*}
    Conversely, we have for the universal quantification that
    \begin{align*}
        \groundingofat{\left(\forall_{\indindexof{0}}\folexformula\right)}{\indexedindvariableof{1},\ldots,\indexedindvariableof{\indorder-1}} = 1
    \end{align*}
    if and only if for any assignment $\indindexofin{0}$ we have
    \begin{align*}
        \groundingofat{\folexformula}{\indexedindvariableof{0},\indexedindvariableof{1},\ldots,\indexedindvariableof{\indorder-1}} = 1 \, .
    \end{align*}
\end{definition}


Let us now show, that existential and universal quantification are coordinatewise transforms (see \defref{def:coordinatewiseTransform}) of contracted grounding tensors.
To this end, let us introduce the greater-$z$ indicator $\greaterthanfunction{z}$, where $z\in\rr$, as the function
\begin{align*}
    \greaterthanfunction : \rr \rightarrow \ozset
    \quad, \quad \greaterthanfunctionof{z}{x} =
    \begin{cases}
        1 & \quad  \text{if} \quad x > z\\
        0 & else
    \end{cases} \, .
\end{align*}

\begin{theorem}
    For any formula $\folexformula$ with variables $\shortindvariablelist$ we have
    \begin{align*}
        \groundingofat{\left(\exists{\indindexof{0}}\folexformula\right)}{\indvariableof{1},\ldots,\indvariableof{\indorder-1}} =
        \coordinatetrafowrtofat{\existquanttrafo}{\contractionof{\groundingof{\folexformula}}{\indvariableof{1},\ldots,\indvariableof{\indorder-1}}}{\indvariableof{1},\ldots,\indvariableof{\indorder-1}}
    \end{align*}
    and
    \begin{align*}
        \groundingofat{\left(\forall{{\indindexof{0}}} \folexformula\right)}{\indvariableof{1},\ldots,\indvariableof{\indorder-1}}=
        \coordinatetrafowrtofat{\universalquanttrafo}{\contractionof{\groundingof{\folexformula}}{\indvariableof{1},\ldots,\indvariableof{\indorder-1}}}{\indvariableof{1},\ldots,\indvariableof{\indorder-1}}
    \end{align*}
\end{theorem}
\begin{proof}
    We proof the claimed equalities to arbitrary slices of the remaining variables, which amount to arbitrary substitutions of the formulas.
    For any indices $\indindexofin{1},\ldots,\indindexofin{\indorder-1}$ we notice, that
    \begin{align*}
        \sbcontractionof{\groundingof{\folexformula}}{\indexedindvariableof{1},\ldots,\indexedindvariableof{\indorder-1}}
        &= \sum_{\indindexofin{0}} \groundingofat{\folexformula}{\indexedindvariableof{0},\ldots,\indexedindvariableof{\indorder-1}} \\
        &= \cardof{\indindexofin{0} \, : \, \groundingofat{\folexformula}{\indexedindvariableof{0},\ldots,\indexedindvariableof{\indorder-1}}=1} \, .
    \end{align*}
    We can thus understand the contracted grounding tensor as storing in its coordinates the count of the coordinate extensions to the zeroth variable, such that the grounding tensor is satisfied.
    This is analogous to our interpretation of contracted propositional formulas as world counts.
    From this it is obvious, that the existential quantification is satisfied, if the count is different from zero, which is captured by the coordinatewise transform with $\existquanttrafo$.
    We therefore arrive at
    \begin{align*}
        \groundingofat{\left(\exists_{\indindexof{0}}\folexformula\right)}{\indexedindvariableof{1},\ldots,\indexedindvariableof{\indorder-1}} =
        \coordinatetrafowrtofat{\existquanttrafo}{\contractionof{\groundingof{\folexformula}}{\indvariableof{1},\ldots,\indvariableof{\indorder-1}}}{\indexedindvariableof{1},\ldots,\indexedindvariableof{\indorder-1}} \, .
    \end{align*}
    The first claim follows, since the assignment to the non-quantified variables was arbitrary.
    The universal quantification is satisfied, when all extensions are satisfied, and the count is $\inddim$.
    Since $\inddim$ is the maximal count, this is captured by the coordinatewise transform with $\universalquanttrafo$ and we get
    \begin{align*}
        \groundingofat{\left(\forall{\indindexof{0}}\folexformula\right)}{\indexedindvariableof{1},\ldots,\indexedindvariableof{\indorder-1}} =
        \coordinatetrafowrtofat{\universalquanttrafo}{\contractionof{\groundingof{\folexformula}}{\indvariableof{1},\ldots,\indvariableof{\indorder-1}}}{\indexedindvariableof{1},\ldots,\indexedindvariableof{\indorder-1}} \, .
    \end{align*}
    With the same argument, the second claim is established.
\end{proof}

% Customized quantifiers
We can extend this discussion towards more generic counting quantifiers, of which the existential and the universal quantifier are extreme cases.
One can define quantifiers by demanding that at least $z\in\nn$ compatible groundings are satisfied, and show that they amount to coordinatewise transforms with $\greaterthanfunction{z}$.
What is more, quantifiers demanding that at most $z\in\nn$ are satisfied would be representable by transforms with an analogously defined function $\ones_{\leq z}$.
Such customized quantifiers appear for example in the $\mathrm{OWL\,2}$ standard of description logics (see \cite{rudolph_foundations_2011} and \secref{sec:kgRepresentation}).

% basis encodings
As will be discussed in \charef{cha:basisCalculus}, any coordinatewise transform can be performed by a contraction of a basis encoding of the tensor with a head vector prepared by the transform function (see \theref{the:tensorFunctionComposition}).
In the case here, a direct implementation would require a dimension of these head variables by $\inddim$, which can be infeasible when having large object sets.

% Prenex
To summarize, let us assume a formula is in its prenex normal form, that is a collection of quantifiers are acting on a qantifier free part.
We can represent its grounding tensor by
\begin{itemize}
    \item Instantiations of the tom groundings with the assigned variables, as contractions of the basis encoding of the world tensor with atom selecting tensors.
    \item Propositional formula acting on the head variables of the predicate instantiations, representing the connectives combining the formula.
    \item Quantifiers as a composition of contractions closing the quantified variable and coordinatewise transforms with the respective greater-than indicators.
\end{itemize}



\subsect{Storage in basis CP decomposition}\label{sec:basisCPgrounding}

In many situations, grounding cores are sparse and representations as single tensor cores comes with a drastic overhead.
We often encounter sparse grounding tensors, where the number of non-zero coordinates (to be investigated by basis CP ranks in \charef{cha:sparseCalculus}) satisfies
\begin{align*}
    \sparsityof{\groundingof{\folexformula}} << \inddim^{\cardof{\indvariableof{\folexformula}}} \, .
\end{align*}
In this case, since most coordinates vanish, the basis CP decomposition (see \secref{sec:basisCP}) enables a representation of the grounding with significantly lower storage demand, see \theref{the:sparseBasisCP}.
This is particularly useful for representing large relational databases, where each object has only a few relations with others, while the majority of possible relations remains unsatisfied.
We depict such CP decomposition of a formula grounding in \theref{fig:groundingCP}.

% Standard KB Encoding and Assumptions
Most logical syntaxes exploit $\ell_0$-sparsity, explicitly storing only known assertions.
The interpretation of unspecified assertions depends on the underlying assumptions.
Under the Closed World Assumption, for example, all unspecified assertions are assumed to be false.

\begin{figure}[h]
    \begin{center}
        \input{./PartII/tikz_pics/fol_models/grounding_decomposition.tex}
    \end{center}
    \caption{Basis CP Decomposition of the grounding of $\folexformula$, following the scheme of \theref{the:sparseBasisCP}.
    Instead of direct storage of the grounding tensor $\groundingof{\folexformula}$, the non-zero coordinates are enumerated by a variable $\datvariable$ and the corresponding coordinates stored in leg-matrices $\legcoreof{\folexformula,\indenumerator}$.}
    \label{fig:groundingCP}
\end{figure}

\subsect{Queries}

A database is understood as a specific fist order logic world, and are operations on such a single world.
Queries are described by a formula $\impformula$, which are asked against a specific world $\dataworld$ to retrieve the grounding $\groundingof{\impformula}$.
The variables of such formulas are called projection variables.
The answer $\groundingof{\impformula}$ of a query is most conveniently represented as a list of solution mappings from the projection variables to objects in the world, such that the query formula is satisfied.
Answering a query by solution mappings corresponds with finding the basis CP Decomposition (see \secref{sec:basisCP}) of $\groundingof{\impformula}$.
We can understand these solution mappings as stored in the leg-matrices $\legcoreof{\folexformula,\indenumerator}$ (see \figref{fig:gorundingCP}).

Let us give with the outer join an example of a popular operation to define queries, which efficient execution and storage can be improved based on considerations in the tensor network formalism.

\begin{definition}[Outer join]
    Let there be a world $\dataworld$ and formulas $\extformulaof{\selindex}$ depending on variables $\indvariableof{\nodesof{\selindex}}$, which have grounding tensors by
    \begin{align*}
        \groundingofat{\extformulaof{\selindex}}{\indvariableof{\node}} \, : \,  \bigtimes_{\node\in\nodesof{\selindex}}[\inddimof{\node}] \rightarrow \ozset \, .
    \end{align*}
    Then their (outer) $\joinsymbol$ is defined as the grounding of their conjunctions, as
    \begin{align*}
        \groundingofat{\joinsymbol\left(\extformulaof{0},\ldots,\extformulaof{\seldim-1}\right)}{\bigcup_{\selindexin}\indvariableof{\nodesof{\selindex}}}
        = \contractionof{\groundingofat{\extformulaof{\selindex}}{\indvariableof{\nodesof{\selindex}}}\,:\,\selindexin}{\bigcup_{l\in[p]}\indvariableof{\nodesof{\selindex}}} \, .
    \end{align*}
\end{definition}

%Visualization and efficiency
We can understand the $\joinsymbol$ of groundings by a factor graph, where each grounding tensor decorates the hyperedge to the node set $\nodesof{\selindex}$.
The projection variable assignment to each formula combined in a $\joinsymbol$ operation provide a basic tensor network format to store the output of the operation.
There are thus situations, in which the solution map storage corresponding with a CP Decomposition comes with unnecessary overheads compared with other formats.

% Coordinatewise transform
We can also understand the $\joinsymbol$ operation as a coordinatewise transform (see \defref{def:coordinatewiseTransform}) with the product as transform function.
To make this connection solid, one would need to extend each joined formula trivially to the variables appearing in other formulas.

% Evaluation similar constraint propagation
The efficiency of evaluating the contraction to a $\joinsymbol$ operation might be improved by understanding it as an Constraint Satisfaction Problem (see \charef{cha:logicalReasoning}).
When applying efficient Message Passing algorithms such as Knowledge Propagation (see \algoref{alg:knowledgePropagation}), the groundings can be sparsified by local constraint propagation operations before turning to more global and more demanding contraction operations.
Here the groundings $\groundingof{\extformulaof{\selindex}}$ would be used to initialize Knowledge Cores $\kcoreof{\edge}$ and sequentially sparsified during the algorithm.

%\begin{example} % WOULD NEED OVERWORK: DRAW!
%	For example take a query with many basic graph patterns with pairwise different projection variables.
%	The global CP Decomposition would come here with an exponential storage overhead compared with storage as a tensor product of CP Decompositions to each Basic graph pattern.
%\end{example}

%% CONFUSING?
%\begin{remark}[Distinguishing from probabilistic queries]
%	Let us distinguish the discussion here from those of queries in probabilistic reasoning, which have two main differences.
%	First, we ask queries against all possible pairs of variables, instead of asking the probability of satisfaction of a specific formula.
%	Second, since we made the epistemologic assumption of knowing possibilities and not probabilities in logics, a query is answered by a truth value.
%	We then only output in the shape of solution mappings the variable assignments where the query formula is true.
% 	Thus, the queries here can be thought of as a batch of probabilistic queries with Boolean answers.
%	% Alternative -> Later?
%	Probabilistic queries can furthermore be understood in terms of the data extraction process described in this section.
%	We can ask the query in probabilistic form (decomposed into atomic formulas) on the resulting empirical distribution.
%	This results in the ratio of the worlds satisfying the query among those worlds satisfying the extraction query $\impformula$.
%\end{remark}


\sect{Representation of Knowledge Graphs}\label{sec:kgRepresentation}

Let us now represent a specific fragment of first-order logic, namely Description Logics which Knowledge Bases are often refered to as Knowledge Graphs.
We here use the $\mathrm{OWL\,2}$ standard, which encodes the syntax of the description logic $\mathcal{SROIQ(D)}$ \cite{rudolph_foundations_2011}.

\subsect{Representation as unary and binary predicates}

% Reduction to binary
Predicates in knowledge graphs are binary (owl:ObjectProperties) and unary (owl:Class).
%Larger formulas are created by logical connections of these atomic formulas using disjunctions, conjunctions etc.
We enumerate the predicates by $[\folpredicateorder]$, the objects in the domain $\worlddomain$ by $[\inddim]$, and extend the unary predicates to binaries by tensor product with $\onehotmapofat{0}{\indvariableof{1}}$.
A Knowledge Graph on the set $\worlddomain$ of constants (owl:NamedIndividuals) is then the tensor
\begin{align*}
    \kgat{\selvariable,\indvariableof{0},\indvariableof{1}} : [\folpredicateorder] \times [\inddim] \times [\inddim] \rightarrow \ozset \, .
\end{align*}


\subsect{Representation as ternary predicate}\label{subsec:knowledgeGraphTernaryRep}

It has been particulary convenient to represent a Knowledge Graph instead as a grounding of a single ternary predicate $\rdf$.
To this end, the predicates $\folpredicateof{\catenumerator}$ and another object $\mathrdftype$ are added to a domain $\worlddomain$, by extending the $\inddim$ and the index interpretation function accordingly.


% RDF triple: Alternative viewpoint to collection of unary and binary predicates!
Following our notation we understand a Knowledge Graph as a grounding of the rdf triple relation $\rdf$ (being a formula of order 3) on a specific world $\kg$ with individuals $\worlddomain$

We then construct a grounding tensor $\kggroundingof{\rdf}$ out of the world $\kgat{\selvariable,\indvariableof{0},\indvariableof{1}}$ by
\begin{align*}
    \kggroundingof{\rdf} : [\inddim] \times [\inddim] \times [\inddim] \rightarrow \ozset
\end{align*}
where
\begin{align*}
    &\kggroundingofat{\rdf}{\indexedindvariableof{s}, \indexedindvariableof{p}, \indexedindvariableof{o}} \\
    &\quad =
    \begin{cases}
        \kgat{\selvariable=\indindexof{s},\indvariableof{0}=\indindexof{o},\indvariableof{1}=0}
        & \text{if} \quad \indindexof{p} = \invindexinterpretationat{\mathrdftype} \\
        \kgat{\selvariable=\indindexof{p},\indvariableof{0}=\indindexof{s},\indvariableof{1}=\indindexof{o}}
        & \text{if} \quad \indindexof{p} = \invindexinterpretationat{\folpredicateof{\catenumerator}} \quad \text{for some} \quad \catenumerator \\
        0  \quad & \text{else}
    \end{cases} \, .
\end{align*}


Slicing the tensor $\kggroundingof{\rdf}$ along the predicate axis retrieves specific information about roles and can be efficiently be performed on these formats.
The role $\mathrdftype$ has a specific meaning, since it contains from a DL perspective classifications (memberships of named concepts).
Further slicing the tensor along object axis therefore results in membership lists for specific classes (concepts).
One can thus regard $\mathrdftype$ as a placeholder for unitary formulas in a space of binary formulas.

% Triple Stores, sparsity
Exploiting the $\ell_0$-sparsity now leads to a so-called triple store, where $\kggroundingof{\rdf}$ is stored by a listing of those triples $\indindexof{\subsymbol},\indindexof{\predsymbol},\indindexof{\objsymbol}$ such that $\kggroundingofat{\rdf}{\indexedindvariableof{s}, \indexedindvariableof{p}, \indexedindvariableof{o}}=1$
A recent implementation of a triple store exploiting these intuitions is $\mathrm{TENTRIS}$, see \cite{pan_tentris_2020}.
In this work, such decompositions are generalized into more generic CP formats, see \charef{cha:sparseCalculus}.
% Approximation of KG Groundings
Approximations of grounding tensors by decompositions leads to embeddings of the individuals such as $\mathrm{Tucker}$, $\mathrm{ComplEx}$ and $\mathrm{RESCAL}$ (see \cite{nickel_review_2016}).

% Sparse representation
%Sparse representation of the grounding tensor to a knowledge graph is of central importance, as investigated in \cite{pan_tentris_2020}.
%We here do basis CP for sparse representation.


% basis encoding
For our purposes of evaluating logical formulas such as $\sparql$ queries we use the basis encoding of the groundings, which are depicted by
\begin{center}
    \begin{tikzpicture}[scale=0.3, thick] % , baseline = -3.5pt

    \draw[->] (0,1)--(0,3) node[midway,left] {\tiny $\headvariable$};
    \draw (-3,1) rectangle (3,-1);
    \node[anchor=center] (text) at (0,0) {\small $\rencodingof{\kggroundingof{\rdf}}$};
    \draw[<-] (-2,-1)--(-2,-3) node[midway,left] {\tiny $\sindvariable$};
    \draw[<-] (0,-1)--(0,-3) node[midway,left] {\tiny $\pindvariable$};
    \draw[<-] (2,-1)--(2,-3) node[midway,left] {\tiny $\oindvariable$};

\end{tikzpicture}
\end{center}




\subsect{$\sparql$ Queries}

The $\sparql$ query language is a syntax to express first-order logic formulas $\folexformula$ and intended to be evaluated given a Knowledge Graph.
We here consider tensor network representations of the $\mathrm{WHERE}{\cdot}$ block.
Given a specific knowledge graph $\kggroundingof{\rdf}$, the execution of query is the interpretation $\groundingof{\folexformula}$, typical represented in a sparse basis CP format where each slice represents a solution mapping.

\subsubsect{Triple Patterns}

\red{Central to $\sparql$ queries are triple patterns, which we understand as slicings of the tensor $\kggroundingof{\rdf}$.}
To each so-called triple pattern we build a corresponding atom creating tensor (see \defref{def:atomCreatingTensor}).
The triple pattern is then evaluated by contraction of the atom creating tensor with $\kggroundingof{\rdf}$.

Let us now provide examples of such pattern tensors.
A unary triple patterns contains a single projection variable, typically related with the subject variable $\sindvariable$ of $\kggroundingof{\rdf}$.
The corresponding pattern tensor is then
\begin{align*}
    \atomcreatorofat{\kgtriple{\provariable}{\mathrdftype}{\folpredicateof{\catenumerator}}}{
        \sindvariable, \pindvariable, \oindvariable, \provariable
    }
    = \identityat{\sindvariable,\provariable}
    \otimes \onehotmapofat{\invindexinterpretationat{\mathrdftype}}{\pindvariable}
    \otimes \onehotmapofat{\invindexinterpretationat{\folpredicateof{\atomenumerator}}}{\oindvariable} \, .
\end{align*}

Binary triple patterns come with two projection variables, typically related with the subject and the object variables $\sindvariable$ and $\oindvariable$.
The pattern tensor to the $\catenumerator$-th predicate is then
\begin{align*}
    \atomcreatorofat{\kgtriple{\provariableof{0}}{\folpredicateof{\catenumerator}}{\provariableof{1}}}{
        \sindvariable, \pindvariable, \oindvariable, \provariableof{0}, \provariableof{1}
    }
    = \identityat{\sindvariable,\provariableof{0}}
    \otimes \onehotmapofat{\invindexinterpretationat{\folpredicateof{\atomenumerator}}}{\pindvariable}
    \otimes \identityat{\oindvariable,\provariableof{1}} \, .
\end{align*}

Contraction with these pattern tensor evaluated the specific triple pattern, and outputs in a boolean tensor the indicator, which objects are members of a specific class (for unary patterns) or which pair of objects are related by a specific relation.
Again, the output of such contractions is a subset encodings of the set of solutions (see \defref{def:subsetEncoding}).

%%%%%%%%%%%% END OF FRIDAY 14.3.
%%%%%%%%%%%%

% Examples
Examples of triple patterns, drawn in \figref{fig:triplePatterns} are
\begin{itemize}
    \item Unary triple pattern with one variable, representing a formula with a single projection variable.
    For the example $\exunarytriple$ see Figure~\ref{fig:triplePatterns}a.
    \begin{align*}
        \atomcreatorofat{\kgtriple{\provariable}{\mathrdftype}{\folpredicateof{\catenumerator}}}{
            \sindvariable, \pindvariable, \oindvariable, \provariable
        }
        = \identityat{\sindvariable,\provariable}
        \otimes \onehotmapofat{\invindexinterpretationat{\mathrdftype}}{\pindvariable}
        \otimes \onehotmapofat{\invindexinterpretationat{\exaunaryrelation}}{\oindvariable}
    \end{align*}
    If and only if the output slice is $\tbasis$, then the corresponding object encoded by the input indices is of class $\exaunaryrelation$.
    \item Binary triple pattern with two variables, representing a formula with two projection variables.
    For the example  $\exbinarytriple$ see Figure~\ref{fig:triplePatterns}b.
    If and only if the output slice is $\tbasis$, then the corresponding object tuple encoded by the input indices has a relation $\exabinaryrelation$.
\end{itemize}

% Projection picture
The composition $\psi (\psi^T)$ of the matrification of the tensor $\psi$ is an orthogonal projection.
That means that applying $\psi (\psi^T)$ is the same map as applying once.


\begin{figure}[h]
    \begin{center}
        \begin{tikzpicture}[scale=0.3,thick] % , baseline = -3.5pt

    \begin{scope}
        [shift={(0,0)}]

        \node[anchor=center] (text) at (-12,2) {$a)$};

        \begin{scope}
            [shift={(-7,2)}]

            \draw (0,-3) rectangle (-6,-5);
            \draw[<-] (-3,-1)--(-3,-3) node[midway,right] {\tiny $\headvariable$};
            \node[anchor=center] (text) at (-3,-4) {$\rencodingof{\kggroundingof{\exunarytriple}}$};
            \draw[<-] (-3,-5)--(-3,-7) node[midway,left] {\tiny $\provariableof{0}$};

        \end{scope}

        \node[anchor=center] (text) at (-5.5,-2) {${=}$};

        \draw[->] (0,1)--(0,3) node[midway,left] {\tiny $\headvariable$};
        \draw (-4,1) rectangle (4,-1);
        \node[anchor=center] (text) at (0,0) {\small $\rencodingof{\kggroundingof{\rdf}}$};

        \draw (-2,-3) rectangle (-4,-5);
        \draw[<-] (-3,-1)--(-3,-3) node[midway,left] {\tiny $\sindvariable$};
        \node[anchor=center] (text) at (-3,-4) {$\delta$};
        \draw[<-] (-3,-5)--(-3,-7) node[midway,left] {\tiny $\provariableof{0}$};

        \draw (-1,-3) rectangle (1,-5);
        \draw[<-] (0,-1)--(0,-3) node[midway,left] {\tiny $\pindvariable$};
        \node[anchor=center] (text) at (0,-4) {$\onehotmapof{\invrdftypesymbol}$};

        \draw (2,-3) rectangle (4,-5);
        \draw[<-] (3,-1)--(3,-3) node[midway,left] {\tiny $\oindvariable$};
        \node[anchor=center] (text) at (3,-4) {$\onehotmapof{\exaunaryrelation}$};

    \end{scope}


    \begin{scope}
        [shift={(24,0)}]

        \node[anchor=center] (text) at (-13,2) {$b)$};

        \begin{scope}
            [shift={(-8,2)}]

            \draw (0.5,-3) rectangle (-6.5,-5);
            \draw[<-] (-3,-1)--(-3,-3) node[midway,right] {\tiny $\headvariable$};
            \node[anchor=center] (text) at (-3,-4) {$\rencodingof{\kggroundingof{\exbinarytriple}}$};

            \draw[<-] (-2,-5)--(-2,-7) node[midway,right] {\tiny $\provariableof{0}$};
            \draw[<-] (-4,-5)--(-4,-7) node[midway,left] {\tiny $\provariableof{1}$};

        \end{scope}

        \node[anchor=center] (text) at (-5.5,-2) {${=}$};

        \draw[->] (0,1)--(0,3) node[midway,left] {\tiny $\headvariable$};
        \draw (-4,1) rectangle (4,-1);
        \node[anchor=center] (text) at (0,0) {\small $\rencodingof{\kggroundingof{\rdf}}$};

        \draw (-2,-3) rectangle (-4,-5);
        \draw[<-] (-3,-1)--(-3,-3) node[midway,left] {\tiny $\sindvariable$};
        \node[anchor=center] (text) at (-3,-4) {$\delta$};
        \draw[<-] (-3,-5)--(-3,-7) node[midway,left] {\tiny $\provariableof{1}$};

        \draw (-1,-3) rectangle (1,-5);
        \draw[<-] (0,-1)--(0,-3) node[midway,left] {\tiny $\pindvariable$};
        \node[anchor=center] (text) at (0,-4) {$\onehotmapof{\exabinaryrelation}$};

        \draw (2,-3) rectangle (4,-5);
        \draw[<-] (3,-1)--(3,-3) node[midway,left] {\tiny $\oindvariable$};
        \node[anchor=center] (text) at (3,-4) {$\delta$};
        \draw[<-] (3,-5)--(3,-7) node[midway,right] {\tiny $\provariableof{0}$};

    \end{scope}

\end{tikzpicture}
    \end{center}
    \caption{Triple patterns of $\sparql$ as tensor networks.
    a) Example of unary triple pattern $\exunarytriple$ specifying whether an individual $\indexinterpretationof{\indindexof{1}}$ is a member of class $C$.
    %Here by $0$ we denote the element $\invindexinterpretationat{\mathrdftype}$
        b) Example of a binary triple pattern $\exbinarytriple$ specifying whether individuals $\indexinterpretationof{\indindexof{1}}$ and $\indexinterpretationof{\indindexof{2}}$ have a relation $R$.
        By $\onehotmapof{\invrdftypesymbol},\onehotmapof{\exaunaryrelation},\onehotmapof{\exabinaryrelation}$ we denote the one-hot encodings of the enumeration of the resources $rdf:type, C$ and $R$.
    }
    \label{fig:triplePatterns}
\end{figure}




\subsubsect{Basic Graph Patterns}

Generic $\sparql$ queries are compositions of triple patterns by logical connectives. % Except for some stuff like regex
These triple patterns possibly share projection variables.
Statements in $\sparql$ can be translated into Propositional Logics combining the triple patterns:
\begin{center}
    \begin{tabular}{|c|c|c|}
        \hline
        \textbf{$\sparql$}                & \textbf{Propositional Logics} & \textbf{Tensor Representation}                                                                   \\
        \hline
        $\{f_1, f_2\}$                    & $f_1\land f_2$                & $\rencodingofat{\land}{\headvariableof{f_1\land f_2},\headvariableof{f_1},\headvariableof{f_2}}$ \\
        \hline
        $\mathrm{UNION}\{f_1, f_2\} $     & $f_1\lor f_2$                 & $\rencodingofat{\lor}{\headvariableof{f_1\lor f_2},\headvariableof{f_1},\headvariableof{f_2}}$   \\
        \hline
        $\mathrm{FILTER}\,\,\mathrm{NOT}\,\,\mathrm{EXISTS}\{f\}$ & $\lnot f$                     & $\rencodingofat{\lnot}{\headvariableof{\lnot f},\headvariableof{f}}$                             \\
        \hline
    \end{tabular}
\end{center}

If a $\sparql$ query consists of these keywords, we find a straight forward corresponding network of triple patterns and encoded logical connectives, by applying our findings of \secref{sec:folConnectiveRepresentation}.
To this end, we prepare for each appearing triple pattern the corresponding pattern tensor, and a copy of $\kggroundingof{\rdf}$.
Here we also copy the term variables $\sindvariable,\pindvariable$ and $\oindvariable$, to ensure that each copy of $\kggroundingof{\rdf}$ shares variables with a single pattern tensor.
Projection variables are not copied, since we need to keep track of them shared among triple patterns.
Then we prepare the basis encoding of logical connectives according to the hierarchy specified in the $\sparql$ query.
Finally we add a $\tbasis$-vector to the final head variable representing the complete $\sparql$ query, to restrict the support to coordiantes corresponding with solution mappings.
We then contract the resulting tensor network, leaving all projection variables open.

If a projection variable is not appearing in the $\mathrm{SELECT}$ statement in front of the $\mathrm{WHERE}\{\cdot\}$-block, we simply exclude it from the open variables of the described contraction.
Note that in that case, the coordinates contain solution counts, i.e. how many assignments to the dropped variable have been a $1$ coordinate.
We can drop this additional information simply by performing a coordinatewise transform with the greater zero indicator $\existquanttrafo$.

% Effective calculus alternative
Here we represented a $\sparql$ query $\impformula$ consistent of multiple triple pattern by instantiating a head variables to each triple pattern.
Alternatively, the more direct hybrid calculus developed in \secref{sec:hybridCalculus} can be applied and the additional head variables avoided.
This is especially compelling, when the $\mathrm{WHERE}\{\cdot\}$-block does not contain further keywords, i.e. it is the conjunction of all triple patterns.
In that case, we avoid the instantiation of head variables (i.e. close the head variables separately by $\tbasis$-vectors) and represent the query by a contraction of all triple pattern tensors.

% Expressivity
We further notice, that any propositional formula acting on the head variables of the triple patterns can be expressed by a hierarchical combination of the key words in the above table.
To find the expression, one can transform a given formula into its conjunctive or disjunctive normal form and apply the statements according to the apperaing operations $\land,\lor$ and $\lnot$.


%% Further $\sparql$ features
%Further $\sparql$ features, which cannot be expressed by a tensor network are:
%\begin{itemize}
%    \item $\mathrm{FILTER}\{\cdot\}$ does not depend on triple patterns (e.g. numeric inequalities, regex functions on strings).
%    We can regard it as another basic formula, which does not result from a slicing of the $\rdf$ grounding tensor.
%    Besides that, we can understand it as formulas and include it in compositions.
%    \item $\mathrm{OPTIONAL}\{\cdot\}$ would result in $\ones$ leg vectors, when there is a missing variable assignment resulting.
%\end{itemize}



\sect{Probabilistic Relational Models}

% MLN in FOL and PL
So far we have studied Markov Logic Networks in Propositional Logics as probability distributions over worlds.
In FOL they define probability distributions over relations in worlds with a fixed set of objects.
More generally, such models are probabilistic relational models (see for an overview \cite{getoor_introduction_2019}.


We in this section treat random worlds in first-order logics with fixed domains $\worlddomain$.

%
We in this section show, when and how we can interpret likelihoods of Markov Logic Networks in First Order Logic in terms of samples of a Markov Logic Network in Propositional Logics.

\subsect{Hybrid First-Order Logic Networks}

% Templates
Following \cite{richardson_markov_2006} Markov Logic Networks in first-order logics are templates for distributions, which instantiate random worlds when choosing a set of objects $\worlddomain$.
Given a fixed set of constants, they then define a distribution over the worlds, which objects correspond with the constants. % this is database semantics!
This applies database semantics, where only those worlds are considered, where the unique name and domain closure assumptions given a set of constants are satisfied.
\red{Here we directly define them as exponential families distributing $\randworld$ for a given set of objects $\worlddomain$.}
\red{To avoid a similar discussion as in \charef{cha:networkRepresentation} we directly allow for boolean base measures and call the distributions Hybrid First-Order Logic Networks.}

\begin{definition}[Hybrid First-Order Logic Networks (HFLN)]
%    A Markov Logic Network is a template of probability distributions defined
    Let there be a set $\folformulaset$ of first-order logic formulas with maximal arity $\individualorder$, which is enumerated by a selection variable $\selvariable$ of dimension $\seldim$.
    Further, let there be a set of objects $\worlddomain$ and a boolean base measure $\basemeasureat{\shortindvariables}$.
    The family of Hybrid First-Order Logic Networks $\expfamilyof{\restfolformulaset,\basemeasure}$ defined by the tuple $(\folformulaset,\worlddomain,\basemeasure)$ is the exponential family of joint distributions to the variables $\randworld$ with the statistics
    \begin{align*}
        \sstat^{\restfolformulaset}_{\selindex}\left[\indexedrandworld\right]
        = \sbcontraction{\groundingof{\enumfolformula}}
    \end{align*}
    and the base measure $\basemeasure$.
%    The Markov Logic Network instantiated for a given set of objects $\worlddomain$ and a base measure $\basemeasure$ is the random world, which is a member of the exponential family with sufficient statistics
%    \begin{align*}
%        \sstatcoordinateofat{\selindex}{\indexedrandworld} = \sbcontraction{\groundingof{\enumfolformula}}
%        %\sstat_{\selindex}(\dataworld)  = \sbcontraction{\groundingof{\folexformula_\selindex}} % Formulas can have different
%    \end{align*}
%    and canonical parameters $\weight$.
\end{definition}

Each element of the family $\expfamilyof{\restfolformulaset,\basemeasure}$ is represented by a canonical parameter $\canparamat{\selvariable}$.

The mean parameter polytope is the convex hull of the vectors
\begin{align*}
    \sencodingofat{\folformulaset}{\indexedrandworld,\selvariable}
\end{align*}
to the worlds $\dataworld$ with $\basemeasureat{\indexedrandworld}=1$.
These vectors store are the counts of satisfied groundings to each formula, that is
\begin{align*}
    \sencodingofat{\folformulaset}{\indexedrandworld,\selvariable} = \cardof{
        \indindexof{\enumfolformula} \, : \, \groundingofat{\enumfolformula}{\indexedindvariableof{\enumfolformula}} = 1
    } \, .
\end{align*}
Each substitution of the variables in $\enumfolformula$ by objects in $\worlddomain$, which satisfies the formula in the world $\dataworld$, therefore provides a factor of $\expof{\canparamat{\indexedselvariable}}$ to the probability of $\dataworld$.

Let us notice, that different to the case of Hybrid Logic Networks treated in \charef{cha:networkRepresentation}, the statistic does not consist of boolean features, when formulas contain variables and we have multiple objects.
One could, however, replace each $\enumfolformula$ by the set of the possible groundings, i.e. substitutions of the formulas variables by any tuple of objects in $\worlddomain$.
The resulting distribution would be an Hybrid Logic Network with boolean statistic, which coincides with the HFLN when posing certain weight sharing conditions on $\canparam$.
The downside of this construction is the increase in the number of features from $\seldim$ to $\sum_{\selindexin} \cardof{\worlddomain}^{\cardof{\indvariableof{\enumfolformula}}}$.
This polynomial in the cardinality of the domain set increase poses significant computational challenges, see \cite{richardson_markov_2006}.
We will in the next sections explore an alternative way to apply the theory of \charef{cha:networkRepresentation} and \charef{cha:networkReasoning}, namely based on importance formulas.


%% Interpretation
%The statistics
%\begin{align*}
%    \sbcontraction{\groundingof{\folexformula_\selindex}} % Formulas can have different
%\end{align*}
%can be interpreted as the number of substitutions to a formula, such that the formula ist satisfied.
%Each substitution satisfying a formula adds a factor of $\expof{\canparam_\selindex}$ to the probability of the respective world before normalization.


%
%When constructing a world tensor to a theory with predicates of different order, we already argued that we extend the arity of predicates by tensor products with $\onehotmapof{0}$.
%To define random world tensors, we then restrict the corresponding base measure to be supported only on those worlds where the extended predicates hold only at the individual $\exindividualof{0}$ at the extended axis.


% Comparison with PL MLN
%We choose extraction formulas $\extformulaof{\atomenumerator}$ such that any formula in the FOL MLN has a propositional equivalent (see \defref{def:propositionalEquivalent}).
%The statistic map is then a formula selecting tensor as in the propositional logic case contracted with the groundings of $\extformulaof{\atomenumerator}$.






\subsect{Base measures by importance formulas}

%\red{Analogous to a guard formula in \cite[Definition 6.11]{koller_probabilistic_2009}!}

The boolean base measure $\basemeasure$ of a Hybrid First-Order Logic Network is the subset encoding of the possible worlds which have a non-vanishing probability with respect to any member of the family.
We now construct specific base measures based on a fixed grounding tensor of an importance formula.
This will reduce the number of object tuples influencing the probability distribution in order to arrive at an interpretation of FOL MLNs as likelihoods to datasets of propositional MLNs.

To this end, we mark pairs of term indices relevant to the distributions by an auxiliary index $\datindexin$.
Given a set $\{\indindexof{[\indorder]}^{\datindex} \, : \, \datindexin \}$ of indices to the important tuples we build a set encoding (see \defref{def:subsetEncoding})
\begin{align*}
    \fixedimpformula = \sum_{\datindexin} \left(
    \bigotimes_{\indenumeratorin} \onehotmapofat{\indindexof{\indenumerator}^{\datindex}}{\indvariableof{\indenumerator}}
    \right) \, .
\end{align*}

% Interpretation as grounding
We interpret the tensor $\fixedimpformula$ as the grounding of a formula, which we call the importance formula.

% Restricting to worlds with identical grounding
To have a constant importance formula we define a syntactic representation and restrict the support of the HFLN to those world coinciding with groundings of the importance formula coinciding with $\fixedimpformula$ by designing a base measure
\begin{align*}
    \fixedimpbm
    = \begin{cases}
          1 & \text{if} \quad \groundingofat{\impformula}{\indvariableof{\impformula}} = \fixedimpformula \\
          0 & \text{else}
    \end{cases} \, .
\end{align*}

% Conditioning on exquery
The base measure restricts the HFLN to be those worlds, where $\groundingof{\impformula}$ is coincides with the fixed tensor $\fixedimpformula$.
Intuitively, $\groundingof{\impformula}$ represents certain evidence about a first-order logic world, whereas other formulas are uncertain.


\begin{assumption}
    \label{ass:importanceBasemeasure}
    Given a base measure $\fixedimpbm$, we assume that there is an importance formula $\impformulaat{\shortindvariables}$ such that
    \begin{align*}
        \fixedimpbm
        = \begin{cases}
              1 & \text{if} \quad \groundingofat{\impformula}{\indvariableof{\impformula}} = \fixedimpformula \\
              0 & \text{else}
        \end{cases} \, .
    \end{align*}
\end{assumption}


\subsect{Decomposition of the log likelihood}


% Extraction query
To reduce the likelihood of a world to we make the assumption that all formulas in a HFLN are of the form
\begin{align}
    \label{eq:folImplicationForm}
%    \folexformula_{\selindex}(\individuals) =
    \enumfolformulaat{\indvariableof{\enumfolformula}}
    = \left( \impformulaat{\shortindvariables} \Rightarrow \headfolformulaofat{\selindex}{\indvariableof{\enumfolformula}} \right)
\end{align}
that is a rule with the importance formula being the premise.
In particular, we assume, that they depend on all term variables $\shortindvariables$.
If this is not the case, we extend the formula trivially on the missing term variables.
When this assumption holds, we can think of the importance formula as a conditions on individuals to satisfy a statistical relation given by $\headfolexformula$.

Towards connecting with propositional logics, we further make the assumption, that we can decompose the formula $\headfolformulaof{\selindex}$ in what we will call extraction formulas.

\begin{assumption}
    \label{ass:propositionalHeads}
    We assume that there exist formulas $\{\extformulaofat{\catenumerator}{\shortindvariables} \, : \, \catenumeratorin\}$, which we refer to as atom extraction formulas, and an importance formula $\impformulaat{\shortindvariables}$ such that the following holds.
    To each first-order logic formula $\enumfolformula$ there is another first-order logic formula $\headfolformulaofat{\selindex}{\indvariableof{\enumfolformula}}$ and a propositional formula $\enumformulaat{\shortcatvariables}$ such that
    \begin{align*}
        \enumfolformulaat{\indvariableof{\enumfolformula}}
        = \left( \impformulaat{\shortindvariables} \Rightarrow \headfolformulaofat{\selindex}{\indvariableof{\enumfolformula}} \right)
    \end{align*}
    and
    \begin{align*}
        \headfolformulaofat{\selindex}{\indvariableof{\enumfolformula}} =
        \contractionof{
            \{\enumformulaat{\shortcatvariables}\} \cup \{\rencodingofat{\extformulaof{\catenumerator}}{\catvariableof{\catenumerator},\shortindvariables} \, : \, \catenumeratorin\}
        }{\indvariableof{\enumformula}} \, .
    \end{align*}
\end{assumption}

We depict the assumption, that any formula is of the form \eqref{eq:folImplicationForm} in the diagram
\begin{center}
    \begin{tikzpicture}[scale=0.35, yscale=1, thick] % , baseline = -3.5pt




\draw (1,-1) rectangle (7,-3);
\node[anchor=center] (text) at (4,-2) {$\groundingof{\left(\impformula\Rightarrow\folexformula\right)}$};

\draw[] (2,-3) -- (2,-5) node[midway,left] {\tiny $\individualvariableof{0}$};
\node[anchor=center] (text) at (4,-4) {$\cdots$};
\draw[] (6,-3) -- (6,-5) node[midway,right] {\tiny $\individualvariableof{\individualorder-1}$};


\node[anchor=center] (text) at (10,-2) {${=}$};



%\draw (1,-1) rectangle (7,-3);
%\node[anchor=center] (text) at (4,-2) {$\rencodingof{\impformula}$};

\begin{scope}[shift={(12,-2)}]

\draw (1,3) rectangle (12,5);
\node[anchor=center] (text) at (6.5,4) {$\exformula$};

\draw[->-] (2.5,1) -- (2.5,3) node[midway,right] {\tiny $\atomicformulaof{0}$};
\draw (1,-1) rectangle (4,1);
\node[anchor=center] (text) at (2.5,0) {$\rencodingof{\groundingof{\extformulaof{0}}}$};
\node[anchor=center] (text) at (2.5,-2) {$\cdots$};

\node[anchor=center] (text) at (6.5,0) {$\cdots$};

\draw[->-] (10.5,1) -- (10.5,3) node[midway,right] {\tiny $\atomicformulaof{\atomorder-1}$};
\draw (8.75,-1) rectangle (12.25,1);
\node[anchor=center] (text) at (10.5,0) {$\rencodingof{\groundingof{\extformulaof{\atomorder\shortminus1}}}$};
\node[anchor=center] (text) at (10.5,-2) {$\cdots$};

\draw[<-] (13,-3) -- (3.5,-3) ;
\draw[<-] (13,-5) -- (1.5,-5) ;

\draw[fill] (11.5,-3) circle (0.15cm);
\draw[->-] (11.5,-3) -- (11.5,-1);

\draw[fill] (9.5,-5) circle (0.15cm);
\draw[->-] (9.5,-5) -- (9.5,-1);

\draw[fill] (3.5,-3) circle (0.15cm);
\draw[->-] (3.5,-3) -- (3.5,-1);

\draw[fill] (1.5,-5) circle (0.15cm);
\draw[->-] (1.5,-5) -- (1.5,-1);


\draw[fill] (7.5,-3) circle (0.15cm);
\draw[<-] (7.5,-3) -- (7.5,-7) node[right] {\tiny $\individualvariableof{\individualorder-1}$} ;

\node[anchor=center] (text) at (6.5,-6) {$\cdots$};

\draw[fill] (5.5,-5) circle (0.15cm);
\draw[<-] (5.5,-5) -- (5.5,-7) node[left] {\tiny $\individualvariableof{0}$} ;


\draw (13,-2) rectangle (15,-6);
\node[anchor=center] (text) at (14,-4) {$\rencodingof{\impformula}$};
\node[anchor=center] (text) at (12,-3.75) {$\vdots$};
\draw[->-] (15,-4) -- (16,-4);
\draw[] (17,-4) -- (16,-4);
\draw (17,-3) rectangle (19,-5);
\node[anchor=center] (text) at (18,-4) {$\tbasis$};

\end{scope}




\node[anchor=center] (text) at (30,-2) {${+}$};





\begin{scope}[shift={(22,1)}]

\draw (13,0) rectangle (15,2);
\node[anchor=center] (text) at (14,1) {$\fbasis$};

\draw[] (14,-1) --(14,0);
\draw[->-] (14,-2) --(14,-1);

\draw (12,-2) rectangle (16,-4);
\node[anchor=center] (text) at (14,-3) {$\rencodingof{\impformula}$};

\draw[<-] (12.5,-4) -- (12.5,-6) node[left] {\tiny $\individualvariableof{0}$} ;
\node[anchor=center] (text) at (14,-5) {$\cdots$};
\draw[<-] (15.5,-4) -- (15.5,-6) node[right] {\tiny $\individualvariableof{\individualorder\shortminus1}$} ;


		
\end{scope}

\end{tikzpicture}
\end{center}
where the second summand depends only on the query $\impformula$ and therefore does not appear in the likelihood.


%\begin{example}[Trivial importance formula]
%	When the importance formula is always satisfied, any tuple of objects contributes to the likelihood. 
%	This original approach to Markov Logic Networks \cite{richardson_markov_2006} however leads to many datapoints which are also dependent on each other.
%\end{example}


% Define probability
Let us now show, how to decompose the probability of a first-order logic world to a HFLN under the above assumptions.
Given a HFLN $\probof{\folmlnparameters}$, the probability of a world $\dataworld$ with $\groundingof{\impformula}=\fixedimpformula$ is % and $\groundingof{\folpredicateof{\folpredicateenumerator}} \prec \fixedimpformula$ as
\begin{align*}
    \probofat{\folmlnparameters}{\indexedrandworld}
    = \frac{1}{\partitionfunctionof{\folmlnparameters}}
    %\expof{\sum_{\folexformulain}\weightof{\folexformula}\contraction{\groundingof{(\impformula\Rightarrow\headfolexformula)}}}
    \expof{\sum_{\selindexin}\canparamat{\indexedselvariable}\contraction{\groundingof{(\impformula\Rightarrow\headfolexformula)}}}
\end{align*}
where the partition function is
\begin{align*}
    \partitionfunctionof{\folmlnparameters} =
    \sum_{\supportedworlds}
    \expof{\sum_{\selindexin}\canparamat{\indexedselvariable}\contraction{\groundingof{(\impformula\Rightarrow\headfolexformula)}}} \, .
    %\expof{\sum_{\folexformulain}  \weightof{\folexformula}  \sum_{\indindexlist\in[\inddim]} \groundingofat{(\impformula\Rightarrow\headfolexformula)}{\indexedshortindvariables} } }
    %\prod_{\individuals\in\worlddomain} \left(\prod_{\folexformulain} \expof{\weightof{\folexformula}(\impformula\Rightarrow\folexformula)(\individuals)} \right)\, .
\end{align*}


Let us now decompose the statistics into constant and varying terms.
We have
\begin{align*}
    \contraction{\groundingof{(\impformula\Rightarrow\headfolexformula)}} =
    \contraction{\groundingof{\impformula\land\headfolexformula}} + \contraction{\groundingof{\lnot\impformula}} \, ,
\end{align*}
where the the second term is constant among the supported worlds and the first can be enumerated by the satisfied substitutions of $\impformula$, that is
\begin{align*}
    \contraction{\groundingof{\impformula\land\headfolexformula}}
    = \sum_{\datindexin}\groundingofat{\headfolexformula}{\indvariableof{[\indorder]} = \indindexof{[\indorder]}^{\datindex}} \, .
\end{align*}


Using these insights we decompose a normalized log likelihood as
\begin{align}
    \label{eq:dataworldLogProb}
    \frac{1}{\datanum} \lnof{\probofat{\folmlnparameters}{\indexedrandworld}}
    = & \frac{1}{\datanum} \sum_{\datindexin} \sum_{\selindexin} \canparamat{\indexedselvariable}
    \groundingofat{\headfolexformula}{\shortindindices = \indindexof{[\indorder]}^{\datindex}} \\
    & - \frac{1}{\datanum} \lnof{
        \frac{\partitionfunctionof{\folmlnparameters}}{
            \expof{\contraction{\weight} \cdot \contraction{\groundingof{\lnot\impformula}}}
        }
    }
%	\sum_{\individuals\in\worlddomain \, : \, \impformula(\individuals)=1} \left(\sum_{\exformula\in\formulaset} \weightof{\exformula} \exformula(\individuals)\right) -  \frac{1}{\datanum}\lnof{\secpartitionfunctionof{\folformulaset,\weight,\worlddomain,\fixedimpformula}} \, . 
\end{align}

% Data identification
We notice a similarity with the likelihood in the case of MLNs in propositional logic.
When we interpret each tuple $\shortindindices\in(\worlddomain)^{\indorder}$ satisfying $\impformulaat{\indexedshortindvariables}=1$ as a datapoint, and choose the formulas
\begin{align*}
    \formulaset = \{\headfolformulaof{\selindex} \, : \, \selindexin \}
\end{align*}
from the propositional equivalents to the formulas $\folformulaset$, the fist term in \eqref{eq:dataworldLogProb} coincides with the first term of the likelihood
\begin{align*}
    \centropyof{\empdistribution}{\expdistof{(\formulaset,\canparam,\ones)}}
    = \dataaverage \sum_{\selindexin} \canparamat{\indexedselvariable} \enumformulaat{\datapoint} - \lnof{\partitionfunctionof{\mlnparameters}}
\end{align*}

However, the partition function couples multiple samples, with possible couplings, and prevents a straight forward interpretation as an empirical dataset.
We in the next section present assumptions on the tuples satisfying $\impformula$, which lead to a factorization of the partition function.

% Partition function simplification
%\subsect{Interpretation as Likelihood of Propositional Dataset}
\subsect{Decomposition of the Partition function}


We now make additional assumptions to decompose the partition function of an HFLN as a product of HLN partition functions.

\begin{assumption}
    \label{ass:independentTuples}
    Let $\fixedimpbm$ be a base measure of worlds such that the vectors
    \begin{align}
        \label{eq:data}
        \left(  \groundingofatwrt{\extformulaof{0}}{\indvariableof{0}=\indindexof{0}^\datindex,\ldots,\indvariableof{\indorder-1}=\indindexof{\indorder-1}^\datindex}{\tilde{\dataworld}}, \ldots,
        \groundingofatwrt{\extformulaof{\atomorder-1}}{\indvariableof{0}=\indindexof{0}^\datindex,\ldots,\indvariableof{\indorder-1}=\indindexof{\indorder-1}^\datindex}{\tilde{\dataworld}}
        \right)
    \end{align}
    for $\datindexin$ are independent and identical distributed by the normation of a boolean base measure $\atombasemeasure$, when drawing $\randworld$ by $\fixedimpbm$.
\end{assumption}

When these assumption holds, we now show that the probability of a first-order logic world coincides with the likelihood of a propositional logic dataset.

\begin{theorem}
    \label{the:FOLworldToPLdataset}
    Let there be a set of formulas $\folformulaset$ and a base measure $\fixedimpbm$ such that \assref{ass:importanceBasemeasure}, \assref{ass:propositionalHeads} and \assref{ass:independentTuples} hold.
    We then have for the likelihood of any by $\fixedimpbm$ supported world $\dataworld$ that
    \begin{align*}
        \frac{1}{\datanum} \lnof{\probofat{\folmlnparameters}{\indexedrandworld}}
        = \centropyof{\empdistribution}{\expdistof{\mlnparameters}}
    \end{align*}
    where $\formulaset$ is the set of propositional equivalents to $\folformulaset$ (see \assref{ass:propositionalHeads}) and $\datamap$ the data map with evaluation at $\datindexin$ by the enumerated non-vanishing coordinates of $\fixedimpformulawith$
    \begin{align*}
        \datapoint
        = \big( \groundingofat{\extformulaof{0}}{\indvariableof{0}=\indindexof{0}^\datindex,\ldots,\indvariableof{\indorder-1}=\indindexof{\indorder-1}^\datindex}, \ldots ,
        \groundingofat{\extformulaof{\atomorder-1}}{\indvariableof{0}=\indindexof{0}^\datindex,\ldots,\indvariableof{\indorder-1}=\indindexof{\indorder-1}^\datindex} \big) \, .
    \end{align*}
\end{theorem}

To show the theorem, we show first in the following lemma the factorization of the partition function of the HFLN.

\begin{lemma}
    \label{lem:FOLpartitionfunctionfactorization}
    Given the assumptions of \theref{the:FOLworldToPLdataset}, we have
    \begin{align*}
        \frac{\partitionfunctionof{\folmlnparameters}}{\expof{\contraction{\canparam} \cdot \contraction{\groundingof{\lnot\impformula}}}}
        = \left(\partitionfunctionof{\mlnparameters,\atombasemeasure}\right)^\datanum \, .
    \end{align*}
\end{lemma}
\begin{proof}
    We have
    \begin{align*}
        \partitionfunctionof{\folmlnparameters}
        &= \expectationofwrt{
            \expof{\sum_{\folexformulain}\weightof{\folexformula}\contraction{\groundingof{(\impformula\Rightarrow\headfolexformula)}}}
        }{\dataworld\sim\fixedimpbm} \\
        &= \expof{\contraction{\weight} \cdot \contraction{\groundingof{\lnot\impformula}}} \cdot
        \expectationofwrt{
            \expof{\sum_{\folexformulain}\weightof{\folexformula}  \sum_{\datindexin} \groundingofat{\headfolexformula}{\datshortindvariables} }
        }{\dataworld\sim\fixedimpbm} \\
        &= \expof{\contraction{\weight} \cdot \contraction{\groundingof{\lnot\impformula}}} \cdot
        \expectationofwrt{
            \prod_{\datindexin} \expof{ \sum_{\folexformulain} \weightof{\folexformula} \cdot \groundingofat{\headfolexformula}{\datshortindvariables} }
        }{\dataworld\sim\fixedimpbm}
    \end{align*}
    Since the substitutions of the atom formulas at the respective object tuples are independent, also the variables
    \begin{align*}
        \expof{\weightof{\folexformula}  \cdot \groundingofat{\headfolexformula}{\datshortindvariables}  }
    \end{align*}
    for $\datindexin$ are independent.
    We therefore get
    \begin{align}
        \label{eq:independentSamplesFOL}
        \partitionfunctionof{\folmlnparameters}
        &= \expof{\contraction{\weight} \cdot \contraction{\groundingof{\lnot\impformula}}} \cdot
        \prod_{\datindexin}
        \expectationofwrt{
            \expof{ \sum_{\folexformulain} \weightof{\folexformula} \cdot \groundingofat{\headfolexformula}{\datshortindvariables} }
        }{\dataworld\sim\fixedimpbm}
    \end{align}
    Each $\groundingofat{\headfolexformula}{\datshortindvariables}$ depends by \assref{ass:propositionalHeads} only on the random tuple $\{\extformulaof{\atomenumerator}[\datshortindvariables] \, : \, \catenumeratorin\}$.
    We build the expectation over all possible values $\shortcatindices$ of this tuple at any $\datindexin$ and get
    \begin{align*}
        & \expectationofwrt{
            \expof{\sum_{\selindexin} \canparamat{\indexedselvariable} \cdot \groundingofat{\headfolformulaof{\selindex}}{\datshortindvariables}}
        }{\dataworld\sim\fixedimpbm} \\
        & \quad = \sum_{\shortcatindices\in\atomstates}
        \probofwrt{\forall{\catenumeratorin} \, : \, \extformulaof{\atomenumerator}[\datshortindvariables] =\catindexof{\atomenumerator}}{\dataworld\sim\fixedimpbm}
        \cdot \expof{\sum_{\selindexin}\canparamat{\indexedselvariable}\cdot\enumformulaat{\indexedshortcatvariables}} \\
        & = \sum_{\shortcatindices\in\atomstates} \atombasemeasure[\indexedshortcatvariables] \cdot
        \expof{\sum_{\selindexin} \canparamat{\indexedselvariable} \cdot \enumformulaat{\indexedshortcatvariables}}
        \\
        & = \partitionfunctionof{\mlnparameters,\atombasemeasure} \, .
    \end{align*}
    We arrive at the claim, when combining this equation with \eqref{eq:independentSamplesFOL}.
\end{proof}

With this lemma, we are now show \theref{the:FOLworldToPLdataset}.

\begin{proof}[Proof of \theref{the:FOLworldToPLdataset}]
    %Use Equation \ref{eq:dataworldLogProb} and the \lemref{lem:FOLpartitionfunctionfactorization}.
    We have for the logarithm of the probability of a world $\dataworld$ given the distribution $\probof{\folmlnparameters}$, that
    \begin{align*}
        \lnof{\probofat{\folmlnparameters}{\indexedrandworld}}
        =  \sum_{\selindexin} \canparamat{\indexedselvariable} \contraction{\groundingof{\enumfolformula}} - \lnof{\partitionfunctionof{\folmlnparameters}}
    \end{align*}
    The first term obeys with \assref{ass:propositionalHeads}
    \begin{align*}
        \sum_{\selindexin} \canparamat{\indexedselvariable} \contraction{\groundingof{\enumfolformula}}
        &=  \contraction{\canparam} \cdot \contraction{\groundingof{\lnot\impformula}}
        + \sum_{\selindexin}\sum_{\datindexin} \canparamat{\indexedselvariable} \cdot \headfolformulaofat{\selindex}{\datshortindvariables} \\
        &= \contraction{\canparam} \cdot \contraction{\groundingof{\lnot\impformula}}
        + \sum_{\selindexin}\sum_{\datindexin} \canparamat{\indexedselvariable} \cdot \enumformulaat{\datshortcatvariables} \, .
    \end{align*}
    With \lemref{lem:FOLpartitionfunctionfactorization} we have under the given assumptions for the second term
    \begin{align*}
        \lnof{\partitionfunctionof{\folmlnparameters}} = \datanum \cdot \lnof{\partitionfunctionof{\mlnparameters,\atombasemeasure}}  + \contraction{\canparam} \cdot \contraction{\groundingof{\lnot\impformula}} \, .
    \end{align*}
    Combining both, we have
    \begin{align*}
        \frac{1}{\datanum} \lnof{\probofat{\folmlnparameters}{\indexedrandworld}}
        =  \dataaverage\sum_{\selindexin} \canparamat{\indexedselvariable} \cdot \enumformulaat{\datshortcatvariables}  - \lnof{\partitionfunctionof{\mlnparameters,\atombasemeasure}}
    \end{align*}
    which coincides with $\centropyof{\empdistribution}{\expdistof{\mlnparameters}}$.
%    Given \assref{ass:propositionalHeads} we can
%    Note that we need to correct the likelihood by the averalge log basemeasure on the data, since that term is appearing in the likelihood of a MLN.
\end{proof}

%%%%%%%%%%%%%%%%%%%%%%%
%%%% END of Monday 17.3.
%%%%%%%%%%%%%%%%%%%%%%

% Independent data investigation
Let us now investigate, in which cases the \assref{ass:independentTuples} of independent data can be matched.

\begin{lemma}
    Let $\impformula$ and $\extformulas$ be quantor and constant free and let the index tuples of the support of $\fixedimpformula$ be pairwise disjoint.
    Then the vectors \eqref{eq:data} are pairwise independent.
\end{lemma}
\begin{proof}
    Then we can reduce each sample as dependent only on an independent random world with domain by the respective objects.
    Quantor and constant-free is needed that this reductions is possible.
\end{proof}


There are situations, where \assref{ass:independentTuples} is violated.
For example, when two object tuples are not disjoint, then some formulas might always coincide on both datapoints, which would violate independence.

In further situations the atom base $\atombasemeasure$ are not the uniform $\ones$:
\begin{itemize}
    \item extraction formula being a) conjunctions of predicates: Probability that they are satisfied decreases
    b) disjunctions of predicates: Probability that they are satisfied increases
    \item extraction formula coinciding with importance formula: Always satisfied, in this case still boolean
    \item extraction formulas contradicting each other, more general not independent from each other
\end{itemize}

Let us notice, that non-boolean base measures could be treated in a same manner, but several developments in this work, such as cross-entropy decompositions in \charef{cha:probReasoning} would receive further terms.



\begin{remark}[Approximation by Independent Samples]
    As observed above, we do not have independent samples in general.
    As a consequence, we cannot apply \lemref{lem:FOLpartitionfunctionfactorization} to decompose the partition function term of the log-probability into factors to each solution map of $\impformula$.
    In this case, it might be still benefitial to use the reduction to the likelihood of a HLN, but needs to understand it as a approximation to the true world probability.

    %
    If the expectations of each sample with respect to the marginalized distributions coincide, the average of empirical distribution also coincides with these (by linearity).
    When the creation of samples has sufficient mixing properties, the empirical distribution converges to this expectation in the asymptotic case of large numbers of samples.

\end{remark}



\sect{Sample extraction from first-order logic worlds}

The decomposition of the likelihood suggests the following approach to generate samples from groundings:
%We propose the following approach to generate datacores from groundings:
\begin{itemize}
    \item Define a query formula $\impformula$, which we decompose in the basis CP decomposition and interpret each slice as the one-hot encoding of the datapoint.
    \item Define for $\atomenumeratorin$ queries $\extformulaof{\atomenumerator}$ generating the the atoms $\atomicformulaof{\atomenumerator}$:
    Predicates along with assignment of variables / constants to its positions.
    \item Contract the groundings of each formula $\extformulaof{\atomenumerator}$ with the grounding of $\impformula$ to build a data core
\end{itemize}


\subsect{Representation by Tensor Networks}

We model the extraction process as a relation between a tuple of individuals and the extracted world in the factored system of atoms $\catvariableof{\atomenumerator}$.

\begin{definition}
    \label{def:extractionRelation}
    Given a first-order logic world $\dataworld$, an importance formula $\impformula$ and extraction formulas $\extformulaof{\catenumerator}$ for $\catenumeratorin$, we define the extraction relation
    \begin{align*}
        \extractionrelation \subset \left(\symindstates\right) \otimes \left(\atomstates\right)
    \end{align*}
    by
    \begin{align*}
        \extractionrelation
        = \{ (\shortindindices, \shortcatindices)
        \, : \,  \groundingofat{\impformula}{\indexedshortindvariables} = 1 \, , \, \forall {\catenumeratorin} : \,  \catindexof{\atomenumerator} = \extformulaofat{\atomenumerator}{\indexedshortindvariables} \} \, .
    \end{align*}
\end{definition}

The encoding of an extraction relation is
\begin{align*}
    \rencodingofat{\extractionrelation}{\shortindvariables,\shortcatvariables} \subset \left(\indspace\right) \otimes \left(\atomspace\right) \,
\end{align*}
and drawn in a contraction diagram by
\begin{center}
    \input{./PartII/tikz_pics/fol_models/extraction_relation.tex}
\end{center}
Here the contraction of $\rencodingof{\impformula}$ with the truth vector $\tbasis$ represents the matching condition posed by $\impformula$ when extracting pairs of individuals.


%% Empirical Distribution
The empirical distribution is then the normalized contraction leaving only the legs to the extracted atomic formulas open, that is
\begin{align*}
    \empdistribution
    = \frac{
        \sbcontractionof{\rencodingof{\extractionrelation}}{\shortcatvariables}
    }{
        \sbcontraction{\rencodingof{\extractionrelation}}
    }  \, .
\end{align*}
Here the number of extracted data is the denominator
\begin{align*}
    \datanum
    = \contraction{\rencodingof{\extractionrelation}}
    = \contraction{\rencodingofat{\impformula}{\headvariableof{\impformula},\shortindvariables},\tbasisat{\headvariableof{\impformula}}}\, .
\end{align*}

We depict this by
\begin{center}
    \input{./PartII/tikz_pics/fol_models/empirical_generation.tex}
\end{center}




\subsect{Basis CP Decomposition of extracted data}

To connect with the empirical distribution introduced in \secref{sec:empDistribution} we now show how the empirical distribution extracted from the interpretations of the formulas $\impformula,\extformulas$ on a first-order logic world $\dataworld$ can be represented by tensor networks.

First of all, we decompose the importance formula into a basis CP format (see \charef{cha:sparseCalculus}), that is a decomposition
\begin{align*}
    \groundingofat{\impformula}{\shortindvariables}
    = \contractionof{
        \{\legcoreofat{\indenumerator}{\indvariableof{\indenumerator},\datvariable} \, : \, \indenumeratorin \}
    }{\shortindvariables}
\end{align*}
such that all $\legcoreofat{\indenumerator}{\indvariableof{\indenumerator},\datvariable}$ are directed and boolean tensors.
Here an auxiliary variables $\datvariable$ taking values in $[\datanum]$ is introduced, which we call the data variable, which enumerates the non-vanishing coordinates of $\groundingof{\impformula}$.
With this decomposition, we can understand the decomposition of $\groundingofat{\impformula}{\shortindvariables}$ as a basis encoding of an term selection map $\secdatamap$ with coordinate maps defined such that
\begin{align*}
    \rencodingofat{\secdatamap_{\indenumerator}}{\indvariableof{\indenumerator},\datvariable}
    = \legcoreofat{\indenumerator}{\indvariableof{\indenumerator},\datvariable} \, .
\end{align*}
We depict this decomposition by:
\begin{center}
    \input{./PartII/tikz_pics/fol_models/impformula_cp.tex}
\end{center}

Based on these construction, we now provide a tensor network decomposition of the extracted empirical distribution.

\begin{theorem}
    Given a first-order logic world $\dataworld$, an importance formula $\impformula$ and extraction formulas $\extformulaof{\catenumerator}$ for $\catenumeratorin$, we have
    \begin{align*}
        \rencodingofat{\extractionrelation}{\shortindvariables,\shortcatvariables} =
        \contractionof{
            \{\rencodingofat{\groundingof{\extformulaof{\atomenumerator}}}{\catvariableof{\catenumerator},\shortindvariables} \, : \, \catenumeratorin\}
            \cup \{\rencodingofat{\secdatamap_{\indenumerator}}{\indvariableof{\indenumerator},\datvariable} \, : \, \indenumeratorin\}
        }{\shortindvariables,\shortcatvariables}
    \end{align*}
    and thus
    \begin{align*}
        \empdistributionat{\shortcatvariables} =
        \frac{1}{\datanum}  \contractionof{
            \{\rencodingofat{\groundingof{\extformulaof{\atomenumerator}}}{\catvariableof{\catenumerator},\shortindvariables} \, : \, \catenumeratorin\}
            \cup \{\rencodingofat{\secdatamap_{\indenumerator}}{\indvariableof{\indenumerator},\datvariable} \, : \, \indenumeratorin\}
        }{\shortcatvariables} \, .
    \end{align*}
\end{theorem}
\begin{proof}
    To show the first claim, let us choose arbitrary state tuples $\shortindindices$ and $\shortcatindices$.
    We then have
    \begin{align*}
        &\contractionof{
            \{\rencodingofat{\groundingof{\extformulaof{\atomenumerator}}}{\catvariableof{\catenumerator},\shortindvariables} \, : \, \catenumeratorin\}
            \cup \{\rencodingofat{\secdatamap_{\indenumerator}}{\indvariableof{\indenumerator},\datvariable} \, : \, \indenumeratorin\}
        }{\indexedshortindvariables,\indexedshortcatvariables} \\
        & \quad  =  \contraction{
            \{\rencodingofat{\groundingof{\extformulaof{\atomenumerator}}}{\indexedcatvariableof{\catenumerator},\indexedshortindvariables} \, : \, \catenumeratorin\}
            \cup \{\rencodingofat{\secdatamap_{\indenumerator}}{\indexedindvariableof{\indenumerator},\datvariable} \, : \, \indenumeratorin\}
        } \, .
    \end{align*}
    This contraction evaluates to $1$, if and only if for all $\catenumeratorin$ we have $\rencodingofat{\groundingof{\extformulaof{\atomenumerator}}}{\catvariableof{\catenumerator},\shortindvariables}=1$ and
    \begin{align*}
        \contraction{\{\rencodingofat{\secdatamap_{\indenumerator}}{\indexedindvariableof{\indenumerator},\datvariable} \, : \, \indenumeratorin\}}  = 1 \, .
    \end{align*}
    The first condition is equal to $\catindexof{\atomenumerator} = \extformulaofat{\atomenumerator}{\indexedshortindvariables}$ for all $\catenumeratorin$ and the second to
    \begin{align*}
        \groundingofat{\impformula}{\indexedshortindvariables} = 1 \, .
    \end{align*}
    Comparing with the definition of the extraction relation (see \defref{def:extractionRelation}), we notice that these conditions are equal to $(\shortindindices,\shortcatindices)\in\extractionrelation$ and therefore to
    \begin{align*}
        \rencodingofat{\extractionrelation}{\indexedshortindvariables,\indexedshortcatvariables} \, .
    \end{align*}
    The first claim follows, since $\rencodingof{\extractionrelation}$ is boolean, as is the contraction of the cores $\rencodingof{\groundingof{\extformulaof{\atomenumerator}}}$ with the cores $\rencodingof{\secdatamap_{\indenumerator}}$, which leaves the outgoing variables $\shortcatvariables$ open.
    The second claim follows from the first using that $\empdistributionat{\shortcatvariables}=\frac{1}{\datanum}\contractionof{\rencodingof{\extractionrelation}}{\shortcatvariables}$.
\end{proof}

To connect with the representation of empirical distributions based on data cores (see \secref{sec:empDistribution}), we now form data cores by contractions with the grounding of extraction formulas with the cores $\rencodingof{\secdatamap_{\indenumerator}}$ (see \figref{fig:datacoreGeneration}),
\begin{align*}
    \datacoreofat{\atomenumerator}{\catvariableof{\catenumerator},\datvariable}
    = \sbcontractionof{
        \{\rencodingofat{\groundingof{\extformulaof{\atomenumerator}}}{\catvariableof{\catenumerator},\shortindvariables}\}
        \cup \{ \legcoreofat{\indenumerator}{\indvariableof{\indenumerator},\datvariable} \, : \, \indenumeratorin\}
    }{\catvariableof{\atomenumerator},\datvariable} \, .
\end{align*}

% Empirical distribution
The empirical distribution is then a tensor network of these tensors, as we show next.

\begin{theorem}
    \label{the:extractionDataCores}
    We have
    \begin{align*}
        \sbcontractionof{\rencodingof{\extractionrelation}}{\shortcatvariables}
        = \contractionof{\{\datacoreofat{\atomenumerator}{\datvariable,\catvariableof{\atomenumerator}} \, : \, \atomenumeratorin\}}{\shortcatvariables}
    \end{align*}
    and thus
    \begin{align*}
        \empdistributionat{\shortcatvariables}
        = \frac{1}{\datanum} \contractionof{\{\datacoreofat{\atomenumerator}{\datvariable,\catvariableof{\atomenumerator}}  \, : \, \atomenumeratorin\}}{\shortcatvariables} \, .
    \end{align*}
\end{theorem}
\begin{proof}
    By \theref{the:extractionrelationDecomposition} we have
    \begin{align*}
        \rencodingofat{\extractionrelation}{\shortindvariables,\shortcatvariables} =
        \contractionof{
            \{\rencodingofat{\groundingof{\extformulaof{\atomenumerator}}}{\catvariableof{\catenumerator},\shortindvariables} \, : \, \catenumeratorin\}
            \cup \{\rencodingofat{\secdatamap_{\indenumerator}}{\indvariableof{\indenumerator},\datvariable} \, : \, \indenumeratorin\}
        }{\shortindvariables,\shortcatvariables} \, .
    \end{align*}
    Since $\rencodingofat{\secdatamap_{\indenumerator}}{\indvariableof{\indenumerator},\datvariable}$ are directed and boolean, they can be copied and separately contracted with each $\groundingof{\extformulaof{\atomenumerator}}$, without changing the contraction.
    We arrive at
    \begin{align*}
        &\rencodingofat{\extractionrelation}{\shortindvariables,\shortcatvariables} \\
        &\quad = \contractionof{
            \big\{\contractionof{
                \{\rencodingofat{\groundingof{\extformulaof{\atomenumerator}}}{\catvariableof{\catenumerator},\shortindvariables}\}
                \cup \{\rencodingofat{\secdatamap_{\indenumerator}}{\indvariableof{\indenumerator},\datvariable} \, : \, \indenumeratorin\}
            }{\catvariableof{\catenumerator},\datvariable} \, : \, \catenumeratorin \big\}
        }{\shortindvariables,\shortcatvariables} \\
        & \quad =  \contractionof{\{\datacoreofat{\atomenumerator}{\datvariable,\catvariableof{\atomenumerator}}  \, : \, \atomenumeratorin\}}{\shortcatvariables} \, ,
    \end{align*}
    which established the claim.
\end{proof}

% Efficient contraction: Do also basis decomposition of the extraction query and use efficient contraction!
%Towards efficient calculation of the data cores, we build a basis CP decomposition of $\groundingof{\impformula}$, where we further demand $\scalarcore=\ones$.
%This is a collection of basis leg cores $\legcoreof{\fixedimpformula,\indenumerator}$ such that
%\begin{align*}
%    \fixedimpformula[\shortindvariablelist]
%    = \contractionof{ \left\{ \legcoreofat{\fixedimpformula,\indenumerator}{\datvariable,\indvariableof{\indenumerator}} \, : \, \indenumeratorin \right\} }{\shortindvariablelist} \, .
%\end{align*}

% Data enumeration -> To representation
%We can further utilize any decomposition of $\impformula$ into a directed and binary CP Format to enumerate the datapoints by the slice index $\datindex$. % Approaches like SPARQL directly give us these by solution mappings.
%Understanding $\impformula$ as a query on the world being the database, such decomposition is given by the set of solution mappings.


\begin{figure}[h]
    \begin{center}
        \begin{tikzpicture}[scale=0.35, yscale=1, thick] % , baseline = -3.5pt


    \draw[->] (4,-1) -- (4,1) node[midway, right] {\tiny $\catvariableof{\atomenumerator}$};
    \draw (3,-1) rectangle (5,-3);
    \node[anchor=center] (text) at (4,-2) {$\datacoreof{\atomenumerator}$};
    \draw[<-] (4,-3) -- (4,-5) node[midway, right] {\tiny $\datvariable$};

    \node[anchor=center] (text) at (7,-2) {${=}$};

    \begin{scope}
        [shift={(10,0)}]

        \draw[->] (3,1) -- (3,3) node[midway, right] {\tiny $\catvariableof{\atomenumerator}$};
        \draw (-1,1) rectangle (7,-1);
        \node[anchor=center] (text) at (3,0) {$\rencodingof{\groundingof{\extformulaof{\atomenumerator}}}$};

        \draw[->] (0,-3) -- (0,-1) node[midway,left] {\tiny $\indvariableof{0}$};
        \draw[->] (3,-3) -- (3,-1) node[midway,left] {\tiny $\indvariableof{1}$};
        \draw[->] (6,-3) -- (6,-1) node[midway,left] {\tiny $\indvariableof{2}$};


    \end{scope}

    \begin{scope}
        [shift={(10,-2)}]

        \coordinate (conposseldec) at (4.5,-5.5);
        \drawvariabledot{4.5}{-5.5}
        \draw[<-] (conposseldec) -- (4.5,-7.5) node[midway, right] {\tiny $\indexvariable$};

        \draw (-1,-1) rectangle (1, -3);
        \node[anchor=center] (text) at (0,-2) {\small $\rencodingof{\secdatamap_0}$};%{\small $\legcoreof{\fixedimpformula,0}$};
        \draw[<-] (0,-3) to[bend right=20] (conposseldec);

        \draw (2,-1) rectangle (4, -3);
        \node[anchor=center] (text) at (3,-2) {\small $\rencodingof{\secdatamap_1}$};%{\small $\legcoreof{\fixedimpformula,1}$};
        \draw[<-] (3,-3) to[bend right=20]  (conposseldec);

        \draw (5,-1) rectangle (7, -3);
        \node[anchor=center] (text) at (6,-2) {\small $\rencodingof{\secdatamap_2}$};%{\small $\legcoreof{\fixedimpformula,2}$};
        \draw[<-] (6,-3) to[bend right=-20]  (conposseldec);

        \draw[<-] (9,1) -- (9,-1) node[midway,left] {\tiny $\indvariableof{3}$};
        \draw (8,-1) rectangle (10, -3);
        \node[anchor=center] (text) at (9,-2) {\small $\rencodingof{\secdatamap_3}$};%{\small $\legcoreof{\fixedimpformula,3}$};
        \draw[<-] (9,-3) to[bend right=-20]  (conposseldec);


        \node[anchor=center] (text) at (12,-2) {$\cdots$};

        \draw[<-] (15,1) -- (15,-1) node[midway,left] {\tiny $\indvariableof{\indorder-1}$};
        \draw (13.5,-1) rectangle (16.5, -3);
        \node[anchor=center] (text) at (15,-2) {\small $\rencodingof{\secdatamap_{\indorder-1}}$};%{\small $\legcoreof{\fixedimpformula,\variableorder-1}$};
        \draw[<-] (15,-3) to[bend left=20]  (conposseldec);


        \draw (8,1) rectangle (16, 3);
        \node[anchor=center] (text) at (12,2) {\small $\ones$};


    \end{scope}


    \node[anchor=center] (text) at (29,-2) {${=}$};


    \begin{scope}
        [shift={(32,0)}]

        \draw[->] (3,1) -- (3,3) node[midway, right] {\tiny $\catvariableof{\atomenumerator}$};
        \draw (-1,1) rectangle (7,-1);
        \node[anchor=center] (text) at (3,0) {$\rencodingof{\groundingof{\extformulaof{\atomenumerator}}}$};

        \draw[->] (0,-3) -- (0,-1) node[midway,left] {\tiny $\indvariableof{0}$};
        \draw[->] (3,-3) -- (3,-1) node[midway,left] {\tiny $\indvariableof{1}$};
        \draw[->] (6,-3) -- (6,-1) node[midway,left] {\tiny $\indvariableof{2}$};


    \end{scope}

    \begin{scope}
        [shift={(32,-2)}]


        \coordinate (conposseldec) at (3,-5.5);
        \drawvariabledot{3}{-5.5}
        \draw[<-] (conposseldec) -- (3,-7.5) node[midway, right] {\tiny $\datvariable$};

        \draw (-1,-1) rectangle (1, -3);
        \node[anchor=center] (text) at (0,-2){\small $\rencodingof{\secdatamap_0}$};%{\small $\legcoreof{\fixedimpformula,0}$};
        \draw[<-] (0,-3) to[bend right=20] (conposseldec);

        \draw (2,-1) rectangle (4, -3);
        \node[anchor=center] (text) at (3,-2) {\small $\rencodingof{\secdatamap_1}$};%{\small $\legcoreof{\fixedimpformula,1}$};
        \draw[<-] (3,-3) to[bend right=0]  (conposseldec);

        \draw (5,-1) rectangle (7, -3);
        \node[anchor=center] (text) at (6,-2) {\small $\rencodingof{\secdatamap_2}$};%{\small $\legcoreof{\fixedimpformula,2}$};
        \draw[<-] (6,-3) to[bend right=-20]  (conposseldec);


    \end{scope}


\end{tikzpicture}
    \end{center}
    \caption{Generation of a data core for the variable $\catvariableof{\catenumerator}$ given an extraction formula $\extformulaof{\catenumerator}$ and an importance formula, which grounding is decomposed into a basis CP format with leg vectors $\rencodingofat{\secdatamap_{\indenumerator}}{\indvariableof{\indenumerator},\datvariable}$.
    Term variables, which are appearing in the importance formula, but not in the extraction formula $\extformulaof{\catenumerator}$ can be treated trivally by contraction with the trivial tensor (here $\indvariableof{4},\ldots,\indvariableof{\indorder-1})$.
    }
    \label{fig:datacoreGeneration}
\end{figure}


% Comment: Exploitation of common structure
When many atom extraction formulas differ only by a constant, we can replace the constant by an auxiliary term variable.
The atoms are then the atomizations of this variable (see \secref{sec:categoricalTN}), treated as a categorical variable, with respect to the constant in the extraction query.
The advantages are that we can avoid the $\rencodingof{}$-formalism and directly model the categorical distributions.

This also enables a batchwise computation of multiple $\sparql$ queries, which differ only in one constant.


%\subsect{Design of the Formulas}
%
%Most intuitive when labeling individuals by classes.
%Extraction formulas $\extformulas$ can then be defined by subclasses of the member of a class and relations between objects of different classes. % Koller calls atomic formulas the template attributes
%We then choose $\formulaset$ as more involved formulas decomposed into connectives acting on these atoms.
%The importance formula $\impformula$ is then designed based on class memberships to ensure, that the arguments of the formulas are always of specific classes. % Koller specifies to each argument of the attributes a class
%
%% Approach
%We propose to
%\begin{itemize}
%    \item Execute an extraction query to get pairs of individuals (the pairDf).
%    \item Propositionalize the FOL Formulas independently on each tuple taking the individuals as a set of constant and filtering on the possible properties of each individuals.
%    (Can understand as adding knowledge that most of the relations do not hold)
%    \item Understand each such generated knowledge base as datapoint and average over them to get the empirical distribution to be fit.
%    \item Fit a MLN describing the statistical relations of unseen results of the extraction query, based on likelihood maximation.
%\end{itemize}




\sect{Generation of first-order logic worlds}

\red{
    So far we have discussed, how MLNs for FOL Knowledge Bases such as Knowledge Graphs can be built by extracting data.
    Conversely, any binary tensor can be interpreted as a Knowledge Graph.
    To be more precise, we follow the intuition that the ones coordinates mark possible worlds compatible with the knowledge about a factored system.
    Each possible world can then be encoded in a subgraph of the Knowledge Graph representing the world.
%
    This amounts to an "inversion" of the data generation process described in the subsection above.
}

In the previous section we have described a way to extract an effective empirical distribution for the likelihood of a first-order logic world given a HFLN.
We now want to investigate methods to reproduce an empirical distribution based on a constructed first-order logic world.

\begin{definition}[Reproduction of Empirical Distributions]
    Given an empirical distribution $\empdistribution\in\atomspace$, we say that a triple $(\dataworld,\impformula,\shortextformulas)$ of a FOL world $\dataworld$ an importance formula $\impformula$ and extraction formulas $\shortextformulas=\{\extformulaof{\atomenumerator}\,:\,\atomenumeratorin\}$ reproduces $\empdistribution$, when
    \begin{align*}
        \empdistribution
        = \normationof{\{\groundingof{\impformula}\}\cup\{\rencodingof{\kggroundingof{\extformulaof{\atomenumerator}}\, : \, \atomenumeratorin}\}}{\shortcatvariables} \, .
    \end{align*}
\end{definition}

% If \datamap is not known
Note that for distribution $\probtensor$ to be reproducable, it needs to have rational coordinates. %, since each coordinate can be interpreted as the frequency of the respective world in the data $\datamap$.
If any only if all coordinates are rational, we find a $\datanum\in\nn$ such that $\imageof{\datanum\cdot\probtensor}\subset\nn$.
We can then interpret $\datanum$ as the number of samples, and construct a sample selector map by understanding each coordinate of $\datanum\cdot\probtensor$ as the number of appearances of the respective world in the samples.

We show different schemes and give examples on Knowledge Graphs, where we provide examples for importance and extraction formulas by $\sparql$ queries.


%\subsect{Example: Generation of Knowledge Graphs} % To generation of FOL worlds?
%
% Having a directed and binary CP decomposition of $\exformula$, each possible world is encoded by a slice.


% Formalization
%\begin{definition}[Reproduction of Empirical Distributions]
%    Given an empirical distribution $\empdistribution\in\bigotimes_{\atomenumeratorin}\rr^2$, we say that a tuple $(\kg,\impformula,\{\extformulas\})$ of a Knowledge Graph $\kg$ and queries $\impformula,\extformulaof{\atomenumerator}$ reproduces $\empdistribution$, when
%    \[\empdistribution = \normationof{\{\kggroundingof{\impformula}\}\cup\{\rencodingof{\kggroundingof{\extformulaof{\atomenumerator}}\, : \, \atomenumeratorin}\}}{\shortcatvariables} \, .  \]
%\end{definition}

%

%In a frequentist interpretation we instantiate each world according to the rate $\probtensor(\atomindices)$.
%This interpretation requires a rounding of the real probabilities by rational numbers.


\subsect{Samples by single objects}

%\subsect{Samples by single objects}

In the first reproduction scheme we construct datapoints by dedicated objects, which represent a sample, that is we choose a domain $\worlddomain=[\datdim]$.

\begin{theorem}
    \label{the:reproducingSingleObjects}
    Let there be an empirical distribution $\empdistribution$ to a sample selector map $\datamap$ (see \defref{def:dataMap}), we construct a world $\dataworld[\selvariable,\indvariable]$ with $\atomorder$ unary predicates by
    \begin{align*}
        \dataworldat{\selvariable,\indvariable}
        = \sum_{\atomenumeratorin} \sum_{\datindexin \, : \datamap_{\atomenumerator}(\datindex)=1} \onehotmapofat{\atomenumerator}{\selvariable} \otimes \onehotmapofat{\datindex}{\indvariable} \, .
    \end{align*}
    We further choose a trivial importance query, that is
    \begin{align*}
        \groundingofat{\impformula}{\indvariable} = \onesat{\indvariable} \, ,
    \end{align*}
    and extraction queries coinciding with the unary predicates, that is for $\atomenumeratorin$
    \begin{align*}
        \extformulaof{\atomenumerator} = \folpredicateof{\atomenumerator} \, .
    \end{align*}
    Then, the triple $(\dataworld,\impformula,\shortextformulas)$ reproduces $\empdistribution$.
%    reproduces with the trivial importance query and extraction queries coinciding with the predicates the dataset $\datamap$.
\end{theorem}
\begin{proof}
    By \theref{the:extractionDataCores} it is enough to show, that the data cores constructed from the data extraction process coincide with those of $\empdistribution$.
    We enumerate to this end the non-vanishing coordinates of $\groundingof{\impformula}$ by the data variable $\datvariable$ taking values $\datindexin$, as
    \begin{align*}
        \groundingofat{\impformula}{\indvariable=\datindex} = 1 \,
    \end{align*}
    and choose
    \begin{align*}
        \secdatamap = \identity \, .
    \end{align*}
    For arbitrary $\atomenumeratorin$ and $\datindexin$ we now have
    \begin{align*}
        \datacoreofat{\atomenumerator}{\catvariableof{\catenumerator},\indexeddatvariable}
        &= \contractionof{
            \rencodingofat{\groundingof{\extformulaof{\atomenumerator}}}{\catvariableof{\catenumerator},\indvariable},
            \legcoreofat{0}{\indvariable,\indexeddatvariable}
        }{\catvariableof{\atomenumerator},\datvariable} \\
        &= \contractionof{
            \rencodingofat{\groundingof{\extformulaof{\atomenumerator}}}{\catvariableof{\catenumerator},\indvariable},
            \onehotmapofat{\secdatamap(\datindex)}{\indvariable}
        }{\catvariableof{\atomenumerator},\indexeddatvariable} \\
        &= \onehotmapofat{\datamap_\atomenumerator(\datindex)}{\catvariableof{\catenumerator}} \, .
    \end{align*}
    This coincides with the slice of the data core of the CP representation of empirical distributions used in \theref{the:empCPRep}.
    Since the slice and the core was arbitrary, the tensor network representations in \theref{the:empCPRep} and \theref{the:extractionDataCores} are equal and thus the triple $(\dataworld,\impformula,\shortextformulas)$ reproduces $\empdistribution$.
\end{proof}


We now give by the next theorem an example of a Knowledge Graph with $\sparql$ queries reproducing and arbitrary empirical distribution.

\begin{theorem}
    \label{the:reproducingKGSingelObjects}
    Let $\empdistribution$ be an empirical distribution to the sample selector $\datamap$.
    We construct a Knowledge Graph of the resources $\worlddomain = \{s_\datindex \, : \, \datindexin\} \cup \{C\} \cup \{C_\atomenumerator \, : \, \atomenumeratorin\}$, where $s_{\datindex}$ represent samples and $C_\atomenumerator$ unary predicates, by
    \begin{align*}
        \kggroundingof{\rdf}
        =
        \sum_{\datindexin}
        \onehotmapof{\indexinterpretationof{s_\datindex}}{\sindvariable}
        \otimes \onehotmapof{\indexinterpretationof{\mathrdftype}}{\pindvariable}
        \otimes \onehotmapof{\indexinterpretationof{C}}{\oindvariable}
        +
        \sum_{\datindexin} \sum_{\atomenumeratorin \, : \, \datamap_{\atomenumerator}(\datindex)=1}
        \onehotmapof{\indexinterpretationof{s_\datindex}}{\sindvariable}
        \otimes \onehotmapof{\indexinterpretationof{\mathrdftype}}{\pindvariable}
        \otimes \onehotmapof{\indexinterpretationof{C_\atomenumerator}}{\oindvariable} \, .
    \end{align*}
    We further define an importance formula by the $\sparql$ query
    \begin{centeredcode}
        \impformula = SELECT \{ ?x \} WHERE \{ ?x \quad \rdftype\quad C \, .\}
    \end{centeredcode}
    and for each $\atomenumeratorin$ an extraction formula by the query
    \begin{centeredcode}
        $\extformulaof{\atomenumerator}$ = SELECT \{ ?x \} WHERE \{ ?x \quad \rdftype \quad $C_\atomenumerator$ \, .\} \, .
    \end{centeredcode}
    Then the triple $(\kg,\impformula,\shortextformulas)$ reproduces $\empdistribution$.
\end{theorem}
\begin{proof}
    We show the theorem analogously to \theref{the:reproducingSingleObjects}, with the slide difference in the importance formula.
    We have for the grounding of $\impformula$ on $\kg$ that
    \begin{align*}
        \kggroundingofat{\impformula}{\indvariable} = \sum_{\datindexin}  \onehotmapof{\indexinterpretationof{s_\datindex}}{\indvariable}
    \end{align*}
    and enumerate the non-vanishing coordinates by $\datvariable$.

    For each extraction formula we have
    \begin{align*}
        \kggroundingofat{\extformulaof{\atomenumerator}}{\indvariable} = \sum_{\datindexin \, : \, \datamap_{\atomenumerator}(\datindex)=1} \onehotmapof{\indexinterpretationof{s_\datindex}}{\indvariable} \,.
    \end{align*}
    It follows that the data cores used in \theref{the:extractionDataCores} are
    \begin{align*}
        \rencodingofat{\datamap_\atomenumerator}{\catvariableof{\atomenumerator},\datindex}
        = \onehotmapofat{0}{\catvariableof{\atomenumerator}} \otimes \left(\sum_{\datindexin \, : \, \datamap_{\atomenumerator}(\datindex)=0} \onehotmapofat{\datindex}{\datvariable}\right)
        +\onehotmapofat{1}{\catvariableof{\atomenumerator}} \otimes \left(\sum_{\datindexin \, : \, \datamap_{\atomenumerator}(\datindex)=1} \onehotmapofat{\datindex}{\datvariable}\right)
    \end{align*}
    and they thus coincide with those in the decomposition in \theref{the:empCPRep}.
    The claim follows therefore with the same argumentation as in the proof of \theref{the:reproducingSingleObjects}.
\end{proof}

%
Let us provide some more insights on the construction of the reproducing Knowledge Graph in \theref{the:reproducingKGSingelObject}.
By the insertions to the one-hot encodings $\onehotmapof{\indexinterpretationof{s_\datindex}}{\sindvariable} \otimes \onehotmapof{\indexinterpretationof{\mathrdftype}}{\pindvariable} \otimes \onehotmapof{\indexinterpretationof{C}}{\oindvariable}$ we mark each sample representing resource by a class and ensure its appearance as a $\mathrm{owl:NamedIndividual}$ in the graph.
The insertions $\onehotmapof{\indexinterpretationof{s_\datindex}}{\sindvariable}\otimes \onehotmapof{\indexinterpretationof{\mathrdftype}}{\pindvariable} \otimes \onehotmapof{\indexinterpretationof{C_\atomenumerator}}{\oindvariable}$ on the other side encode the sample selecting map, by inserting exactly the assertions corresponding with the respective sample.
% 
In this simple Knowledge Graph, Description Logic is expressive enough to represent any formula $\folexformula$ composed of the formulas $\extformulas$.

%
%\begin{theorem}
%    Let there any empirical distribution $\empdistribution\in\bigotimes_{\atomenumeratorin}\rr^2$ and $\datanum\in\nn$ such that $\imageof{\datanum\cdot\empdistribution}\subset\nn$.
%    Then the tuple $(\kg,\impformula,\{\extformulas\})$ defined by a Knowledge Graph
%    \begin{align}
%        \kg =
%        & \bigcup_{\atomindicesin}  \{(
%        s_{j, \atomindices} \quad \mathrm{rdf:type} \quad C ) : j \in [\datanum\cdot\empdistribution(\atomindices)] \}  \\
%        &\bigcup_{\atomindicesin}  \{(
%        s_{j, \atomindices} \quad \mathrm{rdf:type} \quad C_\atomenumerator
%        ) : j \in [\datanum\cdot\empdistribution(\atomindices)], \atomenumeratorin , \atomlegindexof{\atomenumerator}=1\}
%    \end{align}
%    further an importance formula by the query
%    \begin{centeredcode}
%        \impformula = SELECT \{ ?x \} WHERE \{ ?x \quad \rdftype\quad C \, .\}
%    \end{centeredcode}
%    and extraction formulas for each $\atomenumeratorin$ by the query
%    \begin{centeredcode}
%        $\extformulaof{\atomenumerator}$ = SELECT \{ ?x \} WHERE \{ ?x \quad \rdftype \quad $C_\atomenumerator$ \, .\}
%    \end{centeredcode}
%    reproduces $\empdistribution$.
%\end{theorem}
%\begin{proof}
%    With respect to any enumeration of the resources of $\kg$ we have
%    \begin{align}
%        \kggroundingof{\impformula}
%        = \sum_{\atomindicesin} \sum_{j \in [\datanum\cdot\empdistribution(\atomindices)]} \onehotmapof{s_{j, \atomindices} }
%    \end{align}
%    and
%    \begin{align}
%        \kggroundingof{\extformulaof{\atomenumerator}}
%        = \sum_{\atomindicesin \, : \, \atomlegindexof{\atomenumerator} = 1} \sum_{j \in [\datanum\cdot\empdistribution(\atomindices)]} \onehotmapof{s_{j, \atomindices} } \, .
%    \end{align}
%    Summing over the resource variables of these tensors in a contraction we get
%    \begin{align}
%        \contractionof{\{\kggroundingof{\impformula}\}\cup\{\rencodingof{\kggroundingof{\extformulaof{\atomenumerator}}\, : \, \atomenumeratorin}\}}{\shortcatvariables}
%        & = \sum_{\atomenumeratorin}  \datanum\cdot\empdistribution(\atomindices) \cdot \onehotmapof{\atomindices} = \datanum \cdot \empdistribution
%    \end{align}
%    and therefore
%    \begin{align}
%        \normationof{\{\kggroundingof{\impformula}\}\cup\{\rencodingof{\kggroundingof{\extformulaof{\atomenumerator}}}\, : \, \atomenumeratorin\}}{\shortcatvariables} = \empdistribution \, .
%    \end{align}
%\end{proof}







\subsect{Samples by pairs of objects}

%\paragraph{TBox:} The categorical variables of the factored system are the classes.
%We define atomic formulas by the state indicators of each categorical variable as in \secref{sec:categoricalTN}.
%Each such atomic formula corresponds with a sub-class of the classes.
%By definition, each collection of state indicators define thus pairwise disjoint subclasses.
%
%\paragraph{ABox:} The samples are represented by single individuals in the Knowledge Graph.
%Their sub-class memberships corresponding with the categorical variables of the system are instantiated whenever the atom is true in the sample.
%%\subsubsect{Samples by pairs of resources}
%
%\begin{remark}[Refinement of the Samples]
%    We can split each sample node into a pair of individuals.
%    For this we need to specify, which each class membership will be encoded in a unary or binary attribute of the splitted individuals.
%    This specification is possible based on the extraction query and the atomic formulas.
%\end{remark}
%
%%
%Taking any importance query $\impformula$, which has no permutation symmetries, we can instantiate each projection variable for each sample and prepare the links according to the triple patterns.
%When the atom queries $\extformulas$ have different triple patterns compared with $\impformula$, we instantiate those in cases where $\atomlegindexof{\atomenumerator}=1$.


%
We now instantiate multiple objects for each datapoint, one for each variable of the importance formula, i.e. $\worlddomain=[\datdim]\times[\indorder]$
Label individuals $s_{\datindex,\indenumerator}$ by data index and variable index.

\begin{lemma}
    Let there a data map $\datamap$, queries $\impformula,\shortextformulas$ and a first-order logic world containing objects $s_{\datindex,\indenumerator}$ for $\datindexin$ and $\indenumeratorin$
    If
    \begin{align*}
        \kggroundingof{\impformula}
        = \sum_{\datindexin} \bigotimes_{\indenumeratorin} \onehotmapofat{\indexinterpretationof{s_{\datindex,\indenumerator}}}{\indvariableof{\indenumerator}}
    \end{align*}
    and for any $\atomenumeratorin$
    \begin{align*}
        \kggroundingof{\extformulaof{\atomenumerator}}
        = \sum_{\datindex : \datamap_{\atomenumerator}(\datindex)=1} \bigotimes_{\indvariableof{\indenumerator} \in \indvariableof{\extformulaof{\atomenumerator}}}
        \onehotmapofat{\indexinterpretationof{s_{\datindex,\indenumerator}}}{\indvariableof{\indenumerator}} \, .
    \end{align*}
%    \[ \kggroundingof{\extformulaof{\atomenumerator}}
%    = \sum_{\datindex : \datamap^{\atomenumerator}(\datindex)=1} \bigotimes_{\indenumerator \in \extformulaof{\atomenumerator}} \onehotmapof{\datindex,\indenumerator} \, . \]
    Then the tuple $(\kg,\impformula,\{\extformulas\})$ reproduces $\empdistribution$.
\end{lemma}
\begin{proof}
    We notice, that the grounding of the importance formula is in a basis CP format, since by assumption
    \begin{align*}
        \kggroundingof{\impformula}
        = \sum_{\datindexin} \bigotimes_{\indenumeratorin} \onehotmapofat{\indexinterpretationof{s_{\datindex,\indenumerator}}}{\indvariableof{\indenumerator}} \, .
    \end{align*}
    We choose $\datvariable$ to enumerate the non-vanishing entries and get a term selecting map
    \begin{align*}
        \secdatamap_{\indenumerator}(\datindex) = \indexinterpretationof{s_{\datindex,\indenumerator}} \, .
    \end{align*}
    From this we have
    \begin{align*}
        \contractionof{
            \{\rencodingofat{\kggroundingof{\extformulaof{\atomenumerator}}}{\catvariableof{\atomenumerator},\indvariableof{\extformulaof{\atomenumerator}}}\} \cup
            \{\rencodingofat{\secdatamap_{\indenumerator}}{\indvariableof{\indenumerator},\datvariable} \, : \, \indenumeratorin\}
        }{\catvariableof{\catenumerator},\datvariable}
        = \rencodingofat{\datamap_{\atomenumerator}}{\catvariableof{\catenumerator},\datvariable}
    \end{align*}
    and the claim follows with the same argumentation as in the proof of \theref{the:reproducingSingleObjects}.
\end{proof}


%Let us construct a Knowledge Graph
%\begin{align*}
%        \kggroundingof{\rdf}
%        =
%        \sum_{\datindexin}\sum_{\indenumeratorin}
%        \onehotmapof{\indexinterpretationof{s_{\datindex,\indenumerator}}}{\sindvariable}
%        \otimes \onehotmapof{\indexinterpretationof{\mathrdftype}}{\pindvariable}
%        \otimes \onehotmapof{\indexinterpretationof{C}}{\oindvariable}
%        +
%        \sum_{\datindexin} \sum_{\atomenumeratorin \, : \, \datamap_{\atomenumerator}(\datindex)=1} \sum_{\indvariableof{\indenumerator}\in\indvariableof{}}
%        \onehotmapof{\indexinterpretationof{s_{\datindex,\indenumerator}}}{\sindvariable}
%        \otimes \onehotmapof{\indexinterpretationof{\mathrdftype}}{\pindvariable}
%        \otimes \onehotmapof{\indexinterpretationof{C_\atomenumerator}}{\oindvariable} \, .
%\end{align*}
%    We further define an importance formula by the $\sparql$ query
%\begin{centeredcode}
%        \impformula = SELECT \{ ?x_0 \cdots ?x_{\indorder-1} \} WHERE \{ ?x_0 \quad \rdftype\quad C \, .\}
%\end{centeredcode}
%    and for each $\atomenumeratorin$ an extraction formula by the query
%\begin{centeredcode}
%        $\extformulaof{\atomenumerator}$ = SELECT \{ ?x \} WHERE \{ ?x \quad \rdftype \quad $C_\atomenumerator$ \, .\} \, .
%\end{centeredcode}
%    Then the triple $(\kg,\impformula,\shortextformulas)$ reproduces $\empdistribution$.



\sect{Discussion}


% Probabilistic Relational Models
Statistical Models are called Probabilistic Relational Models. % (RUSSELL - Chapter Probabilistic Programming).
Extensions are models that also handle structural uncertainty, i.e. distributions of worlds with varying $\worlddomain$.

% Comparison with network science
In the emerging area of network science \cite{barabasi_network_2016, giovanni_russo_vito_latora_complex_2017}, statistical models for random graphs are investigated.
Statistical Models of first-order logic go beyond the typical single edge type perspective of network science.


%
\begin{remark}[Alternative Representation of empirical distributions]
    So far, we have motivated the representation of empirical distributions based on basis CP decompositions based on data maps.
    In this section, based on the extraction queries, we have observed that empirical distributions might have more efficient representation formats.
    In many applications such as the computation of log-likelihoods we can use any representation of the empirical distribution by tensor networks.
    It is thus not necessary to compute the data cores as above, unless one requires a list of the extracted samples.
\end{remark}



\part{Analysis of the Contraction Calculus}

Based on the logical interpretation we often handle tensor calculus with specific tensors.
Often, they are binary (that is their coordinates are in $\{0,1\}$ corresponding with a Boolean), and sparse (that is having a decomposition with less storage demand).
We investigate it in this part in more depth the properties of such tensors, which where exploited in the previous parts.

\section{Coordinate Calculus} 

In the previous chapters, information to states has been stored in coordinates of a tensor.
To distinguish from other schemes of calculus, we call this scheme of storing and retrieving information the coordinate calculus.
We in this chapter investigate in more depth, which operations can be performed based on such tensors and proof the applied properties.

\red{Add summations and further operations?}

\red{Discuss things like $\expof{\hypercore}$ as coordinatewise operations, by circles in the factor graph diagrams.}

\subsection{One-hot encodings as basis}

\begin{lemma}[Basis of tensor spaces]\label{lem:tensorBasisDecomposition}
	The image of the one-hot encoding map is a linear basis of the tensor space $\facspace$.
	Any element $\exformula\in\facspace$ has a decomposition 
		\[ \exformula[\catvariables] = \sum_{\catindices} \exformula[\indexedcatvariables] \cdot \onehotmapof{\catindices}[\catvariables] \, . \]
	We notice that the coordinates are the weights to the basis elements in the one-hot decomposition.
\end{lemma}
\begin{proof}
	For any $\tildecatindices\in\facstates$ we have
		\[ \formulaat{\shortcatvariables=\tilde{\catindex}_{[\catorder]}}
		= \sum_{\catindices} \exformula[\indexedcatvariables] \cdot \onehotmapofat{\catindices}{\shortcatvariables=\tilde{\catindex}_{[\catorder]}} 
		= \formulaat{\shortcatvariables=\tilde{\catindex}_{[\catorder]}} \cdot \onehotmapofat{\catindices}{\shortcatvariables=\tilde{\catindex}_{[\catorder]}}   \]
\end{proof}


\begin{definition}
	Given any real-valued function 
		\[ \exfunction : \facstates \rightarrow \rr \]
	we define the coordinate encoding by
		\[ \hypercoreof{\exfunction} = \sum_{\catindices\in\facstates} \exfunction(\catindices) \cdot \onehotmapof{\catindices} \, . \]
\end{definition}


\begin{theorem}[Coordinate Calculus]\label{the:coordinateCalculus}
	Given any tensor $\hypercoreat{\catvariables}$ can retrieve its coordinate indexed by $\catindices$ as
		\[ \hypercoreat{\indexedcatvariables} = \sbcontraction{\hypercore, \onehotmapof{\catindices}} \, . \]
\end{theorem}
\begin{proof}
	We use the decomposition in Lemma~\ref{lem:tensorBasisDecomposition} and have
	\begin{align*}
		\contractionof{\{\hypercore, \onehotmapof{\catindices}\}}{\varnothing} 
		& = \sum_{\tildecatindices} \hypercoreat{\tildeindexedcatvariables} \cdot \contractionof{\{\onehotmapof{\tildecatindices} ,\onehotmapof{\catindices} \}}{\catvariables} \\
		& =  \sum_{\tildecatindices} \hypercoreat{\tildeindexedcatvariables} \cdot \delta_{(\tildecatindices),(\catindices)} \\
		& = \hypercoreat{\indexedcatvariables} \, ,
	\end{align*}
	where we used that one-hot encodings are orthonormal.
\end{proof}

% Coordinate Calculus
Coordinate calculus is the representation of real-valued functions as tensors, from which its evaluations can be retrieved by the scheme of Theorem~\ref{the:coordinateCalculus}.


%\red{Incorporate theorem: Contracting tensor network conditioned on indices can be done by contracting the conditioned.}


% Retrieval of Coorinates from tensor networks
Tensors of large orders often admit a decomposition by tensor networks.
We in the next theorem show, how such a decomposition can be exploited for efficient contraction and in particular coordinate retrieval.


\begin{theorem}\label{the:slicedContractionToCores}
	Given a tensor network $\tnetof{\graph}$ on a hypergraph $\graph=(\nodes,\edges)$, disjoint subsets $\nodesa,\nodesb\subset\nodes$ and $\catindexofin{\nodesb}$, we have
		\[ \contractionof{\tnetof{\graph}}{\catvariableof{\nodesa},\indexedcatvariableof{\nodesb}} 
		=  \contractionof{\{
			\sbcontractionof{\hypercoreof{\edge}}{\catvariableof{\edge/\nodesb},\indexedcatvariableof{\nodesb}} \, : \, \edge\in\edges
		\}}{\catvariableof{\nodesa}} \, .
		\]
\end{theorem}
\begin{proof}
	Using a delta tensor, which copies basis vector when contracting with one.
\end{proof}

% Special case of retrieving single coordinates
If we retrieve a single coordinate of a tensor, we have the situation $\nodesa=\varnothing$, $\nodesb=\nodes$.
In that case, Theorem~\ref{the:slicedContractionToCores} shows, that the coordinate is the product of the coordinates of the cores. % Thus no contraction required!



\subsection{Differentiation of Contraction}

We add adiitional variables $\seccatvariable$ selecting a coordinate of a tensor, which is varied in a differentiation.

\begin{lemma}\label{lem:difMNprob}
	For any tensor network $\extnet$ with positive $\hypercoreof{\edge}$ we have
	\begin{align*}
		\difwrt{\hypercoreofat{\edge}{\seccatvariableof{\edge}}} \extnetdist
		& = \sbcontractionof{
	 	\identityat{\seccatvariableof{\edge},\edgevariables}, 
		\frac{\contractionof{\extnet}{\edgevariables}}{\hypercoreofat{\edge}{\edgevariables}}, 
		\normationofwrt{\extnet}{\catvariableof{\nodes/\edge}}{\edgevariables} }{\seccatvariableof{\edge},\nodevariables} \\
		& \quad -  \extnetdist \otimes \sbcontractionof{\frac{\contractionof{\extnet}{\seccatvariableof{\edge}}}{\hypercoreofat{\edge}{\seccatvariableof{\edge}}}
		}{\seccatvariableof{\edge}} \, .
	\end{align*}
\end{lemma}
\begin{proof}
	By multilinearity of tensor network contractions we have
	\begin{align*}
		\difwrt{\hypercoreofat{\edge}{\seccatvariableof{\edge}}} \contractionof{\extnet}{\nodevariables}
		& = \contractionof{\{\identityat{\seccatvariableof{\edge},\edgevariables}\}\cup\{\hypercoreofat{\secedge}{\catvariableof{\secedge}} \, : \, \secedge\neq\edge \}}{\seccatvariableof{\edge},\nodevariables}
	\end{align*}
	and thus	
	\begin{align*}
		\difwrt{\hypercoreofat{\edge}{\seccatvariableof{\edge}}} \contraction{\extnet}
		& = \contractionof{\{\identityat{\seccatvariableof{\edge},\edgevariables}\}\cup\{\hypercoreofat{\secedge}{\catvariableof{\secedge}} \, : \, \secedge\neq\edge \}}{\seccatvariableof{\edge}} \, . 
	\end{align*}
		
	Using both we get
	\begin{align}
		\difwrt{\hypercoreofat{\edge}{\seccatvariableof{\edge}}} \extnetdist 
		& = \difwrt{\hypercoreofat{\edge}{\seccatvariableof{\edge}}}  \frac{\contractionof{\extnet}{\nodevariables}}{\contraction{\extnet}} \nonumber \\
		& = \frac{ \difwrt{\hypercoreofat{\edge}{\seccatvariableof{\edge}}} \contractionof{\extnet}{\nodevariables}}{\contraction{\extnet}} 
		- \frac{ \contractionof{\extnet}{\nodevariables} \difwrt{\hypercoreofat{\edge}{\seccatvariableof{\edge}}} \contraction{\extnet} }{(\contraction{\extnet})^2} \nonumber \\
		& = \frac{ \contractionof{\{\identityat{\seccatvariableof{\edge},\edgevariables}\}\cup\{\hypercoreofat{\secedge}{\catvariableof{\secedge}} \, : \, \secedge\neq\edge \}}{\seccatvariableof{\edge},\nodevariables}}{\contraction{\extnet}} \nonumber \\
		& \quad\quad - \extnetdist \cdot  \frac{\contractionof{\{\identityat{\seccatvariableof{\edge},\edgevariables}\}\cup\{\hypercoreofat{\secedge}{\catvariableof{\secedge}} \, : \, \secedge\neq\edge \}}{\seccatvariableof{\edge}}}{\contraction{\extnet}} \label{eq:differentiatingMNpreresult}
		% = \contractionof{\{\identityat{\seccatvariableof{\edge},\edgevariables}\}\cup\{\hypercoreofat{\secedge}{\catvariableof{\secedge}} \, : \, \secedge\neq\edge \}}{\seccatvariableof{\edge},\nodevariables}
	\end{align}
		
	For the first term we get with a normation equation (see Theorem~\ref{the:normationContractionEQ}) that
	\begin{align*}
		\frac{ \contractionof{\{\identityat{\seccatvariableof{\edge},\edgevariables}\}\cup\{\hypercoreofat{\secedge}{\catvariableof{\secedge}} \, : \, \secedge\neq\edge \}}{\seccatvariableof{\edge},\nodevariables}}{\contraction{\extnet}} 	
		&= \frac{\contractionof{\{\identityat{\seccatvariableof{\edge},\edgevariables}\}\cup\{\hypercoreofat{\secedge}{\catvariableof{\secedge}} \, : \, \secedge\in\edges \}}{\seccatvariableof{\edge},\nodevariables}}{\hypercoreofat{\edge}{\edgevariables}  \cdot \contraction{\extnet}} \\
		&= \frac{
		\sbcontractionof{\identityat{\seccatvariableof{\edge},\edgevariables},\extnetdist}{\seccatvariableof{\edge},\nodevariables}
		}{\hypercoreofat{\edge}{\edgevariables}}  \\ 
		&= \frac{\sbcontractionof{\identityat{\seccatvariableof{\edge},\edgevariables},
			\normationof{\extnet}{\edgevariables},
			\normationofwrt{\extnet}{\catvariableof{\nodes/\edge}}{\edgevariables}
			}{\seccatvariableof{\edge},\nodevariables}
		}{\hypercoreofat{\edge}{\edgevariables}}  \, . 
	\end{align*}
	
	Analogously, we have 
	\begin{align*}
		\frac{ \contractionof{\{\identityat{\seccatvariableof{\edge},\edgevariables}\}\cup\{\hypercoreofat{\secedge}{\catvariableof{\secedge}} \, : \, \secedge\neq\edge \}}{\seccatvariableof{\edge}}}{\contraction{\extnet}} 	
		&= \frac{\sbcontractionof{\identityat{\seccatvariableof{\edge},\edgevariables},
			\normationof{\extnet}{\edgevariables}%,
			%\normationofwrt{\extnet}{\catvariableof{\nodes/\edge}}{\edgevariables}
			}{\seccatvariableof{\edge}}
		}{\hypercoreofat{\edge}{\edgevariables}}  \, . 
	\end{align*}
		
	With \eqref{eq:differentiatingMNpreresult}, we arrive at the claim 
	\begin{align*}
		\difwrt{\hypercoreofat{\edge}{\seccatvariableof{\edge}}} \extnetdist
		& = \sbcontractionof{
	 	\identityat{\seccatvariableof{\edge},\edgevariables}, 
		\frac{\contractionof{\extnet}{\edgevariables}}{\hypercoreofat{\edge}{\edgevariables}}, 
		\normationofwrt{\extnet}{\catvariableof{\nodes/\edge}}{\edgevariables} }{\seccatvariableof{\edge},\nodevariables} \\
		& \quad -  \extnetdist \otimes \sbcontractionof{\frac{\contractionof{\extnet}{\seccatvariableof{\edge}}}{\hypercoreofat{\edge}{\seccatvariableof{\edge}}}
		}{\seccatvariableof{\edge}} \, .
	\end{align*}
	
\end{proof}


% Could put it into contraction equations?
\begin{lemma}\label{lem:difMNExpectation}
	%See Proposition 11.9 in Koller Book.
	For any function $\exfunction(\hypercoreof{\edge})[\nodevariables]$ we have
	\begin{align*}
		 \difwrt{\hypercoreofat{\edge}{\seccatvariableof{\edge}}} &
		\sbcontraction{\extnetdist,\exfunction(\hypercoreof{\edge})[\nodevariables]} \\
		= & 
		\frac{\normationof{\extnet}{\indexedcatvariableof{\edge}}}{\hypercoreofat{\edge}{\indexedcatvariableof{\edge}}} 
		\Big( \sbcontraction{\normationofwrt{\extnet}{\catvariableof{\nodes/\edge}}{\indexedcatvariableof{\edge}}, \exfunction(\hypercoreof{\edge})[\nodevariables,\seccatvariableof{\edge}]} \\
		& \quad \quad \quad \quad \quad - \sbcontraction{\extnetdist, \exfunction(\hypercoreof{\edge})[\nodevariables]}
		\Big) \\
		& + \contraction{ \extnetdist
		\difofwrt{\exfunction(\hypercoreof{\edge})[\nodevariables]}{\hypercoreofat{\edge}{\seccatvariableof{\edge}}}
		}
	\end{align*}
\end{lemma}
\begin{proof}
	By product rule of differentiation we have
	\begin{align*}
		\difwrt{\hypercoreofat{\edge}{\indexedcatvariableof{\edge}}} \sbcontraction{\extnetdist,\exfunction(\hypercoreof{\edge})[\nodevariables]} 
		& =  \sbcontraction{\difwrt{\hypercoreofat{\edge}{\indexedcatvariableof{\edge}}}\extnetdist,\exfunction(\hypercoreof{\edge})[\nodevariables]} \\
		& \quad +  \sbcontraction{\extnetdist,\difwrt{\hypercoreofat{\edge}{\indexedcatvariableof{\edge}}}\exfunction(\hypercoreof{\edge})[\nodevariables]}  \, . 
	\end{align*}
	The claim now follows with the application of Lemma~\ref{lem:difMNprob} on the first term.
\end{proof}






\subsection{Selection Encodings}

Selection encodings as introduced in Definition~\ref{def:selectionEncoding} are best understood in terms of linear mapping interpretations of tensors.
We will first provide by basis encodings a generic relation between the coordinatewise tensor definitions in this work and linear maps.

We then show the utility of this perspective in the representation of composed linear functions.
The results are applicable in the exponential family theory, in the tensor representation of energies and means.

\subsubsection{Tensors as linear maps}

The state sets $\facstates$ can be interpreted as an enumeration of basis elements $\onehotmapof{\catindex}$ of the tensor space $\facspace$.

Along this interpretation, tensors have an interpretation as maps between tensor spaces.

\red{Any tensor and any partition of its variables into two sets can be interpreted as the basis elements of a linear map between the tensor spaces of the respective variables.}

Tensor valued functions on state sets $\facstates$ are an intermediate representation.

\begin{definition}
	Let there be two tensor spaces $V_1$ and $V_2$ with basis by sets $\mathcal{M}_1\subset V_1$ and $\mathcal{M}_2\subset V_2$ of cardinality $\catdimof{1}$ and $\catdimof{2}$, which are enumerated by variables $\individualvariableof{1}$ and $\individualvariableof{2}$.
	The basis encoding of a linear map $\exfunction\in\linmapspace(V_1,V_2)$ is the tensor
		\[ \bencodingof{\exfunction}[\individualvariableof{1},\individualvariableof{2}] \in \rr^{\catdimof{1}} \otimes \rr^{\catdimof{2}} \, . \] 
\end{definition}

% Matrices
Basis encodings are standard linear algebra tools, where matrices are understood as linear maps between vector spaces.

\begin{theorem}\label{the:linearCompositionBasisEncoding}
	If $\linmapof{1}$ is a linear function between $V_1$ and $V_2$  and $\linmapof{2}$ between $V_2$ and $V_3$, and let $\individualvariableof{1},\,\individualvariableof{2}$ and $\individualvariableof{3}$ be enumerations of chosen bases in the spaces.
	We have
	\begin{align*}
		\bencodingofat{\linmapof{2}\circ\linmapof{1}}{\individualvariableof{1},\individualvariableof{3}} 
		= \sbcontractionof{
		\bencodingofat{\linmapof{2}}{\individualvariableof{2},\individualvariableof{3}}, \bencodingofat{\linmapof{1}}{\individualvariableof{1},\individualvariableof{2}}
		}{\individualvariableof{1},\individualvariableof{3}}  \, . 
	\end{align*}
\end{theorem}
\begin{proof}
	By basis decompositions and representations of linear maps as contractions.
\end{proof}

% Matrix Multiplication
A typical instance is matrix multiplication, where matrices understood as representations of linear maps.


\subsubsection{Selection encodings}

Selection encodings (see Definition~\ref{def:selectionEncoding}) are related to basis encodings of linear maps as we show in the next theorem.

\begin{theorem}\label{the:selectionToBasisEncoding}
	Given a function 
		\[ \exfunction : \facstates \rightarrow \parspace \]
	we define a linear map $\linmapof{\exfunction}\in\linmapspace(\facspace,\parspace)$ by the basis elements to $\catindexof{1} \in\facstates$ and $\catindexof{2}\in\parstates$ by
	\begin{align*}
	 	\linmapof{\exfunction}(\onehotmapof{\catindexof{1}},\onehotmapof{\catindexof{2}}) 
		= \sbcontraction{\exfunction(\onehotmapof{\catindexof{1}}),\onehotmapof{\catindexof{2}}} \, .  
	\end{align*}
	We then have
	\begin{align*}
		\sencodingof{\exfunction} = \bencodingof{\linmapof{\exfunction}} \, . 
	\end{align*}
\end{theorem}



%\red{Selection encodings are interpretations of matrifications.
%Along that, maps between factored systems are understood as basis decompositions of linear maps between the tensor spaces.}


% Comparison with relational encodings - definition
While relational encoding works for maps from $\facstates$ to arbitrary sets (which are enumerated), selection encodings as introduced in Definition~\ref{def:selectionEncoding} require and exploit that their image is embedded in a tensor space.

% Slicing
Given a selection encoding of a function, the function is retrieved by slicing with respect to the 
	\[ \exfunction(\catindex) = \sencodingofat{\exfunction}{\indexedcatvariableof{},\selvariable} \, . \]
More generally, we show in the next Lemma how to construct to any tensor and any partition of its variables functions by slicing operations, such that the tensor is the selection encoding of the function.

\begin{lemma}\label{lem:inverseSelectionEncoding} % To be used for MLN - proposal distribution
	Let $\hypercoreat{\nodevariables}$ be a tensor in $\bigotimes_{\nodein}\rr^{\catdimof{\node}}$ and let $\nodesa$, $\nodesb$ be a disjoint partition of $\nodes$, that is $\nodesa\dot{\cup}\nodesb=\nodes$.
	Then the function
		\[ \exfunction : \bigtimes_{\node\in\nodesa}[\catdimof{\node}] \rightarrow \bigotimes_{\node\in\nodesb} \rr^{\catdimof{\node}}  \]
	with coordinates
		\[ \exfunction(\catindexof{\nodesa}) = \hypercoreat{\indexedcatvariableof{\nodesa},\catvariableof{\nodesb}}  \]
	obeys
		\[ \sencodingofat{\exfunction}{\nodevariables} = \hypercoreat{\nodevariables} \, . \]
	Here we have renamed the selection variables $\selvariableof{\nodesa}$ by the categorical variables $\catvariableof{\nodesa}$. 
\end{lemma}
\begin{proof}
	From Theorem~\ref{the:linearCompositionBasisEncoding} using the basis encoding equivalence of Theorem~\ref{the:selectionToBasisEncoding}.
\end{proof}


\begin{example}[Markov Logic Networks and Proposal Distributions]
	% Via inverse selection encodings
	While the statistic of MLN (namely $\fselectionmap$) and the proposal distribution (namely $\tranfselectionmap$) have a common selection encoding, both result from the inverse selection encoding described in Lemma~\ref{lem:inverseSelectionEncoding}.
	We can construct $\tranfselectionmap$ by first building the selection encoding to $\fselectionmap$ and then applying the construction of Lemma~\ref{lem:inverseSelectionEncoding} with $\nodesa=\selvariable$ and $\nodesb=\shortcatvariables$.
\end{example}


% Composition
We use selection encodings to express compositions of functions, based on the next theorem.

\begin{theorem}[Selection Encoding for Linear Compositions]\label{the:linCompSelEncoding}
	Let $\sstat$ be a tensor valued function from $\facstates$ to $\simpleparspace$ with image coordinates $\sstat_\statenumerator$ and let $\exfunction$ be a tensor 
	Then
		\[ \left(\sum_{\statenumeratorin}\exfunction[\selvariableof{\sstat}=\statenumerator]\cdot \sstat_\statenumerator \right) [\shortcatvariables] : \facstates \rightarrow \rr \]
	is represented as
		\[ \left(\sum_{\statenumeratorin}\exfunction[\selvariableof{\sstat}=\statenumerator]\cdot \sstat_\statenumerator \right) [\shortcatvariables] 
		 = \sbcontractionof{\sencodingofat{\sstat}{\shortcatvariables,\selvariableof{\sstat}} , \exfunction[\selvariableof{\sstat}]}{\shortcatvariables} \, . \]
\end{theorem}
\begin{proof}
	The representation holds, since for any $\catindexof{[\atomorder]}\in\facstates$ we have
	\begin{align*}
		\sbcontractionof{\sencodingofat{\sstat}{\shortcatvariables,\selvariableof{\sstat}} , \exfunction[\selvariableof{\sstat}]}{\indexedcatvariableof{[\atomorder]}}  
		= \sum_{\statenumeratorin}\exfunction[\selvariableof{\sstat}=\statenumerator]\cdot\sstat_\statenumerator(\catindexof{[\atomorder]}) \, . 
	\end{align*} 
\end{proof}

% Linear 
This theorem shows, that while relation encodings can represent any composition with another function by a contractions, selection encodings can be used to represent linear transforms.
To see this, we interpret $\sstat$ and $\exfunction$ in Theorem~\ref{the:linCompSelEncoding} as basis decompositions of linear maps.


\subsection{Discussion}

Representations of linear maps is the typical application of tensors, reason for refering to tensor networks as multilinear algebra.


% Properties of Calculus
\section{Directed Tensor Calculus}\label{cha:directedTC}



We in this chapter investigate the properties of tensors, which where arising in probabilistic and logical reasoning.
We observed already before, that
\begin{itemize}
	\item Conditional probability tensors are directed tensors.
	\item Logical formulas are binary tensors.
\end{itemize}

Thus, the set of tensors, which are both directed and binary is of much interest. 
We will show, that they are equal to the set of relational encodings of functions.




\subsection{Directed Tensors}

Directionality as defined in Definition~\ref{def:directedTensor} represents the constraints on the structure of tensors:
Summing over outgoing trivializes the tensor.
%We will use this property to find efficient algorithms.


\begin{definition}[Directed Hypergraph]
	A directed hyperedge is a hyperedge, which node set is split into disjoint sets of incoming and outgoing nodes.
	We say a hypercore $\hypercoreof{\edge}$ decorating a directed hyperedge respects the direction, when it is a conditional probability tensor with respect to the direction of the hyperedge. 
	The hypergraph is acyclic, when there is no nonempty cycle of node tuples $(\node_1,\node_2)$, such that $\node_1$ is an incoming node and $\node_2$ an outgoing node of the same hyperedge.
\end{definition}

%\begin{definition}
%	A directed Tensor Network is a directed hypergraph, which is decorated by tensors respecting the directionality of the respective edges.
%\end{definition}


% Multiple Directions possible
There can be multiple ways to direct a tensor, which an extreme example being the Dirac Delta Tensors.
Further example are relational encodings of invertible functions.

\begin{example}[Dirac Delta Tensors]
	Given a node set $\edge$ colored with a constant dimension $\catdim$ Diracs Delta Tensors are the tensors
		\[ \dirdeltaof{\edge,\catdim} \in \bigotimes_{\node\in\edge} \rr^{\catdim} \]
	with coordinates
	\begin{align}
		\dirdeltaof{\edge,\catdim}_{\catindices} = 
		\begin{cases}
			1 \quad & \text{if} \quad \catindexof{0} = \ldots = \catindexof{\atomorder-1} \\
			0 & \text{else}
		\end{cases} \, . 
	\end{align}
	The contractions with respect to subsets $\secnodes \subset \edge$ are
	\begin{align}
		\contractionof{\{\dirdeltaof{\edge,\catdim}\}}{\catvariableof{\secnodes}} = 
		\begin{cases}
			\catdim & \text{if} \quad \nodes = \varnothing \\ 
			\ones & \text{if} \quad \cardof{\secnodes} = 1\\ 
			\dirdeltaof{\secnodes,\catdim} & \text{else}
		\end{cases} \, .
	\end{align}
	Thus are directed for any orientation of the respective edge with exactly one incoming variable.
\end{example}




% TRUE?
%% Hypernetworks
% We can use Diracs Delta Tensors to represent a tensor network on a hypergraph by a tensor network on a graph (that is edges contain at most two nodes).


\begin{lemma}\label{lem:deltification}
%	Let $\graph$ be a directed hypergraph, such that each node is at most in one hyperedge appearing in the outgoing nodes.
	Let $\graph=(\nodes,\edges)$ be a hypergraph and $\extnet$ a tensor network on $\graph$.
	We build a graph $\secgraph=(\secnodes,\secedges\cup\Delta^{\graph})$ and a tensor network $\tnetof{\secgraph}$ by % ! See Bethe Cluster Graph definition !
	\begin{itemize}
		\item Recolored Edges $\secedges = \{\tilde{\edge} \, : \, \edge\in \edges\}$ where $\tilde{\edge} = \{\node^{\edge} \, : \, \node\in\edge\}$, which decoration tensor $\hypercoreof{\tilde{\edge}}$ has same coordinates as $\hypercoreof{\edge}$
		\item Nodes $\secnodes = \bigcup_{\edge\in\edges}\tilde{\edge}$ %$\secnodes = \bigcup_{\edge\in\edges}\{\node^{\edge} \, : \, \node\in\edge \}$ 
		\item Delta Edges $\Delta^{\graph} =  \big\{ \{\node\} \cup \{\node^{\edge} \, : \, \edge\ni\node \} \, : \, \node\in\nodes \big\} $ each of which decorated by a delta tensor $\delta^{\{\node^{\edge} \, : \, \edge\ni\node \}}$
	\end{itemize}
	 Then we have
	 	\[ \contractionof{\extnet}{\catvariableof{\nodes}} =  \contractionof{\tnetof{\secgraph}}{\catvariableof{\nodes}}  \, . \]
\end{lemma}
\begin{proof}
	For any $\catindexof{\nodes}$ we have 
	\begin{align*}
		 \contractionof{\tnetof{\secgraph}}{\indexedcatvariableof{\nodes}} 
		 & = \contractionof{\{\hypercoreof{\tilde{\edge}}[\catvariableof{\{\node^{\edge} : \node\in\edge\}}] : \edge \in \edges \}\cup 
		 	\{\delta^{\{\node\} \cup \{\node^{\edge} \, : \, \edge\ni\node \}}[\catvariableof{\{\node^{\edge} : \edge\ni\node \}}, \indexedcatvariableof{\node} ]  : \node\in\nodes \}
		 }{\varnothing} \\
		 & =  \contractionof{\{\hypercoreof{\tilde{\edge}}[\catvariableof{\{\node^{\edge} : \node\in\edge\}} = \catindexof{\{\node : \node\in\edge\}} ] : \edge \in \edges \}
		 }{\varnothing} \\
		 & = \contractionof{\extnet}{\indexedcatvariableof{\nodes}} \, ,
	\end{align*}
	which establishes the claim.
\end{proof}



\subsubsection{Normation}


Normed tensors (see Definition~\ref{def:normation}) are directed and directed tensors invariant under normation wrt their incoming and outgoing variable, as we show next.

\begin{theorem}\label{the:normationDirected}
	For any tensor network $\extnet$ on variables $\nodes$ that can be normed with respect to $\innodes$ and $\outnodes$, the normation is directed with $\innodes$ incoming and $\outnodes$ outgoing.
\end{theorem}
\begin{proof}
	We have for any incoming state ${\atomlegindexof{\innodes}\in\bigtimes_{\node\in\innodes}\catdimof{\node}}$ that
	\begin{align*}
		\contractionof{\{\normationofwrt{\extnet}{\innodes}{\outnodes}, \onehotmapof{\atomlegindexof{\innodes}} \}}{\varnothing} 
		& =  \frac{
		\contractionof{\extnet\cup\{\onehotmapof{\atomlegindexof{\innodes}}\}}{\varnothing}
		}{
		\contractionof{\extnet\cup\{\onehotmapof{\atomlegindexof{\innodes}}\}}{\varnothing}
		} \, .
	\end{align*}
	By Definition~\ref{def:directedTensor}, $\normationofwrt{\extnet}{\outnodes}{\innodes}$ is thus directed.
%
%	We have for any basis tensor $\onehotmapof{\atomlegindexof{\secnodes}}$
%	\begin{align}
%		\contractionof{
%			\{ \normationof{\hypercore}{\secnodes}, \onehotmapof{\atomlegindexof{\secnodes}} \}
%		}{
%			\varnothing
%		}
%		= \frac{
%			\contractionof{\{\hypercore, \onehotmapof{\atomlegindexof{\secnodes}} \}}{\varnothing}
%		}{
%			\contractionof{\{\hypercore, \onehotmapof{\atomlegindexof{\secnodes}} \}}{\varnothing}
%		} 
%		= 1 \, . 
%	\end{align}
%	Thus 
%	\begin{align}
%		\contractionof{
%			\{ \normationof{\hypercore}{\secnodes} \}
%		}{
%			\secnodes 
%		}
%		= \ones
%	\end{align}
%	and by Definition~\ref{def:directedTensor}, $\normationof{\hypercore}{\secnodes}$ is directed with the variables $\secnodes$ incoming.
\end{proof}


The normation operation coincides in cases of non-negative tensors with the conditioning of a Markov Network representing a probability distribution.


\subsubsection{Contraction of Directed Tensors}

Let us now investigate, which contractions inherit the directionality of the tensors.

%Next we state that specific contraction of conditional probability tensors are still conditional probability tensors.

%\red{Can be extended to single outgoing legs, by using delta tensors at hyperedges.}

%\begin{theorem}
%	Given a directed acyclic hypergraph, which hyperdedges are decorated by tensor cores respecting the direction.
%	Then the contractions, where all closed nodes appear exactly once as an incoming node and exactly once as an outgoing node, and where all open nodes appear in single hyperedges, are conditional probability tensors
%\end{theorem}
%\begin{proof}
%	It is enough to show this property on the contraction of two hypercores.
%	Since the hypergraph is acyclic, the coinciding nodes are all outgoing on the one and incoming to the other hyperedge.
%	Let the hyperedge with the incoming nodes $\edge_1$ and the one with the outgoing nodes $\edge_2$
%	We need to show that when further contracting the contraction with trivial tensors on the outgoing and basis tensors on the incoming legs we get $1$.
%	For any $\catindex$ and $\seccatindex$ this holds since

%	Here we used in the first equality, that $\hypercoreof{\edge_1}$ is a conditional probability tensor and in the second the same property for $\hypercoreof{\edge_2}$.
%\end{proof}

% Hadamard does not preserve probabilities
%We need to ass as assumption in Theorem~\ref{the:conditionalContractionPreservation}, that each node is to at most one hyperedge and to at most one hyperedge outgoing.
%This is due to the failure of Hadamard products of probability tensors to be probability tensors themself.


%\begin{lemma}\label{lem:twoDirectedContracted}
%	Given two 
%\end{lemma}





\begin{theorem}\label{the:conditionalContractionPreservation}
	Let $\graph=(\nodes,\edges)$ be a directed acyclic hypergraph, such that each node $\node\in\edge$ appears at most in one hyperedge as an outgoing variable and denote $\innodes$ as those nodes, which do not appear as outgoing variables.
	For any tensor network $\extnet$ respecting the direction of $\graph$ we have that
		\[ \contractionof{\extnet}{\catvariableof{\innodes}} = \onesat{\catvariableof{\innodes}} \, , \]
	that is $\contractionof{\extnet}{\catvariableof{\nodes}}$ is a directed tensor with $\innodes$ incoming and $\nodes/\innodes$ outgoing.
\end{theorem}
\begin{proof}
	We show the theorem only for the case of hypergraphs, where variables are appearing at most in two hyperedges.
	If a hypergraph fails to satisfy this assumption, we apply Lemma~\ref{lem:deltification} and add delta tensors copying the variables, which are appearing in multiple tensors, and arrive at a tensor network with nodes appearing in at most two hyperedges.
	When orienting each edge 
	
	% Approach: Contracting with ones
	We show the theorem over induction on the number $n$ of cores.
	\paragraph{$n=1$:} It holds trivially, when $\extnet$ consists of a single core
	\paragraph{$n\rightarrow n-1$:} Let us assume, that the claim holds for graphs with $n-1$ hyperedges and let $\tnetof{\graph}$ be a tensor network with $n$ hyperedges.
	Since the hypergraph is acyclic, we find an edge $\edge\in\edges$ such that all outgoing nodes of $\edge$ are not appearing as an incoming node in any edge. 
	We then apply Theorem~\ref{the:splittingContractions} and get
	\begin{align*}
		\contractionof{\extnet}{\catvariableof{\innodes}} 
		&= \contractionof{
			\tnetof{(\nodes,\edges/\{\edge\})} \cup \{\hypercoreofat{\edge}{\catvariableof{\incomingnodes},\catvariableof{\outgoingnodes}}\}
			}{\catvariableof{\innodes}} \\
		& = \contractionof{
			\tnetof{(\nodes,\edges/\{\edge\})} \cup \{\sbcontractionof{\hypercoreof{\edge}}{\catvariableof{\incomingnodes}} \}
			}{\catvariableof{\innodes}} \\
		& = \contractionof{
			\tnetof{(\nodes,\edges/\{\edge\})} \cup \{\onesat{\catvariableof{\incomingnodes}} \}
			}{\catvariableof{\innodes}} \\
		& \contractionof{
			\tnetof{(\nodes,\edges/\{\edge\})} \}
			}{\catvariableof{\innodes}} \, . 
	\end{align*}
	We then notice that the hypergraph $(\nodes,\edges/\{\edge\})$ has $n-1$ hyperedges and each node appears at most once as an incoming and at most once as an outgoing node.
	Thus, we apply the assumption of the induction and get
	\begin{align*}
		\contractionof{\extnet}{\catvariableof{\innodes}} = \contractionof{
			\tnetof{(\nodes,\edges/\{\edge\})} \}
			}{\catvariableof{\innodes}} = \onesat{\catvariableof{\innodes}} \, . 
	\end{align*}
	
%	% ALTERNATIVE APPROACH	
%	We then iteratively choose two neighbored tensors and replace them by their contraction, which is by Lemma~\ref{lem:twoDirectedContracted} itself directed and not changing the contraction by  Theorem~\ref{the:splittingContractions}.
%	When no such neighbored tensors are left, we have an tensor product of directed tensors, which are not sharing variables.
%	This outer product is trivially directed.
\end{proof}


% Overwork!
Schematically, the direction conservation property argument is depicted as:
\begin{center}
	\begin{tikzpicture}[scale=0.3, thick] % , baseline = -3.5pt


\draw[dashed] (-22,1) rectangle (-14,3); 
\node[anchor=center] (text) at (-18,2) {\small $\ones$};

\draw[->]  (-21,-1)--(-21,1);% node[midway,left] {\tiny $\node_1$}; 
\draw[->]  (-15,-1)--(-15,1);% node[midway,left] {\tiny $\cdots \quad \edge^{out}_1$}; 
\node[anchor=center] (text) at (-18,0) {\small $\cdots$};


\draw (-18,0) ellipse (4 and 0.5);
\node[anchor=center] (text) at (-23,0) {\tiny $\edge^{out}_1$};

\draw (-22,-1) rectangle (-14,-3);
\node[anchor=center] (text) at (-18,-2) {\small $\hypercoreof{\edge_1}$};
\draw[<-]  (-21,-3)--(-21,-5);% node[midway,left]% {\tiny $\node_1$}; 
\draw[<-]  (-19,-3)--(-19,-5) node[midway,left] {\tiny $\cdots$}; 

\draw[<-]  (-17,-3)--(-17,-7);% node[midway,left] {\tiny $\exrandom$}; 
\draw[<-]  (-15,-3)--(-15,-5) node[midway,left] {\tiny $\cdots$};% node[midway,left] {\tiny $\exrandom$}; 
\draw (-15,-5) -- (-15,-7);

\draw (-18,-4) ellipse (4 and 0.5);
\node[anchor=center] (text) at (-23,-4) {\tiny $\edge^{in}_1$};


\draw[dashed] (-18.5,-5) rectangle (-22,-7); 
\node[anchor=center] (text) at (-20.25,-6) {\small $\onehotmapof{\catindex}$};

\draw (-13.5,-6) ellipse (4 and 0.5);
\node[anchor=center] (text) at (-8.5,-6) {\tiny $\edge^{out}_2$};

\draw[dashed] (-10,-5) rectangle (-13,-3); 
\node[anchor=center] (text) at (-11.5,-4) {\small $\ones$};

\draw[<-]  (-10.5,-5)--(-10.5,-7) node[midway,left] {\tiny $\cdots$}; 
\draw[<-]  (-12.5,-5)--(-12.5,-7);

\draw (-17.5,-7) rectangle (-9.5,-9);
\node[anchor=center] (text) at (-13.5,-8) {\small $\hypercoreof{\edge_2}$};
\draw[->]  (-11.5,-11)--(-11.5,-9) node[midway,left] {\tiny $\cdots$}; 
\draw[->]  (-14.5,-11)--(-14.5,-9);

\draw (-13,-10) ellipse (2 and 0.5);
\node[anchor=center] (text) at (-16,-10) {\tiny $\edge^{in}_2$};

\draw[dashed] (-15.5,-11) rectangle (-10.5,-13); 
\node[anchor=center] (text) at (-13,-12) {\small $\onehotmapof{\seccatindex}$};


\node[anchor=center] (text) at (-6,-2) {${=}$};

\node[anchor=center] (text) at (13,-2) {${=} \quad\quad 1 \, \, . $};



\begin{scope}[shift={(21,0)}]

\draw[dashed] (-22,1) rectangle (-14,3); 
\node[anchor=center] (text) at (-18,2) {\small $\ones$};

\draw[->]  (-21,-1)--(-21,1);% node[midway,left] {\tiny $\node_1$}; 
\draw[->]  (-15,-1)--(-15,1);% node[midway,left] {\tiny $\cdots \quad \edge^{out}_1$}; 
\node[anchor=center] (text) at (-18,0) {\small $\cdots$};


\draw (-18,0) ellipse (4 and 0.5);
\node[anchor=center] (text) at (-23,0) {\tiny $\edge^{out}_1$};

\draw (-22,-1) rectangle (-14,-3);
\node[anchor=center] (text) at (-18,-2) {\small $\hypercoreof{\edge_1}$};
\draw[<-]  (-21,-3)--(-21,-5);% node[midway,left]% {\tiny $\node_1$}; 
\draw[<-]  (-19,-3)--(-19,-5) node[midway,left] {\tiny $\cdots$}; 

\draw[<-]  (-17,-3)--(-17,-5);% node[midway,left] {\tiny $\exrandom$}; 
\draw[<-]  (-15,-3)--(-15,-5) node[midway,left] {\tiny $\cdots$};% node[midway,left] {\tiny $\exrandom$}; 


\draw (-18,-4) ellipse (4 and 0.5);
\node[anchor=center] (text) at (-23,-4) {\tiny $\edge^{in}_1$};


\draw[dashed] (-14,-5) rectangle (-17.5,-7); 
\node[anchor=center] (text) at (-15.75,-6) {\small $\tilde{\probtensor}$};


\draw[dashed] (-18.5,-5) rectangle (-22,-7); 
\node[anchor=center] (text) at (-20.25,-6) {\small $\onehotmapof{\catindex}$};


\end{scope}





\end{tikzpicture}
\end{center}



















%\input{PartIII/binary_tensor_calculus.tex}
\section{Contraction Equations}

We have observed, that many concepts and theorems in probability theory and logics can be understood as contraction equations.
We first provide a summary of the used contraction equations.

\subsection{Contraction equations in logical and probabilistic reasoning}

Let us summarize the application of contractions and normation in the definition of
\begin{itemize}
	\item Marginal probabilities (\defref{def:marginalProbability}, \theref{the:marginalContraction})
		\[ \probat{\exrandom} = \sbcontractionof{\probtensor}{\exrandom} \]
	\item Conditional probabilities (\defref{def:conditionalProbability}, \theref{the:conditionalContraction})
		\[ \condprobof{\exrandom}{\secexrandom} = \sbnormationofwrt{\probtensor}{\exrandom}{\secexrandom} \]
	\item The partition function of a Markov Networks 
		\begin{align*}
			\partitionfunctionof{\extnet} = \contractionof{\extnet}{\varnothing}
		\end{align*}
	\item The probability distribution of a Markov Network is (\defref{def:markovNetwork})
		\begin{align*}
			\probtensor^{\extnet} = \normationofwrt{\extnet}{\nodes}{\varnothing}
		\end{align*}
\end{itemize}


Further the following properties have been defined by contraction equations:
\begin{itemize}
	\item $\exrandom$ and $\secexrandom$ are independent when (\defref{def:independence}, \theref{the:independenceProductCriterion})
		\[  \sbcontractionof{\probtensor}{\exrandom,\secexrandom} 
		=  \sbcontractionof{\probtensor}{\exrandom} 
			\otimes  \sbcontractionof{\probtensor}{\secexrandom} \]
	\item $\exrandom$ and $\secexrandom$ are called independent conditioned on $\thirdexrandom$ when (\defref{def:condIndependence}, \theref{the:condIndependenceProductCriterion})
		\[ \sbnormationofwrt{\probtensor}{\exrandom,\secexrandom}{\thirdexrandom} 
		= \sbnormationofwrt{\probtensor}{\exrandom}{\thirdexrandom} 
		\otimes \sbnormationofwrt{\probtensor}{\secexrandom}{\thirdexrandom} \]
\end{itemize}





\subsection{Normation Equations}

\begin{theorem}[Normation as a Contraction Equation]\label{the:normationContractionEQ}
	For any on $\innodes$ normable tensor $\hypercoreat{\catvariableof{\nodes}}$, where $\innodes\dot{\cup}\outnodes=\nodes$, we have
	\begin{align*}
		\sbcontractionof{\hypercore}{\catvariableof{\nodes}} 
		= \sbcontractionof{\sbnormationofwrt{\hypercore}{\catvariableof{\outnodes}}{\catvariableof{\innodes}},\sbcontractionof{\hypercore}{\catvariableof{\innodes}}}{\catvariableof{\nodes}} \, . 
	\end{align*}
\end{theorem}
\begin{proof}
	Let us choose indices $\catindexof{\innodes}$ and $\catindexof{\outnodes}$.
	We have that
	\begin{align*}
		%\sbcontractionof{
		\sbnormationofwrt{\hypercore}{\indexedcatvariableof{\innodes}}{\indexedcatvariableof{\outnodes}}
		%}{\indexedcatvariableof{\innodes},\indexedcatvariableof{\outnodes}} 
		= \frac{
			\sbcontractionof{\hypercore}{\indexedcatvariableof{\innodes},\indexedcatvariableof{\outnodes}} 	
		}{
			\sbcontractionof{\hypercore}{\indexedcatvariableof{\innodes}} 	
		} 
	\end{align*}
	and therefor
	\begin{align*}
		\sbcontractionof{\hypercore}{\indexedcatvariableof{\innodes},\indexedcatvariableof{\outnodes}} = 
		\sbnormationofwrt{\hypercore}{\indexedcatvariableof{\innodes}}{\indexedcatvariableof{\outnodes}}
		\cdot 
		\sbcontractionof{\hypercore}{\indexedcatvariableof{\innodes}} 	
	\end{align*}
	Since the equation holds for arbitrary indices, the claim is established.
\end{proof}


\begin{theorem}[Generic Chain Rule]\label{the:genericChainRule}
	For any Tensor $\hypercoreat{\catvariableof{\nodes}}$ and any total order $\prec$ on the nodes $\nodes$ we have % ! CAN DIRECTLY USE [d] when having the order !
	\begin{align*}
		\hypercoreat{\catvariableof{\nodes}} = 
		\contractionof{
			\{ \sbnormationofwrt{\hypercore}{\catvariableof{\node}}{\catvariableof{\prenodes}}  \, : \nodein \}
		}{\catvariableof{\nodes}}
	\end{align*}
\end{theorem}
\begin{proof}
	We apply \theref{the:normationContractionEQ} on the tensor
	\begin{align*}
		\sbnormationofwrt{\hypercore}{
			\catvariableof{\node},\catvariableof{\afternodes}
		}{
			\indexedcatvariableof{\prenodes}
		} \, ,
	\end{align*}
	where $\nodein$ and $\catindexof{\nodes}$ are chosen arbitrarly.
	For any $\nodein$ we get
	\begin{align*}
		\sbnormationofwrt{\hypercore}{
			\catvariableof{\node},\catvariableof{\afternodes}
			}{
			\catvariableof{\prenodes}
		} 
		= \sbcontractionof{
			\normationofwrt{\hypercore}{
				\catvariableof{\afternodes}
				}{
				\catvariableof{\node},\catvariableof{\prenodes}
				},
			\normationofwrt{\hypercore}{
				\catvariableof{\node}
				}{
				\catvariableof{\prenodes}
				}
		}{
			\catvariableof{\nodes} 
		} \, .
	\end{align*}
	Applying this equation iteratively and making use of the commutation of contractions we get for any $\nodein$
	\begin{align*}
		\sbnormationofwrt{\hypercore}{
			\catvariableof{\node},\catvariableof{\afternodes}
		}{
			\catvariableof{\prenodes}
		}
		= \sbcontractionof{
			\normationofwrt{\hypercore}{
				\catvariableof{\secnode}
			}{
				\catvariableof{\{\thirdnode : \thirdnode \prec \secnode, \thirdnode\neq\secnode\}}
			} 
			\, : \node \prec \secnode
		}{
			\catvariableof{\nodes} 
		} \, .
	\end{align*}
	With the maximal node $\node$, that is the $\node$, such that no $\secnode\in\nodes$ with $\node\prec\secnode$ and $\node\neq\secnode$ exists, this is the claim.
\end{proof}




\subsection{Proof of Hammersley-Clifford Theorem}\label{sec:proofHCTheorem}

\red{Different to the standard case, we show a version of Hammersley-Clifford for hypergraphs.}


\begin{lemma}\label{the:contractionFactorization}
	Let $\hypercoreat{\catvariableof{\nodes}}$ be a positive tensor and $\seccatindexof{\nodes}$ an arbitrary index.
	Then we have 
	\begin{align*}
		\hypercoreat{\catvariableof{\nodes}}
		= \sbcontractionof{
			\big(\sbcontractionof{\hypercore}{\catvariableof{\nodes/\thirdnodes}, \catvariableof{\thirdnodes} = \seccatindexof{\thirdnodes}}\big)^{(-1)^{\cardof{\secnodes}-\cardof{\thirdnodes}}} \, : \, \thirdnodes \subset \secnodes \subset \nodes
		}{\catvariableof{\nodes}} \, ,
	\end{align*}
	where the exponentiation is performed coordinatewise and positivity of $\hypercore$ ensures the well-definedness.
\end{lemma}
\begin{proof}
	It suffices to show, that for an arbitrary index $\catindexof{\nodes}$ be an arbitrary index we have
	\begin{align*}
		\hypercoreat{\indexedcatvariableof{\nodes}}
		= \prod_{\secnodes\subset\nodes} \prod_{\thirdnodes\subset\secnodes}
			\big(\sbcontractionof{\hypercore}{\indexedcatvariableof{\nodes/\thirdnodes}, \catvariableof{\thirdnodes} = \seccatindexof{\thirdnodes}}\big)^{(-1)^{\cardof{\secnodes}-\cardof{\thirdnodes}}} \, .
	\end{align*}
	We do this by applying a logarithm on the right hand side and grouping the terms by $\thirdnodes$ as
	\begin{align*}
		%\lnof{\hypercoreat{\indexedcatvariableof{\nodes}}} 
		& \lnof{\prod_{\secnodes\subset\nodes} \prod_{\thirdnodes\subset\secnodes}
			\sbcontractionof{\hypercore}{\indexedcatvariableof{\nodes/\thirdnodes}, \catvariableof{\thirdnodes} = \seccatindexof{\thirdnodes}}\big)^{(-1)^{\cardof{\secnodes}-\cardof{\thirdnodes}}}} \\
		& = \sum_{\thirdnodes\subset\nodes} \lnof{\sbcontractionof{\hypercore}{\indexedcatvariableof{\nodes/\thirdnodes}, \catvariableof{\thirdnodes} = \seccatindexof{\thirdnodes}}} 
		\left( \sum_{\secnodes\subset\nodes \, : \, \thirdnodes\subset \secnodes} (-1)^{\cardof{\secnodes}-\cardof{\thirdnodes}} \right) \\
		& =  \sum_{\thirdnodes\subset\nodes} \lnof{\sbcontractionof{\hypercore}{\indexedcatvariableof{\nodes/\thirdnodes}, \catvariableof{\thirdnodes} = \seccatindexof{\thirdnodes}}} 
		\left( \sum_{i \in [\cardof{\nodes}-\cardof{\thirdnodes}]}  (-1)^{i}  \binom{\cardof{\nodes}-\cardof{\thirdnodes}}{i}  \right) 
	\end{align*}
	Now, by the generic binomial theorem we have that for $n\in\nn, n \neq 0$
		\[ 0 = (1-1)^n = \sum_{i \in [n]}  (-1)^{i}  \binom{n}{i}   \, . \]
	Therefore, the summands for $\thirdnodes\neq\nodes$ vanish and we have 
	\begin{align*}
			& \lnof{ \prod_{\secnodes\subset\nodes} \prod_{\thirdnodes\subset\secnodes}
			\big(\sbcontractionof{\hypercore}{\indexedcatvariableof{\nodes/\thirdnodes}, \catvariableof{\thirdnodes} = \seccatindexof{\thirdnodes}}\big)^{(-1)^{\cardof{\secnodes}-\cardof{\thirdnodes}}} } \\
			& = \lnof{\hypercoreat{\indexedcatvariableof{\nodes}}}
			\left( \sum_{i \in [0]}  (-1)^{i}  \binom{0}{i}  \right) \\
			& = \lnof{\hypercoreat{\indexedcatvariableof{\nodes}}} \, . 
	\end{align*}
	Applying the exponential function on both sides establishes the claim.
\end{proof}


%\begin{lemma}\label{the:condIndMN}
%	Let $\extnet$ be a tensor network of positive cores and $\catindexof{\nodes}$ an arbitrary index.
%	Then we have
%	\begin{align*}
%		\contractionof{\extnet}{\catvariableof{\nodes}} =
%		\prod_{\secnodes\subset\nodes} \prod_{\thirdnodes\subset\secnodes} 
%		\left(\contractionof{\extnet\cup\{\onehotmapof{\catindexof{\nodes/\thirdnodes}}\}}{\catvariableof{\thirdnodes}}\right)^{(-1)^{\cardof{\secnodes}-\cardof{\thirdnodes}}}
%	\end{align*}
%\end{lemma}
%\begin{proof}
%	Show, that any factor $\contractionof{\extnet\cup\{\onehotmapof{\catindexof{\nodes/\thirdnodes}}\}}{\catvariableof{\thirdnodes}}$ with $\thirdnodes\neq\nodes$ is cancelled (by the $(1-1)^n$ binominal theorem).
%\end{proof}


\begin{lemma}\label{lem:independentContractionFactorization}
	Let $\hypercore$ be a positive tensor, $\secnodes\subset\nodes$ and arbitrary subset and $\catindexof{\secnodes}$ an arbitrary index.
	When there are $\nodesa,\nodesb \in\secnodes$, such that
	\begin{align*}
		\sbnormationofwrt{\hypercore}{\catvariableof{\nodesa,\nodesb}}{\catvariableof{\nodes/\{\nodesa,\nodesb\}}}
		= \sbcontractionof{ 
		\sbnormationofwrt{\hypercore}{\catvariableof{\nodesa}}{\catvariableof{\nodes/\{\nodesa,\nodesb\}}},
		\sbnormationofwrt{\hypercore}{\catvariableof{\nodesb}}{\catvariableof{\nodes/\{\nodesa,\nodesb\}}}
		}{\catvariableof{\secnodes}}
	\end{align*}
	then
	\begin{align*}
 \prod_{\thirdnodes\subset\secnodes} 
		\left(\sbcontractionof{\hypercore}{\indexedcatvariableof{\nodes/\thirdnodes}, \catvariableof{\thirdnodes} = \seccatindexof{\thirdnodes}}\right)^{(-1)^{\cardof{\secnodes}-\cardof{\thirdnodes}}} = 1 \, .
	\end{align*}
\end{lemma}
\begin{proof}
	We abbreviate
		\[ Z_{\thirdnodes} = \sbcontractionof{\hypercore}{\indexedcatvariableof{\nodes/\thirdnodes}, \catvariableof{\thirdnodes} = \seccatindexof{\thirdnodes}} \, . 
 \]
	By reorganizing the sum over $\thirdnodes\subset\secnodes$ into  $\thirdnodes\subset\secnodes/\nodesa\cup\nodesb$ we have
	\begin{align}\label{eq:indContFacProof}
	 	\prod_{\thirdnodes\subset\secnodes} 
		\left(
			Z_{\thirdnodes}
		\right)^{(-1)^{\cardof{\secnodes}-\cardof{\thirdnodes}}} = 
		 \prod_{\thirdnodes\subset\secnodes/\{\nodesa,\nodesb\}} 
		 \left( 
		 	\frac{
				Z_{\thirdnodes} \cdot Z_{\thirdnodes\cup\{\nodesa,\nodesb\}}
			}{
				Z_{\thirdnodes\cup\{\nodesa\}} \cdot Z_{\thirdnodes\cup\{\nodesb\}}
			}
		 \right)^{(-1)^{\cardof{\secnodes}-\cardof{\thirdnodes}}} \, . 
	\end{align}
	From the independence assumption it follows that for any index $\catindex$
	\begin{align*}
		& \sbnormationofwrt{\hypercore}{
			 	\indexedcatvariableof{\nodesa} 
			 }{\indexedcatvariableof{\nodes/\thirdnodes\cup\{\nodesa,\nodesb\}},\catvariableof{\thirdnodes}=\seccatindexof{\thirdnodes},  \indexedcatvariableof{\nodesb} } 
			 \\
		& \quad =  
		\sbnormationofwrt{\hypercore}{
			 	\indexedcatvariableof{\nodesa} 
			 }{\indexedcatvariableof{\nodes/\thirdnodes\cup\{\nodesa,\nodesb\}}, \catvariableof{\thirdnodes}=\seccatindexof{\thirdnodes}} \\
		 & \quad  = 
		\sbnormationofwrt{\hypercore}{
			 	\indexedcatvariableof{\nodesa} 
			 }{\indexedcatvariableof{\nodes/\thirdnodes\cup\{\nodesa,\nodesb\}},\catvariableof{\thirdnodes}=\seccatindexof{\thirdnodes},  \catvariableof{\nodesb} = \seccatindexof{\nodesb}} 
	\end{align*}
	Applying this in each squares bracket term of \eqref{eq:indContFacProof} we get
	\begin{align*}
		\frac{
			Z_{\thirdnodes}
		}{
			Z_{\thirdnodes\cup\{\nodesa\}}
		} 
		& = 
		\frac{
			 \sbnormationofwrt{\hypercore}{
			 	\indexedcatvariableof{\nodesa} 
			 }{\indexedcatvariableof{\nodes/\thirdnodes\cup\{\nodesa,\nodesb\}}, \catvariableof{\thirdnodes}=\seccatindexof{\thirdnodes}, \indexedcatvariableof{\nodesb} } 
		}{
			 \sbnormationofwrt{\hypercore}{
			 	\catvariableof{\nodesa} = \seccatindexof{\nodesa}
			 }{\indexedcatvariableof{\nodes/\thirdnodes\cup\{\nodesa,\nodesb\}} , \catvariableof{\thirdnodes}=\seccatindexof{\thirdnodes}, \indexedcatvariableof{\nodesb}} 
		} \\
		& = 
		\frac{
			 \sbnormationofwrt{\hypercore}{
			 	\indexedcatvariableof{\nodesa} 
			 }{\indexedcatvariableof{\nodes/\thirdnodes\cup\{\nodesa,\nodesb\}}, \catvariableof{\thirdnodes}=\seccatindexof{\thirdnodes}, \catvariableof{\nodesb} = \seccatindexof{\nodesb}} 
		}{
			 \sbnormationofwrt{\hypercore}{
			 	\catvariableof{\nodesa} = \seccatindexof{\nodesa}
			 }{\indexedcatvariableof{\nodes/\thirdnodes\cup\{\nodesa,\nodesb\}}, \catvariableof{\thirdnodes}=\seccatindexof{\thirdnodes},\catvariableof{\nodesb} = \seccatindexof{\nodesb}} 
		} \\
		& = 
		\frac{
			Z_{\thirdnodes\cup\{\nodesb\}}
		}{
			Z_{\thirdnodes\cup\{\nodesa,\nodesb\}}
		} \, . 
	\end{align*}
	Thus, each factor in \eqref{eq:indContFacProof} is trivial, which establishes the claim.
\end{proof}

We are finally ready to proof \theref{the:condIndMN} based on the Lemmata above.

\begin{theorem}[\theref{the:condIndMN}]
	Let $\probat{\catvariableof{\nodes}}$ be a probability distribution and $\graph$ a clique-capturing hypergraph, such that for $\nodesa$, $\nodesb$, $\nodesc$ we have that $\catvariableof{\nodesa}$ is independent of $\catvariableof{\nodesb}$ conditioned on $\catvariableof{\nodesc}$, when $\nodesc$ separates $\nodesa$ and $\nodesb$ in the hypergraph.
	Then there is a Markov Network on $\graph$, which distribution is equal to $\probat{\catvariableof{\nodes}}$.
\end{theorem}
\begin{proof}[Proof of \theref{the:condIndMN}]
	By \lemref{the:contractionFactorization} we have for any index $\catindexof{\nodes}$
	\begin{align*}
		\probat{\indexedcatvariableof{\nodes}} =
		\prod_{\secnodes\subset\nodes} \prod_{\thirdnodes\subset\secnodes} 
		\left(
			\probat{\indexedcatvariableof{\thirdnodes},\catvariableof{\nodes/\thirdnodes}=\seccatindexof{\nodes/\thirdnodes}}
		%	\contractionof{\extnet\cup\{\onehotmapof{\catindexof{\nodes/\thirdnodes}}\}}{\catvariableof{\thirdnodes}}
		\right)^{(-1)^{\cardof{\secnodes}-\cardof{\thirdnodes}}}
	\end{align*}
	For any subset $\secnodes\subset\nodes$, which is not contained in a hyperedge, we find $\nodesa,\nodesb \in\secnodes$ such that $\catvariableof{\nodesa}$ is independendent on $\catvariableof{\nodesb}$ conditioned on $\catvariableof{\secnodes/\{\nodesa,\nodesb\}}$.
	If no such nodes $\nodesa,\nodesb \in\secnodes$ exists, $\secnodes$ would be contained in a hyperedge, since the hypergraph is assumed to be clique-capturing.
	By \lemref{lem:independentContractionFactorization} we then have
	\begin{align*}
	 \prod_{\thirdnodes\subset\secnodes} 
		\left(
			\probat{\indexedcatvariableof{\thirdnodes},\catvariableof{\nodes/\thirdnodes}=\seccatindexof{\nodes/\thirdnodes}}
		\right)^{(-1)^{\cardof{\secnodes}-\cardof{\thirdnodes}}} = 1 \, .
	\end{align*}
	We label by a function 
	\begin{align*}
		\alpha: \{\secnodes : \exists\edge\in\edges: \secnodes \subset \edge \} \rightarrow \edges
	\end{align*}	
	the remaining node subsets by a hyperedge containing the subset.
	We build the tensor
	\begin{align*}
		\hypercoreofat{\edge}{\catvariableof{\edge}} = \prod_{\secnodes \, : \, \alpha(\secnodes) = \edge} \prod_{\thirdnodes\subset\secnodes} 
		\left(
			\probat{\indexedcatvariableof{\thirdnodes},\catvariableof{\nodes/\thirdnodes}=\seccatindexof{\nodes/\thirdnodes}}
		\right)^{(-1)^{\cardof{\secnodes}-\cardof{\thirdnodes}}} \, . 
	\end{align*}
	and get, that 
	\begin{align*}
		\probat{\catvariableof{\nodes}} & = \contractionof{\extnetasset}{\catvariableof{\nodes}} \\
		& = \normationofwrt{\extnetasset}{\catvariableof{\node}}{\varnothing} \, .
	\end{align*}
	We have thus constructed a Markov Network with trivial partition function, which contraction coincides with the probability distribution.
\end{proof}





\subsection{Commutation of Contractions}

We show in the next theorem, that a contractions can be performed by contracting a subnetwork first and then further contracting the result with the rest. 

%% OLD Statement
%\begin{theorem}\label{the:splittingContractions}
%	Let $\tnetof{\graph}$ be a tensor network on a hypergraph $\graph=(\nodes,\edges)$.
%	Let us now split the $\graph$ into two graphs $\graph_1=(\nodes_1,\edges_1)$ and $\graph_2=(\nodes_2,\edges_2)$, such that $\edges_1\dot{\cup}\edges_2=\edges$, and let $\secnodes,\secnodes_2\subset\nodes$ be such that $\nodes_2\cap(\nodes_1\cup\secnodes_2) \subset \secnodes$.
%	%$\nodes_1\cup\nodes_2=\nodes$.
%	If $\nodes_2\cup\secnodes \subset \secnodes_2$ then
%		\[ \contractionof{\tnetof{\graph}}{\secnodes} = \contractionof{\tnetof{\graph_1} \cup \{
%			\contractionof{\tnetof{\graph_2}}{\secnodes_2}
%		\}}{\nodes}   \, . \]
%\end{theorem}
%\begin{proof}
%	By an exchange of summations.
%\end{proof}



\begin{theorem}\label{the:splittingContractions}
	Let $\tnetof{\graph}$ be a tensor network on a hypergraph $\graph=(\nodes,\edges)$.
	Let us now split the $\graph$ into two graphs $\graph_1=(\nodes_1,\edges_1)$ and $\graph_2=(\nodes_2,\edges_2)$, such that $\edges_1\dot{\cup}\edges_2=\edges$, $\nodes_1\cup\nodes_2=\nodes$ and all nodes in $\nodes_2$ are contained in an hyperedge of $\edges_2$.
	We then have
		\[ \contractionof{\tnetof{\graph}}{\catvariableof{\secnodes}} 
		= 
		\contractionof{\tnetof{\graph_1} \cup \{
			\contractionof{\tnetof{\graph_2}}{\catvariableof{\nodes_2\cap(\nodes_1\cup\secnodes)}}
		\}}{\catvariableof{\secnodes}}   \, . \]
\end{theorem}
\begin{proof}
	For any index $\catindexof{\secnodes}$ we show that 
			\[ \contractionof{\tnetof{\graph}}{\indexedcatvariableof{\secnodes}} 
		= 
		\contractionof{\tnetof{\graph_1} \cup \{
			\contractionof{\tnetof{\graph_2}}{\catvariableof{\nodes_2\cap(\nodes_1\cup\secnodes)}}
		\}}{\indexedcatvariableof{\secnodes}}   \, . \]
	By definition we have
	\begin{align*}
		\contractionof{\tnetof{\graph}}{\indexedcatvariableof{\secnodes}} 
		& = \sum_{\catindexof{\nodes/\secnodes}} \prod_{\edge\in\edges} \hypercoreofat{\edge}{\indexedcatvariableof{\edge}} \\
		& = \sum_{\catindexof{\nodes/\secnodes}} 
		 	\left( \prod_{\edge\in\edges_1} \hypercoreofat{\edge}{\indexedcatvariableof{\edge}} \right) 
		 	\cdot \left( \prod_{\edge\in\edges_2} \hypercoreofat{\edge}{\indexedcatvariableof{\edge}}  \right) \\
		& =  \sum_{\catindexof{\nodes_1/\secnodes}} \sum_{\catindexof{\nodes_2/(\secnodes\cup\nodes_1)}} 
			\left( \prod_{\edge\in\edges_1} \hypercoreofat{\edge}{\indexedcatvariableof{\edge}} \right) 
		 	\cdot \left( \prod_{\edge\in\edges_2} \hypercoreofat{\edge}{\indexedcatvariableof{\edge}}  \right) \\
		& =  \sum_{\catindexof{\nodes_1/\secnodes}}  
			\left( \prod_{\edge\in\edges_1} \hypercoreofat{\edge}{\indexedcatvariableof{\edge}} \right) 
		 	\cdot \left( \sum_{\catindexof{\nodes_2/(\secnodes\cup\nodes_1)}}  \prod_{\edge\in\edges_2} \hypercoreofat{\edge}{\indexedcatvariableof{\edge}}  \right) \, .
	\end{align*}
	When contracting the variables $\catvariableof{\nodes_2/(\secnodes\cup\nodes_1)}$ on $\tnetof{\graph_2}$, the variables $\catvariableof{\nodes_2\cap(\secnodes\cup\nodes_1)}$ are left open. 
	We therefore have for any $\catindexof{\nodes_2\cap(\secnodes\cup\nodes_1)}$ 
	\begin{align*}
		\sbcontractionof{\tnetof{\graph_2}}{\indexedcatvariableof{\nodes_2\cap(\secnodes\cup\nodes_1)}} =
		 \left( \sum_{\catindexof{\nodes_2/(\secnodes\cup\nodes_1)}}  \prod_{\edge\in\edges_2} \hypercoreofat{\edge}{\indexedcatvariableof{\edge}}  \right) \, . 
	\end{align*}
	It follows with the above, that 
	\begin{align*}
		\contractionof{\tnetof{\graph}}{\indexedcatvariableof{\secnodes}} 
		& =  \sum_{\catindexof{\nodes_1/\secnodes}}  \left( \prod_{\edge\in\edges_1} \hypercoreofat{\edge}{\indexedcatvariableof{\edge}} \right) \cdot \sbcontractionof{\tnetof{\graph_2}}{\indexedcatvariableof{\nodes_2\cap(\secnodes\cup\nodes_1)}} \\
		& = \contractionof{\tnetof{\graph_1} \cup \{
			\contractionof{\tnetof{\graph_2}}{\catvariableof{\nodes_2\cap(\nodes_1\cup\secnodes)}}
		\}}{\indexedcatvariableof{\secnodes}}   \, . 
	\end{align*}
\end{proof}







\subsection{Support of Contractions}\label{sec:supportContractionEquations}



To state the next theorem we introduce the nonzero function $\nonzerofunction: \rr \rightarrow [2]$ by
\begin{align}
	\nonzeroof{x} = \begin{cases}
	0 & \text{if }x=0 \\
	1 & \text{else}
	\end{cases}
\end{align}
Applied coordinatewise on tensors it marks the nonzero coordinates by $1$.

We show that adding binary tensor cores to an contraction orders the results by the partial ordering introduced in \defref{def:partialFTOrder}

\begin{theorem}[Monotonicity of Tensor Contractions]\label{the:monotonicityBinaryContractions}
	Let $\extnet, \secextnet$ be tensor network of non-negative tensors and $\catvariableof{\secnodes}$ an arbitrary set of random variables. %, and $\tilde{\theta}$ another binary tensor. 
	Then we have
		\[ \nonzeroof{\contractionof{\extnet\cup\secextnet}{\catvariableof{\secnodes}}} \prec
		\nonzeroof{\contractionof{\extnet}{\catvariableof{\secnodes}}} \, .  \]
\end{theorem}
\begin{proof}
	It suffices to show that for any $\catindexof{\secnodes}$ with 
		\[ \nonzeroof{\contractionof{\extnet\cup\secextnet}{\indexedcatvariableof{\secnodes}}}=1 \]
	we also have 
		\[ \nonzeroof{\contractionof{\extnet}{\indexedcatvariableof{\secnodes}}}=1 \, . \]
	For any $\catindexof{\secnodes}$ satisfying the first equation we find an extension $\catindexof{\nodes}$ to all variables of the tensor networks such that
		\[ \contractionof{\extnet\cup\secextnet}{\indexedcatvariableof{\nodes}} > 0 \]
	and it follows that
		\[ \contractionof{\extnet}{\indexedcatvariableof{\nodes}} > 0 \quad\text{and}\quad  \contractionof{\secextnet}{\indexedcatvariableof{\nodes}} > 0  \, . \]
	But this already implies, that 
		\[ \nonzeroof{\contractionof{\extnet}{\indexedcatvariableof{\secnodes}}}=1 \, . \]
\end{proof}

Let us now state an equivalence of the contraction, when we add the result of the same contraction 
\begin{theorem}[Invariance under adding subcontractions]\label{the:invarianceAddingSubcontractions}
	Let $\extnet$ be a tensor network of non-negative tensors with variables $\catvariableof{\nodes}$ and let $\secextnet$ be a subset.
	Then we have for any subset $\catvariableof{\secnodes}$ of $\catvariableof{\nodes}$
		\[ \contractionof{\extnet \cup\{
			\nonzeroof{
			\contractionof{\secextnet}{\catvariableof{\secnodes}}
			}
		\}}{\catvariableof{\nodes}} 
		= \contractionof{\extnet}{\catvariableof{\nodes}}
		\, . \]
	
	%For any sets of leg variables $\randomxof{V},\randomxof{\tilde{V}}$ appearing in $\theta$  we have
	%Then we have
	%	\[ \contractionof{\theta\cup
	%	\nonzeroof{\contractionof{\tilde{\theta}}{\randomxof{\tilde{V}}}}
	%			}{\randomxof{V}} = \contractionof{\theta}{\randomxof{V}} \, . \]
\end{theorem}
\begin{proof}
	For any $\catindexof{\nodes}$ with 
		\[ \contractionof{\extnet}{\indexedcatvariableof{\nodes}} = 0 \]
	we also have 
		\[ \contractionof{\extnet \cup\{
			\nonzeroof{
			\contractionof{\secextnet}{\catvariableof{\secnodes}}
			}
		\}}{\indexedcatvariableof{\nodes}} = 0 \, . \]
	For any $\catindexof{\nodes}$ with 
		\[ \contractionof{\extnet}{\indexedcatvariableof{\nodes}} \neq 0 \]
	we have for the reduction $\catindexof{\secnodes}$ of the index $\catindexof{\nodes}$ that
		\[  \contractionof{\secextnet}{\indexedcatvariableof{\secnodes}} \neq 0 \]
	and thus
	\begin{align*}
		\contractionof{\extnet \cup\{
			\nonzeroof{
			\contractionof{\secextnet}{\catvariableof{\secnodes}}
			}
		\}}{\indexedcatvariableof{\nodes}} 
		= \contractionof{\extnet}{\indexedcatvariableof{\nodes}} \cdot \nonzeroof{
			\contractionof{\secextnet}{\catvariableof{\secnodes}}
			}[\indexedcatvariableof{\secnodes}]
		= \contractionof{\extnet}{\indexedcatvariableof{\nodes}} \, . 
	\end{align*}
%	When the subcore transformed by $\nonzeroof{\cdot}$ contains a zero slice, then this
%	 zero slice is also appearing in the rest contraction.
%	Multiplying a zero slice with zero does not affect the contraction, neither does multiplication with one on any slice.
\end{proof}





\begin{remark}
	Similar statements hold, when dropping the non-negativity assumption on the, but demanding that all variables are left open.
\end{remark}






















% Encoding 
\section{Basis Calculus}\label{cha:tensorEncodings}\label{cha:basisCalculus}

\red{
Basis Calculus stores informations in the selection of basis elements, while coordinate calculus uses the coordinates to each index for storage.
While coordinate calculus is more expressive, basis calculus can be exploited in sparse representations of composed functions.
}

\subsection{Encoding of Subsets and Relations}

Based on the concept of one-hot encodings of states we in this chapter develop the construction of encodings to sets, relations and functions.

\begin{figure}[h]
	\begin{center}
		\begin{tikzpicture}[yscale=0.6]
	\draw[dashed] (-10.5,12) rectangle (5.5,2);
	\node[anchor=center] (text) at (-2.5,11) {Tensors with non-negative coordinates};
	
	\draw[red] (-10,10) rectangle (2.5,5); 
	\node[anchor=center,red] (text) at (-5,9) {Directed Tensors: Conditional probability distributions};
	\draw[blue] (-7.5,7.5) rectangle (5,2.5); 
	\node[anchor=center,blue] (text) at (0,3.5) {Boolean Tensors: Encoding of subsets (see \defref{def:subsetEncoding}) and relations (see \defref{def:daryRelation})};

	\node[anchor=center] (text) at (-2.5,6.5) {Directed and Boolean Tensors: Encoding of functions (see \theref{the:rencodingDirected})};
\end{tikzpicture}
	\end{center}
	\caption{Sketch of the tensors with non-negative coordinates. 
	We investigate in this chapter tensors, which are directed and binary.}
\end{figure}


Here we show how we can use binary numbers to encode the truth of set memberships.

\begin{definition}[Subset Encoding]\label{def:subsetEncoding}
	We say that an arbitrary set $\arbset$ is enumerated by a categorical variables $\individualvariableof{\arbset}$ taking values in $[\catdimof{\arbset}]$, when $\catdimof{\arbset}=\absof{\arbset}$ and there is a bijective function
		\[ \indexinterpretationof{} : [\catdimof{\arbset}] \rightarrow \arbset \, . \]
	Given an set $\arbset$ enumerated by the variable $\individualvariableof{\arbset}$, any subset $\arbsubset\subset\arbset$ is encoded by the tensor $\onehotmapto{\arbsubset}[\individualvariable]$ defined for $\catindex\in[\absof{\arbset}]$ as
	%Let there be a set $\arbset$, which elements we enumerate by $x_{\catindex}$ for $\catindex\in[\absof{\arbset}]$, and let there be a categorical variable $\individualvariable$ taking values in $[\absof{\arbset}]$ selecting the members.
	%We define the encoding of a subset $\arbsubset\subset\arbset$ as the tensor $\onehotmapto{\arbsubset}[\individualvariable]$ with coordinates
% \[ \onehotmapto{\arbsubset}[\individualvariable] : \arbset \rightarrow \{0,1\}\]
%	defined for $x\in\arbset$
	\begin{align}
	 	\onehotmapto{\arbsubset}[\individualvariable={\catindex}] = \begin{cases}
		1 & \text{if } \indexinterpretationofat{}{\catindex} \in \arbsubset \\
		0 & \text{else}
		\end{cases} \, . 
	\end{align}
\end{definition}

%	a representation of the set as
%		\[ \arbset = \left\{x_{\catindex} \, : \, \catindex\in[\absof{\arbset}] \right\} \, . \]



% Decomposition
In a one-hot basis decomposition we have
	\[ \onehotmapto{\arbsubset}[\individualvariable] \coloneqq \sum_{\catindex\in[\cardof{\arbset}] \, : \, \indexinterpretationofat{}{\catindex}\in\arbsubset}\onehotmapofat{\catindex}{\individualvariable} \, . \]
%where $\onehotmapof{x}$ denotes the one-hot encoding of an element $x$ with respect to an enumeration of the elements $\arbset$ by $[\catdim]$ or $\facstates$.

% Explanation
Encoding of subsets as vectors: Each coordinate associated with a possible element, $\{0,1\}$ encoding whether in subset.
The encodings is thus a binary tensor.
Any subset encoding is a binary tensor.

\begin{definition}[Relation Encoding]
	Given two finite sets $\inset$, $\outset$, a relation is a subset of their cartesian product
		\[ \exrelation \subset \inset \times \outset \, . \]
	Given an enumeration of $\inset$ and $\outset$ by the categorical variables $\individualvariableof{\insymbol}$ and $\individualvariableof{\outsymbol}$ and interpretation maps 
	$\indexinterpretationof{\insymbol}$, $\indexinterpretationof{\outsymbol}$
	, we define the encoding of this subset as the tensor $\onehotmapto{\exrelation}[\individualvariableof{\insymbol},\individualvariableof{\outsymbol}]$ with the coordinates
	\begin{align}
		\onehotmapto{\exrelation}[\individualvariableof{\insymbol}=\catindexof{\insymbol},\individualvariableof{\outsymbol}=\catindexof{\outsymbol}]
		= \begin{cases}
		1 & \text{if } (\indexinterpretationofat{\insymbol}{\catindexof{\insymbol}},\indexinterpretationofat{\outsymbol}{\catindexof{\outsymbol}}) \in \exrelation \\
		0 & \text{else}
		\end{cases} \, . 
	\end{align}
\end{definition}

% Decomposition
The relation encoding has a decomposition into one-hot encodings as
	\[ \onehotmapto{\exrelation}[\individualvariable_{\insymbol},\individualvariable_{\outsymbol}]
	 = \sum_{\catindexof{\insymbol},\catindexof{\outsymbol} \, : \, (\indexinterpretationofat{\insymbol}{\catindexof{\insymbol}},\indexinterpretationofat{\outsymbol}{\catindexof{\outsymbol}}) \in \exrelation}
	\onehotmapofat{\catindexof{\insymbol}}{\catvariableof{\insymbol}}  \otimes \onehotmapofat{\catindexof{\outsymbol}}{\catvariableof{\outsymbol}}  \, . \]

Relations are subsets of cartesian products and encodings of relations are the encodings of subsets by vectors.
They have a matrix structure by the cartesian product, which can be further folded to tensors, when the sets itself are cartesian products.


\begin{theorem}
	The relational encoding is a bijection between the set of relations and the set of binary tensors.
\end{theorem}
\begin{proof}
	% =>
	By definition, a relational encoding is the encoding of a subset and thus a binary tensor.
	% <= 
	Any matrification of a binary tensor marks by its $1$ coordinates the elements of a relation.
\end{proof}

% Significance
We can thus understand any matrification of a binary tensor as the encoding of a relation and vice versa.



\subsubsection{Higher order relations}


We can extend this contraction to relations of higher order, and arrive at encoding schemes usable for relational databases.

\begin{definition}\label{def:daryRelation}
	Given sets $\arbset^{\atomenumerator}$ for $\atomenumeratorin$, a $\atomorder$-ary relation is a subset of a their cartesian product, that is
		\[ \exrelation \subset  \bigtimes_{\atomenumeratorin} \arbset^{\atomenumerator} \, . \]
	Given an enumeration of each set $\arbset^{\atomenumerator}$ by a variable $\individualvariableof{\atomenumerator}$ and an interpretation map $\indexinterpretationof{\atomenumerator}$, we define the encoding of the relation as the tensor $\onehotmapto{\exrelation}[\individualvariableof{[\atomorder]}]$ with coordinates
	\begin{align}
		\onehotmapto{\exrelation}[\individualvariableof{0}=\catindexof{0},\ldots,\individualvariableof{\atomorder-1}=\catindexof{\atomorder-1}]
		= \begin{cases}
		1 & \text{if } (\indexinterpretationofat{0}{\catindexof{0}},\ldots,\indexinterpretationofat{\atomorder-1}{\catindexof{\atomorder-1}}) \in \exrelation \\
		0 & \text{else}
		\end{cases} \, . 
	\end{align}
\end{definition}


\begin{example}[Propositional Formulas]
	Propositional formulas are equal to the subset encoding of their models.
	The sets $\arbset^{\atomenumerator}$ are all $[2]$ and are interpreted as the possible assignments to the boolean atoms.
\end{example}


\begin{example}[Relational Databases]
	Relational Databases can be encoded as tensors using the relation encoding scheme.
	Each column is thereby understood as a categorical variable, which values form the sets $\arbsetof{\catenumerator}$.
\end{example}

% Sparse Representations
Let us notice, that the dimensionality of the tensor space used for representing a relation is 
	\[ \prod_{\catenumeratorin} \cardof{\arbsetof{\catenumerator}} \]
and therefore growing exponentially with the number of variables.
Relations are however often sparse, in the sense that 
	\[ \cardof{\exrelation} << \prod_{\catenumeratorin} \cardof{\arbsetof{\catenumerator}} \, . \]
It is therefore often benefitially to choose sparse encoding schemes, for example by restricted CP formats (see Chapter~\ref{cha:sparseTC}) to represent $\onehotmapof{\exrelation}$.

\subsection{Encoding of Functions}

Real valued functions are directly tensors by definition.

\subsubsection{Relational Encoding of Functions}

% Reference to first definition
In Definition~\ref{def:functionRepresentation} we have introduced the relational encoding of functions between the states of factored systems.
We now generalize the representation scheme towards maps between arbitrary unstructured sets.

\begin{definition}[Relational Encoding of Maps]\label{def:functionRelationEncoding}
	Any map
		\[ \exfunction : \inset \rightarrow \outset \]
	can be represented by a relation
		\[ \exrelationof{\exfunction} \coloneqq \left\{ (x,\exfunction(x) \, : \, x \in \inset )\right\} \subset \inset \times \outset \, . \]
	Given a enumeration of the sets by $\individualvariableof{\insymbol}$ and $\individualvariableof{\outsymbol}$ we define the relational encoding of $\exfunction$ as the tensor
		\[ \rencodingofat{\exfunction}{\individualvariableof{\insymbol},\individualvariableof{\outsymbol}} = \onehotmapto{\exrelationof{\exfunction}}\left[\individualvariableof{\inset},\individualvariableof{\outset}\right]  \, . \]
\end{definition}

\begin{remark}[Reduction to images]
	% Image enumeration
	When $\exfunction$ maps into a set of infinite cardinality, we restrict $\outset$ to the image of $\exfunction$ and enumerate the image by a variable $\catvariableof{\exfunction}$.
	This scheme is applied, when $\exfunction$ is itself a tensor, i.e. $\outset=\rr$.
	While the variable $\catvariableof{\exfunction}$ can in general be of the same cardinality as the domain set $\inset$, it will be in $[2]$ when considering binary tensors.
\end{remark}

% Characterization of the directed and binary tensors
We notice, that any relational representation of a function is also a directed tensor with incoming variables to the domain and outgoing variables to the image.
It furthermore holds, that the set of directed and binary tensors is characterized by the relational encoding of functions.
This is shown in the next theorem, by the claim that any binary tensor which is directed is the relational representation of a function.

\begin{theorem}\label{the:rencodingDirected}
	Let $\inset,\outset$ be sets and $\exrelation\subset\inset\times\outset$ a relation.
	If and only if there exists a map $\exfunction:\inset\rightarrow\outset$ such that $\exrelation=\exrelationof{\exfunction}$, the relational encoding $\rencodingof{\exfunction}$ is a directed tensor with $\individualvariableof{\insymbol}$ incoming and $\individualvariableof{\outsymbol}$ outgoing.
%	If and only if the relation is a function between domain and image set, its encoding is directed with domain incoming and image outgoing.
\end{theorem}
\begin{proof}
	% =>
	When $\exfunction$ is a function, we have for any $\indindexofin{\insymbol}$
		\[ \sum_{\indindexofin{\outsymbol}} \rencodingofat{\exfunction}{\indexedindvariableof{\insymbol},\indexedindvariableof{\outsymbol}}
		=  \rencodingof{\exfunction}[\indexedindvariableof{\insymbol},\indvariableof{\outsymbol}=\exfunction(\indindexof{\insymbol})] = 1 \, . \]
	% <=
	Conversely let there be a relation $\exrelation$, such that $\rencodingof{\exrelation}$ is directed. %, we construct a map $\exfunction$ with $\exrelation=\exrelationof{\exfunction}$.
	To this end, we observe that for any $\indindexofin{\insymbol}$ the tensor
		\[  \onehotmapofat{\exrelation}{\indexedindvariableof{\insymbol},\indvariableof{\outsymbol}} \]
	is a binary tensor with coordinate sum one and therefore a basis vector.
	It follows that the function $\exfunction : \inset \rightarrow \outset $ 
		\[ \exfunction(\indindexof{\insymbol}) = \invonehotmapof{\onehotmapofat{\exrelation}{\indexedindvariableof{\insymbol},\indvariableof{\outsymbol}} } \]
	is well-defined.
	We then have by construction
	\begin{align*}
		\rencodingof{\exfunction} 
		= \sum_{\indindexofin{\insymbol}} \onehotmapof{\indindexof{\insymbol}} \otimes \onehotmapof{\exfunction(\indindexof{\insymbol})}
		=  \sum_{\indindexofin{\insymbol}} \onehotmapof{\indindexof{\insymbol}} \otimes \onehotmapofat{\exrelation}{\indexedindvariableof{\insymbol},\indvariableof{\outsymbol}} 
		= \onehotmapof{\exrelation}
	\end{align*}
	and therefore $\exrelation = \exrelationof{\exfunction}$.
%	 with a directed encoding, directionality implies that for any $\incatindex$ there is exactly one $\outcatindex$ such that $\onehotmapto{\exrelation}[\indexedcatvariableof{\insymbol}\indexedcatvariableof{\outsymbol}]=1	
%	Then we define a function $\exfunction$ as the mapping of $\incatindex$ to the corresponding $\outcatindex$ with $\onehotmapto{\exrelation}[\incatindex\outcatindex]=1$ and observe $\exrelation=\exrelationof{\exfunction}$.
\end{proof}

% Grid sets
We are specially interested in sets of states of a factored system, which amounts to the case in Definition~\ref{def:functionRepresentation}.
Those state sets have a decomposition into a cartesian product of $\atomorder$ sets
	\[ \arbset = \facstates \, . \]
The most obvious enumeration of the set $\arbset$ is therefore by the collection of state variables $\{\catvariableof{\atomenumerator} \, : \, \atomenumeratorin \}$.
Functions between states of factored systems with $\atomorder_{\insymbol}$ and $\atomorder_{\outsymbol}$ state variables can be represented by $\atomorder_{\insymbol}+\atomorder_{\outsymbol}$-ary relations and Definition~\ref{def:functionRelationEncoding} has an obvious generalization to this case with multiple enumeration variables.


% Conditional 
Since the relational encoding of any map between factored systems is directed, it can be interpreted by a conditional probability tensor, as we state next.

%% Maps
\begin{corollary}%\label{the:condProbFunctionRepresentation}
	The relational encoding $\rencodingof{\exfunction}$ (see Definition~\ref{def:functionRepresentation}) of a function $\exfunction$ between factored systems is a conditional probability tensor, where the legs to the image system are the conditions and the legs to the target system the distribution legs.
\end{corollary}
%\begin{proof}
%	To proof the claim we need to show that contracting the trivial tensor to the target legs results in the trivial tensor of the image legs, that is
%		\[ \onesof{\shortcatvariables} = \contractionof{\{\rencodingof{\exfunction}\}}{\shortcatvariables} \, . \]
%	This can be shown coordinatewise by sums over the target legs as
%	\begin{align*}
%		\sum_{\seccatindices} \ftensorof{\exfunction}_{\seccatindices,:}
%		 	= &  \sum_{\seccatindices} \sum_{\catindices}  \left(\onehotmapof{\exfunction(\catindices)}\right)_{\seccatindices} \otimes \onehotmapof{\catindices} \\
%			= &  \sum_{\catindices} \onehotmapof{\catindices}  \\
%			= &  \onesof{\shortcatvariables}  \, .
%	\end{align*}
%	Here we used in the second equality, that the function $\exfunction$ maps each index pair $(\catindices)$ of the image system to exactly one index pair $(\seccatindices)$ in the target system.
%	In the third equality we further used that the sum of the one-hot encodings of all states is the trivial tensor.
%\end{proof}

%% Deterministic by construction
These are deterministic conditional probability tensors, in the sense that any slice with respect to the input variables is a basis tensor.
Through contractions with distribution tensors (e.g. distributions in domain systems) they get stochastic.
This is for example the case in the empirical distribution, which can be understood as the forwarding of the uniform distribution on the sample enumeration.








\subsection{Calculus of relational encodings}



\subsubsection{Function Evaluation}

\begin{theorem}[Basis Calculus]\label{the:basisCalculus}
	Retrieving the value of the function at a specific state is then the contraction of the tensor representation with the one-hot encoded state.
	For any state indexed by $(\catindices)$ we have
		\[ \onehotmapofat{\exformula(\catindices)}{\catvariableof{\exformula}} 
		= \contractionof{\{\rencodingof{\exformula},\onehotmapof{\catindices}\}}{\catvariableof{\exformula}} \, .\]
	Thus, we can retrieve the function evaluation by the inverse one-hot mapping as
		\[ \exformula(\catindices) 
		= \invonehotmapof{\contractionof{\{\rencodingof{\exformula},\onehotmapof{\catindices}\}}{\catvariableof{\exformula}}} \, . \]
	This generalizes Example~\ref{exa:atomicFunction} to arbitrary maps between factored systems.
\end{theorem}
\begin{proof}
	By doing the summation in tensor product notation.
\end{proof}

%% Usage: Basis Calculus
We can thus use tensor contractions to calculate the values of functions.
Since basis vectors being the one-hot encoding of the domain system are mapped to basis vectors being the encoding of the image system, we call these contraction basis calculus.

\subsubsection{Composition of function}

We have already used, that combination of propositional formulas by connectives can be represented by contractions.
In a more general perspective, any composition of functions between factored systems is in its relational encoding the contraction of the encoded functions.

\begin{theorem}[Composition of Functions]\label{the:compositionByContraction}
	Let there be two maps between factored systems 
		\[ \exfunction : \bigtimes_{\node\in\nodes_1} [\catdimof{\node}] \rightarrow \bigtimes_{\node\in\nodes_2} [\catdimof{\node}] \]
	and 
		\[ \secexfunction : \bigtimes_{\node\in\nodes_2} [\catdimof{\node}] \rightarrow \bigtimes_{\node\in\nodes_3} [\catdimof{\node}] \]
	with the image system of $\exfunction$ is the domain system of $\secexfunction$.
	Then the relational encoding of the composition 
		\[ \secexfunction(\exfunction) :  \bigtimes_{\node\in\nodes_1} [\catdimof{\node}] \rightarrow \bigtimes_{\node\in\nodes_3} [\catdimof{\node}] \]
	satisfies
		\[ \rencodingof{\secexfunction(\exfunction)} = \contractionof{\{\rencodingof{\exfunction},\rencodingof{\secexfunction}\}}{\nodes_1\cup\nodes_3} \, .  \]
\end{theorem}
\begin{proof}
	By definition we have
	\begin{align*}
		\rencodingof{\secexfunction(\exfunction)} = \sum_{i_1 \in \bigtimes_{\node\in\nodes_1} [\catdimof{\node}]} \onehotmapof{i_1} \otimes \onehotmapof{\secexfunction(\exfunction(i_1))} \, ,
	\end{align*}
	where by $i_1$ we denote in a slide abuse of notation the tuple indexing the state of the domain factored systems of $\exfunction$.
	On the other side, we have that
	\begin{align*}
	 	\contractionof{\{\rencodingof{\exfunction},\rencodingof{\secexfunction}\}}{\catvariableof{\nodes_1}\catvariableof{\nodes_3}} 
		& =
		\sum_{\tilde{i}_2 \in \bigtimes_{\node\in\nodes_2}[\catdimof{\node}]} 
		\left( \sum_{i_1 \in \bigtimes_{\node\in\nodes_1}[\catdimof{\node}]}  \onehotmapof{i_1} \cdot \left(\onehotmapof{\exfunction(i_1)}\right)_{\tilde{i}_2} \right)
		\left( \sum_{i_2 \in \bigtimes_{\node\in\nodes_2}[\catdimof{\node}]}  \left(\onehotmapof{i_2}\right)_{\tilde{i}_2} \cdot \onehotmapof{\secexfunction(i_2)} \right) \\
		& = 
		\sum_{\tilde{i}_2 \in \bigtimes_{\node\in\nodes_2}[\catdimof{\node}]} 
		\left( \sum_{i_1 \in \bigtimes_{\node\in\nodes_1}[\catdimof{\node}]}  \onehotmapof{i_1} \cdot \delta_{\exfunction(i_1),\tilde{i}_2} \right)
		\left( \sum_{i_2 \in \bigtimes_{\node\in\nodes_2}[\catdimof{\node}]}  \delta_{i_2 \tilde{i}_2} \cdot \onehotmapof{\secexfunction(i_2)} \right) \\
		& = 
		\sum_{i_1 \in \bigtimes_{\node\in\nodes_1}[\catdimof{\node}]}
		\sum_{\tilde{i}_2 \in \bigtimes_{\node\in\nodes_2}[\catdimof{\node}]}  \delta_{\exfunction(i_1),\tilde{i}_2} \cdot \left(  \onehotmapof{i_1}   \otimes \onehotmapof{\secexfunction(i_2)} \right) \\
		& = 
		\sum_{i_1 \in \bigtimes_{\node\in\nodes_1} [\catdimof{\node}]} 
		\onehotmapof{i_1} \otimes \onehotmapof{\secexfunction(\exfunction(i_1))} \, .
	\end{align*}
	Here we represented the contraction of the variables in $\nodes_2$ by the summation over another index $i_2$.
	In the last equation we used that the delta tensor does not vanish only for $\tilde{i}_2= \exfunction(i_1)$.
	The claim follows, since both expressions are equal.
\end{proof}

% Iterative usage
We can use Theorem~\ref{the:compositionByContraction} iteratively to further decompose the function $\secexfunction$.
In this way, the relational encoding of a function consistent of multiple compositions can be represented as the contractions of all the functions.
The relational encoding of propositional formulas for instance can in this way be represented as a contraction of the encodings of its logical connectives applied on the respective formula spaces. 
%It is thereby of central importance that for any connective a respective variable is added, which then disappears in the contractions.




\subsubsection{Compositions with real functions}

\red{Follows from composition calculus above with the usage of }
	\[ \hypercore = \sbcontractionof{\rencodingof{\hypercore}, \restrictionofto{Id}{\imageof{\hypercore}} }{\shortcatvariables} \, . \]

We here investigate how the composition of a tensor 
	\[ \hypercore : \facstates \rightarrow \rr \]
with arbitrary functions 
	\[ \chainingfunction: \rr \rightarrow \rr \]
can be represented.
This is for example relevant, when representing the distributions of an exponential family.

% Strategy
Our main strategy is in understanding the tensor $\hypercore$ as a map to its finite image, seen as the enumerated states of a categorical variable building a factored system.
We then use the relational encoding $\rencodingof{\hypercore}$ of this map between factored systems. 

By $\restrictionofto{\chainingfunction}{\mathcal{M}}$ we further denote the restriction of a real function $\chainingfunction$ to an enumerated set $\mathcal{M} =\{x_i \, : \, i \in [\cardof{\mathcal{M}}]\} \subset \rr$, i.e. the vector
	\[ \restrictionofto{\chainingfunction}{\mathcal{M}} : [\cardof{\mathcal{M}}] \rightarrow \rr \]
defined for $i \in [\cardof{\mathcal{M}}]$ as
	\[ \restrictionofto{\chainingfunction}{\mathcal{M}}(i) = \chainingfunction(x_i) \, . \]


\begin{theorem}\label{the:tensorFunctionComposition}
	We have for any tensor $\hypercore$ and real function $\chainingfunction$ (see Figure~\ref{fig:tensorFunctionComposition})
		\[ \chainingfunction\circ\hypercore = \contractionof{\{\rencodingof{\hypercore}, \restrictionofto{\chainingfunction}{\imageof{\hypercore}}\}}{\shortcatvariables} \, . \]
\end{theorem}
\begin{proof}
	We enumerate the image $\mathrm{im}(\hypercore)$ by $\{x_i \, : \, i \in [\cardof{\mathrm{im}(\hypercore)}]\}$.
	For arbitrary but fixed $\catindices\in\facstates$ let $\tilde{i}$ be such that $\hypercore(\catindices) = x_i$.
	Then we have for $\randomxof{\hypercore}$ denoting the image variable of $\rencodingof{\hypercore}$ that
		\[ \contractionof{\{\rencodingof{\hypercore},\onehotmapof{\catindices}\}}{\{\randomxof{\hypercore}\}} = \onehotmapof{\tilde{i}} \]
	and
		\[ \contractionof{\{\rencodingof{\hypercore}, \restrictionofto{\chainingfunction}{\imageof{\hypercore}}, \onehotmapof{\catindices}\}}{\varnothing} = 
		\contractionof{\{\restrictionofto{\chainingfunction}{\imageof{\hypercore}}, \onehotmapof{\tilde{i}}\}}{\varnothing} = 
		\chainingfunction(\hypercore(\catindices)) \, . 
		\]
	Since $\catindices$ was chosen arbitrarly form $\facstates$, this shows that $\chainingfunction\circ\hypercore$ and $ \contractionof{\{\rencodingof{\hypercore}, \restrictionofto{\chainingfunction}{\imageof{\hypercore}}\}}{\shortcatvariables}$ coincide on all inputs and are thus equal.
\end{proof}

\begin{figure}[h]
\begin{center}
	\begin{tikzpicture}[scale=0.35] % , baseline = -3.5pt




\begin{scope}[shift={(-15,0)}]

\drawatomcore{-6}{-8}{$\chainingfunction\circ\hypercore$}


	\begin{scope}[shift={(-6,-12)}]
		\draw[] (0,1)--(0,-1) node[midway,left] {\tiny $\catvariableof{0}$}; 
		\draw[] (1.5,1)--(1.5,-1) node[midway,left] {\tiny $\catvariableof{1}$}; 
		\node[anchor=center] (text) at (3,0) {$\cdots$};
		\draw[] (4,1)--(4,-1) node[midway,right] {\tiny $\catvariableof{\atomorder\shortminus1}$}; 
	\end{scope}

\end{scope}



%\node[anchor=center] (text) at (-14.25,-10) {${=}$};



%\begin{scope}[shift={(-12.5,0)}]
%
%\node[anchor=center] (text) at (1,-7.25) {\small $\chainingfunction$};
%\draw (5.5,-7.25) ellipse (6 and 4.5);
%
%
%\drawatomcore{3.5}{-8}{$\rencodingof{\exformula}$}
%\drawatomindices{3.5}{-12}	
%\draw[] (5.5,-9)--(5.5,-7) node[midway,right] {\tiny $\randomxof{\hypercore}$};
%
%
%\draw (4.75,-3.5) rectangle (6.25,-7);
%\node[anchor=center] (text) at (5.5,-5.25) {$\begin{bmatrix} 
%0 \\
%1
%\end{bmatrix}$};
%
%\end{scope}


\begin{scope}[shift={(-12.5,0)}]

\node[anchor=center] (text) at (-0.5,-10) {${=}$};

%\node[anchor=center] (text) at (0.5,-8) {$\mathrm{log}$};

\drawatomcore{3.5}{-8}{$\rencodingof{\hypercore}$}
\drawatomindices{3.5}{-12}	
\draw (5.5,-9)--(5.5,-6) node[midway,right] {\tiny $\randomxof{\hypercore}$};
\draw[->] (5.5,-9) -- (5.5,-7.5);

\draw (3.25,-4) rectangle (7.5,-6);
\node[anchor=center] (text) at (5.5,-5) {$\restrictionofto{\chainingfunction}{\imageof{\hypercore}}$
%\begin{bmatrix} 
%	\chainingfunction(0) \\
%	\chainingfunction(1)
%\end{bmatrix}
};

\end{scope}

\end{tikzpicture}
\end{center}
\caption{Representation of the composition of a tensor $\hypercore$ with a real function $\chainingfunction$.}
\label{fig:tensorFunctionComposition} 
\end{figure}


\begin{corollary}\label{cor:rhoToNormal}
	For any tensor $\hypercore$ we have
		\[ \hypercore = \contractionof{\rencodingof{\hypercore},\restrictionofto{\mathrm{Id}}{\imageof{\hypercore}}}{\shortcatvariables} \, . \]
\end{corollary}
\begin{proof}
	Directly by using $\chainingfunction=\mathrm{Id}$.
\end{proof}


\begin{corollary}\label{cor:onesHead}
	For any tensor $\hypercore$ we have
		\[ \ones = \contractionof{\rencodingof{\hypercore}}{\shortcatvariables} \, . \]
\end{corollary}
\begin{proof}
	Directly by using $\chainingfunction=\ones$.
\end{proof}

% Replacement of Slicing Theorem
\begin{corollary}\label{cor:directedTrafo}
	Let $\basisslices$ be a directed and binary tensor with incoming variables being $\shortcatvariables$, and $\gentensor$ a tensor, which variables are the outgoing variables of $\basisslices$.
	Let further $\coordinatetrafo:\rr\rightarrow\rr$ be any real function.
	Then
		\[ \coordinatetrafo \circ \contractionof{\{\basisslices,\gentensor\}}{\shortcatvariables} = \contractionof{\{\basisslices,\coordinatetrafo\circ \gentensor\}}{\shortcatvariables} \, . \]
\end{corollary}
\begin{proof}
	Since $\basisslices$ is a directed and binary tensor, we find a map
		\[ \exfunction : \facstates \rightarrow \secfacstates \]
	such that $\basisslices=\rencodingof{\exfunction}$ and a map $V$ such that $\gentensor=\restrictionofto{V}{\imageof{\exfunction}}$.
	Then Theorem~\ref{the:tensorFunctionComposition} implies that 
		\[ \contractionof{\{\basisslices,\gentensor\}}{\shortcatvariables} = V \circ \exfunction \, . \]
	It follows that 
	\begin{align*}
		\coordinatetrafo \circ \contractionof{\{\basisslices,\gentensor\}}{\shortcatvariables} = \coordinatetrafo \circ V \circ \exfunction 
	\end{align*}
	and by another application of Theorem~\ref{the:tensorFunctionComposition} that
	\begin{align*}
		\coordinatetrafo \circ V \circ \exfunction
		& = \contractionof{\rencodingof{\exfunction}, \restrictionofto{\coordinatetrafo \circ V}{\imageof{\exfunction}}}{\shortcatvariables} \\
		& = \contractionof{\{\basisslices,\coordinatetrafo\circ\gentensor\}}{\shortcatvariables} \, . 
	\end{align*}
	The claim follows as a combination of both equations.
\end{proof}



\begin{example}[Shannon entropy of empirical distribution]
%\begin{theorem}
	The Shannon entropy of an empirical distribution can be efficiently computed by contraction of the datatensor with itself along the atom indices, then applying a coordinatewise $\ln$ and averaging.
%\end{theorem}

%\begin{proof}
	This follows from commutations of contraction and coordinatewise contraction (see Corollary~\ref{cor:directedTrafo}), using that the datacores are directed. 
	%Theorem~\ref{the:CoordinateTransform}, since the datatensor is a slicewise basis tensor.
	To be more precise, let $\secdatamap$ be a copy of $\datamap$ with identical image space and copied domain space.
	Then $\secdatacoreof{\atomenumerator}$ are tensor cores with identical outgoing legs to $\datacoreof{\atomenumerator}$, but different incoming legs.
	We have that
	\begin{align*}
		\sentropyof{\empdistribution} 
		& = \contractionof{\{\datacoreof{\atomenumerator}\, : \, \atomenumeratorin\} \cup \{\frac{1}{\datanum}\ones \} \cup \{-\ln \contractionof{\{\secdatacoreof{\atomenumerator}\, : \, \atomenumeratorin\} \cup \{\frac{1}{\datanum}\ones \}}{\shortcatvariables} \} }{\varnothing} \\
		& = \contractionof{
		 \left\{\frac{1}{\datanum}\ones, -\ln \contractionof{\{\datacoreof{\atomenumerator},\secdatacoreof{\atomenumerator}\, : \, \atomenumeratorin\} \cup \{\frac{1}{\datanum}\ones \}}{\shortcatvariables} \right\}
		}{\varnothing}
	\end{align*}
	where in the second equation we used Corollary~\ref{cor:directedTrafo}.
%\end{proof}
\end{example}

\subsubsection{Decomposition in case of structured images}

\red{Here the introduction of multiple head variables, i.e. when}
	\[ \outset = \bigtimes_{\catenumeratorin} \arbsetof{\catenumerator}\]
\red{This is the case for the empirical distributions!}


When the image admits a cartesian representation, the relational encoding can be represented by a contraction of relational encodings to each image coordinate.

\begin{theorem}\label{the:functionDecompositionBasisCP}
	Let $\exfunction$ be a function between factored systems
		\[ \exfunction : [\catdim] \rightarrow  \facstates \]
	and denote by
		\[ \exfunction^\atomenumerator : [\catdim] \rightarrow [\catdimof{\atomenumerator}]\]
	the restrictions of $\exfunction$ to axes of $\facstates$.
	We assign the variable $\catvariable$ to the factored system in the domain system of $\exfunction$ and the variables $\catvariableof{\atomenumerator}$ for $\atomenumeratorin$ to the image system of $\exfunction$.
	
	We then have
	\begin{align*}
		\rencodingofat{\exfunction}{\catvariable,\shortcatvariables}  
		= \contractionof{
		\{\rencodingofat{\exfunction^{\atomenumerator}}{\catvariable,\catvariableof{\atomenumerator}} : \atomenumeratorin \} 
		}{\catvariable,\shortcatvariables} \, . 
	\end{align*}
%	and 
%		\[ \baspluscprankof{\rencodingof{\exfunction}} \leq \catdim \, . \]
\end{theorem}
\begin{proof}
	
	We have 
	\begin{align*}
		\rencodingofat{\exfunction}{\catvariable,\shortcatvariables}  
		= \contractionof{
		\{\rencodingofat{\exfunction^{\atomenumerator}}{\catvariable,\catvariableof{\atomenumerator}} : \atomenumeratorin \} 
		}{\catvariable,\shortcatvariables}
	\end{align*}
	since for any $\catindex\in[\catdim]$ 
	\begin{align*}
		\rencodingofat{\exfunction}{\catvariable=\catindex,\shortcatvariables}  
		= \bigotimes_{\atomenumeratorin} \rencodingofat{\exfunction^{\atomenumerator}}{\catvariable=\catindex,\catvariableof{\atomenumerator}}
		= \contractionof{
		\{\rencodingofat{\exfunction^{\atomenumerator}}{\catvariable,\catvariableof{\atomenumerator}} : \atomenumeratorin \} 
		}{\catvariable=\catindex,\shortcatvariables} \, . 
	\end{align*}
	
	To get a representation in the basis CP format, we rename the variable $\catvariable$ in the cores $\rencodingofat{\exfunction^{\atomenumerator}}{\catvariable,\catvariableof{\atomenumerator}}$ by $\decvariable$ and observe that for a trivial scalar core $\onesat{\decvariable}$ 
	\begin{align*}
		\contractionof{
		\{\rencodingofat{\exfunction^{\atomenumerator}}{\catvariable,\catvariableof{\atomenumerator}} : \atomenumeratorin \} 
		}{\catvariable,\shortcatvariables} 
		=
		\contractionof{ 
		\{\onesat{\decvariable}\} \cup \{\rencodingofat{\exfunction^{\atomenumerator}}{\decvariable,\catvariableof{\atomenumerator}} : \atomenumeratorin \} \cup \{\identityat{\catvariable,\decvariable}\}
		}{\catvariable,\shortcatvariables}  \, . 
	\end{align*}
	
%	It suffices to show for any $\catindex\in[\catdim]$ we have
%		\[ \contractionof{\{\ftensorof{\exfunction},\onehotmapof{\catindex}\}}{\shortcatvariables} = 
%		 	\contractionof{\{\ftensorof{\exfunction^\atomenumerator} \, : \, \atomenumeratorin\}\cup\{\onehotmapof{\catindex}\}}{\shortcatvariables} \, . 
%		  \]
%	But this holds, since
%	\begin{align*} 
%		\contractionof{\{\ftensorof{\exfunction},\onehotmapof{\catindex}\}}{\shortcatvariables} 
%			& = \bigotimes_{\atomenumeratorin} \onehotmapof{\exfunction^{\atomenumerator}(\catindex)} \\
%			& = \bigotimes_{\atomenumeratorin} \contractionof{\{\ftensorof{\exfunction^\atomenumerator},\onehotmapof{\catindex}\}}{\randomxof{\atomenumerator}} \\
%			& = \contractionof{\{\ftensorof{\exfunction^\atomenumerator} \, : \, \atomenumeratorin \}\cup\{\onehotmapof{\catindex}\}}{\shortcatvariables} \, .
%	\end{align*}
\end{proof}





\subsection{Effective Calculus}\label{sec:effectiveCalculus} % -> Part III

% Calculus against the direction
For specific functions, slices of the relational encodings with respect to head variables are basis vectors.
In that case, we can perform basis calculus in the inverse direction than suggested by the directions of the tensors.
We examplify this situation in the following theorem for relational encodings of logical conjunctions and negations.

\begin{figure}
\begin{center}
	\input{./tikz_pics/skeleton_elements.tex}
\end{center}
\caption{Decomposition schemes by effecitve calculus. a) Conjunction, b) Negations.}\label{fig:ConNegDecomposition}
\end{figure}

\begin{theorem}\label{the:effectiveConjunction}
	For any formulas $\exformula,\secexformula$ we have
	\begin{align*}
		\sbcontractionof{
			\rencodingofat{\land}{\catvariableof{\exformula\land\secexformula},\catvariableof{\exformula},\catvariableof{\secexformula}},\onehotmapofat{1}{\catvariableof{\exformula\land\secexformula}}
		}{\catvariableof{\exformula},\catvariableof{\secexformula}}
		= \onehotmapofat{1}{\catvariableof{\exformula}} \otimes \onehotmapofat{1}{\catvariableof{\secexformula}} \, . 
	\end{align*}
	In particular, it holds that (see Figure~\ref{fig:ConNegDecomposition}a)
	\begin{align*}
		(\exformula\land\secexformula)[\shortcatvariables] = \sbcontractionof{\exformula,\secexformula}{\shortcatvariables} \, . 
	\end{align*}
\end{theorem}
\begin{proof}
	We decompose 
	\begin{align*}
		\rencodingofat{\land}{\catvariableof{\exformula\land\secexformula},\catvariableof{\exformula},\catvariableof{\secexformula}}
		= \onehotmapofat{1}{\catvariableof{\exformula\land\secexformula}} \otimes \onehotmapofat{1}{\catvariableof{\exformula}} \otimes \onehotmapofat{1}{\catvariableof{\secexformula}}
		+ \onehotmapofat{0}{\catvariableof{\exformula\land\secexformula}} \left( \onesat{\catvariableof{\exformula},\onesat{\catvariableof{\secexformula}}} -  \onehotmapofat{1}{\catvariableof{\exformula}} \otimes \onehotmapofat{1}{\catvariableof{\secexformula}} \right) 
	\end{align*}
	and get the first claim as
	\begin{align*}
		\sbcontractionof{
			\rencodingofat{\land}{\catvariableof{\exformula\land\secexformula},\catvariableof{\exformula},\catvariableof{\secexformula}},\onehotmapofat{1}{\catvariableof{\exformula\land\secexformula}}
		}{\catvariableof{\exformula},\catvariableof{\secexformula}}
		& = \sbcontractionof{
			\onehotmapofat{1}{\catvariableof{\exformula\land\secexformula}} \otimes \onehotmapofat{1}{\catvariableof{\exformula}} \otimes \onehotmapofat{1}{\catvariableof{\secexformula}},\onehotmapofat{1}{\catvariableof{\exformula\land\secexformula}}
		}{\catvariableof{\exformula},\catvariableof{\secexformula}} \\
		& = \onehotmapofat{1}{\catvariableof{\exformula}} \otimes \onehotmapofat{1}{\catvariableof{\secexformula}} \, . 
	\end{align*}
	To show the second claim we use
	\begin{align*}
		(\exformula\land\secexformula)[\shortcatvariables] 
		&= \sbcontractionof{
			\rencodingofat{\exformula}{\catvariableof{\exformula},\shortcatvariables},
			\rencodingofat{\secexformula}{\catvariableof{\secexformula},\shortcatvariables},
			\rencodingofat{\land}{\catvariableof{\exformula\land\secexformula},\catvariableof{\exformula},\catvariableof{\secexformula}},
			\onehotmapofat{1}{\catvariableof{\exformula\land\secexformula}}
			}{\shortcatvariables} \\
		&  = \sbcontractionof{
			\rencodingofat{\exformula}{\catvariableof{\exformula},\shortcatvariables},
			\rencodingofat{\secexformula}{\catvariableof{\secexformula},\shortcatvariables},
			(\onehotmapofat{1}{\catvariableof{\exformula}}\otimes \onehotmapofat{1}{\catvariableof{\secexformula}})
			%\rencodingofat{\land}{\catvariableof{\exformula},\catvariableof{\secexformula},\catvariableof{\exformula\land\secexformula}}
			}{\shortcatvariables} \\
		&= \sbcontractionof{\exformula,\secexformula}{\shortcatvariables} \, . 
	\end{align*}
\end{proof}

A similar decomposition holds for negations, as we show next.

\begin{theorem}
	For any formula $\exformula$ we have
	\begin{align*}
		\sbcontractionof{
			\rencodingofat{\lnot}{\catvariableof{\exformula},\catvariableof{\lnot\exformula}},\onehotmapofat{1}{\catvariableof{\lnot\exformula}}
		}{\catvariableof{\exformula}}
		= \onehotmapofat{0}{\catvariableof{\exformula}} =  \onesat{\catvariableof{\exformula}} - \onehotmapofat{1}{\catvariableof{\exformula}} \, . 
	\end{align*}
	and
	\begin{align*}
		\sbcontractionof{
			\rencodingofat{\lnot}{\catvariableof{\exformula},\catvariableof{\lnot\exformula}},\onehotmapofat{0}{\catvariableof{\lnot\exformula}}
		}{\catvariableof{\exformula}}
		= \onehotmapofat{1}{\catvariableof{\exformula}} \, . 
	\end{align*}
	In particular, it holds that (see Figure~\ref{fig:ConNegDecomposition}b)
	\begin{align*}
		(\lnot\exformula)[\shortcatvariables] = \onesat{\shortcatvariables} - \formulaat{\shortcatvariables}  \, . 
	\end{align*}
\end{theorem}
\begin{proof}
	Using that for two dimensional variables we have $\onesat{\catvariable}=\onehotmapofat{0}{\catvariable}+\onehotmapofat{1}{\catvariable}\, .$
\end{proof}

% Usage
These theorems provide a mean to represent logical formulas by sums of one-hot encodings.
Since any propositional formula can be represented by compositions of negations and conjunctions, they are universal.
We further notice, that the resulting decomposition is a basis+ CP format, as further discussed in Chapter~\ref{cha:sparseTC}.
In Figure~\ref{fig:DecompositionExample} we provide an example of this decomposition.


\begin{figure}
\begin{center}
	\input{./tikz_pics/skeleton_example.tex}
\end{center}
\caption{
	Example of a decomposition by effective calculus of a formula $\exformula(\catvariableof{1},\catvariableof{2}) = \textcolor{blue}{\lnot} \secexformula^{(1)}(\catvariableof{1},\catvariableof{2}) \textcolor{red}{\land}  \secexformula^{(2)}(\catvariableof{1},\catvariableof{2})$ into a sum of contractions.}
	\label{fig:DecompositionExample}
\end{figure}




\subsection{Applications in Machine Learning}

The neural paradigm of Machine Learning describes the relevance of sparse function to be effective models in the sense of learning and approximation.

% Neural Paradigm by Tensor Network Decompositions
Our model of the neural paradigm are tensor network decompositions, seen as decomposition of functions into smaller functions, which take each other as input.
Summations along input axis are avoided, when having directed and binary tensor networks with basis calculus interpretation.

% Basis Calculus
We have already observed in Theorem~\ref{the:basisCalculus}, that the value of discrete maps can be calculated by contractions of the directed binary relation encodings.
This has been framed as Basis Calculus.
What is more, tensor network decompositions into directed binary tensors correspond with representation of functions as compositions of smaller functions.
We can understand each composition as marking a neuron in an architecture and thus have established a neural perspective on binary directed tensor networks.
 % Before sparse tensor calculus!

\input{PartIII/local_contractions.tex}

% CP sparsity
\section{Sparse Tensor Representations}\label{cha:sparseTC}

We in this chapter investigate, which sparsity notations enable tensors to be representable as contractions of tensor networks.


\subsection{CP Formats}

The CP Decomposition is one way to generalize the ranks of matrices to tensors.
It is oriented on the Singular Value Decomposition of matrices, providing a representation of the matrix as a weighed sum of the tensor product of singular vectors.
Given a tensor of higher order, each such tensor product is over multiple vectors, 

\begin{definition}\label{def:cpFormats}
	A CP Decomposition of rank $\decdim$ of a tensor $\hypercore\in\facspace$ is a collections of tensors $\scalarcoreat{\decvariable}$ and $\legcoreofat{\atomenumerator}{\decvariable,\catvariableof{\atomenumerator}}$ for $\atomenumeratorin$, where $\decvariable$ takes values in $[\decdim]$, such that
		\[  \hypercoreat{\shortcatvariables}
		= \contractionof{
		\{\scalarcoreat{\decvariable}\} \cup \{ \legcoreofat{\atomenumerator}{\decvariable,\catvariableof{\atomenumerator}} \, : \, \atomenumeratorin \}
		}{\shortcatvariables} \, . 
		\]
%	where for each $\decindexin$ and $\atomenumeratorin$ we have $\scalarcoreat{\decindex} \in \rr$ and $\legcoreof{\atomenumerator,\decindex}\in\rr^{\catindexof{\atomenumerator}}$.
	We say that the CP Decomposition is
	\begin{itemize}
		\item directed, when for each $\atomenumerator$ the core $\legcoreof{\atomenumerator}$ is directed with $\decvariable$ incoming and $\catvariableof{\atomenumerator}$ outgoing.
		\item binary, when for each $\atomenumerator$ the core $\legcoreof{\atomenumerator}$ is binary.
		\item basis, where we demand both properties, that is for each $\atomenumeratorin$ and $\decindexin$ 
			\[ \legcoreofat{\atomenumerator}{\inddecvar,\catvariableof{\atomenumerator}}\in \{ \onehotmapofat{[\catindexof{\atomenumerator}]}{\catvariableof{\atomenumerator}} \catindexof{\atomenumerator}\in[\catdimof{\atomenumerator}] \}\, . \]
		\item basis+, when for each $\atomenumeratorin$ and $\decindexin$  %$\legcoreof{\atomenumerator,\decindex}\in\onehotmapof{[\catindexof{\atomenumerator}]}$ or $\legcoreof{\atomenumerator,\decindex}=\ones$.
			\[ \legcoreofat{\atomenumerator}{\inddecvar,\catvariableof{\atomenumerator}}\in \{ \onehotmapofat{[\catindexof{\atomenumerator}]}{\catvariableof{\atomenumerator}} \catindexof{\atomenumerator}\in[\catdimof{\atomenumerator}] \} \cup \{\onesat{\catvariableof{\atomenumerator}}\}\, . \]
	\end{itemize}
	We denote by $\cprankof{\hypercore}$, respectively $\bincprankof{\hypercore}$, $\bascprankof{\hypercore}$ and $\baspluscprankof{\hypercore}$ the minimal cardinality such that $\hypercore$ has a CP Decomposition with directed cores, respectively binary cores, basis cores and basis+ cores.
\end{definition}

% Sum of elementary tensors
We have by definition
	\[ \hypercoreat{\shortcatvariables} = \sum_{\decindexin} \scalarcoreat{\inddecvar} \left( \bigotimes_{\atomenumeratorin} \legcoreofat{\atomenumerator}{\inddecvar,\catvariableof{\atomenumerator}} \right) \, . \]
The right side can be seen as an alternative definition of CP Decompositions by summations of elementary tensors.


\begin{figure}[h]
	\begin{center}
		\input{PartIII/tikz_pics/sparse_tensor_calculus/cp_decomposition.tex}
	\end{center}
	\caption{Tensor Network diagram of a generic CP decomposition (see Definition~\ref{def:cpFormats})}
\end{figure}

We introduce different notions of sparsities based on CP Decomposition with different properties of their leg cores.

\subsubsection{Directed Leg Cores}

This is the canonical CP Decomposition, where the vectors $\legcoreof{\atomenumerator,\decindex}$ are interpreted as generalized singular vectors.
Any CP decomposition can be transformed into a directed CP decomposition without enlarging the index set $\indexset$, simply by diving the vectors by their norms and multiplying it to $\scalarcoreat{\inddecvar}$.

% Directionality
We then have a partially directed Tensor Network representing the decomposed tensor.
The only undirected core is $\scalarcore$, since we do not demand it to be normed.
In many applications applications, however, also the $\scalarcore$ is directed with a single outgoing leg (see for example the empirical distributions as discussed in Section~\ref{sec:empDistribution}).
In that case, also the decomposed tensor is directed with outgoing legs.



\subsubsection{Basis Leg Cores}\label{sec:basisCP}

% From FOL Chapter: Bayesian Network interpretation of Basis CP
%	The basis CP can further be understood as a Bayesian network, where we understand $\dataindex$ as condition and each decomposition core as a conditional probability distribution.
%	We notice that in this interpretation the direction of the dependency is inversed compared with previous representation of grounding tensors in Figure~\ref{fig:groundingCP}. 


Directed and binary leg cores have incoming slices being basis vectors, we thus call them basis CP Decomposition.
This allows the interpretation of the directed and binary CP decomposition in terms of mapping to nonzero coordinates.
We start by defining the number of nonzero coordinates of tensors by the $\ell_0$-norm.

\begin{definition}
	The $\ell_0$-norm counts the nonzero coordinates of a tensor by
		\[ \sparsityof{\hypercore} = \#\big\{ \catindices \, : \, \hypercore_{\catindices }\neq 0 \big\} \, . \]
\end{definition}

The $\ell_0$-norm is not a proper norm itself, but the limit of $\ell_p$-norms (where $p \rightarrow 0$) of the flattened tensor (which are norms for $p\geq1$).

% Interpretation
The $\ell_0$ norm is the number of nonzero coordinates. 
We understand the leg cores as the relational encoding of functions mapping to the slices of these coordinates given an enumeration.
This is consistent with the previous analysis of Chapter~\ref{cha:directedTC}, where we characterized binary and directed cores by the encoding of associated functions.
Based on this idea, we can proof, that any tensor has a directed and binary CP decomposition with rand $\sparsityof{\hypercore}$.


\begin{theorem}\label{the:sparseBasisCP}
	For any tensor $\hypercore$ we have
		\[ \bascprankof{\hypercore} = \sparsityof{\hypercore} \, .  \]	
\end{theorem}
\begin{proof}
	We find a map 
		\[ \exfunction : [\sparsityof{\hypercore}] \rightarrow  \facstates \, , \] 
	which image is the set of nonzero coordinates of $\hypercore$.
	Denoting its image coordinate maps by $\exfunction^{\atomenumerator}$ we have
		\[ \hypercore = \sum_{\dataindexin} \scalarcoreof{\exfunction(\dataindex)} \left( \bigotimes_{\atomenumeratorin} \onehotmapof{\exfunction^\atomenumerator(\dataindex)} \right) \, . \]
	This is a basis CP Decomposition with rank $\sparsityof{\hypercore}$.
	Conversely, any basis CP Decomposition of $\hypercore$ with dimension $r$ would have at most $r$ coordinates different from zero and thus $\sparsityof{\hypercore}\leq r$.
	Thus, there cannot be a CP Decomposition with a dimension $r\leq\sparsityof{\hypercore}$.
\end{proof}

%
The next theorem relates the basis CP decomposition with encodings of $\atomorder$-ary relations (see Definition~\ref{def:daryRelation}).

\begin{theorem}
	If any only if $\hypercore\in\facspace$ has a basis decomposition with slices $\{\catindex_{[\atomorder]}^{\decindex} \, : \, \decindexin \}$ and trivial cores, it coincides with the encoding of the $\atomorder$-ary relation 
		\[ \exrelation = \{\catindex_{[\atomorder]}^{\decindex} \, : \, \decindexin \} \, . \]
%	To each basis CP decompositions with pairwise different slices and trivial scalar cores we find a $d$-ary relation, such that 
\end{theorem}


If in addition $\catdimof{\atomenumerator}=2$, we can interpret basis CP decompositions as propositional formulas.

% Knowledge Bases
\begin{theorem}
	If $\hypercore\in\atomspace$ has a basis decomposition with slices $\{\catindex_{[\atomorder]}^{\decindex} \, : \, \decindexin \}$ and trivial cores, it coincides with the propositional formula
		\[ \formulaat{\shortcatvariables} = 
		\bigvee_{\decindexin} \termof{\catindex_{[\atomorder]}^{\decindex}} \, . \]
\end{theorem}
\begin{proof}
	This is a generalization of Theorem~\ref{the:maximalClausesRepresentation}, which follows from Theorem~\ref{the:tensorToMaxMinTerms}.
\end{proof}


% Storage
The storage demand of any CP decomposition is at most linear in the dimension and the sum of its leg dimension.
When we have a basis CP decomposition, this demand can be further improved.
The basis vectors can be stored by its preimage of the one hot encoding $\onehotmapof{\cdot}$, that is the number of the basis vector in $[\catdim]$.
This reduces the storage demand of each basis vector to the logarithms of the space dimension without the need of storing the full vector.

% Matrix Representation
More precisely, we can store the CP Decomposition by the matrix
	\[ \matrixat{\decvariable,\selvariable} \in \rr^{\datanum \times (\atomorder+1)} \]
defined for $\atomenumeratorin$
	\[ \matrixat{\inddecvar,\selvariable=\atomenumerator} 
	= \invonehotmapof{\legcoreofat{\atomenumerator}{\inddecvar,\catvariableof{\atomenumerator}}}\]
and
	\[ \matrixat{\inddecvar,\selvariable=\atomorder}  
	= \scalarcoreat{\inddecvar} \, . \]
	
This is a typical tabular format to store relational databases.

\subsubsection{Basis+ Leg Cores}

The minimal rank of CP Decompositon is closely related to polynomial sparsity of the map $\hypercore$, which we will define next.

\begin{definition}\label{def:polynomialSparsity}
	A monomial decomposition of a tensor $\hypercore\in\facspace$ 
	%consists of index sets $\indexsetof{\variableset}$ to each $\variableset\subset[\atomorder]$ and values $\scalarcoreat{\variableset, \catindexof{\variableset}}\in\rr$ for each $\catindexof{\variableset}\in \indexsetof{\variableset}$ such that
	is a set $\sliceset$ of tuples $\slicetupleof{}$ where $\slicescalar\in\rr, \variableset\subset[\atomorder]$ and $\catindexof{\variableset}\in\bigtimes_{\atomenumerator\in\variableset} [\catdimof{\atomenumerator}]$ such that
	\begin{align}\label{eq:decIntoMonomials}
		\hypercoreat{\shortcatvariables} = \sum_{\slicetupleof{}\in\sliceset} \slicescalar \cdot \contractionof{\onehotmapof{\catindexof{\variableset}}}{\shortcatvariables} \, .
	\end{align}
%	\begin{align}
%		\hypercore 
%			= \sum_{\variableset\subset[\atomorder]} \sum_{\catindexof{\variableset}\in \indexsetof{\variableset}}  
%			\scalarcoreat{\variableset, \catindexof{\variableset}} \cdot \left( \onehotmapof{\catindexof{\variableset}} \otimes \onesof{[\atomorder]/\variableset} \right)   \, . 
%	\end{align}
	For any tensor $\hypercore\in\facspace$ we define its polynomial sparsity as
	\begin{align*}
		\slicesparsityof{\hypercore} =
		 \min \left\{ \cardof{\sliceset} \, : \, 
		 	\hypercoreat{\shortcatvariables} = \sum_{\slicetupleof{}\in\sliceset} \slicescalar \cdot \contractionof{\onehotmapof{\catindexof{\variableset}}}{\shortcatvariables} 
		 \right\}
	\end{align*}
%	\begin{align}
%		\slicesparsityof{\hypercore} =
%		 \min \left\{ \# \left( \bigcup_{\variableset\subset[\atomorder]} \indexsetof{\variableset} \right) \, : \, \exists \scalarcoreat{\variableset, \catindexof{\variableset}} \in \rr : 
%		 \hypercore 
%		 	= \sum_{\variableset\subset[\atomorder]} \sum_{\catindexof{\variableset}\in \indexsetof{\variableset}}  
%			\scalarcoreat{\variableset, \catindexof{\variableset}} \cdot \left( \onehotmapof{\catindexof{\variableset}} \otimes \onesof{[\atomorder]/\variableset} \right)   
%		 \right\}
%	\end{align}
\end{definition}


% Explanation of monomials
We refer to the terms in a decomposition \eqref{eq:decIntoMonomials} in Definition~\ref{def:polynomialSparsity} as monomials of binary variables, which are enumerated by pairs $(\atomenumerator,\catindexof{\atomenumerator})$ and indicate whether the variable $\catvariableof{\atomenumerator}$ is in state $\catindexof{\atomenumerator}\in[\catdimof{\atomenumerator}]$.
Such indicators are represented by the one-hot encodings
	\[ \onehotmapofat{\catindexof{\atomenumerator}}{\catvariableof{\atomenumerator}} \, . \]
The monomial of multiple such binary variables indicated, whether all variables labelled by a set $\variableset$ are in the state $\catvariableof{\variableset}$, which is represented by
	\[ \onehotmapofat{\catindexof{\variableset}}{\catvariableof{\variableset}} = \bigotimes_{\atomenumerator\in\variableset} \onehotmapofat{\catindexof{\atomenumerator}}{\catvariableof{\atomenumerator}}  \, . \]
The states of the variables labeled by $\atomenumerator\in[\atomorder]/\variableset$ are not specified in the monomial and the monomial is trivially extended to
	\[ \contractionof{\onehotmapof{\catindexof{\variableset}}}{\shortcatvariables}  = \onehotmapofat{\catindexof{\variableset}}{\catvariableof{\variableset}} \otimes \onesat{\catvariableof{[\atomorder]/\variableset}} \, .   \]

%% Interpretation as Monomial Sparsity
%Each tensor $\lambda \cdot \contractionof{\onehotmapof{\catindexof{\variableset}}}{\shortcatvariables}$ is a by $\lambda\in\rr$ weighted monomial of the variables (or their negations) in $\variableset$.
%The polynomial sparsity is thus the minimal number of monomials that sum up to the function $\hypercore$.
%Given any monomial decomposition of a tensor, we can alternatively write
%	\begin{align*}
%		\hypercore = \sum_{\variableset\subset[\atomorder]} \sum_{\catindexof{\variableset}\in \indexsetof{\variableset}}  
%			\scalarcoreat{\variableset, \catindexof{\variableset}} \left( \prod_{\atomenumeratorin} \catvariableof{\atomenumerator} == (\catindexof{\variableset})_{\atomenumerator} \right) \, . 
%	\end{align*}
%where by $ \catvariableof{\atomenumerator} == (\catindexof{\variableset})_{\atomenumerator}$ we denote the atomic variables, indicating whether the variable $\catvariableof{\atomenumerator}$ is in state $ (\catindexof{\variableset})_{\atomenumerator}$.
%
%% Failing to be directed.
%Note, that the leg cores fail to be directed, when for some $\variableset\neq[\atomorder]$ the set $\indexsetof{\variableset}$ is not empty.


\begin{theorem}
	For any tensor $\hypercore\in\facspace$ we have
		\[ \slicesparsityof{\hypercore} = \baspluscprankof{\hypercore} \, . \]
	When $\catindexof{\atomenumerator}=2$ for all $\atomenumeratorin$, we also have
		\[ \bincprankof{\hypercore} = \slicesparsityof{\hypercore}  \, . \]
\end{theorem}
\begin{proof}
	To proof the first claim, we construct a basis+ CP decomposition given a monomial decomposition and vice versa.
	Let there be a tensor $\hypercoreat{\shortcatvariables}$ with a monomial decomposition by $\sliceset$ with $\cardof{\sliceset}=m$ and let us enumerate the elements in $\sliceset$ by $\slicetupleof{\decindex}$ for $\decindexin$.
	 We define for each $\atomenumeratorin$ the tensors
	 \begin{align*}
		\legcoreofat{\atomenumerator}{\decvariable,\catvariableof{\atomenumerator}}
		 = \left( \sum_{\decindexin \, : \, \atomenumerator\in\variableset} \onehotmapofat{\decindex}{\decvariable} \otimes \onehotmapofat{\catindexof{\atomenumerator}^{\decindex}}{\catvariableof{\atomenumerator}} \right)
		 + \left(\sum_{\decindexin \, : \, \atomenumerator\notin\variableset} \onehotmapofat{\decindex}{\decvariable} \otimes \onesat{\catvariableof{\atomenumerator}} \right)
	\end{align*}
	and 
	\begin{align*}
		\scalarcoreat{\decvariable} = \sum_{\decindexin} \slicescalar^{\decindex} \cdot \onehotmapofat{\decindex}{\decvariable}
	\end{align*}
	 and notice that	 
	\begin{align*}
		\hypercoreat{\shortcatvariables} 
		& = \sum_{\decindexin} \slicescalar^{\decindex} \cdot \contractionof{\onehotmapof{\catindexof{\variableset}^{\decindex}}}{\shortcatvariables} \\
		& = \sum_{\decindexin} \left(  \scalarcoreat{\inddecvar} \cdot \bigotimes_{\atomenumeratorin} \legcoreofat{\atomenumerator}{\inddecvar, \catvariableof{\atomenumerator}} \right) \\
		& = \contractionof{
		\{\scalarcoreat{\decvariable}\} \cup \{\legcoreofat{\atomenumerator}{\decvariable,\catvariableof{\atomenumerator}} \, : \, \atomenumeratorin \}
		}{\shortcatvariables} \, . 
	\end{align*}
	By construction this is a basis+ CP decomposition with rank $\decdim$.
	Since any monomial decomposition can be transformed into a basis+ CP decomposition with same rank we have
	\begin{align*}
		\slicesparsityof{\hypercore} \geq \baspluscprankof{\hypercore} \, . 
	\end{align*}
	
	Let there now be a basis+ CP decomposition we define for each $\decindexin$ 
	\begin{align*}
		\variableset^{\decindex} = \{\atomenumeratorin : \legcoreofat{\atomenumerator}{\inddecvar, \catvariableof{\atomenumerator}} \neq \onesat{\catvariableof{\atomenumerator}} \}
		 \quad \text{and} \quad 
		 \catindexof{\variableset}^{\decindex} = \{\invonehotmapof{\legcoreofat{\atomenumerator}{\inddecvar, \catvariableof{\atomenumerator}} } \, : \atomenumerator\in\variableset\}
	\end{align*}
	where by $\invonehotmapof{\cdot}$ we denote the inverse of the one-hot encoding.
	
	We notice that this is a monomial decomposition of $\hypercoreat{\shortcatvariables}$ to the tuple set
	\begin{align*}
		\sliceset = \{(\scalarcoreat{\inddecvar}, \variableset^{\decindex}, \catindexof{\variableset^{\decindex}}^{\decindex} ) \, : \, \decindexin \} \, . 
	\end{align*}
	It follows from this that
	\begin{align*}
		\slicesparsityof{\hypercore} \leq \baspluscprankof{\hypercore} \, 
	\end{align*}
	and the first claim is shown.
	
	The second claim follows from the observation, that whenever $\catindexof{\atomenumerator}=2$ for all $\atomenumeratorin$ the binary CP decompositions with non-vanishing slices $\legcoreofat{\atomenumerator}{\inddecvar, \catvariableof{\atomenumerator}}$ for $\atomenumeratorin$ and $\decindexin$ are also basis+ CP decompositions and vice versa.
%
%	
%
%	We proof the claim by establishing a one-to-one map between any binary CP decomposition of a a tensor $\hypercore$ and a monomial decomposition of $\hypercore$.
%	% CP Decomposition to monomial decomposition
%	Let there be a binary CP Decomposition of $\hypercore$ with the leg tensors $\{\legcoreof{\atomenumerator}\, :\, \atomenumeratorin\}$ and the scalar core $\scalarcore$.
%	For any index $\decindexin$ and $\atomenumeratorin$ we have
%		\[ \legcoreof{\atomenumerator}_{\decindex} \in \{\onehotmapof{0},\onehotmapof{1},\ones\} \, . \]
%	We define for any $\decindexin$ the sets 
%		\[ \variableset^{\decindex}=\big\{\atomenumerator \, : \, \legcoreof{\atomenumerator}_{\decindex} \in \{\onehotmapof{0},\onehotmapof{1}\}  \big\}\]
%	and an index tuple $\catindexof{\variableset}^\decindex \in \bigotimes_{\atomenumerator\in\variableset}[2]$ by
%	\begin{align*}
%		(\catindexof{\variableset}^\decindex)_\atomenumerator = 
%		\begin{cases} 
%			0 & \text{  if  } \legcoreof{\atomenumerator}_{\decindex} = \onehotmapof{0} \\1 & \text{  if  } \legcoreof{\atomenumerator}_{\decindex} = \onehotmapof{1} 
%		\end{cases} \, . 
%	\end{align*}   
%	Then we have by construction that 
%	\begin{align*}
%		\hypercore = \sum_{\decindexin} \scalarcoreat{\decindex} \cdot \left( \onehotmapof{\catindexof{\variableset}^\decindex} \otimes \onesof{[\atomorder]/\variableset^{\decindex}} \right) \, . 
%	\end{align*}
%	When regrouping the sum over the decomposition index by a sum over possible sets $\variableset\subset[\atomorder]$ and a sum over appearing index tuples $\catindexof{\variableset}$, this is a monomial decomposition of $\hypercore$ with
%		\[ \#\left(\indexset\right) = \# \left( \bigcup_{\variableset\subset[\atomorder]} \indexsetof{\variableset} \right) \, . \]
%	 % Monomial decomposition to CP Decomposition
%	 Vise versa we can construct a binary CP Decomposition given any monomial decomposition of $\hypercore$ 
%	 	\[ \hypercore 
%			= \sum_{\variableset\subset[\atomorder]} \sum_{\catindexof{\variableset}\in \indexsetof{\variableset}}  
%			\scalarcoreat{\variableset, \catindexof{\variableset}} \cdot \left( \onehotmapof{\catindexof{\variableset}} \otimes \onesof{[\atomorder]/\variableset} \right)  \, . \]
%	 To this end, we enumerate the set $\bigcup_{\variableset\subset[\atomorder]} \indexsetof{\variableset}$ by an additional index $\decindexin$ and define leg cores by
%	\begin{align*}
%		\legcoreof{\atomenumerator}_{\decindex} = 
%		\begin{cases} 
%			\onehotmapof{0} & \text{  if  }  \atomenumerator \in \variableset^{\decindex} \text{  and  } (\catindexof{\variableset}^\decindex)_\atomenumerator = 0 \\
%			\onehotmapof{1} & \text{  if  } \atomenumerator \in \variableset^{\decindex} \text{  and  } (\catindexof{\variableset}^\decindex)_\atomenumerator = 1 \\
%			\ones & \text{  if   } \atomenumerator \notin \variableset^{\decindex}
%		\end{cases} 
%	\end{align*}   
%	and a scalar core by coordinates
%		\[ \scalarcoreat{\decindex} = \scalarcoreat{\variableset^{\decindex},\catindexof{\variableset}^\decindex} \, . \]
%	Then, the monomial decomposition coincides with a basis CP Decomposition with the same dimension.
%	Thus, both $\bincprankof{\hypercore}$ and $\slicesparsityof{\hypercore}$ are the minima of identical sets and thus identical.
\end{proof}



\begin{example}[Propositional Formulas]
	When all leg dimensions of a binary tensor $\hypercore$ are $2$, we can interpret $\hypercore$ as a logical formula.
	We can use the binary CP decomposition of any tensor $\sechypercore$ with $\nonzeroof{\sechypercore}=\hypercore$ as a CNF of $\hypercore$.
	Finding the sparsest CNF thus amounts to finding the $\sechypercore$ with minimal $\slicesparsityof{\sechypercore}$ such that $\nonzeroof{\sechypercore}=\hypercore$.
\end{example}






\subsection{Representation involving selection architectures}

The set of slice-sparse tensors coincides with the expressivity of specific selection architecture.
We first define a slice selecting tensor and then show its decomposition into a formula selecting neural network.

\begin{definition}
	Given a set of atomic variables $\shortcatvariables$, a slice selecting tensor of maximal cardinality $\selorder$ is the tensor
		\[ \fselectionmapat{\shortcatvariables,\selvariableof{0,0},\ldots,\selvariableof{\selorder-1,0},\selvariableof{0,1},\ldots,\selvariableof{\selorder-1,1}} \]
	with dimensions
		\[ \seldimof{\selenumerator,0} = 2, \seldimof{\selenumerator,1} = \atomorder \]
	and coordinates
	\begin{align*}
		& \fselectionmapat{\shortcatvariables=\shortcatindices,\indexedselvariableof{0,0},\ldots,\indexedselvariableof{\selorder-1,0},\indexedselvariableof{0,1},\ldots,\indexedselvariableof{\selorder-1,1}} \\
		& \quad = \begin{cases}
			1 & \text{if} \quad 
			\forall_{\atomenumerator,\selenumerator} : \big(  \selindexof{\selenumerator,1} = \atomenumerator \land \selindexof{\selenumerator,0} \neq 2 \big) \rightarrow  \selindexof{\selenumerator,0} = \catindexof{\atomenumerator}  \\
			0
		\end{cases} \, . 
	\end{align*}
\end{definition}


\begin{lemma}\label{lem:sliceFromSliceSelector}
	If all input neurons with same selection index are agreeing have the same connective index, the selected formula does not vanish and coincides with a slice to the set
		\[ \sliceset = \{ \atomenumerator \, : \, \exists_{\selenumeratorin}: \selindexof{\selenumerator,1} = \atomenumerator \land \selindexof{\selenumerator,0} \neq 2 \} \]
	and
		\[ \catindexof{\atomenumerator} = \selindexof{\selenumerator,0} \quad \text{if} \quad \selindexof{\selenumerator,1} = \atomenumerator \, . \]
\end{lemma}


\begin{lemma}\label{lem:fsnnRepresentingSliceSelector}
	The slice selection tensor coincides with a formula selecting neural network with neurons
	\begin{itemize}
		\item unary input neuron enumerated by $\selenumerator$, selecting one of the $\shortcatvariables$ with the variable $\selvariableof{\selenumerator,1}$ and selecting a connective in $\{\lnot, \mathrm{Id}, \mathrm{True}\}$ by $\selvariableof{\selenumerator,0}$
		\item $\selorder$-ary output neuron fixed to the $\land$ connective.
	\end{itemize}
\end{lemma}
\begin{proof}
	This can be easily checked on each input coordinate.
\end{proof}








Let us now show, that the expressivity of the slice selecting neural network coincides with the set of tensors with 

\begin{theorem}
	Let $\fselectionmap$ be a slice selecting tensor.
	For any parameter tensor $\canparam$ we have
		\[ \bascprankof{\canparam} \geq \baspluscprankof{\contractionof{\fselectionmap},\canparam} \, . \]
\end{theorem}
\begin{proof}
	\red{Use the above lemma for that on each slice.}
	Show, how a coordinate of $\canparam$ corresponds with a slice determining tuple: $\sliceset$ determined by the selection indices.
\end{proof}


\begin{figure}[h]
	\begin{center}
		\begin{tikzpicture}[thick, scale=0.35] % , baseline = -3.5pt

\drawatomindices{0}{-6}
\draw (-1,1) rectangle (5, -5);
\node[anchor=center] (text) at (2,-2) {$\fselectionmap$};

%\draw[->] (2,-1)--(2,1) node[midway,right] {\tiny ${\atomicformulaof{\parindexof{1}} \land \atomicformulaof{\parindexof{2}}}$}; 

\draw[<-] (5,0.5) -- (7,0.5) node[midway, above] {\tiny $\selvariableof{0,0}$};
\draw[<-] (5,-1)--(7,-1) node[midway,above] {\tiny $\selvariableof{0,1}$}; 
\node[anchor=center] (text) at (6,-1.75) {$\vdots$};
\draw[<-] (5,-3)--(7,-3) node[midway,below] {\tiny $\selvariableof{\sliceorder\shortminus1,0}$}; 
\draw[<-] (5,-4.5)--(7,-4.5) node[midway,below] {\tiny $\selvariableof{\sliceorder\shortminus1,1}$}; 

\draw (7,1) rectangle (9, -5);
\node[anchor=center] (text) at (8,-2) {$\canparam$};


		
\node[anchor=center] (text) at (12,-2) {${=}$};


\begin{scope}[shift={(17,8)}]

	\begin{scope}[shift={(0,-10)}]
	
		\draw (-1,7) rectangle (12, 9);	
		\node[anchor=center] (text) at (5.5,8) {$\land$};
		
		% First leg selector
		\draw[->] (1,5) -- (1,7) node[midway, right] {\tiny $\catvariableof{\lneuron_0}$};	
		\draw (-1,3) rectangle (3, 5);
		\node[anchor=center] (text) at (1,4) {$\rencodingof{\{\notucon,\iducon,\trueucon\}}$};
		
		\draw (3,4) -- (12,4);
		\draw[<-] (12,4) -- (14,4) node[midway, above] {\tiny $\selvariableof{0,0}$};
			
		% SelectorCores
		\draw[->] (1,1) -- (1,3) node[midway, left] {\tiny $\catvariableof{\vselectionsymbol,0}$};
		
		\draw (-1,1) rectangle (3, -1);
		\node[anchor=center] (text) at (1,0) {$\selectorcoreof{0}$};
		\draw (3,0) to[bend left=10]  (12,2.5);
		\draw[<-] (12,2.5) -- (14,2.5) node[midway, above] {\tiny $\selvariableof{0,1}$};
		
		\draw[<-] (0,-1)--(0,-3) node[midway,left] {\tiny $\catvariableof{0}$}; 
		\node[anchor=center] (text) at (1,-2) {$\cdots$};
		\draw[<-] (2,-1)--(2,-3) node[midway,right] {\tiny $\catvariableof{\atomorder\shortminus1}$}; 
		
		\node[anchor=center] (text) at (5.5,0) {$\cdots$};

		% Second leg selector
		\draw (10,5) --(10,7);
		\draw[->] (10,1) -- (10,5) node[midway, right] {\tiny $\catvariableof{\lneuron_{\sliceorder\shortminus 1}}$};	
		\draw (8,-1) rectangle (12, 1);
		\node[anchor=center] (text) at (10,0) {$\rencodingof{\{\notucon,\iducon,\trueucon\}}$};
		\draw[<-] (12,0) -- (14,0) node[midway, below] {\tiny $\selvariableof{\sliceorder\shortminus1,0}$};

%		\draw (9,-1) -- (9,1);
		\draw[->] (10,-3) -- (10,-1) node[midway, left] {\tiny $\catvariableof{\vselectionsymbol,1}$};
		\draw (8,-5) rectangle (12, -3);
		\node[anchor=center] (text) at (10,-4) {$\selectorcoreof{\sliceorder-1}$};
		\draw[<-] (12,-4) -- (14,-4) node[midway, below] {\tiny $\selvariableof{\sliceorder\shortminus1,1}$};
		
		
%		\draw[<-] (9,-1)--(9,-3) node[midway,left] {\tiny $\catvariableof{0}$}; 
%		\node[anchor=center] (text) at (1,-2) {$\cdots$};
%		\draw[<-] (11,-1)--(11,-3) node[midway,right] {\tiny $\catvariableof{\atomorder\shortminus1}$}; 
		%\drawatomindices{7}{-4}	
		
		
		
		% ParameterCores
		\draw (14,5) rectangle (16, -5);
		\node[anchor=center] (text) at (15,0) {$\canparam$};
		
		\node[anchor=center] (text) at (13,1.5) {$\vdots$};
		
		
		\drawvariabledot{7.5}{-7}
		\drawvariabledot{3.5}{-7}

		\draw[] (3.5,-7) to[bend left=25] (0,-3);
		\draw[->] (3.5,-7) to[bend right=10] (9,-5);
		
		\draw[] (7.5,-7) to[bend left=10] (2,-3);
		\draw[->] (7.5,-7) to[bend right=25] (11,-5);


		\draw[->] (3.5,-9)--(3.5,-7) node[midway,left] {\tiny $\catvariableof{0}$}; 
		\node[anchor=center] (text) at (5.5,-8) {$\cdots$};
		\draw[->] (7.5,-9)--(7.5,-7) node[midway,right] {\tiny $\catvariableof{\atomorder\shortminus1}$}; 
		
%		\begin{scope}[shift={(-3.5,8)}]
%			\draw[fill] (7.5,-15) circle (0.25cm);
%			\draw[] (7.5,-15) to[bend left=25] (3.5,-13);
%			\draw[] (7.5,-15) to[bend right=25] (10.5,-13);
%
%			%\draw[fill] (9,-15.25) circle (0.25cm);
%			%\draw[] (9,-15.25) to[bend left=25] (5,-13);
%			%\draw[] (9,-15.25) to[bend right=25] (12,-13);
%
%			\draw[fill] (11.5,-15) circle (0.25cm);
%			\draw[] (11.5,-15) to[bend left=25] (7.5,-13);
%			\draw[] (11.5,-15) to[bend right=25] (14.5,-13);
%
%			%\drawatomindices{7.5}{-16}
%
%		\end{scope}
	
	\end{scope}

\end{scope}

\end{tikzpicture}
	\end{center}
	\caption{Representation of a basis+ Tensor by the contraction of a parameter tensor $\canparam$ with a slice selecting architecture $\fselectionmap$, which has a decomposition as a formula selecting neural network (see Theorem~\ref{lem:fsnnRepresentingSliceSelector}).
	The nonzero coordinates of $\canparam$ represent the (see Lemma~\ref{lem:sliceFromSliceSelector}).}
\end{figure}\label{fig:sliceSelectingNN}


\subsubsection{Applications}

One application is as a parametrization scheme in the approximation of a tensor by a slice-sparse tensor, see Chapter~\ref{cha:tensorApproximation}.


\red{
The approximated parameter can then be used as a proxy energy to be maximized.
When choosing $\selorder=2$, the approximating tensor contains only quadratic slices, which then poses a QUBO problem.
}

\begin{remark}[Extension to arbitrary CP-formats]
	Select at each input neuron a specific leg.
	For finite number of legs, as it is the case in the binary, basis and basis+ formats, we can enumerate all possibilities by the selection variable.
	For the basis+ format, in case of binary leg dimensions, we here exemplified the approach, by enumerating the three possibilities $\onehotmapof{0},\onehotmapof{1},\onesat{1}$.
	This approach, however, fails as a generic representation of the directed format, since the directed legs are continuous and there therefore are infinite choosable legs.
\end{remark}





\subsection{Constructive Bounds on CP Ranks}

After having defined three CP Decompositions, let us investigate bounds on their ranks which proofs come with constructions of the cores.


\subsubsection{Format Transformations}

%% Case of binary legs
Especially useful, when the leg dimensions are two, where the slice decomposition shows decomposition of the tensor into monomials.


\begin{theorem}\label{the:sliceToCP}
	For any tensor $\hypercoreat{\shortcatvariables}\in\facspace$ we have
		\[ \cprankof{\hypercore} \leq \bincprankof{\hypercore} \leq \baspluscprankof{\hypercore} \leq \bascprankof{\hypercore} \, . \]
\end{theorem}	
\begin{proof}
	% First bound
	Since any CP decomposition into binary leg cores can be normed to a CP decomposition with directed leg cores, the first bound holds.
	% Second bound
	The second bound holds analogously, since any CP decomposition with basis leg cores is also a CP decomposition with binary leg cores.
\end{proof}

	%Taking only $\variableset=[\atomorder]$ and the index set being the nonzero coordinates of $\hypercore$ we get 
	%	\[ \slicesparsityof{\hypercore} \leq \#{\catindices: \hypercore_{\catindices}\neq 0} = \sparsityof{\hypercore} \, . \]

%% Tightness of the Bounds
Consider for example the tensor $\ones$ having maximal $\ell_0$-norm being the dimension of the tensor space, but, since it is elementary, a CP decomposition with rank $1$.


\subsubsection{Summation of CP Decompositions}

\begin{theorem}\label{the:CPrankSumBound}
	For any collections of tensors $\{T^{l}[\catvariableof{\nodes}] : l \in [n]\}$ with identical variables and scalars $\lambda^{l} \in \rr$ for $l\in[n]$  we have
		\[ \cprankof{\sum_{l \in [n]} \lambda^{l} \cdot T^{l}} \leq \sum_{l\in[n]}  \cprankof{T^{l}}  \, . \]
	The bound still holds, when we replace on both sides $\cprankof{\cdot}$ by $\bincprankof{\cdot}$, by $\bascprankof{\cdot}$ or by $\baspluscprankof{\cdot}$.
\end{theorem}
\begin{proof}
	Products with scalars do not change the rank, since they just rescale the core $\scalarcore$.
	The sum of CP Decomposition is just the combination of all slices, thus the rank is at most additive.
\end{proof}

\subsubsection{Contractions of CP Decompositions}

More general, we can bound the sparsity of any contraction by the product of sparsities of affected tensors.

\begin{theorem}\label{the:CPrankContractionBound}
	For any tensor network with variables $\nodes$ and edges $\edges$ we have for any subset $\secnodes\subset\nodes$
		\[ \cprankof{\contractionof{\{\hypercoreof{\edge} : \edge\in\edges \}}{\secnodes}} \leq 
		\prod_{\edge\in\edges \, : \, \secnodes\cap\edge \neq \varnothing} \cprankof{\hypercoreof{\edge}} \, . \]
	The bound still holds, when we replace on both sides $\cprankof{\cdot}$ by $\bincprankof{\cdot}$, by $\bascprankof{\cdot}$ or by $\baspluscprankof{\cdot}$.
\end{theorem}


Remarkably, in Theorem~\ref{the:CPrankContractionBound} the upper bound on the CP rank is build only by the ranks of the tensor cores, which have remaining open edges.
We prepare for its proof by first showing the following Lemmata.

\begin{lemma}\label{lem:sparsityGeneralContraction}
	For any tensors $\hypercoreofat{1}{\catvariableof{\nodes_1}}$ and $\hypercoreofat{2}{\catvariableof{\nodes_2}}$ and any set of variables $\secnodes\subset\nodes_1\cup\nodes_2$ we have
		\[ \cprankof{\contractionof{\{\hypercoreof{1},\hypercoreof{2}\}}{\secnodes}} \leq \cprankof{\hypercoreof{1}} \cdot \cprankof{\hypercoreof{2}} \, . \]
	The bound still holds, when we replace on both sides $\cprankof{\cdot}$ by $\bincprankof{\cdot}$, by $\bascprankof{\cdot}$ or by $\baspluscprankof{\cdot}$.
\end{lemma}
\begin{proof}
	By connecting the cores and restoring the binary or basis properties.
\end{proof}

When one core of the contracted tensor network does not contain variables which are left open, we can drastically sharpen the bound provided by Lemma~\ref{lem:sparsityGeneralContraction} as we show next.

\begin{lemma}\label{lem:sparsityDisjointContraction}
	For any tensor network consistent of two tensors $\hypercoreofat{1}{\catvariableof{\nodes_1}}$ and $\hypercoreofat{2}{\catvariableof{\nodes_2}}$ and any set $\secnodes$ with $\secnodes\cap\nodes_2=\varnothing$ we have
		\[ \cprankof{\contractionof{\{\hypercoreof{1},\hypercoreof{2}\}}{\secnodes}} \leq \cprankof{\hypercoreof{1}} \, . \]
	The bound still holds, when we replace on both sides $\cprankof{\cdot}$ by $\bincprankof{\cdot}$ or by $\bascprankof{\cdot}$.
\end{lemma}
\begin{proof}
	We show the lemma by constructing a CP decomposition of $\cprankof{\contractionof{\{\hypercoreof{1},\hypercoreof{2}\}}{\secnodes}} $ for any CP decomposition of $\hypercoreof{1}$.
	Let therefore take any CP decomposition of $\hypercoreof{1}$ consistent of the leg cores $\{\legcoreof{\node} \, : \, \node \in \nodes_1 \}$ and a scalar core $\scalarcore$.
	Then we define a new $\scalarcore$ by
		\[ \tilde{\scalarcore} = \contractionof{\{\scalarcore\}\cup \{\legcoreof{\node} \, : \, \node \in \nodes_1 , \node \notin \secnodes \} \cup \{\hypercoreof{2}\} }{\decvariable} \, . \]
	Then, the leg cores $\{\legcoreof{\node} \, : \, \node \in \secnodes \}$ build with the scalar core $\tilde{\scalarcore}$ a CP decomposition of $\contractionof{\{\hypercoreof{1},\hypercoreof{2}\}}{\secnodes}$.
	% Binary of basis
	When the CP decomposition of $\hypercoreof{1}$ was binary, basis or basis+, this property is also satisfied by the constructed CP decomposition.
	Thus the bound also holds for the ranks $\bincprankof{\cdot}$ or $\bascprankof{\cdot}$.
\end{proof}

\begin{proof}[Proof of Theorem~\ref{the:CPrankContractionBound}]
	Use delta tensor representation to represent contractions by graphs.
	We then iterate through the cores and contract them to the previously contracted tensor, where we apply Lemma~\ref{lem:sparsityGeneralContraction} when the tensor core has variables left open and Lemma~\ref{lem:sparsityDisjointContraction} if not.
\end{proof}


\begin{example}[Composition of formulas with connectives]
	For any formula $\exformula$ we have $1-\exformula$ = $\lnot\exformula$.
	The CP rank bound brings an increase by at most factor $2$ when taking the contraction with $\concoreof{\lnot}$ which has slice sparsity of $2$.
	This is not optimal, since $\lnot\exformula$ has at most an absolute slice sparsity increase of $1$.
	
	For any formulas $\exformula$ and $\secexformula$ we have $\exformula\cdot\secexformula = \exformula\land\secexformula$.
	Here the CP rank bounds on contractions can also be further tightened.
\end{example}


\begin{example}[Distributions of independent variables]
	Independence means factorization, conditional independence means sum over factorizations.
	Again, the $\ell_0$ norm is bounded by the product of the $\ell_0$ norm of the factors.
\end{example}


\subsubsection{Sparse Encoding of Functions}

%Using the proof idea of Theorem~\ref{the:sparseBasisCP}, we can state a more general CP bound on the encoding of functions.

We now state that the basis CP rank of relational encodings is equal to the cardinality of the domain.
The basis CP format can therefore not provide a sparse representation when the factored system contains many categorical variables.

\begin{theorem}\label{the:rencodingBasCP}
	For any function
		\[ \exfunction : \facstates \rightarrow  \secfacstates \]
	between factored systems we have
		\[ \bascprankof{\rencodingof{\exfunction}} =  \facdim \, . \]
\end{theorem}
\begin{proof}
	With Theorem~\ref{the:sparseBasisCP}, the basis CP rank coincides with the number of not vanishing coordinates, which is the cardinality of the domain of $\exfunction$.
\end{proof}

Allowing for trivial leg vectors can decrease the CP rank, as we show next.

\begin{theorem}
	We have
		\[ \baspluscprankof{\rencodingof{\exfunction}} \leq  \sum_{y \in \imageof{\exfunction}} \baspluscprankof{\ones_{\exfunction == y} } \, , \]
	where by $\ones_{\exfunction == y} $ we denote the indicator, whether the function $\exfunction$ evaluates to $y$.
\end{theorem}
\begin{proof}
	We have
		\[ \rencodingof{\exfunction} = \sum_{y \in \imageof{\exfunction}} \ones_{\exfunction == y}[\catvariable]  \otimes \onehotmapofat{y}{\catvariableof{\exfunction}} \, . \]
	For each $y \in \imageof{\exfunction}$ we represent $\onehotmapofat{y}{\catvariableof{\exfunction}}$ in an basis+ CP format with $\baspluscprankof{\ones_{\exfunction == y} } $ summands and arrive at a basis+ CP decomposition of $\rencodingof{\exfunction}$ with $\sum_{y \in \imageof{\exfunction}} \baspluscprankof{\ones_{\exfunction == y} } $ summands.
\end{proof}

The above claim still holds when replacing $\baspluscprankof{\cdot}$ with the ranks $\bascprankof{\cdot}$ or $\bincprankof{\cdot}$.
For the rank $\bascprankof{\cdot}$ it leads to the bound of Theorem~\ref{the:rencodingBasCP}, since summing the number of non zero coordinators of the indicators is the cardinality of the domain.

\begin{example}{Conjunction of variables}
	For the propositional formula $\exformula = \catvariableof{0} \land \catvariableof{1}$ we have
		\[ \rencodingofat{\exformula}{\catvariableof{0},\catvariableof{1}}
		 = \onehotmapofat{1,1}{\catvariableof{0},\catvariableof{1}} \otimes \onehotmapofat{1}{\catvariableof{\exformula}}
		  +  (\onesat{\catvariableof{0},\catvariableof{1}} - \onehotmapofat{1,1}{\catvariableof{0},\catvariableof{1}}) \otimes \onehotmapofat{0}{\catvariableof{\exformula}}  \]
	and thus 
		\[ \baspluscprankof{\rencodingof{\exformula}} \leq 3\]
	while $\bascprankof{\rencodingof{\exformula}} = 4$.
	\red{Especially useful for $d$-ary conjunctions, see Remark~\ref{rem:naryConnectives}!}
\end{example}




\subsubsection{Construction by averaging the incoming legs}

Basis CP Decompositions can be constructed by understanding the variable $\indvariableof{\insymbol}$ of the relational encoding of a function $\exfunction:\inset \rightarrow \outset$ as the slice selection variable.

\begin{example}{Empirical distributions, see Theorem~\ref{the:empCPRep}}
	Let there be a data map 
		\[ \datamap : [\datanum] \rightarrow \facstates \, . \]
	We can use Theorem~\ref{the:functionDecompositionBasisCP} to find a tensor network representation fo $\rencodingof{\datamap}$ as
	\begin{align*}
		\rencodingofat{\datamap}{\catvariable,\shortcatvariables}  
		= \contractionof{
		\{\rencodingofat{\datamap^{\atomenumerator}}{\catvariable,\catvariableof{\atomenumerator}} : \atomenumeratorin \} 
		}{\catvariable,\shortcatvariables} \, . 
	\end{align*}
	This representation is in the CP format, when adding trivial scalar core and and delta tensor to the data index.
	It is furthermore in a basis CP format, since all $\rencodingof{\datamap^{\catenumerator}}$ are directed and binary tensors.
	Normation to get the empirical distribution amounts to setting a slice core with coordinates $\frac{1}{\datanum}$.
\end{example}





%\subsection{Manipulations of Binary CP Decomposition}\label{sec:BinaryCPManipulation}
%
%Since the coordinates on the legs are binary, operations like contractions, slicing and marginalization are especially efficient given binary CP Decompositions.
%
%
%\begin{example}[Hypertrie Format]
%	Hypertries are another efficient implementation of the slicing operations.
%%	Hypertries make use of the Basic TT Decomposition, given any permutation of the tensor legs.
%%	In addition, they eliminate storage redundancies when representing all permuted TT Decompositions, by referencing to same subnetworks (e.g. when slicing wrt leg 1 and then 2 or slicing wrt to 2 and then leg 1 will leave the same tensors to be further decomposed).
%\end{example}






%%% NEEDED?
%\subsection{Basis Tensor Networks}
%
%
%\begin{definition}
%	We call a tensor network, which cores are directed and binary a basis tensor network. %also acyclic?
%\end{definition}
%
%%\begin{definition}
%%	We call a Tensor Network with open legs $V$ basis, when for any tensor core in the network any slicing of the closed legs is parallel to a basic tensor (that is has $\ell_0$ norm of at most $1$).
%%\end{definition}
%
%
%\subsubsection{Basis elementary decomposition}
%
%Elementary tensors are tensor products of vectors.
%Demanding each vector in the product to be a basis vector leads to basis tensors.
%Thus the tensors which poses a basis elementary decompositions coincide with the basis tensors.
%
%\begin{theorem}
%	Given axis dimensions $\catdimof{\atomenumerator}\in\mathbb{N}$ for $\atomenumeratorin$, the one-hot encoding is a bijection between $\facstates$ and the basis tensors of $\facspace$ with unit norm.
%\end{theorem}
%\begin{proof}
%	The one-hot encoding to the basis tensors is injective, since each state is mapped to a different basis tensor.
%	The one-hot encoding is further surjective, since every basis tensor has a preimage state by the indices of the $1$ coordinate.
%	Therefore the one-hot encoding is a bijection.
%\end{proof}
%
%
%\subsubsection{Basis CP Decomposition}\label{sec:basisCP}
%
%We here provide with the CP Decomposition of binary tensors as ways to overcome the storage overhead of $\ell_0$-sparse (of tensor flattening) tensors.
%The key idea is to enumerate the nonzero coordinates by introducing an additional axis carrying the data index.
%Keeping the such introduced hidden rank as constant then results in an elementary tensor, which has a representation with linear demand.
%We can thus represent the vectors of each such elementary tensor in a matrix and get the cores of the CP decomposition.
%
%
%\subsubsection{Basis TT Decomposition}
%
%Exploiting vanishing slices, thus exploiting a form of block $\ell_0$-sparsity (where full slizes are vanishing).
%
%Can be generated from the basic CP decomposition, by the CP cores contracted with partial $\delta$ tensors.
%The rank will thus not be larger than the rank of the $\cpformat$.
%
%
%
%\subsubsection{Basic HT Decomposition}
%
%The constraint that networks need to be basic just affects the leaf cores.






\subsection{Subspaces of formulas}\label{sec:HT}

\red{
Formula Tensors have Tensor Network decomposition, which are best represented in a $\htformat$ decomposition.
We here describe this perspective and show applications of this formalism in the recovery/learning of formula tensors.
}
The decomposition of formula tensors is basic, since atomic formula tensors being on the leafs consist of basic vectors in the respective legs.


\subsubsection{Formula Subspaces}

Each formula tensor defines the subspace of $\atomspace$
\begin{align}
	\subspaceof{\exformula} = \mathrm{span} \left\{ \lnot\exformula,\exformula \right\} = \mathrm{im}\left(\ftensorof{\exformula} \right)
	%\mathrm{span} \left\{ \braket{\atombasisvector_1,\ftensorof{\exformula}}, \braket{\atombasisvector_0,\ftensorof{\exformula}} \right\}
\end{align}

Let us notice that the spanning vector of the subspace $\subspaceof{\exformula}$ are binary tensors summing up to the tensor of ones.

%\subsubsection{Atomic Tensor Spaces}

For each atom $\atomicformulaof{\atomenumerator}$ we have
	\[ \subspaceof{\atomicformulaof{\atomenumerator}} = \rr^2 \, . \]
The tensor space carrying the factored representation of the worlds is thus
\begin{align}
	\bigotimes_{\atomenumeratorin}\subspaceof{\atomicformulaof{\atomenumerator}} \, .
\end{align}

\subsubsection{Formula Decomposition as a Subspace Choice}

Given a formula $\exformula\exconnective\secexformula$ composed of formulas $\exformula$ and $\secexformula$ containing different atoms we have
\begin{align}
	\subspaceof{\exformula\exconnective\secexformula} 
	\subset \subspaceof{\exformula} \otimes \subspaceof{\secexformula}
\end{align}

A connective $\exconnective$ thus determines the selection of a two-dimensional subspace in the four-dimensional tensor product of subspaces to both subformulas.

%\subsubsection{Approximation problems}

Reconstruction of a formula given its formula tensor amounts to finding the HT Decomposition under the constraints of subspace choices according to the allowed logical connectives.

Given a set of positive and negative examples of a formula poses further an approximation problem of the examples by a HT Decomposition.

Advantages of this perspective are
\begin{itemize}
	\item Given a $\htformat$ the best approximation always exists (Theorem 11.58 in \cite{hackbusch_tensor_2012}), but need to further restrict to cores given by logical connectives 
	\item Apply Approximation algorithms: ALS or HOSVD
\end{itemize}


















%\section{Binary Optimization of Sparse Tensors}

Let us now study the problem of searching for the maximal coordinate in a tensor, when the tensor is represented in a basis+ CP Format. 


%\red{Search for the maximal coordinate in a tensor network is a binary optimization problem, given leg dimensions of 2.
%We explain this perspective in this chapter and connect it to the HUBO/QUBO formalism.}


\subsection{Mode search in exponential families}

Mode search 
\begin{align*}
	\max_{\shortcatindices\in\atomstates} \sbcontraction{\sencsstatat{\indexedshortcatvariables,\selvariable},\canparam} 
	= \max_{\meanparam\in\meanset} \sbcontraction{\meanparamat{\selvariable},\canparamat{\selvariable}}
\end{align*}


% Appearance of mode search
The search for maximal coordinates appears in various reasoning tasks:
\begin{itemize}
	\item MAP query as mode search of MLN: $\hypercore$ is the contraction of evidence with the distribution, leaving the query variables open.
	\item Grafting as mode search of proposal distribution: $\hypercore$ is the contraction of the gradient of the likelihood with the relational encoding of the hypothesis.
\end{itemize}
Both tasks have been formulated as mode search problems in exponential families.



\subsection{Higher-Order Unconstrained Binary Optimization (HUBO)}

\red{
Here binary refers to the leg dimensions $\catdimof{\atomenumerator}$ being 2, not to binary coordinates as often refered to in this work.
}


\begin{definition}
	The binary optimization of a tensor $\hypercoreat{\shortcatvariables}\in\atomstates$ is the problem
	\begin{align}\tag{$\mathrm{P}_{\hypercore}$}\label{prob:HUBO}
		\argmax_{\shortcatindices\in\atomstates} \hypercoreat{\indexedshortcatvariables} 
	\end{align}
	
	We call Problem~\ref{prob:HUBO} a Higher Order Unconstrained Binary Optimization (HUBO) problem of order $\sliceorder$ and sparsity $\slicerankwrtof{\sliceorder}{\hypercore}$, when $\hypercore$ has a monomial decomposition (see Definition~\ref{def:polynomialSparsity}) with $\cardof{\variablesetof{\decindex}}\leq\sliceorder$ for all $\decindexin$, that is when $\slicerankwrtof{\sliceorder}{\hypercore}<\infty$.
	
	
\end{definition}


\begin{remark}[Leg dimensions larger than 2]
% Leg dimension needs to be 2
	We demanded leg dimensions $\catdimof{\atomenumerator}=2$ to have binary valued variables $\catvariableof{\catenumerator}$, which is required to connect with the formalism of binary optimization.
	Categorical variables with larger dimensions can be represented by atomization variables, which are created by contractions with categorical constraint tensors (see Section~\ref{sec:categoricalTN}).
\end{remark}


% Interpretation of sparsity
The sparsity $\slicerankwrtof{\sliceorder}{\hypercore}$ is the minimal number of monomials, for which a weighted sum is equal to $\hypercore$.
Thus we interpret Problem~\ref{prob:HUBO} as searching for the maximum in a polynomial consistent of $\slicerankwrtof{\sliceorder}{\hypercore}$ monomial terms.
\red{Each monomial is also refered to as potential.}



%\begin{remark}[Sparsity]% To sparse Tensor Calculus?
%
%The number $\slicesparsityof{\hypercore}$ of a tensor $\hypercore$ defining a HUBO is of central importance to have an effective solution.
%%Here the number of nonzero coordinates coincides with the number of monomials required to represent the polynomial as a sum.
%
%\red{We investigated the Sparsity as the slice sparsity not the vector sparsity in Chapter~\ref{cha:sparseTC}.}
%%We here investigate, whether the same reasoning assumptions used for sparse representation by tensor networks also lead to $\ell_0$-sparse tensors. 
%
%\end{remark}




\subsection{Quadratic Unconstrained Binary Optimization (QUBO)}

\red{Quadratic Unconstrained Binary Optimization problems are HUBOs of order $\sliceorder=2$.}

We refine the monomial decomposition of tensors (see Definition~\ref{def:polynomialSparsity}) by demanding that monomials consist of at most two variables.

\begin{definition}
	We call a monomial decomposition $\sliceset$ of a tensor $\hypercore\in\atomspace$ a quadratic decomposition, if $\cardof{\variableset}\leq 2$ for all $(\lambda,\variableset,\catindexof{\variableset}) \in \sliceset$.
	We denote the smallest cardinality $\cardof{\sliceset}$ among quadratic decompositions of $\hypercore$ by $\quacprankof{\hypercore}$.

	If a tensor $\hypercore\in\bigotimes_{\atomenumeratorin}\rr^2$ has a quadratic decomposition, we call Problem~\ref{prob:HUBO} a Quadratic Unconstrained Binary Optimization (QUBO) problem of sparsity $\quacprankof{\hypercore}$.
\end{definition}

% CP Decompositions
Analogously to monomial decompositions, quadratic decompositions have an equivalence in a CP decomposition of $\hypercore$.
Beyond being binary tensors, the leg cores are further restricted that for each slice $\decindexin$ at most two of them are basis vectors and the rest trivial vectors $\ones$.

% Existence
We notice, that there are tensors, for which no quadratic decomposition exists.
This is already obvious from the fact, that the tensors with a quadratic decomposition build a $\binom{\atomorder}{2}$ dimensional submanifold in the $2^\atomorder$ dimensional tensor space.
This is in contrast with monomial decompositions, where one can always construct a decomposition.



%% OLD THEOREM: FALSE!
%However, for any non-negative tensor $\hypercore$ the Problem~\ref{prob:HUBO} is equivalent to a QUBO problem of possibly larger order as we state next.
%To turn HUBO problems into QUBO we need the slack variable trick, as described in the next lemma.

%\begin{theorem}\label{the:HUBOtoQUBO}
%	Let there be a tensor $\hypercore\in\in\bigotimes_{\atomenumeratorin}\rr^2$, which has a monomial decomposition with dimension $r$ and non-negative scalar core $\scalarcore$.
%	Then, the HUBO defined by $\hypercore$ is equivalent to a QUBO of order at most $\atomorder+r$ and sparsity at most $\atomorder \cdot r $.
%%	The maximal coordinate problem to any tensor $\hypercore\in\bigotimes_{\atomenumeratorin}\rr^2$ is equivalent to a QUBO with at most $\atomorder+\slicesparsityof{\hypercore}$ variables.
%%	\red{Need positive coordinates!}
%\end{theorem}

%To show the theorem we state the following lemma.


We can transform certain HUBO problems in QUBO problems with the usage of auxiliary variables, as we show in the next lemma.

%% Slack variables
\begin{lemma}\label{lem:monomialToQUBO}
	For any $\atomindices\in[2]$ and $\variableset\subset[\atomorder]$ we have 
		\[ \left( \prod_{\atomenumerator\in\variableset} \atomlegindexof{\atomenumerator } \right)  \left(  \prod_{\atomenumerator\notin\variableset} (1- \atomlegindexof{\atomenumerator }) \right)
		=
		\max_{\slackvariable\in[2]} \slackvariable \cdot 2 \cdot \left( \sum_{\atomenumerator\in\variableset}\atomlegindexof{\atomenumerator}  - \cardof{\variableset} - \sum_{\atomenumerator\notin\variableset}\atomlegindexof{\atomenumerator} + \frac{1}{2} \right) \, . % Alternative: no factor 2, but + 1 instead of +1/2 (->pyqubo)
 		\]
\end{lemma}
\begin{proof} %Proof by case distinction
	Only if $\atomlegindexof{\atomenumerator}=1$ for $\atomenumerator\in\variableset$ and $\atomlegindexof{\atomenumerator}=0$ else we have
		\[ \left( \sum_{\atomenumerator\in\variableset}\atomlegindexof{\atomenumerator}  - \cardof{\variableset} - \sum_{\atomenumerator\notin\variableset}\atomlegindexof{\atomenumerator} + \frac{1}{2} \right) \geq 0 \, . \]
	In this case the maximum is taken for $\slackvariable=1$ and we have
		\[ \max_{\slackvariable\in[2]} \slackvariable \cdot 2 \cdot \left( \sum_{\atomenumerator\in\variableset}\atomlegindexof{\atomenumerator}  - \cardof{\variableset} - \sum_{\atomenumerator\notin\variableset}\atomlegindexof{\atomenumerator} + \frac{1}{2} \right) 
		= 1 = \left( \prod_{\atomenumerator\in\variableset} \atomlegindexof{\atomenumerator } \right)  \left(  \prod_{\atomenumerator\notin\variableset} (1- \atomlegindexof{\atomenumerator }) \right) \, . \]
	In all other cases, the maximum is taken for $\slackvariable=0$ and thus vanishes, that is 
		\[ \max_{\slackvariable\in[2]} \slackvariable \cdot 2 \cdot \left( \sum_{\atomenumerator\in\variableset}\atomlegindexof{\atomenumerator}  - \cardof{\variableset} - \sum_{\atomenumerator\notin\variableset}\atomlegindexof{\atomenumerator} + \frac{1}{2} \right) 
		= 0 = \left( \prod_{\atomenumerator\in\variableset} \atomlegindexof{\atomenumerator } \right)  \left(  \prod_{\atomenumerator\notin\variableset} (1- \atomlegindexof{\atomenumerator }) \right) \, . \]
	Thus, the claim holds in all cases.
\end{proof}	


%\begin{proof}[Proof of Theorem~\ref{the:HUBOtoQUBO}]
%	For each summand in the monomial decomposition apply Lemma~\ref{lem:monomialToQUBO}.
%\end{proof}



\subsection{Integer Linear Programming}

Let us now show how optimization problems can be represented as linear programming problems.



\begin{definition}
	A Binary Integer Linear Program (ILP) is a problem of the form
	\begin{align*}
		\max_{x \in\{0,1\}^n} c^T x \quad \text{subject to } \quad A^{upper} x \leq b^{upper} , A^{lower} x \geq b^{lower} 
	\end{align*}
	where $A^{upper}\in\rr^{n^{upper}\times n}$, $b^{upper}\in\rr^{n^{upper}}$, $A^{lower}\in\rr^{n^{lower}\times n}$, $b^{lower}\in\rr^{n^{lower}}$.
\end{definition}



\begin{theorem}
	Given a monomial decomposition $\sliceset=\enumeratedslices$ of a tensor $\hypercore$ we define an Binary ILP as the maximation of 
	\begin{align*}
		\sum_{\decindexin} \slicescalar^{\decindex} \slackvariable^{\decindex} 
	\end{align*}
	with the constraints for any $\decindex$
	\begin{itemize}
		\item 
		\begin{align*}
			\slackvariable^{\decindex}  \leq \catvariableof{\atomenumerator} \quad \text{for} \quad \atomenumerator\in\variableset^j , \catindexof{\atomenumerator} = 1
		\end{align*}
		\item 
		\begin{align*}
			\slackvariable^{\decindex}  \leq (1-\catvariableof{\atomenumerator}) \quad \text{for} \quad \atomenumerator\in\variableset^j , \catindexof{\atomenumerator} = 0
		\end{align*}
		\item 
		\begin{align*}
			\slackvariable^{\decindex} \geq 1 + \sum_{\atomenumerator\in\variableset^{\decindex} : \catindexof{\atomenumerator} = 1} (\catvariableof{\atomenumerator} -1)
		- \sum_{\atomenumerator\in\variableset^{\decindex} : \catindexof{\atomenumerator} = 0} \catvariableof{\atomenumerator} 
		\end{align*}
	\end{itemize}
	The solution $\catindex^{ILP,\sliceset}$ of this ILP and the solution $\catindex^{HUBO,\sliceset}$ of the HUBO coincide on the variables of hypercore, i.e.
		\[ \catindex^{ILP,\sliceset}|_{[d]} =  \catindex^{HUBO,\sliceset} \, . \]
\end{theorem}
\begin{proof}
	We have to show that the constraints are satisfied if and only if $\slackvariable^{\decindex}=\onehotmapofat{\catvariableof{\variableset^{\decindex}}^{\decindex}}{\indexedcatvariableof{\variableset^{\decindex}}}$.
\end{proof}





% Solution Algorithms
\section{Reasoning by Tensor Approximation}\label{cha:tensorApproximation}

Often reasoning requires the execution of demanding contractions of tensors networks, or combinatorical search of maximum coordinates.
We in this chapter investigate methods, to replace hard to be sampled tensor networks by approximating tensor networks, which then serve as a proxy in inference tasks.


\subsection{Approximation of Energy tensors}




\subsubsection{Direct Approximation}

Direct approximation would be
	\[ \argmin_{\canparam\in\Gamma^{\graph}} \|\energytensorat{\shortcatvariables} - \canparamat{\shortcatvariables}\|^2 \]


\subsubsection{Approximation involving Selection Architectures}

Direct approximation would be
	\[ \argmin_{\canparam\in\Gamma^{\graph}} \|\energytensor - \sbcontractionof{\sencodingof{\fselectionmap},\canparam}{\shortcatvariables}\|^2 \]

In a tensor network diagram we depict this as
\begin{center}
    \input{PartIII/tikz_pics/approximation/least_squares.tex}
\end{center}


\begin{example}[Approximate based on a slice sparsity selecting architecture]
	Use a term selecting neural network (conjunction neuron on $\atomorder$ unary neurons selecting a variable and $\mathrm{Id},\lnot,\mathrm{True}$ as connective selector.
	Demand the parameter tensor $\canparam$ to be in a basis CP format, then each slice of the parameter tensor corresponds with the slice of the energy.
	The use the approximation for MAP search.
	Same construction possible for probability tensors, but often more involved to instantiate them as tensor network.
\end{example}


%\begin{figure}[h]



%\caption{Tensor Network Representation of the optimization.}
%\end{figure}



\subsection{Transformation of Maximum Search to Risk Minimization}

\red{By the squares risk trick, maximum coordinate searches involving contractions with boolean tensors can be turned into squares risk minimization problems.}
One can use this for MAP inference of MLN and the proposal distribution.


\subsubsection{Squares Risk Trick}

\begin{lemma}
	Let $\gentensor$ be a Boolean tensor, that is $\imageof{\gentensor}\subset\{0,1\}$.
	Then
		\[ \gentensor = \ones - \left( \gentensor - \ones \right)^2  \]
	where $\ones$ is a tensor with same shape as $\gentensor$ and all coordinates being $1$.
\end{lemma}
\begin{proof}
	Since for each $\shortcatindices\in\facstates$ we have $\gentensor[\shortcatvariables=\shortcatindices]\in\{0,1\}$, it holds that
		\[ \gentensor[\shortcatvariables=\shortcatindices] = 1 - (\gentensor[\shortcatvariables=\shortcatindices]-1)^2 \]
	and thus
		\[ \gentensor[\shortcatvariables] = \onesat{\shortcatvariables} - \left( \gentensor - \ones \right)^2  \]
	Applying this to all coordinates shows the claim.
\end{proof}

% OLD?
%Thus, in combination with the slicing theorem \ref{the:CoordinateTransform} we have 
%	\[ \argmax_{\basisslices} \gentensor\basisslices = \argmin_{\basisslices} \left ( \gentensor\basisslices - \ones \right)^2 \, . \]
%\red{ALS reformulation thus only works under the constraint of basis slices, i.e. single active formulas and datapoint contraction.}



\begin{theorem}\label{the:reweightedLeastSquares}
	Let $\Theta$ be a set of binary tensors in $\facspace$ and $\importancetensor\in\facspace$ arbitrary.
	Then we have
	\begin{align}
		\argmax_{\hypercore\in\Theta} \contraction{\importancetensor,\hypercore} 
		= \argmin_{\hypercore\in\Theta} \contraction{\importancetensor, (\hypercoreat{\shortcatvariables}-\onesat{\shortcatvariables})^2}
		%\sum_{\catindices} \left(\theta_{\catindices} -1 \right)^2 \hypercore_{\catindices} \, . 
	\end{align} 
\end{theorem}
\begin{proof}
	Using the Lemma above, $\hypercore$ is identical to $\onesat{\shortcatvariables}-(\hypercoreat{\shortcatvariables}-\onesat{\shortcatvariables})^2$ and we get
	\begin{align*}
		 \contraction{\importancetensor,\hypercore} 
		 &=  \contraction{\importancetensor,\onesat{\shortcatvariables}}-\contraction{\importancetensor,(\hypercoreat{\shortcatvariables}-\onesat{\shortcatvariables})^2} 
	\end{align*}
	Since the first term does not depend on $\hypercore$, it can be dropped in the maximization problem.
	The $(-1)$ factor then turns the maximization into a minimization problem.
\end{proof}

% Interpretation and Importance Tensor
Corollary~\ref{cor:reweightedLeastSquares} reformulates maximation of binary tensors with respect to an angle to another tensor into minimization of a squares risk.
This squares risk trick is especially useful when combining it with a relaxation of $\Theta$ to differentiably parametrizable sets, since then common squares risk solvers can be applied.
We will call $\hypercore$ in the Corollary~\ref{cor:reweightedLeastSquares} importance tensor, since it manipulates the relevance of each coordinate in the squares loss.

\begin{example}[Proposal distribution maxima]
	The Problem~\ref{prob:steepestAscent} of finding the maximal coordinate can thus be turned into
	\begin{align*}
		\argmax_{\shortselindices} \contractionof{(\empdistribution-\currentdistribution),\fselectionmap}{\shortselvariables=\shortselindices}  
		= \argmin_{\shortselindices} \sbcontraction{(\empdistribution-\currentdistribution),
		\left(\contractionof{\fselectionmap,\onehotmapofat{\shortselindices}{\shortselvariables}}{\shortcatvariables}-\onesat{\shortcatvariables}\right)^2} \, . 
	\end{align*}
\end{example}



%\subsubsection{Squares Loss Reformulation of Contracted Binary Tensors}
%
%% Motivation
%We here make use of the fact, that $\rencodingof{\formulaset}$ is a binary tensor, which is also true for all contractions with elementary selection tensors.
%When, like it is the case here, the set $\Gamma$ of hypothesis directions consists of binary tensors, we can apply the squares risk trick (Corollary~\ref{cor:reweightedLeastSquares}) to reformulate Problem~\ref{prob:steepestAscent} to a least squares problem.
%
%\begin{theorem}
%	When $\Gamma$ consists of binary tensors, %that is
%%		\[ \Gamma \subset \facspace \]
%	then %Problem~\ref{prob:steepestAscent} 
%	\begin{align}
%		\argmax_{\theta\in\Gamma} \braket{\theta, \frac{\partial \lossof{\mlntensor}}{\partial \mlntensor}} =
%		\argmin_{\theta\in\Gamma} \sum_{\catindices} \left(\theta_{\catindices} -1 \right)^2  \left(\frac{\partial \lossof{\mlntensor}}{\partial \mlntensor_{\catindices}}\right) \, . 
%	\end{align}
%\end{theorem}
%\begin{proof}
%	Direct application of Corollary~\ref{the:reweightedLeastSquares}.
%\end{proof}


%Since any formula tensor $\ftensorof{\exformula}$ contains only binary coordinates, we can reformulate the loss as:
%\begin{align}
%	\lossof{\mlnparameters} = &
%		\log\partitionfunctionof{\formulasum\weightof{\exformula}\ftensorof{\exformula}} 
%		+ \formulasum\weightof{\exformula}
%		+ \formulasum\weightof{\exformula}\variablesum  \left( 1 - \braket{\ftensorof{\exformula}, \tensordataof{\variableindex}} \right) \\
%		= & 
%		\log\partitionfunctionof{\formulasum\weightof{\exformula}\ftensorof{\exformula}} 
%		+ \formulasum\weightof{\exformula}
%		+ \formulasum\weightof{\exformula}\variablesum  \left( 1 - \braket{\ftensorof{\exformula}, \tensordataof{\variableindex}} \right)^2
%\end{align}


%\begin{remark}{Interpretation of Datapoints}
%	The datapoints are exactly the nonzero entries of the importance core $\importancetensor$.%, i.e. corresponding with individuals in e.g. an \rdf Knowledge Graph.
%\end{remark}







\subsection{Parametercores}

\red{Parameter cores as tradeoff between expressivity and complexity in sampling.
When not hidden variables, their maximum can be found by }


Parametercores need also an efficient representation.

In case of skeleton expressions with many placeholders further decomposition for algorithmic efficiency are required.
\begin{itemize}
	\item Elementary Format ($\elformat$-Format): 
	\item $\cpformat$-Format: Closest to sum of formula tensors (when all vectors are basis, then have a sum).
	\item $\ttformat$-Format: Showed better heuristic performance in optimization
\end{itemize}

For any tensor network decomposition into cores $\parametercoreof{\parenumerator}$ have the derivative $\frac{\partial}{\partial \parametercoreof{\parenumerator}} \parametertensor$ as the tensor network with out the core $\parametercoreof{\parenumerator}$.

\begin{remark}[Parametercores being basis tensors]
	When the parameter core is a basis tensor, the contraction with the parametercore coincides with the respective formula tensor.
	Thus, we will search for basis tensors optimizing in contractions objectives to specific reasoning tasks, and add them iteratively to the network at hand.
\end{remark}

%% Bayesian Neural Network Perspective
When the parametercores are a tensor network representing a probability tensor by there contraction, we can interpret them as a probability distribution over formulas.



\subsection{Fitting a tensor by a formula tensor}

Task: Given a tensor $\hypercore$, find a formula $\exformula\in\formulaset$ such that it coincides with $\hypercore$.

If $\hypercore$ is a binary tensor, we understand it as a formula and want to find an $\exformula$ such that its number of worlds is maximal, that is solve the problem
	\[ \argmax_{\exformula\in\formulaset}\sbcontraction{\exformula\Leftrightarrow\hypercore}  \, . \]

We can use the squares risk trick and get an equivalent problem
	\[ \argmin_{\exformula\in\formulaset} \| \sbcontractionof{\exformula\Leftrightarrow\hypercore}{\shortcatvariables}  - \onesat{\shortcatvariables} \|^2 \, . \]


%\begin{remark}{Least Squares Loss by Tensor Fitting}
%	\red{Alternative approach to least squares problems: Tensor Fitting}
%	And, if the target is another formula $y$, such that $\exformula$ conincides with $\tilde{f} \iff y $ we have
%		\[ \left(\polynomialof{\exformula}(\datamap)-1\right)^2 = \left(  \polynomialof{\tilde{f}}(\datamap) - y(\datamap) \right)^2  \]
%	This is exactly the least squares loss, which would appear in a supervised interpretation of the learning.
%\end{remark}




\subsection{Alternating Solution of Least Squares Problems}

When the parameter tensor $\canparam$ is only restricted to have a decomposition as a tensor network on $\graph$, we can iteratively update each core.
The resulting algorithm is called Alternating Least Squares (ALS) (see Algorithm \ref{alg:ALS}).

\begin{algorithm}[hbt!]
\caption{Alternating Least Squares (ALS)}\label{alg:ALS}
\begin{algorithmic}
\For{$\atomenumeratorin$}
	\State Set $\varcore{\atomenumerator}$ to a random element in $\rr^{\atomlegdimof{\atomenumerator}}$ 
\EndFor
\While{Stopping criterion is not met}
\For{$\atomenumeratorin$}
	\State Set $\varcore{\atomenumerator}$ to a to a solution of $ \argmin_{\varcore{\atomenumerator}}  \left\|  \importancetensor \chadamard ( \groundingof{\parametrization(\varcore{1},\ldots,\varcore{\atomorder})} - \targettensor ) \right\|^2$
\EndFor
\EndWhile
\end{algorithmic}
\end{algorithm}



%\subsection{Projection onto Basis Tensors}
%\red{This is sampling!}
%We project onto basis tensors to achieve single formulas.


% Statistical Analysis
%\section{Concentration of the Expected Sufficient Statistics}
\section{Uniform Concentration of Random Contractions}\label{cha:widthBounds}

We here derive bounds on the uniform concentration of contractions with random tensors.




	
	
	

The width of a vector $\noisetensor$ is the supremum of contractions with respect to a set $\Gamma$ is
\begin{align}
	\widthwrtof{\Gamma}{\noisetensor}
	= \sup_{\theta\in\Gamma} \sbcontraction{\theta,\noisetensor} \, . 
	%= \sup_{\theta\in\Gamma} \contractionof{\{\theta, \noisetensor\}}{\varnothing} \, . 
\end{align}

We are interested in the random vector
	\[ \noisetensor^{\formulaset,\gendistribution,\datamap} = \sbcontractionof{\empdistribution,\formulaset}{\selvariable} -  \sbcontractionof{\gendistribution,\formulaset}{\selvariable}  \,  \]
	%\essdistof{\empdistribution}-\expectationof{\essdistof{\empdistribution}} \]
which is the difference between the mean parameters given the empirical distribution and the underlying generating distribution.

We derive bounds for the hypothesis $\Gamma$ being the set of basis vectors (i.e. for feature search) and being the set of normed vectors (i.e. for feature calibration).





%\subsection{Binomials}
%\subsubsection{Concentration Bounds for Binomials}





\subsection{Naive bounds given binomial coordinates}


We here investigate width bounds on random tensors $\noisetensor$, which coordinates have marginal distributions by Binomials.

% MLN
This is the case for $\noisetensor^{\fselectionmap,\gendistribution,\datanum}$.

Naive means, that we do not exploit the dependencies of the coordinates on each other, as would be the case in more sophisticated chaining approaches.

\subsubsection{Basis Vectors}

We exploit the sub-Gaussian Norm (Def 2.5.6 in CITE Vershynin Book) to state concentration inequalities.

\begin{definition}[Sub-Gaussian Norm]
	The sub-Gaussian norm of a random variable $X$ is defined as
		\[ \sgnormof{X} = \inf \left\{ C > 0 \, : \expectationof{\frac{X^2}{C^2} } \leq 2 \right\} \, .  \]
\end{definition}

\begin{theorem}
	Any coordinate of $\noisetensor$ is sub-Gaussian with 
		\[ \sgnormof{\noisetensor_i} \leq \frac{C}{\sqrt{\ln2}} \frac{1}{\sqrt{m}} \]
\end{theorem}
\begin{proof}
	Centered Bernoulli is bounded and therefore sub-Gaussian.
	Binomial is a sum and we apply Proposition 2.6.1 in CITE Vershynin Book.
\end{proof}


\begin{theorem}\label{the:basisTensorWidthBound}
	Let 
		\[ \Gamma = \{\onehotmapof{i} : i \in [p] \}\]
	and $\noisetensor$ be a random vector in $\rr^p$, which coordinates have a marginal distribution being centered Binomials with a fixed $\datanum\in\nn$.
	Then
		\[ \sgnormof{\widthatof{\Gamma}{\noisetensor}} \leq C \sqrt{\frac{\ln p}{m}} \, . \]
	where $C>0$ is a universal constant.
\end{theorem}
\begin{proof}
	Supremum of Sub-Gaussian variables.
\end{proof}


\subsubsection{Sphere}

We first provide a Chebyshev bound on the width of the sphere.

\begin{theorem}\label{the:sphereBoundVariance}
	Let $\noisetensor$ be a random tensor with marginal coordinate distributions by binomials with parameters $(\fprobof{\catindex},\datanum)$.

	For any $\failprob>0$, $\precision>0$ and $\datanum\in\nn$ with probability at least $1-\failprob$ we have
		\[ \normof{\frac{\noisetensor-\expectationof{\noisetensor}}{\datanum}} \leq   \precision \, \]
	provided that
		\[ \datanum \geq  \frac{ \sum_{\catindex\in\facstates} \fprobof{\catindex}(1-\fprobof{\catindex})}{\precision^2 \failprob} \, . \]
\end{theorem}
\begin{proof}
	%We estimate with the Cauchy Shwartz inequality
	%	\[ \braket{\theta,\noisetensor} \leq \|\noisetensor\| \|\theta\| \, . \]
	Since the squared norm of the noise is the sum of squared centered and averaged Binomials, we have
		\[  \expectationof{\normof{\noisetensor-\expectationof{\noisetensor}}^2}  
		= \sum_{\catindex\in\facstates} \frac{\fprobof{\catindex}(1-\fprobof{\catindex})}{\datanum} \]
	Here we used that the variance of Binomials with parameters $(\fprobof{\catindex},\datanum)$ is $\fprobof{\catindex}(1-\fprobof{\catindex}) \datanum$.
	
	If follows, that 
		\[ \expectationof{\left(\normof{\frac{\noisetensor-\expectationof{\noisetensor}}{\datanum}}\right)^2} =  \frac{ \sum_{\catindex\in\facstates} \fprobof{\catindex}(1-\fprobof{\catindex})}{\datanum} \, . \]
	
	Then we apply a Chebyshev Bound to get for any $\precision>0$
	\begin{align}
		\probof{\normof{\frac{\noisetensor-\expectationof{\noisetensor}}{\datanum}} > \precision} 
		= \probof{\left(\normof{\frac{\noisetensor-\expectationof{\noisetensor}}{\datanum}}\right)^2 > \precision^2} 
		\leq \frac{ \sum_{\catindex\in\facstates} \fprobof{\catindex}(1-\fprobof{\catindex})}{\datanum \cdot \precision^2}
	\end{align} 
	For a $\failprob>0$ we choose any $\datanum$ with
		\[ \datanum \geq  \frac{ \sum_{\catindex\in\facstates} \fprobof{\catindex}(1-\fprobof{\catindex})}{\precision^2 \failprob} \, \]
	and get 
	\begin{align}
		\probof{\normof{\frac{\noisetensor-\expectationof{\noisetensor}}{\datanum}} > \precision} \leq \failprob \, . 
	\end{align} 
	Thus, we have 
	\begin{align}
		\probof{\normof{\frac{\noisetensor-\expectationof{\noisetensor}}{\datanum}} \leq \precision} = 1 - \probof{\normof{\frac{\noisetensor-\expectationof{\noisetensor}}{\datanum}} > \precision}  \geq 1-\failprob \, . 
	\end{align} 
\end{proof}


% Multinomial
For $\sstat=\identity$ the noise tensor is a rescaled and centered multinomial.

\begin{corollary}
	Let there be multinomial variable with parameters $(\fprob,\datanum)$ where $\fprob\in\facspace$ a positive and normed tensor.
	Let $\datamap$ be a set of independent samples
	For any $\failprob>0$, $\precision>0$ and $\datanum\in\nn$ with probability at least $1-\failprob$ we have
		\[ \normof{\frac{\noisetensor-\expectationof{\noisetensor}}{\datanum}} \leq   \precision \, \]
	provided that
		\[ \datanum \geq  \frac{(1-\sbcontraction{(\fprob)^2})}{\precision^2 \failprob} \, . \]
\end{corollary}
\begin{proof}
	Theorem~\ref{the:sphereBoundVariance} with 
		\[ \sum_{\catindex\in\facstates} \fprobof{\catindex}(1-\fprobof{\catindex}) = \sum_{\catindex\in\facstates} \fprobof{\catindex} - \sum_{\catindex\in\facstates} (\fprobof{\catindex})^2 = 1-\sbcontraction{(\fprob)^2} \, . \]
\end{proof}



\subsubsection{Bounds based on the sub-gaussian norm}

\red{Unclear whether this is needed.}

A faster tail decay can be achieved, when bounding sub-gaussian norms.


\begin{theorem}
	Let $\noisetensor$ be a random vector in $\rr^p$, which coordinates have a marginal distribution being centered Binomials with a fixed $\datanum\in\nn$.
	Then
		\[ \sgnormof{\widthatof{\subsphere}{\noisetensor}} \leq C \sqrt{\frac{p}{m}} \, . \]
	where $C>0$ is a universal constant.
\end{theorem}
\begin{proof}
	Using that each coordinate has Sub-gaussian norm of at most $1$.
	\red{Asymptotically, the binomial tends to a gaussian, which has a smaller sg norm. 
	But the binomial has a sub-exponential regime preventing tighter sg bounds.}

	
	Norm of a Sub-gaussian vector, another application of Proposition 2.6.1 in CITE Vershynin Book.
\end{proof}




\subsection{Chaining bounds given binomial coordinates}

To proceed with the uniform concentration investigation, we need a concentration bound on Binomials.

\begin{theorem}
	For any $p\in[0,1]$ and $\datanum\in\mathbb{N}$ any $X \sim \bidistof{p,\datanum}$ satisfies for any $t>0$
		\[ \probof{X-\expectationof{X} > t}  \leq \expof{- \frac{t^2}{2\datanum p + \frac{2t}{3}} } \]
	and
		\[ \probof{\expectationof{X} - X > t}  \leq \expof{- \frac{t^2}{2\datanum p}} \]
	Thus
		\[ \probof{|\expectationof{X} - X| > t} \leq  2 \expof{- \frac{t^2}{2\datanum p + \frac{2t}{3}} }  \]
\end{theorem}
\begin{proof}
	See e.g.
	\href{https://mathweb.ucsd.edu/~fan/wp/concen.pdf}{https://mathweb.ucsd.edu/~fan/wp/concen.pdf}
%	The proof uses the Chernoff bound applying the moment generating function, which is for Binomial variables X and $\lambda\geq0$
%	\begin{align}
%	 	\expectationof{\expof{\lambda (X - \expectationof{X})}} 
%		= & \left( (1-p) \cdot \expof{\lambda -p} + p \cdot \expof{\lambda (1-p)}\right)^\datanum \\
%		= & \expof{-\lambda p \datanum} \left(1 + p(\expof{\lambda}-1) \right)^\datanum \, .
%	\end{align}
%	The Chernoff bound used for any $\lambda>0,t>0$ the Markov inequality
%	\begin{align}
%		\probof{X-\expectationof{X} > t} 
%		= \probof{\expof{\lambda(X-\expectationof{X})} > \expof{\lambda t} }
%		\leq \frac{\expectationof{\expof{\lambda(X-\expectationof{X})}}}{\expof{\lambda t} } 
%	\end{align}
\end{proof}

The binomial thus has a sub-gaussian and a sub-exponential regime.

\begin{theorem}
	For any $p\in[0,1]$ and $\datanum\in\mathbb{N}$ any $X \sim \bidistof{p,\datanum}$ satisfies for any $t>0$
		\[ \probof{|\expectationof{X} - X| > \sqrt{4\datanum p t} + 2 t} \leq  2 \expof{- t }  \]
\end{theorem}
\begin{proof}
	For any s>0 we choose t>0 such that 
		\[ s = - \frac{t^2}{2\datanum p + \frac{t}{3}}  \]
	and observe 
		\[ \min\left( \frac{t^2}{4\datanum p},\frac{t^2}{2t} \right) \]
	and
		\[ t \leq \max(\sqrt{4\datanum p s},2s) \leq  \sqrt{4\datanum p s} + 2s \, . \]
	With the above bound it holds, that
		\[  \probof{|\expectationof{X} - X| >  \sqrt{4\datanum p s} + 2s}
		\leq \probof{|\expectationof{X} - X| > t}
		\leq 2 \expof{- \frac{t^2}{2\datanum p + \frac{t}{3}} } 
		\leq 2 \expof{-s} \, . \]
\end{proof}

We apply this on the variable  $\braket{\ftensor,\noisetensor}$.

\begin{theorem}
	Let $\ftensor = \sum_{\mlnformulain} \weightof{\exformula}\exformula$, then
	\[ \probof{\datanum |\braket{\ftensor,\noisetensor}|\geq \sqrt{4 \datanum} \cdot \left( \sum_{\mlnformulain} |\weightof{\exformula}| \sqrt{\fprobof{\exformula}} \right) \sqrt{t}  + 2 \cdot \left( \sum_{\mlnformulain} |\weightof{\exformula}|  \right) t } \leq 2|\mlnformulaset | \cdot \expof{- t} \]
\end{theorem}
\begin{proof}
	We can not assume independence of the $\braket{\exformula,\noisetensor}$ (in that case we could use a Bernstein inequality) and instead take the naive bound over all formulas in $\mlnformulaset$ 
	\begin{align}
		& \probof{|\braket{\ftensor,\noisetensor}|\geq
		 \sqrt{4 \datanum} \cdot \sum_{\mlnformulain} |\weightof{\exformula}| \sqrt{\fprobof{\exformula}}\sqrt{t}  
		 + 2 \cdot \sum_{\mlnformulain} |\weightof{\exformula}|  t } \\
		& \quad \quad \leq \probof{\exists \mlnformulain \, : \, |\braket{\exformula,\noisetensor}|\geq  \sqrt{4 \datanum}  |\weightof{\exformula}| \sqrt{\fprobof{\exformula}}\sqrt{t}  + 2 |\weightof{\exformula}|  t  }\\
		& \quad \quad \leq \sum_{\mlnformulain} \probof{ |\braket{\exformula,\noisetensor}|\geq  \sqrt{4 \datanum}  |\weightof{\exformula}| \sqrt{\fprobof{\exformula}}\sqrt{t}  + 2 |\weightof{\exformula}|  t  }\\
		& \quad \quad  \leq 2|\mlnformulaset| \cdot  \expof{-t} \, .
	\end{align}
\end{proof}

We can thus proof uniform concentration bounds given covering bounds of a hypothesis set in $\ell_1$ norm (and the reweighted one).

\begin{remark}[Small Formula Probabilities]	
	Directions with small $\fprobof{\exformula}$ will require larger covering sets and thus have large contributions to the bounds.
	Intuitively, they correspond with exceptional special cases, which need many samples to be observed. 
\red{Strange: Should also $1-\fprobof{\exformula}$ small intuitely be an issue?}
\end{remark}


\subsubsection{Generic Width Bounds}

We define the maps
	\[ \nu^p_2(\ftensor) =  \inf_{\mlnparameters : \sum_{\mlnformulain}\weightof{\exformula}\exformula = \ftensor}  \left( \sum_{\mlnformulain} |\weightof{\exformula}| \sqrt{\fprobof{\exformula}} \right) \]
and
	\[ \nu_1(\ftensor) =  \inf_{\mlnparameters : \sum_{\mlnformulain}\weightof{\exformula}\exformula = \ftensor} \left( \sum_{\mlnformulain} |\weightof{\exformula}| \sqrt{\fprobof{\exformula}} \right)  \]
corresponding to the sub-gaussian and sub-exponential regimes.

%This is almost the mixed tail definition of the thesis, just a factor on the probability

\begin{theorem}
	Let $\mlnformulaset$ be a set of formulas and $W\subset\rr^{|\mlnformulaset|}$ a set of weight vectors.
	Then with probability at least $1- |\mlnformulaset| C_1 \expof{-\frac{u^2}{2}} $ we have for the set
		\[ \Gamma = \big\{ \sum_{\mlnformulain}\weightof{\exformula}\exformula \, : \, (\weightof{\exformula})_{\mlnformulain} \in W \big\} \]
	the bound
		\[ \omega_\Gamma(\noisetensor)  \leq \frac{C_2 \gamma_{2}(\Gamma,\nu^p_2)}{\sqrt{\datanum}} + \frac{C_3 \gamma_{1}(\Gamma,\nu_1)}{\datanum} \, . \]
\end{theorem}



%% OLD: These situations are addressed more directly

%\subsubsection{Recovery of atom assignment in skeleton}
%
%We select for each placeholder in the skeleton an atom of $\variableorder$ possible choices.
%The set of formula tensors resulting from these choices is
%	\[ \Gamma^{\skeleton} = \left\{ \exformula \, : \, \exformula = \skeleton(\atomindices), \atomindices \in [\variableorder] \right\} \, .\]
%We can estimate the cardinality by
%	\[ \cardof{\Gamma^{\skeleton}} \leq \variableorder^\atomorder \, . \]
%This is just an inequality, since assignments of atoms to placeholders of the skeleton can result in identical formulas.
%
%When restricting choices by a candidatesdict, the bound can be sharpened by the product of the cardinality at each placeholder.
%
%\begin{theorem}
%	Let $\skeleton$ be a skeleton formula with $\atomorder$ placeholders and $\variableorder$ atoms, which can be selected at each position.
%	Then we have with probability at least $1-C_1\expof{-\frac{u^2}{2}}$
%		\[ \omega_\skeleton(\noisetensor)  \leq 2C_2 \sqrt{\frac{ \atomorder \ln\variableorder}{\datanum}} + 2C_3\frac{\atomorder\ln\variableorder}{\datanum}  \]
%\end{theorem}
%\begin{proof}
%	Application of the generic Dudleys entropy bound.
%\end{proof}
%
%Thus
%	\[ \datanum \sim \atomorder\ln(\variableorder) \]
%is enough for a sharp Kullback Leibler bound of the solution.
%
%
%\subsection{Recovery of weight parameters }




\appendix
\chapter{Implementation in the \tnreason package}\label{cha:implementation}

We here document the implementation of the discussed concepts in the \python package \tnreason, in the version \curvertnreason
 
 % Name
\tnreason is an abbreviation of \textbf{t}ensor \textbf{n}etwork \textbf{reason}ing, by which we emphasize the capabilities of this package to represent and answer reasoning tasks by tensor network contractions. 

% Installation
The package can be installed either by cloning the repository
\begin{center}
	\href{https://github.com/EnexaProject/enexa-tensor-reasoning}{https://github.com/EnexaProject/enexa-tensor-reasoning}
\end{center}
or by
\begin{lstlisting}
	!pip install tnreason==2.0.0
\end{lstlisting}

\sect{Architecture}

\tnreason is structured in four subpackages and three layers
\begin{itemize}
	\item Layer 1: Storage and numerical manipulations, by subpackage \spengine, "Tensor Networks" -> building "tn" of \tnreason
	\item Layer 2: Specification of workload, subpackage \sprepresentation specific for storage, subpackage \spreasoning specific for manipulations
	\item Layer 3: Applications in reasoning, by subpackage \spapplication, "Reasoning" -> building "reason" of \tnreason
\end{itemize}

We sketch this structure by
\begin{center}
\input{./OtherContent/tikz_pics/implementation/architecture_sketch.tex}
\end{center}


\sect{Implementation of basic notation}

First of all, we explain how the basis notation explained in \charef{cha:notation} is reflected in the implementation.

\subsect{\bncategoricals}
Categorical Variables are identified by strings, which then appear as colors of the corresponding tensor axes.
Their dimension is stored in shapeDicts, but most practically these shapes are stored in the tensors in which variables appear.
Suffixes in the color string (defined in \inlinecode{representation.suffixes}) denote the type of the variable:
\begin{itemize}
	\item Distributed variables with color suffix \disVarSuf: $\catvariableof{\cdot}$
	\item Computed variables with color suffix \comVarSuf: $\headvariableof{\cdot}$
	\item Selection variables with color suffix \selVarSuf: $\selvariableof{\cdot}$
	\item Term variables with color suffix \terVarSuf: $\indvariableof{\cdot}$
\end{itemize}

\subsect{\bntensors}
\paragraph{Tensors} are objects of classes inheriting \inlinecode{engine.TensorCore} with main attributes
\begin{itemize}
	\item \inlinecode{values}: Storing the coordinates of the tensors (individual realization for different cores)
	\item \inlinecode{colors}: List of the variables $[\headvariableof{\formula},\catvariableof{0},\catvariableof{1}]$
	\item \inlinecode{name}: Reflecting the notation such as $\rencodingof{\formula}$
	\item \inlinecode{shape}: Storing the dimension of each appearing variable, as a list of integers with the same length as colors.
\end{itemize}

Suffixes in the name string (defined in \inlinecode{representation.suffixes}) highlight the origin and purpose of the tensor.
Cores are named with suffixes based on their functionality
\begin{itemize}
	\item Computation core with name suffix \comCoreSuf: They represent the computation of a function in basis calculus, and are directed cores.
		Their colors are \inlinecode{[headColors] + [inputColors]}, where \inlinecode{[inputColors]} are either distributed variables or, if having a composition of formulas.
		When the function is a selection augmentation of other functions, selection colors are listed in the end of \inlinecode{[inputColors]}.
	\item Activation core with name suffix \actCoreSuf: two-dimensional vectors representing of the activation core to a formula
\end{itemize}

Exploiting efficient representation tricks we further have the tensor name suffices:
\begin{itemize}
	\item \atoCoreSuf: Atomization core, for sparse representation of categorical constraints
	\item \vselCoreSuf: Variable selection core: For sparse representation of variable selectors
\end{itemize}

Tensors are instantiated by
\begin{lstlisting}
	engine.getCore(coreType)(values, colors, name, shape)
\end{lstlisting}
where \inlinecode{coreType} is a string further specifying a specific implementation of tensors (see for more detail \secref{sec:implementationEngine}).
The default tensor implementation \defaultCoreType is chosen, when \inlinecode{coreType} is not specified.



One-hot encodings are specific tensors created in \sprepresentation.

\subsect{\bncontractions}
\paragraph{Tensor networks} are stored as dictionaries of tensors, where the keys coincide with the names of the corresponding tensors.

\paragraph{Contractions} are implemented in the subpackage \spengine, orienting on \defref{def:contraction}.
Reflected in the notation
\begin{align*}
	\contractionof{\tnetof{\graph}}{\secnodevariables}
\end{align*}
a contraction is defined by
\begin{itemize}
	\item Tensor Network $\tnetof{\graph}$, specified by a dictionary of tensor names as keys and valued by tensor cores.
	\item Open Variables $\secnodes$, specified by a list of colors to the variables.
\end{itemize}
Contraction calls are implemented as
\begin{lstlisting}
	engine.contract(contractionMethod, coreDict, openColors, dimensionDict, evidenceColorDict)
\end{lstlisting}
where the arguments are
\begin{itemize}
	\item \inlinecode{contractionMethod}: str, chooses one of the contraction providers. The default contraction method \defaultContractionMethod is chosen, when
	\item \inlinecode{coreDict}: Dictionary of TensorCores (of the above formats), representing the Tensor Network $\tnetof{\graph}$
	\item \inlinecode{openColors}: List of str, each str identifying a color, that is a variable to be left open in the contraction
	\item \inlinecode{dimensionDict}: Dict valued by int and keys by str, storing dimensions to each variable. This is of optional usage, when a color in openColors does not appear in the coreDict.
	\item \inlinecode{evidenceColorDict}: Dict valued by int and keys by str, indicating sliced variables
\end{itemize}

Coordinates of tensors can be retrieved by
\begin{align*}
	\contractionof{\hypercoreat{\nodevariables}}{\secnodevariables=\catindexof{\secnodes}} \, .
\end{align*}
We implement this by leaving \inlinecode{openColors} empty and passing $\catindexof{\secnodes}$ as the \inlinecode{evidenceColorDict}, as a dictionary with keys by the \inlinecode{str} colors to the variables and values by the corresponding \inlinecode{int} indices.

Graphical illustrations can be generated by
\begin{lstlisting}
	engine.draw_factor_graph(coreDict)
\end{lstlisting}
where \inlinecode{coreDict} is a tensor network to be visualized.


\subsect{\bnencoding}
Encoding schemes are implemented in the subpackage \sprepresentation.



\sect{Subpackage \spengine}\label{sec:implementationEngine}

The \spengine subpackage is for the storage and numerical manipulation of tensors and tensor networks.
We organize the subpackage as the lowest layer of \tnreason, specializing in storage of Tensor Networks and performing the contractions.

\subsect{Cores}

\paragraph{Iterator based Core Initialization}
We orient on basis+ sparse tensor decomposition in the initialization of tensor cores, as discussed in detail in \charef{cha:sparseCalculus}.
An elementary basis+ tensor is specified by tuples
\begin{lstlisting}
	(value, posDict)
\end{lstlisting}
where \inlinecode{posDict} specifies the values to the variables, which do not have a trivial leg vector, and \inlinecode{value} a scalar scaling the basis vector.
Comparing with the notation of \charef{cha:sparseCalculus}, the keys of \inlinecode{posDict} correspond with $\variableset$, the values of \inlinecode{posDict} with $\catindexof{\variableset}$ and \inlinecode{value} corresponds with $\slicescalar$.

A basis+ $\cpformat$ tensor is specified by an iterator \inlinecode{sliceIterator} over elementary basis+ tensors, where the $\cpformat$ rank is the length of the iterator
Given such a representation a tensor is instantiated by
\begin{lstlisting}
	engine.create_from_slice_iterator(shape, colors, sliceIterator, coreType, name)
\end{lstlisting}
where \inlinecode{shape, colors, coreType, name} are used in the call of an empty core by \inlinecode{engine.get_core} and \inlinecode{sliceIterator} used to iterative add the basis+ elementary tensors to create the tensor.

\paragraph{Core Arithmetics}
When executing
\begin{lstlisting}
	exampleCore[posDict] = value
\end{lstlisting}
we add a basis+ tensor specified by \inlinecode{posDict}, that is
\begin{align*}
	\hypercoreat{\shortcatvariables} \algdefsymbol \hypercoreat{\shortcatvariables} + \contractionof{\onehotmapofat{\catindexof{\variableset}}{\catvariableof{\variableset}}}{\shortcatvariables} \, .
\end{align*}
The linear structure of tensors spaces are reflected in sums of tensors implemented with the same \inlinecode{coreType}, as
\begin{lstlisting}
	summed = exampleCore1 + exampleCore2
\end{lstlisting}
and scalar multiplication, where a scalar \inlinecode{value} of type \inlinecode{int} or \inlinecode{float}
\begin{lstlisting}
	multiplied = value * exampleCore
\end{lstlisting}
Both operations are performed as manipulations of the tensors \inlinecode{values}.
Contraction of two cores
\begin{lstlisting}
	contracted = exampleCore1.contract_with(exampleCore2)
\end{lstlisting}
these are especially used in corewise contraction, where \inlinecode{contractionMethod="CoreWiseContractor"}.


\subsect{Contractions}

\textbf{Cores}

%Each Tensor core has attributes
%\begin{itemize}
%	\item values (array-like): storing the value of the coordinates
%	\item colors (list of str): specifying the name of the variables represented by its axes
%	\item name (str): to distinguish from other cores
%\end{itemize}
%The implemented core types differ in the values argument.


\textbf{Polynomial Cores}
Polynomial Cores are implementations of the monomial decomposition or basis+ (see \defref{def:polynomialSparsity}).
Here the each tuple $(\lambda,\variableset,\catvariableof{\variableset})$ is stored as a tuple of the scalar $\lambda$ and a dictionary with $\variableset$ as keys and $\catvariableof{\variableset}$ as values.

\red{The spare cores (Polynomial and Pandas Core) exploit the matrix representation of \remref{rem:matSotrageBasPlus}.}

% Contraction Method List
The supported cores are
\begin{center}
\begin{tabular}{|c|c|c|}
  	\hline
 	\textbf{coreType} & \textbf{Package} & \text{Explanation}  \\
  	\hline
 	\stringof{NumpyTensorCore} 	&  $\mathrm{numpy}$  & Numpy array storing the values\\
  	\hline
 	\stringof{PolynomialCore} 	&  $\mathrm{numpy}$  & Storing the values in a basis+ $\cpformat$ Decomposition\\
  	\hline
\end{tabular}
\end{center}


\textbf{Binary CP Decomposition}

Based on the monomial decomposition $\slicesparsityof{\cdot}$ as specified in \defref{def:polynomialSparsity}.
To store the values of a tensor we store the slices of tensors by the indices $\catindexof{\variableset}$. 

% Trick -> To BinaryCP
Contractions can be performed by partially contracting the cores of the decomposition.
In this way, one can avoids coordinatewise storages of high-order tensors, which can be intractable.

\textbf{Tensor Networks}

Tensor networks $\tnetof{\graph}$ are defined by hypergraphs with hyperedges decorated by tensor cores. 
We store them by dictionaries with values being tensor cores and keys coinciding with the name of each tensor core.


% Contraction Method List
The supported contraction methods are
\begin{center}
\begin{tabular}{|c|c|c|}
  	\hline
 	\textbf{contractionMethod} (str) & \textbf{Package} & \text{Explanation}  \\
  	\hline
 	\stringof{NumpyEinsum} 	&  $\mathrm{numpy}$  & Einstein summation of $\mathrm{numpy}$ arrays\\
  	\hline
 	\stringof{TensorFlowEinsum} 	&  $\mathrm{tensorflow}$  & Einstein summation of $\mathrm{tensorflow}$ tensors\\
  	\hline
	\stringof{TorchEinsum} 	&  $\mathrm{torch}$  & Einstein summation of $\mathrm{torch}$ tensors\\
  	\hline
	\stringof{TentrisEinsum} 	&  $\mathrm{tentris}$  & Einstein summation of $\mathrm{tentris}$ hypertries\\
  	\hline
	\stringof{PgmpyVariableEliminator} 	&  $\mathrm{pgmpy}$  & Variable Elimination of DiscreteFactors in $\mathrm{pgmpy}$\\
  	\hline
	\stringof{CorewiseContractor} 	&  $\mathrm{numpy}$  & Contraction of CP Decompositions stored in $\mathrm{numpy}$ arrays\\
  	\hline	
\end{tabular}
\end{center}


%%\textbf{Einstein Summation}
%Contractions represented as Einstein summation, as implemented in:
%\begin{itemize}
%	\item numpy
%	\item tensorflow
%	\item pytorch
%	\item tentris
%\end{itemize}

%\textbf{Variable Elimination}
%Contractions can be executed by variable elimination as implemented in:
%\begin{itemize}
%	\item pgmpy
%\end{itemize}

%\textbf{Manipulation of Binary CP Decomposition}
%Contraction of tensors in Binary CP Decomposition as in \secref{sec:BinaryCPManipulation}.







\sect{Subpackage \sprepresentation}\label{sec:implementationRepresentation}

The \sprepresentation subpackage consists in a collection of core creation methods.

Here the relational encodings $\rencodingof{\exfunction}$ of various maps $\exfunction$ are created.


We arrange the \sprepresentation subpackage into the second layer of the \tnreason architecture, since it specifies tensor cores which formats are specified in \spengine.




\textbf{Coordinate Calculus}

Main function
\begin{lstlisting}
	engine.coordinatewise_transform(coresList, transformFunction)
\end{lstlisting}

\textbf{Basis Calculus}

Main function
\begin{lstlisting}
	engine.relational_encoding()
\end{lstlisting}
basis calculus then based on contractions.







\subsect{Refinement by infixes}

Both the cores and the colors are further refined by infixes before the suffices to denote specific instantiations.

\begin{itemize}
	\item \selCoreIn: Involving a selection variable
	\item \eviCoreIn: Storing evidence about a variable
	\item \heaIn: Head of a function, typically the variable computed at a activation selector
	\item \funIn: Function selection variables
	\item \posIn+\stringof{i}: Variable selection for argument at position $i$
	\item \datIn: Involving data (data cores and colors)
\end{itemize}

Further infixes are strings denoting atom names and neuron neames.


\subsect{Relational encoding of formulas} % -> To application!

Propositional formulas $\exformula$ are represented in three schemes:
\begin{itemize}
	\item Script language $\synencodingof{\exformula}$ by nested lists (see \secref{subsec:scriptLanguage}).
		Most practical to choose a formula from a neuro-symbolic architecture.
	\item Strings specifying the categorical variables $\catvariableof{\exformula}$.
	\item Representation of formulas by tensor networks being contracted to $\rencodingof{\exformula}$
\end{itemize}

Conversions of the formats:
\begin{itemize}
	\item $\synencodingof{\exformula}$ to color by
		\begin{lstlisting}
			representation.get_formula_color($\synencodingof{\exformula}$)
		\end{lstlisting}
		Here the nested lists are turned in a string by concatenating all elements of a list with \stringof{\_} and adding \stringof{[} and \stringof{]} at the beginning and end of each list.
	\item  $\synencodingof{\exformula}$ to tensor network
		\begin{lstlisting}
			representation.create_raw_cores($\synencodingof{\exformula}$)
		\end{lstlisting}
		This creates the connective cores for the semantic representation of $\rencodingof{\exformula}$.
We encode them by
\end{itemize}

When encoding formulas with hard interpretation, we furthermore add a head core of type \stringof{truthEvaluation} since we have
 	\[ {\exformula} = \sbcontractionof{\rencodingof{\exformula},\tbasis}{\catvariableof{\exformula}} \, . \]



\subsect{Representation of MLNs}

\textbf{Computation Cores} are binary cores relating the variables in a predefined way, which is not changing during reasoning.
\begin{itemize}
	\item Logical interpretation: Cores $\rencodingof{\exconnective}$ \red{Structure Cores are those of the Bayesian Propositional Network}
	\item Categorical constraints: Cores $\categoricalcore$
	%\item Data: Cores $\datacore$
\end{itemize}

\textbf{Activation Cores} encode the weights of the formulas in a Markov Logic Network.
%For proper MLN only have unary cores, which we call headCores.
%Head cores with suffix "headCore" in name.

They are modified during reasoning: Selection of activation cores in structure learning, assigning a weight in parameter estimation.



\subsect{Formula Selecting Networks}

Encoding of Neurons according to \defref{def:fsNeuron}:
\begin{itemize}
	\item Activation selection core with suffix \stringof{actCore} in name.
		 Selection by variable with suffix \stringof{actVar}
	\item Selection of neurons as arguments with suffix \stringof{selCore} in name.
		Each argument of each neuron comes with a control variable with suffix \stringof{selVar}.
\end{itemize}

Encoding of Formula Selecting Neural Networks (\defref{def:fsNeuron}) by creating all formula selecting neurons.

Skeleton expression (\defref{def:skeleton}) are stored with placeholderkeys and the candidatelists by dictionaries with the placeholderkeys and values being the possible symbols.



\sect{Subpackage \spreasoning}\label{sec:implementationReasoning}


The \spreasoning subpackage implements contraction-based reasoning algorithm on representation.ComputationActivationNetworks.
%basic tensor network algorithms with calls of specific execution in \spengine.
As the \sprepresentation subpackage it is arranged in the second layer of the \tnreason architecture, since it specifies the manipulation of tensor networks in the \spengine subpackage.

\subsect{Sampling}

Sampling is performed by MCMC methods calling local sampling methods.

\begin{itemize}
	\item Tensor Network of Structure Cores
	\item Parameter cores: Variable tensor network cores representing basis vectors.
	\item List of importance cores
\end{itemize}

\begin{centeredcode}
	reasoning.Gibbs
\end{centeredcode}

\subsect{Energy-based Algorithms}

These algorithms execute reasoning tasks solely on energy dictionaries, which are created by \inlinecode{representation.ComputationActivationNetwork.get_energy_dict()}.

\begin{centeredcode}
	reasoning.NaiveMeanField
\end{centeredcode}

\begin{centeredcode}
	reasoning.GenericMeanField
\end{centeredcode}

\begin{centeredcode}
	reasoning.EnergyBasedGibbs
\end{centeredcode}




\sect{Subpackage \spapplication}\label{sec:implementationApplication}

With the \spapplication subpackage we provide an interface for reasoning workload.
It builds a third layer, since it used \sprepresentation to represent knowledge by tensor networks and \spreasoning in the execution of reasoning tasks.
%
To have a user-friendly high-level syntax of tensor-network creation a the script language (logical formulas or neuro-symbolic architectures), categorical constraints or data, is introduced.
Given a specification of a formula $\exformula$ in script language $\synencodingof{\cdot}$, the task amounts to building a semantic representation based on the syntactic specification.

\subsect{Script Language}\label{subsec:scriptLanguage}

To specify propositional sentences, neuro-symbolic architectures and Markov Logic Networks, we developed a script language.

\textbf{Propositional Sentences by Nested Lists}

%\textbf{Production Rules}
Are those of Propositional Logics, but instead of brackets we nest the symbols into lists.

% Connectives
\textbf{Connectives} are represented by strings, where the following are supported (see \defref{def:connectives}):
\begin{center}
\begin{tikzpicture}
\node [anchor=center] at (0,0) {
	\begin{tabular}{|c|c|}
  	\hline
 	\textbf{Unary connective $\exconnective$} & \textbf{$\synencodingof{\exconnective}$} \\
  	\hline
 	$\lnot$ 	&  \stringof{not} \\
  	\hline
 	$()$		&  \stringof{id} \\
  	\hline
	\end{tabular}};
\node [anchor=center] at (7,0) {
	\begin{tabular}{|c|c|}
  	\hline
 	\textbf{Binary connective $\exconnective$} & \textbf{$\synencodingof{\exconnective}$} \\
  	\hline
 	$\land$ 		&  \stringof{and} \\
  	\hline
 	$\lor$ 		&  \stringof{or} \\
  	\hline
 	$\Rightarrow$ 	&  \stringof{imp} \\
  	\hline
	 $\oplus$ 		&  \stringof{xor} \\
  	\hline
	 $\Leftrightarrow$ &  \stringof{eq} \\
  	\hline
	\end{tabular}};
\end{tikzpicture}
\end{center}

% WOLFRAM Numbers
Besides these specific connectives we exploit a generic representation scheme of propositional formulas by the so-called Wolfram code orginially designed for the classification of cellular automaton rules \cite{wolfram_statistical_1983} and popularized in the book \cite{wolfram_new_2002}.
Along this, the coordinate encodings of connectives $\exconnective$ with differing arity are flattened and interpreted as a binary number, which is transformed into a decimal number and represented as a string $\synencodingof{\exconnective}$.
We then choose a prefix to encode the arity by
\begin{itemize}
	\item \stringof{u} for unary
	\item \stringof{b} for binary
	\item \stringof{t} for ternary
	\item \stringof{q} for quarternary
\end{itemize}
connectives.
Together, the connective is represented by the string concatenation
	\[  \synencodingof{\exconnective} = \synencodingof{\catorder} + \synencodingof{\exconnective} \, . \]


% Atoms
\textbf{Atomic Formulas} are represented by arbitrary strings, which are not used for the representation of connectives.
We further avoid the symbols \{\stringof{(}, \stringof{)}, \stringof{\_}\} in the names of atoms, to not confuse them with colors of categorical variables.

% Composed Formulas
\textbf{Composed Formulas} $\exformula_1\exconnective,\exformula_2$ are represented by
\begin{centeredcode}
	$\synencodingof{\exformula_1\exconnective,\exformula_2}$ = [$\synencodingof{\exconnective}$, $\synencodingof{\exformula_1}$, $\synencodingof{\exformula_2}$]
\end{centeredcode}
where we apply the conventions
\begin{itemize}
	\item Connectives are at the 0th position in each list
	\item Further entries are either atoms as strings or encoded formulas itself
\end{itemize}

% Backus-Naur
The applied grammar in Backus-Naur form is \\
\begin{tabular}{|l|l|}
  	\hline
 	Unary Connective & \stringof{not} | \stringof{id}\\
  	\hline
 	Binary Connective & \stringof{and} | \stringof{or} | \stringof{imp} | \stringof{xor}  | \stringof{eq} \\
  	\hline
 	Atomic Formula & Set of strings not in Connectives\\
  	\hline
	Complex Formula & Atomic Formula | [Unary Connective, Complex Formula] | \\
	&  [Binary Connective, Complex Formula, Complex Formula] \\
	\hline
\end{tabular}


\begin{example}[Encoding of the Wet Street example]
For example we have
\begin{itemize}
	\item
	Atomic variable $\var{Rained}$ by
		\begin{centeredcode}
			$\synencodingof{\var{Rained}}$
			= \stringof{Rained}
		\end{centeredcode}
	\item
	Negative literal $\lnot\var{Rained}$ by
		\begin{centeredcode}
			$\synencodingof{\lnot\var{Rained}}$
			= [\stringof{not},\stringof{Rained}]
		\end{centeredcode}
	\item
	Horn clause $\left(\var{Rained}\Rightarrow\var{Wet}\right)$ by
		\begin{centeredcode}
			$\synencodingof{\var{Rained}\Rightarrow\var{Wet}}$
			= [\stringof{imp},\stringof{Rained},\stringof{Wet}]
		\end{centeredcode}
	\item
	Knowledge Base
	$(\lnot\var{Rained})\land(\var{Rained}\Rightarrow\var{Wet})$ by
		\begin{centeredcode}
			$\synencodingof{\lnot\var{Rained})\land(\var{Rained}\Rightarrow\var{Wet}}$
			=  [\stringof{and}, [\stringof{not}, \stringof{Rained}], [\stringof{imp}, \stringof{Rained}, \stringof{Wet}]]
		\end{centeredcode}
\end{itemize}
\end{example}




\textbf{Knowledge Bases}

% Should distinguish these in knowledge?
We distinguish here formulas, with propositional logic interpretation and formulas which have a soft logic interpretation.
%\textbf{Facts.}
The formulas with hard interpretation are called facts in a knowledge base $\kb$ and encoded by dictionaries
\begin{centeredcode}
	\{key($\exformula$) : $\synencodingof{\exformula}$ for $\exformula\in\kb$ \}
\end{centeredcode}

\textbf{Markov Logic Networks}

%\textbf{Weighted formulas.}
The formulas with soft interpretation are called weighted formulas and encoded by $\expof{\weightof{\exformula}\cdot\exformula}$.
We thus require a specification of the weights, which we do by adding $\weightof{\exformula}$ as a $\mathrm{float}$ or an $\mathrm{int}$ to the list $\synencodingof{\exformula}$.
We then store Markov Logic Networks by dictionaries
\begin{centeredcode}
	\{key($\exformula$) : $\synencodingof{\exformula}$ + [$\weightof{\exformula}$] for $\exformula\in\formulaset$\}
\end{centeredcode}

\textbf{Neuro-Symbolic Architecture by Nested Lists}

% Generalizing the script language to specify architectures
To specify neuro-symbolic architectures in terms of formula selecting maps, as has been the subject of \charef{cha:formulaSelection} we further exploit the nested list structure of encoding propositional logics.
We replace, in each hierarchy of the nested structure each entry by a list of possible choices.
In this way, we reinterpret the list index as the choice indices $\selindex$ introduced for connective and formula selections (see \defref{def:connectiveSelector} and \ref{def:formulaSelector}).

% Neurons
A connective selector (see \defref{def:connectiveSelector}) is encoded by the list
	\begin{centeredcode}
			$\synencodingof{\exconnective}$
			= [$\synencodingof{\exconnective_{0}}$, $\ldots$, $\synencodingof{\exconnective_{\seldim\shortminus1}}$]
	\end{centeredcode}
and a formula selector (see \defref{def:formulaSelector}) by
	\begin{centeredcode}
			$\synencodingof{\fselectionmap}$
			= [$\synencodingof{\exconnective_{0}}$, $\ldots$, $\synencodingof{\exconnective_{\seldim\shortminus1}}$]
	\end{centeredcode}
A logical neuron of order $\selorder$ (see \defref{def:fsNeuron}), defined by a connective selector $\exconnective$, and a formula selector $\fselectionmap_\atomenumerator$ on each argument $\atomenumerator\in[\selorder]$, is encoded by
		\begin{centeredcode}
			$\synencodingof{\lneuron}$
			= [$\synencodingof{\exconnective}$, $\synencodingof{\fselectionmap_0}$, $\ldots$,  $\synencodingof{\fselectionmap_{\selorder-1}}$]
		\end{centeredcode}
Only the unary $\selorder=1$ and the $\selorder=2$ cases are supported.


% Confusing?
The resulting nested lists indices have an alternating interpretation at each level compared with the elements of each list.
That is, when $\synencodingof{\lneuron}$ is the encoding of a neuron, then any element $x\in\synencodingof{\lneuron}$ represents a list of choices.
When $x$ is not the first element, then each choice is either the encoding $\synencodingof{\catvariable}$ of an atomic formula, or another neuron.

% Find another symbol?
A neural architecture $\larchitecture$ is then represented in the dictionary
\begin{centeredcode}
	$\synencodingof{\larchitecture}$ = \{key($\lneuron$) : $\synencodingof{\lneuron}$ for $\exformula\in\larchitecture$\}
\end{centeredcode}
%To store this structure, we choose dictionaries of neuron spe
%\begin{centeredcode}
%	\{key($\lneuron$) : $\synencodingof{\lneuron}$ for $\exformula\in\formulaset$\}
%\end{centeredcode}
where key($\lneuron$) is a string, which can be used in the formula selections of other neurons.

% Important for well-definedness
It is important that the directed graph of neurons induced by the choice possibilities is acyclic, to ensure well-definedness of the architecture.


% Backus-Naur
In order to represent neuro-symbolic architectures, the grammar of $\synencodingof{\cdot}$ in Backus-Naur Form is extended by the production rules \\
\begin{tabular}{|l|l|}
  	\hline
 	Unary Connectives & [Unary Connective] | [Unary Connective] + Unary Connectives \\
  	\hline
 	Binary Connectives & [Unary Connective] | [Binary Connective] + Binary Connectives \\
	%\hline
	%Neuron Name & Any set of strings not used for atoms or connectives \\
  	\hline
 	Dependency Choice & Atomic Formula | Neuron \\
  	\hline
	Dependency Choices & [Dependency Choice] | [Dependency Choice] + Dependency Choices \\
	\hline
	Neuron & [Unary Connectives, Dependency Choices] | \\
	&  [Binary Connectives, Dependency Choices, Dependency Choices] \\
	\hline
\end{tabular}


\begin{example}[Neuro-Symbolic Architecture for the Wet Street]
	Following the wet street example, we can define a neuron by
	\begin{centeredcode}
		$\synencodingof{\lneuron}$ = [[\stringof{imp},\stringof{eq}],[\stringof{Wet},\stringof{Sprinkler}],[\stringof{Street}]]
	\end{centeredcode}
	from which the formulas
	\begin{centeredcode}
		[\stringof{imp}, \stringof{Wet}, \stringof{Street}] \\
		\hspace{0.25cm} [\stringof{eq}, \stringof{Wet}, \stringof{Street}] \\
		\hspace{1cm}[\stringof{imp}, \stringof{Sprinkler}, \stringof{Street}] \\
		\hspace{1cm}[\stringof{eq}, \stringof{Sprinkler}, \stringof{Street}]
	\end{centeredcode}
	can be chosen.
	Combining this neuron with further neurons, e.g. by the architecture
	\begin{centeredcode}
		$\synencodingof{\larchitecture}$ = \{ \stringof{neur1}: [[\stringof{imp},\stringof{eq}],[\stringof{neur2}],[\stringof{Street}]] , \\
		\hspace{1.8cm}\stringof{neur2}: [[\stringof{lnot},\stringof{id}],[\stringof{Wet},\stringof{Sprinkler}],[\stringof{Street}]] \}
	\end{centeredcode}
	the expressitivity increases.
	In this case, the further neuron provides the flexibility of the first atoms to be replaced by its negation.
\end{example}



\subsect{Distributions}

We encode Markov Networks by specifying a set of tensor cores.
Each distribution needs to have a routine
\begin{lstlisting}
	.create_cores()
\end{lstlisting}
creating the factor cores and 
\begin{lstlisting}
	.get_partition_function()
\end{lstlisting}
calculating the partition function.
Although the partition function can be calculated by the contraction of all cores, we separate the method since there are situations where a faster calculation can be performed.


\textbf{Empirical Distributions} are distributions of sample data.
We represent the values as a CP Format of data cores as specified in \secref{sec:empDistribution}
\begin{lstlisting}
	knowledge.EmpiricalDistribution
\end{lstlisting}
Here the partition function is the number of samples used in the creation of the empirical distribution.


\textbf{HybridKnowledgeBases} are probability distributions, which are specified by propositional formulas in the script language.
\begin{lstlisting}
	knowledge.HybridKnowledgeBase
\end{lstlisting}
They are initialized with arguments
\begin{itemize}
	\item facts: Dictionary of propositional formulas stored as $\synencodingof{\exformula}$ representing hard logical constraints
	\item weightedFormulas: Dictionary of propositional formulas stored as $\synencodingof{\exformula}$+$[\weightof{\exformula}]$ representing soft logical constraints
	\item evidence: Dictionary of atomic formulas, where key are the formulas in string representation and values the certainty in $[0,1]$ (float or int) of the atom being true
	\item categoricalConstraints: Dictionary of categorical constrained, which values are lists of atomic formulas stored as strings $\synencodingof{\atomicformula}$
\end{itemize}


\subsect{Inference}

By
\begin{lstlisting}
	knowledge.InferenceProvider
\end{lstlisting}
taking a distribution from the above as argument.

% Probabilistic Queries
Probabilistic queries as specified \defref{def:queries})  by
\begin{lstlisting}
	.query(variableList, evidenceDict)
\end{lstlisting}

MAP queries by
\begin{lstlisting}
	.exact_map_query()
\end{lstlisting}
or by
\begin{lstlisting}
	.annealed_sample()
\end{lstlisting}
using Simulated Annealing (see Remark~\ref{rem:simulatedAnnealing}) to find an approximate maximum.
The second method circumvents the creation of the coordinatewise representation of the distribution and circumvents therefore, at the expense of potentially approximative solutions, a bottleneck in case of many query variables.

% Entailment Queries
Entailment from the distribution (\defref{def:entailment}) is be decided by
\begin{lstlisting}
	.ask(queryFormula, evidenceDict)
\end{lstlisting}
where queryFormula is the formula $\exformula$ to be tested for entailment in the representation $\synencodingof{\exformula}$.

% Sampling
Samples can be drawn by
\begin{lstlisting}
	.draw_samples(sampleNum, variableList, annealingPattern)
\end{lstlisting}
based on Gibbs sampling, where
\begin{itemize}
	\item sampleNum (int) gives the number of samples to be drawn
	\item variableList (list of str) defines the variables to be represented by the samples (default: all atoms in the distribution)
	\item annealingPattern specifies an annealing pattern 
\end{itemize}


\subsect{Parameter Estimation}

\textbf{EntropyMaximizer} implements Algorithm~\ref{alg:AWO}, which is motivated by the maximum entropy principle (see \secref{sec:maxEntDuality}) to optimize Markov Logic Networks.
The class  
\begin{lstlisting}
	knowledge.EntropyMaximizer
\end{lstlisting}
is initialized with the arguments
\begin{itemize}
	\item expressionsDict: Dictionary of formulas in the format $\synencodingof{\exformula}$ 
	\item satisfactionDict: Dictionary of the satisfaction rates (mean parameters) to be matched by the optimal distribution
\end{itemize}
The optimization is then performed by
\begin{lstlisting}
	.alternating_optimization(sweepNum, updateKeys)
\end{lstlisting}
method, where the iteration in Algorithm~\ref{alg:AWO} through the updateKeys is performed sweepNum times.

\subsect{Structure Learning}

\textbf{Formula Booster} chooses a formula given a formula selecting map.
\begin{lstlisting}
	knowledge.FormulaBooster
\end{lstlisting}
is initialized with the arguments
\begin{itemize}
	\item knowledgeBase: Distribution representing a current model to be improved
	\item specDict: A neuro-symbolic architecture encoded in a dictionary of neurons
\end{itemize}




\bibliographystyle{plainnat}
\bibliography{literature/tensor_logic}


\end{document}