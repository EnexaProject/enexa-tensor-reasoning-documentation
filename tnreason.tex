\documentclass[aps,onecolumn,nofootinbib,pra]{article}

\usepackage{arxiv}
\usepackage{amsmath,amsfonts,amssymb,amsthm,bbm,graphicx,enumerate,times}
\usepackage{mathtools}
\usepackage[usenames,dvipsnames]{color}
\usepackage{hyperref}
\hypersetup{
	breaklinks,
	colorlinks,
	linkcolor=gray,
	citecolor=gray,
	urlcolor=gray,
	pdftitle={The Tensor Network Approach to Efficient and Explainable AI},
	pdfauthor={Alex Goessmann}
}

\usepackage{tikz}
\usepackage{graphicx}
\usepackage{float}
\usepackage{comment}
\usepackage{csquotes}

\usepackage{listings}
\usepackage{verbatim}
\usepackage{etoolbox}
\usepackage{braket}
\usepackage[utf8]{inputenc}
\usepackage[english]{babel}
\usepackage[T1]{fontenc}
\usepackage{amsmath}
\usepackage{amsfonts}
\usepackage{amssymb}
\usepackage{amsthm}
\usepackage{titlesec}
\usepackage{tikz}
\usepackage{mathtools}
\usepackage{fancyhdr}
\usepackage{bbm}
\usepackage{bm}
\usepackage{algpseudocode}
\usepackage{algorithm}
\usepackage{lipsum}

\newtheorem{remark}{Remark}
\newtheorem{theorem}{Theorem}
\newtheorem{lemma}{Lemma}
\newtheorem{corollary}{Corollary}
\newtheorem{definition}{Definition}
\newtheorem{example}{Example}

\newcommand{\var}[1]{\text{\emph{#1}}}

\newcommand{\synencodingof}[1]{S\left(#1\right)} % Syntax encoding!
\newcommand{\stringof}[1]{"#1"}

\newcommand{\rdf}{\mathrm{RDF}}
\newcommand{\mathrdftype}{\mathrm{rdf}\mathrm{type}}
\newcommand{\rdftype}{$\mathrm{rdf}:\mathrm{type}$}

\newcommand{\truesymbol}{\mathrm{True}}
\newcommand{\falsesymbol}{\mathrm{False}}
\newcommand{\truthset}{\{\falsesymbol,\truesymbol\}}
\newcommand{\truthstate}{z}
\newcommand{\truthstateof}[1]{\truthstate_{#1}}
\newcommand{\ozset}{\{0,1\}}
\newcommand{\ozbasisset}{\{\fbasisat{\catvariable},\tbasisat{\catvariable}\}}

\newcommand{\uniquantwrtof}[2]{\forall{#1}:{#2}}
\newcommand{\imppremhead}[2]{\left(#1\right)\Rightarrow\left(#2\right)}

\newenvironment{centeredcode}
{\begin{center}
     \begin{algorithmic}
         \hspace{1cm}}
{\end{algorithmic}\end{center}} % Use for tnreason script language, lstlistings for python code!

\newcommand{\algdefsymbol}{\leftarrow}
\newcommand{\proofrightsymbol}{"$\Rightarrow$"}
\newcommand{\proofleftsymbol}{"$\Leftarrow$"}

\newcommand{\distassymbol}{\sim}
\newcommand{\probtagtypeinst}[2]{\mathrm{P}^{#1}_{#2}}

\newcommand{\skeletoncolor}{blue}
\newcommand{\probcolor}{red}
\newcommand{\concolor}{blue}

\newcommand{\conjunctioncolor}{red}
\newcommand{\negationcolor}{blue}
\newcommand{\nodeminsize}{0.8cm}
\newcommand{\nodegrayscale}{gray!50}

% Basic Symbols
\newcommand{\entropysymbol}{\mathbb{H}}
\newcommand{\sentropyof}[1]{\entropysymbol\left[#1\right]}
\newcommand{\centropyof}[2]{\entropysymbol\left[#1,#2\right]}

\newcommand{\subspacedimof}[1]{\mathrm{dim}(#1)}

\newcommand{\subsphere}{\mathcal{S}}
\newcommand{\rr}{\mathbb{R}}
\newcommand{\nn}{\mathbb{N}}

\newcommand{\closureof}[1]{\overline{#1}}
\newcommand{\interiorof}[1]{{#1}^{\circ}}
\newcommand{\sbinteriorof}[1]{{\left(#1\right)}^{\circ}}

\newcommand{\difofwrt}[2]{\frac{\partial #1}{\partial #2}}
\newcommand{\difwrt}[1]{\difofwrt{}{#1}}
\newcommand{\gradwrt}[1]{\nabla_{#1}}
\newcommand{\gradwrtat}[2]{\nabla_{#1}|_{#2}}

\newcommand{\cardof}[1]{\left|#1\right|}
\newcommand{\absof}[1]{\left|#1\right|}

\newcommand{\imageof}[1]{\mathrm{im}\left(#1\right)}

\newcommand{\convhullof}[1]{\mathrm{conv}\left(#1\right)}
\newcommand{\cubeof}[1]{[0,1]^{#1}}
\newcommand{\dimof}[1]{\mathrm{dim}\left(#1\right)}
\newcommand{\spanof}[1]{\mathrm{span}\left(#1\right)}
\newcommand{\subspaceof}[1]{V^{#1}}

\newcommand{\argmin}{\mathrm{argmin}}
\newcommand{\argmax}{\mathrm{argmax}}

% Help functions
\newcommand{\chainingfunction}{h}
\newcommand{\chainingfunctionof}[1]{\chainingfunction\left(#1\right)}

\newcommand{\greaterthanfunction}[1]{\ones_{>#1}}
\newcommand{\greaterthanfunctionof}[2]{\greaterthanfunction{#1}\left(#2\right)}
\newcommand{\existquanttrafo}{\greaterthanfunction{0}}
\newcommand{\universalquanttrafo}{\greaterthanfunction{\inddim-1}}

\newcommand{\greaterzerofunction}{\greaterthanfunction{0}}
\newcommand{\greaterzeroof}[1]{\greaterzerofunction\left(#1\right)}



\newcommand{\nonzerofunction}{\ones_{\neq0}}
\newcommand{\nonzeroof}[1]{\nonzerofunction\left(#1\right)}
\newcommand{\nonzerocirc}{\nonzerofunction\circ}

% Probability distributions
\newcommand{\expof}[1]{\mathrm{exp}\left[#1\right]}
\newcommand{\probtensor}{\mathbb{P}}
\newcommand{\probtensorof}[1]{\probtensor^{#1}}
\newcommand{\probtensorofat}[2]{\probtensor^{#1}\left[#2\right]}
\newcommand{\secprobtensor}{\tilde{\mathbb{P}}}
\newcommand{\secprobat}[1]{\secprobtensor[#1]}

\newcommand{\probtensorset}{\Gamma}
\newcommand{\bmrealprobof}[1]{\expfamilyof{\identity,#1}} % distributions with support coinciding with the base measure in argument

\newcommand{\gendistribution}{\probtensor^*}
\newcommand{\gendistributionat}[1]{\gendistribution\left[#1\right]}
\newcommand{\currentdistribution}{\tilde{\probtensor}}

\newcommand{\probat}[1]{\probtensor\left[#1\right]}
\newcommand{\probof}[1]{\probtensor^{#1}}
\newcommand{\probofat}[2]{\probof{#1}\left[#2\right]}
\newcommand{\probwith}{\probat{\shortcatvariables}}
\newcommand{\probofwrt}[2]{\probtensor_{#1}\left[#2\right]}


\newcommand{\condprobat}[2]{\mathbb{P}\left[#1|#2\right]}
\newcommand{\condprobof}[2]{\condprobat{#1}{#2}}
\newcommand{\condprobwrtof}[3]{\mathbb{P}^{#1}\left[#2|#3\right]}
\newcommand{\margprobat}[1]{\probat{#1}}
\newcommand{\expectationof}[1]{\mathbb{E}\left[#1\right]}
\newcommand{\expectationofwrt}[2]{\mathbb{E}_{#2}\left[#1\right]}
\newcommand{\lnof}[1]{\ln \left[ #1 \right] }
\newcommand{\sgnormof}[1]{\left\|#1\right\|_{\psi_2}} % subgaussian
\newcommand{\normof}[1]{\left\|#1\right\|_{2}}

\newcommand{\distof}[1]{\mathbb{P}^{#1}}

\newcommand{\ones}{\mathbb{I}}
\newcommand{\onesof}[1]{\ones^{#1}}
\newcommand{\onesat}[1]{\ones\left[#1\right]}
\newcommand{\onesofat}[2]{\onesof{#1}\left[#2\right]}
\newcommand{\oneswith}{\onesat{\shortcatvariables}}
\newcommand{\zerosat}[1]{0\left[#1\right]}
\newcommand{\identity}{\delta}
\newcommand{\identityat}[1]{\identity\left[#1\right]}
\newcommand{\dirdeltaof}[1]{\delta^{#1}}
\newcommand{\dirdeltaofat}[2]{\dirdeltaof{#1}\left[#1\right]}

\newcommand{\deltaof}[1]{\delta_{#1}} % used in coordinate calculus proofs


\newcommand{\indicatorofat}[2]{\ones_{#1}\left[#2\right]}

\newcommand{\exmatrix}{M}
\newcommand{\matrixat}[1]{\exmatrix[#1]}
\newcommand{\matrixofat}[2]{\exmatrix^{#1}\left[#2\right]}

\newcommand{\exvector}{V}
\newcommand{\vectorof}[1]{\exvector^{#1}}
\newcommand{\vectorat}[1]{\exvector[#1]}
\newcommand{\vectorofat}[2]{\exvector^{#1}[#2]}

\newcommand{\restrictionofto}[2]{{#1}|_{#2}}
\newcommand{\restrictionoftoat}[3]{\restrictionofto{#1}{#2}\left[#3\right]}

\newcommand{\idsymbol}{\mathrm{Id}} % ! Different to delta tensor in \identity
\newcommand{\idrestrictedto}[1]{\restrictionofto{\idsymbol}{#1}}

%% KNOWLEDGE GRAPH
\newcommand{\kg}{\mathrm{KG}|_{\dataworld}}
\newcommand{\kgat}[1]{\kg\left[#1\right]}
\newcommand{\kgreptensor}{\rencodingof{\kg}}

\newcommand{\exaunaryrelation}{C}
\newcommand{\exabinaryrelation}{R}

\newcommand{\kgtriple}[3]{\braket{#1,#2,#3}}
\newcommand{\exunarytriple}{\kgtriple{\provariable}{\mathrdftype}{\exaunaryrelation}}
\newcommand{\exbinarytriple}{\kgtriple{\provariableof{0}}{\exabinaryrelation}{\provariableof{1}}}

\newcommand{\atomcreator}{\psi}
\newcommand{\atomcreatorofat}[2]{\atomcreator_{#1}\left[#2\right]}
\newcommand{\provariable}{Z}
\newcommand{\provariableof}[1]{\provariable_{#1}}

\newcommand{\sparql}{\mathrm{SPARQL}}
\newcommand{\joinsymbol}{\mathrm{JOIN}}

\newcommand{\subsymbol}{s}
\newcommand{\predsymbol}{p}
\newcommand{\objsymbol}{o}

\newcommand{\sindvariable}{\indvariableof{\subsymbol}}
\newcommand{\pindvariable}{\indvariableof{\predsymbol}}
\newcommand{\oindvariable}{\indvariableof{\objsymbol}}

\newcommand{\invrdftypesymbol}{\mathrm{typ}}



% Propositional Logics: New square bracket notation
\newcommand{\formula}{f}
\newcommand{\formulaof}[1]{\formula_{#1}}
\newcommand{\formulaat}[1]{\formula\left[#1\right]}
\newcommand{\formulaofat}[2]{\formulaof{#1}\left[#2\right]}



\newcommand{\formulavar}{\headvariableof{\formula}}
\newcommand{\formulacc}{\rencodingof{\formula}} % computation core
\newcommand{\formulaccwith}{\rencodingofat{\formula}{\formulavar,\shortcatvariables}}

\newcommand{\enumformula}{\formulaof{\selindex}}
\newcommand{\enumformulaat}[1]{\enumformula\left[#1\right]}

\newcommand{\enumformulavar}{\headvariableof{\selindex}}

\newcommand{\enumformulacc}{\rencodingof{\enumformula}} % computation core
\newcommand{\enumformulaccwith}{\rencodingofat{\enumformula}{\enumformulavar,\shortcatvariables}}
\newcommand{\enumformulaac}{\actcoreof{\enumformula,\canparamat{\indexedselvariable}}}
\newcommand{\enumformulaacwith}{\actcoreofat{\enumformula,\canparamat{\indexedselvariable}}{\headvariableof{\enumformula}}}

\newcommand{\exformula}{\formula}
\newcommand{\exformulavar}{\headvariableof{\exformula}}
\newcommand{\exformulaat}[1]{\exformula\left[#1\right]}

\newcommand{\formulazerocoordinates}{\shortcatindices\,:\,\formulaat{\shortcatindices}=0}
\newcommand{\formulaonecoordinates}{\shortcatindices\,:\,\formulaat{\shortcatindices}=1}

\newcommand{\secexformula}{h} % Since g is atom
\newcommand{\secexformulavar}{\headvariableof{\secexformula}}
\newcommand{\secexformulaat}[1]{\secexformula\left[#1\right]}

\newcommand{\exformulain}{\exformula\in\formulaset}
\newcommand{\exformulaof}[1]{\exformula\left(#1\right)}

\newcommand{\formulasuperset}{\mathcal{H}}

% First order Logics
\newcommand{\folexformula}{q}
 % When representing \folexformula as \importancequery \rightarrow \headfolexformula
\newcommand{\folformulaset}{\mathcal{Q}}
\newcommand{\folexformulain}{\folexformula\in\folformulaset}
\newcommand{\folexformulaof}[1]{\folexformula_{#1}}
\newcommand{\restfolformulaset}{\restrictionofto{\folformulaset}{\worlddomain}}

\newcommand{\enumfolformula}{\folexformulaof{\selindex}}
\newcommand{\enumfolformulaat}[1]{\enumfolformula\left[#1\right]}

\newcommand{\headfolformula}{h}
\newcommand{\headfolexformula}{\headfolformula}
\newcommand{\headfolformulaof}[1]{\headfolformula_{#1}}
\newcommand{\headfolformulaofat}[2]{\headfolformulaof{#1}\left[#2\right]}

\newcommand{\folpredicate}{g}
\newcommand{\folpredicateof}[1]{\folpredicate_{#1}}
\newcommand{\folpredicates}{\folpredicateof{0},\ldots,\folpredicateof{\folpredicateorder-1}}
\newcommand{\folpredicateenumerator}{\atomenumerator} % Due to PL being a special case
\newcommand{\folpredicateorder}{\atomorder}

\newcommand{\worlddomain}{\arbset} % Snce enumerated
\newcommand{\exindividual}{a}
\newcommand{\secindividual}{b}
\newcommand{\exindividualof}[1]{\exindividual_{#1}}

\newcommand{\atombasemeasure}{\nu}

\newcommand{\individuals}{\exindividualof{\indindexof{0}},\ldots,\exindividualof{\indindexof{\individualorder-1}}}
\newcommand{\individualsof}[1]{\exindividualof{0}^{#1},\ldots,\exindividualof{\individualorder-1}^{#1}} % Do not use, index already in individuals


%% Redundant to individual variables
\newcommand{\individualvariable}{\indvariable}
\newcommand{\individualvariableof}[1]{\indvariableof{#1}}
\newcommand{\individualvariables}{\indvariablelist}

\newcommand{\individualorder}{\indorder}
\newcommand{\individualenumerator}{\indenumerator}
\newcommand{\individualenumeratorin}{\indenumeratorin}

\newcommand{\variableindex}{\indindex}
\newcommand{\variableindexof}[1]{\indindexof{#1}}
\newcommand{\variableenumerator}{\indenumerator}
\newcommand{\variableorder}{\indorder}
\newcommand{\variableenumeratorin}{\indenumeratorin}
\newcommand{\variableindices}{\indindexof{0}\ldots\indindexof{\indorder-1}}

\newcommand{\exconnective}{\circ}
\newcommand{\connectiveof}[1]{\exconnective_{#1}}
\newcommand{\connectiveofat}[2]{\connectiveof{#1}\left[#2\right]}

\newcommand{\folworldsymbol}{W}

\newcommand{\dataworld}{\catindexof{\folworldsymbol}}
\newcommand{\dataworldat}[1]{\dataworld[#1]}
\newcommand{\dataworldwith}{\dataworldat{\selvariable,\shortindvariables}}

\newcommand{\randworld}{\catvariableof{\folworldsymbol}}
\newcommand{\indexedrandworld}{\indexedcatvariableof{\folworldsymbol}}


\newcommand{\groundingofatwrt}[3]{{#1}|_{#3} \left[#2\right]}
\newcommand{\groundingofat}[2]{{#1}|_{\dataworld} \left[#2\right]}
\newcommand{\groundingof}[1]{{#1}|_{\dataworld}}
\newcommand{\kggroundingof}[1]{{#1}|_{\dataworld}}
\newcommand{\kggroundingofat}[2]{\kggroundingof{#1}\left[#2\right]}


% Used in FOL Models
%\newcommand{\gtensor}{\rho} % For decompositions
%\newcommand{\gtensorof}[1]{\gtensor^{#1}}

%% For the TCalculus Theorem
\newcommand{\coordinatetrafo}{\chainingfunction}
\newcommand{\gentensor}{T}
\newcommand{\basisslices}{U}

% Parameters 
\newcommand{\candidatelist}{\mathcal{M}}
\newcommand{\candidatelistof}[1]{\candidatelist^{#1}}

% Data Extraction Spec
\newcommand{\impformula}{p}
\newcommand{\fixedimpformula}{\underline{\impformula}}
\newcommand{\fixedimpformulawith}{\underline{\impformula}\left[\indvariableof{\impformula}\right]}
\newcommand{\fixedimpbm}{\basemeasureofat{\fixedimpformula}{\randworld}}
\newcommand{\supportedworlds}{\dataworld \, : \, \groundingofat{\impformula}{\shortindvariables} = \fixedimpformulawith}
\newcommand{\impformulaat}[1]{\impformula\left[#1\right]}

%\newcommand{\decgroundedimpformula}{\groundingof{\impformula}^{\mathrm{enum}}}


\newcommand{\extformula}{q}
\newcommand{\extformulaof}[1]{\extformula_{#1}}
\newcommand{\extformulaofat}[2]{\extformulaof{#1}\left[#2\right]}
\newcommand{\extformulas}{\extformulaof{0},\ldots,\extformulaof{\atomorder-1}}
\newcommand{\shortextformulas}{\extformulaof{[\atomorder]}}

\newcommand{\extractionrelation}{\exrelation}

\newcommand{\variableset}{A} % Still used in monomial decomposition, NOT for object sets!
\newcommand{\variablesetof}[1]{\variableset^{#1}}

\newcommand{\formulaset}{\mathcal{F}}
\newcommand{\formulasetof}[1]{\formulaset_{#1}}

\newcommand{\secformulaset}{\tilde{\formulaset}}

\newcommand{\hardformulaset}{\kb}
\newcommand{\hfbasemeasure}{\basemeasureof{\hardformulaset}}
\newcommand{\hfbasemeasureat}[1]{\hfbasemeasure\left[#1\right]}
\newcommand{\softformulaset}{\formulaset}


% Formula Selecting
\newcommand{\larchitecture}{\mathcal{A}}
\newcommand{\larchitectureat}[1]{\larchitecture\left[#1\right]}

\newcommand{\inneuronset}{\mathcal{A}^{\mathrm{in}}}
\newcommand{\outneuronset}{\mathcal{A}^{\mathrm{out}}}

\newcommand{\lneuron}{\sigma}
\newcommand{\lneuronof}[1]{\lneuron_{#1}}
\newcommand{\lneuronat}[1]{\lneuron\left[#1\right]}
\newcommand{\lneuractivation}{\lneuron^{\larchitecture}}
\newcommand{\lneuractivationat}[1]{\lneuractivation\left[#1\right]}

\newcommand{\fsnn}{\fselectionmapof{\larchitecture}}
\newcommand{\fsnnat}[1]{\fsnn\left[#1\right]}

\newcommand{\sliceselectionmapof}[1]{\fselectionmapof{\land,#1}}
\newcommand{\sliceselectionmapofat}[2]{\fselectionmapofat{\land,#1}{#2}}
\newcommand{\sliceselectionmapat}[1]{\sliceselectionmapofat{\catorder,\sliceorder}{#1}}

\newcommand{\skeleton}{S}
\newcommand{\skeletonof}[1]{\skeleton\left(#1\right)}
\newcommand{\skeletontensor}{\rencodingof{\skeleton}} %OLD! Use skeleton

\newcommand{\skeletoncore}{S}
\newcommand{\skeletoncoreof}[1]{\skeletoncore^{#1}}

\newcommand{\cselectionsymbol}{C}
\newcommand{\vselectionsymbol}{V}
\newcommand{\sselectionsymbol}{S}

\newcommand{\selinputvariable}{\selvariable}
\newcommand{\cselinputvariable}{\selvariableof{\cselectionsymbol}}
\newcommand{\vselinputvariable}{\selvariableof{\vselectionsymbol}}

\newcommand{\fselectionmap}{\mathcal{H}}
\newcommand{\fselectionmapof}[1]{\fselectionmap_{#1}}
\newcommand{\fselectionmapat}[1]{\fselectionmap\left[#1\right]}
\newcommand{\fselectionmapofat}[2]{\fselectionmap_{#1}\left[#2\right]}

\newcommand{\cselectionmap}{\fselectionmapof{\cselectionsymbol}}
\newcommand{\cselectionmapat}[1]{\fselectionmapofat{\cselectionsymbol}{#1}}

\newcommand{\vselectionmap}{\fselectionmapof{\vselectionsymbol}}
\newcommand{\vselectionmapat}[1]{\fselectionmapofat{\vselectionsymbol}{#1}}
\newcommand{\vselectionheadvar}{\headvariableof{\vselectionsymbol}} % Replacing \catvariableof{\vselectionmap}

\newcommand{\sselectionmap}{\fselectionmapof{\sselectionsymbol}}
\newcommand{\sselectionmapat}[1]{\fselectionmapofat{\sselectionsymbol}{#1}}

\newcommand{\vselectionmapof}[1]{\fselectionmapof{\vselectionsymbol,#1}} % tb deleted!

\newcommand{\tranfselectionmap}{\fselectionmap^T}

% Output variables - Following the catvariable convention
\newcommand{\seloutputvariable}{\randomx}
\newcommand{\cseloutputvariable}{\catvariableof{\cselectionsymbol}}
\newcommand{\vseloutputvariable}{\headvariableof{\vselectionsymbol}}

% Tensor Core Representation
\newcommand{\selectorcore}{{\rencodingof{\vselectionsymbol}}}
\newcommand{\selectorcoreof}[1]{\rencodingof{\vselectionmapof{#1}}}

\newcommand{\selectorcomponentof}[1]{\hypercoreof{\vselectionsymbol_{#1}}} % Since not an relational encoding!
\newcommand{\selectorcomponentofat}[2]{\selectorcomponentof{#1}\left[#2\right]}

\newcommand{\parspace}{\bigotimes_{\selenumeratorin}\rr^{\seldimof{\selenumerator}}}
\newcommand{\simpleparspace}{\rr^{\seldim}}

\newcommand{\unitvectoratof}[2]{e^{(#1)}_{#2}}
\newcommand{\parametrizingunittensor}{e_{\atomindices}} % Not required?

\newcommand{\placeholder}{Z} %% When not used in formulas, take the set for it
\newcommand{\placeholderof}[1]{\placeholder^{#1}}

\newcommand{\atomicformula}{\catvariable}
\newcommand{\atomicformulaof}[1]{\catvariableof{#1}}
\newcommand{\atomicformulaofat}[2]{\catvariableof{#1}\left[#2\right]}
\newcommand{\atomicformulas}{\catvariableof{[\atomorder]}} %{\{\atomicformulaof{\atomenumerator} :  \atomenumeratorin \}}
\newcommand{\enumeratedatoms}{\atomicformulaof{0},\ldots,\atomicformulaof{\atomorder-1}}
\newcommand{\atomformulaset}{\formulasetof{\mlnatomsymbol}}

\newcommand{\clause}{Z^{\lor}}
\newcommand{\clauseof}[2]{\clause_{#1,#2}}
\newcommand{\clauseofat}[3]{\clauseof{#1}{#2}\left[#3\right]}
\newcommand{\maxtermof}[1]{\clause_{#1}}
\newcommand{\maxtermformulaset}{\formulasetof{\mlnmaxtermsymbol}}

\newcommand{\term}{Z^{\land}}
\newcommand{\termof}[2]{\term_{#1,#2}}
\newcommand{\termofat}[3]{\termof{#1}{#2}\left[#3\right]}
\newcommand{\mintermof}[1]{\term_{#1}}
\newcommand{\mintermofat}[2]{\mintermof{#1}\left[#2\right]}
\newcommand{\mintermformulaset}{\formulasetof{\mlnmintermsymbol}}

\newcommand{\indexedplaceholderof}[1]{\placeholderof{#1}_{\atomlegindexof{#1}}}
\newcommand{\indexedplaceholders}{\indexedplaceholderof{1},\ldots,\indexedplaceholderof{\atomorder}}

\newcommand{\atomorder}{d}
\newcommand{\secatomorder}{r}
\newcommand{\atomenumerator}{k}
\newcommand{\secatomenumerator}{l}

\newcommand{\atomenumeratorin}{\atomenumerator\in[\atomorder]}
\newcommand{\secatomenumeratorin}{\secatomenumerator\in[\secatomorder]}
\newcommand{\atomlegindex}{\catindex}
\newcommand{\tatomlegindex}{\tilde{\atomlegindex}}
\newcommand{\atomlegindexof}[1]{\atomlegindex_{#1}}
\newcommand{\tatomlegindexof}[1]{\tatomlegindex_{#1}}
\newcommand{\atomindices}{{\atomlegindexof{0},\ldots,\atomlegindexof{\atomorder-1}}}
\newcommand{\atomindicesin}{\atomindices\in\atomstates}

%% OPTIMIZATION
\newcommand{\targettensor}{Y}
\newcommand{\importancetensor}{I}

%% MARKOV LOGIC NETWORK
\newcommand{\loss}{\mathcal{L}_{\datamap}}
\newcommand{\lossof}[1]{\loss\left(#1\right)}
\newcommand{\mlnformulaset}{\mathcal{F}}
\newcommand{\mlnformulain}{\exformula\in\mlnformulaset}
\newcommand{\weight}{\theta}
\newcommand{\weightof}[1]{\weight_{#1}}
\newcommand{\weightat}[1]{\weight[#1]}

\newcommand{\mlnparameters}{\formulaset,\canparam}
%\newcommand{\mlnparameterswithout}{\tilde{\formulaset},\canparamt}
\newcommand{\mlntrueparameters}{(\formulaset^*,\weight^*)}



% Examples
\newcommand{\mlnatomsymbol}{[\catorder]}
\newcommand{\mlnmintermsymbol}{\land}
\newcommand{\mlnmaxtermsymbol}{\lor}

\newcommand{\partitionfunction}{\mathcal{Z}}
\newcommand{\secpartitionfunction}{\tilde{\mathcal{Z}}}
\newcommand{\partitionfunctionof}[1]{\partitionfunction\left(#1\right)}
\newcommand{\secpartitionfunctionof}[1]{\secpartitionfunction\left(#1\right)}

\newcommand{\mlnprob}{\probtensorof{\mlnparameters}}
\newcommand{\mlnprobat}[1]{\expdistofat{\mlnparameters}{#1}}
\newcommand{\mlnenergy}{\energytensorof{\mlnparameters}}

\newcommand{\folmlnparameters}{\restfolformulaset,\canparam,\basemeasureof{\fixedimpformula}}

% For Probabilistic Analysis
\newcommand{\kldivsymbol}{\mathrm{D}_{\mathrm{KL}}}
\newcommand{\kldivof}[2]{\kldivsymbol\left[ #1 || #2 \right]}

\newcommand{\noisetensor}{\eta}
\newcommand{\noiseat}[1]{\noisetensor\left[#1\right]}
\newcommand{\noiseof}[1]{\noisetensor^{#1,\gendistribution,\datamap}}
\newcommand{\sstatnoise}{\noiseof{\sstat}}
\newcommand{\mintermnoise}{\noiseof{\identity}}
\newcommand{\mlnnoise}{\noiseof{\mlnstat}}
\newcommand{\mlnnoiseat}[1]{\mlnnoise\left[#1\right]}

\newcommand{\fprob}{p} % Drop! This is mean parameter
\newcommand{\fprobof}[1]{\fprob^{(#1)}}

\newcommand{\bidistof}[1]{B\left(#1\right)}
\newcommand{\multidistof}[1]{\underline{B}\left(#1\right)}
\newcommand{\widthwrtof}[2]{\omega_{#1}\left(#2\right)}
\newcommand{\widthatof}[2]{\widthwrtof{#1}{#2}}

\newcommand{\selbasisshort}{\Gamma}
\newcommand{\selbasislong}{\{\onehotmapofat{\selindex}{\selvariable} \,:\, \selindexin \}}

\newcommand{\failprob}{\epsilon}
\newcommand{\precision}{\tau}
\newcommand{\maxgap}{\Delta}
\newcommand{\maxgapof}[1]{\maxgap\left(#1\right)}

%% CONTRACTION 
\newcommand{\invtemp}{\beta}

%% Hard Logic
\newcommand{\kb}{\mathcal{KB}}
\newcommand{\kbvar}{\headvariableof{\kb}}
\newcommand{\kbat}[1]{\kb\left[#1\right]}

\newcommand{\seckb}{\tilde{\kb}}

%% Tensor Network Formats
\newcommand{\elformat}{\mathrm{EL}}
\newcommand{\cpformat}{\mathrm{CP}}
\newcommand{\htformat}{\mathrm{HT}}
\newcommand{\ttformat}{\mathrm{TT}}

\newcommand{\extnet}{\mathcal{T}^{\graph}}
\newcommand{\secextnet}{\mathcal{T}^{\tilde{\graph}}}
\newcommand{\extnetat}[1]{\extnet\left[#1\right]}

\newcommand{\objof}[1]{O\left(#1\right)} % Drop!

\newcommand{\nodevariables}{\catvariableof{\nodes}}
\newcommand{\nodevariablesof}[1]{\catvariableof{\nodesof{#1}}}
\newcommand{\indexednodevariables}{\indexedcatvariableof{\nodes}}
\newcommand{\edgevariables}{\catvariableof{\edge}}
\newcommand{\extnetdist}{\normationof{\extnet}{\nodevariables}}

\newcommand{\extnetasset}{\{\hypercoreofat{\edge}{\catvariableof{\edge}}\, : \, \edge\in\edges\}}
\newcommand{\tnetof}[1]{\mathcal{T}^{#1}}
\newcommand{\tnetofat}[2]{\tnetof{#1}\left[#2\right]}

%% Probability Representation
\newcommand{\randomx}{\catvariable}

\newcommand{\exrandom}{\catvariableof{0}}
\newcommand{\secexrandom}{\catvariableof{1}}
\newcommand{\thirdexrandom}{\catvariableof{2}}

\newcommand{\indexedexrandom}{\indexedcatvariableof{0}}
\newcommand{\indexedsecexrandom}{\indexedcatvariableof{1}}
\newcommand{\indexedthirdexrandom}{\thirdexrandom=\thirdexrandind}

\newcommand{\exrandind}{\catindexof{0}}
\newcommand{\exranddim}{\catdimof{0}}
\newcommand{\exrandindin}{\exrandind\in[\exranddim]}

\newcommand{\secexrandind}{\catindexof{1}}
\newcommand{\secexranddim}{\catdimof{1}}
\newcommand{\secexrandindin}{\secexrandind\in[\secexranddim]}

\newcommand{\thirdexrandind}{\catindexof{2}}
\newcommand{\thirdexranddim}{\catdimof{2}}
\newcommand{\thirdexrandindin}{\thirdexrandind\in[\thirdexranddim]}

% Hidden Markov Models
\newcommand{\randomxof}[1]{\randomx_{#1}} % In combination with atomenumerator or tenumerator
\newcommand{\randome}{E}
\newcommand{\randomeof}[1]{\randome_{#1}}
\newcommand{\tenumerator}{t}
\newcommand{\tdim}{T}
\newcommand{\tenumeratorin}{\tenumerator\in[\tdim]}

%% Exponential families
\newcommand{\expdistof}[1]{\probtensorof{#1}}
\newcommand{\expdistofat}[2]{\expdistof{#1}[#2]}
\newcommand{\expdist}{\probtensorof{(\sstat,\canparam,\basemeasure)}}
\newcommand{\expdistwith}{\probtensorofat{(\sstat,\canparam,\basemeasure)}{\shortcatvariables}}
\newcommand{\expdistat}[1]{\expdist\left[#1\right]}
\newcommand{\stanexpdistof}[1]{\expdistof{(\sstat,#1,\basemeasure)}}
\newcommand{\mlnexpdistof}[1]{\expdistof{(\formulaset,#1,\basemeasure)}}

\newcommand{\expfamilyof}[1]{\Gamma^{#1}}
\newcommand{\expfamily}{\expfamilyof{\sstat,\basemeasure}}

\newcommand{\realizabledistsof}[1]{\Lambda^{#1}}
\newcommand{\hlnsetof}[1]{\realizabledistsof{#1,\elformat}}

\newcommand{\mnexpfamily}{\expfamilyof{\graph,\ones}} % The exponential family of Markov Networks on \graph
\newcommand{\mlnexpfamily}{\expfamilyof{\mlnstat,\ones}}

\newcommand{\basemeasure}{\nu}
\newcommand{\basemeasureof}[1]{\basemeasure^{#1}}
\newcommand{\basemeasureofat}[2]{\basemeasure^{#1}\left[#2\right]}
\newcommand{\basemeasureat}[1]{\basemeasure\left[#1\right]}
\newcommand{\basemeasurewith}{\basemeasureat{\shortcatvariables}}

\newcommand{\secbasemeasure}{\tilde{\nu}}
\newcommand{\secbasemeasureat}[1]{\secbasemeasure\left[#1\right]}

\newcommand{\sstat}{\phi}
\newcommand{\sstatat}[1]{\sstat\left(#1\right)}
\newcommand{\sstatof}[1]{\sstat^{#1}}
\newcommand{\secsstat}{\tilde{\sstat}}
\newcommand{\proposalstat}{\fselectionmap^T}
\newcommand{\mlnstat}{\formulaset}
\newcommand{\naivestat}{\identity}

\newcommand{\sstatcoordinateof}[1]{\sstat_{#1}}
\newcommand{\sstatcoordinate}{\sstatcoordinateof{\selindex}}
\newcommand{\sstatcoordinateofat}[2]{\sstat_{#1}\left[#2\right]}

\newcommand{\sstatheadvariables}{\headvariableof{[\seldim]}}

\newcommand{\sstatcc}{\rencodingof{\sstat}}
\newcommand{\sstatccwith}{\rencodingofat{\sstat}{\sstatheadvariables,\shortcatvariables}}
\newcommand{\sstatac}{\actcoreof{\sstatcoordinateof{\selindex},\canparamat{\indexedselvariable}}}
\newcommand{\sstatacwith}{\actcoreofat{\sstatcoordinateof{\selindex},\canparamat{\indexedselvariable}}{\headvariableof{\selindex}}}

\newcommand{\sstatcatof}[1]{\headvariableof{#1}}

\newcommand{\sstatindof}[1]{\catindexof{\sstatcoordinateof{#1}}}
\newcommand{\sencsstat}{\sencodingof{\sstat}}
\newcommand{\sencsstatat}[1]{\sencodingof{\sstat}\left[#1\right]}
\newcommand{\sencsstatwith}{\sencsstatat{\shortcatvariables,\selvariable}}

\newcommand{\sencfset}{\sencodingof{\formulaset}}
\newcommand{\sencfsetat}[1]{\sencfset\left[#1\right]}

\newcommand{\sencmlnstat}{\sencodingof{\mlnstat}}
\newcommand{\sencproposalstat}{\sencodingof{\proposalstat}}

\newcommand{\canparam}{\theta}
\newcommand{\canparamof}[1]{\canparam_{#1}}
\newcommand{\canparamat}[1]{\canparam\left[#1\right]}
\newcommand{\canparamofat}[2]{\canparamof{#1}\left[#2\right]}
\newcommand{\canparamwith}{\canparamat{\selvariable}}
\newcommand{\canparamwithin}{\canparamwith\in\simpleparspace}

\newcommand{\singlecanparam}{\canparam}

\newcommand{\seccanparam}{\tilde{\canparam}}

\newcommand{\canparamwrtat}[2]{\canparamofat{#1}{#2}}
\newcommand{\estcanparam}{\hat{\canparam}}
\newcommand{\naivecanparam}{\tilde{\canparam}}
\newcommand{\naivecanparamat}[1]{\naivecanparam\left[#1\right]}

\newcommand{\datacanparam}{\canparamof{\datamap}}
\newcommand{\datacanparamat}[1]{\canparamofat{\datamap}{#1}}

\newcommand{\gencanparam}{\canparamof{*}}
\newcommand{\gencanparamat}[1]{\canparamofat{*}{#1}}

\newcommand{\canparamhypothesis}{\Gamma}
\newcommand{\canparamin}{\canparam\in\canparamhypothesis}

\newcommand{\expsolution}{\gencanparam}
\newcommand{\empsolution}{\datacanparam}

\newcommand{\meanparam}{\mu}
\newcommand{\secmeanparam}{\tilde{\mu}}
\newcommand{\meanparamof}[1]{\meanparam_{#1}}
\newcommand{\meanparamat}[1]{\meanparam\left[#1\right]}
\newcommand{\meanparamofat}[2]{\meanparamof{#1}\left[#2\right]}
\newcommand{\meanparamwith}{\meanparamat{\selvariable}}

\newcommand{\meanrepprob}{\probtensor^{\meanparam}}

\newcommand{\meanset}{\mathcal{M}}
\newcommand{\meansetof}[1]{\meanset_{#1}}
\newcommand{\genmeanset}{\meanset_{\sstat,\basemeasure}}
\newcommand{\hlnmeanset}{\meanset_{\mlnstat,\basemeasure}}
\newcommand{\propmeanset}{\meanset_{\propstat,\ones}}

\newcommand{\genmeansetargmax}{\argmax_{\meanparam\in\genmeanset}}
\newcommand{\cangenmeansetargmax}{\genmeansetargmax\contraction{\canparamwith,\meanparamwith}}
\newcommand{\cansstatcatindicesargmax}{\argmax_{\shortcatindices}\contraction{\canparam,\sstat(\shortcatindices)}}

\newcommand{\normalvec}{a}
\newcommand{\normalbound}{b}
\newcommand{\normalvecofat}[2]{\normalvec_{#1}\left[#2\right]}
\newcommand{\normalboundof}[1]{\normalbound_{#1}}
\newcommand{\normalboundofat}[2]{\normalbound_{#1}\left[#2\right]}
\newcommand{\halfspaceparams}{\left( (\normalvecofat{i}{\selvariable},\normalboundof{i}) \, : \, i \in[n]\right)}
\newcommand{\facecondset}{\mathcal{I}}
\newcommand{\faceset}{Q}
\newcommand{\genfacesetof}[1]{\faceset^{#1}_{\sstat,\basemeasure}}

\newcommand{\datamean}{\meanparamof{\datamap}}
\newcommand{\datameanat}[1]{\datamean\left[#1\right]}

\newcommand{\genmean}{\meanparam^*}
\newcommand{\genmeanat}[1]{\genmean[#1]}

\newcommand{\currentmean}{\tilde{\meanparam}}

\newcommand{\cumfunctionwrt}[1]{A^{#1}}
\newcommand{\cumfunctionwrtof}[2]{\cumfunctionwrt{#1}\left(#2\right)}
\newcommand{\cumfunction}{\cumfunctionwrt{(\sstat,\basemeasure)}}
\newcommand{\cumfunctionof}[1]{\cumfunction(#1)}
\newcommand{\dualcumfunction}{\big(\cumfunction\big)^*}

\newcommand{\forwardmapwrt}[1]{F^{#1}}
\newcommand{\forwardmap}{\forwardmapwrt{(\sstat,\basemeasure)}}
\newcommand{\forwardmapwrtof}[2]{\forwardmapwrt{#1}(#2)}
\newcommand{\forwardmapof}[1]{\forwardmapwrtof{(\sstat,\basemeasure)}{#1}}

\newcommand{\backwardmapwrt}[1]{B^{#1}}
\newcommand{\backwardmap}{\backwardmapwrt{(\sstat,\basemeasure)}}
\newcommand{\backwardmapwrtof}[2]{\backwardmapwrt{#1}(#2)}
\newcommand{\backwardmapof}[1]{\backwardmapwrtof{(\sstat,\basemeasure)}{#1}}

\newcommand{\energytensor}{E}
\newcommand{\energytensorofat}[2]{\energytensor^{#1}[#2]}
\newcommand{\energytensorof}[1]{\energytensor^{#1}}
\newcommand{\energytensorat}[1]{\energytensor\left[#1\right]}
\newcommand{\expenergy}{\energytensorofat{(\sstat,\canparam,\basemeasure)}{\shortcatvariables}}
\newcommand{\expenergyat}[1]{\energytensorofat{(\sstat,\canparam,\basemeasure)}{#1}}

\newcommand{\energyhypothesis}{\Theta}
\newcommand{\energyhypothesisof}[1]{\energyhypothesis^{#1}}

%% Logical Reasoning
\newcommand{\kcore}{K}
\newcommand{\kcoreof}[1]{\kcore^{#1}}
\newcommand{\kcoreofat}[2]{\kcore^{#1}\left[#2\right]}


\newcommand{\tbasis}{e_1}
\newcommand{\tbasisat}[1]{\tbasis\left[#1\right]}
\newcommand{\fbasis}{e_0}
\newcommand{\fbasisat}[1]{\fbasis\left[#1\right]}
\newcommand{\nbasis}{\ones}

\newcommand{\graph}{\mathcal{G}}
\newcommand{\graphof}[1]{\graph^{#1}}
\newcommand{\secgraph}{\tilde{\graph}}
\newcommand{\nodes}{\mathcal{V}}
\newcommand{\nodesof}[1]{\nodes^{#1}}
\newcommand{\innodes}{\nodesof{\mathrm{in}}}
\newcommand{\outnodes}{\nodesof{\mathrm{out}}}

\newcommand{\domainsymbol}{k}
\newcommand{\domainedges}{\edgesof{\domainsymbol}}

\newcommand{\graphqueue}{\mathcal{Q}}

\newcommand{\elgraph}{\graphof{\elformat}}
\newcommand{\maxgraph}{\graphof{\mathrm{max}}}

\newcommand{\prenodes}{\{\secnode \, : \, \secnode \prec \node, \secnode\neq\node\}}
\newcommand{\afternodes}{\{\secnode \, : \, \node \prec \secnode, \secnode\neq\node\}}

\newcommand{\incomingnodes}{\edge^{\mathrm{in}}}
\newcommand{\outgoingnodes}{\edge^{\mathrm{out}}}

\newcommand{\nodesa}{A}
\newcommand{\nodesb}{B}
\newcommand{\nodesc}{C}

\newcommand{\nodesone}{\nodesof{1}}
\newcommand{\nodestwo}{\nodesof{2}}
\newcommand{\nodesthree}{\nodesof{3}}

\newcommand{\secnodes}{\tilde{\nodes}}
\newcommand{\secnodesof}[1]{\tilde{\nodes}^{#1}}
\newcommand{\thirdnodes}{\bar{\nodes}}

\newcommand{\node}{v}
\newcommand{\nodein}{\node\in\nodes}
\newcommand{\secnode}{\tilde{\node}}
\newcommand{\thirdnode}{\bar{\node}}

\newcommand{\edges}{\mathcal{E}}
\newcommand{\edgesof}[1]{\edges^{#1}}
\newcommand{\secedges}{\tilde{\edges}}

\newcommand{\edge}{e}
\newcommand{\edgeof}[1]{\edge_{#1}}
\newcommand{\secedge}{\tilde{\edge}}
\newcommand{\thirdedge}{\hat{\edge}}
\newcommand{\edgein}{\edge\in\edges}

\newcommand{\parentsof}[1]{\mathrm{Pa}(#1)}
\newcommand{\nondescendantsof}[1]{\mathrm{NonDes}(#1)}

\newcommand{\neighborsof}[1]{\mathrm{N}(#1)}

\newcommand{\bnnodecore}{\hypercoreof{(\parentsof{\node},\{\node\})}}
\newcommand{\bnedges}{\{(\parentsof{\node},\{\node\}) \, : \, \nodein\}}

\newcommand{\hypercore}{T}
\newcommand{\hypercoreat}[1]{\hypercore\left[#1\right]}
\newcommand{\hypercorewith}{\hypercoreat{\shortcatvariables}}
\newcommand{\hypercorewithin}{\hypercoreat{\shortcatvariables}\in\facspace}
\newcommand{\hyperonecoordinates}{\shortcatindices \, : \, \hypercoreat{\indexedshortcatvariables} = 1}
\newcommand{\hyperzerocoordinates}{\shortcatindices \, : \, \hypercoreat{\indexedshortcatvariables} = 0}

\newcommand{\hypercoreof}[1]{\hypercore^{#1}}
\newcommand{\hypercoreofat}[2]{\hypercoreof{#1}\left[#2\right]}
\newcommand{\sechypercore}{\tilde{\hypercore}}
\newcommand{\sechypercoreof}[1]{\sechypercore^{#1}}
\newcommand{\sechypercoreofat}[2]{\sechypercore^{#1}\left[#1\right]}
\newcommand{\sechypercoreat}[1]{\sechypercore\left[#1\right]}

%% Factored System
\newcommand{\onehotmap}{e}
\newcommand{\onehotmapof}[1]{\onehotmap_{#1}}
\newcommand{\onehotmapofat}[2]{\onehotmap_{#1}\left[#2\right]}
\newcommand{\onehotmapto}[1]{\onehotmapof{#1}} % For encoding of sets, relations
\newcommand{\invonehotmapof}[1]{\onehotmap^{-1}(#1)}

\newcommand{\statevectorof}[1]{v_{#1}}
\newcommand{\statevectorofat}[2]{\statevectorof{#1}\left[#2\right]}

% Greedy
\newcommand{\extendedformulaset}{\formulaset\cup\{\formula\}}
\newcommand{\extendedcanparam}{\tilde{\canparam}\cup\{\weightof{\formula}\}}

\newcommand{\exfunction}{f}
\newcommand{\exfunctionof}[1]{\exfunction_{#1}}
\newcommand{\exfunctiontargetspace}{\bigotimes_{l\in[p]}\rr^{\catdimof{l}}}
\newcommand{\exfunctiontargetvariables}{Y_0,\ldots,Y_{p-1}}
\newcommand{\exfunctionimageelement}{y}
\newcommand{\exfunctionat}[1]{\exfunction(#1)}
\newcommand{\secexfunction}{g}
\newcommand{\secexfunctionat}[1]{\secexfunction(#1)}

\newcommand{\compositionof}[2]{{#1}\circ{#2}}
\newcommand{\compositionofat}[3]{(\compositionof{#1}{#2})(#3)}
%% Message Passing
\newcommand{\cluster}{C}
\newcommand{\clusterof}[1]{\cluster_{#1}}
\newcommand{\clusterenumerator}{i}
\newcommand{\secclusterenumerator}{j}
\newcommand{\thirdclusterenumerator}{\tilde{j}}

\newcommand{\enc}{\clusterof{\clusterenumerator}}
\newcommand{\secenc}{\clusterof{\secclusterenumerator}}
\newcommand{\thirdenc}{\clusterof{\thirdclusterenumerator}}

\newcommand{\clusterorder}{n}
\newcommand{\clusterenumeratorin}{\clusterenumerator\in[\clusterorder]}

\newcommand{\mesfromto}[2]{\delta_{#1 \rightarrow #2}}
\newcommand{\mesfromtoat}[3]{\mesfromto{#1}{#2}\left[#3\right]}
\newcommand{\upmes}[2]{\delta_{#1 \rightarrow #2}}
\newcommand{\downmes}[2]{\delta_{#2 \leftarrow #1}}

% Binary connective symbols
\newcommand{\impbincon}{\Rightarrow}
\newcommand{\eqbincon}{\Leftrightarrow}
\newcommand{\lpasbincon}{\triangleleft}

\newcommand{\notucon}{\lnot}
\newcommand{\iducon}{\mathrm{Id}}
\newcommand{\trueucon}{\mathrm{T}}

\newcommand{\indexinterpretation}{I}
\newcommand{\indexinterpretationof}[1]{\indexinterpretation_{#1}}
\newcommand{\indexinterpretationat}[1]{\indexinterpretation(#1)}
\newcommand{\indexinterpretationofat}[2]{\indexinterpretationof{#1}(#2)}

\newcommand{\invindexinterpretation}{I^{-1}}
\newcommand{\invindexinterpretationof}[1]{I_{#1}^{-1}}
\newcommand{\invindexinterpretationat}[1]{\invindexinterpretation(#1)}
\newcommand{\invindexinterpretationofat}[2]{\invindexinterpretationof{#1}(#2)}

%ILP
\newcommand{\objectivesymbol}{c}
\newcommand{\objofat}[2]{\objectivesymbol^{#1}\left[#2\right]}
\newcommand{\rhssymbol}{b}
\newcommand{\rhsofat}[2]{\rhssymbol^{#1}\left[#2\right]}

% Coordinate Calculus
\newcommand{\coordinatetrafowrtof}[2]{{#1}\left(#2\right)}
\newcommand{\coordinatetrafowrtofat}[3]{\coordinatetrafowrtof{#1}{#2}\left[#3\right]}
%% CONTRACTIONS
\newcommand{\contractionof}[2]{\left\langle #1\right\rangle \left[ #2 \right]}

\newcommand{\breakablecontractionof}[2]{\big\langle #1 \big\rangle \big[ #2 \big]}
%\newcommand{\contractionof}[2]{\contractionof{#1}{#2}}
\newcommand{\contraction}[1]{\contractionof{#1}{\varnothing}}
%\newcommand{\contraction}[1]{\contraction{#1}}
\newcommand{\normalizationofwrt}[3]{\left\langle #1\right\rangle \left[ #2 | #3 \right]}
\newcommand{\breakablenormalizationofwrt}[3]{\big\langle #1 \big\rangle \big[ #2 | #3 \big]}
%\newcommand{\normalizationofwrt}[3]{\normalizationofwrt{#1}{#2}{#3}}
\newcommand{\normalizationof}[2]{\normalizationofwrt{#1}{#2}{\varnothing}}
%\newcommand{\normalizationof}[2]{\normalizationofwrt{#1}{#2}{\varnothing}}

\newcommand{\nzcontractionof}[2]{\nonzerocirc\contractionof{#1}{#2}}

%% ENCODING SCHEMES: Coordinate, basis, selection
\newcommand{\cencodingof}[1]{\chi^{#1}}
\newcommand{\cencodingofat}[2]{\cencodingof{#1}\left[#2\right]}
\newcommand{\cencodingwith}{\cencodingofat{\exfunction}{\shortcatvariables}}

\newcommand{\bencodingof}[1]{\beta^{#1}}
\newcommand{\bencodingofat}[2]{\bencodingof{#1}\left[#2\right]}
\newcommand{\bencodingwith}{\bencodingofat{\exfunction}{\headvariableof{\exfunction},\shortcatvariables}}

\newcommand{\sencodingof}[1]{\sigma^{#1}}
\newcommand{\sencodingofat}[2]{\sencodingof{#1}\left[#2\right]}  
\newcommand{\sencodingwith}{\sencodingofat{\exfunction}{\shortcatvariables,\selvariable}}

\newcommand{\brepresentationof}[1]{\tau^{#1}}
\newcommand{\brepresentationofat}[2]{\brepresentationof{#1}\left[#2\right]}


% Further tensors

\newcommand{\actcore}{\alpha} % activation of a formula, typical exp
\newcommand{\actcoreof}[1]{\actcore^{#1}}
\newcommand{\actcoreat}[1]{\actcore\left[#1\right]}
\newcommand{\actcoreofat}[2]{\actcore^{#1}[#2]}
\newcommand{\actcorewith}{\actcoreofat{\selindex,\canparamat{\indexedselvariable}}{\headvariableof{\selindex}}}

\newcommand{\dirdelta}{\delta}
\newcommand{\dirdeltaof}[1]{\dirdelta^{#1}}
\newcommand{\dirdeltaofat}[2]{\dirdeltaof{#1}\left[#2\right]}
\newcommand{\dirdeltawith}{\dirdeltaofat{[\catorder],\catdim}{\shortcatvariables}}

\newcommand{\onehotmap}{\epsilon}
\newcommand{\onehotmapof}[1]{\onehotmap_{#1}}
\newcommand{\onehotmapofat}[2]{\onehotmap_{#1}\left[#2\right]}
\newcommand{\onehotmapto}[1]{\onehotmapof{#1}} % For encoding of sets, relations
\newcommand{\onehotmapwith}{\onehotmapofat{\shortcatindices}{\shortcatvariables}}
\newcommand{\invonehotmapof}[1]{\onehotmap^{-1}(#1)}

\newcommand{\noisetensor}{\eta}
\newcommand{\noiseat}[1]{\noisetensor\left[#1\right]}
\newcommand{\noiseof}[1]{\noisetensor^{#1,\gendistribution,\datamap}}
\newcommand{\sstatnoise}{\noiseof{\sstat}}
\newcommand{\sstatnoisewith}{\noiseof{\sstat}[\selvariable]}

\newcommand{\canparam}{\theta}
\newcommand{\canparamof}[1]{\canparam_{#1}}
\newcommand{\canparamat}[1]{\canparam\left[#1\right]}
\newcommand{\canparamofat}[2]{\canparamof{#1}\left[#2\right]}
\newcommand{\canparamwith}{\canparamat{\selvariable}}
\newcommand{\indexedcanparam}{\canparamat{\indexedselvariable}}
\newcommand{\canparamwithin}{\canparamwith\in\parspace}

\newcommand{\kcore}{\kappa}
\newcommand{\kcoreof}[1]{\kcore^{#1}}
\newcommand{\kcoreat}[1]{\kcore\left[#1\right]}
\newcommand{\kcoreofat}[2]{\kcore^{#1}\left[#2\right]}
\newcommand{\kcorewith}{\kcoreofat{\edge}{\catvariableof{\edge}}}

\newcommand{\scalarcore}{\lambda} % Scalar core in CP decompositons
\newcommand{\scalarcoreof}[1]{\scalarcore[#1]}
\newcommand{\scalarcoreat}[1]{\scalarcore[#1]}
\newcommand{\scalarcoreofat}[2]{\scalarcore^{#1}[#2]}
\newcommand{\scalarcorewith}{\scalarcoreat{\decvariable}}

\newcommand{\meanparam}{\mu}
\newcommand{\meanparamof}[1]{\meanparam_{#1}}
\newcommand{\meanparamat}[1]{\meanparam\left[#1\right]}
\newcommand{\meanparamofat}[2]{\meanparamof{#1}\left[#2\right]}
\newcommand{\indexedmeanparam}{\meanparamat{\indexedselvariable}}
\newcommand{\meanparamwith}{\meanparamat{\selvariable}}

\newcommand{\basemeasure}{\nu}
\newcommand{\basemeasureof}[1]{\basemeasure^{#1}}
\newcommand{\basemeasureofat}[2]{\basemeasure^{#1}\left[#2\right]}
\newcommand{\basemeasureat}[1]{\basemeasure\left[#1\right]}
\newcommand{\basemeasurewith}{\basemeasureat{\shortcatvariables}}

\newcommand{\tnet}{\tau}
\newcommand{\tnetof}[1]{\tnet^{#1}}
\newcommand{\tnetofat}[2]{\tnetof{#1}\left[#2\right]}
\newcommand{\extnet}{\tnetof{\graph}}
\newcommand{\secextnet}{\tnetof{\tilde{\graph}}}
\newcommand{\extnetat}[1]{\extnet\left[#1\right]}
\newcommand{\extnetwith}{\tnetofat{\graph}{\shortcatvariables}}

\newcommand{\legcore}{\rho} % Leg core in CP decompositions
\newcommand{\legcoreof}[1]{\legcore^{#1}}
\newcommand{\legcoreofat}[2]{\legcoreof{#1}\left[#2\right]}
\newcommand{\legcorewith}{\legcoreofat{\atomenumerator}{\catvariableof{\atomenumerator},\decvariable}}

\newcommand{\energytensor}{\phi}
\newcommand{\energytensorofat}[2]{\energytensor^{#1}[#2]}
\newcommand{\energytensorof}[1]{\energytensor^{#1}}
\newcommand{\energytensorat}[1]{\energytensor\left[#1\right]}
\newcommand{\expenergy}{\energytensorofat{(\sstat,\canparam,\basemeasure)}{\shortcatvariables}}
\newcommand{\expenergyat}[1]{\energytensorofat{(\sstat,\canparam,\basemeasure)}{#1}}
\newcommand{\energytensorwith}{\energytensorat{\shortcatvariables}}

%% Sets of tensors
\newcommand{\expfamilyof}[1]{\Gamma^{#1}}
\newcommand{\expfamily}{\genexpfamily}
\newcommand{\genexpfamily}{\expfamilyof{\sstat,\basemeasure}}
\newcommand{\expfamilywith}{\expfamilyof{\sstat,\basemeasure}}

\newcommand{\realizabledistsof}[1]{\Lambda^{#1}}
\newcommand{\maxrealizabledistsof}[1]{\realizabledistsof{#1,\maxgraph}}
\newcommand{\elrealizabledistsof}[1]{\realizabledistsof{#1,\elformat}}
\newcommand{\realizabledistswith}{\realizabledistsof{\sstat,\graph}}
\newcommand{\hlnsetof}[1]{\realizabledistsof{#1,\elformat}}






%% MAIN VARIABLES
\newcommand{\indvariable}{O} 
\newcommand{\inddim}{r}
\newcommand{\indindex}{o} % was s
\newcommand{\indenumerator}{l}
\newcommand{\indorder}{n} % number of variables 

\newcommand{\selvariable}{L} 
\newcommand{\seldim}{p}
\newcommand{\selindex}{l}
\newcommand{\selenumerator}{s}
\newcommand{\selorder}{n}

\newcommand{\catvariable}{X} 
\newcommand{\catdim}{m}
\newcommand{\catindex}{x} % was i
\newcommand{\catenumerator}{\atomenumerator}
\newcommand{\catorder}{\atomorder}

\newcommand{\headvariable}{Y} % head of a basis encoding
\newcommand{\headdim}{n}
\newcommand{\headindex}{y}

\newcommand{\datvariable}{J} % Can be understood as selvariable selecting a datapoint, catvariable as a random datapoint, indvariable as a abstract object representing the sample, also used at indexvariable!
\newcommand{\datdim}{m}
\newcommand{\datindex}{j}

%% Syntactic Sugar
\newcommand{\indvariableof}[1]{\indvariable_{#1}}
\newcommand{\selvariableof}[1]{\selvariable_{#1}}
\newcommand{\catvariableof}[1]{\catvariable_{#1}}
\newcommand{\headvariableof}[1]{\headvariable_{#1}}

\newcommand{\indvariablelist}{\indvariableof{0},\ldots,\indvariableof{\individualorder-1}}
\newcommand{\catvariablelist}{\catvariableof{0},\ldots,\catvariableof{\atomorder-1}}
\newcommand{\selvariablelist}{\selvariableof{0},\ldots,\selvariableof{\selorder-1}}

\newcommand{\shortindvariablelist}{\indvariableof{[\individualorder]}}
\newcommand{\shortcatvariablelist}{\catvariableof{[\atomorder]}}
\newcommand{\shortselvariablelist}{\selvariableof{[\selorder]}}

\newcommand{\selindices}{\selindexof{0},\ldots,\selindexof{\selorder-1}}

\newcommand{\shortindindices}{\indindexof{[\indorder]}}
\newcommand{\shortcatindices}{\catindexof{[\catorder]}}
\newcommand{\shortselindices}{\selindexof{[\selorder]}}

\newcommand{\inddimof}[1]{\inddim_{#1}}
\newcommand{\seldimof}[1]{\seldim_{#1}}
\newcommand{\catdimof}[1]{\catdim_{#1}}
\newcommand{\headdimof}[1]{\headdim_{#1}}

\newcommand{\indindexof}[1]{\indindex_{#1}}
\newcommand{\selindexof}[1]{\selindex_{#1}}
\newcommand{\catindexof}[1]{\catindex_{#1}} 
\newcommand{\headindexof}[1]{\headindex_{#1}}

\newcommand{\indindexin}{\indindex\in[\inddim]}
\newcommand{\selindexin}{\selindex\in[\seldim]}
\newcommand{\catindexin}{\catindex\in[\catdim]}
\newcommand{\datindexin}{\datindex\in[\datdim]}

\newcommand{\indindexofin}[1]{\indindexof{#1}\in[\inddimof{#1}]}
\newcommand{\catindexofin}[1]{\catindexof{#1}\in[\catdimof{#1}]}
\newcommand{\selindexofin}[1]{\selindexof{#1}\in[\seldimof{#1}]}
\newcommand{\headindexofin}[1]{\headindexof{#1}\in[\headdimof{#1}]}

\newcommand{\indindexlist}{\indindexof{0},\ldots,\indindexof{\indorder-1}}
\newcommand{\catindexlist}{\catindexof{0},\ldots,\catindexof{\atomorder-1}}
\newcommand{\selindexlist}{\selindexof{0},\ldots,\selindexof{\selorder-1}}

\newcommand{\indenumeratorin}{\indenumerator\in[\indorder]}
\newcommand{\selenumeratorin}{\selenumerator\in[\selorder]}
\newcommand{\catenumeratorin}{\catenumerator\in[\catorder]}

\newcommand{\indexedindvariableof}[1]{\indvariableof{#1}=\indindexof{#1}}
\newcommand{\indexedcatvariableof}[1]{\catvariableof{#1}=\catindexof{#1}}
\newcommand{\indexedselvariableof}[1]{\selvariableof{#1}=\selindexof{#1}}
\newcommand{\indexedheadvariableof}[1]{\headvariableof{#1}=\headindexof{#1}}

\newcommand{\indexedindvariable}{\indexedindvariableof{}}
\newcommand{\indexedcatvariable}{\indexedcatvariableof{}}
\newcommand{\indexedselvariable}{\indexedselvariableof{}}

\newcommand{\catstatesof}[1]{[\catdimof{#1}]}

\newcommand{\catspaceof}[1]{\rr^{\catdimof{#1}}}

\newcommand{\indspace}{\bigotimes_{\indenumeratorin}\rr^{\inddim}}
\newcommand{\catspace}{\bigotimes_{\atomenumeratorin} \rr^{\catdimof{\atomenumerator}}}

\newcommand{\selstates}{\bigtimes_{\selenumeratorin}[\seldimof{\selenumerator}]}
\newcommand{\selvectorspace}{\rr^{\seldim}}
\newcommand{\selspace}{\bigotimes_{\selenumeratorin}\rr^{\seldimof{\selenumerator}}}
%%

\newcommand{\datanum}{\datdim}

\newcommand{\datain}{\datindex\in[\datanum]}
\newcommand{\data}{\{\datapointof{\datindex}\}_{\datindexin}}
\newcommand{\dataaverage}{\frac{1}{\datanum}\sum_{\datindexin}}

\newcommand{\catvariables}{\catvariablelist}
\newcommand{\shortcatvariables}{\shortcatvariablelist}
\newcommand{\indexedshortcatvariables}{\shortcatvariables=\shortcatindices}
\newcommand{\shortcatindicesin}{\shortcatindices\in\facstates}
\newcommand{\shortatomindicesin}{\shortcatindices\in\atomstates}
\newcommand{\datshortcatvariables}{\shortcatvariables=\shortcatindices^{\datindex}}

\newcommand{\headvariables}{\headvariableof{[\seldim]}}

\newcommand{\shortindvariables}{\shortindvariablelist}
\newcommand{\indexedshortindvariables}{\shortindvariables=\shortindindices}
\newcommand{\datshortindvariables}{\shortindvariables=\shortindindices^{\datindex}}

\newcommand{\selvariables}{\selvariableof{0},\ldots,\selvariableof{\selorder-1}}
\newcommand{\shortselvariables}{\selvariableof{[\selorder]}}
\newcommand{\indexedshortselvariables}{\shortselvariables=\shortselindices}
\newcommand{\secselenumerator}{\tilde{\selenumerator}}
\newcommand{\secselvariable}{\tilde{\selvariable}}
\newcommand{\secselindex}{\tilde{\selindex}}

\newcommand{\nodestatesof}[1]{\bigtimes_{\node\in#1}\catstatesof{\node}}
\newcommand{\atomstates}{\bigtimes_{\atomenumeratorin}[2]}


\newcommand{\symindstates}{\bigtimes_{\indenumeratorin}[\inddim]}

\newcommand{\facstates}{\bigtimes_{\atomenumeratorin}\catstatesof{\atomenumerator}}
\newcommand{\facdim}{\prod_{\atomenumeratorin}\catdimof{\atomenumerator}}
\newcommand{\secfacstates}{\bigtimes_{\secatomenumerator\in[\secatomorder]}\catstatesof{\secatomenumerator}}

\newcommand{\atomspace}{\bigotimes_{\atomenumeratorin}\rr^2}
\newcommand{\facspace}{\catspace}
\newcommand{\secfacspace}{\bigotimes_{\secatomenumerator\in[\seccatorder]} \rr^{\catdimof{\secatomenumerator}}}

\newcommand{\indexedcatvariables}{\indexedcatvariableof{0},\ldots,\indexedcatvariableof{\atomorder-1}} 
\newcommand{\tildeindexedcatvariables}{\catvariableof{0}=\tilde{\catindex}_0,\ldots,\catvariableof{\atomorder-1}=\tilde{\catindex}_{\atomorder-1}} 

\newcommand{\seccatenumerator}{\tilde{\catenumerator}}
\newcommand{\seccatenumeratorin}{\seccatenumerator\in[\catorder]}

\newcommand{\seccatvariable}{Y} % used as differentiation variable
\newcommand{\seccatindex}{y}
\newcommand{\seccatorder}{p} % Has to coincide with seldim for basis encoding def

\newcommand{\seccatvariableof}[1]{\seccatvariable_{#1}}
\newcommand{\indexedseccatvariableof}[1]{\seccatvariableof{#1}=\seccatindexof{#1}}
\newcommand{\seccatvariables}{\seccatvariableof{0},\ldots,\seccatvariableof{\seccatorder\shortminus1}}
\newcommand{\secshortcatvariables}{\seccatvariableof{[\seccatorder]}}
\newcommand{\indexedseccatvariables}{\indexedseccatvariableof{0}\ldots,\indexedseccatvariableof{\seccatorder-1}} 
\newcommand{\indexedsecshortcatvariables}{\indexedseccatvariableof{[\seccatorder]}}

\newcommand{\catvariablesinset}[1]{\catvariableof{#1}}%{\catvariableof{\node} \, : \, \node \in #1}
\newcommand{\seccatindexof}[1]{\seccatindex_{#1}}

\newcommand{\catindices}{\catindexlist}
\newcommand{\tildecatindexof}[1]{\tilde{\catindex}_{#1}}
\newcommand{\tildecatindices}{\tildecatindexof{0},\ldots,\tildecatindexof{\atomorder-1}}
\newcommand{\seccatindices}{{\seccatindexof{0},\ldots,\seccatindexof{\secatomorder-1}}}
\newcommand{\tildeshortcatindices}{\tildecatindexof{[\catorder]}}

\newcommand{\catindicesof}[1]{{\catindexof{0}^{#1},\ldots,\catindexof{\atomorder-1}^{#1}}}

\newcommand{\catzeropositions}{\{\atomenumerator : \catindexof{\atomenumerator}=0\}}
\newcommand{\catonepositions}{\{\atomenumerator : \catindexof{\atomenumerator}=0\}}

%% Cores
\newcommand{\categoricalmap}{Z}
\newcommand{\categoricalmapat}[1]{\categoricalmap\left[#1\right]}
\newcommand{\categoricalmapof}[1]{\categoricalmap^{#1}}
\newcommand{\categoricalmapofat}[2]{\categoricalmap^{#1}\left[#2\right]}

\newcommand{\categoricalcore}{\bencodingof{\categoricalmap}}
\newcommand{\categoricalcoreof}[1]{\bencodingof{\categoricalmapof{#1}}}
\newcommand{\categoricalcoreofat}[2]{\bencodingof{\categoricalmapof{#1}}\left[#2\right]}

\newcommand{\datamap}{D}
\newcommand{\datamapat}[1]{\datamap(#1)}
\newcommand{\datamapof}[1]{\datamap_{#1}}
\newcommand{\datapointof}[1]{\datamapat{#1}}
\newcommand{\datapoint}{\datapointof{\datindex}}
\newcommand{\dataset}{\left((\catindicesof{\datindex})\,:\,\datindexin\right)}

\newcommand{\secdatamap}{\tilde{\datamap}}
\newcommand{\datacore}{\bencodingof{\datamap}}
\newcommand{\datacoreat}[1]{\datacore\left[#1\right]}
\newcommand{\datacoreof}[1]{\bencodingof{\datamap_{#1}}}
\newcommand{\datacoreofat}[2]{\bencodingof{\datamap_{#1}}[#2]}

\newcommand{\secdatacoreof}[1]{\bencodingof{\secdatamap_{#1}}}
\newcommand{\empdistribution}{\probtensor^{\datamap}}
\newcommand{\empdistributionat}[1]{\empdistribution\left[#1\right]}
\newcommand{\empdistributionwith}{\empdistributionat{\shortcatvariables}}

\newcommand{\varcore}[1]{U^{#1}} % For optimization of tensor network
\newcommand{\varspace}[1]{\rr^{p_{#1}}}
\newcommand{\varcollection}{\big\{\varcore{\atomenumerator}\, :\, \atomenumeratorin \big\}}



\newcommand{\conactcore}{\kcore}
\newcommand{\conactcoreof}[1]{\conactcore^{#1}}


% DecompositionIndex 
\newcommand{\decvariable}{I}
\newcommand{\decvariableof}[1]{\decvariable_{#1}}
\newcommand{\decindex}{i} % Needs to be different to datindex!
\newcommand{\decindexof}[1]{\decindex_{#1}}
\newcommand{\indexeddecvariableof}[1]{\decvariableof{#1}=\decindexof{#1}}
\newcommand{\decdim}{n}
\newcommand{\decdimof}[1]{\decdim_{#1}}
\newcommand{\decindexin}{\decindex\in[\decdim]}
\newcommand{\indexeddecvariable}{\decvariable=\decindex}
\newcommand{\inddecvar}{\indexeddecvariable}

\newcommand{\indexeddatvariable}{\datvariable=\datindex}

% Used in poly sparsity
\newcommand{\indexvariable}{\datvariable} % for datacores used
\newcommand{\indexset}{J}
\newcommand{\indexsetof}[1]{\indexset^{#1}}

\newcommand{\slackvariable}{z}
\newcommand{\slackindex}{z}
\newcommand{\slackindexof}[1]{\slackindex_{#1}}

\newcommand{\rankofat}[2]{\mathrm{rank}^{#1}\left(#2\right)}
\newcommand{\cprankof}[1]{\mathrm{rank}\left(#1\right)}
\newcommand{\bincprankof}[1]{\mathrm{rank}^{\mathrm{bin}}\left(#1\right)}
\newcommand{\slicesparsityof}[1]{\slicerankwrtof{\catorder}{#1}} % former {\tilde{\ell} \left(#1\right)}


\newcommand{\dircprankof}[1]{\mathrm{rank}^{\mathrm{dir}}\left(#1\right)}
\newcommand{\bascprankof}[1]{\mathrm{rank}^{\mathrm{bas}}\left(#1\right)}
\newcommand{\baspluscprankof}[1]{\mathrm{rank}^{\mathrm{bas+}}\left(#1\right)}
\newcommand{\quacprankof}[1]{\mathrm{rank}^{\mathrm{qua}}\left(#1\right)}

\newcommand{\sliceset}{\mathcal{M}}
\newcommand{\slicescalar}{\lambda}
\newcommand{\slicescalarof}[1]{\slicescalar^{#1}}
\newcommand{\slicetupleof}[1]{(\slicescalar^{#1}, \variablesetof{#1}, \catindexof{\variablesetof{#1}}^{#1})}
\newcommand{\enumeratedslices}{\{\slicetupleof{\decindex} \, : \, \decindexin\}}

\newcommand{\sliceorder}{r}
\newcommand{\slicerankwrtof}[2]{\mathrm{rank}^{#1}\left(#2\right)}

\usetikzlibrary {arrows.meta} 
\usetikzlibrary{shapes,positioning}
\usetikzlibrary{decorations.markings}
\usetikzlibrary{calc}

\tikzset{
    midarrow/.style={
        postaction={decorate},
        decoration={markings, mark=at position 0.5 with {\arrow{>}}}
    },
    midbackarrow/.style={
        postaction={decorate},
        decoration={markings, mark=at position 0.5 with {\arrow{<}}}
    },
     ->-/.style={midarrow},
     -<-/.style={midbackarrow}
}

\newcommand{\shortminus}{\scalebox{0.4}[1.0]{$-$}}

\newcommand{\drawvariabledot}[2]{
	\draw[fill] (#1,#2) circle (0.15cm);
}

% Draws indices and below the indices the core
\newcommand{\drawatomindices}[2]{
	\begin{scope}[shift={(#1,#2)}]
		\draw[<-] (0,1)--(0,-1) node[midway,left] {\tiny $\catvariableof{0}$}; 
		\draw[<-] (1.5,1)--(1.5,-1) node[midway,left] {\tiny $\catvariableof{1}$}; 
		\node[anchor=center] (text) at (3,0) {$\cdots$};
		\draw[<-] (4,1)--(4,-1) node[midway,right] {\tiny $\catvariableof{\atomorder\shortminus1}$}; 
	\end{scope}
}
\newcommand{\drawundiratomindices}[2]{
	\begin{scope}[shift={(#1,#2)}]
		\draw[] (0,1)--(0,-1) node[midway,left] {\tiny $\catvariableof{0}$}; 
		\draw[] (1.5,1)--(1.5,-1) node[midway,left] {\tiny $\catvariableof{1}$}; 
		\node[anchor=center] (text) at (3,0) {$\cdots$};
		\draw[] (4,1)--(4,-1) node[midway,right] {\tiny $\catvariableof{\atomorder\shortminus1}$}; 
	\end{scope}
}

\newcommand{\drawatomcore}[3]{
	\begin{scope}[shift={(#1,#2)}]
		\draw (-1,-1) rectangle (5,-3);
		\node[anchor=center] (text) at (2,-2) {#3};
	\end{scope}
}

\pretolerance=500
\tolerance=100 
\emergencystretch=10pt

% Bibliography
\DeclareUnicodeCharacter{FB01}{fi}
\usepackage[round]{natbib}


\newcommand{\red}[1]{\textcolor{red}{#1}}

\begin{document}
\title{The Tensor Network Approach to Efficient and Explainable AI}
\author{Alex Goessmann, DATEV eG}

\maketitle
\date{\today}

\begin{abstract}
	Tensor spaces appear naturally in factored representations of systems, when storing information about a systems state.
	Since the curse of dimensionality prevents feasible generic representations and reasoning, logical and probabilistic reasoning focuses on tradeoffs between the sparsity and the generality of reasoning.
	In this work we present these tradeoffs based on the tensor network formalism and formulate feasible reasoning algorithms based on tensor network contractions.
	We review the classical logical and probabilistic approaches to reasoning in the first part and develop applications in neuro-symbolic AI in the second part.
	In the third parts we investigate in more detail schemes to exploit tensor network contractions for calculus.
\end{abstract}	

\tableofcontents

\chapter{Introduction}\label{cha:introduction}

% Explaining the title
Artificial intelligence is a long-standing dream, which has in recent years received enourmous attention, driven by breakthroughs in large language models.
Among the key priorities towards an economic and trustworthy usage remain the creation of efficient and the explainable models.

% Explainability
Instead of post-hoc explainability of a models inference given specific data, our aim in this work is the intrinsic human understandability of a model.
We are motivated by the theory of logic, which formalization of human thoughts serves as an interface between mechanized reasoning on a machine and human understandability.
Having established this advanced form of explainability enables novel forms of human interactions with a model based on verbalizations, manipulations and guarantees on the models inference output.

% Efficiency
The desire of an efficient model originates more from an economic perspective on the realizability of a model and its power consumption.
Tensors appear naturally as representations of a system with multiple variables, both in logical and probabilistic approaches towards artificial intelligence. % avoid factored at this point!
However, already for moderate numbers of variables, the curse of dimensionality prevents a typical machines memory to store a generic representation.
The careful design of representation formats is therefore a necessary task to avoid the exponential increase of storage demands and balance the expressivity and the efficiency of representation formats.

% Tensor Networks
We in this work exploit the formalism of tensor networks in the creation of efficient representation schemes.
The chosen tensor network formats are motivated from explainable learning architectures and provide a synergy between the aims of efficiency and explainability.
Tensor networks appear as the natural numerical structures in probabilistic graphical models and logical knowledge bases.
After presenting the probabilistic and logical approaches based on the tensor network formalism we develop novel applications schemes towards neuro-symbolic artificial intelligence.

\sect{Background}

Before presenting an overview over the contents, we further motivate this work based on the broach approaches towards artificial intelligence and more recent developments.

\subsect{Classical Approached towards AI}

We start with ontological commitments in the description of a system and follow the book \cite{russell_artificial_2021} distinguishing atomic, factored and structured representations.
While in atomic representation, the states of a systems are enumerated and represented in a single variable, factored representations describe a systems state based on a collection of variables.
In the tensor formalism, each state of a system corresponds with a coordinate of a representing tensor.
The order of the tensor coincides therefore with the number of variables in a system.
In an atomic representation, where there is a single coordinate, each state corresponds with a coordinate of the representing vector being a tensor of order one.
Having a factored representation with two variables requires order two tensors or matrices, where a coordinate is specified by a row and a column index.
Given larger numbers of coordinates now extends this representation picture to tensors of larger orders, which have more abstract axes besides rows and columns.
The generalization of the atomic representation to a factored system thus corresponds with the generalization of vectors towards matrices and tensors of larger orders.
Along this line, we can always transform a factored representation of a system to an atomic one, just by enumerating the states of the factored system and interpreting them by a single variable.
This amounts to the flattening of a representing tensor to a vector.
However, by doing so, we would loose much of the structure of the representation, which we would like to exploit in reasoning processes.

% Structured Representations
A more generic representation of systems are structured representation.
Structured representations involve objects of differing numbers and relations between them.
As a consequence the numbers of variables can differ depending on the state of a system.
This poses a challenge to the tensor representation, since a fixed number of variables is required to motivate a tensor space of representations.
There are approaches to circumvent these difficulty by the development of template models such as Markov Logic Networks \cite{richardson_markov_2006}, which are instantiated on systems with differing number of objects.
We will discuss those in \charef{cha:folModels}.

% Continous vs discrete
In this work we treat discrete systems, where the number of states is finite.
One can understand them as a discretization of continuous variables and many results will generalize by the converse limit to the situation of continuous variables.

% Epistemologic
Besides ontological commitments in the choice of a representation scheme, modelling a system also requires epistemologic commitments, by defining what properties are to be reasoned about.
In logical approaches the properties of states are boolean values representing whether a state is consistent with known constraints.
Probabilistic approaches assign to the coordinates of the tensors numbers in $[0,1]$ encoding the probability of a state.
Compared with logical approached to reasoning, probabilistic approaches thus bear a more expressive modelling.

\subsect{Logic and Explainability in AI}

\textbf{Inductive Logic Programming:}
\begin{itemize}
    \item ILP is a classical task \cite{muggleton_inductive_1994}
    \item Amie \cite{galarraga_amie_2013} is a method of learning Horn clauses using a refinement operator.
    \item Class Expression Learning \cite{lehmann_class_2011} is a more recent approach to assist in the design of reasoning capabilities in Knowledge Graphs.
        However, problems arise from the expressivity of description logics and the efficient choice of formulas from exponentially large hypothesis sets.
    \item CEL has therefore recently received further popularity in combination with reinforcement learning \cite{demir_drill-_2021} and neural networks \cite{kouagou_neural_2022, pesquita_neural_2023}, which are methods searching efficienctly in exponentially large spaces of formulas.
\end{itemize}

\textbf{Statistical Relational AI:}
\begin{itemize}
    \item Classical combination of logical and probabilistic approaches to reasoning \cite{getoor_introduction_2019}
\end{itemize}

\textbf{Knowledge Graphs}
\begin{itemize}
    \item The advent of large Knowledge Graphs enables explainable reasoning methods on structured data. \cite{antoniou_semantic_2012,hogan_knowledge_2021}
    \item Knowledge Graphs are stored in a sparse format, i.e. only true atoms instead of all + truth label.
\end{itemize}


\subsect{Tensor Networks in AI}

\textbf{Tensor Network formats}
\begin{itemize}
    \item TT Format \cite{holtz_manifolds_2012,hackbusch_new_2009}
    \item HT Format \cite{hackbusch_tensor_2012}
    \item CP Format
\end{itemize}

\textbf{Tensor Networks as Regressors}
\begin{itemize}
    \item Dynamical Systems learning \cite{gels_multidimensional_2019, goesmann_tensor_2020}
    \item Supervised learning \cite{stoudenmire_supervised_2016}
\end{itemize}

\textbf{Tensor Representation of Logics}
\begin{itemize}
    \item Tensor Networks have been applied in the automatization of logic reasoning \cite{li_linear_2017, sato_linear_2017} apply Matrix multiplication in reasoning.
    \item \cite{nickel_review_2016} review over relational machine learning and latent features via matrix embeddings.
\end{itemize}

\textbf{Tensor Representation of Knowledge Graphs}
\begin{itemize}
    \item Effective representation of queries
    \item Usage of tensor networks in embeddings \cite{yang_embedding_2015} and using complex extensions \cite{trouillon_complex_2017, trouillon_knowledge_2017}
\end{itemize}

\textbf{Tensor Representation of Graphical Models}
\begin{itemize}
    \item Duality of Graphical Models and Tensor Networks: \cite{robeva_duality_2019}
    \item Expressivity studies \cite{glasser_expressive_2019}
\end{itemize}

\subsect{Infrastructure of AI}

The formalism of tensors and their network decompositions and contractions bears the potential of parallel computations exploited in the AI-dedicated soft and hardware.
\begin{itemize}
    \item Hardware: TPUs beyond GPUs
    \item Software: Tensors as basic data structure in TensorFlow, pyTorch etc., storing neural activations and model weights.
\end{itemize}



\sect{Structure of the work}

The chapters are structured into three parts, and two focuses, as sketched by:
\newcommand{\horDistChapter}{6}
\newcommand{\verDistChapter}{2.25}
\newcommand{\parBlockDistance}{0.25}
\newcommand{\drawchapter}[4]{
    \coordinate (logRepStart) at (#1, #2);
    \draw (logRepStart) rectangle ($(logRepStart) + (blockDiagonal)$);
    \node[anchor=center] (text) at ($(logRepStart) + (toTop) +0.5*(blockDiagonal)$) {\small #3};
    \node[anchor=center] (text) at ($(logRepStart) + (toBottom) +0.5*(blockDiagonal)$) {\small #4};
}

\begin{tikzpicture}[scale=0.9]
    \coordinate (blockDiagonal) at (5,1.75);
    \coordinate (toTop) at (0,0.35);
    \coordinate (toBottom) at (0,-0.35);

    % Representation
    \node[anchor=center] (text) at (-0.5, 2*\verDistChapter+0.35) {\focusonespec};
    \draw[dashed]  (-0.25-1*\horDistChapter, 2*\verDistChapter+1) -- (1*\horDistChapter-0.75, 2*\verDistChapter+1) -- (1*\horDistChapter-0.75, -4*\verDistChapter-1)
    -- (-0.25-1*\horDistChapter, -4*\verDistChapter-1) -- (-0.25-1*\horDistChapter, 2*\verDistChapter+1);

    % Reasoning
    \node[anchor=center] (text) at (-0.5+1.5*\horDistChapter, 2*\verDistChapter+0.35) {\focustwospec};
    \draw[dashed]  (-0.25+1*\horDistChapter, 2*\verDistChapter+1) -- (2*\horDistChapter-0.75, 2*\verDistChapter+1) -- (2*\horDistChapter-0.75, -4*\verDistChapter-1)
    -- (-0.25+1*\horDistChapter, -4*\verDistChapter-1) -- (-0.25+1*\horDistChapter, 2*\verDistChapter+1);

    % Part I
    \node[anchor=center] (text) at (-0.5-0.5*\horDistChapter, 1*\verDistChapter-0.25+0.35) {\parref{par:one}:};
    \node[anchor=center] (text) at (-0.5-0.5*\horDistChapter, 1*\verDistChapter-0.25-0.35) {\partonetext};
    \draw (-0.5-1*\horDistChapter,-0.25) -- (2*\horDistChapter-0.5, -0.25) -- (2*\horDistChapter-0.5, 2*\verDistChapter-0.25) --
    (-0.5-1*\horDistChapter, 2*\verDistChapter-0.25) -- (-0.5-1*\horDistChapter,-0.25);
    \foreach \x/\y/\key/\name in {
        0/1/cha:probRepresentation/\chatextprobRepresentation,
        0/0/cha:logicalRepresentation/\chatextlogicalRepresentation,
        1/1/cha:probReasoning/\chatextprobReasoning,
        1/0/cha:logicalReasoning/\chatextlogicalReasoning
    } {
        \drawchapter{\x * \horDistChapter}{\y *\verDistChapter}{\charef{\key}}{\name}
    }

    % Part II
    \node[anchor=center] (text) at (-0.5-0.5*\horDistChapter, -1.5*\verDistChapter-0.5+0.35) {\parref{par:two}:};
    \node[anchor=center] (text) at (-0.5-0.5*\horDistChapter, -1.5*\verDistChapter-0.5-0.35) {\parttwotext};
    \draw (-0.5-1*\horDistChapter,-0.5) -- (2*\horDistChapter-0.5, -0.5) -- (2*\horDistChapter-0.5, -2*\verDistChapter-0.5) --
    (-0.5-1*\horDistChapter, -2*\verDistChapter-0.5) -- (-0.5-1*\horDistChapter,-0.5);
    \foreach \x/\y/\key/\name in {
        -1/-1/cha:formulaSelection/\chatextformulaSelection,
        0/-1/cha:networkRepresentation/\chatextnetworkRepresentation,
        0/-2/cha:folModels/\chatextfolModels,
        1/-1/cha:networkReasoning/\chatextnetworkReasoning,
        1/-2/cha:concentration/\chatextconcentration
    } {
        \drawchapter{\x * \horDistChapter}{\y *\verDistChapter - \parBlockDistance}{\charef{\key}}{\name}
    }

    % Part III
    \node[anchor=center] (text) at (-0.5-0.5*\horDistChapter, -3.5*\verDistChapter-0.5+0.35) {\parref{par:three}:};
    \node[anchor=center] (text) at (-0.5-0.5*\horDistChapter, -3.5*\verDistChapter-0.5-0.35) {\partthreetext};
    \draw (-0.5 - 1*\horDistChapter, -2*\verDistChapter-0.75) -- (2*\horDistChapter-0.5, -2*\verDistChapter-0.75) -- (2*\horDistChapter-0.5, -4*\verDistChapter-0.75) --
    (-0.5-1*\horDistChapter, -4*\verDistChapter-0.75) -- (-0.5-1*\horDistChapter,-2*\verDistChapter-0.75);
    \foreach \x/\y/\key/\name in {
        -1/-3/cha:coordinateCalculus/\chatextcoordinateCalculus,
        0/-3/cha:basisCalculus/\chatextbasisCalculus,
        0/-4/cha:sparseRepresentation/\chatextsparseCalculus,
        1/-3/cha:approximation/\chatextapproximation,
        1/-4/cha:messagePassing/\chatextmessagePassing
    } {
        \drawchapter{\x * \horDistChapter}{\y *\verDistChapter - 2 * \parBlockDistance}{\charef{\key}}{\name}
    }

\end{tikzpicture}

\textbf{\parref{par:one}: \partonetext} \\
\ \\
The probabilistic and logical approaches towards artificial intelligence are reviewed in the tensor network formalism. \\

Tensors appear naturally in
\begin{itemize}
    \item Logics: Boolean tensors indicating models (propositional case) and interpretation tensors in first order logics
    \item Probability theory: Truth tables, which are tensors of probabilities for joint distibutions of categorical variables.
\end{itemize}

% Classical usage of tensor network decompositions
Tensor network decompositions as representation schemes appear in
\begin{itemize}
    \item Logics: Conjunctions of formulas are Hadamard products of the tensor representation of formulas (Coordinate Calculus/ Effective Calculus)
    \item Probability theory: Graphical models are tensor networks of the factors. Further sparsity schemes apply, when placing restrictions on the structure of each factor.
    \item Data bases: Relations encoded by lists as storage of nonvanishing coordinates of a relation encoding
\end{itemize}

% Classical usage of tensor network contractions
Tensor network contractions as reasoning schemes appear in
\begin{itemize}
    \item Logics: Model counts, used for satisfiablility decisions and entailment
    \item Probability theory: Marginal probability distributions, extended to conditional probability distributions through normations
\end{itemize}

\ \\
\textbf{\parref{par:two}: \parttwotext} \\
\ \\
Motivated by the classical approaches we apply the tensor network formalism towards learning and infering neuro-symbolic models. \\

\textbf{Neurosymbolic AI}
\begin{itemize}
    \item Required for more advanced AI \cite{hochreiter_toward_2022}
    \item Add the paradigm of neural computing to logical reasoning
    \item Potential benefits from Statistical Relational AI \cite{marra_statistical_2024}
\end{itemize}

%\subsect{Neuro-Symbolic AI}

\textbf{Tensor Approaches to Neuro-Symbolic AI}
\begin{itemize}
    \item TensorLog \cite{cohen_tensorlog_2020}
    \item \cite{badreddine_logic_2022} representation of logic using tensor networks and automated differentiation to optimize.
    \item \cite{badreddine_logic_2022} representation of logic using tensor networks and automated differentiation to optimize.
\end{itemize}

%% Decomposition of Neural Networks
In Deep Neural Networks, functions between the input layer and the output layer are decomposed into neurons.
Typical neurons are linear transforms with an activation function.

%% Sparsity by fixed architecture
Sparsity means restriction to functions, which are decomposable into a small number of neurons.
Approximations of generic functions (see the universal approximation theorems) would require large amounts of neurons. % CITE!
When restricting to functions based on a fixed architecture, we restrict to a certain set of functions called the inductive bias of the architecture.

\ \\
\textbf{\parref{par:three}: \partthreetext}\\
\ \\
The applied schemes of calculus using tensor network contractions are investigated in more detail.
\ \\
% Representation
\textbf{\focusonespec}\\
\\\
Here we motivate and investigate the efficient representation of tensors based on tensor network decompositions. \\
\ \\
% Reasoning
\textbf{\focustwospec}\\
\ \\
We develop schemes to efficiently perform inductive and deductive reasoning based on information stored in decomposed tensor.


%\subsect{\focusonespec}
%
%\subsect{\focustwospec}
%
%\subsect{\parref{par:one}: \partonetext}
%
%\subsect{\parref{par:two}: \parttwotext}
%
%\subsect{\parref{par:three}: \partthreetext}
\section{Notation and Basic Concepts}\label{cha:TensorNetworks}

We here provide the fundamental definitions of tensors, which are essentiell for the content in Part~I and Part~II.
In Part~III we will further investigate the properties of tensors focusing on their contractions.

\subsection{Categorical Variables and Representations}

We will in this work investigate systems, which are described by a set of properties, each called a categorical variable. 
This is called an ontological commitment, since it defines what properties a system has.

\begin{definition}
	An atomic representation of a system is described by a categorical variables $\catvariable$ taking values $\catindex$ in a finite set 
		\[  [\catdim]\coloneqq \{0,\ldots, \catdim-1\} \]
	of cardinality $\catdim$.
\end{definition}

% Notation: Large and small literals
We will in this work always notate categorical variables by large literals and indices by small literals, possible with other letters such as $\catvariable,\selvariable,\indvariable,\datvariable$ and corresponding values $\catindex,\selindex,\indindex,\datindex$.

\begin{definition}
	A factored representation of a system is a set of categorical variables $\catvariableof{\atomenumerator}$, where $\atomenumeratorin$, taking values in $[\catdimof{\atomenumerator}]$.
\end{definition}

\subsection{Tensors}

% Gentle introduction sentences
Tensors are multiway arrays and a generalization of vectors and matrices to higher orders.
We will first provide a formal definition as real maps from index sets enumerating the coordinates of vectors, matrices and larger order tensors.

\begin{definition}[Tensor]\label{def:tensor}
	Let there be numbers $\catdimof{\atomenumerator}\in\nn$ for $\atomenumeratorin$ and categorical variables $\catvariableof{\atomenumerator}$ taking their values in $[\catdimof{\atomenumerator}]$.
	We call maps
	\begin{align*}
		\hypercoreat{\catvariables} : \bigtimes_{\atomenumeratorin} [\catdimof{\atomenumerator}] \rightarrow \rr
	\end{align*}
	tensor of order $\atomorder$ and leg dimensions $\catdimof{0},\ldots,\catdimof{\atomorder-1}$.
	Evaluations of these maps at indices $\catindices$ are denoted by
	\begin{align*}
		\hypercoreat{\indexedcatvariables} = \hypercoreat{\catvariables}(\catindices) \, .
	\end{align*}	
%	with coordinates denoted by $\hypercore_{\catindices}$ is called a tensor of order $\atomorder$ and legs with the dimensions $\catdimof{0},\ldots,\catdimof{\atomorder-1}$.
	Tensors $\hypercoreat{\catvariables}$ are elements of the space
	\begin{align*}
		\bigotimes_{\atomenumeratorin} \rr^{\catdimof{\atomenumerator}} \,  
	\end{align*}
	which is, with the operations of coordinatewise summation and scalar multiplication, a linear space called a tensor space.
\end{definition} 

% Non-canonical 
We here introduced tensors in a non-canonical way based on categorical variables assigned to its axis.
While coming as syntactic sugar at this point, this will allow us to define contractions without further specification of axes, based on comparisons of shared categorical variables.
Especially, this eases the implementation of tensor network contractions without the need to further specify a graph (see Appendix~\ref{cha:implementation}).

% Further abbreviations
We abbreviate lists $\catvariables$ of categorical variables by $\shortcatvariables$, that is denote $\hypercoreat{\catvariables}$ by $\hypercoreat{\shortcatvariables}$.
Occasionally, when the categorical variables of a tensor are clear from the context, we will omit the notation of the variables. %further abbreviate $\hypercoreat{\catvariables}$ by $\hypercore$.

\begin{example}[Trivial Tensor]\label{exa:trivialTensor}
	The trivial tensor is defined as the map 
		\[ \onesat{\shortcatvariables} : \facstates \rightarrow \{1\} \subset \rr \]
	with all coordinates being $1$, that is for all $\catindices\in\facstates$
		\[ \onesat{\indexedshortcatvariables} = 1 \, . \]
\end{example}


\subsection{One-hot encodings}

We are now ready to provide the link between tensors and states of systems with factored representations.
To this end, we define the one-hot encoding of a state, which is a bijection between the states and the basis elements of a tensor space.

\begin{definition}[One-hot encodings to Atomic Representations]
	Given an atomic system described by the categorical variable $\catvariable$, we define for each $\catindex\in[\catdim]$ the basis vector $\onehotmapofat{\catindex}{\catvariable}$ by
	\begin{align}
		\onehotmapofat{\catindex}{\catvariable=\tilde{\catindex}} = \begin{cases}
			1 & \text{if} \quad \catindex=\tilde{\catindex} \\
			0 & \text{else} \, .
		\end{cases} 
	\end{align}
	The one-hot encoding of states $\catindex\in[\catdim]$ of the atomic system described by the categorical variable $\catvariable$ is the map
		\[ \onehotmap: [\catdim] \rightarrow \rr^\catdim \]
	which maps $\catindex \in [\catdim]$ to the basis vectors $\onehotmapofat{\catindex}{\catvariable}$.
\end{definition}

% Coordinatewise representation
The basis vectors $\onehotmapofat{\catindex}{\catvariable}$ are tensors of order $1$ and leg dimension $\catdim$ of the structure
\begin{align}
	\onehotmapofat{\catindex}{\catvariable} = \begin{bmatrix}
	0 & \cdots & 0 & 1 &  0 & \cdots & 0
	\end{bmatrix} \, ,
\end{align}
where the $1$ is at the $\catindex$th coordinate of the vector.

% Atomic -> Factored system
We have so far described one-hot representations of the states of a single categorical variable, which would suffice to encode the state of an atomic system.
In a factored system on the other side, we are dealing with multiple categorical variables.

\begin{definition}[One-hot encodings to Factored Representations]
	Let there be a factored system defined by a tuple $(\catvariables)$ of variables taking values in $\facstates$.
	The one-hot encoding of its states is the tensor product of the one-hot encoding to each categorical variables, that is the map
		\[ \onehotmap : \facstates \rightarrow  \facspace \]
	defined by mapping $\catindices=\shortcatindices$ to
	\begin{align*}
		 \onehotmapofat{\shortcatindices}{\shortcatvariables}
		=: \bigotimes_{\atomenumeratorin} \onehotmapofat{\catindexof{\atomenumerator}}{\catvariableof{\atomenumerator}} \, . 
	\end{align*}
	We will call one-hot representations \emph{tensor representations} and depict them as
	\begin{center}
		\begin{tikzpicture}[scale=0.35,thick] % , baseline = -3.5pt


\draw (-12,1) rectangle (-3,3);
\node[anchor=center] (text) at (-7.5,2) {\corelabelsize $\bigotimes_{\atomenumeratorin} \onehotmapof{\catindexof{\atomenumerator}}$};
\draw (-11,1)--(-11,-1) node[midway,right] {\colorlabelsize $\catvariableof{0}$};
\draw (-9.5,1)--(-9.5,-1) node[midway,right] {\colorlabelsize $\catvariableof{1}$};
\node[anchor=center] (text) at (-6.75,0) {$\cdots$};
\draw (-4,1)--(-4,-1) node[midway,right] {\colorlabelsize $\catvariableof{\atomorder\shortminus1}$};


\node[anchor=center] (text) at (-1,2) {\corelabelsize ${=}$};

\draw (1,1) rectangle (3,3);
\node[anchor=center] (text) at (2,2) {\corelabelsize $\onehotmapof{\catindexof{0}}$};
\draw (2,1)--(2,-1) node[midway,right] {\colorlabelsize $\catvariableof{0}$};

\node[anchor=center] (text) at (4.5,2) {$\otimes$};

\draw (6,1) rectangle (8,3);
\node[anchor=center] (text) at (7,2) {\corelabelsize $\onehotmapof{\catindexof{1}}$};
\draw (7,1)--(7,-1) node[midway,right] {\colorlabelsize $\catvariableof{1}$};

\node[anchor=center] (text) at (9.5,2) {$\otimes$};

\node[anchor=center] (text) at (11,2) {$\cdots$};

\node[anchor=center] (text) at (12.5,2) {$\otimes$};


\draw (14,1) rectangle (16,3);
\node[anchor=center] (text) at (15,2) {\corelabelsize $\onehotmapof{\catindexof{\atomorder\shortminus1}}$};
\draw (15,1)--(15,-1) node[midway,right] {\colorlabelsize $\catvariableof{\atomorder\shortminus1}$};





\end{tikzpicture}
	\end{center}
\end{definition}


\begin{remark}[Flattening of Tensors]
	The use the tensor product to represent states of factored systems can be motivated by the reduction to atomic systems by enumeration of the states.
	We have this property reflected in the state encoding of factored systems, since the tensor space $\bigotimes_{\atomenumeratorin}\rr^{\catdimof{\atomenumerator}}$ is isomorphic to the vector spaces $\rr^{\prod_{\atomenumeratorin}\catdimof{\atomenumerator}}$.
	This operation is called flattening (or unfolding) of tensors with many axes to tensors of less axes.
\end{remark}



\subsection{Contractions}

Contractions are the central manipulation operation on sets of tensors.
To introduce them, we will develop a graphical illustration of sets of tensors, which we also call tensor networks.
In Part~III we will further investigate the utility of contractions in representing specific calculations, which demand different encoding schemes.


\subsubsection{Graphical Illustrations}

% Hypergraph as capturing the categorical variable assignment to tensors
Sets of tensor with categorical variables assigned to each legs implicitly carry a notion of a hypergraph.
This perspective is especially useful, when some categorical variables are assigned to axis of multiple tensors, as it will often be the case in the applications considered in this work.
Each variable can then be labeled by a node and each tensor as a hyperedge containing the nodes to its axis variables.
Let us first formally introduce hypergraphs, which are generalizations of graphs allowing edges to be arbitrary nonempty subsets of the nodes, whereas canonical graphs demand a cardinality of two.

\begin{definition}\label{def:hypergraphs}
	A hypergraph is a pair $\graph=(\nodes,\edges)$ of a set of nodes $\nodes$ and a set of edges $\edges$, where each hyperedge $\edge\in\edges$ is a subset of the nodes $\nodes$.
	A directed hypergraph is a pair $\graph=(\nodes,\edges)$, such that each hyperedge $\edge\in\edges$ is the tuple of two disjoint sets $\incomingnodes,\outgoingnodes\subset\nodes$, that is
		\[ \edge = (\incomingnodes,\outgoingnodes)  \, . \]
\end{definition}

% Diagrammatic representation in factor graphs
We will use the standard visualization by factor graphs as a diagrammatic illustration of sets of tensors, where tensors are represented by block nodes and each axis assigned with by a categorical variable $\catvariableof{\atomenumerator}$ represented by a node, see Figure~\ref{fig:tensors}a). 
%We further denote on Each axis of the tensor is represented by a node representing the variable $\catvariableof{\atomenumerator}$ and the tensor $\hypercore$ is associated with the hyperedge $\edge$ connecting all variables.
 %representing the choice of an element in the set $[\catdimof{\atomenumerator}]$
% Hyperedge view
Different simplifications of these factor graph depictions have been evolved in different research fields.
In the tradition of graphical models, which started with the work \cite{pearl_probabilistic_1988}, the categorical variables are highlighted and the tensor blocks just depicted by hyperedges.
To depict dependencies with causal interpretations, the edges are further decorated by directions in the depiction of Bayesian networks, see for example \cite{pearl_causality_2009}.

In the tensor network community on the other hand, a simplification scheme highlighting the tensors as blocks and omitting the depiction of categorical variables has been evolved.
The variables, or sometimes their index or dimension, are then directly assigned to the lines depicting the axes of the tensor blocks.

Both depiction schemes are simplifications of factor graphs, by highlighting the categorical variables in the depiction in Figure~\ref{fig:tensors}b) and the tensors in the depiction in Figure~\ref{fig:tensors}c).
We in this work will prefer the simplification of the tensor network community, depicted in Figure~\ref{fig:tensors}b).

% Duality
In another interpretation (see \cite{robeva_duality_2019}), both simplification schemes are itself interpret as hypergraphs, which are dual to each other.

\begin{figure}[h!]
	\begin{center}
		\begin{tikzpicture}[scale=0.35,thick] % , baseline = -3.5pt


\begin{scope}[shift={(-17,0)}]

\node[anchor=center] (text) at (-3,0) {$a)$};

\draw (-1,1) rectangle (10,-1);
\node[anchor=center] (text) at (4.5,0) {\small $\hypercoreof{\edge}$};

\draw (0,-1)--(0,-3) node[midway,left] {\tiny $\catvariableof{0}$}; 
\draw (3,-1)--(3,-3) node[midway,left] {\tiny $\catvariableof{1}$}; 
\node[anchor=center] (text) at (3,-4) {$\cdots$};
\draw (9,-1)--(9,-3) node[midway,right] {\tiny $\catvariableof{\atomorder\shortminus1}$}; 

\node [circle, draw, thick, fill=gray!50, minimum size = \nodeminsize] (P1) at (0,-4) {\tiny $\catvariableof{0}$};	
\node [circle, draw, thick, fill=gray!50, minimum size = \nodeminsize] (P2) at (3,-4) {\tiny $\catvariableof{1}$};
\node[anchor=center] (text) at (6,-4) {$\cdots$};

\node [circle, draw, thick, fill=gray!50, minimum size = \nodeminsize] (P3) at (9,-4) {};

\node[anchor=center] (text) at (9,-4) {\tiny $\catvariableof{\atomorder-1}$};


\end{scope}


\node[anchor=center] (text) at (-2,0) {$b)$};

\node [circle, draw, thick, fill=gray!50, minimum size = \nodeminsize] (P1) at (0,-3) {\tiny $\catvariableof{0}$};	
\node [circle, draw, thick, fill=gray!50, minimum size = \nodeminsize] (P2) at (3,-3) {\tiny $\catvariableof{1}$};

\node[anchor=center] (text) at (6,-3) {$\cdots$};

\node [circle, draw, thick, fill=gray!50, minimum size = \nodeminsize] (P3) at (9,-3) {};

\node[anchor=center] (text) at (9,-3) {\tiny $\catvariableof{\atomorder-1}$};


\draw (P1) to[bend right=-25] (4.5,0);
\draw (P2) to[bend right=-10] (4.5,0);
\draw (P3) to[bend right=25] (4.5,0);
\node[anchor=center] (text) at (4.5,0.5) {$\edge$};


\begin{scope}[shift={(16,2)}]

\node[anchor=center] (text) at (-3,-2) {$c)$};

\draw (-1,-1) rectangle (5,-3);
\node[anchor=center] (text) at (2,-2) {\small $\hypercoreof{\edge}$};
\draw (0,-3)--(0,-5) node[midway,left] {\tiny $\catvariableof{0}$}; 
\draw (1.5,-3)--(1.5,-5) node[midway,left] {\tiny $\catvariableof{1}$}; 
\node[anchor=center] (text) at (3,-4) {$\cdots$};
\draw (4,-3)--(4,-5) node[midway,right] {\tiny $\catvariableof{\atomorder\shortminus1}$}; 

\end{scope}

%\drawatomcore{3.5}{-8}{$\probtensor$}
%\drawatomindices{3.5}{-12}	
%\draw (5.5,-9)--(5.5,-7) node[midway,right] {\tiny $\catvariableof{\exformula}$};

\end{tikzpicture}
	\end{center}
	\caption{Depiction of Tensors 
	a) As a factor in a factor graph, depicted by a block, and connected to categorical variables assigned to nodes.
	b) Highlighting only the variable dependencies by a hyperedge connecting the variables $\catvariableof{\atomenumerator}$ to each axis $\atomenumeratorin$.
	c) Highlighting the tensor by a blockwise notation with axes denoted by open legs represented by the variables $\catvariableof{\atomenumerator}$.
	}\label{fig:tensors}
\end{figure}


% Diagramatic representation of vectors
To depict vector calculus and its generalizations, we will apply the graphical notation (mainly version b) introduced in Chapter~\ref{cha:TensorNetworks}. 
Along this line, we represent vectors and their generalization to tensors by blocks with legs representing its indices.
The basis vectors being one-hot encodings of states are in this scheme represented by
	\begin{center}
		\begin{tikzpicture}[scale=0.3,thick] % , baseline = -3.5pt

\draw (1,1) rectangle (3,3);
\node[anchor=center] (text) at (2,2) {\small $\onehotmapof{\catindex}$};
\draw (2,-1)--(2,1) node[midway,right] {\tiny $\catvariable$};

\end{tikzpicture}
	\end{center}
where $\tilde{\catindex}$ is an indexed represented by an open leg. 
Assigning $\catindex$ to this index will retrieve the $\catindex$th coordinate (with value $1$), whereas all other assignments will retrieve the coordinate values $0$. 


Drawing on the interpretation of tensors by hyeredges we can continue with the definition of tensor networks.

\begin{definition}\label{def:tensorNetwork}
	Let $\graph=(\nodes,\edges)$ be a hypergraph with nodes decorated by categorical variables $\catvariableof{\node}$ with dimensions
		\[ \catdimof{\node} \in \nn \]	
	and hyperedges $\edge\in\edges$ decorated by core tensors
		\[ \hypercoreofat{\edge}{\catvariableof{\edge}} \in \bigotimes_{\node\in\edge}\rr^{\catdimof{\node}} \, , \]
	where we denote by $\catvariableof{\edge}$ the set of categorical variables $\catvariableof{\node}$ with $\node\in\edge$.
	Then we call the set 
		\[ \tnetofat{\graph}{\catvariableof{\nodes}} = \{\hypercoreofat{\edge}{\catvariableof{\edge}}  \, : \, \edge\in\edges\} \]
	the Tensor Network of the decorated hypergraph $\graph$.
\end{definition}


\begin{figure}
	\begin{center}
		\begin{tikzpicture}[scale=0.35,thick] % , baseline = -3.5pt


\node[anchor=center] (text) at (-2,0) {$a)$};

\node [circle, draw, thick, fill=gray!50, minimum size = \nodeminsize] (P1) at (0,-3) {\tiny $\catvariableof{0}$};	
\node [circle, draw, thick, fill=gray!50, minimum size = \nodeminsize] (P2) at (3,-3) {\tiny $\catvariableof{1}$};
\node [circle, draw, thick, fill=gray!50, minimum size = \nodeminsize] (P3) at (6,-3) {\tiny $\catvariableof{2}$};

\node [circle, draw, thick, fill=gray!50, minimum size = \nodeminsize] (P4) at (9,-3) {\tiny $\catvariableof{3}$};;


\draw (P1) to[bend right=-20] (3,0);
\draw (P2) to[bend right=0] (3,0);
\draw (P3) to[bend right=20] (3,0);
\node[anchor=center] (text) at (3,0.5) {$\edge_0$};

\draw (P2) to[bend right=20] (4.5,-6);
\draw (P3) to[bend right=-20] (4.5,-6);

\node[anchor=center] (text) at (4.5,-6.5) {$\edge_1$};

\draw (P3) to[bend right=20] (7.5,-6);
\draw (P4) to[bend right=-20] (7.5,-6);

\node[anchor=center] (text) at (7.5,-6.5) {$\edge_2$};


\begin{scope}[shift={(25,0)}]

\node[anchor=center] (text) at (-2,0) {$b)$};

\draw (-1,-1) rectangle (5,-3);
\node[anchor=center] (text) at (2,-2) {\small $\hypercoreof{\edge_0}$};
\draw (0,-3)--(0,-5) node[midway,left] {\tiny $\catlegof{0}$}; 
\draw (2,-3)--(2,-5) node[midway,left] {\tiny $\catlegof{1}$}; 
%\draw (3,-3)--(3,-5) node[midway,left] {\tiny $\catlegof{1}$}; 
\draw (4,-3)--(4,-5) node[midway,left] {\tiny $\catlegof{2}$}; 


\draw (6,-1) rectangle (10,-3);
\node[anchor=center] (text) at (8,-2) {\small $\hypercoreof{\edge_2}$};
\draw (7,-3)--(7,-5) node[midway,right] {\tiny $\catlegof{2}$}; 
\draw (9,-3)--(9,-5) node[midway,right] {\tiny $\catlegof{3}$}; 


\draw (1,-7) rectangle (5,-9);
\node[anchor=center] (text) at (3,-8) {\small $\hypercoreof{\edge_1}$};
\draw (2,-5)--(2,-7); % node[midway,left] {\tiny $\catlegof{1}$}; 
\draw (4,-5) to[bend right=20]  (7,-6); % node[midway,left] {\tiny $\catlegof{2}$}; 
\draw (4,-7) to[bend right=-20]  (7,-6); 

\draw[fill] (2,-6) circle (0.25cm);
\draw (2,-6) to[bend right=20] (-1,-8); % node[midway, right]{\tiny $\catlegof{1}$};
\node[anchor=center] (text) at (-2,-8) {\tiny $\catlegof{1}$};

\draw[fill] (7,-6) circle (0.25cm);
\draw (7,-5) -- (7,-6);
\draw (7,-6)--(7,-8) node[midway,right] {\tiny $\catlegof{2}$}; 

\end{scope}


\end{tikzpicture}
	\end{center}
	\caption{
	Example of a tensor network.
	a) Hypergraph with edges $\edge_0=\{\catvariableof{0},\catvariableof{1},\catvariableof{2}\}$, $\edge_1=\{\catvariableof{1},\catvariableof{2}\}$ and $\edge_2=\{\catvariableof{2},\catvariableof{3}\}$ decorated by tensor cores.
	b) Dual tensor network, depicting a contraction with leaving all variables open.
	}\label{fig:network}
\end{figure}

%%
%Diagrammatic notation: Best to do version a) as used in the definition, highlighting that tensors have shared categorical variables with fixed dimensions.




\subsubsection{Tensor Product}

% Diagrams -> Contractions
Let us now exploit the developed graphical representations to define contractions of tensor networks.
The simplest contraction is the tensor product, which maps a pair of two tensors with distinct variables onto a third tensor and has an interpretation by coordinatewise products.
Such a contraction corresponds with a tensor network of two tensors with disjoint variables, depicted as:
\begin{center}
	\begin{tikzpicture}[scale=0.35,thick] % , baseline = -3.5pt


\begin{scope}[shift={(-15,0)}]



\draw (-1,1) rectangle (10,-1);
\node[anchor=center] (text) at (4.5,0) {\corelabelsize $\hypercoreof{\edge_0}$};

\draw (0,-1)--(0,-3) node[midway,left] {\colorlabelsize $\catvariableof{0}$};
\draw (3,-1)--(3,-3) node[midway,left] {\colorlabelsize $\catvariableof{1}$};
\node[anchor=center] (text) at (3,-4) {$\cdots$};
\draw (9,-1)--(9,-3) node[midway,right] {\colorlabelsize $\catvariableof{\atomorder\shortminus1}$};

\node [circle, draw, thick, fill=\nodegrayscale, minimum size = \nodeminsize] (P1) at (0,-4) {\colorlabelsize $\catvariableof{0}$};
\node [circle, draw, thick, fill=\nodegrayscale, minimum size = \nodeminsize] (P2) at (3,-4) {\colorlabelsize $\catvariableof{1}$};
\node[anchor=center] (text) at (6,-4) {$\cdots$};

\node [circle, draw, thick, fill=\nodegrayscale, minimum size = \nodeminsize] (P3) at (9,-4) {};

\node[anchor=center] (text) at (9,-4) {\colorlabelsize $\catvariableof{\atomorder-1}$};



\end{scope}




\draw (-1,1) rectangle (10,-1);
\node[anchor=center] (text) at (4.5,0) {\corelabelsize $\hypercoreof{\edge_1}$};

\draw (0,-1)--(0,-3) node[midway,left] {\colorlabelsize $\seccatvariableof{0}$};
\draw (3,-1)--(3,-3) node[midway,left] {\colorlabelsize $\seccatvariableof{1}$};
\node[anchor=center] (text) at (3,-4) {$\cdots$};
\draw (9,-1)--(9,-3) node[midway,right] {\colorlabelsize $\seccatvariableof{\seccatorder\shortminus1}$};

\node [circle, draw, thick, fill=\nodegrayscale, minimum size = \nodeminsize] (P1) at (0,-4) {\colorlabelsize $\seccatvariableof{0}$};
\node [circle, draw, thick, fill=\nodegrayscale, minimum size = \nodeminsize] (P2) at (3,-4) {\colorlabelsize $\seccatvariableof{1}$};
\node[anchor=center] (text) at (6,-4) {$\cdots$};

\node [circle, draw, thick, fill=\nodegrayscale, minimum size = \nodeminsize] (P3) at (9,-4) {};

\node[anchor=center] (text) at (9,-4) {\colorlabelsize $\seccatvariableof{\seccatorder-1}$};




\end{tikzpicture}
\end{center}

\begin{definition}[Tensor Product]\label{def:tensorProduct}
	Let there be two tensor
	\begin{align*}
		\hypercoreofat{\edge_0}{\shortcatvariables} : \facstates \rightarrow \rr \quad \text{and} \quad  \hypercoreofat{\edge_1}{\secshortcatvariables} : \secfacstates \rightarrow \rr \, 
	\end{align*}
	with different categorical variables assigned to its axes.
	Then there tensor product is the map
	\begin{align*}
		\contractionof{\hypercoreofat{\edge_0}{\shortcatvariables},\hypercoreofat{\edge_1}{\secshortcatvariables}}{\shortcatvariables,\secshortcatvariables} :  \left(\facstates\right) \times \left(\secfacstates\right) \rightarrow \rr
	\end{align*}
	defined for $\catindices\in\facstates$ and $\seccatindices\in\secfacstates$ as
	\begin{align*}
		& \contractionof{\hypercore,\sechypercore}{\indexedcatvariables,\indexedseccatvariables} \\
		&\quad\quad :=  \hypercoreofat{\edge_0}{\indexedcatvariables}\cdot \hypercoreofat{\edge_1}{\indexedseccatvariables} \, .
	\end{align*}
\end{definition}

% Other notations
Other popular standard notations of tensor products (see \cite{kolda_tensor_2009,hackbusch_tensor_2012,cichocki_tensor_2015}) 
	\[ \left(\hypercore \otimes \sechypercore\right) = \left(\hypercore \circ \sechypercore\right)  
	= \contractionof{\hypercoreofat{\edge_0}{\shortcatvariables},\hypercoreofat{\edge_1}{\secshortcatvariables}}{\shortcatvariables,\secshortcatvariables}  \, . \]
We will avoid these notations in this work in favor of a consistent notation capable of depicting generic tensor network contractions.

When the tensor $\hypercoreofat{\edge_1}{\secshortcatvariables}$ coincides with the trivial tensor $\onesat{\secshortcatvariables}$ (see Example~\ref{exa:trivialTensor}), we further make a notation convention to omit that tensor, that is
\begin{align*}
	\contractionof{\hypercoreofat{\edge_0}{\shortcatvariables},\onesat{\secshortcatvariables}}{\shortcatvariables,\secshortcatvariables}  
	= \contractionof{\hypercoreofat{\edge_0}{\shortcatvariables}}{\shortcatvariables,\secshortcatvariables} \, .
\end{align*}


\subsubsection{Generic Contractions}


Contractions of Tensor Networks $\extnet$ are operations to retrieve single tensors by summing products of tensors in a network over common indices.
We will define contractions formally by specifying just the indices not to be summed over.

When some of the variables are not appearing as leg variables, we define the contraction as being a tensor product with the trivial tensor $\ones$ carrying the legs of the missing variables.

\begin{definition}\label{def:contraction}
	Let $\tnetof{\graph}$ be a tensor network on a decorated hypergraph $\graph=(\nodes,\edges)$.
	For any subset $\secnodes\subset\nodes$ we define the contraction  to be the tensor 
	\begin{align}
		\contractionof{\tnetof{\graph}}{\catvariableof{\secnodes}} \in \bigotimes_{\node\in\secnodes} \rr^{\catdimof{\node}}
	\end{align}
	defined coordinatewise by the sum	
	\begin{align}
		\contractionof{\tnetof{\graph}}{\indexedcatvariableof{\secnodes}} =
		\sum_{\catindexof{\nodes/\secnodes} \in\,\nodestatesof{\nodes/\secnodes}}
		\left( \prod_{\edge\in\edges}\hypercoreofat{\edge}{\indexedcatvariableof{\edge}} \right) \, .
	\end{align}
	We call $\catvariableof{\secnodes}$ the open variables of the contraction.
\end{definition}


\begin{figure}
	\begin{center}
		\begin{tikzpicture}[scale=0.35,thick] % , baseline = -3.5pt


%\node[anchor=center] (text) at (-2,0) {$a)$};

\draw (-5,-1) rectangle (9,-3);
\node[anchor=center] (text) at (2,-2) {\small $\contractionof{\{\hypercoreof{\edge_0},\hypercoreof{\edge_1},\hypercoreof{\edge_2}\}}{\catvariableof{1},\catvariableof{3}}$};
\draw (0,-3)--(0,-5) node[midway,left] {\tiny $\catvariableof{1}$}; 
\draw (4,-3)--(4,-5) node[midway,left] {\tiny $\catvariableof{3}$}; 

\node[anchor=center] (text) at (11.5,-2) {${=}$};

\begin{scope}[shift={(15,0)}]

%\node[anchor=center] (text) at (-2,0) {$b)$};

\draw (-1,-1) rectangle (5,-3);
\node[anchor=center] (text) at (2,-2) {\small $\hypercoreof{\edge_0}$};
\draw (0,-3)--(0,-4) node[midway,right] {\tiny $\catvariableof{0}$}; 
\draw (-1,-4) rectangle (1,-6);
\node[anchor=center] (text) at (0,-5) {\small $\ones$};

\draw (2,-3)--(2,-5) node[midway,right] {\tiny $\catvariableof{1}$}; 
\draw (4,-3)--(4,-5) node[midway,right] {\tiny $\catvariableof{2}$}; 


\draw (6,-1) rectangle (10,-3);
\node[anchor=center] (text) at (8,-2) {\small $\hypercoreof{\edge_2}$};
\draw (7,-3)--(7,-5) node[midway,right] {\tiny $\catvariableof{2}$}; 
\draw (9,-3)--(9,-5) node[midway,right] {\tiny $\catvariableof{3}$}; 


\draw (1,-7) rectangle (5,-9);
\node[anchor=center] (text) at (3,-8) {\small $\hypercoreof{\edge_1}$};
\draw (2,-5)--(2,-7); % node[midway,left] {\tiny $\catvariableof{1}$}; 
\draw (4,-5) to[bend right=20]  (7,-6); % node[midway,left] {\tiny $\catvariableof{2}$}; 
\draw (4,-7) to[bend right=-20]  (7,-6); 

\draw[fill] (2,-6) circle (0.15cm);
\draw (2,-6) to[bend right=20] (-1,-8); % node[midway, right]{\tiny $\catvariableof{1}$};
\node[anchor=center] (text) at (-2,-8) {\tiny $\catvariableof{1}$};

\draw[fill] (7,-6) circle (0.15cm);
\draw (7,-5) -- (7,-6);
\draw (7,-6)--(7,-7) node[midway,right] {\tiny $\catvariableof{2}$}; 

\draw (6,-7) rectangle (8,-9);
\node[anchor=center] (text) at (7,-8) {\small $\ones$};

\end{scope}


\end{tikzpicture}
	\end{center}
	\caption{
		Example of a tensor network contraction of all but the variables $\catvariableof{1},\catvariableof{3}$.
		Contraction of variables can always be depicted by closing the open legs with trivial tensors $\ones$ performing index sums.
	}\label{fig:contraction}
\end{figure}

%%
%Diagrammatic notation: Best to do version b), since this is easiest to see how tensors combine to new tensors by contractions.

\begin{remark}[Alternative Notations]
	% Einstein summations
	Contractions can also denoted by the Einstein summations of the indices along connected edges, understood as scalar product in each subspace.
	This is as in Definition~\ref{def:contraction}, just omitting the sums.
	We found it useful in this work to do the diagrammatic representation instead, since it offers a better possibility to depict hierarchical arrangements of shared variables.
\end{remark}


% Mode products
Further notations without usage of axis variables are mode products (see \cite{kolda_tensor_2009,hackbusch_tensor_2012,cichocki_tensor_2015}), often denoted by the operation $\times_n$.
With our more generic variable-based notations, we can capture these more specific contractions by coloring the tensor axes, that is assignment of axis variables.

% Examples
To further gain familiarity with the generic contractions, we show the connection to two more popular examples.

%% Diagrammatic representation of Matrix Vector
\begin{example}{Matrix Vector Products}
	The matrix vector product is a special case of tensor contractions, where a matrix $\matrixat{\exrandom,\secexrandom}$ shares a categorical variable with a vector $\vectorat{\secexrandom}$.
	When leaving the variable unique to the matrix open we get the matrix vector product as
		\[ \contractionof{\matrixat{\exrandom,\secexrandom},\vectorat{\secexrandom}}{\exrandom=\exrandind} = \sum_{\secexrandind\in[\secexranddim]} \matrixat{\exrandom=\exrandind,\secexrandom=\secexrandind} \cdot \vectorat{\secexrandom=\secexrandind} \, .  \]

	Exploiting the diagramatic tensor network visualization we depict matrix vector products by:
	\begin{center}
		\begin{tikzpicture}[scale=0.3,thick,xscale=-1] % , baseline = -3.5pt

\draw (-9,2)--(-7,2) node[midway,above] {\colorlabelsize $\exrandom$};
\draw (-21,1) rectangle (-9,3);
\node[anchor=center] (text) at (-15,2) {\corelabelsize $\contractionof{\matrixat{\exrandom,\secexrandom},\vectorat{\secexrandom}}{\exrandom}$};

\node[anchor=center] (text) at (-5,2) {\corelabelsize ${=}$};

\draw (3,2)--(5,2) node[midway,above] {\colorlabelsize $\exrandom$};
\draw (1,1) rectangle (3,3);
\node[anchor=center] (text) at (2,2) {\corelabelsize $\exmatrix$};
\draw (1,2)--(-1,2) node[midway,above] {\colorlabelsize $\secexrandom$};
\draw (-1,1) rectangle (-3,3);
\node[anchor=center] (text) at (-2,2) {\corelabelsize $\exvector$};

%\node[anchor=center] (text) at (7,1) {$\cdot$};


\end{tikzpicture}
	\end{center}
%	Here the index $j$ is represented by a closed edge, which means that it is eliminated by a sum.
\end{example}


%% Hadamard Product 
\begin{example}{Hadamard Products of Vectors}
	A node appearing in arbitrary many hyperedges denotes a Hadamard product of the axis of the respective decorating tensors.
	To give an example, let $\vectorofat{\catenumerator}{\catvariable}\in\rr^\catdim$ be vectors for $\catenumeratorin$. Their hadamard product is the vector
		\[ \contractionof{\{\vectorofat{\catenumerator}{\catvariable} \, : \, \catenumeratorin\}}{\catvariable}  \in \rr^\catdim \]
	defined by
		\[ \contractionof{\{\vectorofat{\catenumerator}{\catvariable} \, : \, \catenumeratorin\}}{\indexedcatvariable}   
		= \prod_{\atomenumeratorin} \vectorofat{\atomenumerator}{\indexedcatvariable}\, . \]
	In a contraction diagram the Hadamard product is depicted by 
	\begin{center}
		\begin{tikzpicture}[scale=0.3,thick] % , baseline = -3.5pt


\begin{scope}[shift={(-10,0)}]

\draw (-5,1) rectangle (7,3);
\node[anchor=center] (text) at (1,2) {\small $\contractionof{V^{0}[\catvariable],\ldots,V^{\catorder-1}[\catvariable]}{\catvariable}$}; % {\small $\contractionof{\{\vectorofat{\catenumerator}{\catvariable} \, : \, \catenumeratorin\}}{\catvariable}$};
\draw (1,-1)--(1,1) node[midway,right] {\tiny $\catvariable$};

\node[anchor=center] (text) at (9,2) {${=}$};

\end{scope}



\draw (1,1) rectangle (3,3);
\node[anchor=center] (text) at (2,2) {\small $\vectorof{0}$};
\draw (2,-1)--(2,1) node[midway,right] {\tiny $\catvariable$};


\begin{scope}[shift={(5,0)}]

\draw (1,1) rectangle (3,3);
\node[anchor=center] (text) at (2,2) {\small $\vectorof{1}$};
\draw (2,-1)--(2,1) node[midway,right] {\tiny $\catvariable$};

\end{scope}

\node[anchor=center] (text) at (11.5,2) {\small $\cdots$};


\begin{scope}[shift={(15,0)}]

\draw (0.75,1) rectangle (3.25,3);
\node[anchor=center] (text) at (2,2) {\small $\vectorof{\atomorder\shortminus1}$};
\draw (2,-1)--(2,1) node[midway,right] {\tiny $\catvariable$};

\end{scope}


\draw[fill] (9.125,-4.5) circle (0.15cm);

\draw (9.125,-4.5) to[bend right=-20] (2,-1); 
\draw (9.125,-4.5) to[bend right=-20] (7,-1); 
\draw (9.125,-4.5) to[bend right=20] (17,-1); 

\draw (9.125,-4.5) -- (9.125,-6.5) node[midway,right] {\tiny $\catvariable$};; 

\end{tikzpicture}
	\end{center}
\end{example}



\subsubsection{Decompositions}

Tensors can be represented by tensor network decompositions, when the contraction of the network retrieves the tensor.

\begin{definition}\label{def:tnDecomposition}
	%Let $\hypercoreat{\nodevaraibles}$ be a tensor in $\extensorspace$.
	A Tensor Network Decomposition of a tensor $\hypercoreat{\nodevariables}$ is a Tensor Network $\tnetof{\graph}$ such that
		\[ \hypercoreat{\nodevariables}= \contractionof{\tnetof{\graph}}{\nodevariables} \, . \]
	We call the hypergraph $\graph$ the format of the decomposition.
\end{definition}



\subsection{Properties of Tensors}

%% Boolean
We will often encounter situations, where the coordinates of tensors are in $\{0,1\}=[2]$.

\begin{definition}\label{def:booleanTensor} % CALL BOOLEAN INSTEAD?
	We call a tensor $\hypercoreat{\shortcatvariables}$ boolean, when $\imageof{\hypercore}\subset[2]$, i.e. all coordinates are either $0$ or $1$.
\end{definition}

%% Directionality
Directionality represents constraints on the structure of tensors:
Summing over outgoing trivializes the tensor.

\begin{definition}\label{def:directedTensor}
	A Tensor 
		\[ \hypercoreat{\nodevariables} \in \bigotimes_{\nodein}\rr^{\catdimof{\node}} \]
	is said to be directed with incoming variables $\innodes$ and outgoing variables $\outnodes$, where $\nodes=\innodes\dot{\cup}\outnodes$, when
		\[ \sbcontractionof{\hypercore}{\catvariablesinset{\outnodes}} =  \onesat{\catvariablesinset{\innodes}} \]
	where $\onesat{\catvariablesinset{\innodes}}$ denoted the trivial tensor in  $\bigotimes_{\node\in\innodes}\rr^{\catdimof{\node}}$ which coordinates are all $1$.
\end{definition}

While by default all legs are outgoing, we can change the direction by normation.

\begin{definition}\label{def:normation}
	A tensor $\hypercoreat{\nodevariables}$ is said to be normable on $\innodes\subset\nodes$, if for any $\catindexof{\innodes}\in\nodestatesof{\innodes}$ we have
		\[ \sbcontraction{\hypercoreat{\nodevariables},\onehotmapofat{\atomlegindexof{\innodes}}{\catvariableof{\innodes}}} > 0 \, . \]
	The normation of a on $\innodes\subset\nodes$ normable tensor is the tensor
	\begin{align*}
		\sbnormationofwrt{\hypercoreat{\nodevariables}}{\catvariableof{\outnodes}}{\catvariableof{\innodes}} = 
		\sum_{\catindexof{\innodes}\in\nodestatesof{\innodes}} 
		\onehotmapofat{\atomlegindexof{\innodes}}{\catvariableof{\innodes}} \otimes \frac{
		\sbcontractionof{\hypercoreat{\nodevariables},\onehotmapofat{\catindexof{\innodes}}{\catvariableof{\innodes}}}{\catvariableof{\outnodes}}
		}{
		\sbcontraction{\hypercoreat{\nodevariables},\onehotmapofat{\catindexof{\innodes}}{\catvariableof{\innodes}}}
		} 
	\end{align*}
	where $\outnodes = \nodes/\innodes$.
\end{definition}

We will investigate the contractions of directed tensors in Part~III, where we show in Theorem~\ref{the:normationDirected} that normations are directed tensors.


%% Diagrammatic notation
In our graphical tensor notation, we depict directed tensors by directed hyperedges (a), which are decorated by directed tensors (b), for example:
%\red{Draw incoming and outgoing example.}
	\begin{center}
		


\begin{tikzpicture}[scale=0.35,thick] % , baseline = -3.5pt

\node[anchor=center] (text) at (-2,0) {$a)$};

\node [circle, draw, thick, fill=\nodegrayscale, minimum size = \nodeminsize] (P1) at (0,-3) {\colorlabelsize $\catvariableof{0}$};
\node [circle, draw, thick, fill=\nodegrayscale, minimum size = \nodeminsize] (P2) at (3,-3) {\colorlabelsize $\catvariableof{1}$};

%\node[anchor=center] (text) at (6,-3) {$\cdots$};
\node [circle, draw, thick, fill=\nodegrayscale, minimum size = \nodeminsize] (P3) at (6,-3)  {\colorlabelsize $\catvariableof{2}$};

\node [circle, draw, thick, fill=\nodegrayscale, minimum size = \nodeminsize] (P4) at (9,-3)  {\colorlabelsize $\catvariableof{3}$};

\node[anchor=center] (text) at (9,-3) {\colorlabelsize $\catvariableof{3}$};


\draw[midarrow] 
    	(4.5,0) to[bend right=25] (P1);
\draw[midarrow] 
    	(4.5,0) to[bend right=10] (P2);
\draw[midarrow] 
    	(P3) to[bend right=10] (4.5,0);
\draw[midarrow] 
	(P4) to[bend right=25] (4.5,0);
	
\node[anchor=center] (text) at (4.5,0.5) {$\edge$};


\begin{scope}[shift={(20,0)}]

\node[anchor=center] (text) at (-2,0) {$b)$};

\draw (-1,-1) rectangle (7,-3);
\node[anchor=center] (text) at (3,-2) {\corelabelsize $\hypercoreofat{\edge}{\catvariableof{0},\catvariableof{1},\catvariableof{2},\catvariableof{3}}$};
%\draw[->-] (0,-3)--(0,-5) node[midway,left] {\colorlabelsize $\catvariableof{0}$};
%\draw[->-] (1.5,-3)--(1.5,-5) node[midway,left] {\colorlabelsize $\catvariableof{1}$};
%\node[anchor=center] (text) at (3,-4) {$\cdots$};
%\draw[->-] (4,-3)--(4,-5) node[midway,right] {\colorlabelsize $\catvariableof{\atomorder-1}$};


\draw[midarrow]  (0,-3) -- (0,-5) node[midway,left] {\colorlabelsize $\catvariableof{0}$};
\draw[midarrow] 
    (2,-3)--(2,-5) node[midway,left] {\colorlabelsize $\catvariableof{1}$};
\draw[midarrow] 
    (4,-5)--(4,-3) node[midway,left] {\colorlabelsize $\catvariableof{2}$};
\draw[midarrow] 
   (6,-5)--(6,-3) node[midway,right] {\colorlabelsize $\catvariableof{3}$};
\end{scope}



\end{tikzpicture}
	\end{center}



\subsection{Encoding schemes for functions}

Tensors are defined here as real-valued functions on the state set of a system described by categorical variables.
We provide further schemes to represent functions in order to perform sparse calculus and to handle more generic functions.



%
%\subsubsection{Real-valued functions}
%\begin{example}[Uncertainty about States]\label{exa:onehotUncertainty}
%	The uncertainty about the state of a categorical variable $\catvariable$ can be expressed in vectors.
%	For example let there be real numbers $\probof{\catvariable=\catindex} \in [0,1]$ for $\catindex\in[\catdim]$ with $\sum_{\catindex\in[\catdim]}\probof{\catvariable=\catindex}=1$ with the interpretation that $\probof{\catvariable=\catindex}$ is the probability of a system being in state $\catindex$. 
%	We can represent this uncertain state simply by a vector 
%		\[ \probof{\catvariable}\in\rr^{\catdim} \]
%	defined as the sum of one-hot representations weighted by $\probof{\catvariable=\catindex}$
%	\[ \sum_{\catindex\in[\catdim]} \probof{\catvariable=\catindex} \cdot \onehotmapofat{\catindex}{\catvariable} =
%		\begin{bmatrix}
%		\probof{\catvariable=0} & \probof{\catvariable=1} & \cdots & \probof{\catvariable=\catdim-1}
%		\end{bmatrix} \, . 
%	\]
%\end{example}



\subsubsection{Relational encodings}

%We have already observed in Example~\ref{exa:atomicFunction}, that any function of a categorical variable has a representation as a linear function acting on the one-hot encoding of the variable.
Let us now show how we can encode maps between factored systems.
The scheme is described in more generality and detail (encoding of subsets and relations) in Chapter~\ref{cha:tensorEncodings}, see Definition~\ref{def:functionRelationEncoding}.

\begin{definition}[Relation encoding of maps between Factored Systems]\label{def:functionRepresentation}
	Let $\exfunction$ be a function
		\[ \exfunction : \facstates \rightarrow  \secfacstates \]
	which maps the states of a factored system to variables $\catvariables$ to the states of another factored system with variables $\seccatvariables$.
	Then the tensor representation of $\exfunction$ is a tensor
		\[ \rencodingofat{\exformula}{\catvariables,\seccatvariables} \in   \left(\facspace\right) \otimes \left(\secfacspace\right)  \]
	defined by
		\[ \rencodingofat{\exformula}{\catvariables,\seccatvariables}= \sum_{\catindices\in\facstates}  
		  \onehotmapofat{\catindices}{\catvariables} \otimes \onehotmapofat{\exfunction(\catindices)}{\seccatvariables} \, . \]
\end{definition}



% Notation with image categorical variable
%When the categorical variables of the image factored system to a map $\exfunction$ are not specified otherwise, we will denote them by $\catvariableof{\exfunction}$.




\subsubsection{Tensor-valued functions}


%% TO DETAILLED HERE -> Part III?
\begin{definition}[Selection encoding of Maps between Factored Systems]\label{def:selectionEncoding}
	Given a tensor space $\parspace$ described by categorical variables $\selvariables$ and a tensor-valued function
		\[ \exfunction : \facstates \rightarrow \parspace \]
	the selection encoding of $\exfunction$ is a tensor
		\[ \sencodingofat{\exfunction}{\shortcatvariables,\shortselvariables} \in \left(\facspace\right) \otimes \left(\parspace\right) \]
	defined by the basis decomposition
		\[ \sencodingofat{\exfunction}{\shortcatvariables,\shortselvariables} = \sum_{\catindices\in\facstates} \onehotmapofat{\catindices}{\shortcatvariables} \otimes \exfunction(\catindices)[\shortselvariables] \, .  \]
\end{definition}

%%
We call these tensor representation of maps selection encodings, since the coordinate of a function $\exfunction$ to be processed is selected by another argument to $\sencodingof{\exfunction}$.

%\begin{example}[Vector valued functions]\label{exa:atomicFunction} %% CONFUSIN, since already needs selection variables?
%	When using a one-hot representation of the state of a categorical variable, any real valued function has a representation by a real valued matrix acting on the one-hot encoding. 
%	Let there be a vector valued function
%		\[ \exformula : [\catdim] \rightarrow \rr^p \]
%	which maps $\catindex\in[\catdim]$ to the vector
%		\[ \exformula(\catindex)[\selvariable] \in \rr^p \, , \]
%	where we introduced the variable $\selvariable\in[p]$ selecting a coordinate of the image vector.
%	The 
%		\[ \exformula(\catindex)[\selvariable] = 
%		\contractionof{\{\onehotmapof{\catindex}[\catvariable] , \,\concore_{\exformula}[\catvariable,\selvariable]\}}{\selvariable}  \]
%	where $\concore_{\exformula} \in \rr^{\catdim \times p} $ is the matrix defined by the function evaluation vectors of $\exformula$ as
%		\[ \concore_{\exformula}[\catvariable,\selvariable] = \begin{bmatrix}
%			-- & \exformula(0) & -- \\
%			-- & \exformula(1) & -- \\
%			& \vdots &  \\
%			-- & \exformula(\catdim-1) & -- 
%		\end{bmatrix} \, . 
%		\]
%	This can easily be verified, since matrix multiplication with basis vectors amounts to selection of rows (when the basis vector is acting from the left) or columns (when the basis vector is acting from the right).
%	Thus, linear transforms (matrices) acting on the one-hot representation are sufficient to represent any vector valued function of the states of a categorical variable.
%\end{example} 


%% 
We will provide more detail to the tensor representation of functions in Part~III, where we distinguish between embeddings for basis and coordinate calculus. %where we show that domain encodings coincide with selection encodings.







% Tensor-Network Based Reasoning
\part{Decomposition and Inference of Factored Representations}

The computational automation of reasoning is rooted both in the probabilistic and the logical reasoning tradition.
Both draw on the same ontological commitment that systems have a factored structure, that is their states are described by assignments to a set of variables.
Based on this commitment both approaches bear a natural tensor representation of their states and a formalism of the respective reasoning algorithms based on multilinear methods.
%We discuss them in this part separated from each other, and unify them in the next part by Markov Logic Networks.



% Parametrization of Probability Distributions
\section{Probability Distributions}\label{cha:probDecomposition}

In this chapter we will establish relations between the formalism of tensor networks and basic concepts of probability theory.
We will first understand distributions as tensors and connect their marginalizations and conditionings to the tensor operations of contractions and normations.
Then we discuss independence assumptions as examples of contraction equations, which lead to tensor network decompositions known as graphical models.
We then treat more generic exponential families and investigate their representation as tensor networks.

\subsection{Tensor Representation of Distributions}

%% Random Variables: Introduction in Bayesian way by uncertainties
After having discussed how to represent states of factored systems by one-hot encodings, let us now take advantage of these representation by associating properties with these states.
Let there be uncertainties of the assignments $\catindexof{\atomenumerator}$ to the categorical variables $\catvariableof{\atomenumerator}$ of a factored system.
We then understand $\catvariableof{\atomenumerator}$ as random variables, which have a joint distribution defined by the uncertainties of the state assignments.
To capture these uncertainties we now make use of the one-hot representation of factored systems in Chapter~\ref{cha:factoredRepresentation}.

\begin{definition}[Probability Tensor] % From the axioms of Kolmogorov!
	Let there be a factored system defined by a categorical variable $\catvariableof{\atomenumerator}$ for each $\atomenumeratorin$ taking values in $[\catdimof{\atomenumerator}]$. 
	A probability distribution over the states of $\facsystem$ is a tensor
		\[ \probat{\catvariableof{0},\ldots,\catvariableof{\atomorder-1}} : \facstates \rightarrow [0, 1] \subset \rr \]
	such that
		\[ \sum_{\catindices\in\facstates} \probat{\indexedcatvariables} = 1 \, . \]
\end{definition}

We notice that there are two conditions for a tensor to be probability tensor.
First, the tensor needs to have non-negative coordinates and second, the coordinates need to sum to $1$.

%% One-hot Decomposition -> Contraction Equivalences
The probability tensor to the distribution is an object
		\[ \probat{\catvariables} \in \bigotimes_{\atomenumeratorin}\rr^{\catdimof{\atomenumerator}} \]
which is the sum over the one-hot encodings (see Lemma~\ref{lem:tensorBasisDecomposition})
		\[ \probat{\catvariables} = \sum_{\catindices\in\facstates} \probat{\indexedcatvariables} \cdot \onehotmapofat{\catindices}{\catvariables} \, . \]
		
%%
The normation condition of probability tensors can be expressed by the contraction equation $1= \sbcontraction{\probtensor}$ since
\begin{align*}
	1 = \sum_{\catindices}\probat{\indexedcatvariables}
	=  \sum_{\catindices}\sbcontraction{\probtensor, \onehotmapof{\catindices}}
	= \sbcontraction{\probtensor} \, . 
\end{align*}

%% NOT NEEDED
%Using the Coordinate Calculus as described in Theorem~\ref{the:coordinateCalculus} we can retrieve the coordinates of $\probtensor$ storing the probabilities of specific states by the contraction
%\begin{align*}
%	\probat{\indexedcatvariables} = \contractionof{\{\probtensor, \onehotmapof{\catindices}\}}{\varnothing} \, . 
%\end{align*}

%% Coordinates
%The probability tensor stores all probabilities on its coordinates, which are by construction
%	\[ \probtensor_{\catindices} = \probat{\catvariableof{\atomenumerator} = \catindexof{\atomenumerator} \, : \, \atomenumeratorin}  \, . \]
%We here draw on the redundancy of the one-hot encoding of each state of a factored system, which enables us to represent the properties of multiple states in single tensors (see Example~\ref{exa:onehotUncertainty}).

Probability tensors are depicted as
\begin{center}
	


\begin{tikzpicture}[scale=0.35,thick] % , baseline = -3.5pt

\node[anchor=center] (text) at (-2,0) {$a)$};

\node [circle, draw, thick, fill=gray!50, minimum size = \nodeminsize] (P1) at (0,-3) {\tiny $\catvariableof{0}$};	
\node [circle, draw, thick, fill=gray!50, minimum size = \nodeminsize] (P2) at (3,-3) {\tiny $\catvariableof{1}$};

\node[anchor=center] (text) at (6,-3) {$\cdots$};

\node [circle, draw, thick, fill=gray!50, minimum size = \nodeminsize] (P3) at (9,-3) {};

\node[anchor=center] (text) at (9,-3) {\tiny $\catvariableof{\atomorder-1}$};


\draw[midarrow] 
    	(4.5,0) to[bend right=25] (P1);
\draw[midarrow] 
    	(4.5,0) to[bend right=10] (P2);
\draw[midarrow] 
	(4.5,0) to[bend right=-25] (P3);
	
\node[anchor=center] (text) at (4.5,0.5) {$\edge$};


\begin{scope}[shift={(20,0)}]

\node[anchor=center] (text) at (-2,0) {$b)$};

\draw (-1,-1) rectangle (5,-3);
\node[anchor=center] (text) at (2,-2) {\small $\probtensor$};
%\draw[->] (0,-3)--(0,-5) node[midway,left] {\tiny $\catlegof{0}$}; 
%\draw[->] (1.5,-3)--(1.5,-5) node[midway,left] {\tiny $\catlegof{1}$}; 
\node[anchor=center] (text) at (3,-4) {$\cdots$};
%\draw[->] (4,-3)--(4,-5) node[midway,right] {\tiny $\catlegof{\atomorder-1}$}; 


\draw[midarrow]  (0,-3) -- (0,-5) node[midway,left] {\tiny $\catlegof{0}$};
\draw[midarrow] 
    (1.5,-3)--(1.5,-5) node[midway,left] {\tiny $\catlegof{1}$}; 
\draw[midarrow] 
   (4,-3)--(4,-5) node[midway,right] {\tiny $\catlegof{\atomorder-1}$}; 
\end{scope}



\end{tikzpicture}
\end{center}

\subsubsection{Base measures}


From a measure theoretic perspective, probabilities are measurable functions called probability densities, which integrals are $1$. % Add citations?
In our case of finite dimensional state spaces of factored systems, we implicitly used the trivial tensor $\onesat{\shortcatvariables}$ as a base measure, which measures subsets of states by their cardinality and is therefore refered to as state counting base measure.
The distribution tensors $\probat{\shortcatvariables}$ can then be understood as probability densities with respect to this state counting base measure.
We in this work will also consider more general base measures $\basemeasureat{\shortcatvariables}$, which we restrict to be boolean, that is $\basemeasureat{\indexedshortcatvariables}\in\ozset$ for all states $\shortcatindices$.
When understanding $\probtensor$ as a probability density with respect to $\basemeasure$, any probabilistic interpretation will be through the contraction $\contractionof{\probtensor,\basemeasure}{\shortcatvariables}$ and the normation condition reads as
	\[ \contraction{\probtensor,\basemeasure} = 1 \, . \]
Since we restrict to boolean base measures, the contraction effectively manipulates the tensor $\probtensor$ by setting the coordinates $\probat{\indexedshortcatvariables}$ to zero, when $\basemeasureat{\indexedshortcatvariables}=0$.
Therefore, multiple tensors $\probtensor$ will have the same proabilistic interpretation, when $\basemeasureat{\shortcatvariables}\neq\onesat{\shortcatvariables}$.
To avoid this ambiguity, we introduce the notation of representability with respect to a base measure $\basemeasure$, by demanding that such coordinates are zero.

\begin{definition}\label{def:representationBaseMeasure}
	We say that a probability distribution $\probtensor$ is representable with respect to a boolean base measure $\basemeasure$, if for all $\shortcatindices$ with $\basemeasureat{\indexedshortcatvariables}=0$ we have $\probat{\indexedshortcatvariables}=0$.
\end{definition}

When a probability distribution $\probtensor$ is representable with respect to a boolean base measure $\basemeasure$, we have the invariance
	\[ \probat{\shortcatvariables} =  \contractionof{\probtensor,\basemeasure}{\shortcatvariables} \]
and can therefore safely ignore the base measures.

Starting with \charef{cha:logicalRepresentation} we will further investigate boolean tensors and relate them with propositional formulas.
In \charef{cha:logicalReasoning} we will connect the representation and positivity with respect to boolean base measures with the formalism of entailment.

% Positive distribution
We now investigate, which base measures $\basemeasure$ can be chosen for a probability distribution $\probtensor$, such that $\probtensor$ is representable by $\basemeasure$.
Here we want to finde a $\basemeasure$, which is in a sense to be defined minimal amount the base measures, such that $\probtensor$ is representable with respect to them.
For this minimality criterion we will develop in \charef{cha:logicalReasoning} orders based on entailment and show the minimality in \theref{the:minimalRepPosBaseMeasure}.
Here, we just introduce the minimality criterion as positivity of a distribution with respect to a base measure.

\begin{definition}\label{def:positivityBaseMeasure}
	We say that a probability distribution $\probtensor$ is positive with respect to a boolean base measure $\basemeasure$, if the distribution is representable by $\basemeasure$ (i.e. $\contraction{\probtensor,\basemeasure}=1$) and for all $\shortcatindices$ with $\basemeasureat{\indexedcatvariables}=1$ we have $\probat{\indexedcatvariables}>0$.
\end{definition}

%This is a slide abuse of the measure theoretic approach to probability theory, since typically the base measure needs to be defined before considering probability distributions. 


\subsection{Marginal Distribution}

Contractions of probability distributions are related to marginalizations as we introduce next.

\begin{definition}[Marginal Probability]\label{def:marginalProbability}
	Given a distribution $\probat{\exrandom,\secexrandom}$ of the categorical variables $\exrandom$ and $\secexrandom$ the marginal distribution of the categorical variable $\exrandom$ is defined for each $\exrandind$ as the tensor
	\begin{align*}
		\probat{\exrandom} : [\exranddim] \rightarrow \rr
	\end{align*}
	defined for $\exrandind\in[\exranddim]$ by
	\begin{align*}
		\probat{\indexedexrandom} 
		= \sum_{\secexrandind\in[\secexranddim]} \probat{\indexedexrandom,\indexedsecexrandom} \, .
	\end{align*}
\end{definition}

% Sets of variables
\defref{def:marginalProbability} generalizes to marginalizations of sets of variables, since we can always group a set of categorical variables and understand them as a single one.

%% Contractions
\begin{theorem}\label{the:marginalContraction}
	%Given a Tensor Network (see \defref{def:tensorNetwork}) $\{\probtensor\}$ consistent of the variables $\exrandom,\secexrandom$ and hyperedge $\{\exrandom,\secexrandom\}$ decorated with the tensor $\probtensor$.
	For any distribution $\probat{\exrandom,\secexrandom}$ the marginal distribution of the variable $\catvariable$ is the contraction
	\begin{align*}
		\probat{\exrandom} = \sbcontractionof{\probtensor}{\exrandom} \, .
	\end{align*}
	Further, any marginal distribution is a probability distribution.
\end{theorem}
\begin{proof}
	We have $\probat{\exrandom} = \contractionof{\{\probtensor\}}{\exrandom}$ by definition.
	To show that $\probat{\exrandom}$ is a probability distribution, we need to show that $\sbcontraction{\probat{\exrandom}}=1$.
	But this follows from the normation of $\probtensor$ and the commutativity of contractions (see Theorem~\ref{the:splittingContractions} in Chapter~\ref{cha:localContractions}) as
		\[ \sbcontraction{\probat{\exrandom}} = 
		\sbcontraction{
			\sbcontractionof{\probtensor}{\exrandom}
		} =
		 \sbcontraction{\probtensor}
		= 1 \, . 
		\]
\end{proof}

%% Tensor Representation
We depict the sum over the possible values of $\secexrandom$ by contraction of the probability tensor with the trivial tensors $\ones$ as 
\begin{center}
	\begin{tikzpicture}[scale=0.3,thick] % , baseline = -3.5pt

\draw (-19,-1) rectangle (-15,-3);
\node[anchor=center] (text) at (-17,-2) {\small $\margprobof{\exrandom}{\exrandom}$};
\draw[midarrow]  (-17,-3)--(-17,-5) node[midway,left] {\tiny $\exrandom$}; 

\node[anchor=center] (text) at (-13,-2) {${=}$};

\draw (-11,-1) rectangle (-5,-3);
\node[anchor=center] (text) at (-8,-2) {\small $\probof{\exrandom,\secexrandom}$};
\draw[midarrow]  (-10,-3)--(-10,-5) node[midway,left] {\tiny $\exrandom$}; 
\draw[midarrow]  (-6,-3)--(-6,-5) node[midway,left] {\tiny $\secexrandom$};
\draw (-7,-5) rectangle (-5,-7); 
\node[anchor=center] (text) at (-6,-6) {$\ones$};

\end{tikzpicture}
\end{center}
Let us notice, that marginal distributions are probability tensors for themself, which we again denote by a directed leg.
%We here omit the denotation of the nodes in the hypergraph of a Tensor Network and represent a Tensor Network just by the appearing Tensor Cores on the hyperedge.


\subsection{Conditional Probabilities}

Normations of probability distributions result in conditional distributions as we define next.

\begin{definition}[Conditional Probability]\label{def:conditionalProbability}
	Let $\probat{\exrandom,\secexrandom}$ be a distribution of the categorical variables $\exrandom$ and $\secexrandom$, such that $\probtensor$ is normable on $\{\secexrandom\}$.
	Then the distribution of $\exrandom$ conditioned on $\secexrandom$ is defined by
		\[ \condprobof{\indexedexrandom}{\indexedsecexrandom}  
		= \frac{\probat{\indexedexrandom,\indexedsecexrandom}}{\probat{\indexedsecexrandom}} \, . \]
\end{definition}

%The conditional probability
%	\[ \condprobof{\exrandom}{\indexedsecexrandom}  
%	= \frac{\probat{\exrandom,\indexedsecexrandom}}{\probat{\indexedsecexrandom}} \]
%is also a tensor with legs to $\exrandom$ and $\secexrandom$.
%For each one-hot encoding $\onehotmapof{\secexrandind}$ of the assignment $\secexrandind$ to the variable $\secexrandom$ we represent the conditional probability by the diagrams
%\begin{center}
%	\begin{tikzpicture}[scale=0.3, thick] % , baseline = -3.5pt

\draw (-21,-1) rectangle (-15,-3);
\node[anchor=center] (text) at (-18,-2) {\small $\condprobof{\exrandom}{\secexrandom}$};
\draw[midarrow]  (-20,-3)--(-20,-5) node[midway,left] {\tiny $\exrandom$}; 

\draw[midarrow]  (-16,-5)--(-16,-3) node[midway,left] {\tiny $\secexrandom$}; 
\draw[dashed] (-15,-5) rectangle (-17,-7); 
\node[anchor=center] (text) at (-16,-6) {\small $\onehotmapof{\secexrandind}$};

\node[anchor=center] (text) at (-13,-2) {${=}$};


\begin{scope}[shift={(0,6)}]

\draw (-11,-1) rectangle (-5,-3);
\node[anchor=center] (text) at (-8,-2) {\small $\probof{\exrandom,\secexrandom}$};
\draw[midarrow]  (-10,-3)--(-10,-5) node[midway,left] {\tiny $\exrandom$}; 
\draw[midarrow]  (-6,-3)--(-6,-5) node[midway,left] {\tiny $\secexrandom$};
\draw[dashed] (-7,-5) rectangle (-5,-7); 
\node[anchor=center] (text) at (-6,-6) {\small $\onehotmapof{\secexrandind}$};

\end{scope}

\draw (-12,-2) -- (-4,-2);

\begin{scope}[shift={(0,-2)}]

\draw (-11,-1) rectangle (-5,-3);
\node[anchor=center] (text) at (-8,-2) {\small $\probof{\exrandom,\secexrandom}$};
\draw[midarrow]  (-10,-3)--(-10,-5) node[midway,left] {\tiny $\exrandom$}; 
\draw (-11,-5) rectangle (-9,-7); 
\node[anchor=center] (text) at (-10,-6) {$\ones$};
\draw[midarrow]  (-6,-3)--(-6,-5) node[midway,left] {\tiny $\secexrandom$};
\draw[dashed] (-7,-5) rectangle (-5,-7); 
\node[anchor=center] (text) at (-6,-6) {\small $\onehotmapof{\secexrandind}$};

\end{scope}

\end{tikzpicture}
%\end{center}
%Here we denote by the quotient a coordinatewise normation, as sketched by the dashed unit vector. % is contracted before each normation, but we will omit it in future diagrams.
%We depict conditional variables by directed edges, where legs to conditions are incoming while the others outgoing.

%% Normation and Directed Notation
We show in the next theorem, that conditional distributions are calculated by normations.
%We will discuss operations on tensors like conditioning more detail in Chapter~\ref{cha:directedTC} as normation operation of \defref{def:normation}.
%In Theorem~\ref{the:conditionalContraction} we will show that the resulting tensor is directed with incoming variables by the conditions.

\begin{theorem}\label{the:conditionalContraction}
	The tensor $\condprobof{\exrandom}{\secexrandom}$ is the normation of $\probat{\exrandom,\secexrandom}$ on $\secexrandom$  (see \defref{def:normation}), that is
	\begin{align*}
		\condprobof{\exrandom}{\secexrandom}   
		= \sbnormationofwrt{\probtensor}{\exrandom}{\secexrandom} \, . 
	\end{align*}
	Further, for any $\secexrandind\in[\secexranddim]$ the tensor $\condprobof{\exrandom}{\indexedsecexrandom}$ is a probability tensor.
\end{theorem}
\begin{proof}
	The first claim follows from a comparison of \defref{def:conditionalProbability} and \ref{def:normation}.
	The second claim follows from the first and Theorem~\ref{the:normationDirected}.
	Alternatively, the second claim can be showed using the diagrammatic notation as
	\begin{center}
		\begin{tikzpicture}[scale=0.3,thick] % , baseline = -3.5pt

\node[anchor=center] (text) at (-30,-2) {\small $\sum_{\atomlegindexof{\exrandom}} \, \condprobof{X=\atomlegindexof{\exrandom}}{Y=\atomlegindexof{\secexrandom}} \quad {=}$};

\draw (-21,-1) rectangle (-15,-3);
\node[anchor=center] (text) at (-18,-2) {\small $\condprobof{X}{Y}$};
\draw[->]  (-20,-3)--(-20,-5) node[midway,left] {\tiny $X$}; 

\draw[<-]  (-16,-3)--(-16,-5) node[midway,left] {\tiny $Y$}; 
\draw[] (-15,-5) rectangle (-17,-7); 
\node[anchor=center] (text) at (-16,-6) {\small $\onehotmapof{\catindexof{Y}}$};

\draw (-21,-5) rectangle (-19,-7); 
\node[anchor=center] (text) at (-20,-6) {$\ones$};

\node[anchor=center] (text) at (-13,-2) {${=}$};


\begin{scope}[shift={(0,6)}]

\draw (-11,-1) rectangle (-5,-3);
\node[anchor=center] (text) at (-8,-2) {\small $\probof{X,Y}$};
\draw[->]  (-10,-3)--(-10,-5) node[midway,left] {\tiny $X$}; 
\draw (-11,-5) rectangle (-9,-7); 
\node[anchor=center] (text) at (-10,-6) {$\ones$};
\draw[->]  (-6,-3)--(-6,-5) node[midway,left] {\tiny $Y$};
\draw[] (-7,-5) rectangle (-5,-7); 
\node[anchor=center] (text) at (-6,-6) {\small $\onehotmapof{\catindexof{Y}}$};

\end{scope}

\draw (-12,-2) -- (-4,-2);

\begin{scope}[shift={(0,-2)}]

\draw (-11,-1) rectangle (-5,-3);
\node[anchor=center] (text) at (-8,-2) {\small $\probof{X,Y}$};
\draw[->]  (-10,-3)--(-10,-5) node[midway,left] {\tiny $X$}; 
\draw (-11,-5) rectangle (-9,-7); 
\node[anchor=center] (text) at (-10,-6) {$\ones$};
\draw[->]  (-6,-3)--(-6,-5) node[midway,left] {\tiny $Y$};
\draw[] (-7,-5) rectangle (-5,-7); 
\node[anchor=center] (text) at (-6,-6) {\small $\onehotmapof{\catindexof{Y}}$};

\end{scope}

%\node[anchor=center] (text) at (-3,-2) {${=}$};
%
%\draw (-1,-3) rectangle (1,-1); 
%\node[anchor=center] (text) at (0,-2) {$\ones$};
%\draw[<-]  (0,-3)--(0,-5) node[midway,left] {\tiny $Y$};
%\draw[] (-1,-5) rectangle (1,-7); 
%\node[anchor=center] (text) at (0,-6) {\small $\onehotmapof{\catindexof{Y}}$};

\node[anchor=center] (text) at (-1,-2) {${=}\quad 1 \, .$};

%\node[anchor=center] (text) at (9,-7) {${.}$};

\end{tikzpicture}
	\end{center}
\end{proof}



% Contraction Formalism
Theorem~\ref{the:marginalContraction} and \ref{the:conditionalContraction} show that the formalism of contractions and normations is applied in basic operations of probabilistic reasoning.

We can further show, that exactly the directed tensors with non-negative coordinates are conditional probability tensors.

\begin{theorem}\label{the:conditionalDirected}
	Any tensor with non-negative coordinates is a conditional distribution tensor, if and only if it is directed with the condition variables ingoing and the other outgoing.
\end{theorem}
\begin{proof}
	\proofrightsymbol:
	By Theorem~\ref{the:conditionalContraction} a conditional probability tensor $\condprobof{\exrandom}{\secexrandom}$ is the normation of a tensor and by Theorem~\ref{the:normationDirected} a directed tensor.
	Since probability tensors have only non-negative coordinates, their contractions with one-hot encodings also have only non-negative coordinates and also their normations. 
	
	\proofleftsymbol:
	Conversely, let $\hypercoreat{\nodevariables}$ be a directed tensor with $\innodes$ incoming and $\outnodes$ outgoing and non-negative coordinates.
	Then
	\begin{align}
		\probat{\nodevariables} = \frac{1}{\prod_{\node\in\innodes}\catdimof{\node}} \cdot \hypercoreat{\nodevariables}
	\end{align}
	is a probability tensor, since 
	\begin{align*}
		\sum_{\atomlegindexof{\innodes}} \sum_{\atomlegindexof{\outnodes}} \probat{\indexedcatvariableof{\nodes}} =
		\sum_{\atomlegindexof{\innodes}} \sum_{\atomlegindexof{\outnodes}} \frac{1}{\prod_{\node\in\innodes}\catdimof{\node}} \cdot \hypercoreat{\indexedcatvariableof{\nodes}} =
		\sum_{\atomlegindexof{\innodes}} \frac{1}{\prod_{\node\in\innodes}\catdimof{\node}} = 1 \, . 
	\end{align*}
	The conditional probability $\condprobof{\catvariableof{\outnodes}}{\catvariableof{\innodes}}$ coincides with $\hypercore$, since
	\begin{align*}
		\condprobof{\catvariableof{\outnodes}}{\indexedcatvariableof{\innodes}} 
		=& \frac{
		\probat{\catvariableof{\outnodes},\indexedcatvariableof{\innodes}}
		}{
		\sum_{\catindexof{\outnodes}} \probat{\indexedcatvariableof{\outnodes},\indexedcatvariableof{\innodes}}
		} \\
		=& \frac{
		\hypercoreat{\catvariableof{\outnodes},\indexedcatvariableof{\innodes}}
		}{
		\sum_{\catindexof{\outnodes}} \hypercoreat{\indexedcatvariableof{\outnodes},\indexedcatvariableof{\innodes}}
		} 
		= \hypercoreat{\catvariableof{\outnodes},\indexedcatvariableof{\innodes}} \, ,
	\end{align*}
	where in the last equation we used that the denominator is by definition trivial since $\hypercore$ is normed.
\end{proof}


Since conditional probabilities are directed tensors we therefore depict them by
\begin{center}
	\begin{tikzpicture}[scale=0.3,thick] % , baseline = -3.5pt

\draw (-21,-1) rectangle (-15,-3);
\node[anchor=center] (text) at (-18,-2) {\small $\condprobof{\exrandom}{\secexrandom}$};
\draw[midarrow]  (-20,-3)--(-20,-5) node[midway,left] {\tiny $\exrandom$}; 
\draw (-21,-5) rectangle (-19,-7);
\node[anchor=center] (text) at (-20,-6) {\small $\ones$};

\draw[midarrow]  (-16,-5)--(-16,-3) node[midway,left] {\tiny $\secexrandom$}; 

\node[anchor=center] (text) at (-13,-2) {${=}$};

\draw (-11,-1) rectangle (-9,-3);
\node[anchor=center] (text) at (-10,-2) {\small $\ones$};
\draw[midarrow]  (-10,-5)--(-10,-3) node[midway,left] {\tiny $\secexrandom$}; 

\node[anchor=center] (text) at (-8,-6) {${.}$};

\end{tikzpicture}
\end{center}


%
Theorem~\ref{the:conditionalDirected} specifies a broad class of tensors to represent conditional probabilities.
In combination with Theorem~\ref{the:rencodingDirected}, which states that relational encodings are directed, we get that any relational encoding of a function is a conditional probability tensor.

\subsection{Bayes Theorem and the Chain Rule}

So far, we have connected concepts of probability theory such as marginal and conditional probabilities with contractions and normations of tensors.
We will now proceed to show that basic theorems of probability theory translate into more general contraction equations.

\begin{theorem}[Bayes Theorem]\label{the:bayes}
	For any probability distribution $\probat{\exrandom, \secexrandom}$ with positive $\probat{\secexrandom}$ we have
	\begin{align*}
		\probat{\exrandom,\secexrandom} 
		= \contractionof{\condprobof{\exrandom}{\secexrandom},\probat{\secexrandom}}{\exrandom,\secexrandom} \, . 
	\end{align*}
\end{theorem}
\begin{proof}
	Directly from the more generic contraction equation Theorem~\ref{the:normationContractionEQ}, since by assumption of positivity of $\probat{\secexrandom}$, the tensor network $\probtensor$ is normable with respect to $\secexrandom$.
\end{proof}


Probability distributions can be decomposed into conditional probabilities, as we demonstrate in the next theorem.

\begin{theorem}[Chain Rule]\label{the:chainRule}
	For any joint probability distribution $\probtensor$ of the variables $\probat{\catvariableof{0},\ldots,\catvariableof{\atomorder-1}}$ we have
	\begin{align*}
		\probtensor = \sbcontractionof{\condprobof{\catvariableof{\atomenumerator},\ldots,\catvariableof{\atomorder-1}}{\catvariableof{0},\ldots,\catvariableof{\atomenumerator-1}}\, : \, \atomenumeratorin\}}{\enumeratedatoms} 
	\end{align*}
	where for $\atomenumerator=0$ we denote by $ \condprobof{\catvariableof{0}}{\catvariableof{0},\ldots,\catvariableof{-1}}$ the marginal distribution $\probat{\catvariableof{0}}$.
\end{theorem}
\begin{proof}
	This follows from Theorem~\ref{the:genericChainRule}.
%	We apply Theorem~\ref{the:bayes} on the distribution
%	\begin{align*}
%	\condprobof{
%	\catvariableof{\atomenumerator},\ldots,\catvariableof{\atomorder}
%	}{
%	\indexedcatvariableof{1},\ldots,\indexedcatvariableof{\atomenumerator-1}
%	} \, ,
%	\end{align*}
%	where $\atomenumeratorin$ and $\catindexof{[\atomorder]}$ are chosen arbitrarly.
%	For any $\atomenumeratorin$ we get
%	\begin{align*}
%		%\contractionof{\{
%			\condprobof{\catvariableof{\atomenumerator},\ldots,\catvariableof{\atomorder-1}}{\catvariableof{1},\ldots,\catvariableof{\atomenumerator-1}}
%		%\}}{} 
%		= \contractionof{\{
%			\condprobof{\catvariableof{\atomenumerator+1},\ldots,\catvariableof{\atomorder-1}}{\catvariableof{1},\ldots,\catvariableof{\atomenumerator-1}},
%			\condprobof{\catvariableof{\atomenumerator}}{\catvariableof{1},\ldots,\catvariableof{\atomenumerator-1}}	
%		\}}{
%			\catvariableof{[\atomorder]} 
%		} \, .
%	\end{align*}
%	Applying this equation iteratively and making use of the commutation of contractions we get for any $\atomenumeratorin$
%	\begin{align*}
%		\condprobof{\catvariableof{\atomenumerator},\ldots,\catvariableof{\atomorder-1}}{\catvariableof{1},\ldots,\catvariableof{\atomenumerator-1}}
%		= \contractionof{\{
%			\condprobof{\catvariableof{\secatomenumerator}}{\catvariableof{1},\ldots,\catvariableof{\atomenumerator-1}} \, : \, \secatomenumerator = \atomenumerator, \atomenumerator +1 , \ldots \atomorder-1
%		\}}{
%			\catvariableof{[\atomorder]} 
%		} \, .
%	\end{align*}
%	For $\atomenumerator=0$, this is the claim.
\end{proof}






\subsection{Independent Variables}

Independence leads to severe simplifications of conditional probabilities and is thus the key assumption to gain sparse decompositions.
We will demonstrate this here applying the chain rule.

\begin{definition}[Independence]\label{def:independence}
	Given a joint distribution of variables $\exrandom$ and $\secexrandom$, we say that $\exrandom$ is independent from $\secexrandom$ if for any values $\exrandind,\secexrandind$ we have
		\[ \probat{\indexedexrandom,\indexedsecexrandom} 
		= \margprobof{\indexedexrandom}{\exrandom}
		 \cdot 
		 \margprobof{\indexedsecexrandom}{\secexrandom} \, . \]
\end{definition}

We give a criterion on independence based on a contraction equation of the probability distribution in the next theorem.

\begin{theorem}\label{the:independenceProductCriterion}
	Given a probability distribution $\probtensor$, $\exrandom$ is independent from $\secexrandom$, if and only if 
	\begin{align*}
		\probat{\exrandom,\secexrandom} 
		= \sbcontractionof{\contractionof{\probtensor}{\exrandom},\contractionof{\probtensor}{\secexrandom}}{\exrandom,\secexrandom} \, . 
	\end{align*}
\end{theorem}
\begin{proof}
	By Theorem~\ref{the:marginalContraction} we know that marginal probabilities are equivalent to contracted probability distributions, i.e. $\probat{\exrandom} = \contractionof{\{\probtensor\}}{\exrandom} $.
	By orthogonality of one-hot encodings we have that
	\begin{align*}
		\forall \exrandind, \secexrandind : \quad  \probat{\indexedexrandom,\indexedsecexrandom} 
		= \margprobof{\indexedexrandom}{\exrandom}
		 \cdot 
		 \margprobof{\indexedsecexrandom}{\secexrandom} 
	\end{align*}
	is equivalent to 
	\begin{align*}
		\sum_{\exrandind}\sum_{\secexrandind} \probat{\indexedexrandom,\indexedsecexrandom} \cdot \onehotmapofat{\exrandind}{\exrandom}\onehotmapofat{\secexrandind}{\secexrandom}
		= \sum_{\exrandind}\sum_{\secexrandind} 
		\margprobof{\indexedexrandom}{\exrandom}
		 \cdot 
		 \margprobof{\indexedsecexrandom}{\secexrandom} \cdot \onehotmapofat{\exrandind}{\exrandom}\onehotmapofat{\secexrandind}{\secexrandom} \, .
	\end{align*}
	We reorder the summations and arrive at
	\begin{align*}
		\sum_{\exrandind,\secexrandind} 
		\probat{\indexedexrandom,\indexedsecexrandom} \cdot \onehotmapofat{\exrandind,\secexrandind}{\exrandom, \secexrandom}
		= \left(\sum_{\exrandind}\margprobof{\indexedexrandom}{\exrandom} \onehotmapofat{\exrandind}{\exrandom} \right)
		\cdot 
		\left( \sum_{\secexrandind}  \margprobof{\indexedsecexrandom}{\secexrandom} \cdot \onehotmapofat{\secexrandind}{\secexrandom}  \right) 
	\end{align*}
	which is by Lemma~\ref{lem:tensorBasisDecomposition} equal to the claim
	\begin{align*}
		\probat{\exrandom,\secexrandom} = \sbcontractionof{\contractionof{\probtensor}{\exrandom},\contractionof{\probtensor}{\secexrandom}}{\exrandom,\secexrandom} \, . 
	\end{align*}
\end{proof}


% Usage for tensor decompositions
Independent variables result in decompositions of $\probtensor$ in a tensor product of marginal probability tensors. 
Having pairwise independent variables reduces the degrees of freedom from exponentially many in the number of atoms to linear.

In the tensor network decomposition we depict this by
	\begin{center}
		\begin{tikzpicture}[scale=0.3,thick] % , baseline = -3.5pt


\draw (0,1) rectangle (7,-1);
\node[anchor=center] (text) at (3.5,0) {\small $\probof{\exrandom,\secexrandom}$};
\draw[->] (1,-1) -- (1,-3) node[midway, left] {\tiny $\exrandom$};
\draw[->] (6,-1) -- (6,-3) node[midway, left] {\tiny $\secexrandom$};

\node[anchor=center] (text) at (9,0) {\small ${=}$};


\begin{scope}[shift={(11,0)}]

\draw (0,1) rectangle (7,-1);
\node[anchor=center] (text) at (3.5,0) {\small $\probof{\exrandom,\secexrandom}$};
\draw[->] (1,-1) -- (1,-3) node[midway, left] {\tiny $\exrandom$};
\draw[->] (6,-1) -- (6,-3) node[midway, left] {\tiny $\secexrandom$};
\draw (5,-3) rectangle (7,-5);
\node[anchor=center] (text) at (6,-4) {\small $\ones$};

\end{scope}

\node[anchor=center] (text) at (20,0) {\small $\otimes$};

\begin{scope}[shift={(22,0)}]

\draw (0,1) rectangle (7,-1);
\node[anchor=center] (text) at (3.5,0) {\small $\probof{\exrandom,\secexrandom}$};
\draw[->] (1,-1) -- (1,-3) node[midway, left] {\tiny $\exrandom$};
\draw (0,-3) rectangle (2,-5);
\node[anchor=center] (text) at (1,-4) {\small $\ones$};
\draw[->] (6,-1) -- (6,-3) node[midway, left] {\tiny $\secexrandom$};

\end{scope}

\node[anchor=center] (text) at (31,0) {\small ${=}$};

\begin{scope}[shift={(33,0)}]

\draw (0,1) rectangle (4,-1);
\node[anchor=center] (text) at (2,0) {\small $\margprobof{\exrandom}{\exrandom}$};
\draw[->] (2,-1) -- (2,-3) node[midway, left] {\tiny $\exrandom$};

\node[anchor=center] (text) at (6,0) {\small $\otimes$};

\draw (8,1) rectangle (12,-1);
\node[anchor=center] (text) at (10,0) {\small $\margprobof{\secexrandom}{\secexrandom}$};
\draw[->] (10,-1) -- (10,-3) node[midway, left] {\tiny $\secexrandom$};


\end{scope}

\node[anchor=center] (text) at (46,-3) {\small ${.}$};

\end{tikzpicture} 
	\end{center}

Independence is a very strong assumption, which is often too restrictive.
Conditional independence instead is a less demanding assumption, when certain conditional distribution variables are independent. 
This leads to tensor network decompositions with a more realistic assumption.

\begin{definition}[Conditional Independence]\label{def:condIndependence}
	Given a joint distribution of variables $\exrandom$, $\secexrandom$ and $\thirdexrandom$, we say $\exrandom$ is independent from $\secexrandom$ conditioned on $\thirdexrandom$ if for any incides $\exrandind,\secexrandind$ and $\thirdexrandind$
		\[ \condprobof{\indexedexrandom,\indexedsecexrandom}{\indexedthirdexrandom} 
		= \condprobof{\indexedexrandom}{\indexedthirdexrandom} 
		\cdot \condprobof{\indexedsecexrandom}{\indexedthirdexrandom}   \, . \]
\end{definition}

Conditional independence is a relation between conditional probabilities and is therefore equivalent to a normation equation stated next.

\begin{theorem}[Conditional Independence as a Contraction Equation]\label{the:condIndependenceProductCriterion}
	Given a distribution $\probtensor$ of variables $\exrandom$, $\secexrandom$ and $\thirdexrandom$, the variable $\exrandom$ is independent from $\secexrandom$ if and only if the contraction equation
	\begin{align*}
		 \condprobof{\exrandom,\secexrandom}{\thirdexrandom} 
		 = \sbcontractionof{
		 \condprobof{\exrandom}{\thirdexrandom} ,\condprobof{\secexrandom}{\thirdexrandom} 
		 }{\exrandom,\secexrandom,\thirdexrandom}
	\end{align*}
	holds.
\end{theorem}
\begin{proof}
	Directly by Theorem~\ref{the:conditionalContraction} used on the conditional probabilities in \defref{def:condIndependence}.
\end{proof}

We can exploit conditional independence to find tensor network decompositions of probability tensors, as we show in the next theorem.

\begin{corollary}\label{cor:secCriterionCondIndepencence}
	If and only if $\exrandom$ is independent from $\secexrandom$ conditioned on $\thirdexrandom$ the probability distribution $\probtensor$ satisfies
		\[ \probat{\exrandom, \secexrandom, \thirdexrandom} 
		= \contractionof{
			\{ \condprobof{\exrandom}{\thirdexrandom}, \condprobof{\secexrandom}{\thirdexrandom}, \margprobof{\thirdexrandom}{\thirdexrandom} \}
		}{
			\exrandom, \secexrandom, \thirdexrandom
		} \, .
		\]
\end{corollary}
\begin{proof}
	Follows from Theorem~\ref{the:condIndependenceProductCriterion} and Theorem~\ref{the:bayes}.
%	We start with the chain rule decomposition of Theorem~\ref{the:chainRule} and have
%		\[ \probat{\exrandom,\secexrandom,\thirdexrandom} = \probat{\thirdexrandom}  \cdot \condprobof{\exrandom,\secexrandom}{\thirdexrandom} \]
%	Since $\exrandom$ is independent from $\secexrandom$ conditioned on $\thirdexrandom$ we have
%		\[ \condprobof{\exrandom,\secexrandom}{\thirdexrandom}  = \condprobof{\exrandom}{\thirdexrandom}  \cdot \condprobof{\secexrandom}{\thirdexrandom}  \, . \]
%	Converse direction similar.
\end{proof}


\begin{corollary}\label{cor:conditionDropping}
	Whenever $\exrandom$ is independent of $\secexrandom$ given $\thirdexrandom$, we have
	\begin{align*}
		\condprobof{\exrandom}{\secexrandom,\thirdexrandom} = \condprobof{\exrandom}{\thirdexrandom} \, .
	\end{align*}
\end{corollary}


\begin{figure}[h]
\begin{center}
	\begin{tikzpicture}[scale=0.3,thick] % , baseline = -3.5pt


\draw (-2,1) rectangle (7,-1);
\node[anchor=center] (text) at (2.5,0) {\small $\probof{\exrandom,\secexrandom,\thirdexrandom}$};
\draw[->] (-1,-1) -- (-1,-3) node[midway, left] {\tiny $\exrandom$};
\draw[->] (2.5,-1) -- (2.5,-3) node[midway, left] {\tiny $\secexrandom$};
\draw[->] (6,-1) -- (6,-3) node[midway, left] {\tiny $\thirdexrandom$};

\node[anchor=center] (text) at (9,0) {\small ${=}$};

\draw (11,1) rectangle (18,-1);
\node[anchor=center] (text) at (14.5,0) {\small $\condprobof{\exrandom}{\thirdexrandom}$};
\draw[->] (12,-1) -- (12,-3) node[midway, left] {\tiny $\exrandom$};
\draw[<-] (17,-1) -- (17,-3) node[midway, left] {\tiny $\thirdexrandom$};

\draw (21,1) rectangle (25,-1);
\node[anchor=center] (text) at (23,0) {\small $\probof{\thirdexrandom}$};
\draw (23,-1) -- (23,-3) node[midway, left] {\tiny $\thirdexrandom$};

\draw (23,-3) -- (23,-5);
\draw[fill] (23,-5) circle (0.25cm);
\draw[->] (23,-5) -- (23,-7) node[midway, left] {\tiny $\thirdexrandom$};
\draw (17,-3) to[bend right=40] (23,-5);
\draw (29,-3) to[bend right=-40] (23,-5);


\draw (28,1) rectangle (35,-1);
\node[anchor=center] (text) at (31.5,0) {\small $\condprobof{\secexrandom}{\thirdexrandom}$};
\draw[<-] (29,-1) -- (29,-3) node[midway, left] {\tiny $\thirdexrandom$};
\draw[->] (34,-1) -- (34,-3) node[midway, left] {\tiny $\secexrandom$};



\end{tikzpicture} 
\end{center}
\caption{Diagrammatic visualization of the contraction equation in Corollary~\ref{cor:secCriterionCondIndepencence}. Conditional independence of $\exrandom$ and $\secexrandom$ given $\thirdexrandom$ holds if the contraction on the right ride is equal to the probability tensor on the left side.}
\end{figure}



% More of an example?
\begin{theorem}[Markov Chain]\label{the:MarkovChain}
	Let there be a set of variables $\catvariableof{\tenumerator}$ where $\tenumeratorin$.
	When $\catvariableof{\tenumerator}$ is independent of $\catvariableof{0:{\tenumerator-2}}$ conditioned on $\catvariableof{\tenumerator-1}$ (the Markov Property), then
	\begin{align*}
		\probtensor = \contractionof{\{ \condprobof{\catvariableof{\tenumerator}}{\catvariableof{\tenumerator-1}}\, : \, \tenumeratorin \}}{
		\catvariableof{0},\ldots,\catvariableof{\tdim-1}
		} 
	\end{align*}	
%		\[ \probat{\catvariableof{0},\ldots,\catvariableof{\tdim-1}} = %\probat{\catvariableof{1}} 
%		\prod_{\tenumeratorin} \condprobof{\catvariableof{\tenumerator}}{\catvariableof{\tenumerator-1}} \, . \] 
	We depict this decomposition in Figure~\ref{fig:MC}.
\end{theorem}
\begin{proof}
	By the chain rule (Theorem~\ref{the:chainRule}) we have
	\begin{align*}
	 	\probat{\catvariableof{0},\ldots,\catvariableof{\tdim-1}}
		= \contractionof{
		\{ \condprobof{\catvariableof{\tenumerator}}{\catvariableof{0:\tenumerator}} : \tenumeratorin \}
		}{\catvariableof{[\tdim]}}
		%= \contractionof{\{\probat{\catvariableof{0}} \prod_{\tenumeratorin, \tenumerator>1} \condprobof{\catvariableof{\tenumerator}}{\catvariableof{0:\tenumerator}}\}{\catvariableof{[\tdim]}} \, . 
	\end{align*}
	Using the conditional independence of $\catvariableof{\tenumerator}$ and $\catvariableof{0:{\tenumerator-2}}$ conditioned on $\catvariableof{\tenumerator-1}$ we further have by Corollary~\ref{cor:conditionDropping}
		\[ \condprobof{\catvariableof{\tenumerator}}{\indexedcatvariableof{0:\tenumerator}}  = \condprobof{\catvariableof{\tenumerator}}{\indexedcatvariableof{\tenumerator-1}} \, .  \]
	Composing both equalities shows the claim.
\end{proof}

Here we denoted by $\catvariableof{0:\tenumerator}$ the tuple $\catvariableof{0},...,\catvariableof{\tenumerator}$.

\begin{remark}
	Let us notice that the dimensionality dropped drastically through applying the independence assumption.
	The tensor space in the naive representation of any probability distribution has
		\[ \prod_{\tenumeratorin} \catdimof{\tenumerator}\]
	coordinates, while the Markov Chain is represented by
		\[ \sum_{\tenumeratorin}  \catdimof{\tenumerator}\cdot \catdimof{\tenumerator-1} \, . \]
	Replacing exponential scaling with the number of variables to linear scaling is the advantage of tensor network decompositions.
\end{remark}

\begin{figure}[h]
\begin{center}
	\begin{tikzpicture}[scale=0.3,thick] % , baseline = -3.5pt

\node[anchor=center] (text) at (-1,3) {${a)}$};

	\node [circle, draw, thick, fill=gray!50] (T1) at (0,0) {\tiny $\randomxof{0}$};
	\node [circle, draw, thick, fill=gray!50] (T2) at (5,0) {\tiny $\randomxof{1}$};
	\draw[->] (T1) -- (T2);
	\node [circle, draw, thick, fill=gray!50] (T3) at (10,0) {\tiny $\randomxof{2}$};
	\draw[->] (T2) -- (T3);
	\node [circle, draw, thick, fill=gray!50] (T4) at (15,0) {\tiny $\randomxof{3}$};
	\draw[->] (T3) -- (T4);
	\draw[->] (T4) -- (18,0);

	\node[anchor=center] (text) at (19,0) {$\cdots$};

	%\node [circle, draw, thick, fill=gray!50] (T4) at (17,0) {\tiny $\randomxof{\atomorder}$};
	%\draw[->] (14,0) -- (T4);	
			

\begin{scope}[shift={(25,0)}]

\node[anchor=center] (text) at (-3,3) {${b)}$};

\draw (-3.5,-1) rectangle (0, 1);
\node[anchor=center] (text) at (-1.75,0) {\small $\probof{\randomxof{0}}$};
\draw[->] (0,0) -- (2,0);
\draw[fill] (1,0) circle (0.25cm);
\draw[->] (1,0) -- (1,2) node[above] {\tiny $\catlegof{0}$};
\draw (2,-1) rectangle (7, 1);
\node[anchor=center] (text) at (4.5,0) {\small $\condprobof{\randomxof{1}}{\randomxof{0}}$};
\draw[->]  (7,0) -- (9,0);
\draw[fill] (8,0) circle (0.25cm);
\draw[->] (8,0) -- (8,2) node[above] {\tiny $\catlegof{1}$};
\draw (9,-1) rectangle (14, 1);
\node[anchor=center] (text) at (11.5,0) {\small $\condprobof{\randomxof{2}}{\randomxof{1}}$};
\draw[->]  (14,0) -- (16,0);
\draw[fill] (15,0) circle (0.25cm);
\draw[->] (15,0) -- (15,2) node[above] {\tiny $\catlegof{2}$};
\draw (16,-1) rectangle (21, 1);
\node[anchor=center] (text) at (18.5,0) {\small $\condprobof{\randomxof{3}}{\randomxof{2}}$};
\draw[->]  (21,0) -- (23,0);
\draw[fill] (22,0) circle (0.25cm);
\draw[->] (22,0) -- (22,2) node[above] {\tiny $\catlegof{3}$};
\node[anchor=center] (text) at (24,0) {$\cdots$};


\end{scope}

\end{tikzpicture} 
\end{center}
\caption{Depiction of a Markov Chain. 
	a) Dependency Graph (of the corresponding chain Graphical Model).
	b) Dual Tensor Network representing the conditional probability factors.}
\label{fig:MC}
\end{figure}





\subsection{Graphical Models}



We have already depicted conditional dependency assumptions made for Markov Chains in Figure~\ref{fig:MC} and discussed the implied decomposition of the dual tensor networks.
Graphical models provide a more general framework for conditional dependency assumptions and provide a generic approach to exploit independences in finding tensor network decompositions of $\probtensor$.


%Graphical Models are typically depicted by nodes to each variable and edges.
Following the tensor network formalism we in this section introduce graphical models based on hypergraphs.
Whether the hypergraph is directed or not distinguished between Bayesian Networks and Markov Networks.




%\begin{remark}[Further nomenclature]
%	The factors of the graphical models are tensors (since multivariate functions of discrete variables).
%	The edges are associated to each axis of the tensor and carry the variables.
%	Since each edge variable can appear in multiple factors, the Tensor Network is defined on a Hypergraph, where edges are interpreted as Hadamard contractions.
%\end{remark}



\subsubsection{Markov Networks}

While typically Markov Networks are defined on graphs, we define them here on hypergraphs to establish a direct connection to tensor networks defined on the same hypergraph.
Along that line, Markov Networks are tensor networks with non-negative tensors (see \defref{def:tensorNetwork}), which are interpreted as probability distributions after normation.

\begin{definition}[Markov Network]\label{def:markovNetwork}
	Let $\tnetof{\graph}$ be a tensor network of non-negative tensors on a hypergraph $\graph$.
	Then the Markov Network to $\tnetof{\graph}$ is the probability distribution of $\catvariableof{\node}$ defined by the tensor
		\[ \probofat{\graph}{\nodevariables} = \frac{
			\contractionof{\{\hypercoreof{\edge} : \edge \in \edges\}}{\nodevariables} 
		}{
			\contraction{\{\hypercoreof{\edge} : \edge \in \edges\}}
		} = \normationof{\tnetof{\graph}}{\nodevariables} \, . \] 
	We call the denominator
		\[\partitionfunctionof{\tnetof{\graph}} = \contraction{\{\hypercoreof{\edge} : \edge \in \edges\}} \]
	the partition function of the Markov Network.
\end{definition}

% Marginalization and Conditioning
Often, we are only interested in the distribution of a subset of variables, which are called the observable variables, and call the other variables hidden variables.
The marginalization of a Markov Network to $\tnetof{\graph}$ on the variables $\catvariableof{\secnodes}$ is
	\[
		\probofat{\graph}{\catvariableof{\secnodes}}
		= \normationof{\tnetof{\graph}}{\catvariableof{\secnodes}} \, . 
	\]
This can be derived from Theorem~\ref{the:splittingContractions}, which established an equivalence of contractions with sequences of consecutive contractions.


Further, the distribution of $\catvariableof{\secnodes}$ conditioned on $\catvariableof{\thirdnodes}$, where $\secnodes,\thirdnodes$ are disjoint subsets of $\nodes$, is
	\[
		\probtensor^{\graph}\left[ \catvariableof{\secnodes} | \catvariableof{\thirdnodes}\right] 
		= \normationofwrt{\tnetof{\graph}}{\catvariableof{\secnodes}}{\catvariableof{\thirdnodes}} \, . 
	\]

\begin{definition}[Separation of Hypergraph]
	A path in a hypergraph is a sequence of nodes $\node_{\atomenumerator}$ for $\atomenumeratorin$, such that for any $\atomenumerator\in[\atomorder-1]$ we find a hyperedge $\edge\in\edges$ such that $(\node_{\atomenumerator}, \node_{\atomenumerator+1})\subset \edge$.
	Given disjoint subsets $\nodesa$, $\nodesb$, $\nodesc$ of nodes in a hypergraph $\graph$ we say that $\nodesc$ separates $\nodesa$ and $\nodesb$ with respect to $\graph$, when any path starting at a node in $\nodesa$ and ending in a node in $\nodesb$ contains a node in $\nodesc$.
	%when removing the hyperedges which are contained in $\nodesc$ leads to a hypergraph with no path of hyperedges between a node in $\nodesa$ to a node in $\nodesb$.
\end{definition}

To characterize Markov Networks in terms of conditional independencies we need to further define the property of clique-capturing.
This property of clique-capturing established a correspondence of hyperedges with maximal cliques in an alternative graph-based definition of Markov Networks \cite{koller_probabilistic_2009}.

\begin{definition}[Clique-Capturing Hypergraph]\label{def:ccHypergraph}
	We call a hypergraph $\graph$ clique-capturing, when each subset $\secnodes\subset\nodes$ is contained in a hyperedge, if for any $a,b\in\secnodes$ there is a hyperedge $\edge\in\edges$ with $a,b\in\secnodes$.
\end{definition}

Let us now show a characterization of Markov Networks in terms of conditional independencies, which is analogous to Theorem~\ref{the:condIndBN}.

% Characterization
\begin{theorem}\label{the:condIndMN}
	Given a clique-capturing hypergraph $\graph$, the set of positive Markov Networks on the hypergraph coincides with the set of positive probability distributions, such that each for each disjoint subsets of variables $\nodesa$, $\nodesb$, $\nodesc$ we have $\catvariableof{\nodesa}$ is independent of $\catvariableof{\nodesb}$ conditioned on $\catvariableof{\nodesc}$, when $\nodesc$ separates $\nodesa$ and $\nodesb$ in the hypergraph. % called d-separation
\end{theorem}
\begin{proof}
	%=>
	%Given any Markov Network, contracting with $\onehotmapof{\atomlegindexof{\nodesc}}$ turns all hyperedges contained in $\nodesc$ to scalar factors (copying possible).
	Let there be a hypergraph $\graph$, a Markov Network $\extnet$ on $\graph$ and nodes $\nodesa,\nodesb,\nodesc \subset \nodes$, such that $\nodesc$ separates $\nodesa$ from $\nodesb$.
	Let us denote by $\nodes_0$ the nodes with paths to $\nodesa$, which do not contain a node in $\nodesc$, and by $\nodes_1$ the nodes with paths to $\nodesb$, which do not contain a node in $\nodesc$.
	Further, we denote by $\edges_0$ the hyperedges which contain a node in $\nodes_0$ and by $\edges_1$ the hyperedges which contain a node in $\nodes_1$.
	By assumption of separability, both sets $\edges_0$ and $\edges_1$ are disjoint and no node in $\nodesa$ is in a hyperedge in $\edges_1$, respectively no node in $\nodesb$ is in a hyperedge in $\edges_0$, .
	We then have
	\begin{align*}
		\normationofwrt{\extnetasset}{\catvariableof{\nodesa},\catvariableof{\nodesb}}{\indexedcatvariableof{\nodesc}} 
		= & \normationof{\extnetasset\cup\{\onehotmapof{\catindexof{\nodesc}}\}}{\catvariableof{\nodesa},\catvariableof{\nodesb}} \\
		= &  \normationof{\{\hypercoreof{\edge}\, : \, \edge\in\edges_0\}\cup\{\onehotmapof{\catindexof{\nodesc}}\}}{\catvariableof{\nodesa}}
		\otimes \normationof{\{\hypercoreof{\edge}\, : \, \edge\in\edges_1\}\cup\{\onehotmapof{\catindexof{\nodesc}}\}}{\catvariableof{\nodesb}} \, .
	\end{align*}
	By Theorem~\ref{the:condIndependenceProductCriterion}, it now follows that $\catvariableof{\nodesa}$ is independent of $\catvariableof{\nodesb}$ conditioned on $\catvariableof{\nodesc}$.
	%<= HARDER! Hammersley Clifford needed
	The converse direction, i.e. that positive distributions respecting the conditional indpendence assumptions are representable as Markov Networks, is known as the Hammersley Clifford Theorem, 
	which we will proof later in Section~\ref{sec:proofHCTheorem}.
	%for which proof we refer to Theorem~4.8 in KOLLER.
\end{proof}

% Positivity
From the proof of Theorem~\ref{the:condIndMN} Markov Networks with zero coordinates still satisfy the conditional independence assumption.
However, the reverse is not true, that is there are distributions with vanishing coordinates, which satisfy the conditional independence assumptions, but cannot be represented as a Markov Network (see Example~4.4 in \cite{koller_probabilistic_2009}).




\subsubsection{Bayesian Networks}

Compared to Markov Networks, Bayesian Networks impose further conditions on tensor networks representing a distribution.
They assume a directed hypergraph and each tensor decorating the edges to be normed according to the direction.
We will observe, that if the hypergraph is in addition acyclic, then each tensor core coincides with the conditional distribution of the underlying Markov Network.
To introduce Bayesian Networks, we extend \defref{def:hypergraphs} by introducing the property of acyclicity for hypergraphs.

%are described by directed acyclic graphs (DAG).
%The probability distribution is a Hadamard product of conditional probabilities, where each variable has a conditional probability factor conditioned on the parents variables in the graph.
%We introduce Bayesian Networks based on directed hypergraphs (see \defref{def:hypergraphs}) and define further properties.

\begin{definition}
	A directed path is a sequence $\node_{0},\ldots\node_{\secatomorder}$ such that for any $\secatomenumeratorin$ there is an hyperedge $\edge=(\incomingnodes,\outgoingnodes)\in\edges$ such that $\node_{\secatomenumerator}\in\incomingnodes$ and $\node_{\secatomenumerator+1}\in\outgoingnodes$.
	We call the hypergraph $\graph$ acyclic, if there is no path with $\secatomorder>0$ such that $\node_{0}=\node_{\secatomorder}$.
	Given a directed hypergraph $\graph=(\nodes,\edges)$ we define for any node $\nodein$ its parents by
		\[ \parentsof{\node} = \{\secnode \, : \, \exists\edge=(\incomingnodes,\outgoingnodes)\in\edges: \secnode\in\incomingnodes,\node\in\outgoingnodes \} \]
	and its non-descendants $\nondescendantsof{\node}$ as the set of nodes $\secnode$, such that there is no directed path from $\node$ to $\secnode$.
\end{definition}

Based on these additional graphical properties, we now define Bayesian Networks.

\begin{definition}[Bayesian Network]\label{def:bayesianNetwork}
	Let $\graph=(\nodes,\edges)$ be a directed acyclic hypergraph with edges of the form 
		\[ \edges = \bnedges \, . \]
	A \emph{Bayesian Network} is a decoration of each edge $(\parentsof{\node},\{\node\})$ by a conditional probability distribution
		\[ \condprobof{\catvariableof{\node}}{\catvariableof{\parentsof{\node}}} \]
	which represents the probability distribution
	\begin{align*}
		\probat{\nodevariables} = \contractionof{\{\condprobof{\catvariableof{\node}}{\catvariableof{\parentsof{\node}}} \, : \, \nodein\}}{\nodevariables} \, .
	\end{align*}
\end{definition}

%
By definition each tensor decorating a hyperedge is directed with $\catvariableof{\parentsof{\node}}$ incoming and $\catvariableof{\node}$ outgoing.
Thus, the directionality of the hypergraph is reflected in each tensor decorating a directed hyperedge.
This allows us to verify with Theorem~\ref{the:conditionalContractionPreservation} that their contraction defines a probability distribution.

% Contraction -> Now in definition!
%By definition we can represent a Bayesian network by the contraction
%\begin{align*}
%	\probtensorof{\graph} = \sbcontractionof{\{ \condprobof{\catvariableof{\node}}{\catvariableof{\parentsof{\node}}} \, : \, \node\in\nodes\}}{\nodes} \, . 
%\end{align*}

% Dual
%The dual tensor network consists of conditional probability distributions to each node $\node\in\nodes$ (see Figure~\ref{fig:BayesianFactor}b).

\begin{figure}[h]
\begin{center}
	\begin{tikzpicture}[scale=0.35,thick] % , baseline = -3.5pt

\node[anchor=center] (text) at (-1,3) {${a)}$};

	\node [circle, draw, thick, fill=gray!50] (H) at (5,0) {\tiny $\randomxof{\node}$};
	\node [circle, draw, thick, fill=gray!50] (P1) at (0,-5) {\tiny $\randomxof{0}$};	
	\node [circle, draw, thick, fill=gray!50] (P2) at (5,-5) {\tiny $\randomxof{1}$};	
	
	\node[anchor=center] (text) at (10,-5) {$\cdots$};
	\node [circle, draw, thick, fill=gray!50] (Pd) at (15,-5) {\tiny $\randomxof{\atomorder\shortminus1}$};
	
	\node [] (E) at (5,-2) {};	
	
	\draw[midarrow] (P1) -- (5,-2) ;	
	\draw[midarrow] (P2) -- (5,-2) ;	
	\draw[midarrow] (Pd) -- (5,-2) ;	
	\draw[midarrow] (5,-2) -- (H) ;	
			

\begin{scope}[shift={(25,0)}]

\node[anchor=center] (text) at (-3,3) {${b)}$};

\draw[->] (4.5,-1) -- (4.5,1) node[midway, right]{\tiny $\catvariableof{\node}$};
\draw (0,-1) rectangle (9,-4); 
\node[anchor=center] (text) at (4.5,-2.5) {\small $\condprobof{\randomxof{\node}}{\randomxof{[\atomorder]}} $};
\draw[->] (1,-6) -- (1,-4) node[midway, right]{\tiny $\catlegof{0}$};
\draw[->] (2.5,-6) -- (2.5,-4) node[midway, right]{\tiny $\catlegof{1}$};

\node[anchor=center] (text) at (5.5,-5) {$\cdots$};
	
\draw[->] (8,-6) -- (8,-4) node[midway, right]{\tiny $\catlegof{\atomorder\shortminus1}$};

\end{scope}

\end{tikzpicture} 
\end{center}
\caption{Example of a Factor of a Bayesian Network to the node $\catvariableof{\node}$ with parents $\catvariableof{0},\ldots,\catvariableof{\catorder-1}$, as subgraph $a)$ and dual tensor core $b)$.}
\label{fig:BayesianFactor}
\end{figure}


%% Marginalization and Contraction
Marginalization of a Bayesian Network are still Bayesian Networks on a graph where the edges directing to variables, which are not marginalized over, are replaced by directed edges to the children.
Conditioned Bayesian Network do not have a simple Bayesian Network representation, which is why we will treat them as Markov Networks to be introduced next.


\begin{theorem}[Independence Characterization of Bayesian Networks]\label{the:condIndBN}
	A probability distribution $\probat{\nodevariables}$ has a representation by a Bayesian Network on a directed acyclic graph $\graph=(\nodes,\edges)$, if and only if for any $\nodein$ the variables $\catvariableof{\node}$ are independent on $\nondescendantsof{\node}$ conditioned on $\parentsof{\node}$.
\end{theorem}
\begin{proof}
	We choose a topological order $\prec$ on the nodes of $\graph$, which exists since $\graph$ is acyclic.
	
	\proofrightsymbol:
	Let us assume, that the conditional independencies are satisfied and apply the chain rule with respect to that ordering to get
	\begin{align*}
		\probat{\nodevariables} =
		\contractionof{
			\condprobof{\catvariableof{\node}}{\catvariableof{\secnode} : \secnode \prec \node}
		}
		{\nodevariables} \, .
	\end{align*}
	Since $\prec$ is a topological ordering we have
		\[ \parentsof{\node} \subset \{\secnode : \secnode \prec \node\} \]
	We apply the assumed conditional independence with Corollary~\ref{cor:conditionDropping} and get
	\begin{align*}
		\probat{\nodevariables} =
		\contractionof{
			\condprobof{\catvariableof{\node}}{\catvariableof{\parentsof{\node}}}
		}
		{\nodevariables} \, .
	\end{align*}
	
	\proofleftsymbol:
	To show the converse direction, let there be a Bayesian Network $\probat{\nodevariables}$ on $\graph$.
	To show for any node $\node$, that $\catvariableof{\node}$ is independent of $\nondescendantsof{\node}$ conditioned on $\parentsof{\node}$, we reorder the tensors in the contraction
	%with respect to a set $\node_0$ 
	\begin{align*}
		& \condprobof{\catvariableof{\node},\catvariableof{\nondescendantsof{\node}}}{\indexedcatvariableof{\parentsof{\node}}} \\
		& \quad\quad = \normationofwrt{
			\{\condprobof{\catvariableof{\secnode}}{\catvariableof{\parentsof{\secnode}}} \, : \, \secnode\in\nodes\}
		}
		{\catvariableof{\node},\catvariableof{\nondescendantsof{\node}}}
		{\indexedcatvariableof{\parentsof{\node}}} \\
		& \quad\quad  = \normationof{
			\{\condprobof{\catvariableof{\secnode}}{\catvariableof{\parentsof{\secnode}}} \, : \, \secnode\in\nodes\} \cup \{\onehotmapof{\catindexof{\parentsof{\node}}}\}
		}
		{\catvariableof{\node},\catvariableof{\nondescendantsof{\node}}}\\
		&  \quad\quad = \normationof{
			\{\condprobof{\catvariableof{\secnode}}{\catvariableof{\parentsof{\secnode}}} \, : \, \secnode\in\nondescendantsof{\node}\} \cup \{\onehotmapof{\catindexof{\parentsof{\node}}}, \condprobof{\catvariableof{\node}}{\catvariableof{\parentsof{\node}}} \}
		}
		{\catvariableof{\node},\catvariableof{\nondescendantsof{\node}}} \\
		&  \quad\quad =  %\contractionof{
		 \normationof{
			\{\condprobof{\catvariableof{\secnode}}{\catvariableof{\parentsof{\secnode}}} \, : \, \secnode\in\nondescendantsof{\node}\} \cup \{\onehotmapof{\catindexof{\parentsof{\node}}}\}
		}
		{\catvariableof{\nondescendantsof{\node}}} \\
		& \quad\quad  \quad  \cdot \normationof{
			\{\condprobof{\catvariableof{\node}}{\catvariableof{\parentsof{\node}}},\onehotmapof{\catindexof{\parentsof{\node}}}\}
		}
		{\catvariableof{\node}} \\
		& \quad\quad  = \contractionof{\{
		\condprobof{\catvariableof{\nondescendantsof{\node}}}{\indexedcatvariableof{\parentsof{\node}}},
		\condprobof{\catvariableof{\node}}{\indexedcatvariableof{\parentsof{\node}}}
		\}}{\catvariableof{\node},\catvariableof{\nondescendantsof{\node}}}
		%}{\catvariableof{\node},\catvariableof{\nondescendantsof{\node}}}
	\end{align*}
	Here we have dropped in the third equation all tensors to the descendants, since their marginalization is trivial (which can be shown by a leaf-stripping argument).
	In the fourth equation we made use of the fact, that any directed path between the non-descendants and the node is through the parents of the node.
	By Theorem~\ref{the:condIndependenceProductCriterion}, it now follows that $\catvariableof{\node}$ is independent of $\nondescendantsof{\node}$ conditioned on $\parentsof{\node}$.
\end{proof}

\subsubsection{Bayesian Networks as Markov Networks}

Markov Networks are more flexible compared with Bayesian Networks, since any Bayesian Network is a Markov Network by ignoring the directionality of the hypergraph and understanding the conditional distributions as generic tensor cores.
In the next theorem we provide the conditions for the interpretation of a Markov Network as a Bayesian Network.

\begin{theorem}\label{the:MarkovToBayesian}
	Let $\tnetof{\graph}$ be a tensor network on a directed acyclic hypergraph, such that the edges are of the structure
		\[ \edges = \bnedges \]
	and each tensor $\hypercoreof{\edge}$ respects the directionality of the graph, that is each $\hypercoreof{(\parentsof{\node}, \{\node\})}$ is directed with the variables to $\parentsof{\node}$ incoming and $\node$ outgoing.
	Then $\partitionfunctionof{\tnetof{\graph}}=1$ and for each $\node\in\nodes$ we have
		\[ \bnnodecore = \normationofwrt{\tnetof{\graph}}{\catvariableof{\node}}{\catvariableof{\parentsof{\node}}} \, . \]
	In particular, $\tnetof{\graph}$ is a Bayesian Network.
\end{theorem}
\begin{proof}
	We show the claim by induction over the cardinality of $\nodes$.
	
	$\cardof{\nodes}=1$: In this case we find a unique node $\node\in\nodes$ and have $\edges=\{(\varnothing,\{\node\})\}$.
		The tensor $\hypercoreof{(\varnothing,\{\node\})}$ is then normed with no incoming variables and we thus have
			\[ \partitionfunctionof{\tnetof{\graph}} = \contraction{\tnetof{\graph}} = \contraction{\hypercoreof{(\varnothing,\{\node\})}} = 1 \]
		and
			\[ \normationof{\tnetof{\graph}}{\catvariableof{\node}} = \hypercoreof{(\varnothing,\{\node\})} \, .  \]
			
	$\cardof{\nodes}-1 \rightarrow \cardof{\nodes}$: Let there now be a directed hypergraph $\graph=(\nodes,\edges)$ and let us now assume, that the theorem holds for any tensor networks with node cardinality $\cardof{\nodes}-1$.
		Since the hypergraph is acyclic, we find a root $\node\in\nodes$ such that $\node\notin\parentsof{\secnode}$ for $\secnode\in\nodes$.
		We denote $\tnetof{\secgraph}$ the tensor network on the hypergraph $\secgraph=\{\nodes/\{\node\},\edges/\{(\parentsof{\node},\{\node\})\}\}$ with decorations inherited from $\tnetof{\graph}$.
		With Theorem~\ref{the:splittingContractions}, the directionality of $\bnnodecore$ and the induction assumption on $\tnetof{\secgraph}$ we have
		\begin{align*}
			\contraction{\tnetof{\secgraph}\cup\left\{\bnnodecore\right\}}
			 = \contraction{\tnetof{\secgraph}\cup\left\{\contractionof{\bnnodecore}{\catvariableof{\parentsof{\node}}}\right\}}
			 = \contraction{\tnetof{\secgraph}\cup\left\{\onesat{\catvariableof{\parentsof{\node}}}\right\}}
			 = 1
		\end{align*}
		and thus a trivial partition function.
		Since $\node$ does not appear in $\secgraph$, we have for any index $\catindexof{\parentsof{\nodes}}$
		\begin{align*}
			\contractionof{\tnetof{\graph}}{\catvariableof{\node},\indexedcatvariableof{\parentsof{\node}}}
			= \contractionof{\bnnodecore}{\catvariableof{\node},\indexedcatvariableof{\parentsof{\node}}}
			\cdot \contractionof{\tnetof{\secgraph}}{\indexedcatvariableof{\parentsof{\node}}}
		\end{align*}
		and thus, since $\bnnodecore$ is directed, that
		\begin{align*}
			\normationofwrt{\tnetof{\graph}}{\catvariableof{\node}}{\catvariableof{\parentsof{\node}}}
			= \bnnodecore \, .
		\end{align*}
\end{proof}

%\begin{theorem}\label{the:BayesianToMarkov}
%	Any Bayesian Network on a directed graph $\graph=(\nodes,\edges)$ is a Markov Network on a hypergraph $\secgraph=(\nodes,\secedges)$ with identical nodes and hyperedges consistent of  a hyeredge to each node with $\node$ being the only outgoing node and
%		\[  \{\tilde{\node} \, : \, (\tilde{\node},\node) \in \edges\} \,  \]
%	being the incoming nodes.
%	Each hyperedge of the Markov Network is decorated with the conditional probability distribution and the partition function is vanishing.
%\end{theorem}
%\begin{proof}
%	Each conditional probability distribution is associated with the hyperedge constructed to the representative node.
%	The contraction of all conditional probability distributions is the Bayesian Network, which corresponds with the constructed Markov Network due to the trivial partition function.
%\end{proof}

%% Bayesian Network richer
Theorem~\ref{the:MarkovToBayesian} states that Bayesian Networks are a subset of Markov Networks.
While Markov Network allow generic tensor cores, Bayesian Networks impose a local directionality condition on each tensor core by demanding it to be a conditional probability tensor.
In our diagrammatic notation, the local normation of Bayesian Networks is highlighted by the directionality of the hypergraph.
Generic Markov Networks are on undirected hypergraphs, where in general no local directionality condition is assumed.
As a consequence, tasks such as the determination of the partition functions or calculation of conditional distributions involve global contractions.


%% Conditioning
%The representation of Bayesian Networks by Markov Networks is of special interest when representing conditional distributions.
%Bayesian Networks conditioned on evidence are no longer Bayesian Networks on the same graph, but Markov Networks on a hypergraph enriched by the evidence conditioned about.


\subsubsection{Example: Hidden Markov Models}

We here extend the example of Markov Chains from Theorem \ref{the:MarkovChain} to a limited observation of the variables by observations.
Let there be the variables $\catvariableof{\tenumerator}$ (states) and $\randomeof{\tenumerator}$ (observations) with a discrete and finite time $\tenumeratorin$.

The conditional assumptions are 
\begin{itemize}
	\item $\catvariableof{\tenumerator+1}$ is independent of $\catvariableof{0:\tenumerator-1}$ and $\randomeof{0:\tenumerator}$ conditioned on $\catvariableof{\tenumerator}$
	\item $\randomeof{\tenumerator}$ is independent of all other variables conditioned on $\catvariableof{\tenumerator}$
\end{itemize}

Then the probability tensor has the decomposition 
\begin{align}
	\probat{\catvariableof{0:\tdim},\randomeof{0:\tdim}} 
	& = \prod_{\tenumeratorin}
	 \left( \condprobof{\catvariableof{\tenumerator}}{\catvariableof{0:\tenumerator-1},\randomeof{0:\tenumerator-1}} \cdot \condprobof{\randomeof{\tenumerator}}{\catvariableof{0:\tenumerator},\randomeof{0:\tenumerator-1}} \right) \\
	& = \probat{\catvariableof{0}} \cdot \condprobof{\randomeof{0}}{\catvariableof{0}} \cdot \prod_{\tenumeratorin, \tenumerator>0} 
	\left( \condprobof{\catvariableof{\tenumerator}}{\catvariableof{\tenumerator-1}} \cdot \condprobof{\randomeof{\tenumerator}}{\catvariableof{\tenumerator}} \right)
\end{align}
Here we used the Chain Rule decomposition of Theorem~\ref{the:chainRule} in the first equation and the conditional independence assumptions in the second.

We notice, that this is a Bayesian Netowork on a directed acyclic hypergraph $\graph$ consistent in nodes to each state and each observation and directed hyperedges
\begin{itemize}
	\item $(\{\catvariableof{\tenumerator}\}, \{\catvariableof{\tenumerator+1}\})$ for $\tenumerator\in[\tdim-1]$
	\item $(\{\catvariableof{\tenumerator}\}, \{\randomeof{\tenumerator}\})$ for $\tenumeratorin$
\end{itemize}


\begin{figure}[h]
\begin{center}
	\begin{tikzpicture}[scale=0.3,thick] % , baseline = -3.5pt

\node[anchor=center] (text) at (-1,3) {${a)}$};

	\node [circle, draw, thick, fill=gray!50] (T1) at (0,0) {\tiny $\randomxof{0}$};	
	\node [circle, draw, thick, fill=gray!50] (E1) at (0,-5) {\tiny $\randomeof{0}$};
	\draw[->] (T1) -- (E1);	
	\node [circle, draw, thick, fill=gray!50] (T2) at (5,0) {\tiny $\randomxof{1}$};
	\node [circle, draw, thick, fill=gray!50] (E2) at (5,-5) {\tiny $\randomeof{1}$};
	\draw[->] (T2) -- (E2);	
	\draw[->] (T1) -- (T2);	
	\node [circle, draw, thick, fill=gray!50] (T3) at (10,0) {\tiny $\randomxof{2}$};
	\node [circle, draw, thick, fill=gray!50] (E3) at (10,-5) {\tiny $\randomeof{2}$};
	\draw[->] (T3) -- (E3);	
	\draw[->] (T2) -- (T3);
	\node [circle, draw, thick, fill=gray!50] (T4) at (15,0) {\tiny $\randomxof{3}$};
	\node [circle, draw, thick, fill=gray!50] (E4) at (15,-5) {\tiny $\randomeof{3}$};
	\draw[->] (T4) -- (E4);	
	\draw[->] (T3) -- (T4);
	\draw[->] (T4) -- (18,0);

	\node[anchor=center] (text) at (19,0) {$\cdots$};

	%\node [circle, draw, thick, fill=gray!50] (T4) at (17,0) {\tiny $\randomxof{\atomorder}$};
	%\draw[->] (14,0) -- (T4);	
			

\begin{scope}[shift={(25,0)}]

\node[anchor=center] (text) at (-3,3) {${b)}$};

\draw (-3.5,-1) rectangle (0, 1);
\node[anchor=center] (text) at (-1.75,0) {\small $\probof{\randomxof{0}}$};
\draw[->] (0,0) -- (2,0);
\draw[fill] (1,0) circle (0.25cm);
\draw[->] (1,0) -- (1,2) node[above] {\tiny ${\randomxof{0}}$};
\draw[->] (1,0) -- (1,-2);
\draw (-1.5,-2) rectangle (3.5,-4); 
\node[anchor=center] (text) at (1,-3) {\small $\condprobof{\randomeof{0}}{\randomxof{0}}$};
\draw[->] (1,-4) -- (1,-6) node[midway, right]{\tiny ${\randomeof{0}}$};

\draw (2,-1) rectangle (7, 1);
\node[anchor=center] (text) at (4.5,0) {\small $\condprobof{\randomxof{1}}{\randomxof{0}}$};
\draw[->]  (7,0) -- (9,0);
\draw[fill] (8,0) circle (0.25cm);
\draw[->] (8,0) -- (8,2) node[above] {\tiny ${\randomxof{1}}$};
\draw[->] (8,0) -- (8,-2);
\draw (5.5,-2) rectangle (10.5,-4); 
\node[anchor=center] (text) at (8,-3) {\small $\condprobof{\randomeof{1}}{\randomxof{1}}$};
\draw[->] (8,-4) -- (8,-6) node[midway, right]{\tiny ${\randomeof{1}}$};


\draw (9,-1) rectangle (14, 1);
\node[anchor=center] (text) at (11.5,0) {\small $\condprobof{\randomxof{2}}{\randomxof{1}}$};
\draw[->]  (14,0) -- (16,0);
\draw[fill] (15,0) circle (0.25cm);
\draw[->] (15,0) -- (15,2) node[above] {\tiny ${\randomxof{2}}$};
\draw[->] (15,0) -- (15,-2);
\draw (12.5,-2) rectangle (17.5,-4); 
\node[anchor=center] (text) at (15,-3) {\small $\condprobof{\randomeof{2}}{\randomxof{2}}$};
\draw[->] (15,-4) -- (15,-6) node[midway, right]{\tiny ${\randomeof{2}}$};

\draw (16,-1) rectangle (21, 1);
\node[anchor=center] (text) at (18.5,0) {\small $\condprobof{\randomxof{3}}{\randomxof{2}}$};
\draw[->]  (21,0) -- (23,0);
\draw[fill] (22,0) circle (0.25cm);
\draw[->] (22,0) -- (22,2) node[above] {\tiny ${\randomxof{3}}$};
\draw[->] (22,0) -- (22,-2);
\draw (19.5,-2) rectangle (24.5,-4); 
\node[anchor=center] (text) at (22,-3) {\small $\condprobof{\randomeof{3}}{\randomxof{3}}$};
\draw[->] (22,-4) -- (22,-6) node[midway, right]{\tiny ${\randomeof{3}}$};


\node[anchor=center] (text) at (24,0) {$\cdots$};


\end{scope}

\end{tikzpicture} 
\end{center}
\caption{Depiction of a Hidden Markov Model. 
	a) Dependency Graph (of the corresponding chain Graphical Model).
	b) Dual Tensor Network representing the conditional probability factors.}
\label{fig:HMM}
\end{figure}




\subsection{Exponential Families}\label{sec:exponentialFamilies}

% Usage of the selection encoding -> Can also make a theorem out of this
Exponential families are collections of probability distributions, where each coordinate is determined by a base measure and a set $\sstat$ of features as
	\[ \probat{\indexedshortcatvariables}  \propto \basemeasure(\catindex) \cdot \expof{\sum_{\statenumeratorin} \sstatcoordinateofat{\selindex}{\indexedshortcatvariables} \cdot \canparamat{\indexedselvariable}} \, . \]
We use the selection encoding to represent the weighted summation over the statistics, that is the tensor
	\[ \sencsstatat{\shortcatvariables,\selvariable}: \facstates \times [\statorder] \rightarrow \rr \]
with
	\[ \sencsstatat{\indexedshortcatvariables,\indexedselvariable} = \sstatcoordinateofat{\selindex}{\indexedshortcatvariables} \, . \]
We then understand $\canparam$ as a vector to the categorical variable $\selvariable$ and use Theorem~\ref{the:linCompSelEncoding} to get
	\[ \sum_{\statenumeratorin}\canparamat{\indexedselvariable}\cdot \sstatcoordinateofat{\selindex}{\shortcatvariables}
		 = \sbcontractionof{\sencsstatat{\shortcatvariables,\selvariable},\canparamat{\selvariable}}{\shortcatvariables} \, . \]

\begin{definition}
	Given a sufficient statistics 
		\[ \sstat : \facstates \rightarrow \parameterspace\]
	and a boolean base measure
		\[ \basemeasure : \facstates \rightarrow [2] \]
	with $\contraction{\basemeasure}\neq0$,  the set $\expfamily=\{\expdist \, : \, \canparam[\selvariable] \in \simpleparspace\}$ of probability distributions 
		\[ \expdistat{\shortcatvariables} = \normationof{\expof{\sbcontractionof{\sencsstat,\canparam}{\shortcatvariables},\basemeasureat{\shortcatvariables}}}{\shortcatvariables} \]
	is called the exponential family to $\sstat$.
	We further define for each member with parameters $\canparam$ the associated energy tensor
		\[ \expenergy = \sbcontractionof{\sencsstat,\canparam}{\shortcatvariables} \]
	and the cumulant function
		\[ \cumfunctionof{\canparam} = \lnof{\sbcontraction{\basemeasure,\expof{\sbcontractionof{\sencsstat,\canparam}{\shortcatvariables} }} } \, .\]
\end{definition}



% Diverging partition functions avoided here
Since we restrict the discussion to finite state spaces, the distribution $\expdist$ is well-defined for any $\canparam\in\rr^{\statorder}$.
For infinite state space there are sufficient statistics and parameters, such that the partition function $\sbcontraction{\basemeasure,\expof{\sbcontractionof{\sencsstat,\canparam}{\shortcatvariables}}}$ diverges and the normation $\expdist$ is not well-defined.
In that cases, the canonical parameters need to be chosen from a subset where the partition function is finite. 

% Restriction to boolean base measures
As before, we restrict for boolean base measures, which satisfy $\contraction{\basemeasure}\neq0$
We notice, that by positivity of the exponential function, any distribution in an exponential family $\expfamily$ is positive with respect to $\basemeasure$ (see \defref{def:positivityBaseMeasure}).
%In Chapter~\ref{cha:logicalRepresentation} these tensors will be called satisfiable propositional formulas.
In Chapter~\ref{cha:networkRepresentation} we will investigate distributions, where the base measures and the sufficient statistics share a common computation framework.



% Cumulant representation
\begin{lemma}\label{lem:energyCumulantRepresentation}
	For any member of an exponential family $\expfamily$ we have
		\[ \expdistat{\shortcatvariables} 
		= \contractionof{\expof{ \expenergy - \cumfunctionof{\canparam}\cdot \onesat{\shortcatvariables}},\basemeasureof{\shortcatvariables}}{\shortcatvariables} \, . \]
\end{lemma}
\begin{proof}
	By definition we have
	\begin{align*}
		\expdistat{\shortcatvariables} 
		&= \normationof{
		\expof{\sbcontractionof{\sencsstat,\canparam}{\shortcatvariables}},\basemeasureat{\shortcatvariables}
		}{\shortcatvariables} \\
		&= \frac{\contractionof{\expof{\sbcontractionof{\sencsstat,\canparam}{\shortcatvariables},\basemeasureat{\shortcatvariables}}}{\shortcatvariables}
			}{\contraction{\expof{\sbcontractionof{\sencsstat,\canparam	}{\shortcatvariables}},\basemeasureat{\shortcatvariables}}} \\
		&=  \frac{
		\contractionof{\expof{\expenergyat{\shortcatvariables}},\basemeasureat{\shortcatvariables}}{\shortcatvariables}
		}{
		\expof{\cumfunctionof{\canparam}}
		} \\
		& = \contractionof{\expof{ \expenergy - \cumfunctionof{\canparam}\cdot \onesat{\shortcatvariables}},\basemeasureof{\shortcatvariables}}{\shortcatvariables} \, . 
	\end{align*}
\end{proof}


% Minimal statistics

\begin{definition}[Minimal]\label{def:minimalStatistics}
	We say that a statistic $\sstat$ is minimal with respect to a boolean base measure $\basemeasure$, if there is no pair of a nonvanishing vector $\vectorat{\selvariable}$ and a scalar $\lambda\in\rr$ with
		\[ \contractionof{\sencsstatat{\shortcatvariables,\selvariable},\vectorat{\selvariable},\basemeasureat{\shortcatvariables}}{\shortcatvariables} = \lambda\cdot\basemeasureat{\shortcatvariables} \, . \]
\end{definition}




\subsubsection{Tensor Network Representation} 

We can use the relational encoding formalism to represent members of exponential families by a single contraction, as we show next.
The central insight here is a relational encoding of the sufficient statistics, which enables representation by tensor network decomposition, when the sufficient statistic is decomposable.

\begin{theorem}[Generic Representation of Exponential Families]\label{def:expFamilyTensorRep}
	Given any base measure $\basemeasure$ and a sufficient statistic $\sstat$ we enumerate for each coordinate $\selindexin$ the image $\imageof{\sstatcoordinateof{\selindex}}$ by a variable $\sstatcatof{\selindex}$ taking values in $[\cardof{\imageof{\sstatcoordinateof{\statenumerator}}}]$, given an interpretation map
		\[ \indexinterpretationof{\selindex} : 
		[\cardof{\imageof{\sstatcoordinateof{\statenumerator}}}] \rightarrow \imageof{\sstatcoordinateof{\statenumerator}} \, . \]
	
	For any parameter vector $\canparamat{\selvariable}:[\seldim]\rightarrow\rr$ we build the activation cores
		\[ \actcoreofat{\statenumerator}{\sstatcatof{\selindex}=\sstatindof{\selindex}} 
		= \expof{\canparamat{\indexedselvariable} \cdot \indexinterpretationofat{\selindex}{\sstatcatof{\selindex}} } \,   \]
	and have
		\[ \expdist = 
		\normationof{\{\basemeasure,\rencodingof{\sstat}\}\cup\{\actcoreof{\statenumerator} \, : \, \statenumeratorin\}}{\shortcatvariables} \, . 
		\]
%	where we use the vectors $\actcoreof{\statenumerator} : \imageof{\sstatcoordinateof{\statenumerator}} \rightarrow \rr $ defined for $y \in \imageof{\sstatcoordinateof{\statenumerator}}$ by
\end{theorem}
\begin{proof}
	We use an extended image of $\sstat$ by  %	which does not modify the statement of Theorem~\ref{the:tensorFunctionComposition} (since extension to cases, which are never met).
		\[ \imageof{\sstat} = \bigtimes_{\statenumeratorin} \imageof{\sstatcoordinateof{\selindex}} \, . \]
	Theorem~\ref{the:tensorFunctionComposition} implies
		\[ \expof{\sbcontractionof{\sencsstat,\canparam}{\shortcatvariables}}
		= \sbcontractionof{\rencodingof{\sstat}, \restrictionofto{\expof{\braket{\cdot, \weight}}}{\imageof{\sstat}}}{\shortcatvariables} \, . \]
	The claim follows from the equation
		\[ \restrictionoftoat{\expof{\braket{\cdot, \canparam}}}{\imageof{\sstat}}{\catvariableof{\sstatcoordinateof{0}},\ldots,\catvariableof{\sstatcoordinateof{\seldim-1}}}
		= \bigotimes_{\selindexin} \restrictionoftoat{\expof{\cdot \canparamat{\indexedselvariable}}}{\imageof{\sstatcoordinateof{\selindex}}}{\catvariableof{\sstatcoordinateof{\selindex}}}
		= \bigotimes_{\selindexin} \actcoreofat{\selindex}{\catvariableof{\sstatcoordinateof{\selindex}}} \, . \]
\end{proof}


We notice, that the relational encoding is the contraction of the relational encoding of its coordinate maps as 
	\[ \rencodingofat{\sstat}{\shortcatvariables,\sstatcatof{[\seldim]}} = \contractionof{\rencodingof{\sstatcoordinateof{0}},\ldots,\rencodingof{\sstatcoordinateof{\seldim-1}}}{\shortcatvariables,\sstatcatof{[\seldim]}} \, .  \]
We will show this property in Theorem~\ref{the:functionDecompositionBasisCP}.
One strategy to create $\rencodingof{\sstat}$ is thus the creation of the encoding of all its coordinate maps.
When the coordinate maps are sharing common components, a sparser representation can be derived through encodings of the components shared among the coordinate map encodings.


% Core types
A tensor network representation of an exponential family is thus a Markov Network consistent of two types of cores.
Computation cores are relational encodings of statistics $\rencodingof{\sstatcoordinateof{\selindex}}$.
Our intuition is that they compute the hidden variable $\catvariableof{\sstatcoordinateof{\selindex}}$, based on Basis Calculus (see Chapter~\ref{cha:basisCalculus}).
Activation cores $\actcoreof{\selindex}$ exploit the computed variable and provide, when contracted with the relational encoding, a factor 
	\[ \sbcontractionof{\rencodingof{\sstatcoordinateof{\selindex}}, \actcoreofat{\statenumerator}{\catvariableof{\statenumerator}}}{\shortcatvariables}  \]
to the Markov Network reduced to the visible coordinates $\shortcatvariables$.
The activation cores are trivial, i.e. $\actcoreofat{\selindex}{\sstatcatof{\selindex}}=\onesat{\sstatcatof{\selindex}}$, when $\canparamat{\selvariable=\selindex}=0$.
In that case 
	\[  \sbcontractionof{\rencodingof{\sstatcoordinateof{\selindex}}, \actcoreofat{\statenumerator}{\catvariableof{\statenumerator}}}{\shortcatvariables} 
	= \onesat{\shortcatvariables} \]
and both the activation core and the corresponding computation core can be dropped from the network without changing its distribution.

% Interpretation as elementary 
By \theref{def:expFamilyTensorRep} any member of an exponential family is represented by the normed contraction of a collection of unary activation cores contracted with the computation network $\rencodingofat{\sstat}{\headvariableof{[\seldim]},\shortcatvariables}$.
We understand these activation cores as a member of a simple Markov Network distributing the head variables $\headvariableof{[\seldim]}$.
This Markov Network has a graph, where the edges contain single variables, that is $\elgraph=([\seldim],\{\{\selindex\} \, : \, \selindexin\})$.
We generalize these classed of probability distributions now to generic tensor network formats representing activation cores.
%Furthermore, we define 

% Define sets of realizable distributions
\begin{definition}
	Given a statistic $\sstat$, and a hypergraph $\graph=([\seldim],\edges)$ with nodes associated to the coordinates of the statistic, we define the set of realizable distributions by
	\begin{align*}
		\realizabledistsof{\sstat,\graph} = \left\{ \normationof{\{\rencodingofat{\sstat}{\headvariableof{[\seldim]},\shortcatvariables}\} \cup \{\hypercoreofat{\edge}{\headvariableof{\edge}}\}}{\shortcatvariables}  \, : \,\hypercoreofat{\edge}{\headvariableof{\edge}} \in\bigotimes_{\selindex\in\edge}\rr^{\headdimof{\selindex}} \right\} \, .
	\end{align*}
\end{definition}

For unary activation cores, that is for the graph $\elgraph$, we find for any member of $\realizabledistsof{\sstat,\elformat} $
\begin{center}
	\begin{tikzpicture}[scale=0.35,thick,xscale=1] % , baseline = -3.5pt

\draw (-1.25,1) rectangle (1.25,3);
\node[anchor=center] (text) at (0,2) {$\actcoreof{\sstatcoordinateof{0}}$};

\draw (2.75,1) rectangle (5.25,3);
\node[anchor=center] (text) at (4,2) {$\actcoreof{\sstatcoordinateof{\seccatorder\shortminus1}}$};

\draw[->] (0,-1)--(0,1) node[midway,left] {\tiny $\headvariableof{0}$}; 
%\draw[->] (1.5,-1)--(1.5,1) node[midway,left] {\tiny $\headvariableof{1}$}; 
\node[anchor=center] (text) at (2,0) {$\cdots$};
\draw[->] (4,-1)--(4,1) node[midway,right] {\tiny $\headvariableof{\seccatorder\shortminus1}$}; 

\draw (-1,-1) rectangle (5,-3);
\node[anchor=center] (text) at (2,-2) {\small $\rencodingof{\sstat}$};
\draw[<-] (0,-3)--(0,-5) node[midway,left] {\tiny $\catvariableof{0}$}; 
\draw[<-] (1.5,-3)--(1.5,-5) node[midway,left] {\tiny $\catvariableof{1}$}; 
\node[anchor=center] (text) at (3,-4) {$\cdots$};
\draw[<-] (4,-3)--(4,-5) node[midway,right] {\tiny $\catvariableof{\atomorder\shortminus1}$}; 


%\drawatomcore{3.5}{-8}{$\probtensor$}
%\drawatomindices{3.5}{-12}	
%\draw (5.5,-9)--(5.5,-7) node[midway,right] {\tiny $\catvariableof{\exformula}$};

\end{tikzpicture}
\end{center}


\begin{corollary}[Corollary of \theref{def:expFamilyTensorRep}]
	For any base measure $\basemeasure$ and statistic $\sstat$ we have
		\[ \expfamilyof{\sstat,\basemeasure} \subset \realizabledistsof{\sstat,\graph} \, . \]
\end{corollary}

% Proper subset
A natural question is to further ask, whether $\expfamilyof{\sstat,\basemeasure}$ is a proper subset of $\realizabledistsof{\sstat,\graph}$.
This is the case for most pairs $\sstat,\basemeasure$, since members of exponential families are positive with respect to their base measure, while in $\realizabledistsof{\sstat,\graph}$ we allow also for activation cores with vanishing coordinates, which in general do not produce positive distributions.
We will follow these intuitions in the discussion of logical reasoning, starting with \charef{cha:logicalReasoning}, and will use the formats $\realizabledistsof{\sstat,\graph}$ as hybrid formats storing probability distributions and logical knowledge bases.

%% FALSE STATEMENT? 
%We can sum multiples of the trivial tensor on the head cores without changing the distribution as we show next.
%
%\begin{theorem}
%	For any $\statenumeratorin$, the distribution is invariant under replacing $\actcoreofat{\statenumerator}{\selvariableof{\statenumerator}}$ by $\actcoreofat{\statenumerator}{\catvariableof{\statenumerator}}+\lambda\cdot \onesat{\catvariableof{\statenumerator}}$ where $\lambda\in\rr$
%\end{theorem}
%\begin{proof}
%	Follows from linearity in each head core, trivialization by trivial heads and normation.
%	
%	By linearity we have
%	\begin{align*}
%		\sbcontractionof{\rencodingof{\sstatcoordinateof{\selindex}}, (\actcoreofat{\statenumerator}{\catvariableof{\statenumerator}}+\lambda\cdot \onesat{\catvariableof{\statenumerator}})}{\shortcatvariables}
%		= 
%		\sbcontractionof{\rencodingof{\sstatcoordinateof{\selindex}}, \actcoreofat{\statenumerator}{\catvariableof{\statenumerator}}}{\shortcatvariables}
%		+\lambda\cdot  \sbcontractionof{\rencodingof{\sstatcoordinateof{\selindex}}, \onesat{\catvariableof{\statenumerator}}}{\shortcatvariables}
%		=  \sbcontractionof{\rencodingof{\sstatcoordinateof{\selindex}}, \actcoreofat{\statenumerator}{\catvariableof{\statenumerator}}}{\shortcatvariables}
%		+ \lambda \cdot \onesat{\shortcatvariables} \, .
%	\end{align*}
%\end{proof}


\begin{remark}[Comparison of relation and selection encodings]
	% Relation vs Selection encoding
	Relation encodings are in general of higher dimensions than selection encodings.
	In can thus be intractible to instantiate the probability distribution as a tensor networks, while the energy tensor can still be efficiently represented based on selection encodings.
	\red{In this case, energy-based reasoning algorithms are tractible while more direct methods are intractible.}
\end{remark}





\subsubsection{Mean Parameters}

Mean parameters are an alternative way to represent members of exponential families.

\begin{definition}\label{def:meanForwardBackward}
	Let there be an exponential family defined by a statistic $\sstat$ and a boolean base measure.
	We call the tensor
		\[ \meanparam = \sbcontractionof{\probtensor,\sencsstat}{\selvariable} \]
	the mean parameter tensor to a distribution $\probtensor$ of an exponential family.
	The set 
		\[ \genmeanset = \left\{\contractionof{\probtensor,\sencsstat,\basemeasure}{\selvariable} \, : \, 0\prec\probtensor , \, \contractionof{\probtensor,\basemeasure}{\shortcatvariables} 
		= \basemeasureat{\shortcatvariables} \right\} \, , \]
	where $\probtensorset$ denotes the set of all probability distributions,
	is called the convex polytope of realizable mean parameters.
	The map
		\[ \forwardmap :  \simpleparspace\rightarrow\simpleparspace\]
	with $\forwardmapof{\canparam} = \sbcontractionof{\expdist,\sencsstat}{\selvariable}$ is called the forward map of the exponential family and any map
		\[ \backwardmap : \imageof{\forwardmap} \rightarrow \simpleparspace\]
	with $\expdistof{(\sstat,\backwardmapof{\forwardmapof{\canparam}},\basemeasure)} = \expdist$ for any $\canparam\in\rr^{\statorder}$ a backward map.
\end{definition}


% Convex Hull Characterization Polytope
While introduced here as a property of a distribution, the mean parameters will be central to the discussion of probabilistic inference in Chapter~\ref{cha:probReasoning}.
We now provide a simple characterization of the sets of mean parameters based on slices of the selection encoding of the statistic.

\begin{theorem}\label{the:meanPolytopeConvHull}
	For any statistic $\sstat$ the polytope of mean parameters is the convex hull of the slices of $\sencsstat$ with fixed indices to $\shortcatvariables$, that is
	\begin{align*}
		\genmeanset 
		= \convhullof{\sencsstatat{\indexedshortcatvariables,\selvariable} \, : \, \shortcatindices\in\facstates, \, \basemeasureat{\indexedshortcatvariables}=1} \, . 
	\end{align*}	
\end{theorem}
\begin{proof}
	This follows from the fact, that the set of probability distributions is the convex hull of the one-hot encodings and the convex hull of mean parameters is a linear transform of that set.
\end{proof}


% Halfspace Representation
The convex polytope $\genmeanset$ can further be characterized as an intersection of half-spaces, as we state next.

\begin{theorem}\label{the:meanPolytopeHalfspaces}
	For any statistic $\sstat$ and base measure $\basemeasure$ there exists a finite collection $\halfspaceparams$ where $a_i[\selvariable]$ a vector and $b_i\in\rr$ for all $i\in[n]$ such that
	\begin{align*}
		\genmeanset
		= \left\{\meanparamat{\selvariable} \, : \, \forall_{i\in[n]} \, \contraction{\meanparamat{\selvariable},\normalvecofat{i}{\selvariable}}\leq\normalboundof{i} \right\} \, . 
	\end{align*}
\end{theorem}
\begin{proof}
	This is a standard result of combinatorical optimization, see e.g. \cite{ziegler_lectures_2013}.
\end{proof}

The determination of the the vectors $\halfspaceparams$ and is one reason for the intractability of probabilistic inference (see e.g. \cite{wainwright_graphical_2008}).



\begin{definition}\label{def:meanPolytopeFaces}
	Given a mean parameter polytope $\genmeanset$ in the half space representation of Theorem~\ref{the:meanPolytopeHalfspaces}, and any subset $\mathcal{I}\subset[n]$ we say that the set
	\begin{align*}
		\genfacesetof{\facecondset} = \left\{\meanparamat{\selvariable}\in\genmeanset \, : \, \forall_{i\in\mathcal{I}} \, \contraction{\meanparamat{\selvariable},\normalvecofat{i}{\selvariable}}\leq\normalboundof{i} \right\} 
	\end{align*}
	is the face to the constraints $\mathcal{I}$.
\end{definition}


\begin{theorem}\label{the:faceNormal}
	For any non-empty face $\genfacesetof{\facecondset}$ to a subset $\mathcal{I}\subset[n]$ there is a vector $\canparamat{\selvariable}$, which we call a normal of the face, such that
		\[ \genfacesetof{\facecondset} = \argmax_{\meanparam\in\genmeanset} \contraction{\canparamat{\selvariable},\meanparamat{\selvariable}}  \, . \]
	For any collection of positive $\lambda_i$, where $i\in\facecondset$, the vector
		\[ \canparamat{\selvariable} = \sum_{i\in\facecondset} \lambda_i\cdot\normalvecofat{i}{\selvariable}\]
	is a normal for $\genfacesetof{\facecondset}$.		
	%% UNCLEAR if this is true-> Are all possible normals in the span )
%	If $\canparamat{\selvariable}$ is a normal to a face, then for any positive scalars $\lambda[\selvariable]$ also the vector
%		\[ \seccanparam\left[\selvariable\right] = \contractionof{\canparamat{\selvariable},\labmda[\selvariable]}{\selvariable} \]
%	is a normal to the same face.
\end{theorem}
\begin{proof}
	The first claim follows trivially from the second.
	To show the second claim, let there be for $i\in\facecondset$ arbitrary positive scalars $\lambda_i$.
	We use that the face is non-empty, and thus there is a $\meanparamat{\selvariable}$ with
		\[ \contraction{\meanparamat{\selvariable},\normalvecofat{i}{\selvariable}}=\normalboundof{i} \]
	for all $i\in\facecondset$.
	Since for any $\meanparam\in\genmeanset$ 
		\[ \contraction{\meanparamat{\selvariable},\normalvecofat{i}{\selvariable}} \leq \normalboundof{i} \]
	it follows that
		\[ \max_{\meanparam\in\genmeanset} \contraction{\canparamat{\selvariable},\meanparamat{\selvariable}} 
		= \sum_{i\in\facecondset} \lambda_i \cdot \normalboundof{i} \, \, . \]
	The maximum is attained at a $\meanparamat{\selvariable}$, if and only if the equations $\contraction{\meanparamat{\selvariable},\normalvecofat{i}{\selvariable}}=\normalboundof{i}$ are satisfied for $i\in\facecondset$.
	This is equalt to $\meanparam\in\genfacesetof{\facecondset}$.
\end{proof}

% Notation
In a slide abuse of notation, we denote in this case $\genfacesetof{\canparam} = \genfacesetof{\facecondset}$.




\subsubsection{Examples}

% Minterm Exponential Family
\begin{example}[The minterm exponential family]\label{exa:mintermExpFamily}
	When taking as sufficient statistic the identity $\identityat{\shortcatvariables,\selvariableof{[\catorder]}}$, we can represent any positive distribution $\probtensor$ as a member of the exponential family, namely when choosing the canonical parameter
		\[ \canparam = \lnof{\probtensor} \, . \]
	The associated mean parameter is then, after relabeling the variables $\shortcatvariables$ by $\selvariableof{[\catorder]}$,
		\[ \meanparam = \probtensor \,  \]
	and $\meansetof{\identity,\ones}$ coincides with the set of probability tensors.
	For reasons to be explained in the Chapter~\ref{cha:logicalRepresentation} we refer to this family as the minterm exponential family.
\end{example}


% Markov Networks
Given a hypergraph with fixed node decoration, the different decorations of the hyperedges by tensors can be represented by an exponential family, as we show next.

\begin{theorem}[Exponential Representation of Markov Networks]
	For any hypergraph $\graph=(\nodes,\edges)$ we define a sufficient statistics 
		\[ \sstat = \bigtimes_{\edge\in\edges}  \sstatcoordinateof{\edge} \]
	where 
		\[ \sstatcoordinateof{\edge}(\catindexof{\nodes}) = \catindexof{\edge} \, . \]
	Given any Markov Network $\{\hypercoreof{\edge} \, : \, \edge\in\edges\}$ on $\graph$ with positive tensors $\hypercoreof{\edge}$ we define
		\[ \canparam = \bigtimes_{\edge\in\edges} \canparamof{\edge} \]
	where
		\[ \canparamof{\edge} =  \lnof{\hypercoreof{\edge}} \]
	and $\ln$ acts coordinatewise.
	Then, the Markov Network is in the member of the exponential family with trivial base measure, sufficient statistic $\sstat$ and parameters $\canparam$.
\end{theorem}
\begin{proof}
	We have for any $\catindexof{\nodes}$
	\begin{align}
	\prod_{\edge\in\edges} \hypercoreofat{\edge}{\indexedcatvariableof{\edge}}
		= \expof{\sum_{\edge\in\edges} \canparamofat{\edge}{\indexedcatvariableof{\edge}}}
		= \expof{\sum_{\edge\in\edges} \sbcontraction{\canparamofat{\edge}{\catvariableof{\edge}},\sstatcoordinateof{\edge}(\catindexof{\nodes})}}  \, .
	\end{align}
	Using that
		\[ \contractionof{\sstat,\canparam}{\nodevariables} = \sum_{\edge\in\edges} \contractionof{\sstatcoordinateof{\edge},\canparamof{\edge}}{\nodevariables} \]
	we get
	\begin{align}
		\contractionof{\{\hypercoreof{\edge}: \edge\in\edges\}}{\nodevariables} = \expof{\contractionof{\canparam,\sstat}{\nodevariables}} \, .
	\end{align}
	This implies 
	\begin{align}
		\normationof{\{\hypercoreof{\edge}: \edge\in\edges\}}{\nodevariables} = \normationof{\expof{\contractionof{\canparam,\sstat}{\nodevariables}}}{\nodevariables} \, .
	\end{align}
\end{proof}


% Mean parameters
The mean parameter of the Markov Network exponential family is the cartesian product of the marginals $\meanparamofat{\edge}{\catvariableof{\edge}}$ are often refered to as beliefs in the literature.
\red{They are outer bounded by local consistency polytope, which leads to the motivation of message passing algorithms!}

\subsection{Empirical Distributions}\label{sec:empDistribution}

Let us now apply the formalism of probability distributions in tensor network representations to encode data.
\begin{definition}\label{def:dataMap}
	Given a dataset $\dataset$ of samples of the factored system we define the sample selector map
		\[ \datamap : [\datanum] \rightarrow \facstates \]
	elementwise by 
		\[ \datamapof{\dataindex} = (\catindicesof{\dataindex}) \, . \]
	%% Empirical Distribution
	The empirical distribution to the sample selector map $\datamap$ is the probability distribution
	\begin{align*}
		\empdistributionat{\shortcatvariables}
		\coloneqq \sbnormationof{\datacore}{\shortcatvariables} \, ,
	\end{align*}
	where by $\datacore$ we denote the relational encoding (see \defref{def:functionRepresentation}) of the sample selector map, and the distributed variables $\shortcatvariables$ are the head variables of the relational encoding.
\end{definition}

% Sample Selector map (former data tensor)
The relational encoding of the sample selector map is the sum 
	\[ \datacoreat{\datvariable,\shortcatvariables} 
	= \sum_{\dataindexin} \onehotmapofat{\dataindex}{\datvariable} \otimes \onehotmapofat{\catindicesof{\dataindex}}{\shortcatvariables} \, . \]
% Interpretation of the empirical distribution
Each coordinate $\shortcatindices$ of the empirical distribution can be calculated by
\begin{align*}
	\empdistributionat{\indexedshortcatvariables} 
	= \frac{1}{\sbcontraction{\datacore}} \left( \sum_{\dataindexin} \onehotmapofat{\catindicesof{\dataindex}}{\indexedshortcatvariables}  \right) 
	= \frac{\cardof{\dataindexin \, : \, (\catindicesof{\dataindex}) = (\catindices)}}{\cardof{\dataindexin}}
\end{align*}
and is thus interpreted as the frequency of the corresponding world in the data.

%% Basic CP Decomposition 
The relational encoding of the sample selector map is a sum of one-hot encodings of the data indices and the corresponding sample states.
Such sums of basis tensors will be further investigated in Section~\ref{sec:basisCP} as basis CP decompositions.
We now exploit this structure to find efficient tensor network decompositions (see Figure~\ref{fig:DataDecomposition}) based on matrices encoding its variables.


\begin{figure}[h]
\begin{center}
	\begin{tikzpicture}[scale=0.35, thick] % , baseline = -3.5pt


\begin{scope}[shift={(-15,2)}]

\node[anchor=center] (text) at (-1,3) {${a)}$};


\node [circle, draw, thick, fill=gray!50, minimum size = \nodeminsize] (T1) at (0,0) {\tiny $\randomxof{0}$};	
\node [circle, draw, thick, fill=gray!50, minimum size = \nodeminsize] (T2) at (3,0) {\tiny $\randomxof{1}$};	
\node[anchor=center] (text) at (6,0) {\small $\cdots$};
\node [circle, draw, thick, fill=gray!50,minimum size = \nodeminsize] (T3) at (9,0) {};
\node[anchor=center] (text) at (9,0) {\tiny $\randomxof{\atomorder-1}$};	

\node [circle, draw, thick, fill=gray!50, minimum size = \nodeminsize] (C) at (4.5,-5) {};
\node[anchor=center] (text) at (4.5,-5){ \tiny $\datavariable$};	

\draw[->] (C) -- (T1) node [midway,left] {$\edgeof{0}$};
\draw[->] (C) -- (T2) node [midway,right] {$\edgeof{1}$};
\draw[->] (C) -- (T3) node [midway,right] {$\edgeof{\catorder-1}$};

\end{scope}

\node[anchor=center] (text) at (-1,5) {${b)}$};




\drawatomindices{0}{2}
\draw (-1,1) rectangle (5,-1);
\node[anchor=center] (text) at (2,0) {$\datacore$};
\draw[<-] (2,-1) -- (2,-3) node[midway, right] {\tiny $\datavariable$};

\node[anchor=center] (text) at (7,0) {${=}$};


\begin{scope}[shift={(10,2)}]

\newcommand{\conposseldec}{4.5,-5.5}

\draw[fill] (\conposseldec) circle (0.25cm);
\draw[<-] (\conposseldec) -- (4.5,-7.5) node[midway, right] {\tiny $\datavariable$};
%\draw[dashed] (3.5,-7.5) rectangle (5.5, -9.5);
%\node[anchor=center] (text) at (4.5,-8.5) {\small $\frac{1}{\datanum} \ones$};

\draw[<-] (0,1) -- (0,-1) node[midway,left] {\tiny $\catvariableof{0}$};
\draw (-1,-1) rectangle (1, -3);
\node[anchor=center] (text) at (0,-2) {\small $\datacoreof{0}$};
\draw[<-] (0,-3) to[bend right=20] (\conposseldec);


\draw[<-] (3,1) -- (3,-1) node[midway,left] {\tiny $\catvariableof{1}$};
\draw (2,-1) rectangle (4, -3);
\node[anchor=center] (text) at (3,-2) {\small $\datacoreof{1}$};
\draw[<-] (3,-3) to[bend right=20]  (\conposseldec);

\node[anchor=center] (text) at (6,-2) {$\cdots$};

\draw[<-] (9,1) -- (9,-1) node[midway,left] {\tiny $\catvariableof{\atomorder-1}$};
\draw (7.75,-1) rectangle (10.25, -3);
\node[anchor=center] (text) at (9,-2) {\small $\datacoreof{\atomorder-1}$};
\draw[<-] (9,-3) to[bend left=20]  (\conposseldec);




\end{scope}

		


\end{tikzpicture}
\end{center}
\caption{
	Representation of a dataset $\dataset$.
	a) Interpretation as a data selection variable $\datvariable$ selecting states for the variables $\shortcatvariables$.
	b) Corresponding decomposition of the relational encoding $\datacore$ into a tensor network in the basis $\cpformat$ Format (see Section~\ref{sec:basisCP}), where $\hypercoreof{\edgeof{\atomenumerator}}=\datacoreof{\atomenumerator}$.
	%Without the contraction with the dashed $\frac{1}{\datanum}\onesat{\datvariable}$ core, the datacore encodes the distribution conditioned on a datapoint. 
}
\label{fig:DataDecomposition}
\end{figure}


\begin{theorem}\label{the:empCPRep}
	Given a data map $\datamap: [\datanum] \rightarrow \facstates$ we define for $\catenumeratorin$ its coordinate maps
		\[ \datamap_{\catenumerator} : [\datanum] \rightarrow [\catdimof{\catenumerator}] \]
	by
		\[  \datamap_{\catenumerator}(\dataindex) = \catindexof{\catenumerator}^\dataindex \, .  \]
	We then have
	\begin{align*}
		\rencodingofat{\datamap}{\datvariable,\shortcatvariables}  
		= \contractionof{
		\{\rencodingofat{\datamap^{\atomenumerator}}{\datvariable,\catvariableof{\atomenumerator}} : \atomenumeratorin \} 
		}{\datvariable,\shortcatvariables} 
	\end{align*}
	and
	\begin{align*}
	\empdistributionat{\shortcatvariables}
	= \sbcontractionof{\datacoreat{\datvariable,\shortcatvariables}, \frac{1}{\datanum}\onesat{\datvariable}}{\shortcatvariables} 
	= \sbcontractionof{\datacoreofat{0}{\datvariable,\catvariableof{0}},\ldots,\datacoreofat{\catorder-1}{\catvariableof{\catorder-1}}, \frac{1}{\datanum}\onesat{\datvariable}}{\shortcatvariables} \, . 
	\end{align*}
	In a contraction diagram this reads
	\begin{center}
		\begin{tikzpicture}[scale=0.35, thick] % , baseline = -3.5pt






\drawatomindices{0}{2}
\draw (-1,1) rectangle (5,-1);
\node[anchor=center] (text) at (2,0) {$\empdistribution$};


\node[anchor=center] (text) at (7,0) {${=}$};

\node[anchor=center] (text) at (22,-1) {${\cdot}$};

\begin{scope}[shift={(10,2)}]

\newcommand{\conposseldec}{4.5,-5.5}

\draw[fill] (\conposseldec) circle (0.25cm);
\draw[<-] (\conposseldec) -- (4.5,-7.5) node[midway, right] {\tiny $\datavariable$};
\draw (3.5,-7.5) rectangle (5.5, -9.5);
\node[anchor=center] (text) at (4.5,-8.5) {\small $\frac{1}{\datanum} \ones$};

\draw[<-] (0,1) -- (0,-1) node[midway,left] {\tiny $\catvariableof{0}$};
\draw (-1,-1) rectangle (1, -3);
\node[anchor=center] (text) at (0,-2) {\small $\datacoreof{0}$};
\draw[<-] (0,-3) to[bend right=20] (\conposseldec);


\draw[<-] (3,1) -- (3,-1) node[midway,left] {\tiny $\catvariableof{1}$};
\draw (2,-1) rectangle (4, -3);
\node[anchor=center] (text) at (3,-2) {\small $\datacoreof{1}$};
\draw[<-] (3,-3) to[bend right=20]  (\conposseldec);

\node[anchor=center] (text) at (6,-2) {$\cdots$};

\draw[<-] (9,1) -- (9,-1) node[midway,left] {\tiny $\catvariableof{\atomorder-1}$};
\draw (7.75,-1) rectangle (10.25, -3);
\node[anchor=center] (text) at (9,-2) {\small $\datacoreof{\atomorder-1}$};
\draw[<-] (9,-3) to[bend left=20]  (\conposseldec);



\end{scope}

		


\end{tikzpicture}
	\end{center}
\end{theorem}
\begin{proof}
	The first claim is a special case of Theorem~\ref{the:functionDecompositionBasisCP}, to be shown in Chapter~\ref{cha:tensorEncodings}.
	To show the second claim we notice
		\[ \sbcontraction{\datacore} = \sum_{\datindexin} \sbcontraction{\rencodingofat{\datamap}{\datvariable=\datindex,\shortcatvariables}} = \datanum \,  . \]
	With the first claim it now follows that
	\begin{align*}
		\empdistributionat{\shortcatvariables}
		 = \sbnormationof{\datacore}{\shortcatvariables}
		 = \frac{\sbcontractionof{\datacore}{\shortcatvariables}}{\sbcontraction{\datacore}} 
		 =  \contractionof{
		\{\rencodingofat{\datamap^{\atomenumerator}}{\datvariable,\catvariableof{\atomenumerator}} : \atomenumeratorin \} \cup \{ \frac{1}{\datanum} \onesat{\datvariable} \}
		}{\datvariable,\shortcatvariables}  \, . 
	\end{align*}
\end{proof}


The cores $\datacoreof{\atomenumerator}$ are matrices storing the value of the categorical variable $\catvariableof{\atomenumerator}$ in the sample world indexed by $\dataindex$.

% Interpretation
From the proof of Theorem~\ref{the:empCPRep} we notice that the scalar $\frac{1}{\datanum}$ could be assigned with any core in a representation of $\empdistribution$, and the core $\onesat{\datvariable}$ is thus redundant in the contraction representation.
However, creating the core $\frac{1}{\datanum}\onesat{\datvariable}$ provides us with a simple interpretation of the empirical distribution.
We can understand $\frac{1}{\datanum}\onesat{\datvariable}$ as the uniform probability distribution over the samples, which is by the map $\datamap$ forwarded to a distribution over $\facstates$.
The one-hot encoding of each sample is itself a probability distribution, which is understood as conditioned on the respective state of the sample selection variable $\datvariable$.
The conditional distribution $\datacore$ therefore forwards the uniform distribution of the samples to a distribution of the variables $\shortcatvariables$.
In the perspective of a Bayesian Network (see Figure~\ref{fig:DataDecomposition}, the variable $\datvariable$ served as single parent for each categorical variable $\catvariableof{\catenumerator}$.


%%% Inductive vs Deductive perspective
%Each evidence is a probability distribution
%\begin{itemize}
%	\item Inductive Reasoning: When we interpret evidence as a datapoint, they are typically a basis tensor specifying precisely a world.
%	\item Deductive Reasoning: Evidence is a partial observation of the world, typically basis vectors at each variable, but leaving some unspecified ($\ones$).
%	We then interpret the evidence as being a uniform distribution over the worlds not contradicting with the evidence.
%\end{itemize}


\subsection{Discussion and Outlook}

\begin{remark}[Alternative definitions of graphical models]
	In the literature, tensor networks are often called dual to the hypergraphs defining graphical models (see e.g. \cite{robeva_duality_2019}).
	The duality becomes clear, when one interpretes the tensors as cores and their common variables as edges.
	We in this work avoid this ambiguity by directly defining tensor networks as decoration of hyperedges by tensors.
	
	Often, the tensors decorating hyperedges are called factors and their logarithm features \cite{koller_probabilistic_2009}.
	
	Further, we directly use hypergraphs instead of the more canonical association of factors with cliques of a graph.
	This avoids the discussion of non-maximal cliques as decorated with trivial tensors.
	Such hypergraphs follow the same line of though compared with factor graphs, which are bipartite graphs with nodes either corresponding with single variables or with a collection of them affected by a factor.
\end{remark}








\section{Probabilistic Reasoning}\label{cha:probReasoning} 

We have investigated means to store the knowledge about a system and now turn to the retrieval of information, a process called inference.

% 
Contraction of the relational encoding of a function with a Markov Network gives the statistics over the values of the functions.
When contracting the function directly, we get the expectation.

% Message passing
%Another approximation comes from an approximation of the contractions itself. 
One can increase the efficiency of inference algorithms by using approximative contractions.
Here, message passing schemes can be applied as to be introduced in Chapter~\ref{cha:localContractions}.


\subsection{Queries}

% Motivation of queries: Avoid distribution instantiation
In the previous chapter, we have derived efficient representation schemes of probability distributions based on tensor network decompositions.
We have argued that one should avoid naive instantiation of these distributions based on an storage of each coordinates.
In the task of reasoning, we want to retrieve information encoded in the probability distribution.
To derive an efficient approach one therefore needs to avoid instantiating the distribution in a coordiantewise manner in an intermediate step.
We thus formalize a basic reasoning scheme by contractions of the decomposed distributions with query tensors.

\subsubsection{Querying by functions}

We can formalize queries by retrieving expectations of functions given a distribution specified by probability tensors. 
We exploit basis calculus in defining categorical variables $\catvariableof{\exfunction}$ to tensors $\exfunction$, which are enumerating the set $\imageof{\exfunction}$.
More details on this scheme are provided in Chapter~\ref{cha:basisCalculus}, see Definition~\ref{def:functionRelationEncoding} therein.

\begin{definition}\label{def:queries}
	The marginal query of a probability distribution $\probof{\shortcatvariables}$ by a tensor 
		\[ \exfunction : \facstates \rightarrow \rr \]
	is the vector $\probof{\catvariableof{\exfunction}} \in \rr^{\cardof{\imageof{\exfunction}}}$ defined as the contraction
	\begin{align*}
		\probof{\catvariableof{\exfunction}} = \contractionof{\probof{\shortcatvariables},\rencodingofat{\exfunction}{\shortcatvariables,\catvariableof{\exfunction}}}{\catvariableof{\exfunction}} \, . 
	\end{align*}
	
	% Used in connection to mean parameters
	The expectation query of $\probtensor$ by $\exfunction$ is 
	\begin{align*}
		\expectationof{\exfunction} = \sbcontraction{\exfunction, \probtensor} \, . 
	\end{align*}
	
	% Used for sampling
	Given another tensor $\secexfunction: \facstates \rightarrow \rr $ the conditional query of the probability distribution $\probof{\shortcatvariables}$ by the tensor $\exfunction$ conditioned on the tensor $\secexfunction$ is the matrix $\condprobof{\catvariableof{\exfunction}}{\catvariableof{\secexfunction}}\in\rr^{\cardof{\imageof{\exfunction}}}\otimes \rr^{\cardof{\imageof{\secexfunction}}}$ defined as the normation
	\begin{align*}
		\condprobof{\catvariableof{\exfunction}}{\catvariableof{\secexfunction}} 
		= \normationofwrt{\{
		\probof{\shortcatvariables},\rencodingofat{\exfunction}{\shortcatvariables,\catvariableof{\exfunction}},\rencodingofat{\secexfunction}{\shortcatvariables,\catvariableof{\secexfunction}}
		\}}{
		\catvariableof{\exfunction}}{\catvariableof{\secexfunction}
		} \, . 
	\end{align*}
\end{definition}

%% Relation of queries and expectation queries
Expectation queries are contractions of marginal queries with identities, that is
	\[ \expectationof{\exfunction} = \sbcontraction{\probof{\catvariableof{\exfunction}} \idrestrictedto{\imageof{\exfunction}}{\catvariableof{\exfunction}} } \, . \]
This will be shown in more detail in Chapter~\ref{cha:basisCalculus} in Corollary~\ref{cor:rhoToNormal}.

%% Conditional Probabilities and conditional queries
Conditional probabilities are queries, where the tensors $\exfunction$ and $\secexfunction$ are identity mappings in the respective variable state spaces.
Conversely, we can understand the conditional query $\condprobof{\exfunction}{\secexfunction}$ as the conditional probability of $\exfunction$ conditioned on $\secexfunction$, of the underlying Markov Network with cores $\{\probtensor, \rencodingof{\exfunction}, \rencodingof{\secexfunction} \}$ and variables $\catvariableof{\exfunction},\catvariableof{\secexfunction}$ besides the variables distributed by $\probtensor$.

%% Expectations as event queries -> Consistency with $\probof{X=i}$?
We further denote event queries by
	\[  \expectationof{\exfunction=z} = \sbcontraction{\probtensor,\rencodingof{\exfunction},\onehotmapof{z}}\]
where by $\onehotmapof{z}$ be denote the one hot encoding of the state $z$ with respect to some enumeration.
Let us note that they are further contraction of the queries in Definition~\ref{def:queries} since by Theorem~\ref{the:splittingContractions}
\begin{align*}
	 \expectationof{\exfunction=z} 
	& =  \sbcontraction{ \sbcontractionof{\probtensor,\rencodingof{\exfunction}}{\catvariableof{\exfunction}} ,\onehotmapof{z}}\\
	& =  \sbcontraction{ \probof{\exfunction} ,\onehotmapof{z}} \, . 
\end{align*}

%% OLD: Defining queries by 
%\begin{definition}
%	The expectation of functions $\exfunction$ given a probability tensor is the contraction
%		\[ \expectationofwrt{\exfunction(\catvariables)}{\catvariables\sim\probtensor} = 
%			\contractionof{\{\probtensor,\rencodingof{\exfunction}\}}{\{\exfunctiontargetvariables \}} \, . 
%		\]
%\end{definition}
%This is the canonical definition of expectations, since summing function values weighted by the probability of the argument.
%When we have an unnormalized probability distribution $\phi$ the expectation is the quotient
%\begin{align*}
%	\expectationofwrt{\exfunction(\catvariables)}{\catvariables\sim\phi}  = \frac{
%		\contractionof{\{\phi,\ftensorof{\exfunction}\}}{\{\exfunctiontargetvariables \}} 
%	}{
%		\contractionof{\{\phi\}}{\varnothing} 
%	} \, . 
%\end{align*}

%\subsubsection{Conditional Probability Queries}
%
%Typical queries are the computation of an a posteriori distribution given evidence.
%This is just the contraction.
%
%%% As expectation
%The query consists of the one-hot encoding of the evidence and Ids elsewhere.
%The result is then interpreted as another probability distribution, defined as a Markov network and the possible need to normalize with the partition function.
%
%Given evidence, condition the probability tensor on that evidence.






\subsubsection{MAP Queries}

Find the maximal variable of a tensor is a problem, which can be approached by sampling methods as we discuss here.

\begin{definition}
	Given a tensor $\hypercore$ the MAP query is the problem 
	\begin{align}
		\argmax_{\catindices} \hypercoreat{\indexedcatvariables} \, .
	\end{align}
\end{definition}

%Often, the generation of a full (conditioned) probability tensor can be infeasible, if too many variables are queries.
%Having a tensor network decomposition of the probability tensor avoids this generation.

% One hot perspective
By coordinate calculus, we notice that
\begin{align}
	\hypercoreat{\indexedcatvariables} 
	\sbcontraction{\hypercore, \onehotmapof{\catindices}} \, .
\end{align}
Given the image $\Gamma^{\elformat}$ of one-hot encodings, the MAP query problem is equivalent to 
\begin{align}
	\max_{\catindices} \hypercoreat{\indexedcatvariables} 
	= \max_{\theta\in\Gamma^{\elformat}} \sbcontraction{\hypercore, \theta} \, .
\end{align}
We can thus understand MAP queries as a Tensor Network approximation problem, where the approximating tensor are the one-hot encodings of states.

\begin{remark}[MAP queries on energy and probability tensors]
% Usage on energies and probabilities
	Since the exponential function is monotonic, MAP queries on the energy tensor of an exponential family with uniform base measure are equivalent to MAP queries of their energies.
\end{remark}


\subsubsection{Answering queries by energy contractions}

Let us now interpret a probability tensor at hand as a member of an exponential family (see Section~\ref{sec:exponentialFamilies}), which is always possible when taking the naive exponential family.

\begin{lemma}\label{lem:energyContractionQueries} % This is a statement about "full" queries.
	For any probability distribution $\probtensor$ with $\probtensor= \normationof{\expof{\energytensorat{\shortcatvariables}}}{\shortcatvariables}$, disjoint subsets $\nodesa,\nodesb \subset [\catorder]$ with $\nodesa\cup\nodesb=[\catorder]$  and any $\catindexof{\nodesb}$ we have
		\[ \condprobof{\catvariableof{\nodesa}}{\indexedcatvariableof{\nodesb}} 
			= \normationof{
				\expof{\energytensorat{\catvariableof{\nodesa},\indexedcatvariableof{\nodesb}}}
		}{\catvariableof{\nodesa}} \, .\]
\end{lemma}
\begin{proof}
	Since no summation is commuted.
\end{proof}

Thus, it suffices to build the selection encoding of the statistics, and we can avoid the usage of the relational encoding.

% 
We notice, that Lemma~\ref{lem:energyContractionQueries} does not generalize to situations, where $\nodesa\cup\nodesb\neq[\catorder]$, since summation over the indices of the variables $[\catorder]/\nodesa\cup\nodesb$ and contraction do not commute.
%\red{In that case, each summed index produces a factor.}


\begin{lemma}  %\red{TRUE?}
	For any probability distribution $\probtensor$ with $\probtensor= \normationof{\expof{\energytensorat{\shortcatvariables}}}{\shortcatvariables}$, disjoint subsets $\nodesa,\nodesb \subset [\catorder]$ and any $\catindexof{\nodesb}$ we have
		\[ \condprobof{\catvariableof{\nodesa}}{\indexedcatvariableof{\nodesb}} 
			=
			\normationof{
			 \sum_{\catindexofin{[\catorder]/\nodesa\cup\nodesb}} 
				 \expof{\energytensorat{\catvariableof{\nodesa},\indexedcatvariableof{\nodesb},\indexedcatvariableof{[\catorder]/\nodesa\cup\nodesb}}}
		}{\catvariableof{\nodesa}} \, .\]
\end{lemma}
\begin{proof}
	By splitting the contraction into terms to $\nodesa\cup\nodesb$. % and using Lemma~\ref{lem:energyContractionQueries}.
\end{proof}




\subsection{Sampling based on queries}


Let us here investigate how to draw samples from distributions $\probtensor$, based on queries on $\probtensor$.

%Need to generate the full conditional probability distribution by contraction and then sample from it.
Since there are $\prod_{\node\in\nodes}\catdimof{\node}$ coordinates stored in $\probtensor$, naive methods are often infeasible.
One can instead exploit a representation of $\probtensor$ by a Markov network or the energy term in an exponential family for efficient algorithms and sample from local proxy distributions resulting from contractions and interpreted as marginal and conditional probabilities.

\subsubsection{Exact Methods}

Forward Sampling (see Algorithm~\ref{alg:ForwardSampling}) uses a chain decomposition (see Theorem~\ref{the:chainRule}) of a probability distribution to iteratively sample the variables.

\begin{algorithm}[hbt!]
\caption{Forward Sampling}\label{alg:ForwardSampling}
\begin{algorithmic}
\For{$\catenumeratorin$}
	\State Draw $\catindexof{\catenumerator}\in[\catdimof{\catenumerator}]$ from the conditional query
		\[ \condprobof{\catvariableof{\catenumerator}}{\indexedcatvariableof{\seccatenumerator} \, : \, \seccatenumerator < \catenumerator} \]
\EndFor
\end{algorithmic}
\end{algorithm}

%
Forward Sampling is especially efficient, when sampling from a Bayesian Network respecting the topological order of its nodes.
The reason for this lies in trivilizations of all conditional distributions, which heads are not included in the evidence of previously sampled variables.
More technically, we can show that
	\[ \condprobof{\catvariableof{\catenumerator}}{\indexedcatvariableof{\seccatenumerator} \, : \, \seccatenumerator < \catenumerator}  
	= \condprobof{\catvariableof{\catenumerator}}{\indexedcatvariableof{\parentsof{\catenumerator}}} \, , \]
which is only involving a single core of a Bayesian network.
\red{This can be shown using Corollary~\ref{cor:onesHead} to be derived in Chapter~\ref{cha:basisCalculus}.}


%% Comment on rejection Sampling 
%When sampling from conditional probability distributions, one can sample from the conditioned distribution instead.
%However, the conditioning changes the structure of the distribution, and conditioned Bayesian Networks are not Bayesian Networks on the same graph.
%One ways around is rejection sampling, where one samples from the unconditioned distribution and rejects samples not satisfying the event conditioned on.
%When the event conditioned on is of small probability, methods like rejection sampling will come with large runtimes.

\subsubsection{Approximate Methods}

% Problem of many variables
When there are many variables to be sample, the computation of the conditional probability to all variables can be infeasible.
One way to overcome this is Gibbs Sampling: Iteratively resemble single variables given the rest as evidence.

%\subsubsection{Gibbs Sampling}

% Still old: Sample from Marginal
Sample each variable independent from the marginal distribution.
Then, alternate through the variables and sample each variable from the conditional distribution taking the others as evidence.

\begin{algorithm}[hbt!]
\caption{Gibbs Sampling}\label{alg:Gibbs}
\begin{algorithmic}
\For{$\catenumeratorin$}
	\State Draw State for atom $\catenumerator$ from initialization distributions. % In implementation: Initialize with ones and draw -> Avoids zero probability state
\EndFor
\While{Stopping criterion is not met}
\For{$\catenumeratorin$}
	\State Draw $\catindexof{\catenumerator}\in[\catdimof{\catenumerator}]$ from the conditional query
		\[ \condprobof{\catvariableof{\catenumerator}}{\indexedcatvariableof{\seccatenumerator}\, : \seccatenumerator \neq \catenumerator} \]
\EndFor
\EndWhile
\end{algorithmic}
\end{algorithm}


% Energy

Gibbs can be implemented based on the energy tensor $\energytensor$ of the probability tensor, as follows form the Lemma~\ref{lem:energyContractionQueries}.



%	\[ \condprobof{\catvariableof{\catenumerator}}{\{\catvariableof{\seccatenumerator}=\catindexof{\seccatenumerator} \, : \seccatenumerator \neq \catenumerator\}} 
%	= \normationofwrt{\expof{\contractionof{\{\energytensor\}\cup\{\onehotmapof{\catindexof{\seccatenumerator}} \, : \seccatenumerator \neq \catenumerator \}}{\catvariableof{\catenumerator}}}}{\catvariableof{\catenumerator}}{\varnothing}  \, .\]
	


\red{This is in contrast with forward sampling, where we need to sum over many coordinates of the exponentiated energy tensor, which amounts to the representation of the probability distribution as a tensor network using relational encodings.
}
%where the operation with energy tensors and selection encodings is not efficient.}




\subsubsection{Simulated Annealing}

\red{MAP queries are approximated by sampling from annealed distributions: Use $\hypercore$ as the energy tensor, e.g. as parameter tensor to the naive exponential family.}

%\begin{remark}\label{rem:simulatedAnnealing}
% Simulated annealing
	\red{Here by the naive exponential family!}
	Simulated annealing manipulates the probability used to sample $\catindexof{\catenumerator}$ in terms of an inverse temperature parameter $\invtemp$, by
		\[ \probtensor \rightarrow \frac{\expof{\invtemp\cdot\lnof{\probtensor}}}{\contraction{\expof{\invtemp\cdot\lnof{\probtensor}}} } \, . \]
	When the temperature is larger than $1$, the probability of states with low probability increases while the probability of states with large probability decreases and for low temperatures the opposite.
	Simulated annealing, that is the decrease of the temperature to $0$ during Gibbs sampling biases the algorithm towards states with large probability.
%	Tuning this parameter can improve the convergence of Gibbs Sampling.

	% On exponential families
	For any exponential family the transformation 
		\[ \energytensor \rightarrow \invtemp \cdot \energytensor  \]
	can be performed by rescaling the canonical parameters as
		\[ \canparam \rightarrow \invtemp \cdot \canparam \, . \]
%\end{remark}







\subsection{Maximum Likelihood Estimation} % Stuff from Parameter Estimation - Problem that Part I is called inference?

Let us now turn to inductive reasoning tasks, where a probabilistic model is trained on given data.

\subsubsection{Likelihood and Loss}

Given a datapoint $\datamapof{\dataindex}$ consisting of the images of the data selecting map $\datamap$ (see Definition~\ref{def:dataMap}), the likelihood given a Markov Logic Network is denoted as
	\[ \probat{\shortcatvariables = \datamapof{\dataindex}} \, . \]
	
% Independent assumption
When all $\datamapof{\dataindex}$ are drawn independently from $\probat{\shortcatvariablelist}$, we can factorize into
	\[ \probat{\data}  = \prod_{\dataindexin} \probof{\shortcatvariables=\datamapof{\dataindex}} \, . \]

% Logarithm
It is convenient to apply a logarithm on the objective, which does not influence the optimum when optimizing this quantity.
This is especially useful, when investigating the convergence of the objective for $\datanum\rightarrow\infty$ (see Chapter~\ref{cha:mlnConcentration}).

\begin{definition}\label{def:loss}
	We define the loss of a distribution $\probtensor$ as
	\begin{align*}%\label{eq:defLikelihoodLossPL}
		\lossof{\probtensor} 
		= \frac{1}{\datanum} \lnof{\probof{\data}} 
	\end{align*}
\end{definition}

We now state the Maximum Likelihood Problem in the form
\begin{align}\tag{$\mathrm{P}_{\Gamma, \empdistribution}$}\label{prob:parameterMaxLikelihood}
	\argmin_{\probtensor\in\Gamma} \lossof{\probtensor} \, . % Naive Exponential Family perspective!
\end{align}



\subsubsection{Entropic Interpretation}

\begin{definition}[Shannon entropy]
	The information content or the Shannon entropy of a distribution is defined as
		\[ \sentropyof{\probtensor} 
		:= \expectationofwrt{-\lnof{\probof{\shortcatvariables}}}{\shortcatvariables\sim\probtensor} 
		= \sbcontraction{\probtensor,-\lnof{\probtensor}} \, . \]
	%	= - \sum_{\shortcatindices} \probof{\indexedshortcatvariables} \cdot \lnof{\probof{\indexedshortcatvariables}} \, . \]
	We depict this in a tensor network diagram with an ellipsis denoting a coordinatewise transform (see Chapter~\ref{cha:coordinateCalculus}) with a natural logarithm $\ln$ as:
	\begin{center}
		\begin{tikzpicture}[scale=0.3,thick] % , baseline = -3.5pt

\node[anchor=center] (text) at (-8,-5) {\small $\sentropyof{\probtensor}$};

\node[anchor=center] (text) at (-5,-5) {\small ${=}$};

\node[anchor=center] (text) at (-3,-2) {\small $\mathrm{ln}$};
\draw (2,-2) ellipse (6 and 2.75);

\draw (-1,-1) rectangle (5,-3);
\node[anchor=center] (text) at (2,-2) {\small $\probtensor$};
\draw (-1,-7) rectangle (5,-9);
\node[anchor=center] (text) at (2,-8) {\small $\probtensor$};
\draw (0,-5)--(0,-3); 
\draw (0,-5)--(0,-7) node[midway,left] {\tiny $\catvariableof{0}$}; 
\draw (1.5,-5)--(1.5,-3); 
\draw (1.5,-5)--(1.5,-7) node[midway,left] {\tiny $\catvariableof{1}$}; 
\node[anchor=center] (text) at (3,-4) {$\cdots$};
\draw (4,-5)--(4,-3);
\node[anchor=center] (text) at (3,-6) {$\cdots$};
\draw (4,-5)--(4,-7) node[midway,right] {\tiny $\catvariableof{\catorder\shortminus1}$}; 

%\drawatomcore{3.5}{-8}{$\probtensor$}
%\drawatomindices{3.5}{-12}	
%\draw (5.5,-9)--(5.5,-7) node[midway,right] {\tiny $\catvariableof{\exformula}$};

\end{tikzpicture}
	\end{center}
\end{definition}

\begin{definition}[Cross entropy]\label{def:crossEntropy}
	The cross entropy between two distributions is defined as 
		\[ \centropyof{\probtensor}{\secprobtensor} 
		:=  \expectationofwrt{-\lnof{\secprobtensor[\shortcatvariables]}}{\shortcatvariables\sim\probtensor} 
		= \sbcontraction{\probtensor,-\lnof{\secprobtensor}} \, . \]
		%- \sum_{\catindices}  \probof{\indexedcatvariables} \cdot \lnof{\secprobtensor[\indexedshortcatvariables]}  \, . \]
	We depict this in a tensor network diagram with an ellipsis denoting a coordinatewise transform (here the $\ln$) as :
	\begin{center}
		\begin{tikzpicture}[scale=0.3,thick] % , baseline = -3.5pt

\node[anchor=center] (text) at (-8,-5) {\small $\centropyof{\probtensor}{\tilde{\probtensor}}$};

\node[anchor=center] (text) at (-5,-5) {\small ${=}$};

\node[anchor=center] (text) at (-3,-2) {\small $\mathrm{ln}$};
\draw (2,-2) ellipse (6 and 2.75);

\draw (-1,-1) rectangle (5,-3);
\node[anchor=center] (text) at (2,-2) {\small $\tilde{\probtensor}$};
\draw (-1,-7) rectangle (5,-9);
\node[anchor=center] (text) at (2,-8) {\small $\probtensor$};
\draw (0,-5)--(0,-3); 
\draw (0,-5)--(0,-7) node[midway,left] {\tiny $\catvariableof{0}$}; 
\draw (1.5,-5)--(1.5,-3); 
\draw (1.5,-5)--(1.5,-7) node[midway,left] {\tiny $\catvariableof{1}$}; 
\node[anchor=center] (text) at (3,-4) {$\cdots$};
\draw (4,-5)--(4,-3);
\node[anchor=center] (text) at (3,-6) {$\cdots$};
\draw (4,-5)--(4,-7) node[midway,right] {\tiny $\catvariableof{\catorder\shortminus1}$}; 

%\drawatomcore{3.5}{-8}{$\probtensor$}
%\drawatomindices{3.5}{-12}	
%\draw (5.5,-9)--(5.5,-7) node[midway,right] {\tiny $\atomlegindexof{\exformula}$};

\end{tikzpicture}
	\end{center}
\end{definition}

%% Vanishing coordinates case
We here use $\lnof{0}=-\infty$ and $0\cdot \lnof{0} = 0$. 
Then we have $\centropyof{\probtensor}{\secprobtensor} = \infty$ if and only if there is a $\shortcatindices$ such that $\probof{\indexedshortcatvariables}>0$ and $\secprobtensor[\indexedshortcatvariables]=0$.


% KL Divergence
The Gibbs inequality states that
		\[ \centropyof{\probtensor}{\secprobtensor} \geq \sentropyof{\probtensor} \, . \]
The difference between both sides is called the Kullback Leibler Divergence and a useful metric in reasoning, since it vanishes for $\probtensor=\secprobtensor$.

\begin{definition}[Kullback Leibler Divergence]\label{def:KLDivergence}
	The KL divergence between two distributions is defined as 
		\[ \kldivof{\probtensor}{\secprobtensor} = \centropyof{\probtensor}{\secprobtensor} - \sentropyof{\probtensor}  \, . \]
\end{definition}

We are now ready to provide an entropic interpretation of the loss introduced in Definition~\ref{def:loss}.

\begin{lemma}
	Given a data selecting map $\datamap$ and a distribution $\probtensor$ we have
	\begin{align}
		\lossof{\probtensor} =  \centropyof{\empdistribution}{\probtensor} \, . % \sbcontraction{\empdistribution,\lnof{\probtensor}} \, . 
	\end{align}
\end{lemma}
\begin{proof}
	We have
	\begin{align*}
		\lossof{\probtensor} 
		& = \frac{1}{\datanum} \lnof{\probof{\data}} 
		= \frac{1}{\datanum} \sum_{\datain} \lnof{\probof{\shortcatvariables =\datamap(\dataindex)}} 
		= \frac{1}{\datanum} \sum_{\datain} \contraction{\{\lnof{\probtensor},\onehotmapof{\datamap(\dataindex)}\}} \\ 
		& = \sbcontraction{\empdistribution,\lnof{\probtensor}} \, .
	\end{align*}
	Comparing with the negative log likelihood we notice that that loss coincides with the cross-entropy between the empirical distribution $\empdistribution$ and $\probtensor$, i.e.
		\[ \lossof{\probtensor} = \centropyof{\empdistribution}{\probtensor} \, . \]
\end{proof}


% Interpretation of MLE as Cross-Entropy Minimization

We can therefore rewrite Problem~\ref{prob:parameterMaxLikelihood} as minimization of cross-entropies and of Kullback Leibler divergences as
\begin{align*}
	\argmin_{\probtensor\in\Gamma} \lossof{\probtensor} 
	= \argmin_{\probtensor\in\Gamma} \centropyof{\empdistribution}{\probtensor} 
	= \argmin_{\probtensor\in\Gamma} \kldivof{\empdistribution}{\probtensor} \, .
\end{align*}
	


% M-Projection -> A projection since P^2 = P, i.e. P applied on the image is id
Most general, the Maximum Likelihood Problem is the M-Projection of a distribution $\gendistribution$ onto a set $\Gamma$ of probability tensors is
\begin{align}
	\argmax_{\probtensor\in\Gamma} \centropyof{\gendistribution}{\probtensor} 
\end{align}
where the Maximum Likelihood Estimation is the special case $\gendistribution=\empdistribution$.


\begin{example}[Cross entropy with respect to exponential families]
	If $\secprobtensor$ from an exponential family, have
		\[ \centropyof{\probtensor}{\expdist} 
		= \sbcontraction{\probtensor,(\expenergy-\cumfunctionof{\canparam}\cdot \ones)}
		=   \sbcontraction{\probtensor,\expenergy} -\cumfunctionof{\canparam} \, .   \]
\end{example}




\subsection{Forward Mapping in Exponential Families} 


%\red{Integrate: 
%Selection encodings suffice for variational methods, relational encodings of statistics are required for markov network instantiations of exponential families.}


%% Mean parameters are expectation queries
Mean parameter coordinates are expectation queries to $\sstatcoordinateof{\selindex}$, by 
	\[ \meanparamat{\indexedselvariable} = \expectationof{\sstatcoordinateof{\selindex}} \, . \]
	
%% Forward mappings are contractions, variational formulation as an alternative to avoid inefficiencies
Forward mappings have a closed form representation by
	\[ \forwardmapof{\canparam}
	= \sbcontractionof{\sencodingof{\sstat},\normationof{\basemeasure,\expof{\contraction{\sencodingof{\sstat},\canparam}}}{\shortcatvariables}}{\selvariable} \, . \]
% Infeasibility and turn to variational alternatives with selection encodings.
This contraction can, however, be infeasible, since it requires the instantiation of the probability tensor, which can be done by basis encodings of the statistic.
We in this section provide alternative characterization of the forward map and approximations of it, which can be computed based on the selection encoding instead.
Following \cite{wainwright_graphical_2008}, we can characterize the forward mapping to exponential families as a variational problem and provide an alternative characterization to this contraction.



\subsubsection{Variational Formulation}

Besides the direct computation of the mean parameter tensor we can give a variational characterization of the forward mapping.
This is especially useful, when the contraction is intractable, for example because the tensor $\expdist$ is infeasible to create.

\begin{theorem}
	We have
	\begin{align*}
		\forwardmapof{\canparam}
		  = \argmax_{\meanparam\in\meanset}  \sbcontraction{\meanparam,\canparam} + \sentropyof{\probtensor^{\meanparam}} 
	\end{align*}
	where 
	\begin{align*}
		 \meanset = \{  \sbcontractionof{\probtensor,\sencodingof{\sstat}}{\selvariable} \, :  \,  \probtensor \,\, \text{a distribution} \}
	\end{align*}
	and by $\probtensor^{\meanparam}$ we denote a probability tensor reproducing the mean parameter $\meanparam$.
\end{theorem}
\begin{proof}
	Theorem3.4 in \cite{wainwright_graphical_2008}.
\end{proof}


% Forward mapping as gradient of A
Further in \cite{wainwright_graphical_2008}: 
Forward mapping coincides with gradient, i.e. $\meanparam = \nabla \cumfunction(\canparam)$.


\subsubsection{Mean Field Method}



We rewrite 
\begin{align*}
	\max_{\meanparam\in\meanset}  \sbcontraction{\meanparam,\canparam} + \sentropyof{\probtensor^{\meanparam}} 
	=
	\max_{\probtensor} \sbcontraction{\energytensor,\probtensor} + \sentropyof{\probtensor}
\end{align*}
where
	\[ \energytensor = \sbcontractionof{\sencsstat,\canparam}{\shortcatvariables} \, . \]

We now restrict the distributions in the maximum.
Typically we use the family of independent distributions, also called naive mean field method.
The naive mean field is the approximation by distributions of independent random variables $\legcoreof{\catenumerator}$, that is
\begin{align*}
	\argmax_{\legcoreof{\catenumerator} \, : \, \catenumeratorin} \contraction{\{\energytensor\} \cup \{\legcoreof{\catenumerator} \, : \, \catenumeratorin\}}
	+ \sum_{\catenumeratorin} \sentropyof{\legcoreof{\catenumerator}} \, . 
\end{align*}


\begin{theorem}[Update equations for the mean field approximation]
	Keeping all legs but one constant, the problem
	\begin{align*}
		\argmax_{\legcoreof{\catenumerator}} \contraction{\{\energytensor\} \cup \{\legcoreof{\catenumerator} \, : \, \catenumeratorin\}}
		+ \sum_{\catenumeratorin} \sentropyof{\legcoreof{\catenumerator}} 
	\end{align*}
	is solved at 
		\[ \legcoreofat{\catenumerator}{\catvariableof{\catenumerator}} 
			= \normationof{ \expof{ \contractionof{ \{\energytensor[\shortcatvariables] \} \cup
				\{\legcoreofat{\seccatenumerator}{\catvariableof{\seccatenumerator}} \, : \, \seccatenumerator\neq\catenumerator\} }{\shortcatvariables} }
			}{\catvariableof{\catenumerator}} \, . \]
\end{theorem}
\begin{proof}
	We have
	\begin{align*}
		 \difofwrt{\sentropyof{\legcoreof{\catenumerator}}}{\legcoreof{\catenumerator}}
		=  - \lnof{\legcoreofat{\catenumerator}{\catvariableof{\catenumerator}}}
		+ \onesat{\catvariableof{\catenumerator}}
	\end{align*}
	and by multilinearity of tensor contractions
	\begin{align*}
		\difofwrt{\contraction{\{\energytensor\}\cup\{\legcoreof{\seccatenumerator} \, : \, \seccatenumeratorin \}}}{\legcoreof{\catenumerator}}
		=  \contractionof{\{\energytensor\}\cup\{\legcoreof{\seccatenumerator} \, : \, \seccatenumeratorin ,\, \seccatenumerator\neq\catenumerator \}}{\catvariableof{\catenumerator}} \, . 
	\end{align*}
	Combining both, the condition
	\begin{align*}
		0 = \difofwrt{
			\left( \contraction{\{\energytensor\}\cup\{\legcoreof{\seccatenumerator} \, : \, \seccatenumeratorin \}} + \sum_{\catenumeratorin} \sentropyof{\legcoreofat{\catenumerator}} \right)
		}{\legcoreof{\catenumerator}}
	\end{align*}
	is equal to
	\begin{align*}
		\lnof{\legcoreofat{\catenumerator}{\catvariableof{\catenumerator}}} =
		 \onesat{\catvariableof{\catenumerator}} + \contractionof{\{\energytensor\}\cup\{\legcoreof{\seccatenumerator} \, : \, \seccatenumeratorin ,\, \seccatenumerator\neq\catenumerator \}}{\catvariableof{\catenumerator}} \, .
	\end{align*}
	Together with the condition $\sbcontractionof{\legcoreof{\catenumerator}}=1$ this is satisfied at
		\[ \legcoreofat{\catenumerator}{\catvariableof{\catenumerator}} 
			= \normationof{ \expof{ \contractionof{ \{\energytensor\} \cup
				\{\legcoreof{\seccatenumerator} \, : \, \seccatenumerator\neq\catenumerator\} }{\catvariableof{\catenumerator}} }
			}{\catvariableof{\catenumerator}} \, . \]
\end{proof}



Algorithm~\ref{alg:NMF} is the alternation of legwise updates until a stopping criterion is met.

\begin{algorithm}[h!]
\caption{Naive Mean Field Approximation}\label{alg:NMF}
\begin{algorithmic}
\For{$\catenumeratorin$}
	\State 
		\[ \legcoreofat{\catenumerator}{\catvariableof{\catenumerator}} 
		= \normationof{\ones}{\catvariableof{\catenumerator}}  \]
\EndFor
\While{Stopping criterion is not met}
	\For{$\catenumeratorin$}
		\State 
			\[ \legcoreofat{\catenumerator}{\catvariableof{\catenumerator}} 
			= \normationof{ \expof{ \contractionof{ \{\energytensor[\shortcatvariables] \} \cup
				\{\legcoreofat{\seccatenumerator}{\catvariableof{\seccatenumerator}} \, : \, \seccatenumerator\neq\catenumerator\} }{\catvariableof{\catenumerator}} }
			}{\catvariableof{\catenumerator}} \]
\EndFor
\EndWhile
\end{algorithmic}
\end{algorithm}


\subsubsection{Structured Variational Approximation}

%% Structured Variational approximation
More generically, we restrict the maximum over the mean parameters of efficiently contractable distributions and get a lower bound.
In this section we use any Markov Network as the approximating family. 

Let $\graph$ be any hypergraph, we define the problem
\begin{align}\tag{$\mathrm{P}_{\mnexpfamily, \probtensor}$}\label{prob:structuredApproximation}
	\argmax_{\probtensor\in \mnexpfamily} \sbcontraction{\energytensor,\probtensor} + \sentropyof{\probtensor}
\end{align}

We approximate the solution of this problem again by an alternating algorithm, which iteratively updates the cores of the approximating Markov Network. 

\begin{theorem}[Update equations for the structured variational approximation]
	The Markov Network with hypercores $\extnetasset$ is a stationary point for Problem~\ref{prob:structuredApproximation}, if for all $\edgein$
	\begin{align*}
	\hypercoreofat{\edge}{\catvariableof{\edge}}
	= \lambda\cdot \expof{
	\frac{
		\contractionof{\{\energytensor\}\cup\{
		\hypercoreof{\secedge} : \secedge\neq\edge
		\}}{\catvariableof{\edge}} 
	}{
		\contractionof{\{
		\hypercoreof{\secedge} : \secedge\neq\edge
		\}}{\catvariableof{\edge}} 
	}
	+ \sum_{\secedge\neq\edge} 
		\frac{
		\contractionof{\{\lnof{\hypercoreof{\secedge}}\}\cup\{
		\hypercoreof{\secedge} : \secedge\neq\edge
		\}}{\catvariableof{\edge}} 
	}{
		\contractionof{\{
		\hypercoreof{\secedge} : \secedge\neq\edge
		\}}{\catvariableof{\edge}} 
	}
	}
	\end{align*}
	for any $\lambda>0$ (e.g. by the norm).
	Here, the quotient denotes the coordinatewise quotient.
\end{theorem}
\begin{proof}
	We proof the theorem by first order condition on the objective $\sbcontraction{\energytensor,\probtensor} + \sentropyof{\probtensor}$.
	
	\red{Need Proposition 11.9 in Koller Book.}
	
	We have for $\probtensor\in\mnexpfamily$
	\begin{align*}
		\sbcontraction{\energytensor,\probtensor} 
		=  \frac{
			\contraction{\{\energytensor\}\cup\{\hypercoreof{\secedge} : \secedge\in\edges\}} 
		}{
			\contraction{\{\hypercoreof{\secedge} : \secedge\in\edges\}} 			
		}
	\end{align*}
	and thus
	\begin{align*}
		\difwrt{\hypercoreof{\edge}} \sbcontraction{\energytensor,\probtensor} 
		= 
	\end{align*}
	
	
	Further we have
	\begin{align*}
		\sentropyof{\normationof{\{\hypercoreof{\secedge} : \secedge\in\edges\}}{\shortcatvariables}}
		= \left(\sum_{\secedge\in\edges} 
			\frac{ 
			\contraction{\{-\lnof{\hypercoreof{\secedge}}\} \cup\{\hypercoreof{\secedge} : \secedge\in\edges\}}
			}{
			\contraction{\{\hypercoreof{\secedge} : \secedge\in\edges\}}
			}\right)
		+ \lnof{\contraction{\{\hypercoreof{\secedge} : \secedge\in\edges\}}}	
	\end{align*}
	and thus
	\begin{align*}
		\difwrt{\hypercoreof{\edge}} \sentropyof{\normationof{\{\hypercoreof{\secedge} : \secedge\in\edges\}}{\shortcatvariables}}
		= 
	\end{align*}
	
	
	
\end{proof}

%% KL Divergence
The mean field method corresponds with minimization of the KL Divergence to the efficiently contractable family, i.e. the I-projection onto the family.

\begin{theorem}
	For any hypergraph $\graph$ and energy tensor $\energytensor$ we have 
	\begin{align*}
		\argmax_{\probtensor\in \mnexpfamily} \sbcontraction{\energytensor, \probtensor}+ \sentropyof{\probtensor}
		= \argmax_{\probtensor\in \mnexpfamily} \kldivof{\expdistof{(\graph,\canparam)}}{\normationof{\expof{\energytensor}}{\shortcatvariables}}
	\end{align*}
	Problem~\ref{prob:structuredApproximation} is thus the I-projection onto the exponential family $\mnexpfamily$.
\end{theorem}
\begin{proof}
%	This follows from the fact, that the objective is the cross-entropy and the position of the maximum is invariant under substracting $\sentropyof{\probtensor}$.
	By rearranging the objective to the KL divergence.
\end{proof}


\subsubsection{Mode Search by annealing}

Finding the mode of a distribution is related to the forward mapping of $\invtemp\cdot\canparam$: $\meanparam$ to a delta distribution (or in the convex hull of multiple maxima) in the limit.

% Annealing effect on the optimization problem
This is because 
\begin{align*}
	\argmax_{\meanparam\in\meanset}  \sbcontraction{\meanparam,\canparam}
\end{align*}
is taken at an extreme point in $\meanset$ (since linear objective over closed convex set), which is a delta distribution of a set and
\begin{align*}
	\argmax_{\meanparam\in\meanset}  \sbcontraction{\meanparam,\invtemp\cdot\canparam}+ \sentropyof{\probtensor^{\meanparam}} 
	= 
	\argmax_{\meanparam\in\meanset}  \sbcontraction{\meanparam,\canparam} + \frac{1}{\invtemp} \cdot \sentropyof{\probtensor^{\meanparam}} 	
\end{align*}
thus the entropy term is neglectible for large $\invtemp$.
A more precise argument is using a limit of the maxima and can be found in Theorem~8.1 in \cite{wainwright_graphical_2008}





\subsection{Backward Mapping in Exponential Families}



%\begin{theorem}[Moment Matching Criteria]\label{the:MM}
	We have that $\canparam$ is a solution of the backward problem at $\genmean$, if and only if 
		\[ \sbcontractionof{\expdist,\sencsstat}{\selvariable} = \genmeanat{\selvariable} \, . \]
%\end{theorem}

This contraction equation is called moment matching, since the moment of the empirical distribution is matched by the moment of the fitting distribution.

We find one backward mapping as the dual problem to the forward mapping.


\subsubsection{Variational Formulation}

The backward mapping to $\datameanat{\selvariable} = \sbcontractionof{\empdistribution,\sencsstat}{\selvariable}$ is Maximum Likelihood estimation and the solution of the maximum entropy problem.

\begin{lemma}
	Let there be a sufficient statistic $\sstat$.
	The map $\backwardmap: \rr^{\seldim}\rightarrow \rr^{\seldim}$ defined as
	\begin{align*}
		\backwardmapof{\meanparam}
		= \argmax_{\canparam\in\rr^{\seldim}}  \sbcontraction{\meanparam,\canparam} - \cumfunctionof{\canparam} \, . 
	\end{align*}
	is a backward mapping.
\end{lemma}
\begin{proof}
	\red{From duality, see Theorem~3.4 in \cite{wainwright_graphical_2008}.}
	This can be shown by the vanishing gradient reproducing the moment matching condition.
\end{proof}

% Gradient property
In \cite{wainwright_graphical_2008}:
 The objective is the conjugate dual $\dualcumfunction$ of $\cumfunction$, and backward mapping has an expression by the gradient, i.e. $\canparam = \nabla \dualcumfunction(\meanparam)$.






\subsubsection{Connection with Maximum Likelihood Estimation}

% Backward mapping
Backward mapping coincides with the Maximum Likelihood Estimation Problem \eqref{prob:parameterMaxLikelihood}, when we take $\Gamma$ to the distributions in an exponential family $\expfamily$ for a sufficient statistic $\sstat$.

% Cross entropy
The loss is the cross entropy between a distribution with $\meanparam$ and the distribution $\expdistof{(\sstat,\canparam,\basemeasure)}$.

\begin{lemma}
	Let $\sstat\in\facspace\otimes\rr^{\seldim}$ be a sufficient statistic and $\gendistribution\in\facspace$ a probability distribution.
	For any member $\expdist\in\expfamily$ we have
		\[ \centropyof{\gendistribution}{\expdist} = \sbcontraction{\canparam,\genmean} - \cumfunctionof{\canparam} \]
	where 
		\[ \genmean = \sbcontractionof{\gendistribution,\sencsstat}{\selvariableof{\sstat}} \,  \]
	and 
		\[ \cumfunctionof{\canparam} = \lnof{\contraction{\expof{\expenergy}}} \, . \]
	The M-projection of $\gendistribution$ onto $\expfamily$ is  $\expdistof{(\sstat,\estcanparam,\basemeasure)}$ for
		\[ \estcanparam\in \argmax_{\canparam}  \sbcontraction{\canparam,\genmean} - \cumfunctionof{\canparam} \, .  \]
\end{lemma}
\begin{proof}
	By decomposing 
	\begin{align*}
		\expdist 	& = \normationof{\expof{\sbcontractionof{\sencsstat,\canparam}{\shortcatvariables}}}{\shortcatvariables} \\
				& = \frac{\expof{\expenergy}}{\sbcontraction{\expof{\expenergy}}}
	\end{align*}
	we get
	\begin{align*}
		\lnof{\expdist} & = \lnof{\expof{\expenergy}} - \onesat{\shortcatvariables} \cdot \sbcontraction{\expof{\expenergy}} \\ 
		& = \expenergy - \cumfunction(\canparam) \cdot \onesat{\shortcatvariables}  \, .
	\end{align*}
	If follows that
	\begin{align*}
		\centropyof{\gendistribution}{\expdist} 
		&=  \sbcontraction{\gendistribution,\lnof{\expdist}} \\
		&=  \sbcontraction{\gendistribution,\expenergy} - \cumfunction(\canparam) \cdot \sbcontraction{\gendistribution}   \\
		&= \sbcontraction{\canparam, \genmean} - \cumfunction(\canparam) \, . 
	\end{align*}
\end{proof}




%\subsection{Maximum Likelihood and Maximum Entropy for Exponential Families}

Parameter Estimation is the M-Projection of a distribution onto the exponential family.

\begin{theorem}[\cite{wainwright_graphical_2008}]\ref{the:parEstToBackwardMap}
	Given any probability distribution $\probof{\shortcatvariables}$ and a exponential family defined by the sufficient statistic $\sstat$, the M-Projection onto the family is the distribution $\probtensorof{(\sstat,\estcanparam,\basemeasure)}$ where
	\begin{align*}
		\estcanparam = \backwardmapof{\contractionof{\probtensor,\sencsstat}{\selvariable}} \, .
	\end{align*}
\end{theorem}
\begin{proof}
	$\contractionof{\probtensor,\sencsstat}{\selvariable}$ is in $\imageof{\forwardmap}$ and MLE has a variational characterization with maximum at the dual $\estcanparam$, see \cite{wainwright_graphical_2008}.
\end{proof}





\subsubsection{Connection with Maximum Entropy}\label{sec:maxEntDuality}


The Maximum entropy problem with respect to matching expected statistics is
\begin{align}\tag{$\mathrm{P}_{\sentropyof{}, \genmean}$}\label{prob:maxEntropy}
	\argmax_{\probtensor} \sentropyof{\probtensor} \quad \text{subject to} \quad 
	 \sbcontractionof{\probtensor,\sencsstat}{\selvariable} =  \genmeanat{\selvariable}
\end{align}
where the optimization is over all probability distributions $\probtensor$.


\begin{theorem}\label{the:maxEntMaxLikeDuality}
	Let $\sstat$ be a map and $\gendistribution$ be any distribution of $\atomstates$ and define
		\[ \genmeanat{\selvariable} = \sbcontractionof{\gendistribution,\sencsstat}{\selvariable} \, .  \]
	Then the solution of \ref{prob:maxEntropy} coincides with the member $\expdistof{(\sstat,\estcanparam,\basemeasure)}$ of the exponential family $\expfamily$ where
		\[ \estcanparam = \backwardmapof{\genmean} \]
	for a backward map $\backwardmap$ of $\expfamily$.
\end{theorem}
\begin{proof}
	Classical result based on duality of maximum entropy and maximum likelihood, shown e.g. in Koller Book.
\end{proof}

%
Theorem~\ref{the:maxEntMaxLikeDuality} states, that when the maximum entropy problem has a solution (i.e. $\genmean\in\meanset$), then the solution is in the exponential family to the statistic $\sstat$.
\red{This is a main motivation of the usage of exponential families as models.}


\subsubsection{Alternating Algorithms to Approximate the Backward Map}\label{sec:alternatingBackwardMap}


\red{While the forward map always has a representation in closed form by contraction of the probability tensor, the backward map in general fails to have a closed form representation.
Computation of the Backward map can instead be performed by alternating algorithms, as we show here.} % Are these fixpoint iterations?


Alternate through the coordinates of the statistics and adjust $\canparamat{\indexedselvariable}$ to a minimum of the likelihood, i.e. where for any $\selindexin$
\begin{align*}
	0 = \frac{\partial}{\partial \canparamat{\indexedselvariable}} \lossof{\expdist} \, . 
\end{align*}

% Moment matching
This condition is equal to the collection of moment matching equations % (see Theorem~\ref{the:mm})
\begin{align*}
	\sbcontractionof{\expdist,\sencsstat}{\indexedselvariable} = \sbcontraction{\empdistribution,\sencsstat}{\indexedselvariable} \, . 
\end{align*}


\begin{lemma}\label{lem:mmContractionEquation}
	For any sufficient statistic $\sstat$ a parameter vector $\canparam$ and a $\selindexin$ we define
	\begin{align*}
	 	\hypercoreat{\catvariableof{\sstatcoordinateof{\selindex}}} 
		= \contractionof{\{\rencodingof{\sstat}\}\cup\{\headcoreof{\tilde{\selindex}} : \tilde{\selindex} \in [\seldim], \tilde{\selindex}\neq\selindex\}}{\catvariableof{\sstatcoordinateof{\selindex}}} \, . 
	\end{align*}
	Then the moment matching condition for $\sstatcoordinateof{\selindex}$ relative to $\canparam$ and $\meanparam$ is satisfied for any $\canparamat{\indexedselvariable}$ with
	\begin{align*}
		\sbcontraction{\headcoreof{\selindex}, \idrestrictedto{\imageof{\sstatcoordinateof{\selindex}}}, \hypercoreat{\selvariable_\sstat}}
		= \sbcontraction{\headcoreof{\selindex}, \hypercoreat{\selvariable_\sstat}} \cdot \meanparamat{\indexedselvariable} \, . 
	\end{align*}
\end{lemma}
\begin{proof}
	We have
	\begin{align*}
		\expdist = \frac{
			\sbcontractionof{\headcoreof{\selindex}, \hypercore}{\shortcatvariables}
		}{
			\sbcontraction{\headcoreof{\selindex}, \hypercore}
		}
	\end{align*}
	and 
	\begin{align*}
		\sbcontraction{\expdist, \sstatcoordinateof{\selindex}}
		= \frac{
			\sbcontractionof{\headcoreof{\selindex}, \idrestrictedto{\imageof{\sstatcoordinateof{\selindex}}}, \hypercore}{\shortcatvariables}
		}{
			\sbcontraction{\headcoreof{\selindex}, \hypercore}
		} \, . 
	\end{align*}
	Here we used
		\[ \sstatcoordinateof{\selindex} = \sbcontractionof{\headcoreof{\selindex}, \idrestrictedto{\imageof{\sstatcoordinateof{\selindex}}}}{\shortcatvariables} \]
	and redundancies of copies of relational encodings.
	It follows that 
	\begin{align*}
		\sbcontraction{\expdist,\sstatcoordinateof{\selindex}} = \contraction{\empdistribution,\sstatcoordinateof{\selindex}}
	\end{align*}
	is equal to
	\begin{align*}
		\sbcontraction{\headcoreof{\selindex}, \idrestrictedto{\imageof{\sstatcoordinateof{\selindex}}}, \hypercoreat{\catvariableof{\sstatcoordinateof{\selindex}}}}
		= \sbcontraction{\headcoreof{\selindex},\hypercoreat{\catvariableof{\sstatcoordinateof{\selindex}}}} \cdot \meanparamat{\indexedselvariable} \, . 
	\end{align*}	
\end{proof}

% Alternation necessary
The steps have to be alternated until sufficient convergence, since matching the moment to $\selindex$ by modifying $\canparamat{\indexedselvariable}$ will in general change other moments, which will have to be refit.


%Coordinate descent
An alternating optimization is the coordinate descent of the negative likelihood, seen as a function of the coordinates of $\canparam$, see Algorithm~\ref{alg:AMM}.
Since the log likelihood is concave, the algorithm converges to a global minimum.





\begin{algorithm}[h!]
\caption{Alternating Moment Matching}\label{alg:AMM}
\begin{algorithmic}
\State Set $\canparamat{\selvariable}=0$
\State Compute $\datameanat{\selvariable}= \sbcontractionof{\empdistribution,\sencsstat}{\selvariable}$
%\For{$\selindexin$}
%	\State Set $\canparamat{\indexedselvariable}=0$ 
%	\State Compute $\meanparamat{\indexedselvariable}^{\datamap} = \contractionof{\{\empdistribution,\sstatcoordinateof{\selindex}\}}{\varnothing} $ % Or give those as input!
%\EndFor
\While{Stopping criterion is not met}
\For{$\selindexin$}
	\State Compute 
		\begin{align*}
			\hypercoreofat{\selindex}{\catvariableof{\sstatcoordinateof{\selindex}}} 
			= \contractionof{\{\rencodingof{\sstat}\}\cup\{\headcoreof{\tilde{\selindex}} : \tilde{\selindex} \in [\seldim], \tilde{\selindex}\neq\selindex\}}{\catvariableof{\sstatcoordinateof{\selindex}}} 
		\end{align*}
	\State Set $\canparamat{\indexedselvariable}$ to a solution of 
	\begin{align*}
		\sbcontraction{\headcoreof{\selindex},\idrestrictedto{\imageof{\sstatcoordinateof{\selindex}}},\hypercoreof{\selindex}}
		= \sbcontraction{\headcoreof{\selindex},\hypercoreof{\selindex}} \cdot \datameanat{\indexedselvariable} \, . 
	\end{align*}
\EndFor
\EndWhile
\end{algorithmic}
\end{algorithm}


% 
In general, if $\imageof{\sstatcoordinateof{\selindex}}$ contains more than two elements, there exists no closed form solutions.
We will investigate the case of binary images, where there are closed form expressions, later in Section~\ref{sec:alternatingParEstMLN}.


%
The computation of $\hypercore_\selindex$ in Algorithm~\ref{alg:AMM} can be intractable and be replaced by an approximative procedure based on message passing schemes.




% Parametrization of Hard Logic 
\section{Propositional Logics}\label{cha:FormulaTensors}

Propositional logics describes systems with $\atomorder$ binary categorical variables, which are called atoms and denoted by $\atomicformulaof{\atomenumerator}$ for $\atomenumeratorin$.
Indices $\atomlegindexof{\atomenumerator}\in[2]$ to the atoms $\atomenumeratorin$ enumerate the $2^\atomorder$ states of these systems, which are called worlds.
In each world indexed by $\atomindices$ the indices $\atomicformulaof{\atomenumerator}$ encode whether the corresponding variable is $\truesymbol$. 

% Structure
We here choose a semantic centric approach to propositional logic, by defining formulas as binary tensors.
Then we investigate the corresponding syntax of formulas as specification of a tensor network decomposition of the relational encoding of formulas.


\subsection{Encoding of Booleans}

Propositional logic amounts to reason about Boolean variables, which are categorical variables with $2$ possible values.
Before applying this insight in the representation of propositional formulas, we first investigate how Boolean calculus can be represented by multilinear operations.

\subsubsection{Booleans as categorical variables}


%\begin{remark}[Boolean Calculus to Binary Calculus] % This is Coordinate Calculus, Happening on the Coordinates during Binary Tensor Contractions
	%We use an embedding of truth assignments to $\{0,1\}=[2]$ and store in large vectors, restructured as tensors, truths to sets of formulas.
	To represent Booleans by categorical variables $\catvariable$ with two states we use the following standard group homomorphism % (this is standard, also build in python)
		\[ \big(\{\truesymbol,\falsesymbol\},\land\big) \quad \text{and} \quad \big(\{ 1,0\},\cdot\big) \]
	by the map
    		\[ [\cdot]:\{\truesymbol,\falsesymbol\} \rightarrow \{1,0\} \]
	defined as
	    	\[ [\truesymbol] = 1 \quad , \quad [\falsesymbol] = 0 \, . \]
	The multiplication is performed in the binary tensor contractions and can thus be interpreted as the $\land$ connective performed on Boolean coordinates.
%\end{remark}


% Expressivity Issues
While the conjunction of is in this embedding performed by multiplications, operations like the negation
	\[ [\lnot X] = 1 - [X]  \]
are affine linear.
Direct applications of these affine linear operations to perform logical calculus will be discussed later in Section~\ref{sec:effectiveGroundingCalculus}.

However, in this chapter we will circumvent the problems arising with affine linearity by using the one-hot encoding of Booleans.


\subsubsection{One-hot Encoding} % This is what Basis Calculus does! Refer to that here?

Booleans are categorical variables with $\catdim=2$ states, where we interpret the states $[2]=\{0,1\}$ by $\{\truesymbol,\falsesymbol\}$.
The one-hot encoding of Booleans
	\[ \onehotmap: [2] \rightarrow \{\fbasis[\catvariable] ,\tbasis[\catvariable] \}  \subset \rr^2 \]
is thus understood as an encoding of truth values.

%% Expressivity
As discussed before, the one-hot encoding is rich enough to represent any function of the state by a linear function on the encoding.
The truth of formulas is a function of the truth of atomic formulas, and thus representable by linear functions of the one-hot encodings.




\subsection{Semantics of Propositional Formulas}

The epistemological commitments are whether the state is $\truesymbol$ or $\falsesymbol$ reflected by the coordinate of the one-hot encoding being $1$ or $0$.
Intuitively this describes, whether a specific world can be the state of a factored system.

\subsubsection{Formulas}

%% Intro of formulas
Logics is especially strong in interpreting binary tensors representing Propositional Knowledge Bases, based on connections with abstract human thinking.
To make this more precise, we associate each such tensor is associated with a formula $\exformula$ being a composition of the atomic variables with logical connectives as we proof next.


\begin{definition}\label{def:formulas}
	A propositional formula $\formulaat{\catvariables}$ depending on $\atomorder$ atoms $\catvariableof{\atomenumerator}$ is a tensor
		\[ \formulaat{\catvariables} : \atomstates \rightarrow [2] \subset \rr \, . \]
	We call $\atomindices \in \atomstates$ a model of a propositional formula $\formula$, if 
		\[ \formulaat{\indexedcatvariables}=1 \].
	If there is a model to a propositional formula, say the formula is satisfiable.
\end{definition}

% Binary Tensors
The propositional formulas coincide therefore with the binary tensors (see Definition~\ref{def:binaryTensor}).


% Decomposition into model sums
Since propositional formulas are binary valued tensors, the generic decomposition of Lemma~\ref{lem:tensorBasisDecomposition} simplifies to
\begin{align}\label{eq:formulaModelDecomposition}
	\formulaat{\catvariables} = \sum_{\catindices\in\atomstates} \formulaat{\indexedcatvariables} \cdot \onehotmapofat{\catindices}{\catvariables} \\
	= \sum_{\catindices\in\atomstates \, : \, \formulaat{\indexedcatvariables}=1}  \onehotmapofat{\catindices}{\catvariables} \, .
\end{align}
Thus, any propositional formula is the sum over the one-hot encodings of its models.
This is equal to the encoding of the set of models, which will be introduced in Chapter~\ref{cha:tensorEncodings} (see Definition~\ref{def:subsetEncoding}).

We depict this decomposition in the diagrammatic notation by
\begin{center}
	\begin{tikzpicture}[scale=0.35, thick]

    \draw (-1,-1) rectangle (5,-3);
    \node[anchor=center] (text) at (2,-2) {\corelabelsize ${\exformula}$};
    \draw[] (0,-3)--(0,-5) node[midway,left] {\colorlabelsize $\catvariableof{0}$};
    \draw[] (1.5,-3)--(1.5,-5) node[midway,left] {\colorlabelsize $\catvariableof{1}$};
    \node[anchor=center] (text) at (3,-4) {$\cdots$};
    \draw[] (4,-3)--(4,-5) node[midway,right] {\colorlabelsize $\catvariableof{\atomorder\shortminus1}$};


    \node[anchor=center] (text) at (7,-2) {${=}$};

    \node[anchor=center] (text) at (12,-2.5) {${\sum\limits_{\atomindices\in\atomstates}}$};
    \node[anchor=center] (text) at (12,-4) {\colorlabelsize $\exformula(\atomindices)=1$};

    \begin{scope}
        [shift={(19.5,1)}]

        \draw (-3,-2) rectangle (-1,-4);
        \node[anchor=center] (text) at (-2,-3) {\corelabelsize $\onehotmapof{\atomlegindexof{0}}$};
        \draw[->-] (-2,-4)--(-2,-6) node[midway,right] {\colorlabelsize $\catvariableof{0}$};

        \node[anchor=center] (text) at (1,-3) {\corelabelsize $\cdots$};

        \draw (3,-2) rectangle (5,-4);
        \node[anchor=center] (text) at (4,-3) {\corelabelsize $\onehotmapof{\atomlegindexof{\atomorder\shortminus1}}$};
        \draw[->-] (4,-4)--(4,-6) node[midway,right] {\colorlabelsize $\catvariableof{\atomorder\shortminus1}$};

    \end{scope}

\end{tikzpicture}
\end{center}




% Maps to multiple formulas -> Later?
%We can extend the map to factored systems of multiple formulas, by using Definition~\ref{def:formulas} as coordinate maps.
%This is exactly what we will study by Bayesian Propositional Networks.
%We will make use of redundancies in the maps to get an efficient representation based on decompositions.





%% Semantic approach
We here chose a semantic approach to propositional logic in contrary to the standard syntactical approach.
Instead of defining formulas by connectives acting on atomic formulas, we define them here as binary valued functions of the states of a factored system.
They are interpreted by marking possible states as models, given the knowledge of $\exformula$.
The syntactical side will then be introduced later by studying decompositions of formulas.


%\begin{figure}[h]
%\begin{center}
%	\begin{tikzpicture}[scale=0.35, thick]

    \draw (-1,-1) rectangle (5,-3);
    \node[anchor=center] (text) at (2,-2) {\corelabelsize ${\exformula}$};
    \draw[] (0,-3)--(0,-5) node[midway,left] {\colorlabelsize $\catvariableof{0}$};
    \draw[] (1.5,-3)--(1.5,-5) node[midway,left] {\colorlabelsize $\catvariableof{1}$};
    \node[anchor=center] (text) at (3,-4) {$\cdots$};
    \draw[] (4,-3)--(4,-5) node[midway,right] {\colorlabelsize $\catvariableof{\atomorder\shortminus1}$};


    \node[anchor=center] (text) at (7,-2) {${=}$};

    \node[anchor=center] (text) at (12,-2.5) {${\sum\limits_{\atomindices\in\atomstates}}$};
    \node[anchor=center] (text) at (12,-4) {\colorlabelsize $\exformula(\atomindices)=1$};

    \begin{scope}
        [shift={(19.5,1)}]

        \draw (-3,-2) rectangle (-1,-4);
        \node[anchor=center] (text) at (-2,-3) {\corelabelsize $\onehotmapof{\atomlegindexof{0}}$};
        \draw[->-] (-2,-4)--(-2,-6) node[midway,right] {\colorlabelsize $\catvariableof{0}$};

        \node[anchor=center] (text) at (1,-3) {\corelabelsize $\cdots$};

        \draw (3,-2) rectangle (5,-4);
        \node[anchor=center] (text) at (4,-3) {\corelabelsize $\onehotmapof{\atomlegindexof{\atomorder\shortminus1}}$};
        \draw[->-] (4,-4)--(4,-6) node[midway,right] {\colorlabelsize $\catvariableof{\atomorder\shortminus1}$};

    \end{scope}

\end{tikzpicture}
%\end{center}
%\caption{Direct interpretation of a propositional formula $\exformula$ as a tensor.
%	The tensor is the sum of the one hot encodings of its models.
%	While the one hot encodings are directed, their sum is not.}
%\label{fig:formulaDirect} 
%\end{figure}

%% Intro of connectives
%Logical connectives are basic building blocks of such formulas and can be understood by simple computations represented in truth tables.
% Here truth tables?
%We call each combination of atomic formulas with connectives a formula.

\subsubsection{Relational encoding of formulas}


%% Direct and Relational interpretation of $\exformula$
There are two ways to represent formulas by tensors.
One way is to understand $[2]$ as subset of $\rr$ and interpreting the formula directly as a tensor (as in Definition~\ref{def:formulas}).
Another way is to understand $[2]$ as the possible values of a categorical variable.
% Maps between factored systems
Following this second perspective, formulas are maps between factored systems, where the image system is the factored systems of atoms and the target system the atomic system defined by a variable $\catvariableof{\formula}$ representing the formula satisfaction.
%Following this perspective, formulas are maps between the factored systems of atoms and the atomic system of the formula.
We can then build the relational encoding (Definition~\ref{def:functionRepresentation}) of that map to represent the formula (see Figure~\ref{fig:formulaRencoding}).

\begin{definition}[Relation Encoding of Formulas] % Own definition, since a reinterpretation of the formula
	Given a factored system with $\atomorder$ atoms $\catvariables$ and a propositional formula $\formula$, we define the relational encoding of $\formula$ (see Definition~\ref{def:functionRepresentation}) 
		\[ \rencodingofat{\formula}{\catvariables,\catvariableof{\formula}} \in  \left(\bigotimes_{\atomenumeratorin} \rr^2\right) \otimes \rr^2 \]
	by 
	\begin{align} 
		\rencodingofat{\formula}{\catvariables,\catvariableof{\formula}} 
		= & \sum_{\atomindices\in\atomstates}  \onehotmapofat{\atomindices}{\catvariables} \otimes \onehotmapofat{\exformula(\atomindices)}{\catvariableof{\formula}} \, . 
	\end{align}
\end{definition}

%% More general relational encodings
We can build relational encodings more generally of any tensors, where we identify the image of the tensor with the states of a categorical variable.
Exactly for propositional formulas, this construction will lead to Boolean image variables.


\begin{lemma}\label{lem:formulaEncodingDecomposition}
	For any formula $\formula$ we have
		\[ \rencodingofat{\formula}{\shortcatvariables,\catvariableof{\formula}} 
		= \formulaat{\shortcatvariables} \otimes \onehotmapofat{1}{\catvariableof{\formula}} 
		+ \lnot\formulaat{\shortcatvariables} \otimes  \onehotmapofat{0}{\catvariableof{\exformula}} \, . 
		 \]
	In particular
		\[ \formulaat{\shortcatvariables} = \contractionof{\{
		\rencodingofat{\formula}{\shortcatvariables,\catvariableof{\formula}} , \onehotmapofat{1}{\catvariableof{\formula}}
		\}}{\shortcatvariables} \, . \]
\end{lemma}
\begin{proof}
%% Decomposition
	We can decompose relational encodings of formulas into the sum (see Figure~\ref{fig:formulaRencoding}) % ! Not a tensor network decomposition !
	\begin{align} 
		\rencodingof{\exformula} = & \fbasis \otimes \left( \sum_{\atomindices\, : \, \exformula(\atomindices) = 0}  \onehotmapof{\atomindices} \right) \\
		 + & \tbasis \otimes \left( \sum_{\atomindices\, : \,  \exformula(\atomindices) = 1}  \onehotmapof{\atomindices} \right)
	\end{align}
	where the second term sums up the models of $\exformula$ and the first one the models of $\lnot\exformula$.
\end{proof}


% Comparison with direct interpretation
Compared with the direct interpretation of a formula as a tensor and the decomposition into models in Equation~\ref{eq:formulaModelDecomposition}, we notice that the relational encoding also represents encoding of worlds where the formula is not satisfied.
This representation is required to represent arbitrary propositional formulas by contracted tensor networks of its components, as will be investigated in the following sections.


%% Coordinatewise 
The relational decomposition $\rencodingof{\exformula}$ has coordinates 
\begin{align}
		\contractionof{\{\rencodingof{\exformula},\onehotmapof{\atomindices}\}}{\catvariableof{\exformula}} 
		= \begin{cases}
		\tbasis & \text{if the world where $\atomicformulaof{\atomenumerator}=\atomlegindexof{\atomenumerator}$ is a model of $\exformula$}  \\
		\fbasis & \text{else}\, .
		\end{cases}
\end{align} 
The contractions of the relational encoding therefore calculate whether an assignment of atoms is a model of the formula, using basis calculus (see Theorem~\ref{the:basisCalculus}).

\begin{figure}[h]
\begin{center}
	\begin{tikzpicture}[scale=0.35, thick] % , baseline = -3.5pt

\draw[->] (2,-1)--(2,1) node[midway,right] {\tiny $\formulavar$};
\draw (-1,-1) rectangle (5,-3);
\node[anchor=center] (text) at (2,-2) {\small $\rencodingof{\exformula}$};
\draw[<-] (0,-3)--(0,-5) node[midway,left] {\tiny $\randomxof{0}$}; 
\draw[<-] (1.5,-3)--(1.5,-5) node[midway,left] {\tiny $\randomxof{1}$}; 
\node[anchor=center] (text) at (3,-4) {$\cdots$};
\draw[<-] (4,-3)--(4,-5) node[midway,right] {\tiny $\randomxof{\atomorder\shortminus1}$}; 


\node[anchor=center] (text) at (7,-2) {${=}$};

\node[anchor=center] (text) at (12,-2.5) {${\sum\limits_{\atomindices\in\atomstates}}$};

\begin{scope}[shift={(19,-0.5)}]

\draw (-2,1) rectangle (4,-1);
\node[anchor=center] (text) at (1,0) {\small $\onehotmapof{\exformula(\atomindices)}$};
\draw[->] (1,1)--(1,3) node[midway,right] {\tiny $\formulavar$}; 

\draw (-3,-2) rectangle (-1,-4);
\node[anchor=center] (text) at (-2,-3) {\small $\onehotmapof{\atomlegindexof{0}}$};
\draw[->] (-2,-4)--(-2,-6) node[midway,right] {\tiny $\catvariableof{0}$}; 

\node[anchor=center] (text) at (1,-3) {\small $\cdots$};

\draw (3,-2) rectangle (5,-4);
\node[anchor=center] (text) at (4,-3) {\small $\onehotmapof{\atomlegindexof{\atomorder\shortminus1}}$};
\draw[->] (4,-4)--(4,-6) node[midway,right] {\tiny $\catvariableof{\atomorder\shortminus1}$}; 

\end{scope}

\end{tikzpicture}
\end{center}
\caption{Relational encoding of a propositional formula. 
The encoding is a sum of the one hot encodings of all states of the factored system in a tensor product with basis vectors, which encode whether the state is a model of the formula.
The tensor is directed, since any contraction with an encoded state results in the basis vector evaluating the formula, which we called basis calculus.
}
\label{fig:formulaRencoding} 
\end{figure}










\subsection{Syntax of Propositional Formulas}

Relational encodings of propositional formulas are especially useful when representing function compositions by the representation of their components (see Theorem~\ref{the:compositionByContraction}). 
In propositional logics, the syntax of defining propositional formulas is oriented on compositions of formulas by connectives. % Quantifications will be studied in the FOL Chapter.
We in this section investigate the decomposition schemes of relational encodings into tensor networks of component encodings for binary tensors following propositional logic syntax.

\subsubsection{Atomic Formulas}

We call atomic formulas the most granular formulas, which are not splitted into compositions of other formulas.
Our syntactic decomposition of propositional formulas will then investigate, how any propositional formula can be represented by these.

\begin{definition}
	The tensors $\formulaofat{\atomenumerator}{\catvariables}$ defined for $\catindices$ as
		\[ \formulaofat{\atomenumerator}{\indexedcatvariables} = \atomlegindexof{\atomenumerator} \]
	are called atomic formulas.
\end{definition}

\begin{theorem}\label{the:AtomicFTensor}
	The relational encoding of any atomic formula $\formulaofat{\atomenumerator}{\catvariables}$ has a tensor decomposition by
		\[ \rencodingofat{\atomicformulaof{\atomenumerator}}{\catvariables,\catvariableof{\formulaof{\atomenumerator}}}
		= \contractionof{\{\identityat{\catvariableof{\atomenumerator},\catvariableof{\formulaof{\atomenumerator}}}\}}{\catvariables,\catvariableof{\formulaof{\atomenumerator}}} \, . \]
	The decomposition is depicted in a network diagram as
	\begin{center}
		\begin{tikzpicture}[scale=0.35,thick] % , baseline = -3.5pt

\drawatomcore{3.5}{-8}{$\bencodingof{\formulaof{\atomenumerator}}$}
\drawatomindices{3.5}{-12}	
\draw[->-] (5.5,-9)--(5.5,-7) node[midway,right] {\colorlabelsize $\headvariableof{\atomenumerator}$};

\node[anchor=center] (text) at (10,-10) {${=}$};

\draw (12,-9) rectangle (15,-11); 
\node[anchor=center] (text) at (13.5,-10) {\corelabelsize $\ones$};
\draw[-<-] (12.5,-11)--(12.5,-13) node[midway,left] {\colorlabelsize $\catvariableof{0}$};
\node[anchor=center] (text) at  (13.5,-12) {$\cdots$};
\draw[-<-] (14.5,-11)--(14.5,-13) node[midway,right] {\colorlabelsize $\catvariableof{\atomenumerator\shortminus1}$};

\node[anchor=center] (text) at (16.25,-10) {\corelabelsize $\otimes$};

\draw[->-] (18.5,-9)--(18.5,-7) node[midway,right] {\colorlabelsize $\headvariableof{\atomenumerator}$};
\draw (17.5,-9) rectangle (19.5,-11);
\node[anchor=center] (text) at (18.5,-10) {\corelabelsize $\delta$};
\draw[-<-]  (18.5,-11)--(18.5,-13) node[midway,right] {\colorlabelsize $\catvariableof{\atomenumerator}$};

\node[anchor=center] (text) at (20.75,-10) {\corelabelsize $\otimes$};

\begin{scope}[shift={(10,0)}]

\draw (12,-9) rectangle (15,-11); 
\node[anchor=center] (text) at (13.5,-10) {\corelabelsize $\ones$};
\draw[-<-]  (12.5,-11)--(12.5,-13) node[midway,left] {\colorlabelsize $\catvariableof{\atomenumerator+1}$};
\node[anchor=center] (text) at  (13.5,-12) {$\cdots$};
\draw[-<-]  (14.5,-11)--(14.5,-13) node[midway,right] {\colorlabelsize $\catvariableof{\atomorder\shortminus1}$};

\node[anchor=center] (text) at  (16.5,-13) {$.$};

\end{scope}

\end{tikzpicture}
	\end{center}
\end{theorem}
\begin{proof}
	We have by definition
	\begin{align*}
		\rencodingofat{\atomicformulaof{\atomenumerator}}{\catvariables,\catvariableof{\formulaof{\atomenumerator}}}
		=& \sum_{\catindices\in\atomstates} \onehotmapofat{\catindices}{\catvariables} \otimes \onehotmapofat{\formulaofat{\atomenumerator}{\indexedcatvariables}}{\catvariableof{\formulaof{\atomenumerator}}} \\
		=& \left( \onehotmapofat{0,0}{\catvariableof{\atomenumerator},\catvariableof{\formulaof{\atomenumerator}}} +
		\onehotmapofat{1,1}{\catvariableof{\atomenumerator},\catvariableof{\formulaof{\atomenumerator}}} \right) \otimes \onesat{\catvariableof{\secatomenumerator}\, : \, \secatomenumerator \neq \atomenumerator} \\
		=& \contractionof{\{\identityat{\catvariableof{\atomenumerator},\catvariableof{\formulaof{\atomenumerator}}}\}}{\catvariables,\catvariableof{\formulaof{\atomenumerator}}} \, .
%		\ftensorof{\exformula}_{1,\atomindices} = \begin{cases}
%		1 & \text{if $\atomlegindexof{\atomenumerator}=1$}  \\
%		0 & \text{else} \, .
	%\end{cases}
	\end{align*} 
%	Since all atom indices $\secatomenumerator\neq\atomenumerator$ are irrelevant, the formula tensor decomposed into factors with the constant vector $\onesof{\secatomenumerator}$.
\end{proof}


%\begin{figure}[h]
%\caption{Representation of an atomic formula tensor $\atomicformulaof{\atomenumerator}$.}
%\label{fig:FormulaChain} 
%\end{figure}

\begin{remark}[Representation of atomic formula tensors for connective action]
 	Need to represent this as $\braket{\delta \otimes \ones, \truevectorat{\atomenumerator}}$, where the bracket indicates contraction along the $\atomenumerator$th axis.
	The core $\truevectorat{\atomenumerator}$ can be replaced by further operations based on logical connectives.
	The axis $\atomenumerator$ is changed from an axis associated with an atom truth to an axis associated with an formula truth.
\end{remark}


\subsubsection{Syntactical combination of formulas}

Formula tensors are elements of tensor spaces with $\atomorder+1$ axis. 
The number of coordinates thus grows exponentially with the number of atoms, which is
	\[ \dim\left[ \rr^2 \otimes \bigotimes_{\atomenumeratorin}\rr^{2} \right] = 2^{\atomorder +1} \, . \]
When the number of atoms in a theory is large, the naive representation of formula tensors will be thus intractable.
In contrast, most logical formulas appearing in practical knowledge bases are sparse in the sense that they have short representations in a logical syntax.
Motivated by this consideration we now discuss propositional syntax and investigate the sparse decomposition of formula tensors along their formula structure to avoid the curse of dimensionality.

%% Propositional Syntax
In logical syntax formulas are described by atomic formulas recursively connected via connectives. 
We show, that representations of logical connectives $\circ \in \{\lnot, \land, \lor, \oplus, \Rightarrow, \Leftrightarrow\}$ can be represented by feasible tensor cores $\concoreof{\circ}$ contracted along a tensor network.


%More general: Theorem~\ref{the:compositionByContraction} shows that any composition of functions can be expressed by contractions of relational encodings.


\begin{example}\label{exa:connectives}
%	Given formulas $\exformula$ and $\secexformula$ we define $\lnot \exformula$ and $\exformula \exconnective \secexformula$ for $\exconnective \in \{\lnot, \land, \lor, \oplus, \Rightarrow, \Leftrightarrow\}$ by their formula tensors
%	\begin{align}
%		\rencodingof{\lnot\exformula} = \contractionof{\{\concoreof{\lnot},\rencodingof{\exformula}\}}{\atomicformulas\cup\{\catvariableof{\lnot\exformula}\}}
%	\end{align}
%	and 
%	\begin{align}
%		\rencodingof{\exformula\exconnective\secexformula} 
%		= \contractionof{\{\concoreof{\exconnective},\rencodingof{\exformula},\rencodingof{\secexformula}\}}{\atomicformulas\cup\{\catvariableof{\exformula\exconnective\secexformula}\}}
%	\end{align}
	We use the following connectives:
	\begin{itemize}
	\item negation $\lnot: [2]\rightarrow [2]$ by the vector
	\begin{align}
		{\lnot}[\catvariableof{\exformula}] = \begin{bmatrix}
		0  \\
		1  
		\end{bmatrix} 
	\end{align}
%	\begin{align}
%		\rencodingof{\lnot} = \begin{bmatrix}
%		0 & 1 \\
%		1 & 0 
%		\end{bmatrix} 
%	\end{align}
%	and by $\rencodingof{\exconnective}$ the order $3$ tensors
	\item conjunctions $\land:  [2]\times[2] \rightarrow[2]$
		\begin{align}
			\land[\catvariableof{\exformula},\catvariableof{\secexformula}]
			 = \begin{bmatrix}
			0 & 0 \\
			0 & 1 
			\end{bmatrix}
		\end{align}
	\item disjunctions $\lor : [2]\times[2] \rightarrow[2]$
		\begin{align}
			\lor[\catvariableof{\exformula},\catvariableof{\secexformula}]
			 = \begin{bmatrix}
			0 & 1 \\
			1 & 1 
			\end{bmatrix}
		\end{align}
	\item exact disjunction $\oplus:  [2]\times[2] \rightarrow[2]$	
		\begin{align}
			\oplus[\catvariableof{\exformula},\catvariableof{\secexformula}]
			 = \begin{bmatrix}
			0 & 1 \\
			1 & 0 
			\end{bmatrix}
		\end{align}
	\item implications $\Rightarrow:  [2]\times[2] \rightarrow[2]$ 
		\begin{align}
			\Rightarrow[\catvariableof{\exformula},\catvariableof{\secexformula}]
			 = \begin{bmatrix}
			1 & 1 \\
			0 & 1 
			\end{bmatrix}
		\end{align}
	\item biimplication $\Leftrightarrow:  [2]\times[2] \rightarrow[2]$ 
		\begin{align}
			\Leftrightarrow[\catvariableof{\exformula},\catvariableof{\secexformula}]
			 = \begin{bmatrix}
			1 & 0 \\
			0 & 1 
			\end{bmatrix}
		\end{align}
	\end{itemize}
%	\item conjunctions $\land$
%		\begin{align}
%			\rencodingof{\land}_{1,:,:} 
%			 = \begin{bmatrix}
%			0 & 0 \\
%			0 & 1 
%			\end{bmatrix}
%			\quad,\quad			
%			\rencodingof{\land}_{0,:,:} 
%			 = \begin{bmatrix}
%			1 & 1 \\
%			1 & 0 
%			\end{bmatrix} \, .
%		\end{align}
%	\item disjunctions $\lor$
%		\begin{align}
%			\rencodingof{\lor}_{1,:,:} 
%			 = \begin{bmatrix}
%			0 & 1 \\
%			1 & 1 
%			\end{bmatrix}
%			\quad, \quad \rencodingof{\lor}_{0,:,:} 
%			 = \begin{bmatrix}
%			1 & 0 \\
%			0 & 0 
%			\end{bmatrix}
%		\end{align}
%	\item exact disjunction $\oplus$	
%		\begin{align}
%			\rencodingof{\oplus}_{1,:,:} 
%			 = \begin{bmatrix}
%			0 & 1 \\
%			1 & 0 
%			\end{bmatrix}
%			\quad, \quad \rencodingof{\oplus}_{0,:,:} 
%			 = \begin{bmatrix}
%			1 & 0 \\
%			0 & 1 
%			\end{bmatrix}
%		\end{align}
%	\item implications $\Rightarrow$ 
%		\begin{align}
%			\rencodingof{\Rightarrow}_{1,:,:} 
%			 = \begin{bmatrix}
%			1 & 1 \\
%			0 & 1 
%			\end{bmatrix}
%			\quad, \quad \rencodingof{\Rightarrow}_{0,:,:} 
%			 = \begin{bmatrix}
%			0 & 0 \\
%			1 & 0 
%			\end{bmatrix}
%		\end{align}
%	\item biconditionals $\Leftrightarrow$ 
%		\begin{align}
%			\rencodingof{\Leftrightarrow}_{1,:,:} 
%			 = \begin{bmatrix}
%			1 & 0 \\
%			0 & 1 
%			\end{bmatrix}
%			\quad, \quad \rencodingof{\Leftrightarrow}_{0,:,:} 
%			 = \begin{bmatrix}
%			0 & 1 \\
%			1 & 0 
%			\end{bmatrix}
%		\end{align}
%	\end{itemize}
\end{example}


\begin{lemma}\label{lem:compositionByContraction}
	Let there be formulas $\exformula$ and $\secexformula$ depending on categorical variables $\shortcatvariables=\catvariables$ and a map 
		\[ \exconnective: [2]\times[2] \rightarrow[2] \, . \]
	Then we have
	\begin{align*}
		\rencodingofat{\exformula\exconnective\secexformula}{\shortcatvariables,\catvariableof{\exformula\exconnective\secexformula}}
		= \contractionof{\{
		\rencodingofat{\exconnective}{\catvariableof{\exformula},\catvariableof{\secexformula},\catvariableof{\exformula\exconnective\secexformula}},
		\rencodingofat{\exformula}{\shortcatvariables,\catvariableof{\exformula}},
		\rencodingofat{\secexformula}{\shortcatvariables,\catvariableof{\secexformula}} 
		\}}{
		\shortcatvariables,\catvariableof{\exformula\exconnective\secexformula}
		}
	\end{align*}
	and for any map $\exconnective: [2] \rightarrow[2]$
	\begin{align*}
		\rencodingofat{\exconnective\exformula}{\shortcatvariables,\catvariableof{\exconnective\exformula}}
		= \contractionof{\{
		\rencodingofat{\exconnective}{\catvariableof{\exformula},\catvariableof{\exconnective\exformula}},
		\rencodingofat{\exformula}{\shortcatvariables,\catvariableof{\exformula}}
		\}}{
		\shortcatvariables,\catvariableof{\exconnective\exformula}
		} \, . 
	\end{align*}
\end{lemma}
\begin{proof}
	This follows from Theorem~\ref{the:compositionByContraction} to be shown in Chapter~\ref{cha:tensorEncodings}.
\end{proof}


\begin{theorem}[Composition of Formulas]\label{the:compositionByContraction}
	Let there be a set of binary variables $\catvariableof{\nodes}$ including atoms $\catvariables$ and image variables to some formulas.
	For any formula $\formulaat{\catvariables}$, which has a syntactical composition into connectives $\{\exconnective_{l}[\catvariableof{\{\nodes_l \}}] : l \in [p]\}$ taking their inputs by variables $\catvariableof{\{\nodes_l \}}\subset \catvariableof{\nodes}$ and output by a variable $\catvariableof{\exconnective_l}$
	we have that
	\begin{align*}
		\rencodingofat{\formula}{\catvariables,\catvariableof{\formula}} =
		\contractionof{\left\{
		\rencodingofat{\exconnective_l}{\catvariableof{\{\nodes_l \}}, \catvariableof{\exconnective_l} : l \in [p] }
		\right\} }
		{\catvariables,\catvariableof{\formula}} \, . 
	\end{align*}
\end{theorem}
\begin{proof}
	When a variable in $\catvariableof{\nodes}$ appears multiple times as input to connectives, we replace it by a set of copies (which wont change the contraction, since all tensors are binary and Theorem~\ref{the:invarianceAddingSubcontractions} can be applied).
	The claim follows then from iterative application of Lemma~\ref{lem:compositionByContraction}.
\end{proof}

\begin{figure}[h]
\begin{center}
	\begin{tikzpicture}[scale=0.35, thick] % , baseline = -3.5pt

\node[anchor=center] (text) at (2,-4) {$a)$};

\draw[->] (5.5,-5)--(5.5,-3) node[midway,right] {\tiny $\headvariableof{\neg\exformula}$};

\node[anchor=center] (text) at (5.5,-6) {$\rencodingof{\lnot}$};
\draw (4.5,-7) rectangle (6.5,-5);

\draw[->] (5.5,-9)--(5.5,-7) node[midway,right] {\tiny $\formulavar$};


\drawatomcore{3.5}{-8}{$\rencodingof{\exformula}$}
\drawatomindices{3.5}{-12}	




\begin{scope}[shift={(15,0)}]

\node[anchor=center] (text) at (2,-4) {$b)$};

\draw[->] (9.5,-5)--(9.5,-3) node[midway,right] {\tiny $\headvariableof{\exformula\circ\secexformula}$};

\node[anchor=center] (text) at (9.5,-6) {$\rencodingof{\circ}$};
\draw (4.5,-7) rectangle (14.5,-5);

\draw[->] (5.5,-9)--(5.5,-7) node[midway,right] {\tiny $\formulavar$};

\drawatomcore{3.5}{-8}{$\rencodingof{\exformula}$}
\drawatomindices{3.5}{-12}	

\begin{scope}[shift={(8,0)}]

	\draw[->] (5.5,-9)--(5.5,-7) node[midway,right] {\tiny $\secexformulavar$};

	\drawatomcore{3.5}{-8}{$\rencodingof{\secexformula}$}
	\drawatomindices{3.5}{-12}	

\end{scope}

\draw[fill] (7.5,-15) circle (0.25cm);
\draw[] (7.5,-15) to[bend left=25] (3.5,-13);
\draw[] (7.5,-15) to[bend right=25] (11.5,-13);

\draw[fill] (9,-15.25) circle (0.25cm);
\draw[] (9,-15.25) to[bend left=25] (5,-13);
\draw[] (9,-15.25) to[bend right=25] (13,-13);

\draw[fill] (11.5,-15) circle (0.25cm);
\draw[] (11.5,-15) to[bend left=25] (7.5,-13);
\draw[] (11.5,-15) to[bend right=25] (15.5,-13);



\draw[] (7.5,-15)--(7.5,-17) node[midway,left] {\tiny $\catvariableof{0}$}; 
\draw[] (9,-15.25)--(9,-17) node[midway,left] {\tiny $\catvariableof{1}$}; 
\node[anchor=center] (text) at (10.5,-16.5) {$\cdots$};
\draw[] (11.5,-15)--(11.5,-17) node[midway,right] {\tiny $\catvariableof{\atomorder-1}$}; 

\end{scope}

\end{tikzpicture} 
\end{center}
\caption{a) Relational encoding of a negated formula $\exformula$ as a tensor network of the encoded formula and the encoded connective $\lnot$.
b) Relational encoding of a composition of formulas $\exformula, \secexformula$ by a connective $\circ\in\{\land,\lor,\oplus,\Rightarrow,\Leftrightarrow\}$. 
The encoding is a contraction of encodings to  $\exformula, \secexformula$ and $\circ$.}
\label{fig:NegatedFormulaTensor} 
\end{figure}

%\begin{remark}[Universality of connectives]
%	The negation $\lnot$ is the only nontrivial unary connective in propositional logics.
%	In combination with $\land$ it already suffices to represent any other $\atomorder$-connective.
%	This fact can be derived from CNF decompositions and De Morgans rules.
%\end{remark}

% To Logical Inference!
%\begin{remark}[Conditional Probability Interpretation]
%	For all these symbol tensors we have the zeroth component defined by
%		\[ \concoreof{\circ}_{0,:,:}  = \ones - \concoreof{\circ}_{1,:,:} \, .\]
%	This again results from the conditional probability distribution interpretation of each $\concoreof{\circ}$.
%\end{remark}

\begin{remark}[$\atomorder$-ary connectives such as $\land$ and $\lor$]\label{rem:naryConnectives}
	Since the decomposition of relational encoding can be applied to generic function compositions (see Theorem~\ref{the:compositionByContraction}), we can also allow for $\atomorder$-ary connectives
		\[ \exconnective : \bigtimes_{\atomenumeratorin} [2] \rightarrow [2] \,  \]
	in Theorem~\ref{the:compositionByContraction}
	The connectives $\land$ and $\lor$ satisfy associativity and have thus straightforward generalizations to the $\atomorder$-ary case.
	This is because associativity can be exploited to represent the relational encoding by any tree-structured composition of binary $\land$ and $\lor$ connectives.
\end{remark}

%% Maps perspective
In general, any $\atomorder$ary logical connective is a map

%In Example~\ref{exa:connectives} we defined unary ($\atomorder=1$) and binary ($\atomorder=2$) connectives.
%Propositional Syntax describes generic formulas $\exformula$ based on the composition of these maps starting with atomic formulas.
%We can thus apply Theorem~\ref{the:compositionByContraction} recursively to decompose the formula tensors $\ftensorof{\exformula}$ into cores $\concoreof{\exconnective}$.



%% Construction from atomic formula tensors
Propositional syntax consists in the application of connectives on atomic formulas, and recursively on the results of such constructions.
When passed towards connective cores, atomic formula tensors act trivial on the legs and just identify the corresponding atomic formula index $\atomlegindexof{\atomicformulaof{\atomenumerator}}$ with $\atomlegindexof{\atomenumerator}$.
This is due to the fact, that the Hadamard product with the trivial tensor $\ones$ leaves any tensor invariant, and the contraction with the elementary matrix $\delta$ identifies indices with each other.
We can thus savely ignore the atomic formula tensors appearing in the decomposition of formula tensors to non-atomic formulas.
An example of such a decomposition is depicted in Figure~\ref{fig:FTDecomposition}.





\begin{figure}[h]
\begin{center}
	\begin{tikzpicture}[scale=0.35, yscale=-1, thick] % , baseline = -3.5pt

    \begin{scope}
        [shift={(-15,0)}]

        \node[anchor=center] (text) at (-3,-6) {${a)}$};

        \node [circle, draw, thick, fill=\nodegrayscale, minimum size = \nodeminsize] (T1) at (0,2) {\colorlabelsize $\catvariableof{a}$};
        \node [circle, draw, thick, fill=\nodegrayscale, minimum size = \nodeminsize] (T2) at (3,2) {\colorlabelsize $\catvariableof{b}$};
        \node [circle, draw, thick, fill=\nodegrayscale, minimum size = \nodeminsize] (T3) at (6,2) {\colorlabelsize $\catvariableof{c}$};

        \node [circle, draw, thick, fill=\nodegrayscale, minimum size = \nodeminsize] (and) at (1.5,-3) {\colorlabelsize $\headvariableof{a\land b}$};
        \coordinate (lowandcenter) at (1.5, -0.5);
        \node [circle, draw, thick, fill=\nodegrayscale, minimum size = \nodeminsize] (not) at (6,-3) {\colorlabelsize $\headvariableof{\lnot c}$};
        \coordinate (lnotcenter) at (6, -0.5);

        \draw [->-] (T1) -- (lowandcenter);
        \draw [->-] (T2) -- (lowandcenter);
        \draw [->-] (lowandcenter) -- (and);

        \draw [->-] (T3) -- (lnotcenter);
        \draw [->-] (lnotcenter) -- (not);

        \node [circle, draw, thick, fill=\nodegrayscale, minimum size = \nodeminsize] (head) at (3.25,-8) {\colorlabelsize $\headvariableof{a\land b \land \lnot c}$};
        \coordinate (highandcenter) at (3.28, -5.5);

        \draw [->-] (and) -- (highandcenter);
        \draw [->-] (not) -- (highandcenter);
        \draw [->-] (highandcenter) -- (head);
    \end{scope}

    \node[anchor=center] (text) at (-3,-6) {${b)}$};

    \draw[->-] (0,1)--(0,-1) node[midway,left] {\colorlabelsize $\catvariableof{a}$};
    \draw[->-] (1.5,1)--(1.5,-1) node[midway,right] {\colorlabelsize $\catvariableof{b}$};
    \draw[->-] (3,1)--(3,-1) node[midway,right] {\colorlabelsize $\catvariableof{c}$};
    \draw (-1,-1) rectangle (4, -3);
    \node[anchor=center] (text) at (1.5,-2) {\corelabelsize $\bencodingof{a \land b \land \lnot c}$};
    \draw[->-] (1.5,-3)--(1.5,-5) node[midway,right] {\colorlabelsize $\headvariableof{a \land b \land \lnot c}$};

    \node[anchor=center] (text) at (5,-2) {${=}$};


    \begin{scope}
        [shift={(7,0)}]

        \draw[->-] (0,1)--(0,-1) node[midway,left] {\colorlabelsize $\catvariableof{a}$};
        \draw[->-] (3,1)--(3,-1) node[midway,right] {\colorlabelsize $\catvariableof{b}$};
        \draw[->-] (6,1)--(6,-1) node[midway,right] {\colorlabelsize $\catvariableof{c}$};

        \draw (-1,-1) rectangle (4, -3);
        \node[anchor=center] (text) at (1.5,-2) {\corelabelsize $\bencodingof{\land}$};

        \draw[->-] (1.5,-3) --(1.5,-5) node[midway,right]{\colorlabelsize $\headvariableof{a \land b}$};

        \draw (5,-1) rectangle (7, -3);
        \node[anchor=center] (text) at (6,-2) {\corelabelsize $\bencodingof{\lnot}$};

        \draw[->-] (6,-3) --(6,-5) node[midway,right]{\colorlabelsize $\headvariableof{\lnot c}$};

        \draw (0.5,-5) rectangle (6.5,-7);
        \node[anchor=center] (text) at (3.5,-6) {\corelabelsize $\bencodingof{\land}$};

        \draw[->-] (4,-7) -- (4,-9) node[midway,right] {\colorlabelsize $\headvariableof{a \land b \land \lnot c}$};

%\draw (3,-9) rectangle (5,-11);
%\node[anchor=center] (text) at (4,-10) {$\truevectorat{}$};

    \end{scope}

\end{tikzpicture}
\end{center}
\caption{Decomposition of the formula tensor to $\exformula = a \land b \land \lnot c$ into unary (matrix) and binary (third order tensor) cores.
	a) Visualization of $\exformula$ as a graph. %(\red{Exploiting the duality between tensor networks on hypergraphs and graphical models \cite{robeva_duality_2019} })
	b) Dual Tensor Network decomposition of $\exformula$.
	We can make use of the invariance of a Hadamard product with a constant tensor $\ones$ and thus not draw axis to atoms not affected by a formula.}
\label{fig:FTDecomposition}
\end{figure}





\begin{remark}[$\htformat$ Interpretation of Formula Tensor Networks]\label{rem:HTDecomFT}
	The sketched decomposition of the formula tensor into a network is a hierarchical tree decomposition of the formula tensor, which we will describe in more detail in Section~\ref{sec:HT}.
	At each decomposition of a formula into subformulas, two subspaces spanned by the respective atomic spaces are selected. 
\end{remark}


\subsubsection{Syntactical decomposition of formulas}\label{sec:termClauseDecomposition}

% Decomposition in case of missing 
We have seen how the decomposition of complex formulas into connectives acting on the component formulas can be exploited to find effective representations of the semantics by tensor networks.
Here the question arises here, how to perform such decompositions in case of a missing syntactical representation of a formula.
By Definition~\ref{def:formulas} any binary tensor is a formula.
We show in the following, how we can find a syntactic specification of a formula given its tensor.

%
%Let us now show that any formula tensor can be decomposed into a network of these connective symbols and atomic formula tensors.


\begin{definition}[Terms and Clauses]\label{def:clauses}
	Given two disjoint subsets $\nodes_0$ and $\nodes_1$ of the $[\atomorder]$, the corresponding term is the formula defined on the indices $\catindices\in\atomstates$ by
		\[ \termof{\nodes_0}{\nodes_1}
		=\left( \bigwedge_{\atomenumerator\in\nodes_0} \lnot\formulaof{\atomenumerator} \right)  \land \left( \bigwedge_{\atomenumerator\in\nodes_1} \formulaof{\atomenumerator} \right)  \]
	and the corresponding clause is the formula defined on the indices $\catindices\in\atomstates$ by
		\[ \clauseof{\nodes_0}{\nodes_1}
		=\left( \bigvee_{\atomenumerator\in\nodes_0} \formulaof{\atomenumerator} \right)  \lor \left( \bigvee_{\atomenumerator\in\nodes_1} \lnot\formulaof{\atomenumerator} \right)  \, , \]
	where by $\land_{\atomenumerator\in\nodes}$ and $\lor_{\atomenumerator\in\nodes}$ we refer to the $n$-ary connectives $\land$ and $\lor$.
	%We call the clause a minterm, if $\nodes_0\cup\nodes_1 = [\atomorder]$.
	We call the term a minterm and the clause a maxterm, if $\nodes_0\cup\nodes_1 = [\atomorder]$.
\end{definition}

%% 
Terms and Clauses have for any index tuple $\catindexof{[\atomorder]}$ a polynomial representation by
		\[ \termof{\nodes_0}{\nodes_1}[\shortcatvariables=\catindexof{[\atomorder]}] 
		= \left( \prod_{\atomenumerator \in \nodes_0} (1-\catindexof{\atomenumerator}) \right)
		\left(  \prod_{\atomenumerator \in \nodes_1} \catindexof{\atomenumerator} \right) \]
and
		\[ \clauseof{\nodes_0}{\nodes_1}[\shortcatvariables=\catindexof{[\atomorder]}] 
		= 1 - \left( \prod_{\atomenumerator \in \nodes_0} (1-\catindexof{\atomenumerator})\right)
		\left(  \prod_{\atomenumerator \in \nodes_1} \catindexof{\atomenumerator} \right) \, . \]


\begin{lemma}\label{lem:termClauseOneHot}
	Terms are contractions of one-hot encodings, that is for any disjoint subsets $\nodes_0,\nodes_1\subset[\atomorder]$ we have
		\[ \termof{\nodes_0}{\nodes_1}[\shortcatvariables] = \contractionof{\onehotmapof{\{\atomlegindexof{k}=0 : k \in \nodes_0 \} \cup \{\atomlegindexof{k}=1 : k \in \nodes_1\}}}{\shortcatvariables} \, . \]
	Clauses are substractions of one-hot encodings from the trivial tensor, that is for any disjoint subsets $\nodes_0,\nodes_1\subset[\atomorder]$ we have
		\[ \clauseof{\nodes_0}{\nodes_1}[\shortcatvariables] = 
		\onesat{\shortcatvariables} -
		\contractionof{\onehotmapof{\{\atomlegindexof{k}=0 : k \in \nodes_0 \} \cup \{\atomlegindexof{k}=1 : k \in \nodes_1\}}}{\shortcatvariables} \, . \]
\end{lemma}


	
%
The reference of the formulas in the case $\nodes_0\dot{\cup}\nodes_1 = [\atomorder]$ as minterms and maxterms is due to the fact, that minterms are formulas with unique models and maxterms are formulas with a unique world not satisfying the formula.
% Enumeration by $\atomstates$
We use this insight and enumerate maxterms and minterms by the index $\catindex\in\atomstates$ of the unique world where the minterm is satisfied, respectively the maxterm is not satisfied.
For any $\nodes_0\dot{\cup}\nodes_1 = [\atomorder]$ we take the index tuple $\catindices$ where $\catindexof{\atomenumerator}=0$ if $\atomenumerator\in\nodes_0$ and $\catindexof{\atomenumerator}=1$ if $\atomenumerator\in\nodes_1$ and define
\begin{align*}
	\maxtermof{\catindices} = \clauseof{\nodes_0}{\nodes_1} \quad \text{and} \quad \mintermof{\catindices} = \termof{\nodes_0}{\nodes_1} \, .
\end{align*}


\begin{corollary}
	Minterms are basis elements of the tensor space, that is for any $\catindices\in\atomstates$ we have
	\begin{align*}
		\mintermof{\catindices} = \onehotmapof{\catindices}
	\end{align*}
	Maxterms are substraction of basis elements from the trivial tensor, that is for any $\catindices\in\atomstates$ we have
	\begin{align*}
		\maxtermof{\catindices} = \onesat{\shortcatvariables} - \onehotmapof{\catindices} \, .
	\end{align*}
\end{corollary}
\begin{proof}
	Follows from Lemma~\ref{lem:termClauseOneHot}, since when $\nodes_0\cup\nodes_1 = [\atomorder]$ the contractions of the one-hot encodings coincides with the one-hot encoding of a fully specified state.
\end{proof}


Based on this insight, we can decompose any propositional formula into a conjunction of maxterms or a disjunction of minterms as we show next.

\begin{theorem}\label{the:tensorToMaxMinTerms}
	For any binary tensor $\hypercoreat{\shortcatvariables}\in\bigotimes_{\atomenumeratorin}\rr^2$ with two-dimensional axes we have
	\begin{align*}
		\hypercoreat{\shortcatvariables} = \left( \bigvee_{\catindices : \hypercoreat{\shortcatvariables=\catindexof{[\atomorder]}}=1} 
		\termof{
		\{\atomenumerator : \catindexof{\atomenumerator}=0\}
		}{
		\{\atomenumerator: \catindexof{\atomenumerator}=1\}
		} 
		\right)
		[\shortcatvariables] 
	\end{align*}
	and
	\begin{align*}
		\hypercoreat{\shortcatvariables} = \left( \bigwedge_{\catindices : \hypercoreat{\shortcatvariables=\catindexof{[\atomorder]}}=0} 
		\clauseof{
		\{\atomenumerator : \catindexof{\atomenumerator}=0\}
		}{
		\{\atomenumerator: \catindexof{\atomenumerator}=1\}
		} 
		\right)
		[\shortcatvariables] \, .
	\end{align*}
\end{theorem}
\begin{proof}
	To show the representation by minterms we use the decomposition
	\begin{align*}
		\hypercoreat{\shortcatvariables}  = \sum_{\catindexof{[\atomorder]} : \hypercoreat{\catvariableof{[\atomorder]}=\catindexof{[\atomorder]}} = 1} \onehotmapofat{\catindexof{[\atomorder]}}{\shortcatvariables}
	\end{align*}
	and notice that each term in the disjunction modifies the formula by adding respective world $\catindexof{[\atomorder]}$ to the models of the formula.
	To show the representation by maxterms we use the decomposition
	\begin{align*}
		\hypercoreat{\shortcatvariables}  = \onesat{\shortcatvariables} \quad - \sum_{\catindexof{[\atomorder]} : \hypercoreat{\catvariableof{[\atomorder]}=\catindexof{[\atomorder]}} = 0} \onehotmapofat{\catindexof{[\atomorder]}}{\shortcatvariables}
	\end{align*}
	and notice that each term in the conjunction modifies the formula by removing the respective world $\catindexof{[\atomorder]}$ from the models of the formula.	
	Thus, both decompositions are propositional formulas with the same set of models as the formula $\hypercore$ and are thus identical to $\hypercore$.
\end{proof}


% Canonical normal forms
The decompositions found in Theorem~\ref{the:tensorToMaxMinTerms} are also called canonical normal forms to propositional formulas $\hypercoreat{\shortcatvariables}$.

%\begin{theorem}\label{the:FormulaToTensor}
%	For any binary tensor $\hypercore\in\bigotimes_{\atomenumeratorin}\rr^2$ with two-dimensional axes we find a formula $\exformula$ in propositional syntax such that
%		\[ \hypercore = \exformula \, . \]
%\end{theorem}
%\begin{proof}
%	For any such $\hypercore$ we construct the formula
%		\[ \exformula^{\hypercore} = \lor_{\atomindices : \hypercore_{\atomindices}=1} \clauseof{\{\atomenumerator : \atomlegindexof{\atomenumerator=0}\}}{\{\atomenumerator : \atomlegindexof{\atomenumerator=1}\}}  \, . \]
%	Since the clauses are basis vectors and $\hypercore$ is binary we get
%		\[ \exformula^{\hypercore} = \sum_{\atomindices : \hypercore_{\atomindices}=1} \onehotmapof{\atomindices} = \hypercore \, .   \] 
%%	Take for any $\ftensor$ the formula
%%		\[ \exformula = \lor_{\atomindices : \ftensor_{:,\atomindices} = \tbasis}
%%		 \left( \land_{\atomenumerator : \atomlegindexof{\atomenumerator} =1} \atomicformulaof{\atomenumerator}\right)
%%		\left( \land_{\atomenumerator : \atomlegindexof{\atomenumerator} =0} \lnot \atomicformulaof{\atomenumerator} \right) \]
%%	to show the claim.
%\end{proof}

%
%In the proof of Theorem~\ref{the:FormulaToTensor} we only applied the connectives $\land,\lor$ and $\lnot$ in the construction of syntactical specifications of formulas. 
%These connectives thus universal in the sense, that their combinations are representing any binary tensor as in Theorem~\ref{the:FormulaToTensor}.
%This observation can be connected with the theory of normal forms, where arbitrary formulas can be syntactically represented by these connectives only (often called logical equivalent, which in our case means identical formula tensor). 




%% Universality of representations
\begin{remark}[Efficient Representation in Propositional Syntax]
	% Relation with binary CP
	The decomposition in Theorem~\ref{the:tensorToMaxMinTerms} is a basis CP decomposition of the binary tensor and will further be investigated in Chapter~\ref{cha:sparseTC}. 
	The formulas constructed in the proof of Theorem~\ref{the:tensorToMaxMinTerms} are however just one possibility to represent a formula tensor in propositional syntax.
	Typically there are much sparser representations for many formula tensors, in the sense that less connectives and atomic symbols are required.
	Having such a sparser syntactical description of a propositional formula can be exploited to find a shorter conjunctive normal form of the formula and construct a sparse polynomial based on similar ideas as in Theorem~\ref{the:tensorToMaxMinTerms}.
	%One way to eliminate syntactical redundancies are through schemes for decompositions called normal forms, for example the Conjunctive Normal Form (CNF) or the Disjunctive Normal Form (DNF).
	We will provide such constructions in Chapter~\ref{cha:sparseTC}, where we show that dropping the demand of directionality and investigating binary CP Decompositions will improve the sparsity of the polynomial formula representation.
\end{remark}


\subsection{Discussion and Outlook}

Further study of representing Knowledge Bases based on Tensor Networks of its formulas in Section~\ref{sec:hardNetworks} (see Theorem~\ref{the:conDecKB}).




%\subsection{Knowledge Bases as Tensor Networks}
%
%%We here aim at a representation of the semantics of a Knowledge Base, whereas traditional systems store the Knowledge Base exploiting the syntax (i.e.storing the known formulas in the propositional syntax).
%% In this formalism a Knowledge Base is represented by its models, i.e. worlds where it is true.
%
%% Representation of multiple formuals
%Let us investigate how we to store a Knowledge Base of formulas $\exformula\in\formulaset$
%\begin{align}
%	\kb = \land_{\exformulain}\exformula \, .
%\end{align}
%
%One obvious way is to use the scheme of Theorem~\ref{the:compositionByContraction} and contract the relational encoded formulas with a conjunction encoding, that is
%\begin{align}
%	\rencodingofat{\kb}{\shortcatvariables,\catvariableof{\kb}}
%	= \contractionof{
%	\{\rencodingofat{\land}{\catvariableof{\formulaset}, \catvariableof{\kb}}\}
%	\cup\{\rencodingofat{\exformula}{\shortcatvariables,\catvariableof{\exformula}}  : \exformula \in \formulaset \}}{\shortcatvariables,\catvariableof{\kb}}
%\end{align}
%Here we denote by $\land$ the $\cardof{\formulaset}$-ary conjunction, which is well-defined by Remark~\ref{rem:naryConnectives}.
%
%% Simpler by effective calculus
%It is however possible to execute the $\cardof{\formulaset}$-ary conjunction by effective calculus (see Section~\ref{sec:effectiveGroundingCalculus}) and we have
%\begin{align}
%	\kb[\shortcatvariables]= \contractionof{\{\formulaat{\shortcatvariables} : \exformulain \}}{\shortcatvariables} \, .
%\end{align}











\section{Logical Inference} \label{sec:tensorKB}

We approach logical inference by defining probability distributions based on propositional formulas and then apply the methodology introduced in the more generic situation of probabilistic inference.
Logical approaches pay here special attention to situations of certainty, where a state of a variable has probability $1$.
In this situation, we say that the corresponding formula is entailed.
%Such situations are called entailment, and we will investigate how we can find these by contractions.


% From Probabilistic 
We start the discussion by showing how formulas can be interpreted by distributions and define logical entailment based on corresponding probabilistic queries.
This enables us to define logical entailment based on the resulting conditional distributions.


%% Where to put?
\begin{remark}[Interpretation of Contractions in Logical Reasoning]
	The coordinates of contracted binary tensor networks describe whether the by the coordinate indexed world is a model of the Knowledge Base at hand.
	Contractions, which only leave a part variables open, store the counts of the world respecting conditions given by the choice of slices. 
	When contracting without open variables, we thus get the total worldcount.
	
	This is consistent with the probabilistic interpretation of contractions, when applying the frequentist interpretation of probability and defining normed worldcounts as probabilities.
\end{remark}


\subsection{Entailment in Propositional Logics}

\begin{definition}[Entailment of propositional formulas]\label{def:logicalEntailment}
	Given two propositional formulas $\kb$ and $\exformula$ we say that $\kb$ entails $\exformula$, denoted by $\kb\models\exformula$, if any model of $\kb$ is also a model of $\exformula$, that is
		\[ \forall_{\shortcatindices\in\atomstates} \big(\kbat{\indexedcatvariableof{[\catorder]}}=1\big) \rightarrow \big(\formulaat{\indexedcatvariableof{[\catorder]}}=1\big) \, . \]
	If $\kb\models\lnot\exformula$ holds, we say that $\kb$ contradicts $\exformula$.
\end{definition}

%
\red{Entailment can be understood by subset relations of the models of formulas.
This perspective can be applied with subset encodings in Chapter~\ref{cha:tensorEncodings}.
}





\subsubsection{Deciding Entailment by contractions}

\begin{theorem}[Contraction Criterion of Entailment]\label{the:contCriterionLogEntailment}
	We habe $\kb\models\exformula$ if and only if 
		\[ \sbcontraction{\kb,\lnot\exformula} = 0 \, . \]
\end{theorem}
\begin{proof}
	% <= 
	If for a $\atomindices$ we have $\kbat{\indexedcatvariableof{[\catorder]}}=1$ but not $\big(\exformula(\atomindices)=1\big)$, the contraction would be at least $1$.
	% =>
	Conversely, if the contraction is at least one, we would find $\atomindices$ with $\kbat{\indexedcatvariableof{[\catorder]}}=1$ and $\lnot\formulaat{\indexedcatvariableof{[\catorder]}}=1$, therefore $\formulaat{\indexedcatvariableof{[\catorder]}}=0$. 
	It follows that $\kb\models\exformula$ does not hold.
\end{proof}

% Can use relational encoding
To decide whether a formula is entailed, or its negation is entailed (in which case one says that the formula is contradicted) by a single contraction, one can perform the contraction
\begin{align*}
	\hypercore = \sbcontractionof{\kbat{\shortcatvariables},\formulaat{\shortcatvariables,\catvariableof{\exformula}}}{\catvariableof{\exformula}}
\end{align*}
and use that
\begin{align*}
	 \sbcontraction{\kb,\lnot\exformula} = \hypercoreat{\catvariableof{\exformula}=0} 
\end{align*}
and 
\begin{align*}
	 \sbcontraction{\kb,\exformula} = \hypercoreat{\catvariableof{\exformula}=1} \, .  
\end{align*}






\subsubsection{Contraction Knowledge Base}

%\red{To do: Knowledge Base as Markov Network}

We now show how to implement a propositional Knowledge Base with the TELL and ASK operations based on Theorem~\ref{cor:parallelCriterion}.

% Works also for Markov Networks!
\begin{algorithm}[hbt!]
\caption{Contraction Knowledge Base}\label{alg:TensorKB}
ASK(formula $\exformula$)
\begin{algorithmic}
	\State{$\hypercoreat{\formulavar} \leftarrow \sbcontractionof{\kb,\rencodingof{\exformula}}{\formulavar}$}
	\If{$\hypercoreat{\formulavar=0}=0$} 
		\State{return Entailed}
	\EndIf
	\If{$\hypercoreat{\formulavar=1}=0$} 
		\State{return Contradicted}
	\EndIf
	\State{return Contingent}
\end{algorithmic}
TELL(formula $\exformula$)
\begin{algorithmic}
	\If{ASK($\exformula$) returns Contingent:}
%	\If{$\sbcontractionof{\kb,\exformula}>0$ and $\sbcontraction{\kb,\lnot\exformula}>0$} %Consistency + Redundancy check
	\State $\kb \leftarrow \kb\land\exformula$%Add cores of formula tensor $\exformula$ to $\ftensorof{\kb}$ in order to represent $\ftensorof{\kb\cup\{\exformula\}}$
	\EndIf
\end{algorithmic}

\end{algorithm}

\red{
Comment: TELL checks whether the formula to be added is entailed, in which case it is redundant to add, and whether the formula to be added is contradicted, in which case the knowledge base would become unsatisfiable.
}



\subsubsection{Sparse Representation of a Knowledge Base}

Let us now investigate how to sparsely represent a Knowledge Base.
Towards getting insights on this we first show that entailed formulas can be dropped from the Knowledge Base.

\begin{theorem}\label{the:ReduncancyOfEntailed}
	If and only if $\kb \models \exformula$ we have
		\[ \kbat{\shortcatvariables}= \sbcontractionof{\kb,\exformula}{\shortcatvariables}  \, . \]
\end{theorem}
\begin{proof}
	For any world indexed by a coordinate $\atomindices$, $\kbat{\indexedcatvariableof{[\catorder]}}$ indicates whether the world is a model of $\kb$.
	We have entailment, when the models of $\kb\cup\exformula$ coincide with those of $\kb$.
\end{proof}


\begin{remark}[Sparsest Description of a Knowledge Base]
	Given a set of worlds indexed by $\hypercore$, find the sparsest set of formulas $\kb$ such that
		\[ \hypercore = {\kb} \]
	would be benefitial for small computational complexity.
	Since the formula tensors are invariant under entailment, we can drop entailed formulas.
\end{remark}	





\subsection{Formulas as Random Variables}

\red{Aim here: Relate with the probabilistic reasoning concepts of marginal and conditional distributions.}

\red{Given a probability distribution $\probtensor$ of atoms we add a variable by building the Markov Network of $\probtensor$ and $\rencodingof{\exformula}$ to get a joint distribution of the atoms and a query formula $\exformula$}

There are two ways of interpreting formula tensors as conditional probabilities.
The standard one, which we also used above, understands the atomic legs as conditions and calculates the truth of the formula.
Another understands a formula as a condition.

\subsubsection{Conditioning on the atoms}

%% Conditional interpretation -> Formulas as conditional probability ("local")
Our main interpretation understands each tuple of indices $\atomindices$ as conditions of a probability tensor.
Given a truth assignment to the atomic variables $\atomicformulaof{\atomenumerator}$, that is a choice of indices $\atomlegindexof{\atomenumerator}$, determines the truth of the formula.
We thus interpret the formula tensors as defining a conditional probability of $\exformula$ given the atoms $\atomicformulaof{\atomenumerator}$ indexed by $\atomlegindexof{\atomenumerator}$.

\begin{theorem}\label{the:conditionByAtoms}
	The relational encoding of any propositional formula $\exformula$ coincides with the conditional probability of that formula conditioned on the identity on the atoms, that is
		\[ \rencodingof{\exformula} = \condprobof{\formulavar}{\atomicformulas} \, . \]
	We depict this by
	\begin{center}
		\begin{tikzpicture}[scale=0.35,thick] % , baseline = -3.5pt



\draw[->] (2,-1)--(2,1) node[midway,right] {\tiny $\formulavar$};

\draw (-3,-1) rectangle (7,-3);
\node[anchor=center] (text) at (2,-2) {\small $\condprobof{\formulavar}{\atomicformulas}$};
\draw[<-] (0,-3)--(0,-5) node[midway,left] {\tiny $\catvariableof{0}$}; 
\draw[<-] (1.5,-3)--(1.5,-5) node[midway,left] {\tiny $\catvariableof{1}$}; 
\node[anchor=center] (text) at (3,-4) {$\cdots$};
\draw[<-] (4,-3)--(4,-5) node[midway,right] {\tiny $\catvariableof{\atomorder\shortminus1}$}; 


\node[anchor=center] (text) at (9,-2) {${=}$};


\begin{scope}[shift={(12,0)}]

\draw[->] (2,-1)--(2,1) node[midway,right] {\tiny $\formulavar$};
\draw (-1,-1) rectangle (5,-3);
\node[anchor=center] (text) at (2,-2) {\small $\rencodingof{\exformula}$};
\draw[<-] (0,-3)--(0,-5) node[midway,left] {\tiny $\catvariableof{0}$}; 
\draw[<-] (1.5,-3)--(1.5,-5) node[midway,left] {\tiny $\catvariableof{1}$}; 
\node[anchor=center] (text) at (3,-4) {$\cdots$};
\draw[<-] (4,-3)--(4,-5) node[midway,right] {\tiny $\catvariableof{\atomorder\shortminus1}$}; 

\node[anchor=center] (text) at (7,-5) {${\cdot}$};

\end{scope}


\end{tikzpicture}
	\end{center}
\end{theorem}
\begin{proof}
	The distribution $\probtensor$ does not influence the conditional query, since the normation acts on any state.
\end{proof}


% Interpretation of directionality as 
The conditional query $\condprobof{\formulavar}{\enumeratedatoms}$ provides an interpretation of $\rencodingof{\exformula}$ as a conditional probability. 
This is also reflected in the fact that both $\condprobof{\exformula}{\shortcatvariables}$ and $\rencodingof{\exformula}$ are directed, since the first is a normation by Defintion~\ref{def:queries} and the second a encoding of a function.

%The direction of the legs in the formula tensor diagram in Figure~\ref{fig:FormulaTensor} is chosen to highlight the conditional probability interpretation.


This directly implies using Theorem~\ref{the:conditionalMarginalization}  the trivialization of the formula tensor when contracting its head axis indexed by $\atomlegindexof{\exformula}$ with the trivial vector $\ones$, depicted as
\begin{center}
	\begin{tikzpicture}[scale=0.35,thick] % , baseline = -3.5pt



\draw (1,1) rectangle (3,3);
\node[anchor=center] (text) at (2,2) {\small $\ones$};
\draw[->] (2,-1)--(2,1) node[midway,right] {\tiny $\catvariableof{\exformula}$};

\draw (-1,-1) rectangle (5,-3);
\node[anchor=center] (text) at (2,-2) {\small $\ftensorof{\exformula}$};
\draw[<-] (0,-3)--(0,-5) node[midway,left] {\tiny $\catvariableof{0}$}; 
\draw[<-] (1.5,-3)--(1.5,-5) node[midway,left] {\tiny $\catvariableof{1}$}; 
\node[anchor=center] (text) at (3,-4) {$\cdots$};
\draw[<-] (4,-3)--(4,-5) node[midway,right] {\tiny $\catvariableof{\atomorder-1}$}; 


\node[anchor=center] (text) at (7,-2) {${=}$};

\begin{scope}[shift={(10,0)}]

\draw (1,1) rectangle (3,3);
\node[anchor=center] (text) at (2,2) {\small $\ones$};

\draw[->] (2,-1)--(2,1) node[midway,right] {\tiny $\catvariableof{\exformula}$};
\draw (-1,-1) rectangle (5,-3);
\node[anchor=center] (text) at (2,-2) {\small $\condprobof{\exformula}{\{\atomicformulaof{\atomenumerator}\}}$};
\draw[<-] (0,-3)--(0,-5) node[midway,left] {\tiny $\catvariableof{0}$}; 
\draw[<-] (1.5,-3)--(1.5,-5) node[midway,left] {\tiny $\catvariableof{1}$}; 
\node[anchor=center] (text) at (3,-4) {$\cdots$};
\draw[<-] (4,-3)--(4,-5) node[midway,right] {\tiny $\catvariableof{\atomorder\shortminus1}$}; 

\end{scope}

\node[anchor=center] (text) at (17,-2) {${=}$};

\begin{scope}[shift={(20,0)}]

\draw (-1,-1) rectangle (5,-3);
\node[anchor=center] (text) at (2,-2) {\small $\ones$};
\draw[<-] (0,-3)--(0,-5) node[midway,left] {\tiny $\catvariableof{0}$}; 
\draw[<-] (1.5,-3)--(1.5,-5) node[midway,left] {\tiny $\catvariableof{1}$}; 
\node[anchor=center] (text) at (3,-4) {$\cdots$};
\draw[<-] (4,-3)--(4,-5) node[midway,right] {\tiny $\catvariableof{\atomorder\shortminus1}$}; 

\node[anchor=center] (text) at (6,-5) {$.$};

\end{scope}


\end{tikzpicture}
\end{center}



\subsubsection{Conditioning on the formula}

% Defining probability distribution by formulas
Let us now converse the order of conditioning from $\condprobof{\exformula}{\atomicformulas}$ to $\condprobof{\atomicformulas}{\exformula}$.
In this way, we have propositonal formulas defining probability distributions on the factored system of atoms.

Given a Markov Network $\probtensor$ with a single core $\rencodingof{\exformula}$ for a propositional formula $\exformula$.
By definition we have
\begin{align*}
	\condprobof{\enumeratedatoms}{\formulavar} 
	= \sbnormationofwrt{\rencodingof{\exformula}}{\enumeratedatoms}{\formulavar} \, .  
\end{align*}
\begin{center}
	\begin{tikzpicture}[scale=0.35,thick] % , baseline = -3.5pt


\begin{scope}[shift={(-2,0)}]

%\draw[dashed] (1,1) rectangle (3,3);
%\node[anchor=center] (text) at (2,2) {\small $\onehotmapof{\atomlegindexof{\exformula}}$};

\draw[<-] (2,-1)--(2,1) node[midway,right] {\tiny $\catvariableof{\exformula}$};

\draw (-1,-1) rectangle (5,-3);
\node[anchor=center] (text) at (2,-2) {\small $\condprobof{\atomicformulas}{\catvariableof{\exformula}}$};
\draw[->] (0,-3)--(0,-5) node[midway,left] {\tiny $\catvariableof{0}$}; 
\draw[->] (1.5,-3)--(1.5,-5) node[midway,left] {\tiny $\catvariableof{1}$}; 
\node[anchor=center] (text) at (3,-4) {$\cdots$};
\draw[->] (4,-3)--(4,-5) node[midway,right] {\tiny $\catvariableof{\atomorder\shortminus1}$}; 


\node[anchor=center] (text) at (7,-2) {${=}$};

\end{scope}


\node[anchor=center] (text) at (8,-2.25) {$\sum\limits_{\atomlegindexof{\exformula}\in[2]}$};

\draw[] (10,-1) rectangle (12,-3);
\node[anchor=center] (text) at (11,-2) {\small $\onehotmapof{\atomlegindexof{\exformula}}$};

\draw[->] (11,-3)--(11,-5) node[midway,right] {\tiny $\catvariableof{\exformula}$};



\begin{scope}[shift={(15,0)}]

\draw[] (1,1) rectangle (3,3);
\node[anchor=center] (text) at (2,2) {\small $\onehotmapof{\atomlegindexof{\exformula}}$};

\draw[->] (2,-1)--(2,1) node[midway,right] {\tiny $\catvariableof{\exformula}$};
\draw (-1,-1) rectangle (5,-3);
\node[anchor=center] (text) at (2,-2) {\small $\ftensorof{\exformula}$};
\draw[<-] (0,-3)--(0,-5) node[midway,left] {\tiny $\catvariableof{0}$}; 
\draw[<-] (1.5,-3)--(1.5,-5) node[midway,left] {\tiny $\catvariableof{1}$}; 
\node[anchor=center] (text) at (3,-4) {$\cdots$};
\draw[<-] (4,-3)--(4,-5) node[midway,right] {\tiny $\catvariableof{\atomorder\shortminus1}$}; 


\end{scope}


\begin{scope}[shift={(25,0)}]

\draw (-5,-7) -- (0,3);
	
	\draw[] (1,1) rectangle (3,3);
	\node[anchor=center] (text) at (2,2) {\small $\onehotmapof{\atomlegindexof{\exformula}}$};

	\draw[->] (2,-1)--(2,1) node[midway,right] {\tiny $\catvariableof{\exformula}$};
\draw (-1,-1) rectangle (5,-3);
\node[anchor=center] (text) at (2,-2) {\small $\ftensorof{\exformula}$};
\draw[<-] (0,-3)--(0,-5) node[midway,left] {\tiny $\catvariableof{0}$}; 
\draw[<-] (1.5,-3)--(1.5,-5) node[midway,left] {\tiny $\catvariableof{1}$}; 
\node[anchor=center] (text) at (3,-4) {$\cdots$};
\draw[<-] (4,-3)--(4,-5) node[midway,right] {\tiny $\catvariableof{\atomorder\shortminus1}$}; 

\draw (-1,-5) rectangle (5,-7);
\node[anchor=center] (text) at (2,-6) {\small $\ones$};

\end{scope}

\node[anchor=center] (text) at (32,-5) {$.$};

\end{tikzpicture}
\end{center}

% Conditioning on the formula being true
Let us further investigate the slices of $\condprobof{\shortcatvariables}{\exformula}$ with respect to $\exformula$, which define distributions of the states of the factored system.
To this end, let us condition on the event of $\exformula=1$, for which we have the distribution
\begin{align}\label{eq:eventFormulaProb}
	\condprobof{\shortcatvariables}{\exformula=1} = \frac{1}{\sbcontraction{\exformula}} \sum_{\atomindices \in \atomstates \, : \, \formulaat{\indexedcatvariableof{[\catorder]}}=1} \onehotmapof{\atomindices} \, .
\end{align}
With $\sbcontraction{\exformula}$ being the number of models of $\exformula$,  this is the uniform distribution among the models of $\exformula$.
Conversely, when conditioning on the event $\exformula=0$ we get a uniform distribution of the models of $\lnot\exformula$.

% 
The probability distribution in Equation~\eqref{eq:eventFormulaProb} is well defined except for the case that $\sbcontraction{\exformula}=0$.
In this case we have $\exformula=0$ and call $\exformula$ unsatisfiable, since it has no models.

%The probability tensor is well-defined except for the case that $\ftensor_{1,:}$ contains just $0$ coordinates (respectively for $\ftensor_{0,:}$).
%This is an exceptionous situation in logics and called unsatisfiability of the knowledge base.
%\begin{definition}
%	A propositional formula $\exformula$ with $\sbcontraction{\exformula}=0$ is called unsatisfiable.
%\end{definition}
%If the Knowledge Base is inconsistent, the probabilistic interpretation breaks down.
%Thus we will always assume a consistent Knowledge Base when doing probabilistic reasoning.
%An alternative interpretation of formula tensors is the conditional probability of the atomic formulas given the formula at hand.
%To derive the conditional probability tensor we apply the Bayes Theorem
%\begin{align}
%	\condprobof{\{\atomicformulaof{\atomenumerator} = \atomlegindexof{\atomenumerator}\}}{\exformula=\atomlegindexof{\exformula}} 
%	=\frac{
%	\condprobof{\exformula}{\{\atomicformulaof{\atomenumerator} = \atomlegindexof{\atomenumerator}\}}
%	}
%	{
%	\sum_{\atomindices}\condprobof{\exformula}{\{\atomicformulaof{\atomenumerator} = \atomlegindexof{\atomenumerator}\}} \, .
%	}
%\end{align}



%% Uniform interpretation -> KB as probability distribution over its models ("global")
From an epistemological point of view, probability theory is a generalization of logics, since we allow for probability values in the interval $[0,1]$.
The set of distributions being constructed by conditioning on propositional formulas as in Equation~\eqref{eq:eventFormulaProb} correspond within the set of probability distributions with those having constant coordinates on their support.
% More specific
While the probability tensors with nonvanishing coordinates build a $2^\atomorder-1$-dimensional manifold, where the formulas parametrize $2^{2^\atomorder}$ probability tensors, most of which having vanishing coordinates.




\subsubsection{Probability of a function given a Knowledge Base}

% Both directions for entailment
We can now combine the ideas of the previous two subsections and define probabilities of formulas $\exformula$ given the satisfaction of another formula $\kb$, which we call a Knowledge Base.
We have by Theorem~\ref{the:conditionByAtoms} % Again, Markov Network with rencoding of \exformula, \kb build the precise \probtensor
\begin{align*}
	\condprobof{\formulavar}{\kbvar} 
	& = \sbcontractionof{
	\condprobof{\formulavar}{\atomicformulas}, \condprobof{\atomicformulas}{\kbvar}
	}{\formulavar,\kbvar} \\
	& = \sbnormationofwrt{\rencodingof{\exformula} , \rencodingof{\kb}}{\formulavar}{\kbvar}
\end{align*}

% 
Of special interest is the marginal probability of $\formulavar$ given that $\kbvar$ is satisfied, that is
\begin{align*}
	\condprobof{\formulavar}{\kbvar=1} 
	& = \normationof{\{\rencodingof{\exformula} ,\kb\}}{\formulavar}\\
	& = \frac{\contractionof{\{\rencodingof{\exformula},\kb\}}{\formulavar}}{\contraction{\{\kb\}}} \, . 
\end{align*}


% Knowledge Base as Probability
\begin{remark}[Case of Unsatisfiable Knowledge Bases]
	When the Knowledge Base is not satisfiable, one cannot normate it and the probability distribution is not dedfined.
%	We notice, that by the criterion provided by Theorem~\ref{cor:parallelCriterion} we can decide entailment also in the cases where $\kb$ is unsatisfiable.
%	In that case the contraction is the zero tensor, which is parallel to $\tbasis$ and $\fbasis$ and thus entailed and contradicting at the same time.
\end{remark}

%We will now define entailment based on this quantity. 
%\begin{definition}[Entailment]
%	We say that a not unsatisfiable Knowledge Base $\kb$ entails a formula $\exformula$, denoted by $\kb\models\exformula$, if $\condprobof{\formulavar=1}{\kbvar=1}=1$. 
%	If $\kb$ entails $\lnot\exformula$ we say that $\kb$ contradicts $\exformula$.
%	If the Knowledge Base $\kb$ is unsatisfiable, it entails any formula.
%	%
%	More generally, we say that a probability distribution $\probtensor$ entails a formula $\exformula$ if $\probof{\exformula=1}=1$.
%\end{definition}



% 
\begin{theorem}\label{the:probEntailment}
	Given a satisfiable formula $\kb$, we have $\kb\models\exformula$, if and only if 
		\[ \condprobof{\formulavar=0}{\kbvar=1} = 0 \, .  \]
\end{theorem}
\begin{proof}
	Since $\kb$ is satisfiable, we have $\sbcontraction{\kb}>0$ and
		\[ \condprobof{\formulavar=0}{\kbvar=1} = \frac{\sbcontraction{\lnot\exformula, \kb}}{\sbcontraction{\kb}} \, .  \]
	This term vanishes if and only if $\sbcontraction{\lnot\exformula, \kb}$ vanish.
	Thus, the condition is equivalent to the condition in Theorem~\ref{the:contCriterionLogEntailment}.
\end{proof}

Given that $\kb$ is satisfiable, we therefore have $\kb\models\exformula$ if and only if
\begin{align}
	\condprobof{\formulavar}{\kbvar=1} = %\begin{cases}
	\tbasis \, .  %& \text{if }\kb \models \lnot\exformula \\
	%\tbasis & \text{if }\kb \models \exformula \\
	%\notin \{\fbasis,\tbasis\} & \text{else}
	%\end{cases} \, .
\end{align}
We depict this condition by the contraction diagram
%It suffices to check, whether the contraction with the normed Knowledge Base is the basis vector $\tbasis$, respectively $\fbasis$, that is
\begin{center}
	\begin{tikzpicture}[scale=0.35,thick] % , baseline = -3.5pt




\draw[->] (2,-1)--(2,1) node[midway,right] {\tiny $\catvariableof{\exformula}$};
\draw (-1,-1) rectangle (5,-3);
\node[anchor=center] (text) at (2,-2) {\small $\ftensorof{\exformula}$};
\draw[<-] (0,-3)--(0,-5) node[midway,left] {\tiny $\catvariableof{0}$}; 
\draw[<-] (1.5,-3)--(1.5,-5) node[midway,left] {\tiny $\catvariableof{1}$}; 
\node[anchor=center] (text) at (3,-4) {$\cdots$};
\draw[<-] (4,-3)--(4,-5) node[midway,right] {\tiny $\catvariableof{\atomorder\shortminus1}$}; 

\draw (-1.5,-5) rectangle (5.5,-7);
\node[anchor=center] (text) at (2,-6) {\small $\sbnormationof{\kb}{\shortcatvariables}$};

\node[anchor=center] (text) at (9,-4) {\small ${=}$};
%\draw (9,-3) -- (9,-5);
%\draw (9.15,-3) -- (9.15,-5);

\draw[->] (13,-3)--(13,-1) node[midway,right] {\tiny $\catvariableof{\exformula}$};
\draw (12,-5) rectangle (14,-3);
\node[anchor=center] (text) at (13,-4) {\small $\onehotmapof{1}$};

\node[anchor=center] (text) at (16,-6) {$\cdot$};

\end{tikzpicture}
\end{center}


We can omit the normation by $\sbcontraction{\kb}$ when deciding entailment, as we state next.

\begin{corollary}\label{cor:parallelCriterion}
	Given a satisfiable formula $\kb$, we have $\kb\models\exformula$ (respectively $\kb\models\lnot\exformula$), if and only if 
		\[ \sbcontractionof{\kb,\rencodingof{\exformula}}{\formulavar=0} = 0 
		 \quad \text{( respectively }
		 \sbcontractionof{\kb,\rencodingof{\exformula}}{\formulavar=1} = 0 \, . \]
\end{corollary}




%We will draw on this interpretation in the following, when investigating contraction equation equivalent to entailment.


%
Relating entailment to probability distributions motivates an extension of Definition\ref{def:logicalEntailment} of entailment to arbitrary probability distributions.


\begin{definition}\label{def:probEntailment}
	For any propositional formula $\exformula$ and a probability distribution $\probtensor$ we say that $\probtensor$ probabilistically entails $\exformula$, denoted as $\probtensor\models\exformula$, if
		\[ \sbcontractionof{\probtensor,\rencodingof{\exformula}}{\formulavar=0} = 0 . \]
	If $\probtensor\models\lnot\exformula$ we say that $\probtensor$ probabilistically contradicts $\exformula$.
%	Conversely, we 
%		\[ \sbcontractionof{\probtensor,\rencodingof{\exformula}}{\formulavar=1} = 0 . \]
\end{definition}

%
By Theorem~\ref{the:probEntailment} the definition of entailment reduces to propositional formulas by choosing $\probtensor=\sbnormationof{\kb}{\shortcatvariables}$









\subsubsection{Deciding entailment on Markov Networks}



% Generic Probability Tensors
Let us now relate the probabilistic entailment definition \ref{def:probEntailment} with the logical entailment.
Given a generic probability distribution $\probtensor$ we can build a Knowledge Base by the indicator function of the support as 
	\[ \kb^{\probtensor} = \nonzerofunction \circ \probtensor \]
where $\nonzerofunction:\rr\rightarrow \rr$ is defined as $\nonzeroof{x}=1$ if $x\neq0$ and $\nonzeroof{x}=0$ else.

% Generic case of distributions
\begin{theorem}\label{the:entailmentMarkovToKB}
	Any probability distribution $\probtensor$ probabilistically entails a formula $\exformula$, if and only if the Knowledge Base $\kb^{\probtensor}$ logically entails $\exformula$.
\end{theorem}
\begin{proof}
	Whenever $\probtensor$ does not entail $\exformula$ probabilistically we find a state $\shortcatindices\in\atomstates$ such that
		\[ \probof{\shortcatvariables=\shortcatindices} >0 \quad\text{and} \quad \formulaat{\shortcatvariables=\shortcatindices} = 0 \, . \]
	We further have $\probof{\shortcatvariables=\shortcatindices} >0$ if and only if $\kb^{\probtensor}[\shortcatvariables=\shortcatindices]=1$ and
		\[ \big((\kb^{\probtensor}[\indexedcatvariableof{[\catorder]}]=1\big) \rightarrow \big(\formulaat{\indexedcatvariableof{[\catorder]}}=1\big) \, . \]
	is not satisfied.
	Together, $\probtensor\models\exformula$ does not holds if and only if
		\[ \forall \shortcatvariables (\kb^{\probtensor}[\shortcatvariables=\shortcatindices]=1) \rightarrow \big(\formulaat{\indexedcatvariableof{[\catorder]}}=1\big) \,  \]
	is not satisfied. 
	Therefore, probabilistic entailment of $\exformula$ by $\probtensor$ is equivalent to logical entailment of $\exformula$ by $\kb^{\probtensor}$.
\end{proof}




\begin{theorem}\label{the:factorReduction}
	Let $\extnet=\extnetasset$ be a non-negative Tensor Network on a hypergraph $\graph=(\nodes,\edges)$, $\secnodes\subset\nodes$ be a subset and
		\[ \probtensor[\catvariableof{\secnodes}] = \normationof{\{\hypercoreat{\edge} \, : \, \edge\in\edges \}}{\catvariableof{\secnodes}} \]
	and
		\[ \tilde{\probtensor}[\catvariableof{\secnodes}] = \normationof{\{\nonzerofunction \circ \hypercoreat{\edge} \, : \, \edge\in\edges \}}{\catvariableof{\secnodes}} \]
	Then we have for any $\exformula$ that $\probtensor\models\exformula$ if and only if $\tilde{\probtensor}\models\exformula$.
\end{theorem}
\begin{proof}
	We first show
	\begin{align}\label{eq:proofFacReduction}
		 \nonzerofunction\circ\probtensor = \nonzerofunction\circ\tilde{\probtensor} \, . 
	\end{align}
	The claim follows then from Theorem~\ref{the:entailmentMarkovToKB}.
	To show \eqref{eq:proofFacReduction} let there be $\indexedcatvariableof{\secnodes}$ such that $\probtensor[\indexedcatvariableof{\secnodes}]=0$.
	Then for any $\indexedcatvariableof{\nodes}$ extending  $\indexedcatvariableof{\secnodes}$ we have $\contractionof{\{\hypercoreat{\edge} \, : \, \edge\in\edges \}}{\indexedcatvariableof{\nodes}} = 0$ and thus also $\contractionof{\{\nonzerofunction\circ\hypercoreat{\edge} \, : \, \edge\in\edges \}}{\indexedcatvariableof{\nodes}} = 0$ and $\tilde{\probtensor}[\indexedcatvariableof{\secnodes}]=0$.
	One can similarly show, that when $\tilde{\probtensor}[\indexedcatvariableof{\secnodes}]=0$ then also ${\probtensor}[\indexedcatvariableof{\secnodes}]=0$. 
	The support of the distributions $\probtensor$ and $\tilde{\probtensor}$ is thus identical and \eqref{eq:proofFacReduction} holds.
\end{proof}

% Consequence: Reduction of probabilitic entailment to logical entailment.
For any positive tensor $\hypercore$ we have
	\[ \nonzerofunction\circ\hypercoreat{\catvariableof{\edge}} = \onesat{\catvariableof{\edge}} \, , \]
which does not influence the distribution and can be omitted from the Markov Network.
By Theorem~\ref{the:factorReduction}, when deciding eintailment, we can reduce all tensors of a Markov Network to their support and omit those with full support.
Since the support indicating tensors $\nonzerofunction\circ\hypercoreat{\catvariableof{\edge}}$ are Boolean, each is a propositional formula and the Markov Network is turned into a Knowledge Base of their conjunctions.
Deciding probabilstic entailment is thus traced back to logical entailment.

%\subsubsection{Queries by Formulas}
%We have investigated a specific type of query for the definition of entailment.
%More generally, the semantic of logics thus offer a method to state generic queries on arbitrary probability distributions in an interpretable way.

%%  TO DO: Give examples, e.g. correlations by formulas
%
%% Probabilistic Queries
%Deciding entailment is a specific form of a query:
%\begin{itemize}
%	\item Query function is a formula, the one-hot encoding the formula tensor
%	\item Expectations, which are the output of the query, are interpreted whether they are parallel to $\fbasis$ (contradiction), $\tbasis$ (entailment), both (inconsistent KB) or neither (contingent)
%\end{itemize}
%
%
%%We now apply the developed formalism of formula tensors to design a knowledge base.
%Deciding entailment can be done by efficient tensor network contractions of the knowledge base sentences and the query formula in tensor network representation.








%\subsection{Deciding Entailment by Contractions}
%
%In the next Theorem we show how the normations required in the computation of $\condprobof{\exformula}{\kb=1}$ can be avoided when deciding entailment.
%

%\begin{proof}
%	The claim holds in case of unsatisfiable $\kb$, since any $\exformula$ is entailed and $\sbcontractionof{\kb,\rencodingof{\exformula}}{\formulavar}$ is parallel to both $\tbasis$ and $\fbasis$.
%	Let us thus assume, that $\kb$ is satisfiable, in which case we have
%	\begin{align*}
%		 \sbcontractionof{\kb,\rencodingof{\exformula}}{\formulavar}  
%		 & = \onehotmapof{0} \otimes \sbcontractionof{\kb,{\lnot\exformula}}{\formulavar} 
%		 + \onehotmapof{1} \otimes \sbcontractionof{\kb,{\exformula}}{\formulavar} \\
%		 & = \contractionof{\{\kb\}}{\varnothing} \cdot \left(  
%		 \onehotmapof{0} \otimes \frac{\sbcontractionof{\kb,{\lnot\exformula}}{\formulavar}}{\contractionof{\{\kb\}}{\varnothing}}
%		 + \onehotmapof{1} \otimes \frac{
%		 \sbcontractionof{\kb,{\exformula}}{\formulavar}
%		 }{\contractionof{\{\kb\}}{\varnothing}}
%		 \right) \\
%		 & = \contractionof{\{\kb\}}{\varnothing} \cdot \left(  
%		 	\onehotmapof{0} \otimes \condprobof{\exformula=0}{\kb=1}
%		 	+ \onehotmapof{1} \otimes \condprobof{\exformula=1}{\kb=1}
%		 \right) \\
%	\end{align*}
%	Now, if and only if $\kb\models\exformula$ we have $\condprobof{\exformula=1}{\kb=1}=1$ and
%		\[ \condprobof{\exformula=0}{\kb=1} = 1 - \condprobof{\exformula=1}{\kb=1}=0\]
%	and the term $\onehotmapof{0} \otimes \condprobof{\exformula=0}{\kb=1}$.
%	Exactly in this case, we then have $ \sbcontractionof{\kb,\rencodingof{\exformula}}{\formulavar}  \parallel \tbasis$.
%	By the same argument concerning the term $\onehotmapof{1} \otimes \condprobof{\exformula=1}{\kb=1}$, we get  $\sbcontractionof{\kb,\rencodingof{\exformula}}{\formulavar}\parallel\fbasis$ if and only if $\kb\models\lnot\exformula$.
%\end{proof}











%\subsection{Deciding Entailment by Contractions}
%
%\begin{theorem}
%	Given a Knowledge Base $\kb$ and a formula $\exformula$, we have $\kb\models\exformula$ if and only if
%		\[ \contractionof{\{\kb,\rencodingof{\exformula}\}}{\randomxof{\exformula}}  \parallel \tbasis \, . \]
%\end{theorem}
%\begin{proof}
%	We note that  $\contractionof{\{\kb,\rencodingof{\exformula}\}}{\randomxof{\exformula}}  \parallel \tbasis $ is equal to
%		\[ \contractionof{\{\kb,{\lnot\exformula}\}}{\varnothing}  = 0  \, . \]
%	This is equal to $\kb\land\not\exformula$ being inconsistent and therefore equal to $\kb\models\exformula$.
%\end{proof}












\subsection{Deciding Entailment by partial ordering}

% Classical definition of entailment
Classically entailment in propositional logics is defined by a model-theoretic approach.
According to that approach, the entailment statement $\kb\models\exformula$ holds, whenever any model of $\kb$ is also a model of $\exformula$.
We will in the following show, that this is equal to our definition based on probabilistic queries.


% Here for general tensors, not just propositional formulas!
\begin{definition}[Partial ordering of tensors]\label{def:partialFTOrder}
	We say that two tensors $\exformula$ and $\secexformula$ in a tensor space $\facspace$ are partially ordered, denoted by
		\[ {\exformula}\prec{\secexformula} \, , \]
	if for all $\catindices\in\facstates$
		\[ {\exformula}(\catindices) \leq {\secexformula}(\catindices) \, .\]
\end{definition}

We notice, that whenever ${\exformula} \prec{\secexformula}$ holds, for any model $\atomindices$ of $\exformula$ we have
\begin{align*}
	1 = \exformula(\atomindices) \leq \secexformula(\atomindices)
\end{align*}
and thus $\secexformula(\atomindices)=1$.
Therefore any model of $\exformula$ is also a model of $\secexformula$.
We show in the next theorem, that this is equivalent to the entailment statement $\exformula\models\secexformula$.

\begin{theorem}[Partial Ordering Criterion] \label{the:orderingEntailmentCriterion}
	We have $\kb\models\exformula$ if and only if $\kb\prec\exformula$.
\end{theorem}
\begin{proof}
	Directly by definition, since both $\kb$ and $\exformula$ are Boolean and therefore for any $\shortcatindices\in\atomstates$ we have that
		\[ \kbat{\indexedcatvariableof{[\catorder]}} \leq \formulaat{\indexedcatvariableof{[\catorder]}} \]
	is equivalent to 
		\[ \big(\kbat{\indexedcatvariableof{[\catorder]}}=1\big) \rightarrow \big(\formulaat{\indexedcatvariableof{[\catorder]}}=1\big) \, .  \]
	Therefore, $\kb\prec\exformula$ is equivalent to
		\[ \forall_{\shortcatindices\in\atomstates} \big(\kbat{\indexedcatvariableof{[\catorder]}}=1\big) \rightarrow \big(\formulaat{\indexedcatvariableof{[\catorder]}}=1\big) \, , \]
	which is equal to $\kb\models\exformula$.
%\red{Relate to other Theorem!}
%	By Theorem~\ref{cor:parallelCriterion} suffices to show that $\contractionof{\{\kb,\rencodingof{\exformula}\}}{\formulavar}  \parallel \tbasis$ is equivalent to $\kb\prec\exformula$.
%	To show this equivalence we observe
%		\[ \sbcontractionof{\kb,\rencodingof{\exformula}}{\formulavar}(0) = 
%		\contractionof{\{\kb,{\lnot\exformula}\}}{\varnothing} =
%		\# \left\{ i \in\facstates : \kb(i)= 1 \land \exformula(i) = 0 \right\} \, . \]
%	If and only if $\contractionof{\{\kb,\rencodingof{\exformula}\}}{\formulavar}  \parallel \tbasis$ we have $\sbcontraction{\kb,\exformula}=0$, which is equivalent to 
%		\[ \forall i \in\facstates : \lnot\kb(i)= 1 \land \exformula(i) = 0)  \, . \]
%	This is further equivalent to 
%		\[ \forall i \in\facstates : \kb(i) = 1 \rightarrow \exformula(i) = 1)  \]
%	and 
%		\[ \kb \prec \exformula \, . \]
\end{proof}

% Semantic Interpretation
%The partial ordering criterion offers a model-theoretic proof of entailment, since partial ordering is defined through comparison of all models:
%We have
%	\[ \exformula \prec \secexformula \]
%if and only if 
%	\[ \forall i\in \facstates : \exformula_i=1 \rightarrow \secexformula_i = 1 \, , \]
%that is any model of $\exformula$ is also a model of $\secexformula$.



% Partial Ordering
%We can therefore understand partial ordering as a generalization of entailment?





\subsubsection{Monotonicity of Entailment}


%\red{When defining entailment based on Markov Networks, would have clearer statement!}
Vanishing local contractions provide sufficient but not necessary criterion to decide entailment, as we show in the next theorem.

\begin{theorem}[Monotonicity of Entailment]\label{the:monotonEntailment}
	For any Markov Network on the decorated hypergraph $\graph$ and any subgraph $\secgraph$, we have for any formula that $\probtensor^{\graph}\models\exformula$ if $\probtensor^{\secgraph}\models\exformula$.
\end{theorem}	
\begin{proof}
	Based on the reduction to Knowledge Bases by Theorem~\ref{the:entailmentMarkovToKB} and the monotonocity of binary contractions as shown in Theorem~\ref{the:monotonicityBinaryContractions}.
\end{proof}



\begin{remark}
	To make use of Theorem~\ref{the:monotonEntailment} we can exploit any entailment criterion.
	However, there is no claim about entailment being false, when the entailment 
	Theorem~\ref{the:monotonEntailment} therefore just provides a sufficient but not necessary criterion of entailment with respect to $\probtensor^{\graph}$.
\end{remark}



\subsection{Deciding Entailment by local contractions}\label{subsec:LocalEntailment}


Global entailment can become inefficient, when
\begin{itemize}
	\item we are interested in batches of entailment checks. Here we can make use of dynamic programming (store partial contraction results in the Knowledge Cores).
	\item the network is large. Although efficient tensor network contraction often work, they might get infeasible when the tensor network has a large connectivity. For many 
\end{itemize}
An alternative to deciding entailment by global operations is the use of local operations.
Here we interpret a part of the network (for example a single core) as an own knowledge base (with atomic formulas being the roots of the directed subgraph, that is potentially differing with the atoms in the global perspective) and perform entailment with respect to that.

\begin{remark}{Tradeoff between generality and efficiency}
	While generic entailment decision algorithms (those by the full network) can decide any entailment, local algorithms as presented here can only perform some, but therefore more effectively as operating batchwise (dynamically deciding entailment for many leg variables).
	This is a typical phenomenon in logical reasoning and related to decidability.
\end{remark}


\subsubsection{Knowledge Propagation}

Let us now draw on these insights and store partial entailment results in Knowledge Cores, which is a use of the dynamic programming paradigm.
We then iterate over local entailment checks, where we recursively add further entailment checks to be redone due to additional knowledge.
We then call the local checks until convergence Entailment Propagation, since different stadia of knowledge are propagated through the network.
We describe local Knowledge Propagation in a generic way in Algorithm~\ref{alg:KP}.

\begin{algorithm}[hbt!]
\caption{Knowledge Propagation (KP)}\label{alg:KP}
\begin{algorithmic}
\State Tensor Network $\extnet$, $\kcoreof{\edge}=\onesat{\catvariableof{\edge}}$
\While{Stopping Criterion is not met}
	\State Choose $\edge$, subset $M$ of $\extnet$ and of $\{\kcoreof{\edge} : \edge\in\edges \}$ containing $\kcoreof{\edge}$
	\State Update 
		\[ \kcoreof{\edge} \leftarrow \nonzerofunction\circ\contractionof{M}{\catvariableof{\edge}} \]
\EndWhile
\end{algorithmic}
\end{algorithm}

% Interpretation
Each chosen subset $M$ is understood as a local knowledge base, which is then applied for local entailment.

%
%The Knowledge Cores 

%Implementation
There are different ways of implementing Algorithm~\ref{alg:KP}, by choosing an order of local knowledge bases $M$ and a stopping criterion.

\begin{theorem}
	In Entailment Propagation Algorithm~\ref{alg:KP}, $\kcoreof{\edge}$ is monotonically decreasing with respect to the partial ordering and greater than $\tilde{\kcoreof{\edge}}$ defined as
			\[ \tilde{\kcoreof{\edge}} = \nonzeroof{\contractionof{\extnet}{\edge}} \, . \]
\end{theorem}
\begin{proof}
	We deduce the theorem from generic properties of the support of contractions, see Section~\ref{sec:supportContractionEquations}.
	Monotonic decreasing follows from montonocity of tensor contractions, see Theorem~\ref{the:monotonicityBinaryContractions}.
	By Theorem~\ref{the:invarianceAddingSubcontractions} we have during any state of the algorithm 
		\[ \nonzerofunction\circ\contractionof{\extnet}{\catvariableof{\nodes}}  =  
		\nonzerofunction\circ\contractionof{\extnet\cup\{\kcoreof{\edge} : \edge\in\edges\}}{\catvariableof{\nodes}}  \, . 
		\]
	If follows that
		\[ \tilde{\kcoreof{\edge}} =  \nonzerofunction\circ\contractionof{\extnet\cup\{\kcoreof{\edge} : \edge\in\edges\}}{\catvariableof{\edge}} \]
	and by Theorem~\ref{the:monotonicityBinaryContractions}
		\[  \tilde{\kcoreof{\edge}}  \prec \kcoreof{\edge} \, . \]
\end{proof}


\begin{corollary}
	Whenever for a formula $\formulaat{\catvariableof{\secnodes}}$ and a $\kcoreof{\edge}$ we have
		\[ \contractionof{\kcoreof{\edge},\rencodingof{\exformula}}{\catvariableof{\exformula}=0} =0  \]
	then the Markov Network $\extnet$ probabilistically entails $\exformula$.
\end{corollary}


%% Interpretation for leg dimension two
Another way to use Algorithm~\ref{alg:KP} to decide entailement of formulas $\exformula$ is adding each $\rencodingof{\exformula}$ to $\extnet$ and defining Knowledge Cores $\kcoreof{\exformula}[\catvariableof{\exformula}]$.
Since then the Knowledge Core has only two dimensions, there are only four possible cores with the interpretation
\begin{itemize}
	\item $\tbasis$: the formula is known to be true
	\item $\fbasis$: the formula is known to be false
	\item $\ones$: the formula is not known
	\item $0$: the knowledge base is inconsistent
\end{itemize}


%% OLD Criteria 

%\subsection{Deciding entailment by Global Operations}
%
%\begin{corollary}\label{cor:SatisfiabilityCheck}
%	A Knowledge Base $\kb$ is satisfiable, if and only if
%		\[ \contractionof{\ftensorof{\kb}\cup\tbasis^{\kb}}{[]} \geq 0 \, .\]
%	Here $\tbasis^{\kb}$ denotes the tensor with values $\tbasis$ and leg variable $\kb$.
%\end{corollary}
%\begin{proof}
%	It would not be satisfiable if and only if $\formulaset\models\nothing$ and $\ftensorof{\nothing}=0$.
%\end{proof}
%
%More precisely, $\contractionof{\ftensorof{\kb}}{[]} $ is the count of models.
%
%
%%% Satisfaction Check
%
%\begin{theorem}\label{the:EntailmentCheck}
%	We have $\kb\models\exformula$ if and only if
%		\[ \contractionof{\ftensorof{\kb}\cup\tbasis^{\kb} \cup \ftensorof{\exformula} \cup \fbasis^{\exformula}}{[]} = 0 \, .\]
%\end{theorem}
%\begin{proof}
%	It is known that $\kb\models\exformula$ if and only if $\kb\cup\{\lnot\exformula\}$ unsatisfiable (proof by contradiction).
%	Using Corollary \ref{cor:SatisfiabilityCheck} we have $\kb\cup\{\lnot\exformula\}$ unsatisfiable if and only if
%		\[  \contractionof{\ftensorof{\kb\cup\{\lnot\exformula\}}\cup\tbasis^{\kb\cup\{\lnot\exformula\}}}{[]} = 0 \, .\]
%	The claim thus follows from noticing 
%		\[ \contractionof{\ftensorof{\kb\cup\{\lnot\exformula\}}\cup\tbasis^{\kb\cup\{\lnot\exformula\}}}{[]}  = 
%		\contractionof{\ftensorof{\kb}\cup\tbasis^{\kb} \cup \ftensorof{\exformula} \cup \fbasis^{\exformula}}{[]} \, . 
%		\]
%\end{proof}


%% Contraction based criterion % But obvious from Monotonocity!! % Best when having a Markov Network definition of entailment.
%\begin{theorem}[Local Contraction Criterion]\label{the:localEntailmentCriterion}
%	For any subset $\seckb$ of the binary tensor cores of $\kb$, we have $\kb\models\exformula$ (respectively $\kb\models\lnot\exformula$), if the contraction of $\seckb$ with leaving $\atomlegindexof{\exformula}$ open is parallel to $\fbasis$ (respectively parallel to $\fbasis)$, denoted by 
%		\[ \contractionof{\ftensorof{\seckb}}{[\atomlegindexof{\exformula}]}  \parallel \tbasis \quad \text{( respectively }\contractionof{\ftensorof{\seckb}}{[\atomlegindexof{\exformula}]}  \parallel \fbasis \text{)} \, . \]	
%%	 whenever the contraction of the subset with leaving $\atomlegindexof{\exformula}$ open is parallel to $\fbasis$ (respectively parallel to $\fbasis)$, then $\kb\models\exformula$ (respectively $\kb\models\lnot\exformula)$.
%\end{theorem}
%\begin{proof}
%	From the monotonicity of binary tensor contractions with respect to the partial ordering it follows with Theorem~\ref{the:monotonicityBinaryContractions} that
%		\[ \ftensorof{\kb} \prec \ftensorof{\seckb} \, . \]
%	By Theorem~\ref{the:orderingEntailmentCriterion} we further have $\kb\models\seckb$. 
%	Theorem~\ref{cor:parallelCriterion} now implies, that $\seckb\models\exformula$ (respectively $\seckb\models\lnot\exformula$), when the in the claim assumed contraction criterion is satisfied.
%	By monotonicity of entailment we in that case further have $\kb\models\exformula$ (respectively $\seckb\models\lnot\exformula$).
%%	For any knowledge base $\seckb$ such that $\kb\models\seckb$ (seen as sublist of its formulas represented by knowledge cores).
%%	Follows from the monotonicity of propositional logic: When entailment by subset, then also entailment by the full, but not the other way around.
%\end{proof}




% Reasoning on MLN using Tensor Network Decompositions
\part{Neuro-Symbolic Learning}

We now employ tensor networks to define architectures and algorithms for neuro-symbolic reasoning based on the logical and probabilistic foundations.
Markov Logic Networks will be taken as generative models to be learned from data, using formula selecting tensor networks and likelihood optimization algorithms.

\section{Formula Selecting Networks}\label{sec:superposedFT}\label{cha:formulaBatches}\label{cha:architectures}

In this chapter we will investigate efficient schemes to represent collections of formulas with similar structure in one tensor network.

% Relational encoding of the selection map
\begin{definition}
	Given a set of $\parlegdim$ formulas $\{\formulaof{\selindex} : \selindexin\}$, we define the formula selecting map as
		\[  \fselectionmapat{\shortcatvariables,\selvariable} : \atomstates \times [\parlegdim] \rightarrow [2] \]
	defined for $\selindexin$ by
		\[ \fselectionmapat{\shortcatvariables=\atomindices,\selvariable=\selindex} =  \formulaofat{\selindex}{\shortcatvariables=\atomindices} \, . \]
\end{definition}

% Selection Variables
We introduce a selection variable $\selvariable$ and depict the formula selection in Figure~\ref{fig:formulaSelectionMap}.

% Depiction
\begin{figure}[h]
\begin{center}
	\begin{tikzpicture}[scale=0.35, thick] % , baseline = -3.5pt

\begin{scope}[shift={(-20,0)}]
	\node[anchor=center] (text) at (-1,3) {${a)}$};

	\node [circle, draw, thick, fill=gray!50, minimum size = \nodeminsize] (T1) at (0,0) {\tiny $\randomxof{0}$};
	\node [circle, draw, thick, fill=gray!50, minimum size = \nodeminsize] (T2) at (3,0) {\tiny $\randomxof{1}$};
	\node[anchor=center] (text) at (6,0) {${\cdots}$};
	\node [circle, draw, thick, fill=gray!50, minimum size = \nodeminsize] (T3) at (9,0) {};
	\node[anchor=center] (text) at (9,0) {\tiny $\randomxof{\parlegdim\shortminus1}$};
	
	\node [circle, draw, thick, fill=gray!50, minimum size = \nodeminsize] (T4) at (12,3) {};
	\node[anchor=center] (text) at (12,3) {\tiny $\vselectionvariable$};

	
	\node [circle, draw, thick, fill=gray!50, minimum size = \nodeminsize] (S) at (6,3) {};
	\node[anchor=center] (text) at (6,3) {\tiny $\fselectionmap$};
	
	\draw[->] (T1) -- (S);
	\draw[->] (T2) -- (S);
	\draw[->] (T3) -- (S);
	\draw[->] (T4) -- (S);
	
\end{scope}


\node[anchor=center] (text) at (-1,3) {${b)}$};
	

	\begin{scope}[shift={(0,-2)}]
		\draw[<-] (0,1)--(0,-1) node[midway,left] {\tiny $\catvariableof{0}$}; 
		\draw[<-] (1.5,1)--(1.5,-1) node[midway,left] {\tiny $\catvariableof{1}$}; 
		\node[anchor=center] (text) at (3,0) {$\cdots$};
		\draw[<-] (4,1)--(4,-1) node[midway,right] {\tiny $\catvariableof{\parlegdim\shortminus1}$}; 
	\end{scope}
	
\draw (-1,1) rectangle (5,-1);
\node[anchor=center] (text) at (2,0) {$\rencodingof{\fselectionmap}$};
\draw[->] (2,1) -- (2,3) node[midway, right]  {\tiny $\catvariableof{\fselectionmap}$};
\draw[<-] (5,0) -- (7,0) node[midway, above] {\tiny $\fselectionvariable$};




		
\end{tikzpicture}
\end{center}
\caption{Representation of the Formula Selecting map as a 
a) Graphical Model with a selection variable $\fselectionmap$.
b) Dual Tensor Core with selection variable corresponding with an additional axis.}
\label{fig:formulaSelectionMap}
\end{figure}


% Decomposition
A naive representation of the formula selecting map is as a sum
	\[ \fselectionmap = \sum_{\selindexin} \formulaofat{\selindex}{\shortcatvariables}  \otimes \onehotmapofat{\selindex}{\selvariable} \, . \]
Such a representation scheme requires linear resources in the number of formulas.
We will show in the following, that we can exploit common structure in formulas to drastically reduce this resource consumption.



\subsection{Construction schemes}

% Naturality of folding
Let us now investigate efficient schemes to define sets of formulas to be used in the definition of $\fselectionmap$.
We will motivate the folding of the selection variable into multiple selection variables by compositions of selection maps.


\subsubsection{Connective Selecting Tensors}

We represent choices over connectives with a fixed number of arguments by adding a selection variable to the cores and defining each slice by a candidate connective.

% Formal map
\begin{definition}\label{def:connectiveSelector}
	Let $\{\connectiveof{0},\ldots,\connectiveof{\parlegdimof{\cselectionsymbol}-1}\}$ be a set of connectives with $\atomorder$ arguments.
	The associated connective selection map is
		\[ \cselectionmapat{\shortcatvariables,\selvariableof{\cselectionsymbol}}
		: \atomstates \times [\parlegdimof{\cselectionsymbol}] \rightarrow [2] \]
	defined for each $\selindexofin{\cselectionsymbol}$ and $\shortcatindices\in\atomstates$ by 
		\[ \cselectionmapat{\shortcatvariables=\shortcatindices,\indexedselvariableof{\cselectionsymbol}} 
		= \connectiveofat{\selindexof{\cselectionsymbol}}{\shortcatvariables=\shortcatindices}  \, . \]
\end{definition}

We depict the relational encoding of connective selection maps in Figure~\ref{fig:connectiveSelector}.

\begin{figure}[h]
\begin{center}
	\begin{tikzpicture}[scale=0.35, thick] % , baseline = -3.5pt

\begin{scope}[shift={(-20,0)}]
	\node[anchor=center] (text) at (-1,3) {${a)}$};

	\node [circle, draw, thick, fill=gray!50, minimum size = \nodeminsize] (T1) at (0,0) {\tiny $\randomxof{0}$};
	\node [circle, draw, thick, fill=gray!50, minimum size = \nodeminsize] (T2) at (3,0) {\tiny $\randomxof{1}$};
	\node[anchor=center] (text) at (6,0) {${\cdots}$};
	\node [circle, draw, thick, fill=gray!50, minimum size = \nodeminsize] (T3) at (9,0) {};
	\node[anchor=center] (text) at (9,0) {\tiny $\randomxof{\atomorder\shortminus1}$};
	
	\node [circle, draw, thick, fill=gray!50, minimum size = \nodeminsize] (T4) at (12,0) {};
	\node[anchor=center] (text) at (12,0) {\tiny $\cselinputvariable$};

	
	\node [circle, draw, thick, fill=gray!50, minimum size = \nodeminsize] (S2) at (6,4.5) {};
	\node[anchor=center] (text) at (6,4.5) {\tiny $\catvariableof{\cselectionsymbol}$};
	
	\coordinate (S) at (6,2.5);
	\draw[->] (S) -- (S2);

	\draw[->] (T1) -- (S);
	\draw[->] (T2) -- (S);
	\draw[->] (T3) -- (S);
	\draw[->] (T4) -- (S);
	
\end{scope}


\node[anchor=center] (text) at (-1,3) {${b)}$};
	
	
\begin{scope}[shift={(5,2)}]

	\begin{scope}[shift={(0,-2)}]
		\draw[<-] (0,1)--(0,-1) node[midway,left] {\tiny $\catvariableof{0}$}; 
		\draw[<-] (1.5,1)--(1.5,-1) node[midway,left] {\tiny $\catvariableof{1}$}; 
		\node[anchor=center] (text) at (3,0) {$\cdots$};
		\draw[<-] (4,1)--(4,-1) node[midway,right] {\tiny $\catvariableof{\atomorder\shortminus1}$}; 
	\end{scope}
	
\draw (-1,1) rectangle (5,-1);
\node[anchor=center] (text) at (2,0) {$\rencodingof{\cselectionmap}$};
\draw[->] (2,1) -- (2,3) node[midway, right]  {\tiny $\catvariableof{\cselectionsymbol}$};
\draw[<-] (5,0) -- (7,0) node[midway, above] {\tiny $\cselinputvariable$};

\end{scope}
	

\end{tikzpicture}
\end{center}
\caption{Connective Selector.}
\label{fig:connectiveSelector}
\end{figure}

%Following a different perspective: skeleton+atomindices at atomic expression level, atomindices at complex expression level!
%Having an parametrization of binary connectives by $\circ_{\selindex}$ we can define the corresponding connective selector tensor by
%	\[ \concoreof{\circ}_{\selindex,:,:} = \concoreof{\circ_{\selindex}}_{:,:} \, . \]

\begin{remark}[$\htformat$ Interpretation of Superposed Formula Tensor Networks]\label{rem:HTDecomSFT}
	Continuing Remark ~\ref{rem:HTDecomFT}: 
	Superposed Formula Tensors have a decomposition into a $\htformat$ as sketched here, where we distinguish between formula selection subspaces (indices $\selindexof{\parenumerator}$) and atomic subspaces (indices $\atomlegindexof{\atomenumerator})$.
	At each formula selection we thus have a decomposition into three subspaces, two of atomic formulas and one for the formula selection.
\end{remark}




\subsubsection{Variable Selecting Tensor Network}

%\red{Works also for categorical variables! -> Into Contraction Calculus?}

%% Definition
\begin{definition}\label{def:variableSelector}
	The selection of one out of $\seldim$ variables in a list $\catvariableof{[\seldim]}$ is done by variable selecting maps
	\begin{align}
		\vselectionmapat{\catvariableof{[\seldim]},\selvariableof{\vselectionsymbol}}:  \left(\bigtimes_{\selindex\in[\parlegdim]}[2]\right) \times [\seldim]  \rightarrow [2]
	\end{align}
	are defined coordinatewise by
	\begin{align}
		\vselectionmapat{\indexedcatvariableof{0},\ldots,\indexedcatvariableof{\seldim-1},\indexedselvariableof{\vselectionsymbol}} = \catindexof{\selindex} \, .
	\end{align}
\end{definition}
	
% Interpretation as multiplex gate
Variable selecting maps appear in the literature as multiplex gates (see Definition 5.3 in \cite{koller_probabilistic_2009}).

The relational encoding of the variable selection map has a decomposition 
\begin{align*}
	\rencodingofat{\vselectionmap}{\vselectionheadvar,\catvariableof{[\seldimof{\vselectionsymbol}]}}
	= \sum_{\selindexofin{\vselectionsymbol}} 
	\rencodingofat{\atomicformulaof{\selindexof{\vselectionsymbol}}}{\vselectionheadvar,\catvariableof{\selindexof{\vselectionsymbol}}} \otimes  \onehotmapofat{\selindexof{\vselectionsymbol}}{\selvariableof{\vselectionsymbol}} \, . 
\end{align*}
This structure is exploited in the next theorem to derive a tensor network decomposition of $\rencodingof{\vselectionmap}$.

\begin{theorem}[Decomposition of Variable Selecting Maps]\label{the:varSelectorDecomposition}
	Given a list $\catvariableof{[\seldimof{\vselectionsymbol}]}$ of variables, we define for each $\selindexofin{\vselectionsymbol}$ the tensors
		\[ \selectorcomponentofat{\selindexof{\vselectionsymbol}}{\catvariableof{\selindexof{\vselectionsymbol}},\selvariableof{\vselectionsymbol}} 
		= \identityat{\vselectionheadvar,\catvariableof{\selindexof{\vselectionsymbol}}} \otimes \onehotmapofat{\selindexof{\vselectionsymbol}}{\selvariableof{\vselectionsymbol}} 
		+ \onesat{\vselectionheadvar,\catvariableof{\selindexof{\vselectionsymbol}}} \otimes \left(\onesat{\selvariableof{\vselectionsymbol}} - \onehotmapofat{\selindexof{\vselectionsymbol}}{\selvariableof{\vselectionsymbol}} \right) \, . 
		\]
	Then we have (see Figure~\ref{fig:SelectorDecomposition})
		\[ \rencodingofat{\vselectionmap}{\vselectionheadvar,\catvariableof{[\parlegdim]},\selvariableof{\vselectionsymbol}} 
		= \contractionof{
			\{\selectorcomponentofat{\selindexof{\vselectionsymbol}}{\vselectionheadvar,\catvariableof{\selindexof{\vselectionsymbol}},\selvariableof{\vselectionsymbol}} \, : \, \selindexofin{\vselectionsymbol}\}
		}{\vselectionheadvar,\catvariableof{[\parlegdim]},\selvariableof{\vselectionsymbol}} \, . 
		\]
\end{theorem}
\begin{proof}
	We show the equivalence of the tensors on an arbitrary coordinates.
	For $\tilde{\selindex}_{\vselectionsymbol}\in[\seldimof{\vselectionsymbol}]$, $\vselectionheadvar\in[2]$ and $\catindexof{[\seldimof{\vselectionsymbol}]}\in\bigtimes_{\catenumerator\in[\seldimof{\vselectionsymbol}]}[2]$ we have
	\begin{align*}
		& \contractionof{
			\{\selectorcomponentofat{\selindexof{\vselectionsymbol}}{\vselectionheadvar,\catvariableof{\selindexof{\vselectionsymbol}},\selvariableof{\vselectionsymbol}} \, : \, \selindexofin{\vselectionsymbol}\}
		}{\indexedheadvariableof{\vselectionsymbol},\indexedcatvariableof{[\parlegdim]},\selvariableof{\vselectionsymbol} = \tilde{\selindex}_{\vselectionsymbol}} \\
		& \quad = 
		\prod_{\selindexofin{\vselectionsymbol}} \selectorcomponentofat{\selindexof{\vselectionsymbol}}{
			\indexedheadvariableof{\vselectionsymbol},\indexedcatvariableof{\selindexof{\vselectionsymbol}},\selvariableof{\vselectionsymbol}=\tilde{\selindex}_{\vselectionsymbol}
			} \\
		& \quad = \selectorcomponentofat{\tilde{\selindex}_{\vselectionsymbol}}{
			\indexedheadvariableof{\vselectionsymbol},\indexedcatvariableof{\selindexof{\vselectionsymbol}},\selvariableof{\vselectionsymbol}=\tilde{\selindex}_{\vselectionsymbol}
		} \\
		& \quad = 
		\begin{cases}
		 	1 & \text{if} \quad \headindexof{\vselectionsymbol} = \catindexof{\selindexof{\vselectionsymbol}} \\
		 	0 & \text{else}  
		 \end{cases} \\ 
		 & = \rencodingofat{\vselectionmap}{\indexedheadvariableof{\vselectionsymbol},\indexedcatvariableof{[\parlegdim]},\selvariableof{\vselectionsymbol}=\tilde{\selindex}_{\vselectionsymbol}}
	\end{align*}
	In the second equality, we used that the tensor $\selectorcomponentof{\selindexof{\vselectionsymbol}}$ have coordinates $1$ whenever $\tilde{\selindex}_{\vselectionsymbol}\neq\selindexof{\vselectionsymbol}$.
\end{proof}


The decomposition provided by Theorem~\ref{the:varSelectorDecomposition} is in a CP format, as will be further discussed in Chapter~\ref{cha:sparseTC}.
The introduced tensors $\selectorcomponentof{\selindexof{\vselectionsymbol}}$ are Boolean, but not directed and therefore encodings of relations but not functions (see Chapter~\ref{cha:basisCalculus}).

%%% Decomposition
%% ! THIS IS NOT Theorem~\ref{the:functionDecompositionBasisCP}, but works on slice sparsity!

%Using that the encoding $\rencodingof{\atomicformulaof{\selindex}}$ of atomic formulas admits and elementary decomposition (see Theorem~\ref{the:AtomicFTensor}) we notice that Equation~\ref{eq:selectorDecomposition} describes a so-called monomial decomposition, which will be introduced in Definition~\ref{def:polynomialSparsity}.
%We can apply Theorem~\ref{the:sliceToCP} to find a decomposition of $\selectorcore$ in a CP format consisting of cores
%\begin{align}
%	\selectorcoreof{\selindex} = \dirdeltaof{\randomxof{\selindex},\vselectionvariable} \otimes \onehotmapof{\selindex}
%	+ \sum_{\tilde{\selindex}\in[\parlegdim] \, , \, \tilde{\selindex}\neq\selindex} \onesof{\randomxof{\selindex},\vselectionvariable} \otimes \onehotmapof{\tilde{\selindex}} \, . 
%\end{align}	
%The CP decomposition is depicted in Figure~\ref{fig:SelectorDecomposition}.
%
%% Selectorcores are non-functional relational encodings
%We notice, that the selector cores $\selectorcoreof{\selindex}$ are encodings of a relation, which is not a function.
%Therefore, they are binary but not directed tensors.
%Their contraction 
%\begin{align}\label{eq:selectorDecomposition}
% 	\selectorcore = 
%	\contractionof{\{\selectorcomponentof{\selindex}\, : \, \selindexin\}}
%	{\{\randomxof{0},\ldots,\randomxof{\parlegdim-1},\vselectionvariable\}} 
%\end{align}
%is the relational encoding of the function $\vselectionmap$ and thus binary and directed.


%% Interpretation
%The selectorcores $\selectorcoreof{\selindexof{1}}$ are contracted with the parameter cores and select the respective atom when contracted with truth vector tensormultiplied by constant cores (as placeholder for the other possible atoms).
%Decomposed into disconnected strands for each atomkey, which connect on the selection axis and on the atom truth axis.


\begin{figure}[h]
\begin{center}
	\begin{tikzpicture}[scale=0.35, thick] % , baseline = -3.5pt

%\begin{scope}[shift={(-20,0)}]
%	\node[anchor=center] (text) at (-1,3) {${a)}$};
%
%	\node [circle, draw, thick, fill=gray!50, minimum size = \nodeminsize] (T1) at (0,0) {\tiny $\catvariableof{0}$};
%	\node [circle, draw, thick, fill=gray!50, minimum size = \nodeminsize] (T2) at (3,0) {\tiny $\catvariableof{1}$};
%	\node[anchor=center] (text) at (6,0) {${\cdots}$};
%	\node [circle, draw, thick, fill=gray!50, minimum size = \nodeminsize] (T3) at (9,0) {};
%	\node[anchor=center] (text) at (9,0) {\tiny $\catvariableof{\parlegdim\shortminus1}$};
%	
%	\node [circle, draw, thick, fill=gray!50, minimum size = \nodeminsize] (T4) at (12,3) {};
%	\node[anchor=center] (text) at (12,3) {\tiny $\vselectionvariable$};
%
%	
%	\node [circle, draw, thick, fill=gray!50, minimum size = \nodeminsize] (S) at (6,3) {};
%	\node[anchor=center] (text) at (6,3) {\tiny $\vselectionmap$};
%	
%	\draw[->] (T1) -- (S);
%	\draw[->] (T2) -- (S);
%	\draw[->] (T3) -- (S);
%	\draw[->] (T4) -- (S);
%	
%\end{scope}
%
%
%\node[anchor=center] (text) at (-1,3) {${b)}$};
	

	\begin{scope}[shift={(0,-2)}]
		\draw[<-] (0,1)--(0,-1) node[below] {\tiny $\catvariableof{0}$}; 
		\draw[<-] (1.5,1)--(1.5,-1) node[below] {\tiny $\catvariableof{1}$}; 
		\node[anchor=center] (text) at (3,0) {$\cdots$};
		\draw[<-] (4,1)--(4,-1) node[below] {\tiny $\catvariableof{\seldimof{\vselectionsymbol}\shortminus1}$}; 
	\end{scope}
	
\draw (-1,1) rectangle (5,-1);
\node[anchor=center] (text) at (2,0) {$\rencodingof{\vselectionmap}$};
\draw[->] (2,1) -- (2,3) node[midway, right]  {\tiny $\vselectionheadvar$};
\draw[<-] (5,0) -- (7,0) node[midway, above] {\tiny $\selvariableof{\vselectionsymbol}$};


\node[anchor=center] (text) at (9,0) {${=}$};


\begin{scope}[shift={(12,2)}]

\newcommand{\conposseldec}{4.5,1}

\draw[fill] (\conposseldec) circle (0.25cm);
\draw[->] (\conposseldec) -- (4.5,3) node[midway, right]{\tiny $\vselectionheadvar$};

\draw[<-]  (0,-3) -- (0,-5);
\draw (0,-5) -- (0,-7) node[midway,left] {\tiny $\catvariableof{0}$};
\draw (-1,-1) rectangle (1, -3);
\node[anchor=center] (text) at (0,-2) {\small $\selectorcomponentof{0}$};
\draw[] (0,-1) to[bend right=-20] (\conposseldec);
\draw[] (1,-1.5) -- (12,-1.5) ; 

\draw[<-]  (3,-4) -- (3,-5);
\draw[] (3,-5) -- (3,-7) node[midway,left] {\tiny $\catvariableof{1}$};
\draw (2,-2) rectangle (4, -4);
\node[anchor=center] (text) at (3,-3) {\small $\selectorcomponentof{1}$};
\draw[] (3,-2) to[bend right=-20]  (\conposseldec);
\draw[] (4,-3) to[bend right=3]  (12,-1.5);


\node[anchor=center] (text) at (6,-3.5) {$\cdots$};

\draw[<-]  (9,-5) -- (9,-7) node[midway,left] {\tiny $\catvariableof{\seldimof{\vselectionsymbol}\shortminus1}$};
\draw (7.55,-3) rectangle (10.45, -5);
\node[anchor=center] (text) at (9,-4) {\small $\selectorcomponentof{\seldimof{\vselectionsymbol}\shortminus1}$};
\draw[] (9,-3) to[bend left=-20]  (\conposseldec);
\draw[] (10.45,-4) to[bend right=5]  (12,-1.5);

\draw[fill] (12,-1.5) circle (0.25cm);
\draw[<-] (12.25,-1.5) -- (14,-1.5) node[midway,above] {\tiny $\selvariableof{\vselectionsymbol}$};
\end{scope}

		


\end{tikzpicture}
\end{center}
\caption{Decomposition of the relational encoding of a variable selecting tensor into a network of tensors defined in Theorem~\ref{the:varSelectorDecomposition}.
	The decomposition is in a $\cpformat$-Format (see Chapter~\ref{cha:sparseTC}. %, when grouping the indices  $\selindexof{\parenumerator}$ and $\atomlegindexof{\atomicformulaof{\selindexof{\parenumerator}}}$).
	%To ease the notation, we here use $\concoreof{\parenumerator}$ to denote $\ftensorof{\concoreof{\parenumerator}}$.
}
\label{fig:SelectorDecomposition}
\end{figure}




\subsection{State Selecting Tensors}

As an alternative, one can select a state of a categorical variable $\catvariable$.

\begin{definition}
	Given a categorical variable $\catvariableof{\sselectionsymbol}$ with dimension $\catdimof{\sselectionsymbol}$ and a selection variable $\selvariableof{\sselectionsymbol}$ with dimension $\seldimof{\sselectionsymbol}=\catdimof{\sselectionsymbol}$ the state selecting tensor 
		\[ \sselectionmapat{\catvariableof{\sselectionsymbol},\selvariableof{\sselectionsymbol}} : [\catdimof{\sselectionsymbol}] \times [\seldimof{\sselectionsymbol}] \rightarrow [2] \]
	is defined on $\catindexofin{\sselectionsymbol}$ and $\selindexofin{\sselectionsymbol}$ by
	\begin{align*}
		\sselectionmapat{\indexedcatvariableof{},\indexedselvariableof{\sselectionsymbol}} = 
		\begin{cases}
			1 & \text{if} \quad \catindex = \selindexof{\sselectionsymbol} \\
			0 & \text{else}
		\end{cases} \, . 
	\end{align*}
\end{definition}

% Comment: Alternative based on categorical constraints to be introduced later
State selecting tensors can also be realized by variable selecting tensors.
In Section~\ref{sec:categoricalTN} we will describe methods to build atomic variables indicating the states of a categorical variable.
This would, however, increase the number of variables in a tensor network and can thus lead to an exponential overhead of dimensions.
State selecting tensors can therefore be seen as a mean to avoid such dimension increases.

\red{Comment: State Selectors can be integrated in Variable Selection framework. In this perspective, Variable selection networks are the specific case to $X=1$. }


%% OLD Alternative
%Such categorical variable cores have the advantage of avoiding a full atomization of the categorical variable, which is the creation of atoms reproducing the values
%	\[ \catvariable==\catindexof{\catvariable} \, . \]
%By representing categorical variable choice, one can thus avoid an increase of the order of the encoded tensors, which avoids intractabilities.
%Categorical selection cores can further be integrated in the decomposition scheme \eqref{eq:selectorDecomposition}. 







\subsection{Composition of formula selecting maps}
%\subsection{Folding of the Selection Variable}

We will now parametrize the sets $\formulaset$ with additional indices and define formula selector maps subsuming all formulas.
To handle large sets of formulas, we further fold the selection variable into tuples of selection variables.

\begin{definition}%\label{def:formulaSelector}
	Let there be a formula $\formulaof{\selindexlist}$ for each index tuple in $\selindexlist\in\selstates$, where $\selorder,\seldimof{0},\ldots,\seldimof{\selorder-1}\in\nn$.
	The folded formula selector map (see Figure~\ref{fig:foldedSelector}) is the map 
		\[ \fselectionmapat{\shortcatvariables,\shortselvariables} : \left(\atomstates\right) \times \left(\selstates\right) \rightarrow [2] \]
	with the coordinates at the indices $\shortcatindices\in\atomstates$, $\shortselindices\in\selstates$
		\[  \fselectionmapat{\shortcatvariables=\shortcatindices,\shortselvariables=\shortselindices} 
		= \formulaofat{\shortselindices}{\shortcatvariables=\shortcatindices} \, . \]
\end{definition}

% Formula Section based on skeleton expressions
We will find formula selector maps by composition variables selector maps (Definition~\ref{def:variableSelector}) and connective selector maps (Definition~\ref{def:connectiveSelector}).
This is especially useful to provide efficient decompositions of relational encodings. 

\begin{figure}[h]
\begin{center}
	\begin{tikzpicture}[scale=0.35,thick] % , baseline = -3.5pt

\begin{scope}[shift={(-25,-4)}]
	\node[anchor=center] (text) at (0,7) {${a)}$};

	\node [circle, draw, thick, fill=gray!50, minimum size = \nodeminsize] (T1) at (0,0) {\tiny $\randomxof{0}$};
	\node [circle, draw, thick, fill=gray!50, minimum size = \nodeminsize] (T2) at (3,0) {\tiny $\randomxof{1}$};
	\node[anchor=center] (text) at (6,0) {${\cdots}$};
	\node [circle, draw, thick, fill=gray!50, minimum size = \nodeminsize] (T3) at (9,0) {};
	\node[anchor=center] (text) at (9,0) {\tiny $\randomxof{\atomorder-1}$};
	



	\node [circle, draw, thick, fill=gray!50, minimum size = \nodeminsize] (S2) at (6,6) {};
	\node[anchor=center] (text) at (6,6) {\tiny $\headvariableof{\fselectionmap}$};
	
	\node [circle, draw, thick, fill=gray!50, minimum size = \nodeminsize] (T4) at (12,6) {};
	\node[anchor=center] (text) at (12,6) {\tiny $\selvariableof{\parorder-1}$};
	
	%\node [circle, draw, thick, fill=gray!50, minimum size = \nodeminsize] (S) at (6,3) {};
	\node[anchor=center] (text) at (12,3.5) {$\vdots$};
	
	\node [circle, draw, thick, fill=gray!50, minimum size = \nodeminsize] (T5) at (12,1) {};
	\node[anchor=center] (text) at (12,1) {\tiny $\selvariableof{0}$};
	
	\coordinate (S) at (6,2.5);
	\draw[->] (S) -- (S2);
	
	\draw[->] (T1) -- (S);
	\draw[->] (T2) -- (S);
	\draw[->] (T3) -- (S);
	\draw[->] (T4) -- (S);
	\draw[->] (T5) -- (S);
	
\end{scope}



\node[anchor=center] (text) at (-3,3) {${b)}$};


\drawatomindices{0}{-4}
\draw (-1,3) rectangle (5, -3);
\node[anchor=center] (text) at (2,0) {$\rencodingof{\fselectionmap}$};

\draw[->] (2,3)--(2,5) node[midway,right] {\tiny $\headvariableof{\fselectionmap}$};

\draw[<-] (5,-2)--(7,-2) node[midway,below] {\tiny $\selvariableof{0}$}; 
\draw[<-] (5,-0.5)--(7,-0.5) node[midway,below] {\tiny $\selvariableof{1}$}; 
\node[anchor=center] (text) at (6,0.75) {$\vdots$};
\draw[<-] (5,2)--(7,2) node[midway,above] {\tiny $\selvariableof{\parorder\shortminus1}$}; 


%\draw (7,3) rectangle (9, -3);
%\node[anchor=center] (text) at (8,0) {$\canparam$};


\end{tikzpicture}
\end{center}
\caption{Relational encoding of the folded map $\fselectionmap$.}
\label{fig:foldedSelector}
\end{figure}




\subsubsection{Formula Selecting Neuron}


% Motivating foldings by composition
The folding of the selection variable is motivated by the composition of selection maps.
We call the composition of a connective selection with variable selection maps for each argument a formula selecting neuron.


\begin{definition}\label{def:fsNeuron}
	Given an order $\selorder\in\nn$ let there be a connective selector $\cselectionvariable$ selecting connectives of order $\selorder$ and let $\vselectionmapof{0},\ldots,\vselectionmapof{\selorder-1}$ be a collection of variable selectors.
	The corresponding logical neuron is the map
	\begin{align*}
		\lneuronat{\shortcatvariablelist,\shortselvariablelist} 
		: \left(\atomstates\right) \times [\seldimof{\cselectionsymbol}] \times \left( \bigtimes_{\selenumeratorin} [\seldimof{\selenumerator}]\right) \rightarrow [2] 
	\end{align*}
	defined for $\shortcatindices\in\atomstates$, $\selindexof{\cselectionsymbol}\in[\parlegdimof{\cselectionsymbol}]$ and 
	$\parindices\in \bigtimes_{\parenumeratorin} [\parlegdimof{\parenumerator}]$ by
	\begin{align*}
		\lneuron(\atomindices, \selindexof{\cselectionsymbol}, \parindices) = 
		\cselectionmap(\vselectionmapof{0}(\atomindices, \selindexof{0}),\ldots,\vselectionmapof{\parorder-1}(\atomindices,\selindexof{\parorder-1}), \selindexof{\cselectionsymbol}) \, .
	\end{align*}
\end{definition}

% Tensor Network Decomposition
Each neuron has a tensor network decomposition by a connective selector tensor and a variable selector tensor network for each argument, as we state in the next theorem.

\begin{theorem}{Decomposition of formula selecting neurons}\label{the:neuronDecomposition}
	Let $\lneuron$ a logical neuron, defined for a connective selector $\cselectionvariable$ and variable selectors $\vselectionmapof{0},\ldots,\vselectionmapof{\selorder-1}$.
	Then we have (see Figure~\ref{fig:neuronDecomposition} for the example of $\selorder=2$):
	\begin{align*}
		&\rencodingofat{\lneuron}{\headvariableof{\lneuron},\shortcatvariables,\selvariableof{\cselectionsymbol},\selvariableof{\vselectionsymbol,0},\ldots,\selvariableof{\vselectionsymbol,\selorder-1}} \\
		&\quad = \langle\{\rencodingofat{\cselectionmap}{
				\headvariableof{\lneuron},\headvariableof{\vselectionsymbol,0},\ldots,\headvariableof{\vselectionsymbol,\selorder-1}}, \\
		& \quad\quad\quad\rencodingofat{\vselectionmapof{0}}{
				\headvariableof{\vselectionsymbol,0},\shortcatvariables,\selvariableof{\vselectionsymbol,0}},\ldots,
				\rencodingofat{\vselectionmapof{\selorder-1}}{
					\headvariableof{\vselectionsymbol,\selorder-1},\shortcatvariables,\selvariableof{\vselectionsymbol,\selorder-1}}
				\} \rangle
		\left[\headvariableof{\lneuron},\shortcatvariables, \selvariableof{\cselectionsymbol},\selvariableof{\vselectionsymbol,0},\ldots,\selvariableof{\vselectionsymbol,\selorder-1}\right] \, .
	\end{align*}
\end{theorem}
\begin{proof}
	By composition Theorem~\ref{the:compositionByContraction}.
\end{proof}




\red{Example of a formula selecting neuron:}
Given a skeleton expression and a set of candidates at each placeholder, we parameterize a set of formulas by the assignment of candidate atoms to each placeholder position.
Let us denote the set of formulas, which are generated through choosing atoms from $\candidatelistof{\parenumerator}$ for the skeleton formula $\skeleton$ by
		\[ \formulasetof{\skeleton} \coloneqq 
	 \left\{ \skeletonof{\placeholderof{1},\ldots,\placeholderof{\atomorder}} \, : \, \placeholderof{\atomenumerator} \in \candidatelistof{\atomenumerator} \right\} \]

%We now enumerate at each position $\parenumerator$ the list of candidates $\candidatelistof{\parenumerator}$ using an index $\selindexof{\parenumerator}\in[\parlegdimof{\parenumerator}]$ and parametrize the choice of the $\selindexof{\parenumerator}$ for the placeholder $\placeholderof{\parenumerator}$ by unit vectors
%	\[ \unitvectoratof{\parenumerator}{\selindexof{\parenumerator}} \in \rr^{\parlegdimof{\parenumerator}} \, . \]
%We thus have a parameter space $\parameterspace$ parametrizing the possible assignments to the skeleton in its basis vectors.

\begin{figure}[h]
\begin{center}
	\begin{tikzpicture}[scale=0.45, yscale=1.2, thick] % , baseline = -3.5pt

\begin{scope}[shift={(-15,0)}]

\node[anchor=center] (text) at (-3,6) {${a)}$};

	\node [circle, draw, thick, fill=gray!50] (T1) at (0,0) {\tiny $\atomicformulaof{0}$};
	\node [circle, draw, thick, fill=gray!50] (T2) at (3,0) {\tiny $\atomicformulaof{1}$};
	\node [circle, draw, thick, fill=gray!50] (T3) at (6,0) {\tiny $\atomicformulaof{2}$};
	\node [circle, draw, thick, fill=gray!50] (T4) at (9,0) {\tiny $\atomicformulaof{3}$};
	
	\node [circle, draw, thick, fill=gray!50] (ph) at (1.5,3) {\tiny $\headvariableof{\vselectionsymbol,0}$};%{\tiny $\placeholderof{0}$};
	\node [circle, draw, thick, fill=gray!50] (sel) at (-1.5,3) {\tiny $\selvariableof{\vselectionsymbol,0}$};
	
	\node [circle, draw, thick, fill=gray!50] (ph2) at (6,3) {\tiny  $\headvariableof{\vselectionsymbol,1}$};
	\node [circle, draw, thick, fill=gray!50] (sel2) at (9,3) {\tiny $\selvariableof{\vselectionsymbol,1}$};	
	

	\node [circle, draw, thick, fill=gray!50] (sel3) at (0.25,6) {\tiny $\selvariableof{\cselectionsymbol}$};
	\node [circle, draw, thick, fill=gray!50] (head) at (3.25,6) {\tiny $\headvariableof{\lneuron}$};

	\coordinate (S3) at (3.25,4.5);
	\draw[->] (S3) -- (head);
	\draw [->] (sel3) -- (S3);
	\draw [->] (ph2) -- (S3);
	\draw [->] (ph) -- (S3);
	
	\coordinate (S1) at (1.5,1.5);
	\draw[->] (S1) -- (ph);
	\draw [->] (T1) -- (S1);
	\draw [->] (T2) -- (S1);
	\draw [->] (T3) -- (S1);	
	
	\coordinate (S2) at (6,1.5);
	\draw[->] (S2) -- (ph2);
	\draw [->] (sel) -- (S1);		
	\draw [->] (sel2) -- (S2);

	\draw [->] (T2) -- (S2);
	\draw [->] (T3) -- (S2);	

	\draw [->] (T4) -- (S2);

%	\draw [->] (ph) -- (head);			
\end{scope}


\node[anchor=center] (text) at (-3,6) {${b)}$};

\draw (-1,1) rectangle (4, 4.5);
\node[anchor=center] (text) at (1.5,2.75) {\small $\rencodingof{\lneuron}$}; %{\small $\ftensorof{\placeholderof{0} \placeholderof{1} \placeholderof{2}}$};
\draw[->] (1.5,4.5)--(1.5,6) node[midway,right] {\tiny $\headvariableof{\lneuron}$}; %{\tiny $\catvariableof{\placeholderof{0} \placeholderof{1} \placeholderof{2}}$};

\draw[<-] (4,4)--(5.5,4) node[midway,above] {\tiny $\selvariableof{\vselectionsymbol,0}$}; 
\draw[<-] (4,2.75)--(5.5,2.75) node[midway,above] {\tiny $\selvariableof{\vselectionsymbol,1}$}; 
\draw[<-] (4,1.5)--(5.5,1.5) node[midway,above] {\tiny $\selvariableof{\cselectionsymbol}$}; 

\draw[<-] (-0.5,1)--(-0.5,-0.5) node[midway,left] {\tiny $\catvariableof{0}$}; 
\draw[<-] (0.75,1)--(0.75,-0.5) node[midway,left] {\tiny $\catvariableof{1}$}; 
\draw[<-] (2,1)--(2,-0.5) node[midway,left] {\tiny $\catvariableof{2}$}; 
\draw[<-] (3.25,1)--(3.25,-0.5) node[midway,left] {\tiny $\catvariableof{3}$}; 


\node[anchor=center] (text) at (6.5,2.75) {${=}$};


\draw (7.5,1) rectangle (10.5,2.5);
\node[anchor=center] (text) at (9,1.75) {\small $\selectorcoreof{0}$};
\draw[->] (9,2.5)--(9,3.5) node[midway,left] {\tiny $\headvariableof{\vselectionsymbol,0}$};

\draw[<-] (7.75,1)--(7.75,-0.5) node[midway,left] {\tiny $\catvariableof{0}$}; 
\draw[<-] (9,1)--(9,-0.5) node[midway,left] {\tiny $\catvariableof{1}$}; 
\draw[<-] (10.25,1)--(10.25,-0.5) node[midway,left] {\tiny $\catvariableof{2}$}; 
\draw[<-] (10.5,1.75)--(11.5,1.75) node[midway,above] {\tiny $\selvariableof{\vselectionsymbol,0}$}; 

\draw (12.5,1) rectangle (15.5,2.5);
\node[anchor=center] (text) at (14,1.75) {\small $\selectorcoreof{1}$};
\draw[->] (14,2.5)--(14,3.5) node[midway,left] {\tiny $\headvariableof{\vselectionsymbol,1}$};

\draw[<-] (12.75,1)--(12.75,-0.5) node[midway,left] {\tiny $\catvariableof{1}$}; 
\draw[<-] (14,1)--(14,-0.5) node[midway,left] {\tiny $\catvariableof{2}$}; 
\draw[<-] (15.25,1)--(15.25,-0.5) node[midway,left] {\tiny $\catvariableof{3}$}; 
\draw[<-] (15.5,1.75)--(16.5,1.75) node[midway,above] {\tiny $\selvariableof{\vselectionsymbol,1}$}; 

\draw (8.5,5) rectangle (14.5,3.5);
\node[anchor=center] (text) at (11.5,4.25) {\small $\rencodingof{\cselectionmap}$};
\draw[->] (11.5,5)--(11.5,6) node[midway,left] {\tiny $\headvariableof{\lneuron}$};  %{\tiny $\catvariableof{\placeholderof{0} \placeholderof{1} \placeholderof{2}}$};
\draw[<-] (14.5,4.25)--(15.5,4.25) node[midway,above] {\tiny $\selvariableof{\cselectionsymbol}$}; 

\end{tikzpicture}
\end{center}
\caption{Example of a logical neuron $\lneuron$ of order $\selorder=2$.
	a) Selection and categorical variables and their interdependencies visualized in a hypergraph.
	b) Relational encoding of the logical neuron and tensor network decomposition into variable selecting and connective selecting tensors.
}
\label{fig:neuronDecomposition}
\end{figure}


\subsubsection{Formula Selecting Neural Network}

% Enhancement of the Expressivity
Single neurons have a limited expressivity, since for each choice of the selection variables they can just express single connectives acting on atomic variables.
The expressivity is extended to all propositional formulas, when allowing for networks of neurons, which can select each others as input arguments.


\begin{definition}\label{def:fsNeuralNetwork}

%	We call a graph consistent of nodes decorated by formula selecting neurons and directed edges representing the argument dependencies of the neuron on other neurons, an architecture graph.
%	An acyclic architecture graph is called a formula selecting neural network.	
%	Formula selecting neurons, which are not included by other formula selecting neurons are called output neurons and collected in the variables $\catvariableof{\larchitecture}$. 
%	A logical neural network is a collection of logical neurons, such that the network graph (nodes: neurons, edges: directed representing argument dependencies) is acyclic (a DAG).
	
	An architecture graph $\graphof{\larchitecture}=(\nodesof{\larchitecture},\edgesof{\larchitecture})$ is an acyclic directed hypergraph with nodes appearing at most once as outgoing nodes.
	Nodes appearing only as outgoing nodes are input neurons and are labeled by $\inneuronset$ and nodes not appearing as outgoing nodes are the output neurons in the set $\outneuronset$ (see Figure~\ref{fig:architectureGraph} for an example).

	Given an architecture graph $\graphof{\larchitecture}=(\nodesof{\larchitecture},\edgesof{\larchitecture})$, a \emph{formula selecting neural network} $\fsnn$ is a tensor network of logical neurons at each $\lneuron\in\nodesof{\larchitecture}/\inneuronset$, such that each neuron depends on variables $\catvariableof{\parentsof{\lneuron}}$ and on selection variables $\selvariableof{\lneuron}$.
	The collection of all selection variable is notated by $\selvariableof{\larchitecture}$.

	The activation tensor of each neuron $\lneuron\in\nodesof{\larchitecture}/\inneuronset$ is
	\begin{align*}
		\lneuractivationat{\catvariableof{\inneuronset},\selvariableof{\larchitecture}} 
		= \contractionof{
			\{\rencodingof{\tilde{\lneuron}} \, : \, \tilde{\lneuron}\in\nodesof{\larchitecture}/\inneuronset \} \cup \{\onehotmapofat{1}{\headvariableof{\lneuron}}\}
		}{\catvariableof{\inneuronset},\selvariableof{\larchitecture}} \, . 
	\end{align*}
		
	The activation tensor of the formula selecting neural network is the contraction
	\begin{align*}
		\fsnnat{\catvariableof{\inneuronset},\selvariableof{\larchitecture}} 
		= \contractionof{
			\{\rencodingofat{\lneuractivation}{\headvariableof{\lneuron},\catvariableof{\parentsof{\lneuron}},\selvariableof{\larchitecture}} \, : \, \lneuron\in\nodesof{\larchitecture}/\inneuronset \} \cup \{\onehotmapofat{1}{\headvariableof{\lneuron}} \, : \, \lneuron\in\outneuronset\}
		}{\catvariableof{\inneuronset},\selvariableof{\larchitecture}} \, . 
	\end{align*}
	
	The expressivity of a formula selecting neural network $\fsnn$ is the formula set
	\begin{align*}
		\formulasetof{\larchitecture} = \left\{ \larchitectureat{\catvariableof{\inneuronset},\indexedselvariableof{\larchitecture}}  : \selindexof{\larchitecture}\in\selstates \right\} \, . 
	\end{align*}
	
\end{definition}

% ? Extend by activation cone stuff
The activation tensor of each neuron depends in general on the activation tensor of its ancestor neurons with respect to the directed graph $\graphof{\larchitecture}$, and thus inherits the selection variables.

% Architecture graph -> Tensor Network
We notice that the architecture graph is a scheme to construct the variable dependency graph of the tensor network $\formulasetof{\larchitecture}$.
To this end, we replace each neuron $\lneuron\in\nodesof{\larchitecture}/\inneuronset$ by an output variable $\headvariableof{\lneuron}$ and further add selection variables $\selvariableof{\lneuron}$ to the directed edges, that is to each directed hyperedge $(\{\lneuron\}, \parentsof{\lneuron})\in\edgesof{\larchitecture}$ we construct a directed hyperedge $(\{\headvariableof{\lneuron}\}, \catvariableof{\parentsof{\lneuron}}\cup\selvariableof{\lneuron})$.

\begin{figure}[h]
\begin{center}
	\begin{tikzpicture}[scale=0.45, yscale=1.2, thick] % , baseline = -3.5pt


	\node [circle, draw, thick, fill=gray!50] (T1) at (0,0) {\tiny $\lneuronof{0}$};
	\node [circle, draw, thick, fill=gray!50] (T2) at (3,0) {\tiny $\lneuronof{1}$};
	\node [circle, draw, thick, fill=gray!50] (T3) at (6,0) {\tiny $\lneuronof{2}$};
	\node [circle, draw, thick, fill=gray!50] (T4) at (9,0) {\tiny $\lneuronof{3}$};
	
	\node [circle, draw, thick, fill=gray!50] (ph) at (1.5,3.5) {\tiny $\lneuronof{4}$};
	\node [circle, draw, thick, fill=gray!50] (ph2) at (6,2.5) {\tiny  $\lneuronof{5}$};	
	
	\node [circle, draw, thick, fill=gray!50] (head) at (3.25,6) {\tiny $\lneuronof{6}$};
	\node [circle, draw, thick, fill=gray!50] (head2) at (0,6) {\tiny $\lneuronof{7}$};

	\draw[->] (ph) -- (head2);

	\coordinate (S3) at (3.25,4.5);
	\draw[->] (S3) -- (head);
	%\draw [->] (sel3) -- (S3);
	\draw [->] (ph2) -- (S3);
	\draw [->] (ph) -- (S3);
	
	\coordinate (S1) at (1.5,1.5);
	\draw[->] (S1) -- (ph);
	\draw [->] (T1) -- (S1);
	\draw [->] (T2) -- (S1);
	\draw [->] (T3) -- (S1);	
	\draw [->] (ph2) -- (S1);	
	
	\coordinate (S2) at (6,1.5);
	\draw[->] (S2) -- (ph2);
	%\draw [->] (sel) -- (S1);		
	%\draw [->] (sel2) -- (S2);

	\draw [->] (T2) -- (S2);
	\draw [->] (T3) -- (S2);	

	\draw [->] (T4) -- (S2);



\end{tikzpicture}
\end{center}
\caption{Example of an architecture graph $\graphof{\larchitecture}$ with input neurons $\inneuronset=\{\lneuronof{0},\lneuronof{1},\lneuronof{2},\lneuronof{3}\}$ and output neurons $\outneuronset=\{\lneuronof{6},\lneuronof{7}\}$
}
\label{fig:architectureGraph}
\end{figure}


\begin{theorem}
	Given fixed selection variables $\selvariableof{\larchitecture}$, the formula selecting neural network is the conjunction of output neurons, that is
	\begin{align*}
		\fsnnat{\catvariableof{\inneuronset},\selvariableof{\larchitecture}} = \bigwedge_{\lneuron\in\outneuronset} \lneuronat{\catvariableof{\inneuronset},\selvariableof{\larchitecture}} \, . 
	\end{align*}
\end{theorem}
\begin{proof}
	By effective calculus (see Theorem~\ref{the:effectiveConjunction}), we have
		\[ \sbcontractionof{\rencodingofat{\land}{\catvariableof{\land},\shortcatvariables},\onehotmapofat{1}{\catvariableof{\land}}}{\shortcatvariables} = \bigotimes_{\catenumeratorin} \onehotmapofat{1}{\catvariableof{\catenumerator}} \]
	and thus
	\begin{align*}
		\fsnnat{\catvariableof{\inneuronset},\selvariableof{\larchitecture}}
		= \contractionof{
			\{\rencodingof{\lneuron} \, : \, \lneuron\in\nodesof{\larchitecture}/\inneuronset \} \cup \{\rencodingofat{\land}{\catvariableof{\land},\headvariableof{\lneuron}  \, : \, \lneuron\in\outneuronset}, \onehotmapofat{1}{\catvariableof{\land}}\}
		}{\catvariableof{\inneuronset},\selvariableof{\larchitecture}} \, . 
	\end{align*}
\end{proof}


% Combination of decompositions
By the commutation of contractions, we can further use Theorem~\ref{the:neuronDecomposition} to decompose each tensor $\rencodingof{\lneuron}$ into connective and variable selecting components to get a sparse representation of a formula selecting neural network $\fsnn$.

%% Now as the definition!
%\begin{theorem}{Decomposition of formula selecting neural networks}\label{the:architectureDecomposition}
%	We have
%		\[ \rencodingof{\larchitecture} = \contractionof{\{\rencodingof{\lneuron} \, : \, \lneuron \in \larchitecture\}}{\catvariableof{\larchitecture},\shortcatvariables,\selvariableof{\larchitecture}} \]
%\end{theorem}
%\begin{proof}
%	By composition Theorem~\ref{the:compositionByContraction}.
%	%\red{In addition: $X_{\larchitecture}$ specifying the headneurons! }
%\end{proof}

%% Now as the definition!
%% Relation between $\lneuron$ and $\rencodingof{\larchitecture}$
%Another useful property of encoded formula selecting architecture, is that we can retrieve any neuron by a simple contraction, as we show next.
%
%\begin{theorem}\label{the:formulaRetrieval}
%	Any neuron $\lneuron\in\larchitecture$ is retrieved by the contraction 
%		\[ \lneuron = \contractionof{\rencodingof{\larchitecture},\onehotmapof{1}[X_{\lneuron}]}{X\cup Z} \, . \]
%\end{theorem}
%\begin{proof}
%	First use the head neutralization property (Corollary~\ref{cor:onesHead}) in a parent stripping argument.
%	Then we are left with an architecture with $\lneuron$ being the only output neuron and use Corollary~\ref{cor:rhoToNormal} (we have $\restrictionofto{\mathrm{Id}}{[2]}=\onehotmapof{1}$).
%\end{proof}

% Alternative: Headneuron retrieval
%In case of multiple output neurons, the retrieval needs to be performed separately as in Theorem~\ref{the:formulaRetrieval}, since contracting basis vectors $\onehotmapof{1}$ at multiple output neurons will retrieve the conjunction of those output neurons.




%\subsubsection{Skeleton Expressions}
%
%When only allowing for argument selections at the leaf level of the network, we get a skeleton expression.
%
%\begin{definition}\label{def:skeleton}
%	A skeleton expression
%		\[ \skeleton(\placeholderof{0},\ldots,\placeholderof{\parorder-1}) \] 
%	is a composition of atom and connective selector maps, which are denoted by placeholders $\placeholderof{\parenumerator}$, where $\parenumerator\in[\parorder]$..
%	Each placeholder has a by $\selindexof{\parenumerator}$ enumerated list $\candidatelistof{\parenumerator}$ with cardinality $\parlegdimof{\parenumerator}= \cardof{\candidatelistof{\parenumerator}}$ of possible symbols denoting atoms or connectives to be placed in at this position.
%	This defines a map
%		\[ \skeleton : \left(\facstates\right) \times \left(\secfacstates\right) \rightarrow \{0,1\} \]
%	where $\skeleton(\atomindices,\parindices)$ denotes the formula given the selection of placeholders by $\parindices$, which is evaluated at the atoms $\atomindices$.
%\end{definition}

%\begin{definition}
%	A skeleton expression is a formula
%		\[ \skeleton(\placeholderof{0},\ldots,\placeholderof{\parorder-1}) \]
%	where instead of atoms and connectives there are placeholders $\placeholderof{\parenumerator}$, where $\parenumerator\in[\parorder]$.
%	Each skeleton has for each placeholder $\placeholderof{\parenumerator}$ a set $\candidatelistof{\parenumerator}$ of candidate atoms to be plugged in the placeholders.
%	We denote its cardinality to be $\parlegdimof{\parenumerator}= \cardof{\candidatelistof{\parenumerator}}$ and enumerate the elements $\placeholderof{\parenumerator}_{\selindexof{\parenumerator}}$ in each candidates list by an index $\selindexof{\parenumerator}\in[\parlegdimof{\parenumerator}]$.
%\end{definition}






%% ANOTHER EXAMPLE:





%\begin{definition}
%	Given a skeleton expression, the skeleton tensor is the map from the parameter space to the space of formula tensors, defined by
%	\begin{align}
%		 \skeletontensor : \parameterspace \rightarrow  \modelspace \quad , \quad
%		 \skeletontensor\left( \bigotimes_{\parenumeratorin}\unitvectoratof{\parenumerator}{\selindexof{\parenumerator}} \right) = \ftensorof{\skeletonof{\placeholderof{\parenumerator}_{\selindexof{\parenumerator}}\, : \, \parenumeratorin}}
%	\end{align}
%\end{definition}
%
%Using the canonical duality of tensors as maps and elements of tensor spaces, we can reinterpret is as a tensor \red{Domain representation of skeleton map}
%	\[ \skeletontensor \in \bigotimes_{\parenumerator\in[\parorder]} \rr^{\parlegdimof{\parenumerator}} \otimes  \modelspace  \, , \]
%which is the superposed formula tensor to a skeleton based parametrization.
%In the following, we investigate how to efficiently represent the skeleton tensor $\ftensorof{\skeleton}$ as a tensor network.









\subsection{Application of Formula Selecting Networks}

There are two main applications of formula selecting networks.
First, when contracting the selection variables with a weight tensor we get a weighted sum of the parametrized formulas.
Second, when contracting the categorical variables with a distribution or a knowledge base, we get a tensor storing the satisfaction rates respectively the world counts of the parametrized formulas.

\subsubsection{Representation of selection encodings}

\red{In technical perspective: FSN provide efficient representation of $\sencodingof{\formulaset}$ 
-> Use for exponential families, structure learning.}

\begin{lemma}\label{lem:relToSelFSN}
	Given a set $\{\formulaof{\selindexlist} : \selindexlist\in\selstates\}$ of propositional formulas we define the statistic
		 \[ \formulaset : \catindices \rightarrow (\formulaof{\selindexlist}(\catindices))_{\selindexlist} \, . \]
	and the formula selecting map
		\[ \fselectionmap: \catindices , \selindexlist \rightarrow \formulaof{\selindexlist} (\catindices) \, . \]
	Then 
		\[ \sencodingofat{\formulaset}{\shortcatvariablelist, \shortselvariablelist} = \fselectionmap\left[\shortcatvariablelist, \shortselvariablelist \right] \, .  \]
		%\[ \sencodingofat{\formulaset}{\shortcatvariablelist, \shortselvariablelist}
		 %= \sbcontractionof{\rencodingofat{\fselectionmap}{\headvariableof{\fselectionmap}, \shortcatvariablelist, \shortselvariablelist}
		 %, \onehotmapofat{1}{\headvariableof{\fselectionmap}}}{\shortcatvariablelist, \shortselvariablelist} \, . \]
\end{lemma}
\begin{proof}
	For any indices $\shortselindices\in\selstates$ and $\shortcatindices\in\atomstates$ we have
	\begin{align*}
		\sencodingofat{\formulaset}{\shortcatvariablelist=\shortcatindices, \shortselvariablelist=\shortselindices}
		=  \formulaof{\selindexlist}(\catindices) =  \fselectionmap\left[\shortcatvariablelist=\shortcatindices, \shortselvariablelist=\shortselindices \right] \, . 
	\end{align*}
\end{proof}

%% Reason for relational encodings and selection encodings.
Technically, relational encodings have been exploited to derive decompositions based on basis calculus.
Selection encodings on the other hand enable the application of formula selecting networks as superpositions of formulas.



\subsubsection{Efficient Representation of Formulas}

\red{Weight contracted at the selection variables is elementary, then single formula retrieved.}

% Exponentially many formulas represented by linear demand
Formula Selecting Neural Networks are means to represent exponentially many formulas with linear (in sum of candidates list lengths) storage.
Their contraction with probability tensor networks, is thus a batchwise evaluation of exponentially many formulas.
This is possible due to redundancies in logical calculus due to modular combinations of subformulas.

% Retrieve functions
We can retrieve specific formulas by slicing the selection variables, i.e. for $\parindices$ we have
	\[ \exformula_{\parindices}[\shortcatvariables] = \fselectionmapat{\shortcatvariables,\selvariable=\parindices} \, .  \]

In a tensor network diagram we depict this by
\begin{center}
	\begin{tikzpicture}[thick, scale=0.35] % , baseline = -3.5pt

\node[anchor=east] (text) at (-3,0) {${\exformula_{\parindices}} \quad\quad {=}$};

\drawatomindices{0}{-4}
\draw (-1,3) rectangle (5, -3);
\node[anchor=center] (text) at (2,0) {$\rencodingof{\fselectionmap}$};

\draw[->] (2,3)--(2,5) node[midway,right] {\tiny $\catvariableof{\fselectionmap}$}; 
\draw (1,5) rectangle (3,7);
\node[anchor=center] (text) at (2,6) {$\tbasis$};

\draw[<-] (5,-2.25)--(7,-2.25) node[midway,below] {\tiny $\selvariableof{0}$}; 
\draw[<-] (5,-0.5)--(7,-0.5) node[midway,below] {\tiny $\selvariableof{1}$}; 
\node[anchor=center] (text) at (6,1) {$\vdots$};
\draw[<-] (5,2.25)--(7,2.25) node[midway,above] {\tiny $\selvariableof{\parorder\shortminus1}$}; 

\draw (7,1.5) rectangle (9,3);
\node[anchor=center] (text) at (8,2.25) {$\onehotmapof{\parindexof{\parorder\shortminus1}}$};

\draw (7,0.25) rectangle (9,-1.25);
\node[anchor=center] (text) at (8,-0.5) {$\onehotmapof{\parindexof{1}}$};

\draw (7,-1.5) rectangle (9,-3);
\node[anchor=center] (text) at (8,-2.25) {$\onehotmapof{\parindexof{0}}$};

\end{tikzpicture}
\end{center}

% Interpretation by dynamic programming
Another perspective on the efficient formula evaluation by selection tensor networks is dynamic computing.
Evaluating a formula requires evaluations of its subformulas, which are done by subcontractions and saved for different subformulas due to the additional selection legs.

% Storage problem of solutions
However, we need to avoid contracting the tensor with leaving all selection legs open, since this would require exponential storage demand.

% Sparse algorithm
We can avoid this storage bottleneck by extending the contractions by additional cores leaving less variable legs open.
This is the case when contracting gradients of the parameter tensor networks in alternating least squares approaches.
Other methods avoiding the bottleneck can be constructed by MCMC sampling, for example Gibbs Sampling.
Here we only need to vary local components of the formula reflected in keeping only single variable legs open.



\subsubsection{Batch contraction of parametrized formulas}

Given a set $\formulaset$ of formulas, we build a formula selecting network parametrizing the formulas.
The contraction 
\begin{align*}
	\contractionof{\extnet,\fselectionmap}{\shortselvariables} 
\end{align*}
is a tensor containing the contractions of the formulas $\formulaof{\shortselindices}$ with an arbitrary tensor network $\extnet$ as
\begin{align*}
	\sbcontraction{\extnet,\formulaof{\shortselindices}} = \sbcontractionof{\extnet,\fselectionmap}{\shortselvariables=\shortselindices} \, . 
\end{align*}


\subsubsection{Average contraction of parametrized formulas}

We show in the next two examples, how a full contraction of the formula selecting map with a probability distribution or a knowledge base can be interpreted.

\begin{example}[Average satisfaction of formulas]
	The average of the formula satisfactions in $\formulaset$ giben a probability tensor $\probtensor$ is 
		\[ \frac{1}{\prod_{\selenumeratorin}\seldimof{\selenumerator}} \cdot \sbcontraction{\probtensor,\sencodingof{\formulaset}} \, . \]
\end{example}


\begin{example}[Deciding whether any formula is not contradicted]
	For example: We want to decide, whether there is a formula in $\formulaset$ not contradicted by a Knowledge base $\kb$.
	This is the case if and only if 
		\[ \sbcontraction{\kb,\sencodingof{\formulaset}} = 0 \, .  \]
	We use Lemma~\ref{lem:relToSelFSN} to get that $\sencodingof{\formulaset}=\fselectionmap$.
	When the formulas are representable in a folded scheme, we find tensor network decompositions of $\fselectionmap$ and exploit them along efficient representations of $\kb$ in an efficient calculation of $\sbcontraction{\kb,\sencodingof{\formulaset}} $.
	This is further equal to 
		\[ \kb \models \lnot \left( \bigvee_{\exformula\in\formulaset} \exformula\right) \, . \]
\end{example}



%\subsubsection{Neuro-Symbolic Architectures}
%
%%% Neuro-Symbolic Architecture
%We understand selector tensor networks as a neuro-symbolic architecture, where the selector variables are understood as parameters and the processed variables as neural activation variables.
%The orientation of the tensor network organizes the variables in layers.




\subsection{Examples of formula selecting neural networks}

%Here we provide examples of exponential families
%\red{See further: Chapter~\ref{cha:energyRepresentation} studying the exact representation of energy as a weighted superposition of formulas.}

%\subsubsection{Boltzmann Machines}
%The inverse is the representation of a given distribution by a Boltzmann machine?


\subsubsection{Correlation}


For example (see Figure \ref{fig:AndSupFTDecomposition}) consider the logical neuron with single activation candidate $\{\land\}$ and two variable selectors selecting $\catorder$ atomic variables $\shortcatvariables$.
The expressivity of this network is the set of all conjunctions of the atoms
	\[ \{\catvariableof{\atomenumerator} \land \catvariableof{\secatomenumerator} \, : \, \atomenumerator,\secatomenumerator\in[\atomorder] \} \]


% Covariance measure
Contracting with a probability distribution, we use the tensor
	\[ \hypercoreat{\selvariableof{\vselectionsymbol,0},\selvariableof{\vselectionsymbol,1}} = \sbcontractionof{\fsnn}{\selvariableof{\vselectionsymbol,0},\selvariableof{\vselectionsymbol,1}} \]
to read of covariances as
	\[ \mathrm{Cov}(\catvariableof{\atomenumerator},\catvariableof{\secatomenumerator}) = \hypercoreat{\selvariableof{\vselectionsymbol,0}=\atomenumerator,\selvariableof{\vselectionsymbol,1}=\secatomenumerator}  - 
	\hypercoreat{\selvariableof{\vselectionsymbol,0}=\atomenumerator,\selvariableof{\vselectionsymbol,1}=\atomenumerator}  \cdot \hypercoreat{\selvariableof{\vselectionsymbol,0}=\secatomenumerator,\selvariableof{\vselectionsymbol,1}=\secatomenumerator} \, .  \]
	
	
	

%	\[ \skeleton = \placeholderof{1} \land \placeholderof{2} \]
%with the candidates for each placeholder being a set of $\atomorder$ atoms.

\begin{figure}[h]
\begin{center}
	\begin{tikzpicture}[thick, scale=0.35] % , baseline = -3.5pt

\drawatomindices{0}{-4}
\draw (-1,-1) rectangle (5, -3);
\node[anchor=center] (text) at (2,-2) {$\rencodingof{\atomicformulaof{\parindexof{1}} \land \atomicformulaof{\parindexof{2}}}$};

\draw[->] (2,-1)--(2,1) node[midway,right] {\tiny ${\headvariableof{\parindexof{1}} \land \headvariableof{\parindexof{2}}}$};

\draw[<-] (5,-2.5)--(7,-2.5) node[midway,below] {\tiny $\fselectionvariable_{1}$}; 
\draw[<-] (5,-1.5)--(7,-1.5) node[midway,above] {\tiny $\fselectionvariable_{0}$}; 

\draw (7,-1) rectangle (9, -3);
\node[anchor=center] (text) at (8,-2) {$\canparam$};


		
\node[anchor=center] (text) at (12,-2) {${=}$};


\begin{scope}[shift={(17,8)}]
	\begin{scope}[shift={(0,-10)}]

		\draw[->] (5.5,5) -- (5.5,7) node[midway, right] {\tiny ${\headvariableof{\parindexof{1}} \land \headvariableof{\parindexof{2}}}$};
		\draw (1,3) rectangle (10, 5);
		\node[anchor=center] (text) at (5.5,4) {$\rencodingof{\land}$};

			
		% SelectorCores
		\draw[->] (2,1) -- (2,3) node[midway, left] {\tiny ${\headvariableof{\parindexof{2}}}$};
		\draw (-1,1) rectangle (5, -1);
		\node[anchor=center] (text) at (2,0) {$\selectorcoreof{2}$};
		\draw (5,0) -- (12,0);
		\draw[<-] (12,0) -- (14,0) node[midway, above] {\tiny $\fselectionvariable_{1}$};
		\begin{scope}[shift={(0,-2)}]
			\draw[<-] (0,1)--(0,-3) node[midway,left] {\tiny $\catvariableof{0}$}; 
			\draw[<-] (1.5,1)--(1.5,-3) node[midway,left] {\tiny $\catvariableof{1}$}; 
			\node[anchor=center] (text) at (3,0) {$\cdots$};
			\draw[<-] (4,1)--(4,-3) node[midway,right] {\tiny $\catvariableof{\atomorder\shortminus1}$}; 
		\end{scope}


		\draw (9,-1) -- (9,1);
		\draw[->] (9,1) -- (9,3) node[midway, left] {\tiny ${\headvariableof{\parindexof{1}}}$};
		\draw (6,-3) rectangle (12, -1);
		\node[anchor=center] (text) at (9,-2) {$\selectorcoreof{1}$};
		\draw[<-] (12,-2) -- (14,-2) node[midway, above] {\tiny $\fselectionvariable_{0}$};
		\drawatomindices{7}{-4}	
		
		% ParameterCores
		\draw (14,1) rectangle (16, -3);
		\node[anchor=center] (text) at (15,-1) {$\canparam$};
		
		\begin{scope}[shift={(-3.5,8)}]
			\draw[fill] (7.5,-15) circle (0.25cm);
			\draw[] (7.5,-15) to[bend left=25] (3.5,-13);
			\draw[] (7.5,-15) to[bend right=25] (10.5,-13);

			\draw[fill] (9,-15.25) circle (0.25cm);
			\draw[] (9,-15.25) to[bend left=25] (5,-13);
			\draw[] (9,-15.25) to[bend right=25] (12,-13);

			\draw[fill] (11.5,-15) circle (0.25cm);
			\draw[] (11.5,-15) to[bend left=25] (7.5,-13);
			\draw[] (11.5,-15) to[bend right=25] (14.5,-13);

			\drawatomindices{7.5}{-16}

		\end{scope}

	\end{scope}
\end{scope}

\end{tikzpicture}
\end{center}
\caption{Superposition of the encoded formulas $\rencodingof{\atomicformulaof{\selindexof{1}} \land \atomicformulaof{\selindexof{2}}}$ with weight $\canparam_{\selindexof{1} \selindexof{2}}$}
\label{fig:AndSupFTDecomposition}
\end{figure}



\subsubsection{Conjunctive and Disjunctive Normal Forms}%\label{sec:CNFasFormulaSelection}

% Architecture
We can represent any propositional knowledge base by the following scheme:
Literal selecting neurons by connective identity/negation (selecting positive/negative literal) and selecting one of the atoms.
Single output neuron representing the disjunction combining the literal selecting neurons.
Number of neurons defined by the maximal clause size plus one.
Smaller clauses can be covered when adding False as a possible choice (The respective neuron has to choose the identity, otherwise the full clause will be trivial).

% Parameter
The parameter core is in the basis CP format and each slice selects a clause of the knowledge base.
When taking the slice values to infinity (e.g. by an annealing procedure), the represented member of the exponential family converges to the uniform distribution of the models of the knowledge base.


\red{Useful to derive basis+ CP format based on CNF!}


% Representation by selection tensor networks
\begin{remark}[Minterms and Maxterms]
	All minterms and maxterms can be represented by a two layer selection tensor networks without variable selection in two layers.
	The bottom layer has an $\lnot/\mathrm{Id}$ connective selection neuron to each atom and the upper layer consists of a single $\atomorder$ary conjunction.
\end{remark}




\subsection{Extension to variables of larger dimension}

Connective selecting tensors: Can encode arbitrary functions $h_{\selindex}$ of discrete variables, but need $\catvariableof{\cselectionmap}$ to be an enumeration of the, in particular to be of dimension
	\[ \catdimof{\cselectionmap} = \cardof{ \cup_{\selindexin} \imageof{h_{\selindex}} } \, . \]
	
Variable selecting tensors can be understood as specific cases of connective selecting tensors and can thus also be generalized in a straight forward manner by
	\[ \catdimof{\cselectionmap} = \cardof{ \cup_{\selindexin} \imageof{h_{\selindex}} } \, .  \]
	
State selecting tensors are directly 


\begin{example}[Discretization of a continuous neuron]
	Let there be a neuron
		\[ \sigma( w, y) : \rr \times \rr^{\catorder} \rightarrow \rr \, .\]
	When $w \in \arbsetof{weight}\subset \rr^{\catorder}$ and $x \in \arbsetof{x}\subset \rr^{\catorder}$ have
		\[ \cardof{\sigma( \contractionof{w, x} )} \leq \cardof{\arbsetof{weight}} \cdot \cardof{\arbsetof{x}} \, . \] 

	To represent the discretization of the neuron, we use the subset encoding scheme of Definition~\ref{def:subsetEncoding}.
	The variables $\indvariableof{weight}$ indexing $\arbsetof{weight}$ will be understood as selection incoming variables and the variables $\indvariableof{weight}$ indexing $\arbsetof{weight}$ as categorical incoming variables.
	We further define a variable $\indvariableof{\sigma}$ indexing $\imageof{\restrictionofto{\sigma}{\arbsetof{weight}\times\arbsetof{x}}}$ and have a tensor
		\[ \rencodingofat{\sigma}{\indvariableof{\sigma},\indvariableof{\arbsetof{x}},\indvariableof{weight}} \, . \] 

	If the neuron is of the form
		\[ \sigma(w,x) = \psi(\sum_i w_i \cdot x_i)\]
	a decomposition into multiplication at each coordinate and summation of the results, with relational encodings for each, can be done.
\end{example}


% Representation
\section{Representing Logic Networks}

Markov Logic Networks exploit the efficiency and interpretability of logical calculus as well as the expressivity of graphical models. 

%Markov Logic Networks are probability functions of truth assignments to logical functions.
%They respect propositional logic as hard constraints, but have beyond that freedom to shape probability distributions on possible situations.
%To capture these properties, we define them as graphical models with structure cores representing propositional logics and activation cores representing the specification of probability distributions.
% We in this part employ them to combine the probabilistic and the logical paradigm.



\subsection{Markov Logic Networks as Exponential Families}

We introduce Markov Logic Networks in the formalism of exponential families (see Section~\ref{sec:exponentialFamilies}).

\begin{definition}[Markov Logic Networks]
	Markov Logic Networks are exponential families $\mlnexpfamily$ with sufficient statistics by functions
		\[ \formulaset : \atomstates \rightarrow \bigtimes_{\exformulain}[2] \subset \rr^{\cardof{\formulaset}} \]
	defined coordinatewise by propositional formulas $\exformulain$.
\end{definition}

% Binary Statistics as propositional formula
Since the image of each feature is contained in $[2]$, they are propositional formulas (see Def.~\ref{def:formulas}).

% Characterization of MLNs among exponential families: When choosing binary features
Conversely, any binary feature $\sstatcoordinateof{\statenumerator}$ of an exponential family defines a propositional formula (see Definition~\ref{def:formulas}).
Thus, any exponential family of distributions of $\atomstates$, such that $\imageof{\sstatcoordinateof{\statenumerator}}\subset\{0,1\}$ for all $\statenumeratorin$ is a set of Markov Logic Networks with fixed formulas.

% Formula Selecting Networks
%We will further study the sparse representation of formula sets in Chapter~\ref{cha:architectures}.


The sufficient statistics consistent in a map $\formulaset$ of formulas brings the following advantages:
\begin{itemize}
	\item Numerical Advantage: The sufficient statistics is decomposable into logical connectives. 
	If the formulas are sparse (in the sense of limited number of connectives necessary in their representation), this gives rise to efficient tensor network decompositions of the relational encoding.
	\item Statistical Advantage: Since each formula is Boolean valued, the coordinates of the sufficient statistic are Bernoulli variables. 
	Due to their boundedness, they and their averages (by Hoeffdings inequality) are sub-Gaussian variables with favorable concentration properties (absence of heavy tails).
\end{itemize}


\begin{remark}[Alternative Definitions]
	We here defined MLNs on propositional logic, while originally they are defined in FOL.
	The relation of both frameworks will be discussed further in Chapter~\ref{cha:folModels}.
	Besides that we allow for degeneracy and improper MLNs.
\end{remark}


%\subsubsection{Interpretation as Graphical Models}






\subsection{Tensor Network Representation of MLNs}

Based on the previous discussion on the representation of exponential families by tensor networks in Section~\ref{sec:exponentialFamilies} we now derive a representation for Markov Logic Networks.

\begin{theorem}[Relational Encodings for Markov Logic Networks]
	A Markov Logic Network to a set of formulas $\formulaset = \{\formulaof{\selindex} \, : \, \selindexin\}$ is represented as
	\begin{align*}
		\mlnprobat{\shortcatvariables} = 
		\normationof{
			\{\rencodingofat{\formulaof{\selindex}}{\catvariableof{\formulaof{\selindex}},\shortcatvariables} : \selindexin \} 
			\cup \{ 
			\headcoreofat{\formulaof{\selindex},\canparamat{\selvariable=\selindex}}{\catvariableof{\formulaof{\selindex}}} 
			: \selindexin \}
		}{\shortcatvariables}
	\end{align*}
	where we denote for each $\selindexin$ a head core
	\begin{align*}
		\headcoreofat{\formulaof{\selindex},\canparamat{\indexedselvariableof{}}}{\catvariableof{\formulaof{\selindex}}} 
		= \begin{bmatrix} 1 & \expof{\canparamat{\indexedselvariableof{}}} \end{bmatrix}[\catvariableof{\formulaof{\selindex}}] \, .
	\end{align*}
\end{theorem}
\begin{proof}
	The claim follows from Theorem~\ref{def:expFamilyTensorRep} and the following contraction equations.
	We have with the grouped variable $\catvariableof{\formulaset} = \{\catvariableof{\formulaof{\selindex}}\, : \, \selindexin\}$
	\begin{align*}
		\rencodingofat{\formulaset}{\shortcatvariables,\catvariableof{\formulaset}}
		= \contractionof{\{\rencodingofat{\formulaof{\selindex}}{\catvariableof{\formulaof{\selindex}},\shortcatvariables} : \selindexin \}}{\shortcatvariables,\catvariableof{\formulaset}} \, .
	\end{align*}
	Since we have a Markov Logic Network we have $\imageof{\formulaof{\selindex}}\subset [2]$ and thus
	\begin{align*}
	 	\headcoreofat{\formulaof{\selindex},\canparamat{\indexedselvariableof{}}}{\indexedcatvariableof{\formulaof{\selindex}}} 
		= \begin{cases}
			1 & \text{for} \quad \catvariableof{\formulaof{\selindex}} = 0 \\
			\expof{\canparamat{\indexedselvariableof{}}} & \text{for} \quad \catvariableof{\formulaof{\selindex}} = 1
		\end{cases}  
	\end{align*}
	Using these equations, the claim follows from Theorem~\ref{def:expFamilyTensorRep}.
\end{proof}

\begin{figure}[h]
\begin{center}
	\begin{tikzpicture}[thick, scale=0.35] % , baseline = -3.5pt

\drawundiratomindices{0}{-4}
\draw (-2,-1) rectangle (6, -3);
\node[anchor=center] (text) at (2,-2) {$\expof{\canparamat{\selvariable=\selindex}\cdot\formulaof{\selindex}}$};

		
\node[anchor=center] (text) at (10,-2) {${=}$};


\begin{scope}[shift={(15,-2)}]

		\draw (-0.5,3) rectangle (4.5, 5);
		\node[anchor=center] (text) at (2,4) {$\headcoreof{\formulaof{\selindex},\canparamat{\selvariable=\selindex}}$};

		\draw[fill] (2,2.25) circle (0.25cm);
		\draw[] (2,2.25) -- (2,3);
		\draw[->] (2,1) -- (2,2.5) node[midway, left] {\tiny $\catvariableof{\formulaof{\selindex}}$};
		
		\draw (-1,1) rectangle (5, -1);
		\node[anchor=center] (text) at (2,0) {$\rencodingof{\formulaof{\selindex}}$};

		\drawatomindices{0}{-2}

\end{scope}

\end{tikzpicture}
\end{center}
\caption{Factor of a Markov Logic Network to a formula $\formulaof{\selindex}$.}
% Where $\headcoreofat{\formulaof{\selindex}}{\catvariableof{\formulaof{\selindex}}} =\begin{bmatrix} 1 & \expof{\weightof{\exformula}} \end{bmatrix}[\catvariableof{\exformula}] $}
\label{fig:mlnFactor}
\end{figure}

% 
Since any member of an exponential family is a Markov Network with tensors to each coordinate of the statistic, also Markov Logic Networks are Markov Networks.

\begin{corollary}\label{cor:MLNasMN}
	Given a set $\formulaset$ of formulas on atomic variables $\catvariableof{\nodes}$, we construct a $\graph=(\nodes,\edges)$, where $\nodes$ are decorated by the atoms and
		\[ \edges = \{ \nodesof{\formula}: \formula\in\formulaset \} \, , \]
	where by $\nodesof{\formula}$ we denote the minimal set such that there exists a tensor $\hypercoreat{\catvariableof{\nodesof{\formula}}}$ with
		\[ \formulaat{\catvariableof{\nodes}} = \hypercoreat{\catvariableof{\nodesof{\formula}}} \otimes \onesat{\catvariableof{\nodes/\nodesof{\formula}}} \, . \]		
	Any Markov Logic Network $\mlnparameters$ is then a Markov Network given the graph $\graphof{\formulaset}$
	$\{\expof{\canparamat{\selvariable=\selindex}\cdot\formulaof{\selindex}}
\, :\,\selindexin\}$.
\end{corollary}


% MLN as graphical models
Markov Logic Networks are Markov Networks with the factors given in a restricted form from the weighted truth of a formula.
Each formula is seen as a factor of the graphical model.

There are two sparsity mechanisms drastically reducing the number of parameters (and loosing generality):
\begin{itemize}
	\item Factors/Formulas contain only subsets of atoms (already in Corollary~\ref{cor:MLNasMN} exploited):
		The underlying assumptions of conditional independence loss generality.
	\item Structure in the factors: In MLN each factor corresponds with a formula evaluated on possible worlds.
		Again, any possible factor can be represented by a formula, but we concentrate on small formulas (see Theorem \ref{the:FormulaToTensor}).
\end{itemize}


% 
\red{We can extend the set of variables, by including the hidden formulas, and get a Markov Network of the relational encodings of connectives and headcores.
Here hidden variables are additional variables facilitating the decomposition, but not appearing in open variables of contractions when doing reasoning.
One can then exploit redundancies and make sure that every subresult is computed just once, by dropping relational encodings with identical head functions.
}


\begin{figure}[h]
\begin{center}
	\begin{tikzpicture}[scale=0.35, thick, yscale=-1] % , baseline = -3.5pt

\draw (-5,-3) rectangle (1, -5);
\node[anchor=center] (text) at (-2,-4) {$\expof{\canparamat{0}\cdot\formulaof{0}}$};

\draw (3,-3) rectangle (9, -5);
\node[anchor=center] (text) at (6,-4) {$\expof{\canparamat{1}\cdot\formulaof{1}}$};


\draw[->] (0,1)--(0,-1)  node[midway,left] {\tiny $\catvariableof{a}$}; 
\drawvariabledot{0}{-1}
\draw[->] (0,-1) to[bend right=25] (-4,-3);
\draw[->] (0,-1) to[bend left=25] (4,-3);

\draw[->]  (2,1)--(2,-1) node[midway,right] {\tiny $\catvariableof{b}$}; 
\drawvariabledot{2}{-1}
\draw[->] (2,-1) to[bend right=25] (-2,-3);
\draw[->] (2,-1) to[bend left=25] (6,-3);

\draw[->] (4,1)--(4,-1) node[midway,right] {\tiny $\catvariableof{c}$}; 
\drawvariabledot{4}{-1}
\draw[->] (4,-1) to[bend right=25] (0,-3);
\draw[->] (4,-1) to[bend left=25] (8,-3);


\node[anchor=center] (text) at (12.5,-2) {${=}$};


\begin{scope}[shift={(20,0)}]

\draw[->] (0,1)--(0,-1)  node[midway,left] {\tiny $\catvariableof{a}$}; 
\draw[->]  (3,1)--(3,-1) node[midway,right] {\tiny $\catvariableof{b}$}; 
\draw[->] (6,1)--(6,-1) node[midway,right] {\tiny $\catvariableof{c}$}; 

	
\draw (-1,-1) rectangle (4, -3);
\node[anchor=center] (text) at (1.5,-2) {$\concoreof{\lor}$};

\draw[->] (1.5,-3) --(1.5,-4.5) node[midway,right]{\tiny $\catvariableof{a \lor b}$};
\drawvariabledot{1.5}{-4.5}
\draw[->] (1.5,-4.5) --(1.5,-6) ;

\draw[] (1.5,-4.5) -- (0,-4.5);

\draw[]  (-7, -3.5) rectangle (0, -6.5);
\node[anchor=center,] (text) at (-3.5,-5) {$\begin{bmatrix} 
1 \\
\expof{\canparamat{0}}
\end{bmatrix}$};


\draw (5,-1) rectangle (7, -3);
\node[anchor=center] (text) at (6,-2) {$\concoreof{\lnot}$};

\draw[->] (6,-3) --(6,-4.5) node[midway,right]{\tiny $\catvariableof{\lnot c}$};
\drawvariabledot{6}{-4.5}
\draw[->] (6,-4.5) --(6,-6);	
	
	
\draw (0.5,-6) rectangle (6.5,-8);
\node[anchor=center] (text) at (3.5,-7) {$\concoreof{\lor}$};
	
\draw[<-] (4,-9.5) -- (4,-8) node[midway,right] {\tiny $\catvariableof{a \lor b \lor \lnot c}$};

\drawvariabledot{4}{-9.5}
\draw[] (4,-9.5) -- (2.75,-9.5);

\draw[]  (-4.25, -8.5) rectangle (2.75, -11.5);
\node[anchor=center,] (text) at (-0.75,-10) {$\begin{bmatrix} 
1 \\
\expof{\canparamat{1}}
\end{bmatrix}$};



\end{scope}

\end{tikzpicture}
\end{center}
\caption{Example of a decomposed Markov Network representation of a Markov Logic Network with formulas $\{\formulaof{0} = a\lor b, \formulaof{1} = a \lor b \lor \lnot c\}$.
	Since both formulas share the subformula $a\lor b$, their contracted factors have a representation by a connected tensor network.}
% Where $\headcoreofat{\formulaof{\selindex}}{\catvariableof{\formulaof{\selindex}}} =\begin{bmatrix} 1 & \expof{\weightof{\exformula}} \end{bmatrix}[\catvariableof{\exformula}] $}
\label{fig:mlnDecRep}
\end{figure}


%\begin{theorem}[Selection encodings for Energy representation]
%	\red{More the definition of exponential families.}
%	The energy of Markov Logic Networks is the contraction
%		\[ \mlnenergy = \sbcontractionof{\sencodingof{\formulaset},\canparam}{\shortcatvariables} \, . \]
%\end{theorem}


%When the further factor cores contain only one variable, we can label them by the formula $\exformula$ corresponding with that node $[1,\expof{\weightof{\exformula}}]$.
%\begin{definition}{Markov Logic Networks}
%	Given a set of formulas $\mlnformulaset$ with weights $\weight:\mlnformulaset\rightarrow\rr$ the Markov Logic Network is the distribution
%		\[ \mlnprobat{\indexedcatvariables}= \frac{1}{\partitionfunctionof{\mlnparameters}} \expof{
%			\sum_{\mlnformulain} \exformula(\atomindices) \weightof{\exformula}
%			} \]
%	where the partition function
%		\[ \partitionfunctionof{\mlnparameters} = \sum_{\atomindicesin}  \expof{
%			\sum_{\mlnformulain} \exformula(\atomindices) \weightof{\exformula}
%		} \]
%	ensures the normation.
%\end{definition}

%where the partition function is the contraction 
%\begin{align}
%	\partitionfunctionof{\mlnparameters} = \sbcontraction{\expof{\mlnenergy}} \, . 
%\end{align}



\subsubsection{Energy tensors of Markov Logic Networks}

%% Tensor Representation of MLN
With the energy tensor
\begin{align}
	\mlnenergy 
	= \sum_{\selindexin} \canparamat{\selvariable=\selindex} \cdot \formulaofat{\selindex}{\shortcatvariables} 
	= \sbcontractionof{\sencodingofat{\formulaset}{\shortcatvariables,\selvariable},\canparamat{\selvariable}}{\shortcatvariables} 
\end{align}
the MLN is the distribution
\begin{align}
	\mlnprobat{\shortcatvariables} = \normationofwrt{\expof{\mlnenergy}}{\shortcatvariables}{\varnothing} \, . 
\end{align}

In case of a common structure of the formulas in a Markov Logic Network, Formula selecting networks can be applied to represent their energies.

% Energy representation
%The weighted sum of formulas is then the energy of the Markov Logic Network.
We represent the superposition of formulas as a contraction with s parameter tensor.
Given a factored parametrization of formulas $\exformula_{\parindices}$ with indices $\selindexof{\parenumerator}$ we have the superposition by the network representation:
\begin{center}
	\begin{tikzpicture}[thick, scale=0.35] % , baseline = -3.5pt

\node[anchor=east] (text) at (-3,0) {$\sum_{\parindexof{[\parorder]}\in\parstates} \canparamat{\selvariableof{[\parorder]}=\parindexof{[\parorder]}} {\exformula_{\parindexof{[\parorder]}}} \quad {=}$};

%\node[anchor=center] (text) at (0.5,-8) {$\mathrm{log}$};

%\drawatomcore{3.5}{-8}{$\rencodingof{\fselectionmap}$}
%\drawatomindices{3.5}{-12}	
%
%
%\drawatomcore{3.5}{-4}{$\canparam$}
%\drawparindices{3.5}{-8}	



\drawatomindices{0}{-4}
\draw (-1,3) rectangle (5, -3);
\node[anchor=center] (text) at (2,0) {$\rencodingof{\fselectionmap}$};

\draw[->] (2,3)--(2,5) node[midway,right] {\tiny $\catvariableof{\fselectionmap}$}; 
\draw (1,5) rectangle (3,7);
\node[anchor=center] (text) at (2,6) {$\tbasis$};

\draw[<-] (5,-2)--(7,-2) node[midway,below] {\tiny $\selvariableof{0}$}; 
\draw[<-] (5,-0.5)--(7,-0.5) node[midway,below] {\tiny $\selvariableof{1}$}; 
\node[anchor=center] (text) at (6,0.75) {$\vdots$};
\draw[<-] (5,2)--(7,2) node[midway,above] {\tiny $\selvariableof{\parorder\shortminus1}$}; 

\draw (7,3) rectangle (9,-3);
\node[anchor=center] (text) at (8,0) {$\canparam$};

\end{tikzpicture}
\end{center}


% Representation 
If the number of atoms and parameters gets large, it is important to represent the tensor ${\exformula_{\parindices}}$ efficiently in tensor network format and avoid contractions.
To avoid inefficiency issues, we also have to represent the parameter tensor $\canparam$ in a tensor network format to improve the variance of estimations (see Chapter~\ref{cha:mlnConcentration}) and provide efficient numerical algorithms.

% Fail of full probability representation
However, when required to instantiate the probability distribution of a Markov Logic Network as a tensor network, we need to exponentiate and normate the energy tensor, a task for which relational encodings are required.
For such tasks, contractions of formula selecting networks are not sufficient and each formula with a nonvanishing weight needs to be instantiated as a factor tensor of a Markov Network. 






\subsection{Expressivity}\label{sec:MLNMaxMintermRep}

Based on Markov Logic Networks containing only maxterms and minterms (see Definition~\ref{def:clauses}), we here provide an expressivity study.
There are $2^{\atomorder}$ maxterms and $2^{\atomorder}$ minterms which are enough to represent any probability distribution as we show next.

\begin{theorem}\label{the:maximalClausesRepresentation}\label{the:mintermExpressivityMLN}
	Let there be a positive probability tensor %distribution of the worlds to the atoms $\shortcatvariables$ with probability tensor
		 \[ \probof{\shortcatvariables} \in \bigotimes_{\atomenumeratorin}\rr^2 \, . \] %= \frac{\expof{\mlntensor}}{\partitionfunctionof{\mlntensor}} \, . \]
	Then the Markov Logic Network of minterms (see Definition~\ref{def:clauses})
		\[ \mintermformulaset = \{\mintermof{\atomindices} \, : \, \atomindices\in\atomstates \}\]
	with parameters %with nonzero weights at the maxterms indexed by $\atomindicesin$
		\[ \canparamat{\selvariableof{0}=\catindexof{0},\ldots,\selvariableof{\atomorder-1}=\catindexof{\atomorder-1}}% \weightof{\mintermof{\atomindices}} 
		= \ln \probof{\indexedcatvariables} \]
	coincides with $\probof{\shortcatvariables}$.

	Further, the Markov Logic Network of maxterms
		\[ \maxtermformulaset = \{\maxtermof{\atomindices} \, : \, \atomindices\in\atomstates \}\]
	with wparameters
		\[ \canparamat{\selvariableof{0}=\catindexof{0},\ldots,\selvariableof{\atomorder-1}=\catindexof{\atomorder-1}} %\weightof{\maxtermof{\atomindices}} 
		= - \ln\probof{\indexedcatvariables} \]
	coincides with $\probof{\shortcatvariables}$.
\end{theorem}
\begin{proof}
	It suffices to show, that in both cases of choosing $\formulaset$ by minterms or maxterms with the respective parameters
		\[ \mlnenergy =  \ln\probof{\shortcatvariables} \]
	and therefore
		\[ \mlnprobat{\shortcatvariables} 
		= \sbnormationof{\expof{\mlnenergy}}{\shortcatvariables} 
		=  \sbcontractionof{\expof{\mlnenergy}}{\shortcatvariables} 
		= \probof{\shortcatvariables}\, . \]
	
	In the case of minterms, we notice that for any $\atomindicesin$
		\[ \mintermof{\atomindices}[\shortcatvariables] = \onehotmapofat{\atomindices}{\shortcatvariables} \]
	and thus with the weights in the claim
		\[ \sum_{\atomindicesin} 
		\left( \ln \probof{\indexedcatvariables} \right) \cdot \mintermof{\atomindices}[\shortcatvariables] 
		= \ln\probof{\shortcatvariables} \, .
		 \]

	For the maxterms we have analogously
		\[ \maxtermof{\atomindices}[\shortcatvariables] = \onesat{\shortcatvariables} - \onehotmapofat{\catindices}{\shortcatvariables} \]
	and thus that the maximal clauses coincide with the one-hot encodings of respective states.
	We thus have
	\begin{align*}
		& \sum_{\atomindicesin} 
		\left( - \ln \probof{\indexedcatvariables} \right) \cdot \maxtermof{\atomindices}[\shortcatvariables] \\
		& =
		\left(  \sum_{\nodes_0\subset [\atomorder]} 
		\left( - \ln \probof{\indexedcatvariables} \right) \cdot \onesat{\shortcatvariables} \right) \\
		& \quad + 
		\left(  \sum_{\nodes_0\subset [\atomorder]} 
		\left(  \ln \probof{\indexedcatvariables} \right) \cdot 
		\onehotmapofat{\catindices}{\shortcatvariables} 
		\right) 
		 \\
		 & = \ln\probof{\shortcatvariables} + \lambda \cdot  \onesat{\shortcatvariables}\,,
	\end{align*}
	where $\lambda$ is a constant.
\end{proof}

% Redundant parametrization
In general, this representation is redundant, since any offset of the weight by $\lambda\cdot\ones$ results in the same distribution.
However, the only $\bar{\canparam}$ are multiples of $\onesat{\shortcatvariables}$.

% Comparison with previous schemes
Theorem~\ref{the:maximalClausesRepresentation} is the analogue in Markov Logic to Theorem~\ref{the:tensorToMaxMinTerms}, which shows that any binary tensor has a representation by a logical formula, to probability tensors.
Here we require positive distributions for well-defined energy tensors.


\begin{remark}[Representation of Markov Networks]
% Composition of Markov Networks
	If a probability distribution is representable as a Markov Network, we only need to activate clauses and terms, which variables are contained in factors.
	\red{Make a theorem out of that?}
\end{remark}

%\subsection{Markov Logical Networks as Graphical Models}









\subsection{Examples}


\subsubsection{Distribution of independent variables}

We show next, the independent positive distributions are representable by tuning the $\atomorder$ weights of the atomic formulas and keeping all other weights zero.

\begin{theorem}\label{the:independentAtomicMLN}
	Let $\probat{\shortcatvariables}$ be a positive probability distribution, such that disjoint subsets of atoms are independent from each other.
	Then $\probat{\shortcatvariables}$ is the Markov Logic Network of atomic formulas
		\[ \atomformulaset = \{\atomicformulaof{\catenumerator} \, : \, \catenumeratorin \} \]
	and parameters
		\[ \canparamat{\selvariable=\catenumerator} 
		= \lnof{\frac{
		\contractionof{\probtensor}{\catvariableof{\catenumerator}=1}
		}{
		\contractionof{\probtensor}{\catvariableof{\catenumerator}=0}
		}} \]
%	Any distribution such that the atom satisfaction is independent from each other is reproducable by a MLN with nonzero weights only for the atomic formulas.
\end{theorem}
\begin{proof}
%	Using the independent assumptions, the probability tensor factorizes into normed vectors to each atom, with are transformed atomic formulas (leaving out the neutral ones tensors).
%	We then find a weight to each atom such that the vector is reproduced by the contraction with the activation core.
	
	By Theorem~\ref{the:independenceProductCriterion} we get a decomposition 
		\[ \probat{\shortcatvariables} = \bigotimes_{\catenumeratorin} \probofat{\catenumerator}{\catvariableof{\catenumerator}} \,  \]
	where 
		\[ \probofat{\catenumerator}{\catvariableof{\catenumerator}} = \sbcontractionof{\probtensor}{\catvariableof{\catenumerator}} \, . \]
	
	By assumption of positivity, the vector $\probofat{\catenumerator}{\catvariableof{\catenumerator}}$ is positive for each $\catenumeratorin$ and the parameter
		%\[ \canparam^\catenumerator = \lnof{\frac{\probofat{\catenumerator}{\catvariableof{\catenumerator}=1}}{\probofat{\catenumerator}{\catvariableof{\catenumerator}=0}}} \]
		\[ \canparam^\catenumerator 
		= \lnof{\frac{
		\probofat{\catenumerator}{\catvariableof{\catenumerator}=1}
		}{
		\probofat{\catenumerator}{\catvariableof{\catenumerator}=0}
		}} \]
	well-defined.
	
	We then notice, that 
		\[ \expdistofat{(\{\atomicformulaof{\catenumerator}\},\canparam^{\catenumerator})}{\catvariableof{\catenumerator}} 
		= \probofat{\catenumerator}{\catvariableof{\catenumerator}}\]
	and therefore with the parameter vector of dimension $\seldim=\catorder$ defined as
		\[ \canparamat{\selvariable} = \sum_{\catenumeratorin} \canparam^{\catenumerator} \cdot \onehotmapofat{\catenumerator}{\selvariable}  \]
	we have
	\begin{align*}
	 	 \expdistofat{(\{\atomicformulaof{\catenumerator} \, : \, \catenumeratorin\},\canparam)}{\shortcatvariables} 
		& = \bigotimes_{\catenumeratorin} \expdistofat{(\{\atomicformulaof{\catenumerator}\},\canparam^{\catenumerator})}{\catvariableof{\catenumerator}} \\
		& = \bigotimes_{\catenumeratorin} \probofat{\catenumerator}{\catvariableof{\catenumerator}} \\
		& = \probat{\shortcatvariables} \, . 
	\end{align*}
\end{proof}

%In general, the statistic to an atomic formula measures the marginal distribution. -> To Parameter Estimation

% Failing to be positive -> Hybrid networks
In Theorem~\ref{the:independentAtomicMLN} we made the assumption of positive distributions.
If the distribution fails to be positive, we still get a decomposition into distributions of each variable, but there is at least one factor failing to be positive.
Such factors need to be treated by hybrid logic networks, that is they are base measure for an exponential family coinciding with a logical literal (see Chapter~\ref{cha:hardNetworks}.

% Energy representation
All atomic formulas can be selected by a single variable selecting tensor, that is
	\[ \energytensorofat{(\{\atomicformulaof{\catenumerator} \, : \, \catenumeratorin\},\canparam)}{\shortcatvariables}
	= \sbcontractionof{\vselectionmapat{\shortcatvariables,\selvariable},\canparamat{\selvariable}}{\shortcatvariables} \, . 
	\]
	
% Holds also more generally for any formula! -> Place it earlier?
In case of negative $\canparamat{\catenumerator}$ it is convenient to replace $\atomicformulaof{\catenumerator}$ by $\lnot\atomicformulaof{\catenumerator}$, in order to facilitate the interpretation.
The probability distribution is left invariant, when also replacing $\canparamat{\catenumerator}$ by $-\canparamat{\catenumerator}$.



\subsubsection{Boltzmann machines as MLNs}

%\red{Add sufficient statistics?}

A Boltzmann machine is a member of an exponential family with the energy tensor
	\[ \energytensorofat{W,b}{\indexedcatvariables} = 
	\sum_{\atomenumerator,\secatomenumerator \in [\atomorder]} 
		W[\selvariableof{\vselectionsymbol,0}=\atomenumerator, \selvariableof{\vselectionsymbol,1}=\secatomenumerator] \catindexof{\atomenumerator} \catindexof{\secatomenumerator} 
	+ \sum_{\atomenumerator,\secatomenumerator \in [\atomorder]} b[\selvariableof{\vselectionsymbol,0}=\atomenumerator] \, . \]


%sufficient statistic 
%	\[ \sstat : \atomstates \rightarrow (\rr^{\catorder}\otimes \rr^{\catorder}) \times \rr^{\catorder} \]
%by interaction term
%	\[ \sstat(\shortcatindices) = (\catindexof{\atomenumerator} \Leftrightarrow \catindexof{\secatomenumerator})_{\atomenumerator,\secatomenumerator \in[\atomorder]} \]
%and by potential term
%	\[ \sstat(\shortcatindices) =  (\catvariableof{\atomenumerator})_{\atomenumeratorin} \, . \]

We notice, that this coincides with the energy tensor of a Markov Logic Network with formula set 
	\[ \formulaset = \{ \catvariableof{\atomenumerator} \Leftrightarrow \catvariableof{\secatomenumerator} \, : \, \atomenumerator,\secatomenumerator \in[\atomorder] \} 
	\cup \{ \catvariableof{\atomenumerator}\, : \, \atomenumeratorin \} \, \]
with cardinality $\atomorder^2+\atomorder$.

Each formula is in the expressivity of an architecture consisting of a single binary logical neuron selecting any variable of $\shortcatvariables$ in each argument and selecting connectives $\{\eqbincon,\lpasbincon\}$, where by $\lpasbincon$ we refer to a connective passing the first argument, defined for $\catindexofin{0}, \catindexofin{1}$ as 
	\[ \lpasbincon[\indexedcatvariableof{0},\indexedcatvariableof{1}] = \vselectionmapat{\indexedcatvariableof{0},\catvariableof{1},\selvariableof{\vselectionsymbol}=0} \, . \]

The weight is
	\[ \canparam 
	= \onehotmapofat{0}{\selvariableof{\cselectionsymbol}} \otimes W 
	+ \onehotmapofat{1}{\selvariableof{\cselectionsymbol}} \otimes b[\selvariableof{\vselectionsymbol,0}] \otimes  \onehotmapofat{0}{\selvariableof{\vselectionsymbol,0}} 
	\]
	
And we have
	\[ \energytensorofat{W,b}{\shortcatvariables} = 
	\sbcontractionof{\fsnnat{\shortcatvariables,\selvariableof{\cselectionsymbol},\selvariableof{\vselectionsymbol,0},\selvariableof{\vselectionsymbol,1}}, \canparamat{\selvariableof{\cselectionsymbol},\selvariableof{\vselectionsymbol,0},\selvariableof{\vselectionsymbol,1}}}{\shortcatvariables} \, . \]


\begin{figure}[h]
\begin{center}
	\begin{tikzpicture}[thick, scale=0.35] % , baseline = -3.5pt

\drawatomindices{0}{-4}
\draw (-1,1) rectangle (5, -3);
\node[anchor=center] (text) at (2,-1) {$\sencodingof{\formulaset}$};

%\draw[->] (2,-1)--(2,1) node[midway,right] {\tiny ${\atomicformulaof{\parindexof{1}} \land \atomicformulaof{\parindexof{2}}}$}; 

\draw[<-] (5,0.5) -- (7,0.5) node[midway, above] {\tiny $\selvariableof{\cselectionsymbol}$};
\draw[<-] (5,-1)--(7,-1) node[midway,above] {\tiny $\selvariableof{\vselectionsymbol,1}$}; 
\draw[<-] (5,-2.5)--(7,-2.5) node[midway,below] {\tiny $\selvariableof{\vselectionsymbol,0}$}; 

\draw (7,1) rectangle (9, -3);
\node[anchor=center] (text) at (8,-1) {$\canparam$};


		
\node[anchor=center] (text) at (12,-2) {${=}$};


\begin{scope}[shift={(17,8)}]

% SkeletonCores

%% Would be required to match the formula tensor example, but would get messy!

%\drawatomindices{0}{0}
%\draw (-1,-1) rectangle (5, -3);
%\node[anchor=center] (text) at (1.5,-2) {$\skeleton^{\land}$};
%\drawatomindices{0}{-4}


%\drawatomindices{7}{0}
%\draw (6,-1) rectangle (12, -3);
%\node[anchor=center] (text) at (9,-2) {$\skeleton^{\lnot}$};
%\drawatomindices{7}{-4}

%\drawatomindices{3.5}{-4}	
%\draw (-1,-5) rectangle (12,-7);
%\node[anchor=center] (text) at (5.5,-6) {$\skeletoncoreof{\land}$};
%\drawatomindices{0}{-8}	
%\drawatomindices{7}{-8}	


	\begin{scope}[shift={(0,-10)}]
	
		\draw (4.5,7) rectangle (6.5, 9);	
		\node[anchor=center] (text) at (5.5,8) {$\onehotmapof{1}$};
		
		\draw[->] (5.5,5) -- (5.5,7) node[midway, right] {\tiny $\catvariableof{\lneuron}$};	
		\draw (1,3) rectangle (10, 5);
		\node[anchor=center] (text) at (5.5,4) {$\rencodingof{\{\eqbincon,\lpasbincon \}}$};
		\draw (10,4) -- (12,4);
		\draw[<-] (12,4) -- (14,4) node[midway, above] {\tiny $\selvariableof{\cselectionsymbol}$};
			
		% SelectorCores
		\draw[->] (2,1) -- (2,3) node[midway, left] {\tiny $\catvariableof{\vselectionsymbol,0}$};
		\draw (-1,1) rectangle (5, -1);
		\node[anchor=center] (text) at (2,0) {$\selectorcoreof{1}$};
		\draw (5,0) -- (12,0);
		\draw[<-] (12,0) -- (14,0) node[midway, above] {\tiny $\selvariableof{\vselectionsymbol,1}$};
		\begin{scope}[shift={(0,-2)}]
			\draw[<-] (0,1)--(0,-3) node[midway,left] {\tiny $\catvariableof{0}$}; 
			\draw[<-] (1.5,1)--(1.5,-3) node[midway,left] {\tiny $\catvariableof{1}$}; 
			\node[anchor=center] (text) at (3,0) {$\cdots$};
			\draw[<-] (4,1)--(4,-3) node[midway,right] {\tiny $\catvariableof{\atomorder\shortminus1}$}; 
		\end{scope}


		\draw (9,-1) -- (9,1);
		\draw[->] (9,1) -- (9,3) node[midway, left] {\tiny $\catvariableof{\vselectionsymbol,1}$};
		\draw (6,-3) rectangle (12, -1);
		\node[anchor=center] (text) at (9,-2) {$\selectorcoreof{0}$};
		\draw[<-] (12,-2) -- (14,-2) node[midway, above] {\tiny $\selvariableof{\vselectionsymbol,0}$};
		\drawatomindices{7}{-4}	
		
		% ParameterCores
		\draw (14,5) rectangle (16, -3);
		\node[anchor=center] (text) at (15,1) {$\canparam$};
		
		\begin{scope}[shift={(-3.5,8)}]
			\draw[fill] (7.5,-15) circle (0.25cm);
			\draw[] (7.5,-15) to[bend left=25] (3.5,-13);
			\draw[] (7.5,-15) to[bend right=25] (10.5,-13);

			\draw[fill] (9,-15.25) circle (0.25cm);
			\draw[] (9,-15.25) to[bend left=25] (5,-13);
			\draw[] (9,-15.25) to[bend right=25] (12,-13);

			\draw[fill] (11.5,-15) circle (0.25cm);
			\draw[] (11.5,-15) to[bend left=25] (7.5,-13);
			\draw[] (11.5,-15) to[bend right=25] (14.5,-13);

			\drawatomindices{7.5}{-16}

		\end{scope}
	
	\end{scope}

\end{scope}

\end{tikzpicture}
\end{center}
\caption{Tensor network representation of the energy of a Boltzmann machine}
\label{fig:boltzmannEnergy}
\end{figure}


%where by $(\cdot,\cdot)|_{0}$
%To connect with the formalism of Boltzmann machines, let us identify the visible units of a Boltzmann machines with the atoms in a propositional theory.

%Boltzmann machines are then reproduced by taking $\atomorder^2+\atomorder$ many formulas, namely those measuring the correlations and the marginal distributions.
%To be more precise, the correlation between atom $\atomicformulaof{\atomenumerator}$ and $\atomicformulaof{\secatomenumerator}$ is measured by the satisfaction rate of the formula 
%	\[ \exformula_{\atomenumerator,\secatomenumerator} = \atomicformulaof{\atomenumerator} \leftrightarrow \atomicformulaof{\secatomenumerator}\]

%\begin{theorem}
%	Any Boltzmann machine over $\atomorder$ units with interaction matrix $U\in\rr^{\atomorder\times\atomorder}$ and potential term $b\in\rr^{\atomorder}$ (MacKay Book notation) is a MLN where the only nonzero weights are 
%		\[ \weightof{\atomicformulaof{\atomenumerator} } = b_{\atomenumerator} \quad, \quad \atomenumeratorin \]
%	and 
%		\[ \weightof{ \exformula_{\atomenumerator,\secatomenumerator} } = U_{\atomenumerator, \secatomenumerator} \quad , \quad \atomenumerator,\secatomenumerator \in [\atomorder]\] 
%\end{theorem}

\red{
Often Boltzmann machines are formulated with hidden variables.
To average those out, one needs to instantiate the probability distribution instead of the energy tensor and leave only visible variables open in a contraction.
}


Markov Logic Networks go beyond the Boltzmann machines already for binary formulas, by the flexibility to capture further dependencies beyond the correlation.
We can use any binary logical connective and have an associated formula where we can put a weight on.



%\begin{remark}[Hopfield networks]
%	Also interesting for MLNs is a Hopfield perspective.
%	Having an initialization the update can be interpreted as a Gibbs sampling step at temperature $0$ (since deterministic update).
%\end{remark}


%\begin{remark}[Representation by Formula Selecting Networks]
%	When choosing an architecture with a single neurons selecting on both arguments any atom and having the logical biconditional as fixed connective.
%	The sufficient statistics of the empirical distribution is the correlation matrix between the atoms.
%\end{remark}

%
%\subsubsection{MLNs as higher-order deep Boltzmann Machines}
%
%\red{Unclear whether this makes sense: Hidden nodes depend deterministically on the observed!}
%
%We understand 
%\begin{itemize}
%	\item Atoms as observed nodes (might be redundant, since they are formulas themselves)
%	\item Formulas (including the atomic formulas) as hidden nodes
%\end{itemize}
%
%Non-atomic formulas have a hard coded deterministic dependency on the atomic formulas.
%
%The MLN distribution is given by associating a potential to some hidden nodes.








\subsection{Applications}

Markov Logic Networks as neuro-symbolic architectures:
\begin{itemize}
	\item Neural Paradigm here by decompositions of logical formulas into their connectives.
		In more generality by decompositions of sufficient statistics into composed functions, using Basis Calculus.
		Deeper nodes as carrying correlations of lower nodes.
	\item Symbolic Paradigm by interpretability of propositional logics.
\end{itemize}


Markov Logic Networks as trainable Machine Learning models:
\begin{itemize}
	\item Expressivity: Can represent any distributions, as shown by Theorem~\ref{the:maximalClausesRepresentation}, with $2^d$ formulas.
	\item Efficiency: Can only handle small subsets of possible formulas, since their possible number is huge.
		Tensor networks provide means to efficiently represent formulas depending on many variables and reason based on contractions.
	\item Differentiability: Distributions are differentiable functions of their weights, see Parameter Estimation Chapter. 
		The log-likelihood of data is therefore also differentiable function of their weights and we can exploit first-order methods in their optimization.
	\item Structure Learning: We need to find differentiable parametrizations of logical formulas respecting a chosen architecture.
		In Chapter~\ref{cha:formulaBatches} such representations are described based on Selector Tensor Networks.
\end{itemize}





%\subsection{Symbolic Paradigm by Logics}
%Logical formulas (which are maps from $\atomstates$ to $[2]$ as explained before) correspond with a function to be represented by a neural network.

%\subsubsection{Neural Paradigm by Formula Decompositions}

%% Decompositions
%In logics, formula can be decomposed into logical connectives acting on smaller formulas as has been shown in Chapter~\ref{cha:FormulaTensors}.







\section{Hard and Hybrid Logic Networks}\label{cha:hardNetworks} % To be dropped in the unification with the MLN chapter

\red{Hard Logic Networks are Knowledge Bases, Hybrid Logic Networks are exponential families on Knowledge Bases.
This makes it impossible to build energy tensors without basemeasures capturing vanishing coordinates.}



\red{Work in Theorem~\ref{the:factorReduction} to reduce entailment!}

% Hard logic vs markov logic
While exponential families are positive distributions, in logics probability distributions can assign states zero probability.
As a consequence, Markov Logic Networks have a soft logic interpretation in the sense that violation of activated formulas have nonzero probability.
We here discuss their hard logic counterparts, where worlds not satisfying activated formulas have zero probability.

Further we investigate, how both hard and soft logic factors can be combined to hybrid networks.

%The Tensor Network decomposition of formula tensors is analogously constructed to a graphical representation of the formulas.
%We thus develop in this section the interpretation of formula tensor decompositions as Bayesian and Markov Propositional Networks.



\subsection{The limit of hard logic}\label{sec:hardLogicLimit} % To be merged with the above

The probability function of Markov Logic Networks with positive weights mimiks the tensor network representation of the knowledge base, which is the conjunction of the formulas. 
The maxima of the probability function coincide with the models of the corresponding knowledge base, if the latter is satisfiable.
However, since the Markov Logic Network is defined as a normed exponentiation of the weighted formula sum, it is a positive distribution whereas uniform distributions among the models of a knowledge base assign zero probability to world failing to be a model.
Since both distributions are tensors in the same space to a factored system, we can take the limits of large weights and observe, that Markov Logic Networks indeed converge to normed knowledge bases.


% Limit of Activation core
\begin{lemma}
	When taking the limit of large weights $\weightof{\exformula}\rightarrow\infty$ we observe a coordinatewise (in the sense of a convergence of each coordinate of the tensor) convergence 
	\begin{align}
%	\frac{1}{\partitionfunctionof{\weightof{\exformula}\exformula}} \expof{\weightof{\exformula}\exformula} 
		\normationofwrt{\expof{\weightof{\exformula}\cdot \exformula}}{\shortcatvariables}{\varnothing} \rightarrow  \normationofwrt{\exformula}{\shortcatvariables}{\varnothing}
%	\frac{1}{\braket{\exformula,\ones}}\exformula
	\end{align}
\end{lemma}
\begin{proof}
	We have 
	\begin{align*}
		\partitionfunctionof{\mlnparameters} = (\prod_{\atomenumerator} \catdimof{\atomenumerator} - \contractionof{\exformula}{\varnothing}) + \contractionof{\exformula}{\varnothing} \cdot \expof{\weightof{\exformula}}
	\end{align*}
	and therefore for any $\atomindices\in\atomstates$ with $\exformula(\atomindices)=1$
	\begin{align*}
		\normationofwrt{\expof{\weightof{\exformula}\cdot \exformula}}{\indexedcatvariables}{\varnothing} 
		&= \frac{
			\expof{\weightof{\exformula}}
			}{
			(\prod_{\atomenumerator} \catdimof{\atomenumerator} - \contractionof{\exformula}{\varnothing}) + \contractionof{\exformula}{\varnothing} \cdot \expof{\weightof{\exformula}}
			} \\
		& \rightarrow \frac{1}{\contractionof{\exformula}{\varnothing}} 
		= \normationofwrt{\exformula}{\indexedcatvariables}{\varnothing} \, . 
	\end{align*}
	For any $\atomindices\in\atomstates$ with $\exformula(\atomindices)=0$ we have on the other side
	\begin{align*}
		\normationofwrt{\expof{\weightof{\exformula}\cdot \exformula}}{\indexedcatvariables}{\varnothing} 
		&= \frac{
			1
			}{
			(\prod_{\atomenumerator} \catdimof{\atomenumerator} - \contractionof{\exformula}{\varnothing}) + \contractionof{\exformula}{\varnothing} \cdot \expof{\weightof{\exformula}}
			} \\
		& \rightarrow 0
		= \normationofwrt{\exformula}{\indexedcatvariables}{\varnothing} \, . 
	\end{align*}
\end{proof}

\begin{theorem}
	Let $\formulaset$ be a formulaset and $\canparam$ a positive parameter vector.
	If the formula
		\[ \kb = \bigwedge_{\exformulain} \exformula \]
	is satisfiable we have in the limit $\invtemp\rightarrow\infty$ the coordinatewise convergence
		\[ \expdistofat{(\formulaset,\invtemp\cdot\canparam)}{\shortcatvariables} \rightarrow \normationofwrt{\kb}{\shortcatvariables} \, . \]
\end{theorem}
\begin{proof}
	Since $\kb$ is satisfiable we find $\catindices\in\atomstates$ with
		\[  \contractionof{\expof{\sum_{\exformulain}\invtemp\cdot \weightof{\exformula} \cdot \exformula}}{\indexedcatvariables} = \expof{\invtemp \cdot \sum_{\exformulain}\weightof{\exformula}}  \]
	and the partition function obeys
		\[ \contractionof{\expof{\sum_{\exformulain}\invtemp\cdot \weightof{\exformula} \cdot \exformula}}{\varnothing} \geq  \expof{\invtemp \cdot \sum_{\exformulain}\weightof{\exformula}}  \, . \]
	For any state $\catindices\in\atomstates$ with $\kb(\catindices)=0$ we find $\secexformula\in\formulaset$ with $\secexformula(\catindices)=0$ and have
	\begin{align*}
	 	\frac{
		\contractionof{\expof{\sum_{\exformulain}\invtemp\cdot \weightof{\exformula} \cdot \exformula}}{\indexedcatvariables}
		}{
		\contractionof{\expof{\sum_{\exformulain}\invtemp\cdot \weightof{\exformula} \cdot \exformula}}{\varnothing}
		} 
		\leq  
	 	\frac{
		\expof{\invtemp\cdot \sum_{\exformulain : \exformula\neq \secexformula}\weightof{\exformula}}
		}{
		\expof{\invtemp\cdot \sum_{\exformulain}\weightof{\exformula}}
		} 
		= \expof{\invtemp \cdot \weightof{\secexformula}} \rightarrow 0 \, . 
	\end{align*}
	The limit of the distribution has thus support only on the models of $\kb$. 
	Since any model of $\kb$ has same energy at any $\invtemp$ the limit is a uniform distribution and coincides therefor with
		\[ \normationofwrt{\kb}{\shortcatvariables} \, . \]
\end{proof}


\begin{remark}[More generic situation of simulated annealing]
	The process of taking $\invtemp\rightarrow\infty$ is known as simulated annealing, see Chapter~\ref{cha:probReasoning}.
	From the discussion there we have the more general statement, that the limiting distribution is the uniform distribution among the maxima of $\expdistofat{(\formulaset,\canparam)}{\shortcatvariables}$.
	If the formula $\kb$ is not satisfiable the normation $\normationofwrt{\kb}{\shortcatvariables}{\varnothing}$ does not exist and the limit distribution has another syntactical representation, to be gained e.g. by minterm or maxterm representation (see Theorem~\ref{the:tensorToMaxMinTerms}).
\end{remark}





%To make this convergence precise, we define the uniform distribution 
%\begin{align}
%	\expdistofat{\kb}{\datapoint}
%	= \begin{cases} 
%	\frac{1}{\braket{\ftensorof{\kb},\ones}} & \text{if } \braket{\ftensorof{\kb},\datapoint} = 1 \\
%	0 &  \text{if } \braket{\ftensorof{\kb},\datapoint} = 0
%	\end{cases}
%\end{align}

%\begin{theorem}
%	Given a MLN parameterized by $\mlnparameters$, we have for $\lambda\rightarrow\infty$
%		\[ \kldivof{\expdistofat{(\formulaset,\lambda\cdot\weight)}{\datapoint}}{\expdistofat{\kb}{\datapoint}} \rightarrow 0 \, .\]
%\end{theorem}
%\begin{proof}
%	Follows directly from the convergence at each core.
%\end{proof}














\subsection{Hard Logic Networks}

Hard Logic Network coincide with Knowledge Bases.
We use $\land$ symmetry to represent them as a contraction of the formulas building the Knowledge Base as conjunction.

\begin{theorem}[Conjunction Decomposition of Knowledge Bases]\label{the:conDecKB}
	For a Knowledge Base
		\[ \kb = \bigwedge_{\exformula\in\formulaset} \exformula \]
	we have
		\[ \kbat{\shortcatvariables} = \contractionof{\formulaat{\shortcatvariables}}{\shortcatvariables}   \]
	and
		\[ \kbat{\shortcatvariables} = \contractionof{\{\rencodingofat{\exformula}{\catvariableof{\exformula},\shortcatvariables} \, : \, \exformula\in\formulaset\} \cup \{\onehotmapofat{1}{\catvariableof{\exformula}} \, : \, \exformula\in\formulaset\} }{\shortcatvariables} \, .  \]
\end{theorem}
\begin{proof}
	By the $\land$-symmetry, see effective calculus and 
		\[ \formulaat{\shortcatvariables} =  \contractionof{\{\rencodingofat{\exformula}{\catvariableof{\exformula},\shortcatvariables}, \onehotmapofat{1}{\catvariableof{\exformula}}\} }{\shortcatvariables} \]
\end{proof}

\begin{remark}{$\land$ symmetry does not generalize to Markov Logic Networks}
	% Comparison to Markov Logic
	In Markov Logic, similar decompositions are not possible.
	For example, consider a MLN with a single formula $\atomicformulaof{0}\land\atomicformulaof{1}$ and nonvanishing weight $\canparam$.
	This does not coincide with the distribution of a MLN of two formulas $\atomicformulaof{0}$ and $\atomicformulaof{1}$.
	To see this, we notice that with respect to the distribution of the first MLN, both variables are not independent, while for any MLN constructed by the two atomic formulas they are.
\end{remark}


%It is known, that there are symmetries in the syntactical represention of Knowledge Bases.
%
%There is a lot of redundancy in the activation of Knowledge Cores describing exactly the same models.
%
%\begin{theorem}[$\land$-symmetry]\label{the:landSymmetry}
%	We observe that the contraction of an $\land$ core with $\tbasis$  is equivalent with $\tbasis$ cores on all the connected subformulas.
%\end{theorem}
%\begin{proof}
%	By equality of the Knowledge Base contraction in both ways: The missing subformulas behave the same if they are activated, since they then are contrained to the same subnetworks somewhere else. 
%	%\red{Find better arguments for missing subformulas when having the larger core.}
%\end{proof}
%
%\begin{theorem}[$\lnot$-symmetry]
%	Similarly the contraction of an $\lnot$ core with $\tbasis$ or $\tbasis$ has the same result as with $\tbasis$ or $\tbasis$ on the subformula.
%\end{theorem}
%
%We call the application of these in changing the Knowledge Cores without changing the contracted network as the representation symmetry.


\subsubsection{Conjunctive Normal Representation}

\red{Poly reps here?}

We can now apply the representation symmetries to represent a propositional Knowledge Base in conjunctive normal form.
A Knowledge Base in Conjunctive Normal Form is a conjunction of clauses, where clauses are disjunctions of literals being atoms (positive literals) or negated atoms (negative literals).




%One tensor representation of a Knowledge Base is the association of the Knowledge Core $\tbasis$ at the formula being the Knowledge Base itself.
%We can use the $\land$ symmetry (Theorem~\ref{the:landSymmetry}) to propagate $\tbasis$ to all clause cores and get an alternative representation.
%Those are especially interesting when using Modus Ponens/Resolution as local sub-KB reasoners (see Section~\ref{subsec:LocalEntailment}).


\subsubsection{Polynomial Representation of Formulas}

\red{We can further derive representation schemes for Knowledge Bases, which are in conjunctive normal forms.}

Formulas can be represented as sparse polynomials, which will be discussed in more detail in Chapter~\ref{cha:sparseTC} (see Definition~\ref{def:polynomialSparsity}).

\begin{lemma}\label{lem:clauseTermBasisPlus}
	Any term is representable by a single monomial and any clause is representable by at most two monomials. %, any term of basis+ with rank 1. %Use also \baspluscprankof{}
\end{lemma}
\begin{proof}
	Let $\nodes_0$ and $\nodes_1$ be disjoint subsets of $\nodes$, then we have
	\begin{align*}
		\termof{\nodes_0}{\nodes_1} = \onehotmapofat{
			\{\catindexof{\atomenumerator} = 0 : \atomenumerator\in\nodes_0\} \cup \{\catindexof{\atomenumerator} = 1 : \atomenumerator\in\nodes_1\}
		}{\catvariableof{\nodes_0\cup\nodes_1}} \otimes \onesat{\catvariableof{\nodes/(\nodes_0\cup\nodes_1)}}
	\end{align*}
	and
	\begin{align*}
		\clauseof{\nodes_0}{\nodes_1} = \onesat{\catvariableof{\nodes}} - \onehotmapofat{
			\{\catindexof{\atomenumerator} = 0 : \atomenumerator\in\nodes_0\} \cup \{\catindexof{\atomenumerator} = 1 : \atomenumerator\in\nodes_1\}
		}{\catvariableof{\nodes_0\cup\nodes_1}}
		\otimes \onesat{\catvariableof{\nodes/(\nodes_0\cup\nodes_1)}} \, . 
	\end{align*}
	We notice, that any tensors $\ones$ and $\onehotmapof{\catindex}\otimes \ones$ habe basis+-rank of $1$ and therefore $\termof{\nodes_0}{\nodes_1}$ of $1$ and $\clauseof{\nodes_0}{\nodes_1}$ of at most $2$.
\end{proof}


We apply Lemma~\ref{lem:clauseTermBasisPlus} to show the following sparsity bound on the energy tensor of Markov Logic Networks.

\begin{theorem}
	Any formula $\exformula$ with a conjunctive normal form of $n$ clauses satisfies
		\[ \slicesparsityof{\exformula} \leq 2^{n} \, . \]
	For any set $\formulaset$ of formulas each with a conjunctive normal form of $n_{\exformula}$ clauses satisfies for any $\weight$
		\[ \slicesparsityof{\sum_{\exformulain}\weightof{\exformula}\cdot\exformula} \leq \sum_{\exformulain}2^{n_{\exformula}} \, . \]
\end{theorem}
\begin{proof}
	Let $\exformula$ have a CNF with clauses indexed by $l\in[n]$ and each clause represented by subsets $\nodes_0^l, \nodes_1^l$, that is
		\[ \exformula = \bigwedge_{l \in [n]}  \clauseof{\nodes_0^l}{\nodes_1^l} \, . \]
	We now use the rank bound of Theorem~\ref{the:CPrankContractionBound} and Lemma~\ref{lem:clauseTermBasisPlus} to get
	\begin{align*}
		\slicesparsityof{\exformula} \leq \prod_{l \in [n]}  \slicesparsityof{\clauseof{\nodes_0^l}{\nodes_1^l}} \leq 2^n \, . 
	\end{align*}
	
	Given a collection of formulas $\formulaset$, each with a CNF of $n_{\exformula}$ clauses we apply Theorem~\ref{the:CPrankSumBound} and get
	\begin{align*}
		\slicesparsityof{\sum_{\exformulain}\weightof{\exformula}\cdot\exformula} \leq \sum_{\exformulain}\slicesparsityof{\exformula} \leq \sum_{\exformulain}2^{n_{\exformula}} \, . 
	\end{align*}
\end{proof}






\subsection{Hybrid Logic Network}

Markov Logic Networks are by definition positive distributions.
In contrary, Hard Logic Networks model uniform distributions over model sets of the respective Knowledge Base and therefore have vanishing coordinates.
We now show how to combine both approaches by defining Hybrid Logic Networks.
\red{
We orient on Example 3.6 in \cite{wainwright_graphical_2008} and choose hard constraints as a base measure $\hfbasemeasure$, probabilistic soft formulas as sufficient statistics as before.
}

\begin{definition}
	Given a set of formulas $\softformulaset$ with weights $\canparam$ and set $\hardformulaset$ of formulas, which conjunction is satisfiable, the hybrid logic network is the probability distribution
	\begin{align*}
		\probtensorof{(\softformulaset,\canparam,\hfbasemeasure)}[\shortcatvariables] 
		= \normationof{
		\{\exformula : \exformula\in\hardformulaset\} \cup \{\expof{\weightof{\exformula}\cdot\exformula} : \exformula\in\softformulaset\}
		}{\shortcatvariables} \, ,
	\end{align*}
	which is the member of the exponential family with statistic by $\softformulaset$ and the base measure
		\[ \hfbasemeasure[\shortcatvariables] = \contractionof{\{\formula : \formula \in \hardformulaset\}}{\shortcatvariables} \, .\]
\end{definition}

The assumption of a satisfiable set $\hardformulaset$ is necessary, as we show next.

\begin{theorem}
	If any only if $\bigwedge_{\formula\in\hardformulaset}\formula$ is satisfiable, the tensor 
		\[  \contractionof{
		\{\exformula : \exformula\in\hardformulaset\} \cup \{\expof{\weightof{\exformula}\cdot\exformula} : \exformula\in\softformulaset\}
		}{\shortcatvariables} \]
	is normable.
\end{theorem}
\begin{proof}
	We need to show that
	\begin{align}\label{eq:tbsWellDefinedHLN}
		\contraction{\{\exformula : \exformula\in\hardformulaset\} \cup \{\expof{\weightof{\exformula}\cdot\exformula} : \exformula\in\softformulaset\}} > 0 \, . 
	\end{align}
	Since the conjunction of $\hardformulaset$ is satisfiable we find a $\shortcatindices$ with $\formulaat{\indexedcatvariableof{[\catorder]}}=1$ for all $\exformula\in\hardformulaset$.
	Then 
	\begin{align*}
		 \contractionof{\{\exformula : \exformula\in\hardformulaset\} \cup \{\expof{\weightof{\exformula}\cdot\exformula} : \exformula\in\softformulaset\}}{\indexedcatvariableof{[\catorder]}}  
		 & = \left( \prod_{\exformula\in\hardformulaset}\formulaat{\indexedcatvariableof{[\catorder]}} \right) 
		 \cdot \left( \prod_{\exformula\in\softformulaset}\expof{\weightof{\exformula}\cdot\exformula}[\indexedcatvariableof{[\catorder]}] \right) \\
		 & =  \left( \prod_{\exformula\in\softformulaset}\expof{\weightof{\exformula}\cdot\exformula}[\indexedcatvariableof{[\catorder]}] \right) \\
		 & > 0 \, . 
	\end{align*}
	Condition \eqref{eq:tbsWellDefinedHLN} follows from this and the Hybrid Logic Network is well-defined.
\end{proof}


%\subsubsection{Representation as Exponential Families}

%We call graphical models which contain cores from a Markov Logik Network and of a Hard Logic Network a Hybrid Logic Network.

%Hybrid logic networks are exponential family, where the hard logic factors build a base measure and the probabilistic logic components a contracted statistics function.
%\begin{theorem}
%	A hybrid logic network $\probtensorof{(\softformulaset,\canparam,\hardformulaset)}$ is in the exponential family with base measure
%		\[ \hfbasemeasure[\shortcatvariables] = \contractionof{\{\formula : \formula \in \hardformulaset\}}{\shortcatvariables} \] % Do not need a normed base measure!
%	and sufficient statistic $\softformulaset$ the element with parameters $\canparam$.
%\end{theorem}

%
%By a slight abuse of notation, we denote by $\hardformulaset$ both the set of propositional formulas and their contractions.



\subsubsection{Tensor Network Representation}


We can employ the formula decompositions to represent both probabilistic facts of the MLN and hard facts (seen as the limit of large weights).

\begin{theorem}\label{the:hybridNetworkRepresentation}
	For any hybrid logic network we have
	\begin{align*}
		\probtensorof{(\softformulaset,\canparam,\hardformulaset)}[\shortcatvariables] 
		= \normationof{
		\{\rencodingofat{\exformula}{\catvariableof{\exformula},\shortcatvariables} : \exformula\in\softformulaset\cup\hardformulaset \}
		\cup \{\onehotmapofat{1}{\catvariableof{\exformula}} : \exformula\in\hardformulaset \}
		\cup \{\headcoreofat{\exformula}{\catvariableof{\exformula}} : \exformula\in\softformulaset \}
		}{\shortcatvariables} \, . 
	\end{align*}
\end{theorem}
\begin{proof}
	By Lemma~\ref{lem:formulaEncodingDecomposition}.
\end{proof}

%% Overwork: Allow for infinite weights?
%Capturing hard and soft constraints at the same time, we can use a weight to each formula:
%\begin{itemize}
%	\item $\weightof{\exformula}=0$: The formula is neutral and does not influence the probability distribution.
%	Techniacally, the formula tensor is contracted with a $\ones$ head with the result being a $\ones$ world tensor, which leaves other products invariant under Hadamard products.
%	\item $\weightof{\exformula}=\infty$: The formula is a hard constraint. 
%	\item $\weightof{\exformula} \in (0,\infty)$: The fromula is a probabilistic constraint.
%\end{itemize}



%The reason for this is the Slicing Theorem, enabling the operations by both (exponentiation and selection of one slice) by the head cores.
For an example see Figure~\ref{fig:ActivatedHeads}.

\begin{figure}[h]
\begin{center}
	\begin{tikzpicture}[scale=0.35, thick, yscale=-1] % , baseline = -3.5pt


\draw[<-] (0,-1)--(0,1) node[midway,left] {\tiny $\catvariableof{a}$}; 
\draw[<-] (1.5,-1)--(1.5,1) node[midway,right] {\tiny $\catvariableof{b}$}; 
\draw[<-] (3,-1)--(3,1) node[midway,right] {\tiny $\catvariableof{c}$}; 
\draw (-1,-1) rectangle (4, -3);
\node[anchor=center] (text) at (1.5,-2) {$\partitionfunction \cdot \probtensor$};



\node[anchor=center] (text) at (5.5,-2) {${=}$};


\begin{scope}[shift={(11,0)}]

\draw[->] (0,1)--(0,-1)  node[midway,left] {\tiny $\catvariableof{a}$}; 
\draw[->]  (3,1)--(3,-1) node[midway,right] {\tiny $\catvariableof{b}$}; 
\draw[->] (6,1)--(6,-1) node[midway,right] {\tiny $\catvariableof{c}$}; 
	
\draw (-1,-1) rectangle (4, -3);
\node[anchor=center] (text) at (1.5,-2) {$\concoreof{\lor}$};

\draw[->] (1.5,-3) --(1.5,-4.5) node[midway,right]{\tiny $\catvariableof{a \lor b}$};
\drawvariabledot{1.5}{-4.5}
\draw[->] (1.5,-4.5) --(1.5,-6) ;


\draw[fill, \probcolor] (1.5,-4.5) circle (0.15cm);
\draw[\probcolor] (1.5,-4.5) -- (-0.25,-4.5);
\draw[\probcolor]  (-6.75, -3.5) rectangle (-0.25, -6.5);
\node[anchor=center,\probcolor] (text) at (-3.5,-5) {$\begin{bmatrix} 
1 \\
\expof{\canparam}
\end{bmatrix}$};

\draw (5,-1) rectangle (7, -3);
\node[anchor=center] (text) at (6,-2) {$\concoreof{\lnot}$};

\draw[->] (6,-3) --(6,-4.5) node[midway,right]{\tiny $\catvariableof{\lnot c}$};
\drawvariabledot{6}{-4.5}
\draw[->] (6,-4.5) --(6,-6);	


	
\draw (0.5,-6) rectangle (6.5,-8);
\node[anchor=center] (text) at (3.5,-7) {$\concoreof{\lor}$};
	
\draw[<-] (4,-9.5) -- (4,-8) node[midway,right] {\tiny $\catvariableof{a \lor b \lor \lnot c}$};

\draw[fill,\concolor] (4,-9.5) circle (0.15cm);
\draw[\concolor] (4,-9.5) -- (2.25,-9.5);
\draw[\concolor]  (0.25, -8.5) rectangle (2.25, -11.5);
\node[anchor=center,\concolor] (text) at (1.25,-10) {$\begin{bmatrix} 
0 \\
1
\end{bmatrix}$};

%\draw (3,-9) rectangle (5,-11);
%\node[anchor=center] (text) at (4,-10) {$\truevectorat{}$};

\end{scope}

\end{tikzpicture}
\end{center}
\caption{Diagram of a formula tensor with activated heads, containing \textcolor{\concolor}{hard constraint cores} and \textcolor{\probcolor}{probabilistic weight cores} .} %along \textcolor{\inactivecolor}{inactive cores}.}
\label{fig:ActivatedHeads} 
\end{figure}



\begin{remark}{Probability interpretation using the Partition function}
	The tensor networks here represent unnormalized probability distributions.
	The probability distribution can be normed by the quotient with the naive contraction of the network, the partition function.
\end{remark}


\subsubsection{Reasoning Properties}



\begin{theorem}
	Let $(\softformulaset,\canparam,\hardformulaset)$ define a Hybrid Logic Network.
	Given a query formula $\exformula$ we have that 
		\[ \probtensorof{(\softformulaset,\canparam,\hardformulaset)} \models \exformula \]
	if and only if
		\[ \hardformulaset \models \exformula \, . \]
\end{theorem}
\begin{proof}
	Application of Theorem~\ref{the:factorReduction} on the representation of Hybrid Logic Networks as Markov Networks in Theorem~\ref{the:hybridNetworkRepresentation}.
\end{proof}


%% Now in theorems
%\begin{itemize}
%	\item Entailment queries answered on the hard logic parts alone.
%	\item Well defined distributions, when hard logic formulas are satisfiable.
%	\item Redundant hard formulas (Redundancy, whenever the contractions unchanged): If entailled by the rest of the hard logic formulas. 
%	\item Redundant soft formulas (Redundancy, whenever the normations unchanged):  Either if entailled or contradicted by the hard logic formulas
%\end{itemize}



Formulas in $\softformulaset$, which are entailed or contradicted by $\hardformulaset$ are redundant 

\begin{theorem}
	If for a formula $\exformula$ and $\hardformulaset$ we have  %and only if
		\[ \hardformulaset \models \exformula \, \quad \text{or} \quad \hardformulaset \models \lnot\exformula \]
	then for any $(\softformulaset,\canparam,\hardformulaset)$
		\[ \probofat{(\softformulaset/\{\exformula\},\tilde{\canparam},\hardformulaset)}{\shortcatvariables} =  \probofat{(\softformulaset,\canparam,\hardformulaset)}{\shortcatvariables}  \, , \]
	where $\tilde{\canparam}$ denotes the tensor $\canparam$, where the coordinate to $\exformula$ is dropped, if $\exformula\in\softformulaset$.
\end{theorem}
\begin{proof}
	Isolate the factor to the hard formula, which is constant for all situations.
\end{proof}

%% Now in the 
A similar statement holds for the hard formulas itself, as shown in Theorem~\ref{the:ReduncancyOfEntailed}.
However, notice that if $\hardformulaset/\{\exformula\}\models\lnot\exformula$, then $\hardformulaset\cup\{\exformula\}$ is not satisfiable and a hybrid logic network cannot be defined for $\hardformulaset\cup\{\exformula\}$ as hard logic formulas.

%If the conjunction of $\hardformulaset/\{\exformula\}$ entails $\exformula$, we can erase $\exformula$ from $\hardformulaset$ without changing the contraction, therefore without changing the base measure of the Hybrid Logic Network.



\subsubsection{Expressivity}

Hybrid Logic Networks extend the expressivity result of Theorem~\ref{the:mintermExpressivityMLN} to arbitrary probability tensors, dropping the positivity constraints for Markov Logic Networks.

\begin{theorem}\label{the:mintermExpressivityHLN}
	Let $\probat{\shortcatvariables}$ a possibly not positive probability tensor we build a base measure
		\[ \hfbasemeasure = \nonzeroof{\probat{\shortcatvariables}} \]
	and a parameter tensor
	\begin{align*}
		\canparamat{\selvariableof{[\catorder]}=\shortcatindices}
		= \begin{cases}
			0 & \text{if} \quad \probat{\shortcatvariables=\shortcatindices} = 0  \\
			\lnof{\probat{\shortcatvariables=\shortcatindices}} & \text{else} 
		\end{cases} \, . 
	\end{align*}
	Then the probability tensor is the member of the minterm exponential family with base measure $\hardformulaset$ and parameter $\canparam$, that is
		\[ \probof{(\mintermformulaset,\canparam,\hfbasemeasure)}\]
\end{theorem}
\begin{proof}
	It suffices to show that 
		\[ \sbcontractionof{\hfbasemeasure, \expof{\contractionof{
		\sencodingof{\mintermformulaset}\canparam
		}{
		\shortcatvariables
		}}}{\shortcatvariables} = \probat{\shortcatvariables} \, . \]
	For indices $\shortcatindices$ with $\probat{\shortcatvariables=\shortcatindices}=0$ we have $\hfbasemeasureat{\shortcatvariables=\shortcatindices}=0$ and thus also 
		\[ \sbcontractionof{\hfbasemeasure, \expof{\contractionof{
		\sencodingof{\mintermformulaset}\canparam
		}{
		\shortcatvariables
		}}}{\shortcatvariables=\shortcatindices} = 0 \, . \]
	For indices $\shortcatindices$ with $\probat{\shortcatvariables=\shortcatindices}>0$ we have $\hfbasemeasureat{\shortcatvariables=\shortcatindices}=1$ and
	\begin{align*}
		 \sbcontractionof{\hfbasemeasure, \expof{\contractionof{
		\sencodingof{\mintermformulaset}\canparam
		}{
		\shortcatvariables
		}}}{\shortcatvariables=\shortcatindices} 
		&= \prod_{\selindexof{[\catorder]}} \expof{\canparamat{\selvariableof{[\catorder]}=\selindexof{[\catorder]}} \cdot \mintermofat{\selindexof{[\catorder]}}{\shortcatvariables=\shortcatindices}} \\
		&=  \expof{\canparamat{\selvariableof{[\catorder]}=\shortcatindices}} \\
		&=  \probat{\shortcatvariables=\shortcatindices} \, .
	\end{align*}
\end{proof}




\subsection{Categorical Constraints}\label{sec:categoricalTN}

\red{Also called atomization of categorical variables.}

%% Categorical variables with more possibilities
We made the assumption that all categorical variables in factored systems to be represented by propositional logics take binary values (i.e. $\catdim=2$).
In cases where a categorical variable $\catvariable$ takes multiple values we define for each $\catindex$ an atomic formula $\catvariableof{\catindex}$ representing whether $\catvariable$ is assigned by $\catindex$ in a specific state.
	%\[ \catvariableof{\catindex} =  (\catvariable = \catindex \, . \] Confusing notation
Following this construction we have the constraint that exactly one of the atoms $\catvariableof{\catindex}$ is $1$ at each state.

%% Capture constraint
To capture the constraints resulting from this construction we introduce auxiliary parts. % of Bayesian Propositional Networks.
Such constraints can also be expressed by a formula but would result in an unnecessary large tensor network.


%% Categorical selection map
\begin{definition}[Categorical Constraint]
	Given a list $\catvariableof{0},\ldots,\catvariableof{\catdim-1}$ of binary variables and a categorical variable $\catvariable$ with dimension $\catdim$ a categorical constraint is a tensor $\categoricalmap[\catvariable,\catvariableof{[\catdim]}]$ defined as
%	A categorical constraint taking values in $[\catdim]$ maps to its atoms by
%		\[ \categoricalmap : [\catdim] \rightarrow \bigtimes_{\catindex\in[\catdim]}[2]  \]
	\begin{align*}
		 \categoricalmap(\catindex,\catindexof{\variableset}) 
		 = \begin{cases} 
		 	1 & \text{if} \quad \catindexof{[\catdim]} = \onehotmapof{\catindex} \\
			0 & \text{else} \, . 
		 \end{cases}
	\end{align*}
% where the $1$ is at position $\catindex$.
\end{definition}

%% Decomposition
With Theorem~\ref{the:functionDecompositionBasisCP} the tensor representation of $\categoricalcore$ decomposes in a basis CP format (see Figure~\ref{fig:CategoricalDecomposition}b) of if its coordinate maps $\categoricalmap_{\catindex}$, where $\catindex\in[\catdim]$.
For the cores
\begin{align}
	\categoricalcoreof{\catindex} = \onehotmapofat{\catindex}{\catvariable} \otimes \onehotmapofat{1}{\catvariableof{\catindex}} + (\onesat{\catvariable}- \onehotmapofat{\catindex}{\catvariable} ) \otimes \onehotmapofat{0}{\catvariableof{\catindex}} 
\end{align}	
we have by Theorem~\ref{the:functionDecompositionBasisCP}
\begin{align*}
	\rencodingofat{\categoricalmap}{\catvariable, \catvariableof{0}, \ldots, \catvariableof{\catdim-1}} 
	= \contractionof{\{\rencodingof{\categoricalmap(\catindex)} \, : \, \catindex\in[\catdim]\}}{\catvariable, \catvariableof{0}, \ldots, \catvariableof{\catdim-1}} \, . 
\end{align*}


In the next theorem we show how a categorical constraint can be enforced in a tensor network by adding the tensor $\categoricalmap$ to a contraction.

\begin{theorem}
	For any tensor $\hypercoreat{\shortcatvariables}$ and a categorical constraint defined by an ordered subset $\catvariableof{\variableset}\subset\shortcatvariables$, a variable $\catvariable\in\shortcatvariables$ we have
	\begin{align*}
	 	\contractionof{\{\hypercoreat{\shortcatvariables}, \categoricalmap\}}{\indexedcatvariables} 
		= \begin{cases}
			\hypercoreat{\indexedcatvariables} & \text{if} \quad \catindexof{\variableset} = \onehotmapof{\catindex} \\
			0 & \text{else} \, . 
		\end{cases}
	\end{align*}
\end{theorem}
\begin{proof}
	For any $\catindexof{[\atomorder]}$ we have
		\[ \contractionof{\{\hypercoreat{\shortcatvariables}, \categoricalmap\}}{\indexedcatvariables}  = 
			\hypercoreat{\indexedcatvariableof{[\atomorder]}} \cdot \categoricalmap[\indexedcatvariableof{},\indexedcatvariableof{\variableset}] \, . 
		\]
	If $\catindexof{\variableset} = \onehotmapof{\catindex}$ we have $\categoricalmap[\indexedcatvariableof{},\indexedcatvariableof{\variableset}] = 1$ and thus
		\[ \contractionof{\{\hypercoreat{\shortcatvariables}, \categoricalmap\}}{\indexedcatvariables}  =  \hypercoreat{\indexedcatvariableof{[\atomorder]}}  \, . \]
	If $\catindexof{\variableset} \neq \onehotmapof{\catindex}$ then $\categoricalmap[\indexedcatvariableof{},\indexedcatvariableof{\variableset}] = 0$ and  
		\[ \contractionof{\{\hypercoreat{\shortcatvariables}, \categoricalmap\}}{\indexedcatvariables}  = 0 \, . \]
\end{proof}




%We define the corresponding network cores as
%\begin{align}
%	\categoricalcoreof{\catindex} =
%	\begin{cases}
%		\tbasis & \text{ if } \catvariable = \catindex \\
% 		\fbasis & \text{ else }
%	\end{cases} \, . 
%\end{align}	

%Similar to atom selector tensor, but different core definition ($\fbasis$ instead of $\ones$, when $\parindexof{}$ is not matching the core position).

%% CONFUSING!
%We represent the categorical constraint as another variable $\randomxof{\categoricalmap}$, which when known defines the values of the corresponding variables $\randomxof{\atomenumerator}$ (see Figure~\ref{fig:CategoricalDecomposition}a).


\begin{figure}[h]
\begin{center}
	\begin{tikzpicture}[scale=0.35, thick] % , baseline = -3.5pt

\begin{scope}[shift={(-15,2)}]

\node[anchor=center] (text) at (-1,3) {${a)}$};


\node [circle, draw, thick, fill=gray!50] (T1) at (0,0) {\tiny $\randomxof{0}$};	
\node [circle, draw, thick, fill=gray!50] (T2) at (3,0) {\tiny $\randomxof{1}$};	
\node[anchor=center] (text) at (6,0) {\small $\cdots$};
\node [circle, draw, thick, fill=gray!50] (T3) at (9,0) {\tiny $\randomxof{\atomorder-1}$};	

\node [circle, draw, thick, fill=gray!50] (C) at (4.5,-5) {\tiny $\randomxof{\categoricalmap}$};	

\draw[->] (C) -- (T1);
\draw[->] (C) -- (T2);
\draw[->] (C) -- (T3);

\end{scope}

\node[anchor=center] (text) at (-1,5) {${b)}$};


\drawatomindices{0}{2}
\draw (-1,1) rectangle (5,-1);
\node[anchor=center] (text) at (2,0) {\small $\categoricalcore$};
\draw[->] (2,-3) -- (2,-1) node[midway,left] {\tiny $\randomxof{\categoricalmap}$};

\node[anchor=center] (text) at (7,0) {${=}$};


\begin{scope}[shift={(10,2)}]

\newcommand{\conposseldec}{4.5,-5.5}

\draw[fill] (\conposseldec) circle (0.25cm);
\draw (\conposseldec) -- (4.5,-7.5) node[midway, right] {\tiny $\randomxof{\categoricalmap}$};
%!TEX encoding = UTF-8 Unicode\draw[dashed] (3.5,-7.5) rectangle (5.5, -9.5);
%\node[anchor=center] (text) at (4.5,-8.5) {\small $\ones$};

\draw[<-] (0,1) -- (0,-1) node[midway,left] {\tiny $\randomxof{0}$};
\draw (-1,-1) rectangle (1, -3);
\node[anchor=center] (text) at (0,-2) {\small $\categoricalcoreof{0}$};
\draw[<-] (0,-3) to[bend right=20] (\conposseldec);


\draw[<-] (3,1) -- (3,-1) node[midway,left] {\tiny $\randomxof{1}$};
\draw (2,-1) rectangle (4, -3);
\node[anchor=center] (text) at (3,-2) {\small $\categoricalcoreof{1}$};
\draw[<-] (3,-3) to[bend right=20]  (\conposseldec);

\node[anchor=center] (text) at (6,-2) {$\cdots$};

\draw[<-] (9,1) -- (9,-1) node[midway,left] {\tiny $\randomxof{\atomorder-1}$};
\draw (7.75,-1) rectangle (10.25, -3);
\node[anchor=center] (text) at (9,-2) {\small $\categoricalcoreof{\atomorder-1}$};
\draw[<-] (9,-3) to[bend left=20]  (\conposseldec);




\end{scope}

		


\end{tikzpicture}
\end{center}
\caption{Representation of a categorical constraint in a $\cpformat$ Format tensor network.
	a) Representation of the dependency of the graphical model.
	b) Tensor Representation with further network decomposition.
	We average by contraction with the dashed tensor $\ones$, if we do not specify the active atom.
	}
\label{fig:CategoricalDecomposition}
\end{figure}

\begin{remark}[Combination of Constraints]
	We can combine constraint cores by Hadamard products in the dual tensor network representation, as long as they can be satisfied together.
	An example, where this is not the case, are the categorical constraints to the three sets
		\[ \{\randomxof{0},\randomxof{1},\randomxof{2},\randomxof{3}\} \, , \, \{\randomxof{0},\randomxof{1}\}\, ,\,\{\randomxof{2},\randomxof{3}\} \, . \] 
	Besides the categorical cores also the datacores have a similar bayesian network affecting the atoms by another hidden variable.
	Combining both is welldefined, only when all datapoints satisfy the categorical constraints (that is only one of the atoms in each constraint is active).
\end{remark}


\begin{example}[Sudoku]
	An interesting example, where categorical constraints are combined is Sudoku, the game of assigning numbers to a grid.
	For a $n\in\nn$ we define variables $\catvariableof{i,j,k}$ where $i,j,k\in[n^2]$ and $\catdimof{i,j,k}=2$.
	By understanding $i$ as a line index and $j$ as a column index, they are ordered in a grid as sketched in Figure~\ref{fig:sudokuGrid} in the case $n=3$.
	
%	We further define atomization variables
%		\[ \catvariableof{i,j,k} = (\catvariableof{i,j} == k) \, . \]
%	\red{These are also at each $i,j$ categorical constraints!}
	
	
	At each position there is a single number, that is for each $i,j\in[n^2]$ have a constraint
		\[ \{\catvariableof{i,j,k} \, : \, k \in [n^2] \} \]
	
	Here the $n^6$ variables are the random variables whether a specific position has a specific number assigned.
	The $3\cdot n^2$ constraints are 
	\begin{itemize}
		\item Each number $k$ appears exactly once in each row $i\in[n^2]$:
			\[ \{\catvariableof{i,j,k}  \, : \, j \in [n^2] \} \]
		\item Each number $k$ appears exactly once in each column $j\in[n^2]$:
			\[ \{\catvariableof{i,j,k}  \, : \, i \in [n^2] \} \]
		\item Each number appears exactly once in each square $s,r\in[n]$
			\[ \{\catvariableof{i+n\cdot s,j+n\cdot r,k}  \, : \, i,j \in [n] \} \]
	\end{itemize}
	
	In total we have $3\cdot n^2 + n^4$ constraints for $n^6$ variables.

	\begin{figure}\label{fig:sudokuGrid} % ! Still without k index of the variables.
	\begin{center}
		\begin{tikzpicture}[scale=0.9]
% Draw the main grid

%\node[anchor=center] (text) at (0,9) {${a)}$};

\draw[very thick] (0,0) rectangle (9,9); % Outer border
\foreach \x in {1,2,...,8} {
    \draw[thin] (\x,0) -- (\x,9); % Vertical lines
    \draw[thin] (0,\x) -- (9,\x); % Horizontal lines
}
% Thicker lines for 3x3 subgrids
\foreach \x in {3,6} {
    \draw[very thick] (\x,0) -- (\x,9); % Vertical thick lines
    \draw[very thick] (0,\x) -- (9,\x); % Horizontal thick lines
}

% Add variables in the middle of each square
\foreach \i in {0,1,...,8} {
    \foreach \j in {0,1,...,8} {
        \node[circle, draw, thick, fill=gray!50, inner sep = 0.5pt, minimum size=0.6cm, align=center] 
        at (\j+0.5,8-\i+0.5) {$X_{\i,\j}$};
    }
}

%\begin{scope}[shift={(2,0)}]
%
%\node[anchor=center] (text) at (9,9) {${b)}$};
%	
%% Draw a line of variables (horizontal)
%\foreach \k in {0,1,...,8} {
%    \node[circle, draw, thick,  fill=gray!50, inner sep=0pt, 
%    minimum size=0.6cm, align=center] 
%    at (10+\k,8.5) {$X_{i,\k}$};
%}
%
%\node[anchor=center] (text) at (9,7) {${c)}$};
%
%% Draw a column of variables (vertical)
%\foreach \l in {0,1,...,8} {
%    \node[circle, draw, thick, fill=gray!50, inner sep=0pt, 
%    minimum size=0.6cm, align=center] 
%    at (10,-\l+6) {$X_{\l,j}$};
%}
%
%\node[anchor=center] (text) at (9,7) {${d)}$};
%
%% Draw a square of variables (3x3)
%\foreach \i in {0,1,2} {
%    \foreach \j in {0,1,2} {
%        \node[circle, draw, thick, fill=gray!50, inner sep=0pt, 
%        minimum size=0.6cm, align=center] 
%        at (12+\j,6-\i) {$X_{\i,\j}$};
%    }
%}
%
%\end{scope}

\end{tikzpicture}

	\end{center}
	\caption{Sudoku grid of categorical variables, here drawn in the standard case of $n=3$, each with dimension $\catdim=n^2=9$.}
	\end{figure}

	\red{Reasoning by Entailment propagation! 
	Also, probabilistic choices possible when exact (!) contraction at a position not a basis vector, then can choose one possibility.}

\end{example}















% Training
\section{Unconstrained Parameter Estimation}\label{cha:parameterEstimation}

\red{In this chapter we investigate unconstrained parameter estimated for Markov Logic Networks and Hybrid Logic Networks, which are special cases of the backward maps introduced in Chapter~\ref{cha:probDecomposition}.
In the next chapter we impose sparsity constrains on the parameters.}

We estimate the canonical parameters $\canparam$ in an exponential family.
We first discuss parameter estimation in the more generic situation of exponential families and then discuss the more specific situation of Markov Logic Networks.

% Reduction to backward map
The parameters optimizing the likelihood, will be shown to coinciding by the backward mapping acting on the expectation of the sufficient statistics (see Theorem~\ref{the:parEstToBackwardMap}).
This is in most generality true for the parameters of the M-projection of any distribution onto the exponential family.
We therefore investigate methods to compute the backward mapping, in most generality by alternating algorithms and in the special case of Markov Logic Networks by closed form representations.



\subsection{Learning Markov Logic Networks} % Check for redundancy with the mln introduction chapter!



% Repetition and result transfering
Markov Logic Networks are exponential families with statistics by a set $\formulaset$ of propositional formulas.
We furthermore allow for propositional formulas as base measures, to also include the discussion of Hybrid Logic Networks.

%Based on this fact, we can apply the theory of probabilistic inference, developed in Chapter~\ref{cha:probReasoning}.

% Special example: MLE
The Maximum Likelihood Problem on MLNs is the M-projection
\begin{align*}
	\argmax_{\canparamat{\selvariable}\in\rr^{\seldim}} \quad 
	\centropyof{\probtensor}{\expdistof{(\sstat,\basemeasure,\canparam)}}	
\end{align*}
in the base $\probtensor=\empdistribution$ for a data map $\datamap$.

% Backward map
The M-projection coincides, after dropping constant terms in case of non-trivial base measure, with the backward map
\begin{align*}
	\argmax_{\canparamat{\selvariable}\in\rr^{\seldim}} \quad 
	\sbcontraction{\canparamat{\selvariable},\meanparamat{\selvariable}} - \cumfunctionof{\canparamat{\selvariable}} 
\end{align*}
where
\begin{align*}
	\meanparam{\selvariable} = \sbcontractionof{\sencodingof{\formulaset},\probtensor}{\selvariable} 
	\quad \text{and} \quad
	\cumfunctionof{\canparamat{\selvariable}} = \sbcontraction{\expof{ \sbcontractionof{\sencodingofat{\formulaset}{\shortcatvariables,\selvariable},\canparamat{\selvariable}}{\shortcatvariables} }, \basemeasure} \, . 
\end{align*}

% Special example: MaxEnt
The Maximum Entropy Problem for Markov Logic Networks is
\begin{align}
	\argmax_{\probtensor} \quad \sentropyof{\probtensor} 
	\quad \text{subject to} \quad  
	\sbcontractionof{\probtensor,\sencodingof{\formulaset}}{\selvariable}
	 =  \meanparamat{\selvariable} 
\end{align}


\begin{corollary}[of Theorem~\ref{the:maxEntMaxLikeDuality}]
	\red{Works only for $\meanparamat{\indexedselvariable}\in(0,1)$}
	Among all distributions $\probtensor$ of $\atomstates$ satisfying $\sbcontractionof{\probtensor,\sencodingof{\formulaset}}{\selvariable}
	 = \sbcontractionof{\empdistribution,\sencodingof{\formulaset}}{\selvariable}$ the Markov Logic Network with formulas $\formulaset$ and weights $\canparam$ being the solution of the maximum likelihood problem has minimal entropy.
\end{corollary}

% Unique property of MLN
We notice, that the solution of the maximum entropy problem is thus a Markov Logic Network.
This is remarkable, because this motivates our restriction to Markov Logic Networks as those distributions with maximal entropy given satisfaction rates of formulas in $\formulaset$.


\begin{remark}[Bayesian approach]
	% MAP
	\red{When treating $\canparam$ as a random tensor, which prior distribution is given and posteriori distribution wanted, we have a more involved Bayesian approach.}
	When having a prior $\probof{\mlnparameters}$ over the Markov Logic Networks we alternatively want to find the parameters $\mlnparameters$ solving the maximum a posteriori problem
	\begin{align}
		\argmax_{\mlnparameters} \mlnprobat{\data}\cdot \probof{\mlnparameters}\, . 
	\end{align}
\end{remark}



\subsection{Mean parameters}

The convex polytope of realizable mean parameters (see Definition~\ref{def:meanForwardBackward}) is for a statistic $\formulaset$ of propositional formulas
	\[ \meansetof{\formulaset} = \left\{ \sbcontractionof{\probtensor,\sencodingof{\formulaset}}{\selvariable} \, : \, \probtensor\in\probtensorset \right\} \, ,\]
where by $\probtensorset$ we denote the set of all probability distributions.

For any $\meanparam\in\meansetof{\formulaset}$ we have one of the following:
\begin{itemize}
	\item All $\meanparamat{\indexedselvariable}\notin\{0,1\}$: Then reproducible by a Markov Logic Network, $\meanparam\in\meanset$
	\item At least one $\meanparamat{\indexedselvariable}\notin\{0,1\}$ and one $\meanparamat{\indexedselvariable}\in\{0,1\}$: Then reproducible by aHybrid Logic Network
	\item All $\meanparamat{\indexedselvariable}\in\{0,1\}$: Then reproducible by a Hard Logic Network
\end{itemize}



\begin{lemma}\label{lem:meanHLNrealizabilityCondition}
	Let $\meanparam$ be in the boundary of $\meansetof{\formulaset}$, that is $\meanparam\in\closureof{\meanset}_{\formulaset}/\interiorof{\meanset}_{\formulaset}$.
	%Then for at least one $\selindexin$ we have $\meanparamat{\indexedselvariable}\in\{0,1\}$. %True?
	If $\meanparam\in\meansetof{\formulaset}$ then the formula
		\[ \basemeasureofat{\formulaset,\meanparam}{\shortcatvariables} \coloneqq \bigwedge_{\selindexin \, : \, \meanparamat{\indexedselvariable}\in\{0,1\}}
		\lnot^{(1-\meanparamat{\indexedselvariable})} \enumformulaat{\shortcatvariables} \]
	is satisfiable, that is $\sbcontraction{\basemeasureof{\formulaset,\meanparam}}>0$. % Can also use probabilistic entailment for that: $\probtensor\models\basemeasure$
\end{lemma}
\begin{proof}
	For any $\probtensor$ reproducing $\meanparam$ satisfies $\probtensor\models\enumformula$ if $\meanparamat{\indexedselvariable}=1$ and $\probtensor\models\lnot\enumformula$ if $\meanparamat{\indexedselvariable}=0$ and thus $\probtensor\models\basemeasureof{\formulaset,\meanparam}$.
	If $\basemeasureof{\formulaset,\meanparam}$ is not satisfiable, this would imply $\sbcontraction{\probtensor}=0$ which is a contradiction to $\probtensor$ being a probability distribution.
\end{proof}



\begin{figure}[h]\label{fig:}
\begin{center}
	\begin{tikzpicture}[scale=0.35]
    % Define points
    
    \node[below] at (-1,7) {$\interiorof{\meanset}_{\formulaset}$};
    
    \coordinate (A) at (0,0);
    
    \node[below] at (A) {$\meanparam_1$};
    \draw[fill] (A) circle (0.15cm);
    
    \coordinate (B) at (12,2.5);
    \path (A) -- (B) coordinate[pos=0.7] (P1);

    \node[below] at (P1) {$\meanparam_2$};
    \draw[fill] (P1) circle (0.15cm);
       
    \coordinate (P2) at (2,10); 
    \node[below] at (P2) {$\meanparam_3$};
    \draw[fill] (P2) circle (0.15cm);
    
    \coordinate (C) at (7.5,12);
    \path (B) -- (C) coordinate[pos=0.5] (P4);
    \node[left] at (P4) {$\closureof{\meanset}_{\formulaset}/\interiorof{\meanset}_{\formulaset}$};
    \coordinate (D) at (-3,12);
    \coordinate (E) at (-10,5);

    	\draw[thick] (A) -- (B) -- (C) -- (D) -- (E) -- cycle;

	\draw[dashed] (-10,0) -- (12,0) -- (12,12) -- (-10,12) -- (-10,0);
	    \node[left] at (-10,10) {$[0,1]^\seldim$};
\end{tikzpicture}


\end{center}
\caption{Sketch of the convex polytope ${\meansetof{\formulaset}}$ as a subset of the $\seldim$-dimensional cube $[0,1]^\seldim$ (here as a 2-dimensional projection) with example mean 	parameters $\meanparam_1,\meanparam_2,\meanparam_3$.
	The boundary points $\meanparam_1,\meanparam_2\in\closureof{\meanset}_{\formulaset}/\interiorof{\meanset}_{\formulaset}$ are examples of mean parameters, which can be realized by a Hard Logic Networks (respectively Hybrid Logic Network). %, if the criterion of Lem ma~\ref{lem:meanHLNrealizabilityCondition} is satisfied.
	Any extreme point $\meanparam_1\in\meansetof{\formulaset}\cup\{0,1\}^{\seldim}$ is realizable by a Hard Logic Network, while a non-extreme boundary point $\meanparam_2\in\meansetof{\formulaset}/\{0,1\}^{\seldim}$ is realizable by a Hybrid Logic Network.
	Any interior point $\meanparam_3\in\interiorof{\meansetof{\formulaset}}$ is realizable by a Markov Logic Network.
} 
\label{fig:DataDecomposition}
\end{figure}



\subsection{Parameter Estimation in Hybrid Networks}

\red{
Modify alternating weight optimization to deal with situations $\meanparamat{\indexedselvariable}\in\{0,1\}$: Add those as facts as base measures.
}

When there are facts, also situations $\hypercoreat{\catvariable=1}\in\{0,1\}$ can appear.
It that case the formula is entailed or contradicted by the facts, and dropping should be considered in both cases.

The max entropy - max likelihood duality still holds for hybrid logic networks as we show next.

\begin{theorem}
	Given a set of formulas $\tilde{\formulaset}$ and $\tilde{\meanparam}$, with coordinates $\tilde{\meanparam}_\selindex\in[0,1]$ in the closed interval $[0,1]$.
	If the corresponding maximum entropy problem is feasible, its solution is a hybrid logic network with 
	\begin{itemize}
		\item $\hardformulaset= \{\enumformula : \selindexin, \meanparamat{\indexedselvariable} = 1\} \cup  \{\lnot\enumformula : \selindexin, \meanparamat{\indexedselvariable} = 0\} $
		\item $\softformulaset = \{\enumformula : \selindexin, \meanparamat{\indexedselvariable} \in (0,1)\}$
		\item $\canparam$ being the backward map evaluated at the vector $\meanparam$ consisting of the coordinates of $\tilde{\meanparam}$ not in $\{0,1\}$
	\end{itemize}
\end{theorem}
\begin{proof}
	Feasible distributions have a density with base measure by $\hardformulaset$, we therefore reduce the set of distributions in the argmax to those with density to the base measure.
	The max entropy is a max entropy problem with respect to that base measure, where we only keep the constraints to the mean parameters different from $\{0,1\}$ (those are trivially satisfied).
	The statement then follows from the generic property (see Sec3.1 in \cite{wainwright_graphical_2008}).
\end{proof}










%% More general Exponential Families
%More generally sufficient statistics are any maps from the event space to define so called exponential families.
%Given a set $\formulaset$ of formulas, Markov Logic Networks are exponential families with $\formulaset$ being the sufficient statistics.





\subsection{Alternating Algorithms to Approximate the Backward Map}\label{sec:alternatingParEstMLN}

Let us now introduce an implementation of the Alternating Moment Matching Algorithm~\ref{alg:AMM} in case of Markov Logic Networks.
To solve the moment matching condition at a formula $\enumformula$ we refine Lemma~\ref{lem:mmContractionEquation} in the following.

\begin{lemma}\label{ref:lemMMinMLN}
	Let there be a base measure $\basemeasure$, a formula selecting map $\formulaset=\{\enumformula \, : \, \selindexin\}$ and weights $\canparam$, and choose $\selindexin$ such that $\enumformula  \notin \{\onesat{\shortcatvariables},\zerosat{\shortcatvariables}\}$.	
	The moment matching condition relative to $\canparam$, $\selindexin$ and $\datameanat{\indexedselvariable}\in(0,1)$ is then satisfied, if
	\begin{align} \label{sol:momentMatchingExformula}
	 	\weightat{\indexedselvariable} = \lnof{
		\frac{\datameanat{\indexedselvariable}}{(1-\datameanat{\indexedselvariable})} 
		\cdot \frac{\hypercoreat{\catvariableof{\enumformula }=0}}{\hypercoreat{\catvariableof{\enumformula }=1}} 
		} 
	\end{align}
	where by $\hypercoreat{\catvariableof{\enumformula }}$ we denote the contraction 
	\begin{align*}
	 	\hypercoreat{\catvariableof{\enumformula}} 
		= \contractionof{\{\rencodingof{\enumformula} \, : \, \selindexin\}
		\cup\{\headcoreof{\tilde{\selindex}} : \tilde{\selindex} \in [\seldim], \tilde{\selindex}\neq\selindex\}
		\cup\{\basemeasure\}}{\catvariableof{\enumformula}} \, . 
	\end{align*}
\end{lemma}
\begin{proof}
	Since $\imageof{\enumformula}\subset[2]$ we have
	\begin{align*}
		\idrestrictedto{\imageof{\enumformula}} = \onehotmapofat{1}{\catvariableof{\enumformula}}
	\end{align*}
	and the moment matching condition is by Lemma~\ref{lem:mmContractionEquation} satisfied if
	\begin{align*}
		\sbcontraction{\headcoreof{\selindex}, \onehotmapof{1}, \hypercore}
			= \sbcontraction{\headcoreof{\selindex},\hypercore} \cdot \datameanat{\indexedselvariable} \, . 
	\end{align*}
	This is equal to 
	\begin{align*}
		\expof{\canparamat{\indexedselvariable}} \cdot \hypercoreat{\catvariableof{\enumformula}=1}
		= \left( \expof{\canparamat{\indexedselvariable}} \cdot \hypercoreat{\catvariableof{\enumformula}=1} + \hypercoreat{\catvariableof{\enumformula}=0} \right) \cdot \datameanat{\indexedselvariable} \, . 
	\end{align*}
	Rearranging the equations this is equal to 
	\begin{align*}
	 	\hypercoreat{\catvariableof{\enumformula}} 
		= \contractionof{\{\rencodingof{\enumformula}\}
		\cup\{\headcoreof{\tilde{\selindex}} : \tilde{\selindex} \in [\seldim], \tilde{\selindex}\neq\selindex\}
		\cup\{\basemeasure\}}{\selvariable} \, . 
	\end{align*}
	We notice that the right side is well defined, since we have by assumption $\datameanat{\indexedselvariable}, (1- \datameanat{\indexedselvariable}) \neq 0$ and $\hypercoreat{\catvariableof{\enumformula}=0}, \hypercoreat{\catvariableof{\enumformula}=1} \neq 0$ since Markov Logic networks are positive distributions and $\enumformula \notin \{\onesat{\shortcatvariables},\zerosat{\shortcatvariables}\}$.
\end{proof}





%% OLD DERIVATION: Now as a special case of Exponential Families
%To solve Problem~\ref{prob:parameterMaxLikelihood} we choose a coordinate ascent approach.
%The partial derivative of the negative log-likelihood is
%\begin{align}
%	\frac{\partial}{\partial \weightof{\texformula}} \lossof{\formulaset,\weight} 
%	&= \frac{\partial}{\partial \weightof{\texformula}} \left[
%		\left(\sum_{\exformulain}\contractionof{\{\empdistribution,\weightof{\exformula}\cdot\exformula\}}{\varnothing} \right)
%		- \lnof{\contractionof{\{\expof{\weightof{\exformula}\cdot\exformula} \, : \, \exformulain\}}{\varnothing}}
%		\right] \\
%	& = 	\contractionof{\{\empdistribution,\texformula\}}{\varnothing} 
%		- \frac{\contractionof{\{\texformula \}\cup\{\expof{\weightof{\exformula}\cdot\exformula} \, : \, \exformulain\}}{\varnothing}}{
%		\contractionof{\{\expof{\weightof{\exformula}\cdot\exformula} \, : \, \exformulain\}}{\varnothing}
%		} \\
%	& = 	\essparof{\empdistribution}_{\texformula} - \essparof{\mlnprob}_{\texformula} \, .  
%\end{align}
%We notice that the last term is dependent on $\weightof{\exformula}$ and solve
%\begin{align}
%		\frac{\partial}{\partial \weightof{\texformula}} \lossof{\formulaset,\weight} = 0
%\end{align}
%which is equal to 
%\begin{align}\label{eq:momentMatchingExformula}
%	\essparof{\empdistribution}_{\texformula} = \essparof{\mlnprob}_{\texformula} \, .  
%\end{align}
%This is called a moment matching condition to the moment representing the formula $\exformula$.
%
%
%% Solution
%Equation \ref{eq:momentMatchingExformula} has a solution in closed form by
%\begin{align} \label{sol:momentMatchingExformula}
%	\weightof{\texformula} = \lnof{\frac{\essparof{\empdistribution}_{\texformula}}{\big(1-\essparof{\empdistribution}_{\texformula}\big)} \cdot \frac{z_1}{z_2} }  
%\end{align}
%where
%\begin{align}
%	\begin{bmatrix}
%	z_1 \\
%	z_2
%	\end{bmatrix}
%	= \contractionof{\{\texformula \}\cup\{\expof{\weightof{\exformula}\cdot\exformula} \, : \, \exformulain, \exformula\neq\texformula\}}{\randomxof{\texformula}} \, . 
%\end{align}



%% Hard network reference!
In the case $\datameanat{\indexedselvariable}\in\{0,1\}$ the moment matching conditions are not satisfiable for $\canparamat{\indexedselvariable}\in\rr$.
But, we notice, that in the limit $\canparamat{\indexedselvariable}\rightarrow \infty $ (respectively $-\infty$) we have
	\[ \meanparamat{\indexedselvariable} \rightarrow  1 \quad \text{(respectively $0$)}\, ,  \]
and the moment matching can be satisfied up to arbitrary precision.
In Chapter~\ref{cha:hardNetworks} we will allow infinite weights and interpret the corresponding factors by logical formulas.
As a consequence, we will able to fit graphical models, which we will call hybrid networks on arbitrary satisfiable mean parameters.

%
The cases $\hypercoreat{\catvariableof{\enumformula}=1}=0$, respectively $\hypercoreat{\catvariableof{\enumformula}=1}=0$ only appear for nontrivial formulas when the distribution is not positive. 
This is not the case for Markov Logic Networks, but will happen when formulas are added as cores of a Markov Network.
This situation will be investigated in Chapter~\ref{cha:hardNetworks}.


% Concave likelihood 
Since the likelihood is concave (see \cite{koller_probabilistic_2009}), there are not local maxima the coordinate descent could run into and coordinate descent will give a monotonic improvement of the likelihood. 

We suggest an alternating optimization by Algorithm~\ref{alg:AWO}, solving the moment matching equation iteratively for all formulas $\exformulain$ and repeat the optimization until a convergence criterion is met.


\begin{algorithm}[hbt!]
\caption{Alternating Weight Optimization (AWO)}\label{alg:AWO}
\begin{algorithmic}
%% INPUT: Numerated formula set, mean parameter $\datameanat{\selvariable}$
%\For{$\exformula\in\formulaset$}
\State $\kb = \ones$, $\secnodes=\varnothing$
\For{$\selindexin$}
	\If{$\meanparamat{\indexedselvariable}=1$}
		\[ \hardformulaset \algdefsymbol \hardformulaset \cup \{\enumformula\}\]
	%\EndIf
	\ElsIf{$\meanparamat{\indexedselvariable}=1$}
		\[ \hardformulaset \algdefsymbol \hardformulaset \cup \{\enumformula\}\]
	\Else
		\State $\secnodes\algdefsymbol \secnodes \cup \{\selindex\}$
		\State Compute
		\[ \hypercoreat{\catvariableof{\enumformula}}
		\algdefsymbol \sbcontractionof{\rencodingof{\enumformula}}{\catvariableof{\enumformula}} \]
		\State Set 
		\begin{align*}
	 		\canparamat{\indexedselvariable} 
			\algdefsymbol \lnof{
			\frac{\datameanat{\indexedselvariable}}{(1-\datameanat{\indexedselvariable})} 
			\cdot \frac{\hypercoreat{\catvariableof{\enumformula}=0}}{\hypercoreat{\catvariableof{\enumformula}=1}} 
			} 
		\end{align*}
	\EndIf
\EndFor
\If {$\sbcontraction{\kb}=0$}
	 \State \textbf{raise} "Inconsistent Knowledge Base"
\EndIf
%\For{$\selindexin$}
%	\State Compute
%		\[ \hypercoreat{\catvariableof{\enumformula}}
%		\algdefsymbol \sbcontractionof{\rencodingof{\enumformula}}{\catvariableof{\enumformula}} \]
%	\State Set 
%		\begin{align*}
%	 		\canparamat{\indexedselvariable} 
%			\algdefsymbol \lnof{
%			\frac{\datameanat{\indexedselvariable}}{(1-\datameanat{\indexedselvariable})} 
%			\cdot \frac{\hypercoreat{\catvariableof{\enumformula}=0}}{\hypercoreat{\catvariableof{\enumformula}=1}} 
%			} 
%		\end{align*}
%\EndFor
\While{Convergence criterion is not met}
%\For{$\exformula\in\formulaset$}
\For{$\selindex\in\secnodes$}
	\State Compute
	\begin{align*}
	 	\hypercoreat{\catvariableof{\enumformula}} 
		= \contractionof{\{\rencodingof{\enumformula} \, : \, \selindexin\}
		\cup\{\headcoreof{\tilde{\selindex}} : \tilde{\selindex} \in [\seldim], \tilde{\selindex}\neq\selindex\}
		\cup\{\basemeasure\}}{\catvariableof{\enumformula}}
	\end{align*}
	\State Set 
	\begin{align*}
	 	\canparamat{\indexedselvariable} = \lnof{
		\frac{\datameanat{\indexedselvariable}}{(1-\datameanat{\indexedselvariable})} 
		\cdot \frac{\hypercoreat{\catvariableof{\enumformula}=0}}{\hypercoreat{\catvariableof{\enumformula}=1}} 
		} 
	\end{align*}
	%\State Set $\weightof{\texformula}$ to the optimal weight when keeping $\weightof{\exformula}$ for $\exformula\neq\texformula$ constant, i.e. to the solution of the moment matching by Equation~\ref{sol:momentMatchingExformula}.
\EndFor
\EndWhile
\end{algorithmic}
\end{algorithm}


% Independent formulas
In the initialization phase of Algorithm~\ref{alg:AWO}, each parameters is initialized relative to a uniform distribution. 
The algorithm would be finished, if the variables $\catvariableof{\exformula}$ are independent.
This would be the case, if the Markov Logic Network consists of atomic formulas only.
When they fail to be independent, the adjustment of the weights influence the marginal distribution of other formulas and we need an alternating optimization.
% 
This situation corresponds with couplings of the weights by a partition contraction, which does not factorize into terms to each formula.


% Inference
Solving Equation~\ref{sol:momentMatchingExformula} requires inference of a current model by answering the query to the formula $\texformula$.
This can be a bottleneck and circumvented by approximative inference, see e.g. CAMEL \cite{ganapathi_constrained_2008}.



\begin{remark}[Grouping of coordinates with trivial sum]
	When having a set of coordinates, such that the coordinate functions are binary and sum to the trivial tensor, one can find simultaneous updates to the canonical parameters, such that the partition function is staying invariant.
	Given a parameter $\canparam^t$ we compute
		\[ \meanparam^t = \contractionof{\expdistof{(\sstat,\canparam^t)}, \sstat}{\selvariable} \]
	and build the update
		\[ \canparam^{t+1} = \canparam^t + \lnof{\meanparam^{\datamap}}{\meanparam^t} \, . \]
	Then, $\canparam^{t+1}$ satisfies the moment matching equations for all coordinates in the set.
	
	
	The assumptions are met when taking all features to any hyperedge in a Markov Network seen as an exponential family.
	In that case, the update algorithm is refered to as  Iterative Proportional Fitting \cite{wainwright_graphical_2008}.
	Further, when activating both $\exformula$ and $\lnot\exformula$.
\end{remark}


\subsection{Forward and backward mappings of MLNs in closed form}

\red{We recall from Chapter~\ref{cha:probReasoning}, that while forward mappings are always in closed form by contractions, backward mapping in general fail.
We here investigate specific examples, where closed forms can be derived for both.}

We have formulated parameter estimation as a maximum entropy problem constrained to matching expected sufficient statistics.
Let us discuss situations, where the forward and backward mappings are available in closed form and parameter estimation can thus be solved by application of the inverse on the expected sufficient statistics with respect to the empirical distribution.
% Usage
When the backward map $\backwardmap$ is available in closed form, we directly get optimal parameters by the inversion acting on the satisfaction rate and can avoid iterative algorithms of parameter estimation.

\subsubsection{Maxterms and Minterms}

Minterms (respectively maxterms) are ways in propositional logics to get a syntactical formula representation based on a formula to each world which is a model (respectively fails to be a model).
We have already studied in Section~\ref{sec:MLNMaxMintermRep} how to represent any distribution as a MLN of maxterms (respectively minterms), see Theorem~\ref{the:maximalClausesRepresentation}.

We use the tuple enumeration of the maxterms and minterms by $\atomstates$ introduced in Section~\ref{sec:termClauseDecomposition}.
With respect to this enumeration the canonical parameters and mean parameters are tensors in $\bigotimes_{\atomenumeratorin}\rr^2$. 
%% Interpretation of the mean parameters
Since the statistic of the minterm family is the identity, the mean parameters for the minterm family are
	\[ \meanparamat{\selvariableof{[\atomorder]}=\catindexof{[\atomorder]}} 
	= \probat{\catindexof{[\atomorder]}} 
	\]
and therefore after a relabeling of categorical variables to selection variables $\meanparam=\probtensor$.
For maxterms we have analogously
	\[ \meanparamat{\selvariableof{[\atomorder]}=\catindexof{[\atomorder]}} 
	= 1-\probat{\catindexof{[\atomorder]}} 
	\]
and $\meanparam = \onesat{}-\probtensor$.
We can use these insights to provide a characterization of the forward and backward maps of the minterm and maxterm family.

\begin{theorem}
	Given the Markov Logic Networks to the formula sets
		\[ \mintermformulaset := \{ \mintermof{\atomindices} \, : \, \atomindicesin\} \quad \text{and} \quad 
		\maxtermformulaset := \{ \maxtermof{\atomindices} \, : \, \atomindicesin\}  \]
	of all minterms, respectively of all mapterms, the forward mapping are
		%\[ \forwardmapwrt{\mintermformulaset}: \bigotimes_{\atomenumeratorin}\rr^{2} \rightarrow \bigotimes_{\atomenumeratorin}\rr^{2} \]
		\[ \forwardmapwrt{\mlnmintermsymbol}(\canparam) = \normationofwrt{\expof{\canparam}}{\shortcatvariables}{\varnothing}  
		\quad \text{and} \quad 
		 \forwardmapwrt{\mlnmaxtermsymbol}(\canparam) = \normationofwrt{\expof{-\canparam}}{\shortcatvariables}{\varnothing} \, , \]
	where in a slight abuse of notation we assigned the variables $\shortcatvariables$ to the canonical parameters $\canparam$.

	Possible choices of the backward mappings are
		%\[ \backwardmapwrt{\mlnmintermsymbol}: \bigotimes_{\atomenumeratorin}\rr^{2} \rightarrow \bigotimes_{\atomenumeratorin}\rr^{2} \]
		\[ \backwardmapwrt{\mlnmintermsymbol}(\meanparam) = \lnof{\meanparam} 
			\quad \text{and} \quad 
			\backwardmapwrt{\maxtermformulaset}(\meanparam) = -\lnof{\meanparam} \, .
		 \]
\end{theorem}
\begin{proof}
	For the minterms we use that
		\[ \mintermformulaset[\shortcatvariables,\catvariableof{\mintermformulaset}]  = \identityat{\shortcatvariables,\catvariableof{\maxtermformulaset}}\] 
	and get
		\[ \forwardmapwrt{\mlnmintermsymbol}(\canparam) 
		= \normationof{
		\expof{\contractionof{\{\mintermformulaset, \canparam\}}{\shortcatvariables}}
		}{\shortcatvariables}
		= 
		\normationof{\expof{\canparam}}{\shortcatvariables} \, . 
		\]
	
	We notice that for any $\meanparam$ in the image of the forward map we have
		\[ \forwardmapwrt{\mlnmintermsymbol}(\backwardmapwrt{\mlnmintermsymbol}(\meanparam)) = \meanparam \]
	%We notice that for any $\canparam\in\rr^{\seldim}$ we have
	%	\[ \backwardmapwrt{\mlnmintermsymbol}( \forwardmapwrt{\mlnmintermsymbol}(\canparam) ) = \canparam - \contractionof{\expof{\canparam}}{\varnothing} \cdot \onesat{\shortcatvariables}
	%	\]
%	and thus $\canparam$ and $\backwardmapwrt{\mintermformulaset}( \forwardmapwrt{\mintermformulaset}(\canparam) )$ are representing the same member of the exponential family (see Theorem~\ref{the:tensorRepUniqueness}).	
	Therefor, $\backwardmapwrt{\mintermformulaset}$ is indeed a backward mapping to the exponential family of minterms.
	
	For the maxterms we use that
		\[ \maxtermformulaset[\shortcatvariables,\catvariableof{\maxtermformulaset}] = \onesat{\shortcatvariables,\catvariableof{\maxtermformulaset}}-\identityat{\shortcatvariables,\catvariableof{\maxtermformulaset}} \]
	and get
	\begin{align*}
		\forwardmapwrt{\mlnmaxtermsymbol}(\canparam) 
		& = \normationof{
		\expof{\contractionof{\{\mintermformulaset, \canparam\}}{\shortcatvariables}}
		}{\shortcatvariables} \\
		& = \normationof{\{
		\expof{\contractionof{\{\ones, \canparam\}}{\shortcatvariables}}, 
		\expof{-\contractionof{\canparam}{\shortcatvariables}} \}
		}{\shortcatvariables} \\
		& = \normationof{
		\expof{-\canparam}
		}{\shortcatvariables}
	\end{align*}
	where we used, that $\expof{\contractionof{\{\ones, \canparam\}}{\shortcatvariables}}$ is a multiple of $\onesat{\shortcatvariables}$ and is thus eliminated in the normation.
	For any $\meanparam\in\imageof{\forwardmapwrt{\mlnmaxtermsymbol}}$ we have
		\[ \forwardmapwrt{\mlnmaxtermsymbol}(\backwardmapwrt{\mlnmaxtermsymbol}(\meanparam) ) 
		= \meanparam
		%-\lnof{\expof{-\canparam}} + \contractionof{\expof{-\canparam}}{\varnothing} \cdot \onesat{\shortcatvariables}
		%= \canparam + \contractionof{\expof{-\canparam}}{\varnothing} \cdot \onesat{\shortcatvariables}
		\]
	and $\backwardmapwrt{\mlnmintermsymbol}$ is thus a backward map for the exponential family of maxterms.
\end{proof}

	
% Fitting arbitrary distributions
Any positive probability distribution can thus be fitted by minterms when we choose $\canparam=\lnof{\probtensor}$, respectively by maxterms when we choose $\canparam=\ones-\lnof{\probtensor}$.
Thus, we have identified a subset of $2^{\atomorder}$ formulas, which is rich enough to fit any distribution.





\subsubsection{Atomic formulas}

% Repeat atomic formulas
Let us now derive a closed form backward mapping for the statistic
	\[ \atomformulaset := \{\atomicformulaof{\atomenumerator}: \atomenumeratorin\} \, . \]

The mean parameters coincide with the queries on the atomic formulas, that is the marginal 
	\[ \meanparamat{\selvariable=\atomenumerator} = \probat{\catvariableof{\atomenumerator}=1}  \, . \]

\begin{theorem}
	Given a Markov Logic Network with the statistic $\atomformulaset$ of atomic formulas, the forward mapping from canonical parameters to mean parameters is the coordinatewise sigmoid, that is
		\[ \forwardmapwrtof{\mlnatomsymbol}{\canparamat{\selvariable}} = \frac{\expof{\canparamat{\selvariable}}}{\onesat{\selvariable}+\expof{\canparamat{\selvariable}}}   \]
	where the quotient is performed coordinatewise.

	A backward mapping is the coordinatewise logit, that is
		\[ \backwardmapwrt{\mlnatomsymbol}(\meanparamat{\selvariable}) 
		= \lnof{\frac{
			\meanparamat{\selvariable}
			}{
			\onesat{\selvariable}-\meanparamat{\selvariable}
			}}  \, . \]
\end{theorem}
\begin{proof}
	We have for any $\canparamat{\selvariable}\in\rr^{\atomorder}$
		\[ \probofat{(\atomformulaset,\canparam)}{\shortcatvariables} 
		= \bigotimes_{\atomenumeratorin} \normationof{\expof{\canparamat{\selvariable=\atomenumerator}\cdot \atomicformulaof{\atomenumerator}}}{\catvariableof{\atomenumerator}}  \, . \]

	
	For any $\atomenumeratorin$ it therefore holds, that
	\begin{align*}
		\forwardmapwrtof{\mlnatomsymbol}{\canparamat{\selvariable}}[\selvariable=\atomenumerator] 
		&=\sbcontraction{\atomicformulaof{\atomenumerator},  \probofat{(\atomformulaset,\canparam)}{\shortcatvariables}} \\
		&=\sbcontraction{\atomicformulaof{\atomenumerator},  \normationof{\expof{\canparamat{\selvariable=\atomenumerator}\cdot \atomicformulaof{\atomenumerator}}}{\catvariableof{\atomenumerator}}} \\
		& = \frac{\expof{\canparamat{\selvariable=\atomenumerator}}}{1+\expof{\canparamat{\selvariable=\atomenumerator}}} \, .
	\end{align*}

	Since the coordinatewise logit is the inverse function of the coordinatewise sigmoid the map
	\begin{align*}
		\backwardmapwrtof{\mlnatomsymbol}{\meanparamat{\selvariable}}[\selvariable=\atomenumerator] 
		& = \lnof{\frac{\meanparamat{\selvariable=\atomenumerator}}{1- \meanparamat{\selvariable=\atomenumerator}}}
	\end{align*}
	satisfies for any $\meanparam$ in the image of the forward map
	\begin{align*}
		\forwardmapwrt{\mlnatomsymbol}(\backwardmapwrt{\mlnatomsymbol}(\meanparam)) = \meanparam 
	\end{align*}
	and is therefore a backward map.
\end{proof}


% Representation by selection tensor networks
In a selection tensor networks they are represented by a single neuron with identity connective and variable selection to all atoms.
\red{To architectures: Redo the discussions on these examples.}
	
% Interpretation of the result
The maximum likelihood estimator of a positive probability distribution by the MLN of atomic formulas is therefore the tensor product of the marginal distributions.


\begin{remark}
	\red{By Independence Decomposition we reduce to a system of atomic MLN.
	The minterms of such MLNs are the literals.
	By redundancy (literals sum up to $\ones$), it suffices to take only the positive or the negative literal.}
%	We set the weights of $\weightof{\lnot\atomicformulaof{\atomenumerator}}=0$ (corresponding with a gauge normation of the energy offset symmetry). % Not needed!
\end{remark}

















\subsection{Constrained parameter estimation in the minterm family}

% Naive exponential family
We approach structure learning as constrained parameter estimation in the naive exponential family (see Example~\ref{exa:mintermExpFamily}), which coincides with the minterm family $\formulasetof{\mlnmintermsymbol}$.
The minterm family is defined by the statistic $\sstat = \identityat{\shortcatvariables, \selvariableof{[\catorder]}}$ and has energy tensors coinciding with the canonical parameters.

% Convex polytope characterization
\red{For the minterm family, we have as mean parameter set the convex hull of one-hot encodings. 
Each basis vector is an extreme point is an extreme point.
}


By Theorem~\ref{the:mintermExpressivityMLN} all positive distributions are member of the minterm markov logic network family.
This expressivity result was generalized to arbitrary distributions, when allowing for formulas as basemeasures by Theorem~\ref{the:mintermExpressivityHLN}.

Finding the distribution maximizing the likelihood of data would then be the empirical distribution.
In this case we would have $\datameanat{\selvariableof{[\catorder]}=\shortcatindices} = \empdistributionat{\shortcatvariables=\shortcatindices}$ and the maximum likelihood distribution is found by the problem
\begin{align*}
	\argmax_{\canparam\in\facspace}  \sbcontraction{\canparam,\empdistribution} - \cumfunctionof{\canparam} \, 
\end{align*}
which is solved at $\canparam=\lnof{\empdistribution}$ with $\probtensorof{(\identity,\lnof{\empdistribution})}= \empdistribution$.
This follows from $\lossof{\probtensorof{(\identity,\canparam)}}=\kldivof{\empdistribution}{\probtensorof{(\identity,\canparam)}}$, which is by Gibbs inequality minimized at $\probtensorof{(\identity,\canparam)}=\empdistribution$, which is the case for $\canparam = \lnof{\empdistribution}$.

We here allow for $\lnof{0}=-\infty$, with the convention of $\expof{-\infty}=0$, to handle datasets where specific worlds are not represented. 
\red{Better: Use Theorem~\ref{the:mintermExpressivityHLN} with basemeasure dropping non appearing data.}


% Regularization
To avoid this overfitting situation, we regularize by restricting the parameter to be a set $\energyhypothesis\subset\facspace$ and state
\begin{align}\tag{$\mathrm{P}_{\energyhypothesis, \empdistribution}$}\label{prob:restrictedNaiveMLE}
	\argmax_{\canparam\in\energyhypothesis}  \sbcontraction{\canparam,\empdistribution} - \cumfunctionof{\canparam} \, . 
\end{align}

Problem~\ref{prob:restricedNaiveMLE} has two important types of instantiation, which we discuss in the next sections.

\subsubsection{Parameter Estimation}

% Parameter Estimation
\red{Projecting onto the markov logic family to the statistic $\formulaset$ is the instance of Problem~\ref{prob:restricedNaiveMLE} with the hypothesis choice}
%When the $\formulaset$ is known we take $\energyhypothesis$ as the linear hull 
	\[ \energyhypothesisof{\formulaset} = \spanof{\{\formula : \formula\in\formulaset \}} \, . \]
Then, the problem is the parameter estimation problem studied in Section~\ref{sec:parameterEstimation}.
To see this, we reparametrize by the coefficient vectors of the elements in the span, which are then understood as the canonical parameter of the respective distribution in the markov logic family to $\formulaset$.


\begin{remark}[Overparametrization]
	Taking $\formulaset$ to consist of all propositional formulas, we get a massive overparametrization: 
	The essential statistics maps to a $2^{\left(2^\atomorder \right)}$ dimensional real vector space.
	All possible distributions of the $\atomorder$ atomic variables are mapped to an $2^\atomorder-1$ dimensional submanifold, where also the essential statistics maps to.

	Thus, to identify probabilistic knowledge bases, we need to drastically restrict the shape of formulas allowed.
	It is in principle impossible to decide which formulas to be activated, based only on statistics and not on prior assumptions.

	%The nodes of a Markov Propositional Network are all formulas in a propositional theory and the hyperedges all possible decompositons.
	When having $\atomorder$ atoms, there are $2^{\atomorder}$ states in the factored system.
	Since each state can either be a model of a formula or not, there are
		\[ \cardof{\formulaset} = 2^{\big(2^\atomorder \big)} \]
	formulas.
	Having, for example, $\atomorder=10$, then $\cardof{\formulaset}>10^{308}$.


	% Regularization by sparsity
	One regularization is by allowing only a small number of formulas to be active.
	This corresponds with regularization with $\sparsityof{\canparam}$.
	The problem is then non-convex.


	% Regularization by formula size
	A further regularization strategy is the restriction of the size of the possible formulas to maintain interpretability. 
	Thus, we choose small formula selection networks.
\end{remark}




\subsubsection{Structure Learning}

% Structure Learning
The problem of structure learning arises, when the set of parameters in Problem~\ref{prob:restricedNaiveMLE} is choosen as 
	\[ \energyhypothesisof{\formulasuperset}= \bigcup_{\formulaset\in\formulasuperset} \spanof{\formulaset} \, .  \] %\energyhypothesisof{\formulaset}\, . 
In this case, the problem in general fails to be convex.

% Subspace instuition
Each formula set $\formulaset$ represents a subspace in the parameters of the minterm family, which is spanned by the propositional formulas $\exformula\in\formulaset$.

%\red{Intuition by subspaces in the minterm parameters, which are selected by a nonlinear objective, to distinguish from compressed sensing.}







\subsection{Greedy Structure Learning}


%Motivation 
It can be impracticle to learn all formulas at once, since the set $\formulasuperset$ often grows combinatorically, for example when choosing as a powerset of formulas.
\red{Further, we need to avoid overfitting and carefully choose a hypothesis.}
To avoid intractabilities and overfitting, one can choose a greedy approach and learn in addition formulas $\exformula$ when already having learned a set $\formulaset$ of formulas.
We in this section assume a current model $\currentdistribution$, which is a generic positive distribution not necessarily a Markov Logic Network. % or Hybrid Logic Network.

% 
We will use the effective selection tensor network representation of exponentially many formulas described in Chapter~\ref{cha:formulaBatches} and select from them a small subset.

%\red{Alternative discussion: Can use current distribution as base measure and apply moment matching as first order condition.}


\subsubsection{Greedy formula inclusions}

Having a current set of formulas $\formulaset$ we want to choose the best $\formula\in\fselectionmap$ to extend the set of formulas to $\formulaset\cup\{\formula\}$ in a way minimizing the cross entropy.
Given this, add each step we solve the greedy cross entropy minimization
\begin{align}\label{prob:perfectGreedy}\tag{$\mathrm{P}_{\datamap,\formulaset,\fselectionmap}$}
	\argmin_{\formula\in\fselectionmap} \argmin_{\canparam\in\rr^{\cardof{\formulaset}+1}} 
	\centropyof{\empdistribution}{\expdistof{(\formulaset\cup\{\formula\},\canparam,\basemeasure)}} \, . 
\end{align}


A brute force solution would require parameter estimation for each candidate in $\fselectionmap$.
We provide two more efficient approximative heuristics in the following (see Chapter~20 in \cite{koller_probabilistic_2009}).


\subsubsection{Gain Heuristic}

In the gain heuristic, only the parameters of the new formula are optimized and the others left unchanged.
This amounts to 
\begin{align}\label{prob:greedyGain}\tag{$\mathrm{P}^{\mathrm{gain}}_{\datamap,\formulaset,\fselectionmap}$}
	\argmin_{\formula\in\fselectionmap} \left ( \min_{\canparamat{\cardof{\formulaset}}\in\rr} 
	\centropyof{\empdistribution}{\expdistof{(\formulaset\cup\{\formula\},\canparam,\basemeasure)}} \right) \, . 
\end{align}
Here we denote by $\canparam$ the first $\cardof{\formulaset}$ coordinates of the M-projection $\currentdistribution$  of $\empdistribution$ onto $\formulaset$ and the variable new coordinate at position $\canparamat{\cardof{\formulaset}}$.

\begin{lemma}
	The gain heuristic objective is an upper bound on the true greedy objective. 
\end{lemma}
\begin{proof}
Since
\begin{align*}
	\argmin_{\formula\in\fselectionmap} \left( \argmin_{\canparam\in\rr^{\cardof{\formulaset}+1}} 
	\centropyof{\empdistribution}{\expdistof{(\formulaset\cup\{\formula\},\canparam,\basemeasure)}} \right)
	\leq 	\argmin_{\formula\in\fselectionmap} \left ( \argmin_{\canparamat{\cardof{\formulaset}}\in\rr} 
	\centropyof{\empdistribution}{\expdistof{(\formulaset\cup\{\formula\},\canparam,\basemeasure)}} \right) \, . 
\end{align*}
\end{proof}


% Minterm family interpretation
Further, this is Problem~\eqref{prob:restrictedNaiveMLE} in the case
\begin{align*}
	\energyhypothesis = \lnof{\currentdistribution} + \cup_{\formula\in\formulaset} \spanof{\formula} \, .
\end{align*}



% For single formula
Let us choose a formula $\formula\in\formulaset$ and consider Problem~\ref{prob:restrictedNaiveMLE}  in the case
\begin{align*}
	\energyhypothesisof{\formula} = \lnof{\currentdistribution} + \spanof{\formula} \, . 
\end{align*}
This is parameter estimation on the exponential family with the single feature $\formula$ and the base measure $\currentdistribution$.
Therefore we can apply the theory of Chapter~\ref{cha:probReasoning} and characterize the solution by the $\weight$ satisfying the moment matching condition
\begin{align*}
	\contraction{\currentdistribution, \normationof{\expof{\weight}}{\shortcatvariables} } = \contraction{\empdistribution, \formula} \, . 
\end{align*}
We state the solution of this condition in the next theorem.

\begin{theorem}
	Problem~\eqref{prob:greedyGain} is solved at any
	\begin{align*}
		\hat{\canparam} = \weightof{\hat{\formula}} \cdot \hat{\formula}
	\end{align*}
	where the formula $\hat{\formula}$ is in
	\begin{align*}
		\hat{\formula} \in \argmax_{\formula\in\formulaset} \kldivof{\sbcontraction{\empdistribution,\formula}}{\sbcontraction{\currentdistribution,\formula}}
	\end{align*}
	and $\weightof{\hat{\formula}}$ is the weight of $\hat{\formula}$ in the solution of Problem~\ref{prob:restrictedNaiveMLE} with $\Gamma = \currentdistribution + \mathrm{span}(\exformula)$.
	Here we denote by $\kldivof{p_1}{p_2}$ the Kullback-Leibler divergence between Bernoulli distributions with parameters $p_1,p_2\in[0,1]$, that is
		\[ \kldivof{p_1}{p_2} = p_1 \cdot \lnof{\frac{p_1}{p_2}} + (1-p_1) \cdot \lnof{\frac{(1-p_1)}{(1-p_2)}}  \]
\end{theorem}	
\begin{proof}
	% Solution of the problem restricted to 
	For any formula $\formula$, the inner minimum of Problem~\eqref{prob:greedyGain} is by Lemma~\ref{ref:lemMMinMLN} taken at 
		\[ \weightof{\formula} = \lnof{\frac{\datamean}{(1-\datamean)}\cdot \frac{(1-\currentmean)}{\currentmean}}  \]
	where
		\[ \currentmean = \sbcontraction{\currentdistribution,\formula} \]
	and
		\[ \datamean = \sbcontraction{\empdistribution,\formula} \, . \]
	
	The difference of the likelihood at the current distribution and the optimum is
	\begin{align*}
		\centropyof{\empdistribution}{\currentdistribution}
		- \centropyof{\empdistribution}{\expdistof{(\extendedformulaset,\extendedcanparam,\basemeasure)}}
		= \datamean \cdot \weightof{\formula} - \cumfunctionwrtof{\extendedformulaset,\basemeasure}{\extendedcanparam} \, .
	\end{align*}
	
	% Loss gain at optimum
	We use the representation scheme of Theorem~\ref{the:hybridNetworkRepresentation} and get
	\begin{align*}
		\sbcontraction{\currentdistribution, \expof{\weightof{\formula} \cdot \formula}}
		& = \sbcontraction{\currentdistribution, \rencodingofat{\formula}{\catvariableof{\formula}}, \headcoreofat{\formula}{\catvariableof{\formula}}} \\
		& = (1-\currentmean) + \currentmean\cdot \expof{\weightof{\formula}} \\
		& = (1 - \currentmean) + \frac{\datamean \cdot (1-\currentmean)}{(1-\datamean)} \\
		& = (1-\currentmean) \cdot \frac{1}{(1-\datamean)} \, . 
	\end{align*}
	% Refining the cumulant term
	It follows, that
	\begin{align*}
		\cumfunctionwrtof{\extendedformulaset,\basemeasure}{\extendedcanparam}
		& = \lnof{\sbcontraction{\currentdistribution, \expof{\weightof{\formula} \cdot \formula}}} \\
		& = \lnof{1-\currentmean} - \lnof{1-\datamean} \, . 
	\end{align*}
	% Refining the mean product term
	We further have
	\begin{align*}
		\datamean \cdot \weightof{\formula}
		= \datamean \cdot \left[ \lnof{\frac{\datamean}{(1-\datamean)}\cdot \frac{(1-\currentmean)}{\currentmean}}  \right]	
		= \datamean \lnof{\datamean} - \datamean \lnof{1-\datamean} + \datamean \lnof{1-\currentmean} - \datamean \lnof{\currentmean}
	\end{align*}
	and arrive at
	\begin{align*}
		\centropyof{\empdistribution}{\currentdistribution}
		- \centropyof{\empdistribution}{\expdistof{(\exformula,\weightof{\formula},\currentdistribution)}}
		& =  \datamean \lnof{\datamean} - \datamean \lnof{1-\datamean} + \datamean \lnof{1-\currentmean} - \datamean \lnof{\currentmean}
		-  \lnof{1-\currentmean} - \lnof{1-\datamean} \\
		& = \left( -\datamean \lnof{\currentmean} - (1-\datamean) \lnof{1-\currentmean} \right)  - \left( -\datamean \lnof{\datamean} - (1-\datamean) \lnof{1-\datamean} \right) \, . 
	\end{align*}
	By definition, this is the Kullback-Leibler divergence between Bernoulli distributions with parameters $\datamean$ and $\currentmean$.
	%
	Since the gain in the likelihood loss when restricting to $\energyhypothesis = \spanof{\formula}$ is thus given by $\kldivof{\sbcontraction{\empdistribution,\formula}}{\sbcontraction{\currentdistribution,\formula}}$, we have that Problem~\ref{prob:restrictedNaiveCE}  in the case $\energyhypothesis = \bigcup_{\formula\in\formulaset}\spanof{\formula}$ is solved at $\estcanparam = \weightof{\hat{\formula}}\cdot \hat{\formula}$ where
		\[ \hat{\formula} = \kldivof{\sbcontraction{\empdistribution,\formula}}{\sbcontraction{\currentdistribution,\formula}} \, . \]
\end{proof}

\red{Thus, we solve the grain heuristic with a coordinatewise transform of the mean parameter tensors to $\empdistribution$ and $\currentdistribution$, using the Bernoulli Kullback-Leibler divergence as transform function.}


% Interpretation
One therefore takes the formula, which marginal distribution in the current model and the targeted distribution are differing at most, measured in the KL divergence.

% Optimization method
One optimization method would thus be the computation of the mean parameters to both distribution, building the coordinatewise KL divergence and choosing the maximum. 
Since we need to evaluate each coordinate, this can be intractable for large sets of formulas.


% Further weight optimization
Further improvement of the model can be achieved by iteratively optimizing the other weights as well, since their corresponding moment matching conditions might be violated after the integration of a new formula.
This would require the computation of backward mappings for each candidate formula, for which we only have an alternating approach in general.



\subsubsection{Gradient heuristic and the proposal distribution}

\red{Advantage: Might avoid formulawise calculus, when sampling from proposal distribution. 
Brute force solution of gain heuristic require formulawise approach.}

We now derive a heuristic of choosing features based on the maximal coordinate of the gradient when differentiating the canonical parameter in the minterm family.
To prepare for this, we build the gradient of the loss
%For the naive exponential family 
\begin{align*}
	\lossof{\expdistof{(\naivestat, \naivecanparam)}} 
	%= \frac{1}{\datanum} \sum_{\dataindexin}\lnof{\expdistofat{(\naivestat, \naivecanparam)}{\shortcatvariables=\datamapof{\dataindex}}}
	= \contraction{\empdistribution, \sencodingof{\naivestat}, \naivecanparam} - \lnof{\contraction{\expof{\contractionof{\sencodingof{\naivestat}, \naivecanparam}{\shortcatvariables}}}} 
\end{align*}
as
\begin{align*}
	\gradwrt{\naivecanparamat{\selvariable}} \lossof{\expdistof{(\naivestat, \naivecanparam)}}
	&= \contractionof{\sencodingof{\naivestat},\empdistribution}{\selvariable} - \contractionof{\sencodingof{\naivestat},\expdistof{(\naivestat, \naivecanparam)}}{\selvariable} \\
	&= \empdistribution - \expdistof{(\naivestat, \naivecanparam)} \, . 
\end{align*}

%% Single feature
%Given a feature $\exfunction[\shortcatvariables]$ we vary the naive parameters by a function on $\canparam\in\rr$ by
%\begin{align*}
%	 \naivestat(\canparam) %=  \mlntensor + \weight_{\parindices} \ftensorof{\exformula_{\parindices}}
%	= \naivestat(0) + \canparam\cdot\exfunction
%\end{align*}
%and get a likelihood gradient of
%\begin{align*}
%	 \frac{\partial \lossof{\expdistof{(\naivestat(\canparam), \naivecanparam)}}}{\partial\canparam} 
%	 &= \sbcontraction{
%	 	\frac{\partial\lossof{\expdistof{(\naivestat, \naivecanparam)}}}{\partial\naivecanparam}|_{\naivecanparam(0)},
%		\frac{\partial\naivecanparam(\canparam)}{\partial\canparam} 
%	 }  \\
%	 &= \contraction{\empdistribution,\exfunction} -   \contraction{\expdistof{(\naivestat, \naivecanparam)},\exfunction} \, .
%\end{align*}


%% Positive and Negative Search
The gradient shows the typical decomposition into a positive and a negative phase.
While the positive phase comes from the data term and prefers directions of large data support, the negative phase originates in the partition function and draws the gradient away from directions already supported by the current model $\expdistof{(\naivestat, \naivecanparam)}$.
%% Regularization functionality
The negative phase is a regularization, by comparing with what has already been learned.
When nothing has been learned so far, we can take the current model to be the uniform distribution, which is the naive exponential family with vanishing canonical parameters. 



%% Collection of features by selection
Given a set $\fselectionmap$ of features we vary $\naivecanparam$ by the function
\begin{align*}
	 \exfunction(\canparam) = \naivecanparam + \sbcontractionof{\canparam,\sencodingof{\fselectionmap}}{\shortcatvariables} \, . 
\end{align*}
At $\canparam=0$ we have the gradient of the loss of the parametrized formula by
\begin{align*}
	 \gradwrtat{\canparam}{0} 
	 \lossof{\expdistof{(\naivestat,\exfunction(\canparam),\basemeasure)}}
	 &= \sbcontraction{
	 	 \gradwrtat{\exfunction(\canparam)}{\naivecanparam}  \lossof{\expdistof{(\naivestat,\exfunction(\canparam),\basemeasure)}},
		 \gradwrtat{\canparam}{0}  \exfunction(\canparam)
	 }  \\
	 &= \sbcontractionof{\empdistribution,\sencodingof{\sstat}}{\selvariable} -   \sbcontractionof{\expdistof{(\naivestat, \naivecanparam, \basemeasure)},\sencodingof{\sstat}}{\selvariable} \, . 
\end{align*}


%% Grafting
We want to choose the formula, which is best aligned with the gradient of the log-likelihood, that is using a formula selecting map $\fselectionmap$
\begin{align} \label{prob:greedyGrad} \tag{$\mathrm{P}^{\mathrm{grad}}_{\datamap,\formulaset,\fselectionmap}$} 
	\argmax_{\selindex\in[\seldim]} \sbcontractionof{\empdistribution,\fselectionmap}{\indexedselvariable} 
	- \sbcontractionof{\expdistof{(\naivestat, \naivecanparam, \basemeasure)},\fselectionmap}{\indexedselvariable} \, . 
\end{align}
This method is known as the gradient heuristic or grafting.
% Mean parameter interpretation
The objective of Problem~\eqref{prob:greedyGrad} has another interpretation by the difference of the mean parameter $\datamean$ and $\currentmean$ of the projections of the empirical and current distributions on the family to $\fselectionmap$. % ! NOT the proposal family, those have transposed statistic

%% Formula alignment perspective
Problem~\eqref{prob:greedyGrad} is further equivalent to the formula alignment
\begin{align*}
	\argmax_{\formula\in\fselectionmap} \sbcontraction{\formula,\empdistribution-\currentdistribution} \, . 
\end{align*}
The objective can be interpreted as the difference of the satisfaction probability of the formula with respect to the empirical distribution and the current distribution.
%We can choose selection architectures to efficiently parametrize the formulas in the hypothesis $\fselectionmap$ and rewrite the problem as
%\begin{align*}
%	\argmax_{\selindexin} \contractionof{ \gradwrtat{\canparam}{\canparam=0} \lossof{\expdist}}{\indexedselvariable}
%\end{align*}
%This is thus equivalent to the problem \ref{prob:greedyGrad}, when taking all formulas selectable by $\formulaset$ as the hypothesis $\Gamma$.












\subsubsection{Iterations}

Let us now iterate the search for a best formula at a current model with the optimization of weights after each step.
The result is Algorithm~\ref{alg:greedyStructureLearning}, which is a greedy algorithm adding iteratively the currently best feature.

\begin{algorithm}[hbt!]
\caption{Greedy Structure Learning}\label{alg:greedyStructureLearning}
\begin{algorithmic}
	\State Initialize
		\[ \currentdistribution \algdefsymbol \frac{1}{\prod_{\catenumeratorin}\catdimof{\atomenumerator}} \cdot \onesat{\shortcatvariables} \quad, \quad \formulaset = \varnothing \]
	\While{Stopping criterion is not met}
		\State Structure Learning: Compute a (approximative) solution $\hat{\formula}$ to Problem~\ref{prob:restrictedNaiveMLE} and add the formula to $\formulaset$, i.e.
				\[ \formulaset \algdefsymbol \formulaset \cup \{\hat{\formula}\} \]
			Extend dimension of $\selvariable$ by one, by $\formulaof{\seldim}=\hat{\formula}$ and $\canparamat{\seldim}=0$
		\State Weight Estimation: Estimate the best weights for the added formula and recalibrate the weights of the previous formulas, by calling Algorithm~\ref{alg:AWO}.
				\[ \currentdistribution \algdefsymbol \expdistof{\formulaset, \canparam} \]
\EndWhile
\end{algorithmic}
\end{algorithm}



%% Energy Storage -> Useful after learning for energy-based inference
When having used the same learning architecture multiple times, the energy of the corresponding formulas are all representable by a formula selecting architecture.
Their energy term is therefore a contraction of the selecting tensor with a parameter tensor $\canparam$ in a basis CP decomposition with rank by the number of learned formulas.
When mutiple selection architectures have been used, the energy is a sum of such contractions.
% 
Let us note, that this representation is useful after learning, when performing energy-based inference algorithms on the result.
During learning, one needs to instantiate the proposal distribution, which requires instantiation of the probability tensor.
\red{However, one could alternate data energy-based and use this as a particle-based proxy for the probability tensor.}


\begin{remark}[Sparsification by Thresholding]
	To maintain a small set of active formulas, one could combine greedy learning approaches with thresholding on the coordinates of $\canparam$.
	This is a standard procedure in Iterative Hard Thresholding algorithms of Compressed Sensing, but note that here we do not have a linear in $\canparam$ objective.
\end{remark}




\subsection{Proposal distribution}


% Proposal distribution
Let us now understand the likelihood gradient as the energy tensor of a probability distribution, which we call the proposal distribution.

\begin{definition}[Proposal Distribution]
	Let there be a base distribution $\currentdistribution$, a targeted distribution $\empdistribution$ and a formula selecting map $\fselectionmap[\shortcatvariables, \selvariable]$.
	The proposal distribution at inverse temperature $\invtemp>0$ is the distribution of $\selvariable$ defined by
	\begin{align*}
		\normationof{\expof{\sbcontractionof{\invtemp\cdot(\empdistribution-\currentdistribution),\fselectionmap}{\selvariable}} }{\selvariable} \, . 
	\end{align*}
	The proposal distribution is the member of the exponential family with statistics $\fselectionmap$ and parameter $\invtemp\cdot(\empdistribution-\currentdistribution)$.
\end{definition}


%. Exponential family
The proposal distribution is in the exponential family with sufficient statistic by the formula selecting map $\fselectionmap$, namely the member with the canonical parameters $\canparam=\empdistribution-\currentdistribution$.
Of further interest are tempered proposal distributions, which are in the same exponential family with canonical parameters $\invtemp\cdot(\empdistribution-\currentdistribution)$ where $\invtemp>0$ is the inverse temperature parameter.

% MLN
As Markov Logic Networks, the proposal distributions are in exponential families with the sufficient statistic defined in terms of formula selecting maps.
While Markov Logic Networks contract the maps on the selection variables $\selvariable$, the proposal distributions contract them along the categorical variables $\catvariable$ to define energy tensors.

% Methods to solve mode search
The grafting Problem~\eqref{prob:greedyGrad} is the search for the mode of the proposal distribution.
To solve grafting, we thus need to answer a MAP query, for which we can apply the methods introduced in Chapter~\ref{cha:probReasoning}, such as Gibbs Sampling or Mean Field Approximations in combination with annealing.


\subsubsection{Mean parameter polytope}

The mean parameter polytope of the proposal distribution with statistic $\proposalstat$ is the convex hull of the formulas in $\formulaset$, that is
\begin{align*}
	\meansetof{\proposalstat}
	= \convhullof{\sencodingof{\proposalstat}{\indexedselvariable,\shortcatvariables} \, : \, \selindexin{}}
	= \convhullof{\formulaat{\shortcatvariables} \, : \, \formula\in\fselectionmap}
\end{align*}





\subsection{Discussion}

\begin{remark}[Bayesian approach]
	We only treated the estimation of a single resulting distribution by the data, while in a Bayesian approach one typically considers an uncertainty over possible distributions.
	% MAP
	\red{When treating $\canparam$ as a random tensor, which prior distribution is given and posteriori distribution wanted, we have a more involved Bayesian approach.}
	When having a prior $\probof{\mlnparameters}$ over the Markov Logic Networks we alternatively want to find the parameters $\mlnparameters$ solving the maximum a posteriori problem
	\begin{align}
		\argmax_{\mlnparameters} \mlnprobat{\data}\cdot \probof{\mlnparameters}\, . 
	\end{align}
\end{remark}




% Success Guarantees
\section{Probabilistic Success Guarantees}\label{cha:mlnConcentration}

When drawing data independently from a random distribution, we in this chapter derive guarantees, that the 

%Uniform concentration bounds require concentration bounds on 
%	\[ \sbcontraction{\theta,\fluctuationtensor} \]
%for tensors $\theta\in\Gamma$.
%
%To show such bounds we use the formula decomposition of any $\theta$ into a set
%	\[ \sum_{\mlnformulain} \weightof{\exformula}\exformula \, .\]	
	
	
\subsection{Fluctuations}

A random tensor is a random element of a tensor space $\facspace$, drawn from a probability distribution on $\facspace.$
In contrast to the discrete distributions investigated previously in this work, the random tensors are in most generality continuous distributions. % However, when drawing data they are 

\subsubsection{Fluctuation of the empirical distribution}

% Random one hot encodings
When drawing random states $\datamapof{\dataindex}\in\facstates$ by a distribution $\gendistribution$, we use the one-hot encoding to forward each random state to the random tensor
	\[ \onehotmapofat{\datamapof{\dataindex}}{\shortcatvariables} \, . \]
The expectation of this random tensor is
\begin{align*}
	\expectationof{\onehotmapof{\datamapof{\dataindex}}} 
	= \sum_{\catindices\in\facstates} \gendistribution[\indexedcatvariableof{[\atomorder]}] \onehotmapofat{\catindices}{\shortcatvariables} 
	= \gendistribution[\shortcatvariables] \, . 
\end{align*}
	
The empirical distribution is then the average of independent random one-hot encodings, namely the random tensor
	\[ \empdistribution = \frac{1}{\datanum} \sum_{\dataindexin}  \onehotmapofat{\datamapof{\dataindex}}{\shortcatvariables} \, . \]
To avoid confusion let us strengthen, that in this chapter we interpret $\empdistribution$ as a random tensor taking values in $\facspace$, whereas each supported value of $\empdistribution$ is an empirical distribution taking values in $\facstates$.


% Expectation -> Does not make use of independence here!
When the marginal of each datapoint is $\gendistribution$, the expectation of the empirical distribution is
\begin{align*}
	\expectationof{\empdistribution} 
	= \frac{1}{\datanum} \sum_{\dataindexin}  \expectationof{\onehotmapof{\datamapof{\dataindex}}}
	= \gendistribution \, . 
\end{align*}

% Law of large numbers
From the law of large numbers it follows, that in the limit of $\datanum\rightarrow\infty$ at any coordinate $\catindex\in\facstates$ almost everywhere
	\[ \empdistribution[\indexedcatvariableof{[\atomorder]}] \rightarrow \expectationof{\empdistribution[\indexedcatvariableof{[\atomorder]}]} =  \gendistribution[\indexedcatvariableof{[\atomorder]}] \, . \]

% Fluctuation
At finite $\datanum$ the empirical distribution differs from the by the difference
	\[ \empdistribution - \gendistribution \]
which we call a fluctuation tensor.


\subsubsection{Fluctuation tensors and their widths}

Let us now investigate random tensors, which result from the forwarding of the fluctuation of the empirical distribution by sufficient statistics.

\begin{definition}
	Given a statistic $\sstat$, $\datanum\in\nn$ and and a dataset we define the fluctuation tensor as the random tensor
		\[ \expfamfluctuation = \sbcontractionof{\empdistribution-\gendistribution,\sencsstat}{\selvariable} \]
	where $\datamap$ is a collection of $\datanum$ independent samples of $\gendistribution$.
\end{definition}

% Naive Ex
The fluctuation of the empirical distribution around the generating distribution corresponds in this notation with the naive exponential family, taking the identity as statistics.
% Appearances
Besides this, fluctuation tensors appears in Markov Logic Networks as fluctuations of random mean parameters and in proposal distributions as fluctuation of random energy tensor.
We will discuss these examples in the following sections.


\subsubsection{Naive Exponential Family}

\red{This is the minterm exponential family!}

In case of the naive exponential family, we have $\sstat=\identityat{\shortcatvariables,\selvariable}$ and the fluctuation tensor is
	\[ \naivefluctuation = \empdistribution - \gendistribution \, .  \]

% Multinomial
This fluctuation tensor is related to tensor encodings of multinomial distributions, which we now define as multinomial random tensors.

\begin{definition}
	A multinomial random tensor is the sum of the one-hot encodings of independent random variables $Z_\dataindex$ each distributed by $\probtensor$
		\[ Z^{\probtensor, \datanum} = \sum_{\dataindexin} \onehotmapof{Z_\dataindex} \, . \] 
\end{definition}

\begin{lemma}\label{lem:multinomialEmpdistFluctuation}
	The fluctuation $\empdistribution - \gendistribution$ is a by $\frac{1}{\datanum}$ rescaled centered multinomial random tensor with parameters $\gendistribution$ and $\datanum$. % Needs some more explanation based on one-hot encodings?
\end{lemma}
\begin{proof}
	By the above construction we have
		\[  \empdistribution - \gendistribution = \frac{1}{\datanum} \left( \onehotmapofat{\datamapof{\dataindex}}{\shortcatvariables} - \expectationof{\onehotmapofat{\datamapof{\dataindex}}{\shortcatvariables}} \right) \, .  \]
\end{proof}


\subsubsection{Mean parameter in Markov Logic Networks}

The mean parameter of the M-projection of the empirical distribution on the family of Markov Logic Networks with statistic $\fselectionmap$ is the random tensor
\begin{align*}
	\meanparam^\datamap =  \sbcontractionof{\mlnstat,\empdistribution}{\selvariable} \, . 
\end{align*}

The expectation of this random tensor is
\begin{align*}
	\expectationof{\meanparam^\datamap} 
	=  \sbcontractionof{\mlnstat,\expectationof{\empdistribution}}{\selvariable} 
	=  \sbcontractionof{\mlnstat,\gendistribution}{\selvariable} 
	=  \meanparam^* \, ,  
\end{align*}
where we used that the expectation and contraction operation can be commuted due to the multilinearity of contractions.

% Fluctuation of mean parameter
The fluctuation of this mean parameter is
\begin{align*}
	\meanparam^\datamap - \expectationof{\meanparam^\datamap} =  \sbcontractionof{\mlnstat,\empdistribution-\gendistribution}{\selvariable} \, . 
\end{align*}
We notice, that this is the fluctuation tensor $\mlnfluctuation$.


%\begin{example}[Proposal distribution]
\subsubsection{Energy tensor in proposal distributions}

The fluctuation tensor appears as a fluctuation of the energy of the proposal distribution.
For the expected energy it holds that
\begin{align*}
	\expectationof{\energytensorof{\proposalstat,\empdistribution-\currentdistribution}} 
	= \expectationof{\sbcontractionof{\proposalstat,\empdistribution-\currentdistribution}{\selvariable}} 
	= \sbcontractionof{\proposalstat,\expectationof{\empdistribution-\currentdistribution}}{\selvariable}
	= \sbcontractionof{\proposalstat,\gendistribution-\currentdistribution}{\selvariable} 
	= \expectationof{\energytensorof{\proposalstat,\gendistribution-\currentdistribution}} \, . 
\end{align*}

% Fluctuation
The fluctuation of this random tensor is
\begin{align*}
	\expectationof{\energytensorof{\proposalstat,\empdistribution-\currentdistribution}}  - \expectationof{\energytensorof{\proposalstat,\gendistribution-\currentdistribution}} 
	= \expectationof{\energytensorof{\proposalstat,\empdistribution-\gendistribution}}
\end{align*}
and coincides with $\proposalfluctuation$.
	




\subsubsection{Binary Features}

In all the above example we have statistics consistent of binary features.
In this case the marginal distributions of the coordinates of $\expfamfluctuation$ are scaled and centered binomials, which we investigate now.

\begin{lemma}
	Let $\sstat$ be a statistic and let for $\statenumeratorin$ $\sstat_\statenumerator$ be a binary feature, i.e. let $\imageof{\sstat_\statenumerator}\subset \{0,1\}$.
	Then, the marginal distribution of the coordinate $\expfamfluctuation[\selvariable=\statenumerator]$ is 
		\[\frac{1}{\datanum}\left(\bidistof{\fprobof{\statenumerator},\datanum}- \fprobof{\statenumerator}\right)  \, , \] 
	where by $\bidistof{\fprobof{\statenumerator},\datanum}$ we denote the binomial distribution with mean parameter
		\[ \fprobof{\statenumerator} = \sbcontraction{\sstat_\statenumerator, \gendistribution} \, . \]
\end{lemma}
\begin{proof}
	We notice that when forwarding a random sample $\datamapof{\dataindex}$ of $\gendistribution$ is the random tensor
		\[ \onehotmapofat{\datamapof{\dataindex}}{\shortcatvariables} \, \]
	and since $\imageof{\sstat_\statenumerator}\subset \{0,1\}$ the contraction
		\[ \sbcontraction{\sstat_\statenumerator, \onehotmapofat{\datamapof{\dataindex}}{\shortcatvariables}} \]
	is a random variable taking values in $\{0,1\}$.
	The variable therefore follows a Bernoulli distribution with mean parameter
		\[ \fprobof{\statenumerator} = \expectationof{\sbcontraction{\sstat_\statenumerator, \onehotmapofat{\datamapof{\dataindex}}{\shortcatvariables}}} = \sbcontraction{\sstat_\statenumerator, \gendistribution}  \, . \]
\end{proof}


%\begin{lemma}	
%	Let $\exformula$ be a tensor with coordinates in $\{0,1\}$.
%	The random variable $\sbcontraction{\exformula,\fluctuationtensor}$ is distributed by 
%		\[ \frac{1}{\datanum}\left(\bidistof{\fprobof{\exformula},\datanum}- \fprobof{\exformula}\right)  \]
%	where $\bidistof{\fprobof{\exformula},\datanum}$ is the binomial distribution with $\fprobof{\exformula}$ being the probability of $\exformula$ satisfied.
%\end{lemma}
%\begin{proof}
%	For a $\datapoint$ the contraction
%		\[ \braket{\exformula,\datapoint} \]
%	follows a Bernoulli distribution with 
%		\[  \probof{\braket{\exformula,\datapoint}=1} = \probof{\exformula = \mathrm{true}} \, . \]	
%	With the assumption of independent data, these scalar products sum up to a Binomial, which is then centered and averaged.
%\end{proof}
%
%If the data is generated by $\mlnprobof{\cdot}{\expsolution}$ we have
%	\[ \fprobof{\exformula} = \frac{1}{\partitionfunctionof{\expsolution}}  \braket{\exformula,\expof{\expsolution}} \, . \]


\subsubsection{Widths of random tensors}

% Widths
In the following we will use the supremum of contractions with random tensors in the derivation of success guarantees for learning problems.
Such quantities are called widths.

\begin{definition}
	Given a set $\Gamma\subset\facspace$ and $\fluctuationtensor$ a random tensor taking values in $\facspace$ we define the width as the random variable
		\[ \widthwrtof{\Gamma}{\fluctuationtensor} = \sup_{\theta\in\Gamma} \absof{\sbcontraction{\theta,\fluctuationtensor}} \, . \]	
\end{definition}



\subsection{Expected and Empirical Risk Minimization}

\red{We take a frequentist approach here and study the distributions of estimated parameters depending on random data.}

\subsubsection{Maximum Likelihood Estimation on random data}

% Aim: Recovery Guarantees
We here investigate the statistical errors of the maximum likelihood estimators.
The empirical distribution is understood as an estimation of an underlying distribution.
When sampling by independent copies of the underlying distribution, its mean is the underlying distribution.
However, due to fluctuations around this mean the solution of estimation problems with respect to the empirical distribution does in general not coincide with the solution given the underlying distribution.

%
To be more precise, when generating data $\data$ by independent copies of a distribution $\gendistribution$ we define the minimization problems
\begin{align}\tag{$P_{\data,\Gamma}$}\label{prob:empEstimation}
	\empsolution = \argmin_{\theta\in\Gamma} \centropyof{\empdistribution}{\expdistof{\theta}}
\end{align}
and
\begin{align}\tag{$P_{\mathbb{E},\Gamma}$}\label{prob:expEstimation}
	\expsolution = \argmin_{\theta\in\Gamma} \centropyof{\gendistribution}{\expdistof{\theta}}
\end{align}
where by $\expdistof{\theta}$ we denote the element of an exponential family with sufficient statistics $\sstat$ and parameter $\theta$.


%% Examples
%In feature calibration $\sstat$ is the concatenation of all formulas and $\Gamma$ is the vector space $\rr^{\cardof{\formulaset}}$ of possible weigths.
%In feature selection we would choose $\sstat$ by a formula selecting tensor network and $\Gamma$ as the set of basis tensors representing the selection of a specific formula. %% TRUE ?? Doing Gradient 
%We then optimize 


% Concentration
We have at each hypothesis $\theta\in\Gamma$
\begin{align}
	\expectationof{\centropyof{\empdistribution}{\expdistof{\theta}}} = \centropyof{\gendistribution}{\expdistof{\theta}}
\end{align}
and thus, the objective in Problem~\ref{prob:empEstimation} converges in the limit $\datanum\rightarrow\infty$ by the law of large numbers at each hypothesis $\theta\in\Gamma$ to the objective in Problem~\ref{prob:expEstimation}.

To use this insight in the derivation of bounds of the distance of $\empsolution$ and $\expsolution$, we need to quantifying this convergence 
\begin{itemize}
	\item Non-asymptotically: Since we typically have access to limited amounts of data, that is finite $m$, we need to quantify the concentration in non-asymptotic cases.
	\item Uniform: Since the problems are optimized at extreme situations, the convergence of the objective has to happen uniformally at multiply $\theta\in\Gamma$.
\end{itemize}

%% Formalization 
%To be more precise let $\expsolution$ be the tensor representing the MLN $\mlntrueparameters$ and $\Gamma$ a hypothesis tensors.
%We then seek to get the tensor $\mlntrueparameters$ maximizing the likelihood function by solving
%\begin{align}\tag{$P_{\loss,\Gamma}$}\label{prob:empMLNrecovery}
%	\argmin_{\theta\in\Gamma} - \variablesum\log\mlnprobof{\datapointof{\variableindex}}{\theta}
%\end{align}

%This is an empirical risk minimization problem given the loss function 
%	\[ \lossof{\theta} = - \probof{\cdot | \theta}\]
%and the empirical risk
%	\[ \loss_{\data}\left( \theta \right) = - \variablesum\log\mlnprobof{\datapointof{\variableindex}}{\theta} \, . \]
%
%In our probabilistic analysis we assume, that the datapoints are drawn independently from a Markov logic network with parameters $\mlntrueparameters$.
%Then we derive recovery guarantees based on nonasymptotic convergence bounds of $\loss_{\data} $ to its expectation.


\subsubsection{Solution of the Expected Problem}

% Motivation
Let us first investigate the solution of Problem~\ref{prob:expEstimation}, to get a reference for Problem~\ref{prob:empEstimation}.

%The expectation of the loss is then
%\begin{align}
%	\expectationof{\variablesum\log\mlnprobof{\datapointof{\variableindex}}{\mlnparameters}} 
%	= \expectationof{\log\mlnprobof{\datapointof{}}{\mlnparameters}}
%\end{align}
%where the expectation is performed over $\datapointof{}$ distributed by $\mlnprobof{\datapointof{}}{\mlntrueparameters}$.
%This is the cross entropy between the generative distribution by $\mlntrueparameters$ and the hypothesis $\mlnparameters$.
%
%The expected loss analogon to Problem \ref{prob:empMLNrecovery} is
%\begin{align}\tag{$P_{\expectationof{\loss},\Gamma}$}\label{prob:expMLNrecovery}
%	\argmin_{\theta\in\Gamma} \expectationof{\log\mlnprobof{\datapointof{}}{\theta}}
%\end{align}

The solution does not change when manipulating the objective as
\begin{align}
		\argmin_{\theta\in\Gamma} \centropyof{\gendistribution}{\expdistof{\theta}}
		= \argmin_{\theta\in\Gamma} \centropyof{\gendistribution}{\expdistof{\theta}} -  \centropyof{\gendistribution}{\gendistribution}
\end{align}

We notice that the objective of the right side problem is the Kullback-Leibler divergence
	\begin{align}
		 \kldivof{\gendistribution}{\expdistof{\theta}}
%		\kldivof{\expdistof{\expsolution}}{\expdistof{\theta}} = \expectationof{\log
%		\left[
%		\frac{
%		\mlnprobof{\datapointof{}}{\expsolution} 
%		}{
%		\mlnprobof{\datapointof{}}{\theta}
%		}
%		\right]} \, .
	\end{align}
Thus, Problem~\ref{prob:expEstimation} coincides with the approximation of $\gendistribution$ by an element $\expdistof{\theta}$ with $\theta\in\Gamma$ in the Kullback-Leibler divergence.

We decompose:
\begin{align}
	\kldivof{\expdistof{\expsolution}}{\expdistof{\theta}} 
	= \log\left[\frac{\partitionfunctionof{\theta}}{\partitionfunctionof{\expsolution}}\right] 
	+ \frac{1}{\partitionfunctionof{\expsolution}} \braket{\expsolution-\theta,\expof{\expsolution}}
\end{align}
Some insights can be drawn based on this decomposition:
\begin{itemize}
	\item The first term vanishes, when both partition functions are the same.
		We can always adjust $\theta$ by constant offsets on all coordinates (of course without changing the distribution), such that the partition functions are equal.
		This is done by the map
			\[ \theta \rightarrow \theta + \lambda \cdot \ones \]
		where we choose $\lambda\in\rr$ by
			\[ \lambda = \frac{\partitionfunctionof{\expsolution}}{\partitionfunctionof{\theta}} \]
		to ensure a vanishing first term.		
	\item In the second term the differences in coordinates are weighted by exponentiated solution $\expsolution$.
		Where the probability mass is small errors in $\theta$ have a small influence on the Kullback-Leibler divergence.
\end{itemize}


\begin{theorem}
	Let us assume $\gendistribution = \expdistof{\expsolution}$ for some $\expsolution\in\Gamma$. 
	Then one solution of Problem \ref{prob:expEstimation} coincides with $\expsolution$ up to constant offsets.
\end{theorem}
\begin{proof}
	This follows directly from the Gibbs inequality, since 
	\begin{align}
		\kldivof{\expdistof{\expsolution}}{\expdistof{\theta}} \leq 0
	\end{align}
	with equality if only if $\expdistof{\expsolution}$ and $\expdistof{\theta}$ are equal.
	The uniqueness follows from Theorem \ref{the:tensorRepUniqueness}
\end{proof}









\subsubsection{Recovery Guarantee based on Widths}

\begin{theorem}
	Let us assume $\expsolution\in\Gamma$ and that $\partitionfunctionof{\theta}$ is constant among $\theta\in\Gamma$.
%	We define
%	\begin{align}
%		\omega_{\Gamma} = \sup_{\theta\in\Gamma} \braket{\theta-\expsolution, \expectationof{\datapointof{}} - \variablesum\datapointof{\variableindex}}
%	\end{align}
	Then for any solution $\hat{\theta}$ of the empirical problem we have
	\begin{align}
		\kldivof{\expdistof{\theta^*}}{\expdistof{\hat{\theta}}} \leq \widthwrtof{\Gamma}{\fluctuationtensor} \, .
	\end{align}
\end{theorem}
\begin{proof}
	First we notice
	\begin{align}
		\argmin_{\theta\in\Gamma} \loss\theta = \argmin_{\theta\in\Gamma} \loss\theta - \loss\expsolution 
	\end{align}
	When $\expsolution\in\Gamma$ the minimum of the empirical loss with respect to $\expsolution$ is negative since
	\begin{align}
		\loss\empsolution - \loss\expsolution \leq \loss\expsolution-\loss\expsolution \leq 0
	\end{align}
	We separate expectations and fluctuations and get
	\begin{align}
		\kllossof{\empsolution} \leq \kllossof{\empsolution} - (\loss\empsolution - \loss\expsolution) \leq \omega_{\Gamma} \, .
	\end{align}
\end{proof}


%We recognize $\omega_\Gamma$ to be a width of the random tensor
%	\[ \fluctuationtensor  = \expectationof{\datapointof{}}-\variablesum\datapointof{\variableindex}  \, . \]



% Width 
The supremum of the differences between expected and empirical risks is the width of the fluctuation tensor, as we state next.

\begin{lemma}
	For any $\Gamma$ and $\datamap$ we have
	\begin{align*}
		\widthwrtof{\Gamma}{\fluctuationtensor^{\identity, \gendistribution, \datamap}} 
		= \sup_{\theta\in\Gamma} \centropyof{\empdistribution}{\expdistof{\theta}} - \centropyof{\gendistribution}{\expdistof{\theta}} 
	\end{align*}
\end{lemma}
\begin{proof}
	Using the decomposition of cross entropy in the naive exponential family 
	 	\[ \centropyof{\empdistribution}{\expdistof{\theta}}=\contractionof{\probtensor,\lnof{\gendistribution}} - \cumfunctionof{\lnof{\gendistribution}} \, . \]
\end{proof}


\begin{corollary}
	At the solution $\hat{\theta}$ of Problem~\ref{prob:empEstimation} we have
		\[ \centropyof{\empdistribution}{\expdistof{\hat{\theta}}} - \centropyof{\gendistribution}{\expdistof{\hat{\theta}}} \leq  \widthwrtof{\Gamma}{\fluctuationtensor^{\identity, \gendistribution, \datamap}} \, . \] 
\end{corollary}






\subsection{Guarantees for Mode of the Proposal Distribution}

Let us now derive probabilistic guarantees, that the mode of the proposal distribution at the empirical and the generating distribution are equal.

\begin{definition}
	The max gap of a tensor $V$ is the quantity
		\[ \maxgap(V) = \min_{i\neq i^{max}} V[X = i^{max}] - V[X=i] \]
	where
		\[ i^{max} \in \argmax_{i} V[X = i] \, . \]
\end{definition}

If multiple maxima exist, the



\begin{theorem}\label{the:probGuaranteeProposalDist}
	Whenever the energy tensor of the expected proposal distribution has a gap of $\maxgap$, then for every $\failprob>0$ the mode of the expected proposal distribution coincides with the empirical proposal distribution with probability at least $1-\expof{-\frac{1}{\failprob^2}}$, provided that
		\[ \datanum > C\frac{\left(\sum_{\atomenumeratorin}\lnof{\catdimof{\atomenumerator}}\right)}{\maxgap^2} \]
	where $C$ is a universal constant.
\end{theorem}

To proof the theorem we first use a deterministic recovery guarantee involving the width of the fluctuation tensor and then apply the width bound of Theorem~\ref{the:basisTensorWidthBound}.

\begin{lemma}\label{lem:detGuaranteeProposalDist}
	Whenever 
	\begin{align*}
		\maxgap(\contractionof{\{\gendistribution,\fselectionmap\}}{\selvariable}) 
		> 2 \cdot  \widthatof{\{\onehotmapof{\catindices} :\catindices\in\facstates\}}{\fluctuationtensor^{\fselectionmap,\gendistribution,\datamap}} \, , 
	\end{align*}
	then the mode of the proposal distribution to the empirical distribution coincides with the mode of the proposal distribution to the generating distribution.
\end{lemma}
\begin{proof}
	If different, then the expected objective at the solutions of the empirical and expected is at least $\maxgap$.
	But at the empirical, the difference it as most twice the width different, so that is a contradiction to the assumption.
\end{proof}


\begin{proof}[Proof of Theorem~\ref{the:probGuaranteeProposalDist}]
	Given the assumed bound, the sub-gaussian norm of the width is upper bounded by $C_2\cdot \maxgap$, thus for any $\failprob>0$ we have
		\[  \widthatof{\{\onehotmapof{\catindices} :\catindices\in\facstates\}}{\fluctuationtensor^{\fselectionmap,\gendistribution,\datamap}}  < 2 \maxgap \]
	with probability at least $1-\expof{-\frac{1}{\failprob^2}}$.
	The claim thus follows with Lemma~\ref{lem:detGuaranteeProposalDist}.
\end{proof}




\begin{example}[Gap of a MLNs with single formulas]
	Let there be the MLN of a maxterm $\formula$ with $\atomorder$ variables, and let $\formulaset$ be the maxterm selecting tensor, then 
		\[ \maxgap(
		%\energytensorof{(\{\formula\},\weightof{\formula})}
		\energytensorof{(\formulaset, \expdistof{(\{\formula\},\weightof{\formula})} - \normationof{\ones}{\shortcatvariables} )}
		) = \frac{1}{2^{\atomorder}-1 + \expof{-\weightof{\formula}}}  \]
	If $\weightof{\formula}>0$ we have an exponentially small gap.
	Thus, for the above Lemma to apply, the width needs to be exponentially in $\atomorder$ small.
	
	
	Let there be the MLN of a minterm $\formula$ with $\atomorder$ variables, then 
		\[ \maxgap(
%		\energytensorof{(\{\formula\},\weightof{\formula})}
		\energytensorof{(\formulaset, \expdistof{(\{\formula\},\weightof{\formula})} - \normationof{\ones}{\shortcatvariables} )}
		) = \frac{1}{1+(2^{\atomorder}-1)\cdot\expof{-\weightof{\formula}}}  \]
	For large $\weightof{\formula}$ and $\atomorder$, the gap tends to $1$.
\end{example}






%%% OLD
%\begin{theorem}
%	When the expected sufficient statistics of $\gendistribution$ is gapped by $\maxgap>0$ we  have
%		\[ \expsolution = \empsolution \]
%	with probability at least
%		\[ 1- XXXX \, . \]
%\end{theorem}
%\begin{proof}
%	We show the theorem by proofing that within the stated probabilistic bounds we have
%		\[ \maxgap > 2 \widthatof{\Gamma}{\fluctuationtensor} \, . \]
%	In case of this event, it follows that $\expsolution = \empsolution$.
%	The probabilistic bound can be shown based on the sub-Gaussian norm of $\widthatof{\Gamma}{\fluctuationtensor}$ bounded in Chapter~\ref{cha:widthBounds}.
%\end{proof}






\subsection{Guarantees for Parameter Estimation}

\red{This is mean parameter fluctuation interpretation of the random tensor.}


\begin{lemma}\label{lem:meanParamDistance}
	For any $\fselectionmap$ and $\datamap$ drawn from $\gendistribution$ we have
	\begin{align*}
		\normof{\meanparam^\datamap - \meanparam^*} 
		= \widthwrtof{\subsphere}{\fluctuationtensor^{\fselectionmap,\gendistribution,\datamap}} \, ,
	\end{align*}
	where $\meanparam^\datamap=\sbcontractionof{\fselectionmap,\empdistribution}{\selvariable}$ and $\meanparam^*=\sbcontractionof{\fselectionmap,\gendistribution}{\selvariable}$.
\end{lemma}

%
We can thus apply the sphere bounds.


\begin{theorem}
	For any $\failprob\in(0,1)$ we have the following with probability at least $1-\failprob$.
	Let $\hat{\canparam}$ and $\precision>0$, then
		\[ \absof{\centropyof{\gendistribution}{\expdistof{\empsolution}} - \centropyof{\empdistribution}{\expdistof{\empsolution}}} \leq \tau \cdot \normof{\empsolution} \]
	provided that
		\[ \datanum \geq \frac{\sbcontraction{\meanparam^*}-\sbcontraction{(\meanparam^*)^2}}{\failprob \precision^2} \, . \]
\end{theorem}
\begin{proof}
	We have by Cauchy Schwartz 
		\[ \absof{\sbcontraction{\meanparam^\datamap - \meanparam^*,\empsolution}} \leq \normof{\meanparam^\datamap - \meanparam^*} \cdot \normof{\empsolution}\]
	and with Lemma~\ref{lem:meanParamDistance}
		\[ \absof{\sbcontraction{\meanparam^\datamap - \meanparam^*,\empsolution}} \leq \widthwrtof{\subsphere}{\fluctuationtensor^{\fselectionmap,\gendistribution,\datamap}} \cdot \normof{\empsolution} \, . \]
	We show in Part III that in Theorem~\ref{the:sphereBoundVariance} that
		\[  \widthwrtof{\subsphere}{\fluctuationtensor^{\fselectionmap,\gendistribution,\datamap}} \leq \tau \]
	when $\datanum$ satisfies the assumed lower bound, from which the claim follows.
\end{proof}








% Structured Extensions
\chapter{\chatextfolModels}\label{cha:folModels}

We in this chapter extend the tensor network based treatment of \propositionalLogic{} towards \firstOrderLogic{}.
In contrast to \propositionalLogic{}, worlds in \firstOrderLogic{} contain objects, which have relations between each others.
We show in this chapter, how these relations are captured by the \substitutionStructure{} of each world and derive tensor representations of each world.
The \substitutionStructure{} is then combined with the \semanticStructure{}, which enumerates possible worlds in a theory.
We then generalize \HybridLogicNetworks{} to the situation of \firstOrderLogic{}s and show in particular, that the likelihood of \HybridLogicNetworks{} on a single \firstOrderLogic{} world is in some cases equivalent to the likelihood of a dataset in propositoinal logics.

\sect{Syntax and Semantics of \firstOrderLogic{}}

%Since \firstOrderLogic{} follows structured representations of a system, a \firstOrderLogic{} world consists in objects and relations between them.
The framework of \firstOrderLogic{} generalizes \propositionalLogic{} analogously to the generalization of factored system representations towards structured system representations \cite{russell_artificial_2021}.
This more expressive framework is designed to reason about relations and functions between objects.
To capture this framework by tensors we here first define the syntax and then investigate tensor representations of the semantics.

\subsect{Syntax}

A \firstOrderLogic{} theory consists in a finite set of constant symbols $\worlddomain$, a set of function symbols $\{\folfunctionof{\secatomenumerator}\wcols\secatomenumeratorin\}$, a set of predicate symbols $\{\folpredicateof{\atomenumerator}\wcols\atomenumeratorin\}$ and an arity map assigning the arity $\indorderof{\secatomenumerator}$ to the function $\folfunctionof{\secatomenumerator}$ and the arity $\indorderof{\atomenumerator}$ to the predicate $\folpredicateof{\atomenumerator}$.

\subsect{Tensor representation of semantics}

We here follow the model-theoretic semantics of \firstOrderLogic{}, where sets of possible worlds are considered.
% Database
We only treat in this work database semantics, where we assume that the domain $\worlddomain$ of each world has a one-to-one mapping to the constants of the theory and is therefore constant among the worlds.
Database semantics is central to combine the \semanticStructure{} with the \substitutionStructure{}.
Under this assumption, the dimension of the \substitutionStructure{} does not depend on the specific world, and we can combine both structures in a single tensor space to be defined in the next sections.

% Index interpretation of world domain
To each world there is a world domain $\worlddomain$ of objects, which we assume to be finite.
We exploit the set-encoding formalism discussed in more detail in \charef{cha:basisCalculus} and use bijective index interpretation maps
\begin{align*}
    \indexinterpretation \defcols [\inddim] \rightarrow \worlddomain \, .
\end{align*}
A so-called term variable $\indvariable$ takes states $\indindexin$, which represent objects
\begin{align*}
    \indexinterpretationat{\indindex} \in \worlddomain \, .
\end{align*}

% Relations
The relations between objects are described by $\indorder$-ary predicates $\folpredicate$.
Given a specific world $\worldindices$ the truth of relations is represented by boolean tensors
\begin{align*}
    \groundingof{\folpredicate} \defcols \symindstates\rightarrow\ozset \, .
\end{align*}
Given a tuple $\indindexlist\in\symindstates$ the boolean
\begin{align*}
    \groundingofat{\folpredicate}{\indexedindvariableof{0},\ldots,\indexedindvariableof{\indorder-1}} \in\ozset
\end{align*}
is called a grounding and encodes, whether the relation $\folpredicate$ is satisfied in the world $\worldindices$ for the objects $\indexinterpretationat{\indindexof{0}},\ldots,\indexinterpretationat{\indindexof{\indorder-1}}$.

% Functions
Functions in \firstOrderLogic{} are object-valued maps
\begin{align*}
    \groundingof{\folfunction}\defcols \worlddomain^{\inddim} \rightarrow \worlddomain \, .
\end{align*}
While relations are represented by their coordinate encodings, functions are represented by directed basis encodings
\begin{align*}
    \bencodingofat{\groundingof{\folfunction}}{\indvariableof{\folfunction},\shortindvariables}
\end{align*}
where to each object tuple $\indexinterpretationat{\indindexof{0}},\ldots,\indexinterpretationat{\indindexof{\indorder-1}}$ the vector $\bencodingofat{\groundingof{\folfunction}}{\indvariableof{\folfunction},\indexedshortindvariables}$ is the one-hot encoding of the object, that is
\begin{align*}
    \bencodingofat{\groundingof{\folfunction}}{\indvariableof{\folfunction},\indexedshortindvariables}
    = \onehotmapofat{\groundingofat{\folfunction}{\indexinterpretationat{\indindexof{0}},\ldots,\indexinterpretationat{\indindexof{\indorder-1}}}}{\indvariableof{\folfunction}} \, .
\end{align*}


%\subsect{Database semantics}

% Database Semantics
%\begin{align*}
%    \bigotimes_{\atomenumeratorin,\shortindindices\in[\inddim]^{\indorder}} \rr^2 \, .
%\end{align*}

\subsect{Two levels of tensor representation}

In comparison with \propositionalLogic{}, \firstOrderLogic{} bears two natural tensor structures.
\begin{itemize}
    \item \textbf{\SemanticStructure{}:} As in \propositionalLogic{}, we enumerate possible worlds by a collection of variables $\catvariable$.
    This is a representation of the model-theoretic semantics, where subsets of worlds are represented by boolean tensors.
    \item \textbf{\SubstitutionStructure{}:}
    Different to \propositionalLogic{}, a formula can have variables and the evaluation of a formula on a world is represented by its grounding tensor.
    Given a world which contains objects in the domain $\worlddomain$, we can substitute variables by objects in the domain.
    In this way, each predicate with $\inddim$ variables is represented in that world as a boolean tensor of order $\inddim$.
\end{itemize}
% Russel reference
While the \semanticStructure{} appears already in factored representations of systems, the \substitutionStructure{} arises in more generality when treating structured representations of systems \cite{russell_artificial_2021}.



\sect{\SubstitutionStructure{}}

We in this section investigate the structure of terms and formulas in a single \firstOrderLogic{} world, which we for now index by $\worldindices$.
As we have derived in \charef{cha:logicalRepresentation} for \propositionalLogic{}, we develop efficient tensor network representations of the representing tensors based on the corresponding syntax.

\subsect{Grounding tensors}

Given a \firstOrderLogic{} world $\worldindices$, arbitrary formulas are interpreted in terms of the satisfactions of their groundings.
We define their semantic first, and then relate their syntactical decomposition to tensor networks, similar to our approach to \propositionalLogic{} in \charef{cha:logicalRepresentation}.

\begin{definition}[Grounding of a first-order formula given a world]
    Given a specific world $\worldindices$, with a domain $\worlddomain$ enumerated by $[\inddim]$, the grounding of a formula $\folexformula$ with variables $\indvariableof{\folexformula}$  is the tensor
    \begin{align*}
        \groundingofat{\folexformula}{\indvariableof{\folexformula}} :
        \bigtimes_{\indenumerator\in[\indvariableof{\folexformula}]} [\inddim] \rightarrow \ozset \, .
    \end{align*}
    Each coordinate represents thereby the boolean, whether the substitution of the variables in the formula is satisfied given a world $\worldindices$, that is
    \begin{align*}
        \groundingofat{\folexformula}{\indexedindvariableof{\folexformula}} = 1
    \end{align*}
    if and only if the substitution of $\folexformula$ with the variables $\indvariableof{\folexformula}$ replaced by the objects $\indexinterpretationat{\indindexof{\indenumerator}}$ is satisfied on the world $\worldindices$.
\end{definition}


%% Basis encoding
%When interpreting this map as a basis encoding, formulas are tensors in the tensor space
%\begin{align*}
% 	\left(\bigotimes_{\atomenumeratorin, \indindexlist\in[\inddim]} \rr^{2} \right) \otimes
%	\left(  \bigotimes_{\atomenumeratorin, \indindexof{0},\ldots,\indindexof{\indorder_{\folexformula}}\in[\inddim]} \rr^{2} \right) \, .
%\end{align*}

\subsect{Terms}

Terms are object valued maps on $\worlddomain^{n}$.
The basis encoding of each term corresponds is a boolean and directed tensor.
% Constants: Functions with $n=0$
Constants are the functions with $n=0$, and are represented by basis vectors in $\rr^{\cardof{\worlddomain}}$.
Given database semantics, this basis vector does not vary among different worlds.
% General terms
Most general terms are combinations of functions and constants and their basis encodings are acyclic contractions of basis encodings to functions.


%\subsect{Substitution by slicing}
% Slicing interpretation
%Slicing the grounding tensor of a formula a first-order formula amounts to substitution of the respective variable by the constant at the enumeration index.

%\subsect{Syntactical Decomposition of quantifier-free formulas}

\subsect{Formula synthesis by connectives}\label{sec:folConnectiveRepresentation}

In order to have a sound semantic, the grounding of \firstOrderLogic{} formulas is determined by the syntax of the formula, i.e. a decomposition of the formula into connectives and quantifiers acting on atomic formulas.

% Formulas as maps from worlds to groundings
Quantifier-free formulas are connectives acting on atomic formulas.
We can describe them as in the case of \propositionalLogic{} in the $\bencodingof{}$-formalism.
While the atomic formulas where delta tensors copying states, they are more involved here.

\begin{theorem}
    For any connective $\exconnective$ and formulas $\folexformula_1$ and $\folexformula_2$ we have
    \begin{align}
        &\groundingofat{(\folexformula_1\exconnective\folexformula_2)}{\indvariableof{\folexformula_1}\cup\indvariableof{\folexformula_2}} \\
        &\quad=
        \contractionof{
            \bencodingofat{\groundingof{\folexformula_1}}{\headvariableof{\folexformula_1},\indvariableof{\folexformula_1}},
            \bencodingofat{\groundingof{\folexformula_2}}{\headvariableof{\folexformula_2},\indvariableof{\folexformula_2}},
            \bencodingofat{\exconnective}{\headvariableof{\folexformula_1\exconnective\folexformula_2}, \headvariableof{\folexformula_1}, \headvariableof{\folexformula_2}},
            \tbasisat{\headvariableof{\folexformula_1\exconnective\folexformula_2}}
        }
        {\shortindvariablelist} \, .
    \end{align}
\end{theorem}
\begin{proof}
    This directly follows from \theref{the:compositionByContraction}.
%	By the semantic interpretation of the groundings, which has to be sound.
\end{proof}

% Shared variables
Here, variables can be shared by the connected formulas, therefore the variables in the combined formula are unions of the possible not disjoint variables of the connected formulas.

%% Propositional interpretation
%When we understand the head variables in the basis encoding of atoms as the categorical variables, and get a similar interpretation of the tensor network decomposition as in the propositional case.
%\subsect{Propositionalization}

When interpreting the head variables of relational encoded atomic formulas as the atoms of a propositional theory, we find a propositional formula $\exformula$ associated with any decomposable \firstOrderLogic{} formula.

\begin{definition}
    \label{def:propositionalEquivalent}
    Given a formula $\folexformula$ in \firstOrderLogic{}, we say that a propositional formula $\formulaat{\shortcatvariables}$ is the propositional equivalent to $\folexformula$ given atomic formulas $\extformulaof{\atomenumerator}$ in \firstOrderLogic{}, when for any world $\worldindices$ we have
    \begin{align*}
        \groundingofat{\folexformula}{\indvariableof{\folexformula}}
        = \contractionof{
            \{\bencodingofat{\groundingof{\extformulaof{\atomenumerator}}}{\catvariableof{\atomenumerator},\indvariableof{\extformulaof{\atomenumerator}}} : \atomenumeratorin\}
            \cup \{\formulaat{\shortcatvariables}\}
        }{\indvariableof{\folexformula}} \, .
    \end{align*}
    We here denote the head variables of the basis encoding to $\bencodingof{\groundingof{\extformulaof{\atomenumerator}}}$ by $\catvariableof{\atomenumerator}$ to highlight their interpretation as propositional atoms.
\end{definition}

We depict the relation of a grounding tensor to a propositional formula as:
\begin{center}
    \begin{tikzpicture}[scale=0.35, yscale=1, thick] % , baseline = -3.5pt



%\draw[] (2,-1) -- (2,1) node[midway,left] {\tiny $\atomicformulaof{0}$};
%\node[anchor=center] (text) at (4,0) {$\cdots$};
%\draw[] (6,-1) -- (6,1) node[midway,right] {\tiny $\atomicformulaof{\atomorder-1}$};

\draw (1,-1) rectangle (7,-3);
\node[anchor=center] (text) at (4,-2) {$\groundingof{\folexformula}$};

\draw[] (2,-3) -- (2,-5) node[midway,left] {\tiny $\individualvariableof{0}$};
\node[anchor=center] (text) at (4,-4) {$\cdots$};
\draw[] (6,-3) -- (6,-5) node[midway,right] {\tiny $\individualvariableof{\individualorder-1}$};


\node[anchor=center] (text) at (10,-2) {${=}$};



%\draw (1,-1) rectangle (7,-3);
%\node[anchor=center] (text) at (4,-2) {$\rencodingof{\impformula}$};

\begin{scope}[shift={(12,-2)}]

\draw (1,3) rectangle (12,5);
\node[anchor=center] (text) at (6.5,4) {$\exformula$};

\draw[->] (2.5,1) -- (2.5,3) node[midway,right] {\tiny $\atomicformulaof{0}$};
\draw (1,-1) rectangle (4,1);
\node[anchor=center] (text) at (2.5,0) {$\rencodingof{\groundingof{\extformulaof{0}}}$};
\node[anchor=center] (text) at (2.5,-2) {$\cdots$};

\node[anchor=center] (text) at (6.5,0) {$\cdots$};

\draw[->] (10.5,1) -- (10.5,3) node[midway,right] {\tiny $\atomicformulaof{\atomorder-1}$};
\draw (8.75,-1) rectangle (12.25,1);
\node[anchor=center] (text) at (10.5,0) {$\rencodingof{\groundingof{\extformulaof{\atomorder\shortminus1}}}$};
\node[anchor=center] (text) at (10.5,-2) {$\cdots$};

\draw[-] (11.5,-3) -- (3.5,-3) ;
\draw[-] (9.5,-5) -- (1.5,-5) ;

\draw[fill] (11.5,-3) circle (0.25cm);
\draw[->] (11.5,-3) -- (11.5,-1);

\draw[fill] (9.5,-5) circle (0.25cm);
\draw[->] (9.5,-5) -- (9.5,-1);

\draw[fill] (3.5,-3) circle (0.25cm);
\draw[->] (3.5,-3) -- (3.5,-1);

\draw[fill] (1.5,-5) circle (0.25cm);
\draw[->] (1.5,-5) -- (1.5,-1);


\draw[fill] (7.5,-3) circle (0.25cm);
\draw[<-] (7.5,-3) -- (7.5,-7) node[right] {\tiny $\individualvariableof{\individualorder-1}$} ;

\node[anchor=center] (text) at (6.5,-6) {$\cdots$};

\draw[fill] (5.5,-5) circle (0.25cm);
\draw[<-] (5.5,-5) -- (5.5,-7) node[left] {\tiny $\individualvariableof{0}$} ;


%\draw (13,-2) rectangle (15,-6);
%\node[anchor=center] (text) at (14,-4) {$\rencodingof{\impformula}$};
%\node[anchor=center] (text) at (12,-3.75) {$\vdots$};
%\draw[->] (15,-4) -- (16,-4);
%\draw[] (17,-4) -- (16,-4);
%\draw (17,-3) rectangle (19,-5);
%\node[anchor=center] (text) at (18,-4) {$\tbasis$};

\end{scope}




		


\end{tikzpicture}
\end{center}


\subsect{Quantifiers}

Existential and universal quantifiers appear in generic \firstOrderLogic{} and are besides substitutions further means to reduce the number of variables in a formula.
%They are not representable as linear transform of the respective quantifier-free formula.


% Definition of existential and universal quantifiction needed!
The semantics of existential quantification consists in a formula being true, if at least one state of the quantified variable is true, as we define next.

\begin{definition}
    Given a grounding tensor
    \begin{align*}
        \groundingofat{\folexformula}{\indvariableof{0},\ldots,\indvariableof{\indorder-1}} \,
    \end{align*}
    the existential and universal quantification with respect to the first variable are the tensors
    \begin{align*}
        \groundingofat{\left(\exists_{\indindexof{0}}\folexformula\right)}{\indvariableof{1},\ldots,\indvariableof{\indorder-1}} \quad \text{and} \quad
        \groundingofat{\left(\forall_{\indindexof{0}}\folexformula\right)}{\indvariableof{1},\ldots,\indvariableof{\indorder-1}} \,
    \end{align*}
    with coordinates as follows.
    For an assignment $\indindexof{1},\ldots,\indindex$ to the non-quantified variables we have
    \begin{align*}
        \groundingofat{\left(\exists_{\indindexof{0}}\folexformula\right)}{\indexedindvariableof{1},\ldots,\indexedindvariableof{\indorder-1}} = 1
    \end{align*}
    if and only if there is an assignment $\indindexofin{0}$ such that
    \begin{align*}
        \groundingofat{\folexformula}{\indexedindvariableof{0},\indexedindvariableof{1},\ldots,\indexedindvariableof{\indorder-1}} = 1 \, .
    \end{align*}
    Conversely, we have for the universal quantification that
    \begin{align*}
        \groundingofat{\left(\forall_{\indindexof{0}}\folexformula\right)}{\indexedindvariableof{1},\ldots,\indexedindvariableof{\indorder-1}} = 1
    \end{align*}
    if and only if for any assignment $\indindexofin{0}$ we have
    \begin{align*}
        \groundingofat{\folexformula}{\indexedindvariableof{0},\indexedindvariableof{1},\ldots,\indexedindvariableof{\indorder-1}} = 1 \, .
    \end{align*}
\end{definition}


Let us now show, that existential and universal quantification are coordinatewise transforms (see \defref{def:coordinatewiseTransform}) of contracted grounding tensors.
To this end, let us introduce the greater-$z$ indicator $\greaterthanfunction{z}$, where $z\in\rr$, as the function
\begin{align*}
    \greaterthanfunction{z} : \rr \rightarrow \ozset
    \quad, \quad \greaterthanfunctionof{z}{x} =
    \begin{cases}
        1 & \ifspace x > z\\
        0 & \text{else}
    \end{cases} \, .
\end{align*}

\begin{theorem}
    For any formula $\folexformula$ with variables $\shortindvariablelist$ we have
    \begin{align*}
        \groundingofat{\left(\exists{\indindexof{0}}\folexformula\right)}{\indvariableof{1},\ldots,\indvariableof{\indorder-1}} =
        \coordinatetrafowrtofat{\existquanttrafo}{\contractionof{\groundingof{\folexformula}}{\indvariableof{1},\ldots,\indvariableof{\indorder-1}}}{\indvariableof{1},\ldots,\indvariableof{\indorder-1}}
    \end{align*}
    and
    \begin{align*}
        \groundingofat{\left(\forall{{\indindexof{0}}} \folexformula\right)}{\indvariableof{1},\ldots,\indvariableof{\indorder-1}}=
        \coordinatetrafowrtofat{\universalquanttrafo}{\contractionof{\groundingof{\folexformula}}{\indvariableof{1},\ldots,\indvariableof{\indorder-1}}}{\indvariableof{1},\ldots,\indvariableof{\indorder-1}}
    \end{align*}
\end{theorem}
\begin{proof}
    We proof the claimed equalities to arbitrary slices of the remaining variables, which amount to arbitrary substitutions of the formulas.
    For any indices $\indindexofin{1},\ldots,\indindexofin{\indorder-1}$ we notice, that
    \begin{align*}
        \contractionof{\groundingof{\folexformula}}{\indexedindvariableof{1},\ldots,\indexedindvariableof{\indorder-1}}
        &= \sum_{\indindexofin{0}} \groundingofat{\folexformula}{\indexedindvariableof{0},\ldots,\indexedindvariableof{\indorder-1}} \\
        &= \cardof{\indindexofin{0} \, : \, \groundingofat{\folexformula}{\indexedindvariableof{0},\ldots,\indexedindvariableof{\indorder-1}}=1} \, .
    \end{align*}
    We can thus understand the contracted grounding tensor as storing in its coordinates the count of the coordinate extensions to the zeroth variable, such that the grounding tensor is satisfied.
    This is analogous to our interpretation of contracted propositional formulas as world counts.
    From this it is obvious, that the existential quantification is satisfied, if the count is different from zero, which is captured by the coordinatewise transform with $\existquanttrafo$.
    We therefore arrive at
    \begin{align*}
        \groundingofat{\left(\exists_{\indindexof{0}}\folexformula\right)}{\indexedindvariableof{1},\ldots,\indexedindvariableof{\indorder-1}} =
        \coordinatetrafowrtofat{\existquanttrafo}{\contractionof{\groundingof{\folexformula}}{\indvariableof{1},\ldots,\indvariableof{\indorder-1}}}{\indexedindvariableof{1},\ldots,\indexedindvariableof{\indorder-1}} \, .
    \end{align*}
    The first claim follows, since the assignment to the non-quantified variables was arbitrary.
    The universal quantification is satisfied, when all extensions are satisfied, and the count is $\inddim$.
    Since $\inddim$ is the maximal count, this is captured by the coordinatewise transform with $\universalquanttrafo$ and we get
    \begin{align*}
        \groundingofat{\left(\forall{\indindexof{0}}\folexformula\right)}{\indexedindvariableof{1},\ldots,\indexedindvariableof{\indorder-1}} =
        \coordinatetrafowrtofat{\universalquanttrafo}{\contractionof{\groundingof{\folexformula}}{\indvariableof{1},\ldots,\indvariableof{\indorder-1}}}{\indexedindvariableof{1},\ldots,\indexedindvariableof{\indorder-1}} \, .
    \end{align*}
    With the same argument, the second claim is established.
\end{proof}

% Customized quantifiers
We can extend this discussion towards more generic counting quantifiers, of which the existential and the universal quantifier are extreme cases.
One can define quantifiers by demanding that at least $z\in\nn$ compatible groundings are satisfied, and show that they amount to coordinatewise transforms with $\greaterthanfunction{z}$.
What is more, quantifiers demanding that at most $z\in\nn$ are satisfied would be representable by transforms with an analogously defined function $\ones_{\leq z}$.
Such customized quantifiers appear for example in the $\mathrm{OWL\,2}$ standard of description logics (see \cite{rudolph_foundations_2011} and \secref{sec:kgRepresentation}).

% basis encodings
As will be discussed in \charef{cha:basisCalculus}, any coordinatewise transform can be performed by a contraction of a basis encoding of the tensor with a head vector prepared by the transform function (see \theref{the:tensorFunctionComposition}).
In the case here, a direct implementation would require a dimension of these head variables by $\inddim$, which can be infeasible when having large object sets.

% Prenex
To summarize, let us assume a formula is in its prenex normal form, that is a collection of quantifiers are acting on a qantifier free part.
We can represent its grounding tensor by
\begin{itemize}
    \item Instantiations of the tom groundings with the assigned variables, as contractions of the basis encoding of the world tensor with atom selecting tensors.
    \item Propositional formula acting on the head variables of the predicate instantiations, representing the connectives combining the formula.
    \item Quantifiers as a composition of contractions closing the quantified variable and coordinatewise transforms with the respective greater-than indicators.
\end{itemize}



\subsect{Storage in basis CP decomposition}\label{sec:basisCPgrounding}

In many situations, grounding cores are sparse and representations as single tensor cores comes with a drastic overhead.
We often encounter sparse grounding tensors, where the number of non-zero coordinates (to be investigated by basis CP ranks in \charef{cha:sparseRepresentation}) satisfies
\begin{align*}
    \sparsityof{\groundingof{\folexformula}} << \inddim^{\cardof{\indvariableof{\folexformula}}} \, .
\end{align*}
In this case, since most coordinates vanish, the basis CP decomposition (see \secref{sec:basisCP}) enables a representation of the grounding with significantly lower storage demand, see \theref{the:sparseBasisCP}.
This is particularly useful for representing large relational databases, where each object has only a few relations with others, while the majority of possible relations remains unsatisfied.
We depict such CP decomposition of a formula grounding in \theref{fig:groundingCP}.

% Standard KB Encoding and Assumptions
Most logical syntaxes exploit $\ell_0$-sparsity, explicitly storing only known assertions.
The interpretation of unspecified assertions depends on the underlying assumptions.
Under the Closed World Assumption, for example, all unspecified assertions are assumed to be false.

\begin{figure}[t]
    \begin{center}
        \begin{tikzpicture}[scale=0.35, yscale=-1, thick] % , baseline = -3.5pt





%\drawatomindices{0}{2}


\draw (-1,1) rectangle (5,-1);
\node[anchor=center] (text) at (2,0) {$\groundingof{\exformula}$};

%\draw[->] (2,-1) -- (2,-3) node[midway, right] {\tiny $\atomlegindexof{\exformula}$};
%\draw[] (3,-3) rectangle (1, -5);
%\node[anchor=center] (text) at (2,-4) {\small $\tbasis$};
%\draw[->] (4,-1) -- (4,-3) node[midway, right] {\tiny $\datindex$};
%\draw[dashed] (3,-3) rectangle (5, -5);
%\node[anchor=center] (text) at (4,-4) {\small $\ones$};

\draw[] (0,1) -- (0,3) node[midway,left] {\tiny $\exindividualof{0}$};
\draw[] (1.5,1) -- (1.5,3) node[midway,left] {\tiny $\exindividualof{1}$};
\node[anchor=center] (text) at (2.75,2) {$\cdots$};
\draw[] (4,1) -- (4,3) node[midway,right] {\tiny $\exindividualof{\variableorder\shortminus1}$};

\node[anchor=center] (text) at (7,0) {${=}$};


\begin{scope}[shift={(10,2)}]

\newcommand{\conposseldec}{4.5,-5.5}

\draw[fill] (\conposseldec) circle (0.25cm);
\draw (\conposseldec) -- (4.5,-7.5) node[midway, right] {\tiny $\datindex$};
\draw[] (3.5,-7.5) rectangle (5.5, -9.5);
\node[anchor=center] (text) at (4.5,-8.5) {\small $\ones$};

\draw[<-] (0,1) -- (0,-1) node[midway,left] {\tiny $\exindividualof{0}$};
\draw (-1,-1) rectangle (1, -3);
\node[anchor=center] (text) at (0,-2) {\small $\legcoreof{\exformula,0}$};
\draw[<-] (0,-3) to[bend right=20] (\conposseldec);


\draw[<-] (3,1) -- (3,-1) node[midway,left] {\tiny $\exindividualof{1}$};
\draw (2,-1) rectangle (4, -3);
\node[anchor=center] (text) at (3,-2) {\small $\legcoreof{\exformula,1}$};
\draw[<-] (3,-3) to[bend right=20]  (\conposseldec);

\node[anchor=center] (text) at (6,-2) {$\cdots$};

\draw[<-] (9,1) -- (9,-1) node[midway,left] {\tiny $\exindividualof{\variableorder\shortminus1}$};
\draw (8,-1) rectangle (10, -3);
\node[anchor=center] (text) at (9,-2) {\small $\legcoreof{\exformula,\variableorder\shortminus1}$};
\draw[<-] (9,-3) to[bend left=20]  (\conposseldec);




\end{scope}

		


\end{tikzpicture}
    \end{center}
    \caption{Basis CP Decomposition of the grounding of $\folexformula$, following the scheme of \theref{the:sparseBasisCP}.
    Instead of direct storage of the grounding tensor $\groundingof{\folexformula}$, the non-zero coordinates are enumerated by a variable $\datvariable$ and the corresponding coordinates stored in leg-matrices $\legcoreof{\folexformula,\indenumerator}$.}
    \label{fig:groundingCP}
\end{figure}

\subsect{Summary}

Let us summarize the tensor encodings of the representations of the different concepts, given a single \firstOrderLogic{} world:
\begin{center}
    \begin{tabular}{|p{\threecolumnwidth}|p{\threecolumnwidth}|p{\threecolumnwidth}|}
        \hline
        \textbf{Concept}        & \textbf{Representation}                       & \textbf{Decomposition}                             \\
        \hline
        Constant Symbol         & Basis vector                                  &                                                    \\
        \hline
        Function Symbol         & \BasisEncoding{}: Boolean and directed tensor &                                                    \\
        Term                    & ""                                            & Acyclic tensor network of represented functions    \\
        \hline
        Predicate Symbol        & \CoordinateEncoding{}: Boolean tensor         &                                                    \\
%        Relation & Boolean tensor \\
        Quantifier-free Formula & ""                                            & Contraction of represented terms with relations    \\
        Formula                 & ""                                            & Transform of corresponding quantifier-free formula \\
        \hline
    \end{tabular}
\end{center}


\subsect{Example: Relational Databases}

A database is understood as a specific \firstOrderLogic{} world, and are operations on such a single world.
Queries are described by a formula $\impformula$, which are asked against a specific world $\worldindices$ to retrieve the grounding $\groundingof{\impformula}$.
The variables of such formulas are called projection variables.
The answer $\groundingof{\impformula}$ of a query is most conveniently represented as a list of solution mappings from the projection variables to objects in the world, such that the query formula is satisfied.
Answering a query by solution mappings corresponds with finding the basis CP Decomposition (see \secref{sec:basisCP}) of $\groundingof{\impformula}$.
We can understand these solution mappings as stored in the leg-matrices $\legcoreof{\folexformula,\indenumerator}$ (see \figref{fig:groundingCP}).

Let us give with the outer join an example of a popular operation to define queries, which efficient execution and storage can be improved based on considerations in the tensor network formalism.

\begin{definition}[Outer join]
    Let there be a world $\worldindices$ and formulas $\extformulaof{\selindex}$ depending on variables $\indvariableof{\nodesof{\selindex}}$, which have grounding tensors by
    \begin{align*}
        \groundingofat{\extformulaof{\selindex}}{\indvariableof{\node}} \, : \,  \bigtimes_{\node\in\nodesof{\selindex}}[\inddimof{\node}] \rightarrow \ozset \, .
    \end{align*}
    Then their (outer) $\joinsymbol$ is defined as the grounding of their conjunctions, as
    \begin{align*}
        \groundingofat{\joinsymbol\left(\extformulaof{0},\ldots,\extformulaof{\seldim-1}\right)}{\bigcup_{\selindexin}\indvariableof{\nodesof{\selindex}}}
        = \contractionof{\groundingofat{\extformulaof{\selindex}}{\indvariableof{\nodesof{\selindex}}}\,:\,\selindexin}{\bigcup_{l\in[p]}\indvariableof{\nodesof{\selindex}}} \, .
    \end{align*}
\end{definition}

%Visualization and efficiency
We can understand the $\joinsymbol$ of groundings by a factor graph, where each grounding tensor decorates the hyperedge to the node set $\nodesof{\selindex}$.
The projection variable assignment to each formula combined in a $\joinsymbol$ operation provide a basic tensor network format to store the output of the operation.
There are thus situations, in which the solution map storage corresponding with a CP Decomposition comes with unnecessary overheads compared with other formats.

% Coordinatewise transform
We can also understand the $\joinsymbol$ operation as a coordinatewise transform (see \defref{def:coordinatewiseTransform}) with the product as transform function.
To make this connection solid, one would need to extend each joined formula trivially to the variables appearing in other formulas.

% Evaluation similar constraint propagation
The efficiency of evaluating the contraction to a $\joinsymbol$ operation might be improved by understanding it as an Constraint Satisfaction Problem (see \charef{cha:logicalReasoning}).
When applying efficient Message Passing algorithms such as Knowledge Propagation (see \algoref{alg:knowledgePropagation}), the groundings can be sparsified by local constraint propagation operations before turning to more global and more demanding contraction operations.
Here the groundings $\groundingof{\extformulaof{\selindex}}$ would be used to initialize Knowledge Cores $\kcoreof{\edge}$ and sequentially sparsified during the algorithm.

%\begin{example} % WOULD NEED OVERWORK: DRAW!
%	For example take a query with many basic graph patterns with pairwise different projection variables.
%	The global CP Decomposition would come here with an exponential storage overhead compared with storage as a tensor product of CP Decompositions to each Basic graph pattern.
%\end{example}

%% CONFUSING?
%\begin{remark}[Distinguishing from probabilistic queries]
%	Let us distinguish the discussion here from those of queries in probabilistic reasoning, which have two main differences.
%	First, we ask queries against all possible pairs of variables, instead of asking the probability of satisfaction of a specific formula.
%	Second, since we made the epistemologic assumption of knowing possibilities and not probabilities in logics, a query is answered by a truth value.
%	We then only output in the shape of solution mappings the variable assignments where the query formula is true.
% 	Thus, the queries here can be thought of as a batch of probabilistic queries with Boolean answers.
%	% Alternative -> Later?
%	Probabilistic queries can furthermore be understood in terms of the data extraction process described in this section.
%	We can ask the query in probabilistic form (decomposed into atomic formulas) on the resulting empirical distribution.
%	This results in the ratio of the worlds satisfying the query among those worlds satisfying the extraction query $\impformula$.
%\end{remark}







\sect{\SemanticStructure{}}

We now investigate the \semanticStructure{} of a \firstOrderLogic{} theory, when restricting to database semantics.
This involves the representation of collections of worlds, where each has a \substitutionStructure{} as discussed above.

\subsect{World enumerating variables}

Given database semantics, we have a finite set of worlds, which we enumerate by tuples of variables $\catvariable$.

\begin{definition}[World enumerating variables]
    \label{def:worldEnumeratingVariables}
    Given a \firstOrderLogic{} theory with constant symbols $\worlddomain$, function symbols $\{\folfunctionof{\secatomenumerator}\wcols\secatomenumeratorin\}$ and predicate symbols $\{\folpredicateof{\atomenumerator}\wcols\atomenumeratorin\}$, we enumerate the world under database semantics by a tuple
    \begin{align*}
        \worldvariables
        = \left(\bigtimes_{\seccatenumeratorin}\bigtimes_{\indindexof{\seccatenumerator}\in[\inddim]^{\indorderof{\seccatenumerator}+1}} \catvariableof{\seccatenumerator,\indindexof{\seccatenumerator}} \right)
        \times \left(\bigtimes_{\catenumeratorin}\bigtimes_{\indindexof{\catenumerator}\in\in[\inddim]^{\indorderof{\catenumerator}}} \catvariableof{\catenumerator,\indindexof{\catenumerator}} \right)
    \end{align*}
    of boolean variables.
    In the world indexed by $\worldindices$, the grounding tensor of the function $\folfunctionof{\secatomenumerator}$ is
    \begin{align*}
        \bencodingofat{\groundingof{\folfunctionof{\secatomenumerator}}}{\indvariableof{\secatomenumerator},\indvariableof{[\inddimof{\secatomenumerator}]}}
        = \sum_{\indindexof{\seccatenumerator}\in[\inddim]^{\indorderof{\seccatenumerator}+1}\wcols\indindexof{\seccatenumerator}=1} \onehotmapofat{\indindexof{\seccatenumerator}}{\indvariableof{\seccatenumerator}}
    \end{align*}
    and of the predicate $\folpredicateof{\atomenumerator}$
    \begin{align*}
        \groundingofat{\folpredicateof{\atomenumerator}}{\indvariableof{[\inddimof{\atomenumerator}]}}
        = \sum_{\indindexof{\seccatenumerator}\in[\inddim]^{\indorderof{\atomenumerator}}\wcols\indindexof{\atomenumerator}=1}
        \onehotmapofat{\indindexof{\atomenumerator}}{\indvariableof{\atomenumerator}} \, .
    \end{align*}
\end{definition}

We further have the restriction that each basis encoded function is a directed tensor.
To restrict to those worlds, where this is true, we further introduce the validation base measure $\basemeasureat{\worldvariables}$ with coordinates
\begin{align*}
    \basemeasureat{\indexedworldvariables} =
    \begin{cases}
        1 & \ifspace \uniquantwrtof{\secatomenumeratorin}{\sum_{\indindexof{\secatomenumerator}\in[\inddim]^{\indorderof{\seccatenumerator}+1}}=1} \\
        0 & \text{else}
    \end{cases} \, .
\end{align*}

\subsect{Representation of terms and formulas}

Combining its \substitutionStructure{} and \semanticStructure{} we represent a \firstOrderLogic{} term $\folterm$ with arity $\inddimof{\folterm}$ as a tensor
\begin{align*}
    \bencodingofat{\folterm}{\indvariableof{\folterm},\indvariableof{[\inddimof{\folterm}]},\worldvariables} =
    \sum_{\worldindices\wcols\basemeasureat{\indexedworldvariables}=1}
    \bencodingofat{\groundingof{\folterm}}{\indvariableof{\folterm},\indvariableof{[\inddimof{\folterm}]}}
    \otimes \onehotmapofat{\worldindices}{\worldvariables} \, .
\end{align*}
and a formula as the tensor
\begin{align*}
    \folexformulawith =
    \sum_{\worldindices\wcols\basemeasureat{\indexedworldvariables}=1} \groundingofat{\folexformula}{\indvariableof{\folexformula}}
    \otimes \onehotmapofat{\worldindices}{\worldvariables} \, .
\end{align*}

From \defref{def:worldEnumeratingVariables} the representation of predicate and function symbols are directly derived from the $\worldvariables$, that is
\begin{align*}
    \folpredicateofat{\atomenumerator}{\indvariableof{[\inddimof{\atomenumerator}]},\indexedworldvariables}
    = \sum_{\indindexof{\seccatenumerator}\in[\inddim]^{\indorderof{\seccatenumerator}+1}\wcols\indindexof{\seccatenumerator}=1}
    \onehotmapofat{\indindexof{\seccatenumerator}}{\indvariableof{\seccatenumerator}}
\end{align*}
and
\begin{align*}
    \bencodingofat{\folfunctionof{\secatomenumerator}}{\indvariableof{\secatomenumerator},\indvariableof{[\inddimof{\secatomenumerator}]},\indexedworldvariables}
    = \sum_{\indindexof{\seccatenumerator}\in[\inddim]^{\indorderof{\seccatenumerator}+1}\wcols\indindexof{\seccatenumerator}=1} \onehotmapofat{\indindexof{\seccatenumerator}}{\indvariableof{\seccatenumerator}}
\end{align*}

\subsect{Case of \propositionalLogic{}}

\PropositionalLogic{} as discussed in (see \charef{cha:logicalRepresentation}) is the special case of \firstOrderLogic{} with empty sets of constant and function symbols, and $\indorderof{\atomenumerator}=0$ for the predicate symbols.
With the convention $[]^{0}=[2]$ the worlds are enumerated by the tuple
\begin{align*}
    \worldvariables = \bigtimes_{\atomenumeratorin} \catvariableof{\atomenumerator}
\end{align*}
which we have in previous chapters abbreviated by $\shortcatindices$.


To represent logical formulas as sets of possible worlds, and distributions of worlds, we applied in \parref{par:one} one-hot encodings of possible worlds.
For the case of \propositionalLogic{}, this is
\begin{align*}
    \onehotmapofat{\worldindices}{\shortcatvariables}
    = \bigotimes_{\atomenumeratorin} \onehotmapofat{\catindexof{\atomenumerator}}{\catvariableof{\atomenumerator}} \, .
\end{align*}

\subsect{One-hot encoding of worlds}

Let us now generalize the one-hot encodings of propositional logic worlds to worlds of \firstOrderLogic{}.
To encode the boolean tensors $\worldindices$ describing \firstOrderLogic{}s as basis elements of a tensor space, we take the one-hot encoding
\begin{align*}
    \onehotmap \defcols \worldindexset \rightarrow \worldtensorspace
\end{align*}
defined by
\begin{align*}
    \onehotmapofat{\worldindices}{\worldvariables}
    =
    \left(\bigotimes_{\seccatenumeratorin}\bigotimes_{\indindexof{\seccatenumerator}\in[\inddim]^{\indorderof{\seccatenumerator}+1}}
        \onehotmapofat{\catindexof{\seccatenumerator,\indindexof{\seccatenumerator}}}{\catvariableof{\seccatenumerator,\indindexof{\seccatenumerator}}}\right)
    \otimes\left(\bigotimes_{\atomenumeratorin}\bigotimes_{\indindexof{\atomenumerator}\in[\inddim]^{\indorderof{\atomenumerator}}}
        \onehotmapofat{\catindexof{\catenumerator,\indindexof{\catenumerator}}}{\catvariableof{\catenumerator,\indindexof{\catenumerator}}} \right)
\end{align*}
This is a tensor of order $\catorder\cdot\inddim^{\indorder}$, in a tensor space of dimension $2^{\left(\catorder\cdot\inddim^{\indorder}\right)}$.
Storage of such tensors in naive formats would not be possible.
However, the basis $\cpformat$ format discussed in \charef{cha:sparseRepresentation} still provides storage with demand linear in the order $\catorder\cdot\inddim^{\indorder}$.


\subsect{Probability distributions}

Having established the formalism of one-hot encodings also in the case of \firstOrderLogic{} worlds, we can now proceed with the definition of distributions and formulas, analogously to the development in \parref{par:one}.
Probability distributions over worlds coinciding on their domain are then non-negative and normed tensors $\probat{\worldvariables}$ where each coordinate of a world $\worldindices$ is captured by a boolean random variable $\catvariableof{\atomenumerator,\shortindindices}$, indicating whether the $\atomenumerator$-th predicate holds on the object tuple indexed by $\shortindindices$.

% High-dimensional - watch out for repetitions!
We notice, that by definition these probability distributions are distributions of
\begin{align*}
    \left(\sum_{\catenumeratorin}\inddim^{\indorderof{\catenumerator}}\right) +  \left(\sum_{\seccatenumeratorin}\inddim^{\indorderof{\seccatenumerator}}\right)
\end{align*}
Booleans.
% One-hot encodings minimal
Unfortunately, it is not possible to design encoding spaces of smaller dimension, when our aim is to get any distribution over possible worlds by an element in the encoding space.
This is due to the fact, that one-hot encodings provide a basis in the tensor space, as will be shown in \charef{cha:coordinateCalculus}.
The reason for the large encoding space dimension is therefore rooted in the equal number of possible worlds and not in an overhead in the dimension of the one-hot encoding space.
We will later in this chapter investigate methods to handle such high-dimensional distributions in the formalism of exponential families.





\sect{Representation of Knowledge Graphs}\label{sec:kgRepresentation}

Let us now represent a specific fragment of \firstOrderLogic{}, namely Description Logics which Knowledge Bases are often referred to as Knowledge Graphs.
We here use the $\mathrm{OWL\,2}$ standard, which encodes the syntax of the description logic $\mathcal{SROIQ(D)}$ \cite{rudolph_foundations_2011}.

\subsect{Representation as unary and binary predicates}

% Reduction to binary
Predicates in knowledge graphs are binary (owl:ObjectProperties) and unary (owl:Class).
%Larger formulas are created by logical connections of these atomic formulas using disjunctions, conjunctions etc.
We enumerate the predicates by $[\folpredicateorder]$, the objects in the domain $\worlddomain$ by $[\inddim]$, and extend the unary predicates to binaries by tensor product with $\onehotmapofat{0}{\indvariableof{1}}$.
A Knowledge Graph on the set $\worlddomain$ of constants (owl:NamedIndividuals) is then the tensor
\begin{align*}
    \kgat{\selvariable,\indvariableof{0},\indvariableof{1}} : [\folpredicateorder] \times [\inddim] \times [\inddim] \rightarrow \ozset \, .
\end{align*}


\subsect{Representation as ternary predicate}\label{subsec:knowledgeGraphTernaryRep}

It has been particulary convenient to represent a Knowledge Graph instead as a grounding of a single ternary predicate $\rdf$.
To this end, the predicates $\folpredicateof{\catenumerator}$ and another object $\mathrdftype$ are added to a domain $\worlddomain$, by extending the $\inddim$ and the index interpretation function accordingly.


% RDF triple: Alternative viewpoint to collection of unary and binary predicates!
Following our notation we understand a Knowledge Graph as a grounding of the rdf triple relation $\rdf$ (being a formula of order 3) on a specific world $\kg$ with individuals $\worlddomain$

We then construct a grounding tensor $\kggroundingof{\rdf}$ out of the world $\kgat{\selvariable,\indvariableof{0},\indvariableof{1}}$ by
\begin{align*}
    \kggroundingof{\rdf} : [\inddim] \times [\inddim] \times [\inddim] \rightarrow \ozset
\end{align*}
where
\begin{align*}
    &\kggroundingofat{\rdf}{\indexedindvariableof{s}, \indexedindvariableof{p}, \indexedindvariableof{o}} \\
    &\quad =
    \begin{cases}
        \kgat{\selvariable=\indindexof{s},\indvariableof{0}=\indindexof{o},\indvariableof{1}=0}
        & \ifspace \indindexof{p} = \invindexinterpretationat{\mathrdftype} \\
        \kgat{\selvariable=\indindexof{p},\indvariableof{0}=\indindexof{s},\indvariableof{1}=\indindexof{o}}
        & \ifspace \indindexof{p} = \invindexinterpretationat{\folpredicateof{\catenumerator}} \quad \text{for some} \quad \catenumerator \\
        0  \quad & \text{else}
    \end{cases} \, .
\end{align*}


Slicing the tensor $\kggroundingof{\rdf}$ along the predicate axis retrieves specific information about roles and can be efficiently be performed on these formats.
The role $\mathrdftype$ has a specific meaning, since it contains from a DL perspective classifications (memberships of named concepts).
Further slicing the tensor along object axis therefore results in membership lists for specific classes (concepts).
One can thus regard $\mathrdftype$ as a placeholder for unitary formulas in a space of binary formulas.

% Triple Stores, sparsity
Exploiting the $\ell_0$-sparsity now leads to a so-called triple store, where $\kggroundingof{\rdf}$ is stored by a listing of those triples $\indindexof{\subsymbol},\indindexof{\predsymbol},\indindexof{\objsymbol}$ such that $\kggroundingofat{\rdf}{\indexedindvariableof{s}, \indexedindvariableof{p}, \indexedindvariableof{o}}=1$
A recent implementation of a triple store exploiting these intuitions is $\mathrm{TENTRIS}$, see \cite{pan_tentris_2020}.
In this work, such decompositions are generalized into more generic CP formats, see \charef{cha:sparseRepresentation}.
% Approximation of KG Groundings
Approximations of grounding tensors by decompositions leads to embeddings of the individuals such as $\mathrm{Tucker}$, $\mathrm{ComplEx}$ and $\mathrm{RESCAL}$ (see \cite{nickel_review_2016}).

% Sparse representation
%Sparse representation of the grounding tensor to a knowledge graph is of central importance, as investigated in \cite{pan_tentris_2020}.
%We here do basis CP for sparse representation.


% basis encoding
For our purposes of evaluating logical formulas such as $\sparql$ queries we use the basis encoding of the groundings, which are depicted by
\begin{center}
    \begin{tikzpicture}[scale=0.3, thick] % , baseline = -3.5pt

    \draw[->-] (0,1)--(0,3) node[midway,left] {\tiny $\headvariable$};
    \draw (-3,1) rectangle (3,-1);
    \node[anchor=center] (text) at (0,0) {\small $\rencodingof{\kggroundingof{\rdf}}$};
    \draw[-<-] (-2,-1)--(-2,-3) node[midway,left] {\tiny $\sindvariable$};
    \draw[-<-] (0,-1)--(0,-3) node[midway,left] {\tiny $\pindvariable$};
    \draw[-<-] (2,-1)--(2,-3) node[midway,left] {\tiny $\oindvariable$};

\end{tikzpicture}
\end{center}




\subsect{$\sparql$ Queries}

The $\sparql$ query language is a syntax to express \firstOrderLogic{} formulas $\folexformula$ and intended to be evaluated given a Knowledge Graph.
We here consider tensor network representations of the $\mathrm{WHERE}{\cdot}$ block.
Given a specific knowledge graph $\kggroundingof{\rdf}$, the execution of query is the interpretation $\groundingof{\folexformula}$, typical represented in a sparse basis CP format where each slice represents a solution mapping.

\subsubsect{Triple Patterns}

\red{Central to $\sparql$ queries are triple patterns, which we understand as slicings of the tensor $\kggroundingof{\rdf}$.}
To each so-called triple pattern we build a corresponding creation tensor. %(see \defref{def:atomCreatingTensor}).
The triple pattern is then evaluated by contraction of the atom creating tensor with $\kggroundingof{\rdf}$.

Let us now provide examples of such pattern tensors.
A unary triple patterns contains a single projection variable, typically related with the subject variable $\sindvariable$ of $\kggroundingof{\rdf}$.
The corresponding pattern tensor is then
\begin{align*}
    \atomcreatorofat{\kgtriple{\provariable}{\mathrdftype}{\folpredicateof{\catenumerator}}}{
        \sindvariable, \pindvariable, \oindvariable, \provariable
    }
    = \identityat{\sindvariable,\provariable}
    \otimes \onehotmapofat{\invindexinterpretationat{\mathrdftype}}{\pindvariable}
    \otimes \onehotmapofat{\invindexinterpretationat{\folpredicateof{\atomenumerator}}}{\oindvariable} \, .
\end{align*}

Binary triple patterns come with two projection variables, typically related with the subject and the object variables $\sindvariable$ and $\oindvariable$.
The pattern tensor to the $\catenumerator$-th predicate is then
\begin{align*}
    \atomcreatorofat{\kgtriple{\provariableof{0}}{\folpredicateof{\catenumerator}}{\provariableof{1}}}{
        \sindvariable, \pindvariable, \oindvariable, \provariableof{0}, \provariableof{1}
    }
    = \identityat{\sindvariable,\provariableof{0}}
    \otimes \onehotmapofat{\invindexinterpretationat{\folpredicateof{\atomenumerator}}}{\pindvariable}
    \otimes \identityat{\oindvariable,\provariableof{1}} \, .
\end{align*}

Contraction with these pattern tensor evaluated the specific triple pattern, and outputs in a boolean tensor the indicator, which objects are members of a specific class (for unary patterns) or which pair of objects are related by a specific relation.
Again, the output of such contractions is a subset encodings of the set of solutions (see \defref{def:subsetEncoding}).

%%%%%%%%%%%% END OF FRIDAY 14.3.
%%%%%%%%%%%%

% Examples
Examples of triple patterns, drawn in \figref{fig:triplePatterns} are
\begin{itemize}
    \item Unary triple pattern with one variable, representing a formula with a single projection variable.
    For the example $\exunarytriple$ see Figure~\ref{fig:triplePatterns}a.
    \begin{align*}
        \atomcreatorofat{\kgtriple{\provariable}{\mathrdftype}{\folpredicateof{\catenumerator}}}{
            \sindvariable, \pindvariable, \oindvariable, \provariable
        }
        = \identityat{\sindvariable,\provariable}
        \otimes \onehotmapofat{\invindexinterpretationat{\mathrdftype}}{\pindvariable}
        \otimes \onehotmapofat{\invindexinterpretationat{\exaunaryrelation}}{\oindvariable}
    \end{align*}
    If and only if the output slice is $\tbasis$, then the corresponding object encoded by the input indices is of class $\exaunaryrelation$.
    \item Binary triple pattern with two variables, representing a formula with two projection variables.
    For the example  $\exbinarytriple$ see Figure~\ref{fig:triplePatterns}b.
    If and only if the output slice is $\tbasis$, then the corresponding object tuple encoded by the input indices has a relation $\exabinaryrelation$.
\end{itemize}

% Projection picture
The composition $\psi (\psi^T)$ of the matrification of the tensor $\psi$ is an orthogonal projection.
That means that applying $\psi (\psi^T)$ is the same map as applying once.


\begin{figure}[t]
    \begin{center}
        \begin{tikzpicture}[scale=0.3,thick] % , baseline = -3.5pt

    \begin{scope}
        [shift={(0,0)}]

        \node[anchor=center] (text) at (-12,2) {$a)$};

        \begin{scope}
            [shift={(-7,2)}]

            \draw (0,-3) rectangle (-6,-5);
            \draw[<-] (-3,-1)--(-3,-3) node[midway,right] {\tiny $\headvariable$};
            \node[anchor=center] (text) at (-3,-4) {$\rencodingof{\kggroundingof{\exunarytriple}}$};
            \draw[<-] (-3,-5)--(-3,-7) node[midway,left] {\tiny $\provariableof{0}$};

        \end{scope}

        \node[anchor=center] (text) at (-5.5,-2) {${=}$};

        \draw[->] (0,1)--(0,3) node[midway,left] {\tiny $\headvariable$};
        \draw (-4,1) rectangle (4,-1);
        \node[anchor=center] (text) at (0,0) {\small $\rencodingof{\kggroundingof{\rdf}}$};

        \draw (-2,-3) rectangle (-4,-5);
        \draw[<-] (-3,-1)--(-3,-3) node[midway,left] {\tiny $\sindvariable$};
        \node[anchor=center] (text) at (-3,-4) {$\delta$};
        \draw[<-] (-3,-5)--(-3,-7) node[midway,left] {\tiny $\provariableof{0}$};

        \draw (-1,-3) rectangle (1,-5);
        \draw[<-] (0,-1)--(0,-3) node[midway,left] {\tiny $\pindvariable$};
        \node[anchor=center] (text) at (0,-4) {$\onehotmapof{\invrdftypesymbol}$};

        \draw (2,-3) rectangle (4,-5);
        \draw[<-] (3,-1)--(3,-3) node[midway,left] {\tiny $\oindvariable$};
        \node[anchor=center] (text) at (3,-4) {$\onehotmapof{\exaunaryrelation}$};

    \end{scope}


    \begin{scope}
        [shift={(24,0)}]

        \node[anchor=center] (text) at (-13,2) {$b)$};

        \begin{scope}
            [shift={(-8,2)}]

            \draw (0.5,-3) rectangle (-6.5,-5);
            \draw[<-] (-3,-1)--(-3,-3) node[midway,right] {\tiny $\headvariable$};
            \node[anchor=center] (text) at (-3,-4) {$\rencodingof{\kggroundingof{\exbinarytriple}}$};

            \draw[<-] (-2,-5)--(-2,-7) node[midway,right] {\tiny $\provariableof{0}$};
            \draw[<-] (-4,-5)--(-4,-7) node[midway,left] {\tiny $\provariableof{1}$};

        \end{scope}

        \node[anchor=center] (text) at (-5.5,-2) {${=}$};

        \draw[->] (0,1)--(0,3) node[midway,left] {\tiny $\headvariable$};
        \draw (-4,1) rectangle (4,-1);
        \node[anchor=center] (text) at (0,0) {\small $\rencodingof{\kggroundingof{\rdf}}$};

        \draw (-2,-3) rectangle (-4,-5);
        \draw[<-] (-3,-1)--(-3,-3) node[midway,left] {\tiny $\sindvariable$};
        \node[anchor=center] (text) at (-3,-4) {$\delta$};
        \draw[<-] (-3,-5)--(-3,-7) node[midway,left] {\tiny $\provariableof{1}$};

        \draw (-1,-3) rectangle (1,-5);
        \draw[<-] (0,-1)--(0,-3) node[midway,left] {\tiny $\pindvariable$};
        \node[anchor=center] (text) at (0,-4) {$\onehotmapof{\exabinaryrelation}$};

        \draw (2,-3) rectangle (4,-5);
        \draw[<-] (3,-1)--(3,-3) node[midway,left] {\tiny $\oindvariable$};
        \node[anchor=center] (text) at (3,-4) {$\delta$};
        \draw[<-] (3,-5)--(3,-7) node[midway,right] {\tiny $\provariableof{0}$};

    \end{scope}

\end{tikzpicture}
    \end{center}
    \caption{Triple patterns of $\sparql$ as tensor networks.
    a) Example of unary triple pattern $\exunarytriple$ specifying whether an individual $\indexinterpretationof{\indindexof{1}}$ is a member of class $C$.
    %Here by $0$ we denote the element $\invindexinterpretationat{\mathrdftype}$
    b) Example of a binary triple pattern $\exbinarytriple$ specifying whether individuals $\indexinterpretationof{\indindexof{1}}$ and $\indexinterpretationof{\indindexof{2}}$ have a relation $R$.
    By $\onehotmapof{\invrdftypesymbol},\onehotmapof{\exaunaryrelation},\onehotmapof{\exabinaryrelation}$ we denote the one-hot encodings of the enumeration of the resources $rdf:type, C$ and $R$.
    }
    \label{fig:triplePatterns}
\end{figure}




\subsubsect{Basic Graph Patterns}

Generic $\sparql$ queries are compositions of triple patterns by logical connectives. % Except for some stuff like regex
These triple patterns possibly share projection variables.
Statements in $\sparql$ can be translated into \propositionalLogic{} combining the triple patterns:
\begin{center}
    \begin{tabular}{|c|c|c|}
        \hline
        \textbf{$\sparql$}                                        & \textbf{\PropositionalLogic{}} & \textbf{Tensor Representation}                                                                   \\
        \hline
        $\{f_1, f_2\}$                                            & $f_1\land f_2$                 & $\bencodingofat{\land}{\headvariableof{f_1\land f_2},\headvariableof{f_1},\headvariableof{f_2}}$ \\
        \hline
        $\mathrm{UNION}\{f_1, f_2\} $                             & $f_1\lor f_2$                  & $\bencodingofat{\lor}{\headvariableof{f_1\lor f_2},\headvariableof{f_1},\headvariableof{f_2}}$   \\
        \hline
        $\mathrm{FILTER}\,\,\mathrm{NOT}\,\,\mathrm{EXISTS}\{f\}$ & $\lnot f$                      & $\bencodingofat{\lnot}{\headvariableof{\lnot f},\headvariableof{f}}$                             \\
        \hline
    \end{tabular}
\end{center}

If a $\sparql$ query consists of these keywords, we find a straight forward corresponding network of triple patterns and encoded logical connectives, by applying our findings of \secref{sec:folConnectiveRepresentation}.
To this end, we prepare for each appearing triple pattern the corresponding pattern tensor, and a copy of $\kggroundingof{\rdf}$.
Here we also copy the term variables $\sindvariable,\pindvariable$ and $\oindvariable$, to ensure that each copy of $\kggroundingof{\rdf}$ shares variables with a single pattern tensor.
Projection variables are not copied, since we need to keep track of them shared among triple patterns.
Then we prepare the basis encoding of logical connectives according to the hierarchy specified in the $\sparql$ query.
Finally we add a $\tbasis$-vector to the final head variable representing the complete $\sparql$ query, to restrict the support to coordiantes corresponding with solution mappings.
We then contract the resulting tensor network, leaving all projection variables open.

If a projection variable is not appearing in the $\mathrm{SELECT}$ statement in front of the $\mathrm{WHERE}\{\cdot\}$-block, we simply exclude it from the open variables of the described contraction.
Note that in that case, the coordinates contain solution counts, i.e. how many assignments to the dropped variable have been a $1$ coordinate.
We can drop this additional information simply by performing a coordinatewise transform with the greater zero indicator $\existquanttrafo$.

% Effective calculus alternative
Here we represented a $\sparql$ query $\impformula$ consistent of multiple triple pattern by instantiating a head variables to each triple pattern.
Alternatively, the more direct hybrid calculus developed in \secref{sec:hybridCalculus} can be applied and the additional head variables avoided.
This is especially compelling, when the $\mathrm{WHERE}\{\cdot\}$-block does not contain further keywords, i.e. it is the conjunction of all triple patterns.
In that case, we avoid the instantiation of head variables (i.e. close the head variables separately by $\tbasis$-vectors) and represent the query by a contraction of all triple pattern tensors.

% Expressivity
We further notice, that any propositional formula acting on the head variables of the triple patterns can be expressed by a hierarchical combination of the key words in the above table.
To find the expression, one can transform a given formula into its conjunctive or disjunctive normal form and apply the statements according to the apperaing operations $\land,\lor$ and $\lnot$.


%% Further $\sparql$ features
%Further $\sparql$ features, which cannot be expressed by a tensor network are:
%\begin{itemize}
%    \item $\mathrm{FILTER}\{\cdot\}$ does not depend on triple patterns (e.g. numeric inequalities, regex functions on strings).
%    We can regard it as another basic formula, which does not result from a slicing of the $\rdf$ grounding tensor.
%    Besides that, we can understand it as formulas and include it in compositions.
%    \item $\mathrm{OPTIONAL}\{\cdot\}$ would result in $\ones$ leg vectors, when there is a missing variable assignment resulting.
%\end{itemize}



\sect{Probabilistic Relational Models}

% MLN in FOL and PL
So far we have studied \MarkovLogicNetworks{} in \propositionalLogic{} as probability distributions over worlds.
In FOL they define probability distributions over relations in worlds with a fixed set of objects.
More generally, such models are probabilistic relational models (see for an overview \cite{getoor_introduction_2019}.


We in this section treat random worlds in \firstOrderLogic{} with fixed domains $\worlddomain$.

%
We in this section show, when and how we can interpret likelihoods of \MarkovLogicNetworks{} in \firstOrderLogic{} in terms of samples of a \MarkovLogicNetwork{} in \propositionalLogic{}.

\subsect{\HybridFOLNetworks{}}

% Templates
Following \cite{richardson_markov_2006} \MarkovLogicNetworks{} in \firstOrderLogic{} are templates for distributions, which instantiate random worlds when choosing a set of objects $\worlddomain$.
Given a fixed set of constants, they then define a distribution over the worlds, which objects correspond with the constants. % this is database semantics!
This applies database semantics, where only those worlds are considered, where the unique name and domain closure assumptions given a set of constants are satisfied.

We count the number of true groundings to a formula by
\begin{align*}
    \countquantifier\enumfolformula|_{\worldindices}
    = \contraction{\groundingofat{\enumfolformula}{\indvariableof{\enumfolformula}}} \, .
\end{align*}
Using these counts as statistics, we now define \HybridFOLNetworks{} as a Computation Activation Network.

\begin{definition}[\HybridFOLNetworks{} (HFLN)]
    Let there be a set $\worlddomain$ of objects and a set $\folformulaset$ of \firstOrderLogic{} formulas.
    Further let $\hybridparam$ as in \defref{def:hybridLogicNetwork} be a tuple of a subset $\hardlegset\subset[\seldim]$, a tuple $\headindexof{\hardlegset}\in\bigtimes_{\selindex\in\hardlegset}[2]$ and $\canparamwithin$.
    We then define the \HybridFOLNetwork{} as the distribution
    \begin{align*}
        \probofat{\folhlnparameters}{\worldvariables}
        &= \breakablenormalizationof{\{\bencodingofat{\countquantifier\enumfolformula}{\headvariableof{\selindex},\worldvariables}\wcols\selindexin\} \\
        & \quad \cup \{\actcorewith\wcols\selindexin\}
        \cup \{\kcoreofat{\selindex,\hardparam}{\headvariableof{\selindex}}\wcols\selindexin\}
        }{\worldvariables}
    \end{align*}
    where
    \begin{align*}
        \kcoreofat{\selindex,\hardparam}{\headvariableof{\selindex}}
        = \begin{cases}
              \fbasisat{\headvariableof{\selindex}} & \ifspace \selindex\in\hardlegset \ncond \headindexof{\selindex} = 0 \\
              \onehotmapofat{\headdimof{\selindex}-1}{\headvariableof{\selindex}} & \ifspace \selindex\in\hardlegset \ncond \headindexof{\selindex} = 1 \\
              \onesat{\headvariableof{\selindex}} & \text{else} \, .
        \end{cases}
    \end{align*}
\end{definition}



The mean parameter polytope is the convex hull of the vectors
\begin{align*}
    \sencodingofat{\folformulaset}{\indexedworldvariables,\selvariable}
\end{align*}
to the worlds $\worldindices$ with $\basemeasureat{\indexedworldvariables}=1$.
These vectors store are the counts of satisfied groundings to each formula, that is
\begin{align*}
    \sencodingofat{\folformulaset}{\indexedworldvariables,\selvariable} = \cardof{
        \{\indindexof{\enumfolformula} \wcols \groundingofat{\enumfolformula}{\indexedindvariableof{\enumfolformula}} = 1 \}
    } \, .
\end{align*}

\subsect{Propositionalization}

% Propositionalization
Let us notice, that different to the case of propositional \HybridLogicNetworks{} treated in \charef{cha:networkRepresentation}, the statistic does not consist of boolean features, when formulas contain variables and we have multiple objects.
One could, however, replace each $\enumfolformula$ by the set of the possible groundings, i.e. substitutions of the formulas variables by any tuple of objects in $\worlddomain$.
The resulting distribution would be an \HybridLogicNetwork{} with boolean statistic, which coincides with the \HybridFOLNetwork{} when posing certain weight sharing conditions on $\canparam$.
The downside of this construction is the increase in the number of features from $\seldim$ to $\sum_{\selindexin} \cardof{\worlddomain}^{\cardof{\indvariableof{\enumfolformula}}}$.
This polynomial in the cardinality of the domain set increase poses significant computational challenges, see \cite{richardson_markov_2006}.

We now show that the \HybridFOLNetwork{} can be propositionalized to a \HybridLogicNetwork{} in \propositionalLogic{}.

\begin{lemma}
    We have
    \begin{align*}
        &\contractionof{\bencodingofat{\countquantifier\enumfolformula}{\headvariableof{\selindex},\worldvariables},\actcorewith}{\worldvariables} \\
        &\quad = \contractionof{
            \{\bencodingofat{\groundingofat{\enumfolformula}{\worldindices,\indindexof{\enumfolformula}}}{\headvariableof{\indindexof{\enumfolformula}}} \wcols \indindexofin{\enumfolformula}\}
            \cup
            \{\actcoreofat{\selindex,\indexedcanparam}{\headvariableof{\indindexof{\enumfolformula}}} \wcols \indindexofin{\enumfolformula}\}
        }{\worldvariables}
    \end{align*}
    and
    \begin{align*}
        &\contractionof{\bencodingofat{\countquantifier\enumfolformula}{\headvariableof{\selindex},\worldvariables},\kcoreofat{\selindex,\hardparam}{\headvariableof{\selindex}}}{\worldvariables} \\
        &\quad = \contractionof{
            \{\bencodingofat{\groundingofat{\enumfolformula}{\worldindices,\indindexof{\enumfolformula}}}{\headvariableof{\indindexof{\enumfolformula}}} \wcols \indindexofin{\enumfolformula}\}
            \cup
            \{\kcoreofat{\selindex,\hardparam}{\headvariableof{\indindexof{\enumfolformula}}} \wcols \indindexofin{\enumfolformula}\}
        }{\worldvariables} \, .
    \end{align*}
\end{lemma}
\begin{proof}
    The first claim holds, since for any world $\worldindices$ we have
    \begin{align*}
        &\contractionof{\bencodingofat{\countquantifier\enumfolformula}{\headvariableof{\selindex},\worldvariables},\actcorewith}{\indexedworldvariables} \\
        &\quad = \prod_{\indindexof{\enumfolformula}\wcols\groundingofat{\enumfolformula}{\worldindices,\indindexof{\enumfolformula}}=1} \expof{\indexedcanparam} \\
        &\quad = \contractionof{
            \{\bencodingofat{\groundingofat{\enumfolformula}{\worldindices,\indindexof{\enumfolformula}}}{\headvariableof{\indindexof{\enumfolformula}}} \wcols \indindexofin{\enumfolformula}\}
            \cup
            \{\actcoreofat{\selindex,\indexedcanparam}{\headvariableof{\indindexof{\enumfolformula}}} \wcols \indindexofin{\enumfolformula}\}
        }{\indexedworldvariables} \, .
    \end{align*}
    For $\selindex\notin\hardlegset$ the second claim is trivial, since any contraction of a basis encoding with a trivial head is trivial.
    If $\selindex\in\hardlegset$ and $\headindexof{\selindex}=0$ then
    \begin{align*}
        &\contractionof{\bencodingofat{\countquantifier\enumfolformula}{\headvariableof{\selindex},\worldvariables},\kcoreofat{\selindex,\hardparam}{\headvariableof{\selindex}}}{\indexedworldvariables} \\
        &\quad=
        \begin{cases}
            1 & \ifspace \forall \indindexof{\enumfolformula} \defcols  \groundingofat{\enumfolformula}{\worldindices,\indindexof{\enumfolformula}} = 0\\
            0 & \text{else}
        \end{cases} \\
        &\quad= \contractionof{
            \{\bencodingofat{\groundingofat{\enumfolformula}{\worldindices,\indindexof{\enumfolformula}}}{\headvariableof{\indindexof{\enumfolformula}}} \wcols \indindexofin{\enumfolformula}\}
            \cup
            \{\kcoreofat{\selindex,\hardparam}{\headvariableof{\indindexof{\enumfolformula}}} \wcols \indindexofin{\enumfolformula}\}
        }{\indexedworldvariables} \, .
    \end{align*}
    For $\headindexof{\selindex}=1$ this holds analogously.
\end{proof}

Each substitution of the variables in $\enumfolformula$ by objects in $\worlddomain$, which satisfies the formula in the world $\worldindices$, therefore provides a factor of $\expof{\canparamat{\indexedselvariable}}$ to the probability of $\worldindices$.


%\red{Here we directly define them as exponential families distributing $\worldvariables$ for a given set of objects $\worlddomain$.}
%\red{To avoid a similar discussion as in \charef{cha:networkRepresentation} we directly allow for boolean base measures and call the distributions \HybridFOLNetworks{}.}
%
%\begin{definition}[\HybridFOLNetworks{} (HFLN)]
%    Let there be a set $\folformulaset$ of \firstOrderLogic{} formulas with maximal arity $\indorder$, which is enumerated by a selection variable $\selvariable$ of dimension $\seldim$.
%    Further, let there be a set of objects $\worlddomain$ and a boolean base measure $\basemeasureat{\shortindvariables}$.
%    The family of \HybridFOLNetworks{} $\expfamilyof{\restfolformulaset,\basemeasure}$ defined by the tuple $(\folformulaset,\worlddomain,\basemeasure)$ is the exponential family of joint distributions to the variables $\worldvariables$ with the statistics
%    \begin{align*}
%        \sstat^{\restfolformulaset}_{\selindex}\left[\indexedworldvariables\right]
%        = \contraction{\groundingof{\enumfolformula}}
%    \end{align*}
%    and the base measure $\basemeasure$.
%%    The \MarkovLogicNetwork{} instantiated for a given set of objects $\worlddomain$ and a base measure $\basemeasure$ is the random world, which is a member of the exponential family with sufficient statistics
%%    \begin{align*}
%%        \sstatcoordinateofat{\selindex}{\indexedworldvariables} = \contraction{\groundingof{\enumfolformula}}
%%        %\sstat_{\selindex}(\worldindices)  = \contraction{\groundingof{\folexformula_\selindex}} % Formulas can have different
%%    \end{align*}
%%    and canonical parameters $\weight$.
%\end{definition}
%
%Each element of the family $\expfamilyof{\restfolformulaset,\basemeasure}$ is represented by a canonical parameter $\canparamat{\selvariable}$.


%We will in the next sections explore an alternative way to apply the theory of \charef{cha:networkRepresentation} and \charef{cha:networkReasoning}, namely based on importance formulas.


%% Interpretation
%The statistics
%\begin{align*}
%    \contraction{\groundingof{\folexformula_\selindex}} % Formulas can have different
%\end{align*}
%can be interpreted as the number of substitutions to a formula, such that the formula ist satisfied.
%Each substitution satisfying a formula adds a factor of $\expof{\canparam_\selindex}$ to the probability of the respective world before normalization.


%
%When constructing a world tensor to a theory with predicates of different order, we already argued that we extend the arity of predicates by tensor products with $\onehotmapof{0}$.
%To define random world tensors, we then restrict the corresponding base measure to be supported only on those worlds where the extended predicates hold only at the individual $\exindividualof{0}$ at the extended axis.


% Comparison with PL MLN
%We choose extraction formulas $\extformulaof{\atomenumerator}$ such that any formula in the FOL MLN has a propositional equivalent (see \defref{def:propositionalEquivalent}).
%The statistic map is then a formula selecting tensor as in the propositional logic case contracted with the groundings of $\extformulaof{\atomenumerator}$.






\subsect{Conditioning on an importance formulas}

%\red{Analogous to a guard formula in \cite[Definition 6.11]{koller_probabilistic_2009}!}

%The boolean base measure $\basemeasure$ of a \HybridFOLNetwork{} is the subset encoding of the possible worlds which have a non-vanishing probability with respect to any member of the family.
%We now construct specific base measures based on a fixed grounding tensor of an importance formula.
%This will reduce the number of object tuples influencing the probability distribution in order to arrive at an interpretation of FOL MLNs as likelihoods to datasets of propositional MLNs.

To reduce the number of object tuples influencing the probability distribution, we now condition \HybridFOLNetworks{} on situations where a formula, called the importance formula, has a fixed grounding tensor.

To this end, we mark pairs of term indices relevant to the distributions by an auxiliary index $\datindexin$.
Given a set $\{\indindexof{[\indorder]}^{\datindex} \wcols \datindexin \}$ of indices to the important tuples we build a set encoding (see \defref{def:subsetEncoding})
\begin{align*}
    \fixedimpformula = \sum_{\datindexin} \left(
    \bigotimes_{\indenumeratorin} \onehotmapofat{\indindexof{\indenumerator}^{\datindex}}{\indvariableof{\indenumerator}}
    \right) \, .
\end{align*}

% Interpretation as grounding
We interpret the tensor $\fixedimpformula$ as the grounding of a formula, which we call the importance formula.

% Restricting to worlds with identical grounding
To have a constant importance formula we define a syntactic representation and restrict the support of the \HybridFOLNetwork{} to those world coinciding with groundings of the importance formula coinciding with $\fixedimpformula$ by designing a base measure
\begin{align*}
    \fixedimpbm
    = \begin{cases}
          1 & \ifspace \groundingofat{\impformula}{\indvariableof{\impformula}} = \fixedimpformula \\
          0 & \text{else}
    \end{cases} \, .
\end{align*}

% Conditioning on exquery
The base measure restricts the \HybridFOLNetwork{} to be those worlds, where $\groundingof{\impformula}$ is coincides with the fixed tensor $\fixedimpformula$.
Intuitively, $\groundingof{\impformula}$ represents certain evidence about a \firstOrderLogic{} world, whereas other formulas are uncertain.


%\begin{assumption}
%    \label{ass:importanceBasemeasure}
%    Given a base measure $\fixedimpbm$, we assume that there is an importance formula $\impformulaat{\shortindvariables}$ such that
%    \begin{align*}
%        \fixedimpbm
%        = \begin{cases}
%              1 & \ifspace \groundingofat{\impformula}{\indvariableof{\impformula}} = \fixedimpformula \\
%              0 & \text{else}
%        \end{cases} \, .
%    \end{align*}
%\end{assumption}


\subsect{Decomposition of the log likelihood}


% Extraction query
To reduce the likelihood of a world to we make the assumption that all formulas in a \HybridFOLNetwork{} are of the form
\begin{align}
    \label{eq:folImplicationForm}
%    \folexformula_{\selindex}(\individuals) =
    \enumfolformulaat{\indvariableof{\enumfolformula}}
    = \left( \impformulaat{\shortindvariables} \Rightarrow \headfolformulaofat{\selindex}{\indvariableof{\enumfolformula}} \right)
\end{align}
that is a rule with the importance formula being the premise.
In particular, we assume, that they depend on all term variables $\shortindvariables$.
If this is not the case, we extend the formula trivially on the missing term variables.
When this assumption holds, we can think of the importance formula as a conditions on individuals to satisfy a statistical relation given by $\headfolexformula$.

Towards connecting with \propositionalLogic{}, we further make the assumption, that we can decompose the formula $\headfolformulaof{\selindex}$ in what we will call extraction formulas.

\begin{assumption}
    \label{ass:propositionalHeads}
    We assume that there exist formulas $\{\extformulaofat{\catenumerator}{\worldvariables,\shortindvariables} \wcols \catenumeratorin\}$, which we refer to as atom extraction formulas, and an importance formula $\impformulaat{\shortindvariables}$ such that the following holds.
    To each \firstOrderLogic{} formula $\enumfolformula$ there is another \firstOrderLogic{} formula $\headfolformulaofat{\selindex}{\indvariableof{\enumfolformula}}$ and a propositional formula $\enumformulaat{\shortcatvariables}$ such that
    \begin{align*}
        \enumfolformulaat{\worldvariables,\shortindvariables}
        = \left( \impformulaat{\worldvariables,\shortindvariables} \Rightarrow \headfolformulaofat{\selindex}{\worldvariables,\shortindvariables} \right)
    \end{align*}
    and
    \begin{align*}
        \headfolformulaofat{\selindex}{\shortindvariables} =
        \contractionof{
            \{\enumformulaat{\shortcatvariables}\} \cup \{\bencodingofat{\extformulaof{\catenumerator}}{\catvariableof{\catenumerator},\worldvariables,\shortindvariables}
            \wcols \catenumeratorin\}
        }{\shortindvariables} \, .
    \end{align*}
\end{assumption}

We depict the assumption, that any formula is of the form \eqref{eq:folImplicationForm} in the diagram
\begin{center}
    \begin{tikzpicture}[scale=0.35, yscale=1, thick] % , baseline = -3.5pt




\draw (1,-1) rectangle (7,-3);
\node[anchor=center] (text) at (4,-2) {$\groundingof{\left(\impformula\Rightarrow\folexformula\right)}$};

\draw[] (2,-3) -- (2,-5) node[midway,left] {\tiny $\individualvariableof{0}$};
\node[anchor=center] (text) at (4,-4) {$\cdots$};
\draw[] (6,-3) -- (6,-5) node[midway,right] {\tiny $\individualvariableof{\individualorder-1}$};


\node[anchor=center] (text) at (10,-2) {${=}$};



%\draw (1,-1) rectangle (7,-3);
%\node[anchor=center] (text) at (4,-2) {$\rencodingof{\impformula}$};

\begin{scope}[shift={(12,-2)}]

\draw (1,3) rectangle (12,5);
\node[anchor=center] (text) at (6.5,4) {$\exformula$};

\draw[->] (2.5,1) -- (2.5,3) node[midway,right] {\tiny $\atomicformulaof{0}$};
\draw (1,-1) rectangle (4,1);
\node[anchor=center] (text) at (2.5,0) {$\rencodingof{\groundingof{\extformulaof{0}}}$};
\node[anchor=center] (text) at (2.5,-2) {$\cdots$};

\node[anchor=center] (text) at (6.5,0) {$\cdots$};

\draw[->] (10.5,1) -- (10.5,3) node[midway,right] {\tiny $\atomicformulaof{\atomorder-1}$};
\draw (8.75,-1) rectangle (12.25,1);
\node[anchor=center] (text) at (10.5,0) {$\rencodingof{\groundingof{\extformulaof{\atomorder\shortminus1}}}$};
\node[anchor=center] (text) at (10.5,-2) {$\cdots$};

\draw[<-] (13,-3) -- (3.5,-3) ;
\draw[<-] (13,-5) -- (1.5,-5) ;

\draw[fill] (11.5,-3) circle (0.25cm);
\draw[->] (11.5,-3) -- (11.5,-1);

\draw[fill] (9.5,-5) circle (0.25cm);
\draw[->] (9.5,-5) -- (9.5,-1);

\draw[fill] (3.5,-3) circle (0.25cm);
\draw[->] (3.5,-3) -- (3.5,-1);

\draw[fill] (1.5,-5) circle (0.25cm);
\draw[->] (1.5,-5) -- (1.5,-1);


\draw[fill] (7.5,-3) circle (0.25cm);
\draw[<-] (7.5,-3) -- (7.5,-7) node[right] {\tiny $\individualvariableof{\individualorder-1}$} ;

\node[anchor=center] (text) at (6.5,-6) {$\cdots$};

\draw[fill] (5.5,-5) circle (0.25cm);
\draw[<-] (5.5,-5) -- (5.5,-7) node[left] {\tiny $\individualvariableof{0}$} ;


\draw (13,-2) rectangle (15,-6);
\node[anchor=center] (text) at (14,-4) {$\rencodingof{\impformula}$};
\node[anchor=center] (text) at (12,-3.75) {$\vdots$};
\draw[->] (15,-4) -- (16,-4);
\draw[] (17,-4) -- (16,-4);
\draw (17,-3) rectangle (19,-5);
\node[anchor=center] (text) at (18,-4) {$\tbasis$};

\end{scope}




\node[anchor=center] (text) at (30,-2) {${+}$};





\begin{scope}[shift={(22,1)}]

\draw (13,0) rectangle (15,2);
\node[anchor=center] (text) at (14,1) {$\fbasis$};

\draw[] (14,-1) --(14,0);
\draw[->] (14,-2) --(14,-1);

\draw (12,-2) rectangle (16,-4);
\node[anchor=center] (text) at (14,-3) {$\rencodingof{\impformula}$};

\draw[<-] (12.5,-4) -- (12.5,-6) node[left] {\tiny $\individualvariableof{0}$} ;
\node[anchor=center] (text) at (14,-5) {$\cdots$};
\draw[<-] (15.5,-4) -- (15.5,-6) node[right] {\tiny $\individualvariableof{\individualorder\shortminus1}$} ;


		
\end{scope}

\end{tikzpicture}
\end{center}
where the second summand depends only on the query $\impformula$ and therefore does not appear in the likelihood.


%\begin{example}[Trivial importance formula]
%	When the importance formula is always satisfied, any tuple of objects contributes to the likelihood. 
%	This original approach to \MarkovLogicNetworks{} \cite{richardson_markov_2006} however leads to many datapoints which are also dependent on each other.
%\end{example}


\subsect{Reduction to \propositionalLogic{}}

We now make additional assumptions to decompose the partition function of an \HybridFOLNetwork{} as a product of \HybridLogicNetwork{} partition functions.

\begin{assumption}
    \label{ass:independentTuples}
    Given an importance formula $\impformula$ and a tensor $\fixedimpformula$ we consider the base measure
    \begin{align*}
        \fixedimpbm
        = \begin{cases}
              1 & \ifspace \groundingofat{\impformula}{\indvariableof{\impformula}} = \fixedimpformulawith \\
              0 & \text{else}
        \end{cases}
    \end{align*}
    of worlds with grounding of $\impformula$ by $\fixedimpformula$.
    Let further enumerate by $\datindexin$ the support of $\fixedimpformula$, that is $\sampleind$ are the indices of $\shortindvariables$ with $\fixedimpformulaat{\shortindvariables=\sampleind}=1$.
    We assume that with respect to $\fixedimpformula$ the variables
    \begin{align}
        \left(\groundingofat{\extformulaof{\atomenumerator}}{\shortindvariables=\sampleind}\right)
    \end{align}
    are for $\atomenumeratorin$ and $\datindexin$ independent and uniformly distributed.
\end{assumption}

When the above assumption holds, we now show that the probability of a \firstOrderLogic{} world with respect to a \HybridFOLNetwork{} coincides with the likelihood of a dataset in a propositional \HybridLogicNetwork{}.

\begin{theorem}
    \label{the:FOLworldToPLdataset}
    Let there be a set of formulas $\folformulaset$ such that \assref{ass:propositionalHeads} and \assref{ass:independentTuples} hold with an importance formula $\impformula$ and a tensor $\fixedimpformula$.
    Then for any $\hybridparam$ and any world $\catindexof{\folworldsymbol}$ with $\groundingof{\impformula}=\fixedimpformula$ we have
    \begin{align*}
        \frac{1}{\datanum} \lnof{\condprobwrtof{\folhlnparameters}{\indexedworldvariables}{\groundingof{\impformula}=\fixedimpformula}}
        = \centropyof{\empdistribution}{\probof{\hlnparameters}} -
        \frac{\lnof{\contraction{\fixedimpbm}}}{\datanum}
    \end{align*}
    where $\formulaset$ is the set of propositional equivalents to $\folformulaset$ (see \assref{ass:propositionalHeads}) and $\datamap$ the data map with evaluation at $\datindexin$ by the enumerated non-vanishing coordinates of $\fixedimpformulawith$
    \begin{align*}
        \datapoint
        = \big(\groundingofat{\extformulaof{0}}{\shortindvariables=\sampleind},\ldots,\groundingofat{\extformulaof{\atomorder-1}}{\shortindvariables=\sampleind}\big) \, .
    \end{align*}
\end{theorem}

To show the theorem, we show first in the following lemma the factorization of the partition function of the \HybridFOLNetwork{}.

% NEW
\begin{lemma}
    \label{lem:FOLpartitionfunctionfactorization}
    Under the assumptions of \theref{the:FOLworldToPLdataset}, we have
    \begin{align*}
        &\contraction{
            \bencodingofat{\restfolformulaset}{\headvariables,\worldvariables},\hypercoreofat{\hybridparam}{\headvariables},\fixedimpbm
        } \\
        &\quad=\contraction{\fixedimpbm} \cdot
        \left(  \prod_{\shortindindices\wcols\fixedimpformulaat{\indexedshortindvariables}=0} \hypercoreofat{\selindex,\hybridparam}{\headvariable=1}\right) \cdot \\
        &\quad\quad \left(\frac{1}{2^{\atomorder}}
        \cdot \contraction{\hlnstatccwith,\hypercoreofat{\hybridparam}{\headvariables}}
        \right)^{\datanum} \, .
    \end{align*}
\end{lemma}
\begin{proof}
    Using the assumption on the structure of the formulas we have
    \begin{align*}
        \groundingof{\countquantifier\enumfolformula}
        = \groundingof{\countquantifier\lnot\impformula} + \groundingof{\countquantifier(\impformula\land\headfolformulaof{\selindex})}
    \end{align*}
    and for any $\worldindices$
    \begin{align*}
        &\contractionof{\bencodingofat{\countquantifier\enumfolformula}{\headvariableof{\selindex}},\hypercoreofat{\selindex,\hybridparam}{\headvariable},\fixedimpbm}{\indexedworldvariables} \\
        &\quad = \left(\prod_{\shortindindices\wcols\fixedimpformulaat{\indexedshortindvariables}=0} \hypercoreofat{\selindex,\hybridparam}{\headvariable=1}\right) \cdot
        \contractionof{\bencodingofat{\countquantifier(\impformula\land\headfolformulaof{\selindex})}{\headvariableof{\selindex}},\hypercoreofat{\selindex,\hybridparam}{\headvariableof{\selindex}},\fixedimpbm}{\indexedworldvariables} \\
    \end{align*}
    Here by $\hypercoreofat{\selindex,\hybridparam}{\headvariable}$ we denote the two-dimensional realization of the activation core.
    We use that the grounding tensor of $\impformula$ is constant among the by $\fixedimpbm$ supported worlds and thus
    \begin{align*}
        &\contractionof{\bencodingofat{\countquantifier(\impformula\land\headfolformulaof{\selindex})}{\headvariableof{\selindex}},\hypercoreofat{\selindex,\hybridparam}{\headvariableof{\selindex}},\fixedimpbm}{\worldvariables} \\
        % Using that impformula is constant
        &\quad =
        \contractionof{
            \bigcup_{\datindexin}\left\{\bencodingofat{\groundingofwrt{\headfolformulaof{\selindex}}{\sampleind}}{\headvariableof{\selindex,\datindex}},\hypercoreofat{\selindex,\hybridparam}{\headvariableof{\selindex,\datindex}}\right\}
            \cup\{\fixedimpbm\}
        }{\worldvariables} \, .
    \end{align*}
    For each $\datindexin$ we have by \assref{ass:propositionalHeads}
    \begin{align*}
        % Using that headformula decomposes to propositional
        &\contractionof{
            \bencodingofat{\groundingofwrt{\headfolformulaof{\selindex}}{\sampleind}}{\headvariableof{\selindex,\datindex}},\hypercoreofat{\selindex,\hybridparam}{\headvariableof{\selindex,\datindex}}
        }{\worldvariables} \\
        & \quad =
        \contractionof{
            \{\bencodingofat{\groundingofwrt{\extformulaof{\atomenumerator}}{\sampleind}}{\headvariableof{\atomenumerator,\datindex}} \wcols \atomenumeratorin \}
            \cup \{\bencodingofat{\enumformula}{\formulavar,\headvariableof{\atomenumerator,\datindex}},\hypercoreofat{\selindex,\hybridparam}{\formulavar}\}
        }{\worldvariables} \, .
    \end{align*}
    From \assref{ass:independentTuples} we know
    \begin{align*}
        \contractionof{
            \{\bencodingofat{\groundingofwrt{\extformulaof{\atomenumerator}}{\sampleind}}{\headvariableof{\atomenumerator,\datindex}}
            \wcols \atomenumeratorin \ncond \datindexin \} \cup \{\fixedimpbm\}
        }{\headvariableof{[\atomorder]\times[\datanum]}}
        = \frac{\contraction{\fixedimpbm}}{2^{\atomorder\cdot\datanum}} \cdot \onesat{\headvariableof{[\atomorder]\times[\datanum]}} \, .
    \end{align*}
    Combining the above we get
    \begin{align*}
        &\contraction{
            \bencodingofat{\restfolformulaset}{\headvariables,\worldvariables},\hypercoreofat{\hybridparam}{\headvariables},\fixedimpbm
        } \\
        &\quad=
        \frac{\contraction{\fixedimpbm}}{2^{\atomorder\cdot\datanum}} \cdot \\
        &\quad\quad \left(  \prod_{\shortindindices\wcols\fixedimpformulaat{\indexedshortindvariables}=0} \hypercoreofat{\selindex,\hybridparam}{\headvariable=1}\right)
        \prod_{\datindexin}\contraction{
            \bencodingofat{\hlnstat}{\headvariables,\headvariableof{[\atomorder]\times\{\datindex\}}},\hypercoreofat{\hybridparam}{\headvariables}
        } \\
        &\quad=\contraction{\fixedimpbm} \cdot
        \left(  \prod_{\shortindindices\wcols\fixedimpformulaat{\indexedshortindvariables}=0} \hypercoreofat{\selindex,\hybridparam}{\headvariable=1}\right) \\
        &\quad\quad \left(\frac{1}{2^{\atomorder}}
        \cdot \contraction{\hlnstatccwith,\hypercoreofat{\hybridparam}{\headvariables}}
        \right)^{\datanum} \, . \qedhere
    \end{align*}
\end{proof}

% For
We notice, that for the event $\groundingof{\impformula}=\fixedimpformula$ to be of non-vanishing probability, we need to have $\headindexof{\hardlegset}=\ones_\hardlegset$ or $\fixedimpformulawith=\onesat{\indvariableof{\impformula}}$.
With this lemma, we are now show \theref{the:FOLworldToPLdataset}.

\begin{proof}[Proof of \theref{the:FOLworldToPLdataset}]
    We have
    \begin{align*}
        \lnof{\condprobwrtof{\folhlnparameters}{\indexedworldvariables}{\groundingof{\impformula}=\fixedimpformula}}
        &= \lnof{\contractionof{
            \bencodingofat{\restfolformulaset}{\headvariables,\worldvariables},\hypercoreofat{\hybridparam}{\headvariables},\fixedimpbm
        }{\indexedworldvariables}} \\
        &\quad - \lnof{\contraction{
            \bencodingofat{\restfolformulaset}{\headvariables,\worldvariables},\hypercoreofat{\hybridparam}{\headvariables},\fixedimpbm
        }}
    \end{align*}
    While the second term is decomposed by \lemref{lem:FOLpartitionfunctionfactorization} we now derive a decomposition of the first term.
    By \assref{ass:propositionalHeads} and $\groundingof{\impformula}=\fixedimpformula$ we have
    \begin{align*}
        &\contractionof{
            \bencodingofat{\restfolformulaset}{\headvariables,\worldvariables},\hypercoreofat{\hybridparam}{\headvariables},\fixedimpbm
        }{\indexedworldvariables} \\
        &\quad =
        \left(\prod_{\shortindindices\wcols\fixedimpformulaat{\indexedshortindvariables}=0} \hypercoreofat{\selindex,\hybridparam}{\headvariable=1}\right)
        \cdot \\
        &\quad\quad \prod_{\datindexin}
        \contraction{
            \{\bencodingofat{\groundingofwrt{\extformulaof{\atomenumerator}}{\sampleind}}{\headvariableof{\atomenumerator},\indexedworldvariables} \wcols \atomenumeratorin \}
            \cup \{\hlnstatccwith,\hypercoreof{\hybridparam}{\headvariables}\}
        } \, .
    \end{align*}
    With \lemref{lem:FOLpartitionfunctionfactorization} we then have
    \begin{align*}
        &\frac{1}{\datanum}\lnof{\condprobwrtof{\folhlnparameters}{\indexedworldvariables}{\groundingof{\impformula}=\fixedimpformula}} \\
        &\quad= \frac{1}{\datanum}\sum_{\datindexin} \lnof{\contraction{
            \{\bencodingofat{\groundingofwrt{\extformulaof{\atomenumerator}}{\sampleind}}{\headvariableof{\atomenumerator},\indexedworldvariables} \wcols \atomenumeratorin \}
            \cup \{\hlnstatccwith,\hypercoreof{\hybridparam}{\headvariables}\}
        }} \\
        &\quad\quad - \lnof{\contraction{
            \{\bencodingofat{\groundingofwrt{\extformulaof{\atomenumerator}}{\sampleind}}{\headvariableof{\atomenumerator},\worldvariables} \wcols \atomenumeratorin \}
            \cup \{\hlnstatccwith,\hypercoreof{\hybridparam}{\headvariables}\}
        }} - \frac{\lnof{\contraction{\fixedimpbm}}}{\datanum} \\
        &\quad = \centropyof{\empdistribution}{\probof{\hlnparameters}} - \frac{\lnof{\contraction{\fixedimpbm}}}{\datanum} \, . \qedhere
    \end{align*}
\end{proof}

% Independent data investigation
Let us now investigate, in which cases the \assref{ass:independentTuples} of independent data can be matched.

\begin{example}
    If the $\impformula$ and $\extformulas$ are predicates applied on variables, and the index tuples $\sampleind$ are pairwise different, then \assref{ass:independentTuples} is met.
    This is the case, since the values $\groundingof{\extformulaofat{\atomenumerator}{\shortindvariables=\sampleind}}$ are determined by different variables in $\worldvariables$.
\end{example}



There are situations, where \assref{ass:independentTuples} is violated.
\begin{itemize}
    \item extraction formula being a) conjunctions of predicates: Probability that they are satisfied decreases
    b) disjunctions of predicates: Probability that they are satisfied increases
    \item extraction formula coinciding with importance formula: Always satisfied, in this case still boolean
    \item extraction formulas contradicting each other, more general not independent from each other
\end{itemize}

%Let us notice, that non-boolean base measures could be treated in a same manner, but several developments in this work, such as cross-entropy decompositions in \charef{cha:probReasoning} would receive further terms.



\begin{remark}[Approximation by Independent Samples]
    As argued above, we do not have independent samples in general.
    As a consequence, we cannot apply \lemref{lem:FOLpartitionfunctionfactorization} to decompose the partition function term of the log-probability into factors to each solution map of $\impformula$.
    In this case, it might be still benefitial to use the reduction to the likelihood of a HLN, but needs to understand it as a approximation to the true world probability.

    %
    If the expectations of each sample with respect to the marginalized distributions coincide, the average of empirical distribution also coincides with these (by linearity).
    When the creation of samples has sufficient mixing properties, the empirical distribution converges to this expectation in the asymptotic case of large numbers of samples.

\end{remark}



\subsect{Sample extraction from \firstOrderLogic{} worlds}

We have observed that in certain situations the log-likelihood of a \firstOrderLogic{} world with respect to a \HybridFOLNetwork{} coincides with the likelihood of a data set in a propositional \HybridLogicNetwork{}.
Let us now investigate the extraction process of these data set.
%The decomposition of the likelihood suggests the following approach to generate samples from groundings:
%%We propose the following approach to generate datacores from groundings:
%\begin{itemize}
%    \item Define an importance formula $\impformula$, which we decompose in the basis CP decomposition and interpret each slice as the one-hot encoding of the datapoint.
%    \item Define for $\atomenumeratorin$ extraction formulas $\extformulaof{\atomenumerator}$ generating the atoms $\catvariableof{\atomenumerator}$.
%    %Predicates along with assignment of variables / constants to its positions.
%    \item Contract the groundings of each formula $\extformulaof{\atomenumerator}$ with the grounding of $\impformula$ to build a data core.
%\end{itemize}
%\subsubsect{Representation by Tensor Networks}
We model the extraction process as a relation between a tuple of individuals and the extracted world in the factored system of atoms $\catvariableof{\atomenumerator}$.

\begin{definition}
    \label{def:extractionRelation}
    Given a \firstOrderLogic{} world $\worldindices$, an importance formula $\impformula$ and extraction formulas $\extformulaof{\catenumerator}$ for $\catenumeratorin$, we define the extraction relation
    \begin{align*}
        \extractionrelation \subset \left(\symindstates\right) \otimes \left(\atomstates\right)
    \end{align*}
    by
    \begin{align*}
        \extractionrelation
        = \{ (\shortindindices, \shortcatindices)
        \wcols  \groundingofat{\impformula}{\indexedshortindvariables} = 1 \ncond \uniquantwrtof{\catenumeratorin}{\catindexof{\atomenumerator} = \extformulaofat{\atomenumerator}{\indexedshortindvariables}} \} \, .
    \end{align*}
\end{definition}

The encoding of an extraction relation is the tensor
\begin{align*}
    \bencodingofat{\extractionrelation}{\shortindvariables,\shortcatvariables} \subset \left(\indspace\right) \otimes \left(\atomspace\right) \,
\end{align*}
and drawn in a contraction diagram by
\begin{center}
    \begin{tikzpicture}[scale=0.35, yscale=1, thick] % , baseline = -3.5pt


    \draw[] (2,-1) -- (2,1) node[midway,left] {\tiny $\catvariableof{0}$};
    \node[anchor=center] (text) at (4,0) {$\cdots$};
    \draw[] (6,-1) -- (6,1) node[midway,right] {\tiny $\catvariableof{\atomorder-1}$};

    \draw (1,-1) rectangle (7,-3);
    \node[anchor=center] (text) at (4,-2) {$\bencodingof{\extractionrelation}$};

    \draw[] (2,-3) -- (2,-5) node[midway,left] {\tiny $\individualvariableof{0}$};
    \node[anchor=center] (text) at (4,-4) {$\cdots$};
    \draw[] (6,-3) -- (6,-5) node[midway,right] {\tiny $\individualvariableof{\individualorder-1}$};


    \node[anchor=center] (text) at (10,-2) {${=}$};


    \begin{scope}
        [shift={(12,0)}]

        \draw[->-] (2.5,1) -- (2.5,3) node[midway,right] {\tiny $\catvariableof{0}$};
        \draw (1,-1) rectangle (4,1);
        \node[anchor=center] (text) at (2.5,0) {$\bencodingof{\groundingof{\extformulaof{0}}}$};
        \node[anchor=center] (text) at (2.5,-2) {$\cdots$};

        \node[anchor=center] (text) at (6.5,0) {$\cdots$};

        \draw[->-] (10.5,1) -- (10.5,3) node[midway,right] {\tiny $\catvariableof{\atomorder-1}$};
        \draw (8.75,-1) rectangle (12.25,1);
        \node[anchor=center] (text) at (10.5,0) {$\bencodingof{\groundingof{\extformulaof{\atomorder\shortminus1}}}$};
        \node[anchor=center] (text) at (10.5,-2) {$\cdots$};

        \draw[-<-] (13,-3) -- (3.5,-3) ;
        \draw[-<-] (13,-5) -- (1.5,-5) ;

        \drawvariabledot{11.5}{-3}
        \draw[->-] (11.5,-3) -- (11.5,-1);

        \drawvariabledot{9.5}{-5}
        \draw[->-] (9.5,-5) -- (9.5,-1);

        \drawvariabledot{3.5}{-3}
        \draw[->-] (3.5,-3) -- (3.5,-1);

        \drawvariabledot{1.5}{-5}
        \draw[->-] (1.5,-5) -- (1.5,-1);

        \drawvariabledot{7.5}{-3}
        \draw[-<-] (7.5,-3) -- (7.5,-7) node[right] {\tiny $\individualvariableof{\individualorder-1}$} ;

        \node[anchor=center] (text) at (6.5,-6) {$\cdots$};

        \drawvariabledot{5.5}{-5}
        \draw[-<-] (5.5,-5) -- (5.5,-7) node[left] {\tiny $\individualvariableof{0}$} ;


        \draw (13,-2) rectangle (17,-6);
        \node[anchor=center] (text) at (15,-4) {$\bencodingof{\groundingof{\impformula}}$};
        \node[anchor=center] (text) at (12,-3.75) {$\vdots$};
        \draw[->-] (17,-4) -- (18,-4);
        \drawvariabledot{18}{-4}
        \draw[] (18,-4) -- (19,-4);
        \draw (19,-3) rectangle (21,-5);
        \node[anchor=center] (text) at (20,-4) {$\tbasis$};

    \end{scope}


\end{tikzpicture}
\end{center}
Here the contraction of $\bencodingof{\impformula}$ with the truth vector $\tbasis$ represents the matching condition posed by $\impformula$ when extracting pairs of individuals.

%% Empirical Distribution
The empirical distribution of the extracted data is then the normalized contraction leaving only the legs to the extracted atomic formulas open, that is
\begin{align*}
    \empdistribution
    = \frac{
        \contractionof{\bencodingof{\extractionrelation}}{\shortcatvariables}
    }{
        \contraction{\bencodingof{\extractionrelation}}
    }  \, .
\end{align*}
Here the number of extracted data is the denominator
\begin{align*}
    \datanum
    = \contraction{\bencodingof{\extractionrelation}}
    = \contraction{\bencodingofat{\impformula}{\headvariableof{\impformula},\shortindvariables},\tbasisat{\headvariableof{\impformula}}}\, .
\end{align*}

We depict this by
\begin{center}
    
\begin{tikzpicture}[scale=0.35, yscale=1, thick] % , baseline = -3.5pt


    \draw[->] (2,-1) -- (2,1) node[midway,left] {\tiny $\catvariableof{0}$};
    \node[anchor=center] (text) at (4,0) {$\cdots$};
    \draw[->] (6,-1) -- (6,1) node[midway,right] {\tiny $\catvariableof{\atomorder-1}$};

    \draw (1,-1) rectangle (7,-3);
    \node[anchor=center] (text) at (4,-2) {$\empdistribution$};
    \node[anchor=center] (text) at (-1,-2) {$\datanum \,\, \cdot $};

    \node[anchor=center] (text) at (10,-2) {${=}$};

    \begin{scope}
        [shift={(12,0)}]

        \draw[->] (2.5,1) -- (2.5,3) node[midway,right] {\tiny $\catvariableof{0}$};
        \draw (1,-1) rectangle (4,1);
        \node[anchor=center] (text) at (2.5,0) {$\rencodingof{\groundingof{\extformulaof{0}}}$};
        \node[anchor=center] (text) at (2.5,-2) {$\cdots$};

        \node[anchor=center] (text) at (6.5,0) {$\cdots$};

        \draw[->] (10.5,1) -- (10.5,3) node[midway,right] {\tiny $\catvariableof{\atomorder-1}$};
        \draw (8.75,-1) rectangle (12.25,1);
        \node[anchor=center] (text) at (10.5,0) {$\rencodingof{\groundingof{\extformulaof{\atomorder\shortminus1}}}$};
        \node[anchor=center] (text) at (10.5,-2) {$\cdots$};

        \draw[<-] (13,-3) -- (3.5,-3) ;
        \draw[<-] (13,-5) -- (1.5,-5) ;

        \drawvariabledot{11.5}{-3}
        \draw[->] (11.5,-3) -- (11.5,-1);

        \drawvariabledot{9.5}{-5}
        \draw[->] (9.5,-5) -- (9.5,-1);

        \drawvariabledot{3.5}{-3}
        \draw[->] (3.5,-3) -- (3.5,-1);

        \drawvariabledot{1.5}{-5}
        \draw[->] (1.5,-5) -- (1.5,-1);

        \draw (13,-2) rectangle (17,-6);
        \node[anchor=center] (text) at (15,-4) {$\rencodingof{\groundingof{\impformula}}$};
        \node[anchor=center] (text) at (12,-3.75) {$\vdots$};
        \draw[->] (17,-4) -- (18,-4);
        \drawvariabledot{18}{-4}
        \draw[] (18,-4) -- (19,-4);
        \draw (19,-3) rectangle (21,-5);
        \node[anchor=center] (text) at (20,-4) {$\tbasis$};

    \end{scope}


\end{tikzpicture}
\end{center}

%\subsubsect{Decomposition of extracted data}

To connect with the empirical distribution introduced in \secref{sec:empDistribution} we now show how the empirical distribution extracted from the interpretations of the formulas $\impformula,\extformulas$ on a \firstOrderLogic{} world $\worldindices$ can be represented by tensor networks.

First of all, we decompose the importance formula into a basis $\cpformat$ format (see \charef{cha:sparseRepresentation}), that is a decomposition
\begin{align*}
    \groundingofat{\impformula}{\shortindvariables}
    = \contractionof{
        \{\legcoreofat{\indenumerator}{\indvariableof{\indenumerator},\datvariable} \, : \, \indenumeratorin \}
    }{\shortindvariables}
\end{align*}
such that all $\legcoreofat{\indenumerator}{\indvariableof{\indenumerator},\datvariable}$ are directed and boolean tensors.
Here an auxiliary variables $\datvariable$ taking values in $[\datanum]$ is introduced, which we call the data variable, which enumerates the non-vanishing coordinates of $\groundingof{\impformula}$.
With this decomposition, we can understand the decomposition of $\groundingofat{\impformula}{\shortindvariables}$ as a basis encoding of an term selection map $\secdatamap$ with coordinate maps defined such that
\begin{align*}
    \bencodingofat{\secdatamap_{\indenumerator}}{\indvariableof{\indenumerator},\datvariable}
    = \legcoreofat{\indenumerator}{\indvariableof{\indenumerator},\datvariable} \, .
\end{align*}
We depict this decomposition by:
\begin{center}
    \begin{tikzpicture}[scale=0.35, thick] % , baseline = -3.5pt

    \begin{scope}
        [shift={(0,2)}]
        \draw[] (0,1)--(0,-1) node[midway,left] {\tiny $\indvariableof{0}$};
        \draw[] (1.5,1)--(1.5,-1) node[midway,left] {\tiny $\indvariableof{1}$};
        \node[anchor=center] (text) at (3,0) {$\cdots$};
        \draw[] (4,1)--(4,-1) node[midway,right] {\tiny $\indvariableof{\indorder\shortminus1}$};
    \end{scope}


    \draw (-1,1) rectangle (5,-1);
    \node[anchor=center] (text) at (2,0) {\small ${\groundingof{\impformula}}$};


    \node[anchor=center] (text) at (7,0) {${=}$};


    \begin{scope}
        [shift={(10,2)}]



        \coordinate (conposseldec) at (4.5,-5.5);
        \drawvariabledot{4.5}{-5.5}

        \draw (conposseldec) -- (4.5,-7.5) node[midway, right] {\tiny ${\datvariable}$}; % Unclear, whether this is the best notation!
        \draw (3.5,-7.5) rectangle (5.5, -9.5);
        \node[anchor=center] (text) at (4.5,-8.5) {\small $\ones$};

        \draw[-<-] (0,1) -- (0,-1) node[midway,left] {\tiny $\indvariableof{0}$};
        \draw (-1,-1) rectangle (1, -3);
        \node[anchor=center] (text) at (0,-2) {\small $\secdatacoreof{0}$};
        \draw[-<-] (0,-3) to[bend right=20] (conposseldec);


        \draw[-<-] (3,1) -- (3,-1) node[midway,left] {\tiny $\indvariableof{1}$};
        \draw (2,-1) rectangle (4, -3);
        \node[anchor=center] (text) at (3,-2) {\small $\secdatacoreof{1}$};
        \draw[-<-] (3,-3) to[bend right=20]  (conposseldec);

        \node[anchor=center] (text) at (6,-2) {$\cdots$};

        \draw[-<-] (9,1) -- (9,-1) node[midway,left] {\tiny $\indvariableof{\indorder-1}$};
        \draw (7.75,-1) rectangle (10.25, -3);
        \node[anchor=center] (text) at (9,-2) {\small $\secdatacoreof{\indorder-1}$};
        \draw[-<-] (9,-3) to[bend left=20]  (conposseldec);

    \end{scope}


\end{tikzpicture}
\end{center}

Based on these construction, we now provide a tensor network decomposition of the extracted empirical distribution.

\begin{theorem}
    \label{the:extractionrelationDecomposition}
    Given a \firstOrderLogic{} world $\worldindices$, an importance formula $\impformula$ and extraction formulas $\extformulaof{\catenumerator}$ for $\catenumeratorin$, we have
    \begin{align*}
        \bencodingofat{\extractionrelation}{\shortindvariables,\shortcatvariables} =
        \contractionof{
            \{\bencodingofat{\groundingof{\extformulaof{\atomenumerator}}}{\catvariableof{\catenumerator},\shortindvariables} \, : \, \catenumeratorin\}
            \cup \{\bencodingofat{\secdatamap_{\indenumerator}}{\indvariableof{\indenumerator},\datvariable} \, : \, \indenumeratorin\}
        }{\shortindvariables,\shortcatvariables}
    \end{align*}
    and thus
    \begin{align*}
        \empdistributionat{\shortcatvariables} =
        \frac{1}{\datanum}  \contractionof{
            \{\bencodingofat{\groundingof{\extformulaof{\atomenumerator}}}{\catvariableof{\catenumerator},\shortindvariables} \, : \, \catenumeratorin\}
            \cup \{\bencodingofat{\secdatamap_{\indenumerator}}{\indvariableof{\indenumerator},\datvariable} \, : \, \indenumeratorin\}
        }{\shortcatvariables} \, .
    \end{align*}
\end{theorem}
\begin{proof}
    To show the first claim, let us choose arbitrary state tuples $\shortindindices$ and $\shortcatindices$.
    We then have
    \begin{align*}
        &\contractionof{
            \{\bencodingofat{\groundingof{\extformulaof{\atomenumerator}}}{\catvariableof{\catenumerator},\shortindvariables} \, : \, \catenumeratorin\}
            \cup \{\bencodingofat{\secdatamap_{\indenumerator}}{\indvariableof{\indenumerator},\datvariable} \, : \, \indenumeratorin\}
        }{\indexedshortindvariables,\indexedshortcatvariables} \\
        & \quad  =  \contraction{
            \{\bencodingofat{\groundingof{\extformulaof{\atomenumerator}}}{\indexedcatvariableof{\catenumerator},\indexedshortindvariables} \, : \, \catenumeratorin\}
            \cup \{\bencodingofat{\secdatamap_{\indenumerator}}{\indexedindvariableof{\indenumerator},\datvariable} \, : \, \indenumeratorin\}
        } \, .
    \end{align*}
    This contraction evaluates to $1$, if and only if for all $\catenumeratorin$ we have $\bencodingofat{\groundingof{\extformulaof{\atomenumerator}}}{\catvariableof{\catenumerator},\shortindvariables}=1$ and
    \begin{align*}
        \contraction{\{\bencodingofat{\secdatamap_{\indenumerator}}{\indexedindvariableof{\indenumerator},\datvariable} \, : \, \indenumeratorin\}}  = 1 \, .
    \end{align*}
    The first condition is equal to $\catindexof{\atomenumerator} = \extformulaofat{\atomenumerator}{\indexedshortindvariables}$ for all $\catenumeratorin$ and the second to
    \begin{align*}
        \groundingofat{\impformula}{\indexedshortindvariables} = 1 \, .
    \end{align*}
    Comparing with the definition of the extraction relation (see \defref{def:extractionRelation}), we notice that these conditions are equal to $(\shortindindices,\shortcatindices)\in\extractionrelation$ and therefore to
    \begin{align*}
        \bencodingofat{\extractionrelation}{\indexedshortindvariables,\indexedshortcatvariables} \, .
    \end{align*}
    The first claim follows, since $\bencodingof{\extractionrelation}$ is boolean, as is the contraction of the cores $\bencodingof{\groundingof{\extformulaof{\atomenumerator}}}$ with the cores $\bencodingof{\secdatamap_{\indenumerator}}$, which leaves the outgoing variables $\shortcatvariables$ open.
    The second claim follows from the first using that $\empdistributionat{\shortcatvariables}=\frac{1}{\datanum}\contractionof{\bencodingof{\extractionrelation}}{\shortcatvariables}$.
\end{proof}

To connect with the representation of empirical distributions based on data cores (see \secref{sec:empDistribution}), we now form data cores by contractions with the grounding of extraction formulas with the cores $\bencodingof{\secdatamap_{\indenumerator}}$ (see \figref{fig:datacoreGeneration}),
\begin{align*}
    \datacoreofat{\atomenumerator}{\catvariableof{\catenumerator},\datvariable}
    = \contractionof{
        \{\bencodingofat{\groundingof{\extformulaof{\atomenumerator}}}{\catvariableof{\catenumerator},\shortindvariables}\}
        \cup \{ \legcoreofat{\indenumerator}{\indvariableof{\indenumerator},\datvariable} \, : \, \indenumeratorin\}
    }{\catvariableof{\atomenumerator},\datvariable} \, .
\end{align*}

% Empirical distribution
The empirical distribution is then a tensor network of these tensors, as we show next.

\begin{theorem}
    \label{the:extractionDataCores}
    We have
    \begin{align*}
        \contractionof{\bencodingof{\extractionrelation}}{\shortcatvariables}
        = \contractionof{\{\datacoreofat{\atomenumerator}{\datvariable,\catvariableof{\atomenumerator}} \, : \, \atomenumeratorin\}}{\shortcatvariables}
    \end{align*}
    and thus
    \begin{align*}
        \empdistributionat{\shortcatvariables}
        = \frac{1}{\datanum} \contractionof{\{\datacoreofat{\atomenumerator}{\datvariable,\catvariableof{\atomenumerator}}  \, : \, \atomenumeratorin\}}{\shortcatvariables} \, .
    \end{align*}
\end{theorem}
\begin{proof}
    By \theref{the:extractionrelationDecomposition} we have
    \begin{align*}
        \bencodingofat{\extractionrelation}{\shortindvariables,\shortcatvariables} =
        \contractionof{
            \{\bencodingofat{\groundingof{\extformulaof{\atomenumerator}}}{\catvariableof{\catenumerator},\shortindvariables} \, : \, \catenumeratorin\}
            \cup \{\bencodingofat{\secdatamap_{\indenumerator}}{\indvariableof{\indenumerator},\datvariable} \, : \, \indenumeratorin\}
        }{\shortindvariables,\shortcatvariables} \, .
    \end{align*}
    Since $\bencodingofat{\secdatamap_{\indenumerator}}{\indvariableof{\indenumerator},\datvariable}$ are directed and boolean, they can be copied and separately contracted with each $\groundingof{\extformulaof{\atomenumerator}}$, without changing the contraction.
    We arrive at
    \begin{align*}
        &\bencodingofat{\extractionrelation}{\shortindvariables,\shortcatvariables} \\
        &\quad = \contractionof{
            \big\{\contractionof{
                \{\bencodingofat{\groundingof{\extformulaof{\atomenumerator}}}{\catvariableof{\catenumerator},\shortindvariables}\}
                \cup \{\bencodingofat{\secdatamap_{\indenumerator}}{\indvariableof{\indenumerator},\datvariable} \, : \, \indenumeratorin\}
            }{\catvariableof{\catenumerator},\datvariable} \, : \, \catenumeratorin \big\}
        }{\shortindvariables,\shortcatvariables} \\
        & \quad =  \contractionof{\{\datacoreofat{\atomenumerator}{\datvariable,\catvariableof{\atomenumerator}}  \, : \, \atomenumeratorin\}}{\shortcatvariables} \, ,
    \end{align*}
    which established the claim.
\end{proof}

% Efficient contraction: Do also basis decomposition of the extraction query and use efficient contraction!
%Towards efficient calculation of the data cores, we build a basis CP decomposition of $\groundingof{\impformula}$, where we further demand $\scalarcore=\ones$.
%This is a collection of basis leg cores $\legcoreof{\fixedimpformula,\indenumerator}$ such that
%\begin{align*}
%    \fixedimpformula[\shortindvariablelist]
%    = \contractionof{ \left\{ \legcoreofat{\fixedimpformula,\indenumerator}{\datvariable,\indvariableof{\indenumerator}} \, : \, \indenumeratorin \right\} }{\shortindvariablelist} \, .
%\end{align*}

% Data enumeration -> To representation
%We can further utilize any decomposition of $\impformula$ into a directed and binary CP Format to enumerate the datapoints by the slice index $\datindex$. % Approaches like SPARQL directly give us these by solution mappings.
%Understanding $\impformula$ as a query on the world being the database, such decomposition is given by the set of solution mappings.


\begin{figure}[t]
    \begin{center}
        \begin{tikzpicture}[scale=0.35, yscale=1, thick] % , baseline = -3.5pt


    \draw[->-] (4,-1) -- (4,1) node[midway, right] {\tiny $\catvariableof{\atomenumerator}$};
    \draw (3,-1) rectangle (5,-3);
    \node[anchor=center] (text) at (4,-2) {$\datacoreof{\atomenumerator}$};
    \draw[-<-] (4,-3) -- (4,-5) node[midway, right] {\tiny $\datvariable$};

    \node[anchor=center] (text) at (7,-2) {${=}$};

    \begin{scope}
        [shift={(10,0)}]

        \draw[->-] (3,1) -- (3,3) node[midway, right] {\tiny $\catvariableof{\atomenumerator}$};
        \draw (-1,1) rectangle (7,-1);
        \node[anchor=center] (text) at (3,0) {$\rencodingof{\groundingof{\extformulaof{\atomenumerator}}}$};

        \draw[->-] (0,-3) -- (0,-1) node[midway,left] {\tiny $\indvariableof{0}$};
        \draw[->-] (3,-3) -- (3,-1) node[midway,left] {\tiny $\indvariableof{1}$};
        \draw[->-] (6,-3) -- (6,-1) node[midway,left] {\tiny $\indvariableof{2}$};


    \end{scope}

    \begin{scope}
        [shift={(10,-2)}]

        \coordinate (conposseldec) at (4.5,-5.5);
        \drawvariabledot{4.5}{-5.5}
        \draw[-<-] (conposseldec) -- (4.5,-7.5) node[midway, right] {\tiny $\indexvariable$};

        \draw (-1,-1) rectangle (1, -3);
        \node[anchor=center] (text) at (0,-2) {\small $\rencodingof{\secdatamap_0}$};%{\small $\legcoreof{\fixedimpformula,0}$};
        \draw[-<-] (0,-3) to[bend right=20] (conposseldec);

        \draw (2,-1) rectangle (4, -3);
        \node[anchor=center] (text) at (3,-2) {\small $\rencodingof{\secdatamap_1}$};%{\small $\legcoreof{\fixedimpformula,1}$};
        \draw[-<-] (3,-3) to[bend right=20]  (conposseldec);

        \draw (5,-1) rectangle (7, -3);
        \node[anchor=center] (text) at (6,-2) {\small $\rencodingof{\secdatamap_2}$};%{\small $\legcoreof{\fixedimpformula,2}$};
        \draw[-<-] (6,-3) to[bend right=-20]  (conposseldec);

        \draw[<-] (9,1) -- (9,-1) node[midway,left] {\tiny $\indvariableof{3}$};
        \draw (8,-1) rectangle (10, -3);
        \node[anchor=center] (text) at (9,-2) {\small $\rencodingof{\secdatamap_3}$};%{\small $\legcoreof{\fixedimpformula,3}$};
        \draw[<-] (9,-3) to[bend right=-20]  (conposseldec);


        \node[anchor=center] (text) at (12,-2) {$\cdots$};

        \draw[<-] (15,1) -- (15,-1) node[midway,left] {\tiny $\indvariableof{\indorder-1}$};
        \draw (13.5,-1) rectangle (16.5, -3);
        \node[anchor=center] (text) at (15,-2) {\small $\rencodingof{\secdatamap_{\indorder-1}}$};%{\small $\legcoreof{\fixedimpformula,\variableorder-1}$};
        \draw[<-] (15,-3) to[bend left=20]  (conposseldec);


        \draw (8,1) rectangle (16, 3);
        \node[anchor=center] (text) at (12,2) {\small $\ones$};


    \end{scope}


    \node[anchor=center] (text) at (29,-2) {${=}$};


    \begin{scope}
        [shift={(32,0)}]

        \draw[->-] (3,1) -- (3,3) node[midway, right] {\tiny $\catvariableof{\atomenumerator}$};
        \draw (-1,1) rectangle (7,-1);
        \node[anchor=center] (text) at (3,0) {$\rencodingof{\groundingof{\extformulaof{\atomenumerator}}}$};

        \draw[->-] (0,-3) -- (0,-1) node[midway,left] {\tiny $\indvariableof{0}$};
        \draw[->-] (3,-3) -- (3,-1) node[midway,left] {\tiny $\indvariableof{1}$};
        \draw[->-] (6,-3) -- (6,-1) node[midway,left] {\tiny $\indvariableof{2}$};


    \end{scope}

    \begin{scope}
        [shift={(32,-2)}]


        \coordinate (conposseldec) at (3,-5.5);
        \drawvariabledot{3}{-5.5}
        \draw[<-] (conposseldec) -- (3,-7.5) node[midway, right] {\tiny $\datvariable$};

        \draw (-1,-1) rectangle (1, -3);
        \node[anchor=center] (text) at (0,-2){\small $\rencodingof{\secdatamap_0}$};%{\small $\legcoreof{\fixedimpformula,0}$};
        \draw[<-] (0,-3) to[bend right=20] (conposseldec);

        \draw (2,-1) rectangle (4, -3);
        \node[anchor=center] (text) at (3,-2) {\small $\rencodingof{\secdatamap_1}$};%{\small $\legcoreof{\fixedimpformula,1}$};
        \draw[<-] (3,-3) to[bend right=0]  (conposseldec);

        \draw (5,-1) rectangle (7, -3);
        \node[anchor=center] (text) at (6,-2) {\small $\rencodingof{\secdatamap_2}$};%{\small $\legcoreof{\fixedimpformula,2}$};
        \draw[<-] (6,-3) to[bend right=-20]  (conposseldec);


    \end{scope}


\end{tikzpicture}
    \end{center}
    \caption{Generation of a data core for the variable $\catvariableof{\catenumerator}$ given an extraction formula $\extformulaof{\catenumerator}$ and an importance formula, which grounding is decomposed into a basis CP format with leg vectors $\bencodingofat{\secdatamap_{\indenumerator}}{\indvariableof{\indenumerator},\datvariable}$.
    Term variables, which are appearing in the importance formula, but not in the extraction formula $\extformulaof{\catenumerator}$ can be treated trivally by contraction with the trivial tensor (here $\indvariableof{4},\ldots,\indvariableof{\indorder-1})$.
    }
    \label{fig:datacoreGeneration}
\end{figure}


% Comment: Exploitation of common structure
When many atom extraction formulas differ only by a constant, we can replace the constant by an auxiliary term variable.
The atoms are then the atomizations of this variable (see \secref{sec:categoricalTN}), treated as a categorical variable, with respect to the constant in the extraction query.
The advantages are that we can avoid the $\bencodingof{}$-formalism and directly model the categorical distributions.

This also enables a batchwise computation of multiple $\sparql$ queries, which differ only in one constant.


%\subsect{Design of the Formulas}
%
%Most intuitive when labeling individuals by classes.
%Extraction formulas $\extformulas$ can then be defined by subclasses of the member of a class and relations between objects of different classes. % Koller calls atomic formulas the template attributes
%We then choose $\formulaset$ as more involved formulas decomposed into connectives acting on these atoms.
%The importance formula $\impformula$ is then designed based on class memberships to ensure, that the arguments of the formulas are always of specific classes. % Koller specifies to each argument of the attributes a class
%
%% Approach
%We propose to
%\begin{itemize}
%    \item Execute an extraction query to get pairs of individuals (the pairDf).
%    \item Propositionalize the FOL Formulas independently on each tuple taking the individuals as a set of constant and filtering on the possible properties of each individuals.
%    (Can understand as adding knowledge that most of the relations do not hold)
%    \item Understand each such generated knowledge base as datapoint and average over them to get the empirical distribution to be fit.
%    \item Fit a MLN describing the statistical relations of unseen results of the extraction query, based on likelihood maximation.
%\end{itemize}




\sect{Generation of \firstOrderLogic{} worlds}

\red{
    So far we have discussed, how Probabilistic Relational Models for \firstOrderLogic{} Knowledge Bases such as Knowledge Graphs can be built by extracting data.
    Conversely, any binary tensor can be interpreted as a Knowledge Graph.
    To be more precise, we follow the intuition that the ones coordinates mark possible worlds compatible with the knowledge about a factored system.
    Each possible world can then be encoded in a subgraph of the Knowledge Graph representing the world.
%
    This amounts to an "inversion" of the data generation process described in the subsection above.
}

In the previous section we have described a way to extract an effective empirical distribution for the likelihood of a \firstOrderLogic{} world given a \HybridFOLNetwork{}.
We now want to investigate methods to reproduce an empirical distribution based on a constructed \firstOrderLogic{} world.

\begin{definition}[Reproduction of Empirical Distributions]
    Given an empirical distribution $\empdistribution\in\atomspace$, we say that a triple $(\worldindices,\impformula,\shortextformulas)$ of a \firstOrderLogic{} world $\worldindices$ an importance formula $\impformula$ and extraction formulas $\shortextformulas=\{\extformulaof{\atomenumerator}\,:\,\atomenumeratorin\}$ reproduces $\empdistribution$, when
    \begin{align*}
        \empdistribution
        = \normalizationof{\{\groundingofat{\impformula}{\shortindvariables}\}\cup
        \{\bencodingofat{\kggroundingof{\extformulaof{\atomenumerator}}}{\catvariableof{\catenumerator},\shortindvariables} \wcols \atomenumeratorin\}
        }{\shortcatvariables} \, .
    \end{align*}
\end{definition}

% If \datamap is not known
Note that for distribution $\probtensor$ to be reproducible, it needs to have rational coordinates. %, since each coordinate can be interpreted as the frequency of the respective world in the data $\datamap$.
If any only if all coordinates are rational, we find a $\datanum\in\nn$ such that $\imageof{\datanum\cdot\probtensor}\subset\nn$.
We can then interpret $\datanum$ as the number of samples, and construct a sample selector map by understanding each coordinate of $\datanum\cdot\probtensor$ as the number of appearances of the respective world in the samples.

We show different schemes and give examples on Knowledge Graphs, where we provide examples for importance and extraction formulas by $\sparql$ queries.


%\subsect{Example: Generation of Knowledge Graphs} % To generation of \firstOrderLogic{} worlds?
%
% Having a directed and binary CP decomposition of $\exformula$, each possible world is encoded by a slice.


% Formalization
%\begin{definition}[Reproduction of Empirical Distributions]
%    Given an empirical distribution $\empdistribution\in\bigotimes_{\atomenumeratorin}\rr^2$, we say that a tuple $(\kg,\impformula,\{\extformulas\})$ of a Knowledge Graph $\kg$ and queries $\impformula,\extformulaof{\atomenumerator}$ reproduces $\empdistribution$, when
%    \[\empdistribution = \normalizationof{\{\kggroundingof{\impformula}\}\cup\{\bencodingof{\kggroundingof{\extformulaof{\atomenumerator}}\, : \, \atomenumeratorin}\}}{\shortcatvariables} \, .  \]
%\end{definition}

%

%In a frequentist interpretation we instantiate each world according to the rate $\probtensor(\atomindices)$.
%This interpretation requires a rounding of the real probabilities by rational numbers.


\subsect{Samples by single objects}

%\subsect{Samples by single objects}

In the first reproduction scheme we construct datapoints by dedicated objects, which represent a sample, that is we choose a domain $\worlddomain=[\datdim]$.

\begin{theorem}
    \label{the:reproducingSingleObjects}
    Let there be an empirical distribution $\empdistribution$ to a sample selector map $\datamap$ (see \defref{def:dataMap}), we construct a world $\worldindices[\selvariable,\indvariable]$ with $\atomorder$ unary predicates by
    \begin{align*}
        \worldindices[{\selvariable,\indvariable}]
        = \sum_{\atomenumeratorin} \sum_{\datindexin \wcols \datamap_{\atomenumerator}(\datindex)=1} \onehotmapofat{\atomenumerator}{\selvariable} \otimes \onehotmapofat{\datindex}{\indvariable} \, .
    \end{align*}
    We further choose a trivial importance query, that is
    \begin{align*}
        \groundingofat{\impformula}{\indvariable} = \onesat{\indvariable} \, ,
    \end{align*}
    and extraction queries coinciding with the unary predicates, that is for $\atomenumeratorin$
    \begin{align*}
        \extformulaof{\atomenumerator} = \folpredicateof{\atomenumerator} \, .
    \end{align*}
    Then, the triple $(\worldindices,\impformula,\shortextformulas)$ reproduces $\empdistribution$.
%    reproduces with the trivial importance query and extraction queries coinciding with the predicates the dataset $\datamap$.
\end{theorem}
\begin{proof}
    By \theref{the:extractionDataCores} it is enough to show, that the data cores constructed from the data extraction process coincide with those of $\empdistribution$.
    We enumerate to this end the non-vanishing coordinates of $\groundingof{\impformula}$ by the data variable $\datvariable$ taking values $\datindexin$, as
    \begin{align*}
        \groundingofat{\impformula}{\indvariable=\datindex} = 1 \,
    \end{align*}
    and choose
    \begin{align*}
        \secdatamap = \identity \, .
    \end{align*}
    For arbitrary $\atomenumeratorin$ and $\datindexin$ we now have
    \begin{align*}
        \datacoreofat{\atomenumerator}{\catvariableof{\catenumerator},\indexeddatvariable}
        &= \contractionof{
            \bencodingofat{\groundingof{\extformulaof{\atomenumerator}}}{\catvariableof{\catenumerator},\indvariable},
            \legcoreofat{0}{\indvariable,\indexeddatvariable}
        }{\catvariableof{\atomenumerator},\datvariable} \\
        &= \contractionof{
            \bencodingofat{\groundingof{\extformulaof{\atomenumerator}}}{\catvariableof{\catenumerator},\indvariable},
            \onehotmapofat{\secdatamap(\datindex)}{\indvariable}
        }{\catvariableof{\atomenumerator},\indexeddatvariable} \\
        &= \onehotmapofat{\datamap_\atomenumerator(\datindex)}{\catvariableof{\catenumerator}} \, .
    \end{align*}
    This coincides with the slice of the data core of the CP representation of empirical distributions used in \theref{the:empCPRep}.
    Since the slice and the core was arbitrary, the tensor network representations in \theref{the:empCPRep} and \theref{the:extractionDataCores} are equal and thus the triple $(\worldindices,\impformula,\shortextformulas)$ reproduces $\empdistribution$.
\end{proof}


We now give by the next theorem an example of a Knowledge Graph with $\sparql$ queries reproducing and arbitrary empirical distribution.

\begin{theorem}
    \label{the:reproducingKGSingelObjects}
    Let $\empdistribution$ be an empirical distribution to the sample selector $\datamap$.
    We construct a Knowledge Graph of the resources $\worlddomain = \{s_\datindex \, : \, \datindexin\} \cup \{C\} \cup \{C_\atomenumerator \, : \, \atomenumeratorin\}$, where $s_{\datindex}$ represent samples and $C_\atomenumerator$ unary predicates, by
    \begin{align*}
        \kggroundingof{\rdf}
        =
        \sum_{\datindexin}
        \onehotmapof{\indexinterpretationof{s_\datindex}}{\sindvariable}
        \otimes \onehotmapof{\indexinterpretationof{\mathrdftype}}{\pindvariable}
        \otimes \onehotmapof{\indexinterpretationof{C}}{\oindvariable}
        +
        \sum_{\datindexin} \sum_{\atomenumeratorin \, : \, \datamap_{\atomenumerator}(\datindex)=1}
        \onehotmapof{\indexinterpretationof{s_\datindex}}{\sindvariable}
        \otimes \onehotmapof{\indexinterpretationof{\mathrdftype}}{\pindvariable}
        \otimes \onehotmapof{\indexinterpretationof{C_\atomenumerator}}{\oindvariable} \, .
    \end{align*}
    We further define an importance formula by the $\sparql$ query
    \begin{centeredscript}
        \impformula = SELECT \{ ?x \} WHERE \{ ?x \quad \rdftype\quad C \, .\}
    \end{centeredscript}
    and for each $\atomenumeratorin$ an extraction formula by the query
    \begin{centeredscript}
        $\extformulaof{\atomenumerator}$ = SELECT \{ ?x \} WHERE \{ ?x \quad \rdftype \quad $C_\atomenumerator$ \, .\} \, .
    \end{centeredscript}
    Then the triple $(\kg,\impformula,\shortextformulas)$ reproduces $\empdistribution$.
\end{theorem}
\begin{proof}
    We show the theorem analogously to \theref{the:reproducingSingleObjects}, with the slide difference in the importance formula.
    We have for the grounding of $\impformula$ on $\kg$ that
    \begin{align*}
        \kggroundingofat{\impformula}{\indvariable} = \sum_{\datindexin}  \onehotmapof{\indexinterpretationof{s_\datindex}}{\indvariable}
    \end{align*}
    and enumerate the non-vanishing coordinates by $\datvariable$.

    For each extraction formula we have
    \begin{align*}
        \kggroundingofat{\extformulaof{\atomenumerator}}{\indvariable} = \sum_{\datindexin \, : \, \datamap_{\atomenumerator}(\datindex)=1} \onehotmapof{\indexinterpretationof{s_\datindex}}{\indvariable} \,.
    \end{align*}
    It follows that the data cores used in \theref{the:extractionDataCores} are
    \begin{align*}
        \bencodingofat{\datamap_\atomenumerator}{\catvariableof{\atomenumerator},\datindex}
        = \onehotmapofat{0}{\catvariableof{\atomenumerator}} \otimes \left(\sum_{\datindexin \, : \, \datamap_{\atomenumerator}(\datindex)=0} \onehotmapofat{\datindex}{\datvariable}\right)
        +\tbasisat{\catvariableof{\atomenumerator}} \otimes \left(\sum_{\datindexin \, : \, \datamap_{\atomenumerator}(\datindex)=1} \onehotmapofat{\datindex}{\datvariable}\right)
    \end{align*}
    and they thus coincide with those in the decomposition in \theref{the:empCPRep}.
    The claim follows therefore with the same argumentation as in the proof of \theref{the:reproducingSingleObjects}.
\end{proof}

%
Let us provide some more insights on the construction of the reproducing Knowledge Graph in \theref{the:reproducingKGSingelObjects}.
By the insertions to the one-hot encodings $\onehotmapof{\indexinterpretationof{s_\datindex}}{\sindvariable} \otimes \onehotmapof{\indexinterpretationof{\mathrdftype}}{\pindvariable} \otimes \onehotmapof{\indexinterpretationof{C}}{\oindvariable}$ we mark each sample representing resource by a class and ensure its appearance as a $\mathrm{owl:NamedIndividual}$ in the graph.
The insertions $\onehotmapof{\indexinterpretationof{s_\datindex}}{\sindvariable}\otimes \onehotmapof{\indexinterpretationof{\mathrdftype}}{\pindvariable} \otimes \onehotmapof{\indexinterpretationof{C_\atomenumerator}}{\oindvariable}$ on the other side encode the sample selecting map, by inserting exactly the assertions corresponding with the respective sample.
% 
In this simple Knowledge Graph, Description Logic is expressive enough to represent any formula $\folexformula$ composed of the formulas $\extformulas$.

%
%\begin{theorem}
%    Let there any empirical distribution $\empdistribution\in\bigotimes_{\atomenumeratorin}\rr^2$ and $\datanum\in\nn$ such that $\imageof{\datanum\cdot\empdistribution}\subset\nn$.
%    Then the tuple $(\kg,\impformula,\{\extformulas\})$ defined by a Knowledge Graph
%    \begin{align}
%        \kg =
%        & \bigcup_{\atomindicesin}  \{(
%        s_{j, \atomindices} \quad \mathrm{rdf:type} \quad C ) : j \in [\datanum\cdot\empdistribution(\atomindices)] \}  \\
%        &\bigcup_{\atomindicesin}  \{(
%        s_{j, \atomindices} \quad \mathrm{rdf:type} \quad C_\atomenumerator
%        ) : j \in [\datanum\cdot\empdistribution(\atomindices)], \atomenumeratorin , \atomlegindexof{\atomenumerator}=1\}
%    \end{align}
%    further an importance formula by the query
%    \begin{centeredscript}
%        \impformula = SELECT \{ ?x \} WHERE \{ ?x \quad \rdftype\quad C \, .\}
%    \end{centeredscript}
%    and extraction formulas for each $\atomenumeratorin$ by the query
%    \begin{centeredscript}
%        $\extformulaof{\atomenumerator}$ = SELECT \{ ?x \} WHERE \{ ?x \quad \rdftype \quad $C_\atomenumerator$ \, .\}
%    \end{centeredscript}
%    reproduces $\empdistribution$.
%\end{theorem}
%\begin{proof}
%    With respect to any enumeration of the resources of $\kg$ we have
%    \begin{align}
%        \kggroundingof{\impformula}
%        = \sum_{\atomindicesin} \sum_{j \in [\datanum\cdot\empdistribution(\atomindices)]} \onehotmapof{s_{j, \atomindices} }
%    \end{align}
%    and
%    \begin{align}
%        \kggroundingof{\extformulaof{\atomenumerator}}
%        = \sum_{\atomindicesin \, : \, \atomlegindexof{\atomenumerator} = 1} \sum_{j \in [\datanum\cdot\empdistribution(\atomindices)]} \onehotmapof{s_{j, \atomindices} } \, .
%    \end{align}
%    Summing over the resource variables of these tensors in a contraction we get
%    \begin{align}
%        \contractionof{\{\kggroundingof{\impformula}\}\cup\{\bencodingof{\kggroundingof{\extformulaof{\atomenumerator}}\, : \, \atomenumeratorin}\}}{\shortcatvariables}
%        & = \sum_{\atomenumeratorin}  \datanum\cdot\empdistribution(\atomindices) \cdot \onehotmapof{\atomindices} = \datanum \cdot \empdistribution
%    \end{align}
%    and therefore
%    \begin{align}
%        \normalizationof{\{\kggroundingof{\impformula}\}\cup\{\bencodingof{\kggroundingof{\extformulaof{\atomenumerator}}}\, : \, \atomenumeratorin\}}{\shortcatvariables} = \empdistribution \, .
%    \end{align}
%\end{proof}







\subsect{Samples by pairs of objects}

%\paragraph{TBox:} The categorical variables of the factored system are the classes.
%We define atomic formulas by the state indicators of each categorical variable as in \secref{sec:categoricalTN}.
%Each such atomic formula corresponds with a sub-class of the classes.
%By definition, each collection of state indicators define thus pairwise disjoint subclasses.
%
%\paragraph{ABox:} The samples are represented by single individuals in the Knowledge Graph.
%Their sub-class memberships corresponding with the categorical variables of the system are instantiated whenever the atom is true in the sample.
%%\subsubsect{Samples by pairs of resources}
%
%\begin{remark}[Refinement of the Samples]
%    We can split each sample node into a pair of individuals.
%    For this we need to specify, which each class membership will be encoded in a unary or binary attribute of the splitted individuals.
%    This specification is possible based on the extraction query and the atomic formulas.
%\end{remark}
%
%%
%Taking any importance query $\impformula$, which has no permutation symmetries, we can instantiate each projection variable for each sample and prepare the links according to the triple patterns.
%When the atom queries $\extformulas$ have different triple patterns compared with $\impformula$, we instantiate those in cases where $\atomlegindexof{\atomenumerator}=1$.


%
We now instantiate multiple objects for each datapoint, one for each variable of the importance formula, i.e. $\worlddomain=[\datdim]\times[\indorder]$
Label individuals $s_{\datindex,\indenumerator}$ by data index and variable index.

\begin{lemma}
    Let there a data map $\datamap$, queries $\impformula,\shortextformulas$ and a \firstOrderLogic{} world containing objects $s_{\datindex,\indenumerator}$ for $\datindexin$ and $\indenumeratorin$
    If
    \begin{align*}
        \kggroundingof{\impformula}
        = \sum_{\datindexin} \bigotimes_{\indenumeratorin} \onehotmapofat{\indexinterpretationof{s_{\datindex,\indenumerator}}}{\indvariableof{\indenumerator}}
    \end{align*}
    and for any $\atomenumeratorin$
    \begin{align*}
        \kggroundingof{\extformulaof{\atomenumerator}}
        = \sum_{\datindex : \datamap_{\atomenumerator}(\datindex)=1} \bigotimes_{\indvariableof{\indenumerator} \in \indvariableof{\extformulaof{\atomenumerator}}}
        \onehotmapofat{\indexinterpretationof{s_{\datindex,\indenumerator}}}{\indvariableof{\indenumerator}} \, .
    \end{align*}
%    \[ \kggroundingof{\extformulaof{\atomenumerator}}
%    = \sum_{\datindex : \datamap^{\atomenumerator}(\datindex)=1} \bigotimes_{\indenumerator \in \extformulaof{\atomenumerator}} \onehotmapof{\datindex,\indenumerator} \, . \]
    Then the tuple $(\kg,\impformula,\{\extformulas\})$ reproduces $\empdistribution$.
\end{lemma}
\begin{proof}
    We notice, that the grounding of the importance formula is in a basis CP format, since by assumption
    \begin{align*}
        \kggroundingof{\impformula}
        = \sum_{\datindexin} \bigotimes_{\indenumeratorin} \onehotmapofat{\indexinterpretationof{s_{\datindex,\indenumerator}}}{\indvariableof{\indenumerator}} \, .
    \end{align*}
    We choose $\datvariable$ to enumerate the non-vanishing entries and get a term selecting map
    \begin{align*}
        \secdatamap_{\indenumerator}(\datindex) = \indexinterpretationof{s_{\datindex,\indenumerator}} \, .
    \end{align*}
    From this we have
    \begin{align*}
        \contractionof{
            \{\bencodingofat{\kggroundingof{\extformulaof{\atomenumerator}}}{\catvariableof{\atomenumerator},\indvariableof{\extformulaof{\atomenumerator}}}\} \cup
            \{\bencodingofat{\secdatamap_{\indenumerator}}{\indvariableof{\indenumerator},\datvariable} \, : \, \indenumeratorin\}
        }{\catvariableof{\catenumerator},\datvariable}
        = \bencodingofat{\datamap_{\atomenumerator}}{\catvariableof{\catenumerator},\datvariable}
    \end{align*}
    and the claim follows with the same argumentation as in the proof of \theref{the:reproducingSingleObjects}.
\end{proof}


%Let us construct a Knowledge Graph
%\begin{align*}
%        \kggroundingof{\rdf}
%        =
%        \sum_{\datindexin}\sum_{\indenumeratorin}
%        \onehotmapof{\indexinterpretationof{s_{\datindex,\indenumerator}}}{\sindvariable}
%        \otimes \onehotmapof{\indexinterpretationof{\mathrdftype}}{\pindvariable}
%        \otimes \onehotmapof{\indexinterpretationof{C}}{\oindvariable}
%        +
%        \sum_{\datindexin} \sum_{\atomenumeratorin \, : \, \datamap_{\atomenumerator}(\datindex)=1} \sum_{\indvariableof{\indenumerator}\in\indvariableof{}}
%        \onehotmapof{\indexinterpretationof{s_{\datindex,\indenumerator}}}{\sindvariable}
%        \otimes \onehotmapof{\indexinterpretationof{\mathrdftype}}{\pindvariable}
%        \otimes \onehotmapof{\indexinterpretationof{C_\atomenumerator}}{\oindvariable} \, .
%\end{align*}
%    We further define an importance formula by the $\sparql$ query
%\begin{centeredscript}
%        \impformula = SELECT \{ ?x_0 \cdots ?x_{\indorder-1} \} WHERE \{ ?x_0 \quad \rdftype\quad C \, .\}
%\end{centeredscript}
%    and for each $\atomenumeratorin$ an extraction formula by the query
%\begin{centeredscript}
%        $\extformulaof{\atomenumerator}$ = SELECT \{ ?x \} WHERE \{ ?x \quad \rdftype \quad $C_\atomenumerator$ \, .\} \, .
%\end{centeredscript}
%    Then the triple $(\kg,\impformula,\shortextformulas)$ reproduces $\empdistribution$.



\sect{Discussion}


% Probabilistic Relational Models
Statistical Models are called Probabilistic Relational Models. % (RUSSELL - Chapter Probabilistic Programming).
Extensions are models that also handle structural uncertainty, i.e. distributions of worlds with varying $\worlddomain$.

% Comparison with network science
In the emerging area of network science \cite{barabasi_network_2016, giovanni_russo_vito_latora_complex_2017}, statistical models for random graphs are investigated.
Statistical Models of \firstOrderLogic{} go beyond the typical single edge type perspective of network science.


%
\begin{remark}[Alternative Representation of empirical distributions]
    So far, we have motivated the representation of empirical distributions based on basis CP decompositions based on data maps.
    In this section, based on the extraction queries, we have observed that empirical distributions might have more efficient representation formats.
    In many applications such as the computation of log-likelihoods we can use any representation of the empirical distribution by tensor networks.
    It is thus not necessary to compute the data cores as above, unless one requires a list of the extracted samples.
\end{remark}



\part{Contraction Calculus}

Based on the logical interpretation we often handle tensor calculus with specific tensors.
Often, they are binary (that is their coordinates are in $\{0,1\}$ corresponding with a Boolean), and sparse (that is having a decomposition with less storage demand).
We investigate it in this part in more depth the properties of such tensors, which where exploited in the previous parts.

\chapter{\chatextcoordinateCalculus} \label{cha:coordinateCalculus}

In the previous chapters, information to states has been stored in coordinates of a tensor.
To distinguish from other schemes of calculus such as the basis calculus (see \charef{cha:basisCalculus}), we call this scheme of storing and retrieving information the coordinate calculus.
%We in this chapter investigate in more depth, which operations can be performed based on such tensors and proof the applied properties.

\sect{One-hot encodings as basis}

Let us first show, that the one-hot encodings, which we have used to motivate tensor representations, build an orthonormal basis of the respective tensor spaces.

\begin{lemma}%[Basis of tensor spaces]
    \label{lem:tensorBasisDecomposition}
    The image of the one-hot encoding map is an orthonormal basis of the tensor space $\facspace$, that is for any $\shortcatindices,\tildeshortcatindices\in\facstates$ we have
    \begin{align*}
        \contraction{\onehotmapofat{\shortcatindices}{\shortcatvariables},\onehotmapofat{\tildeshortcatindices}{\shortcatvariables}}
        = \deltaof{\shortcatindices,\tildeshortcatindices}
        \coloneqq
        \begin{cases}
            1 & \ifspace \shortcatindices=\tildeshortcatindices \\
            0 & \text{else}
        \end{cases} \, .
    \end{align*}
    Any element $\hypercore\in\facspace$ has a decomposition
    \begin{align*}
        \hypercoreat{\shortcatvariables}
        = \sum_{\shortcatindicesin} \hypercoreat{\indexedshortcatvariables} \cdot \onehotmapofat{\shortcatindices}{\shortcatvariables} \, .
    \end{align*}
    We notice that the coordinates are the weights to the basis elements in the one-hot decomposition.
\end{lemma}
\begin{proof}
    The first claim follows from an elementary decomposition of one-hot encodings and the orthogonality of basis vectors as
    \begin{align*}
        \contraction{\onehotmapofat{\shortcatindices}{\shortcatvariables},\onehotmapofat{\tildeshortcatindices}{\shortcatvariables}}
        = \prod_{\catenumeratorin} \contraction{\onehotmapofat{\catindexof{\atomenumerator}}{\catvariableof{\atomenumerator}},\onehotmapofat{\tildecatindexof{\atomenumerator}}{\catvariableof{\atomenumerator}}}
        = \prod_{\catenumeratorin} \delta_{\catindexof{\atomenumerator},\tildecatindexof{\atomenumerator}}
        = \deltaof{\shortcatindices,\tildeshortcatindices} \, .
    \end{align*}
    To show the second claim, it is enough to notice that for any $\tildeshortcatindices\in\facstates$ we have
    \begin{align*}
        \sum_{\shortcatindicesin} \hypercoreat{\indexedshortcatvariables} \cdot \onehotmapofat{\shortcatindices}{\shortcatvariables=\tildeshortcatindices}
        &= \sum_{\shortcatindicesin} \hypercoreat{\indexedshortcatvariables} \cdot \deltaof{\shortcatindices,\tildeshortcatindices} \\
        &=   \hypercoreat{\shortcatvariables=\tildeshortcatindices} \, . \qedhere
    \end{align*}
\end{proof}

Any tensor can be understood as a coordinate encoding of a real-valued function, as we define next.

\begin{definition}\label{def:coordinateEncoding}
    Given any real-valued function
    \begin{align*}
        \exfunction \defcols \facstates \rightarrow \rr
    \end{align*}
    we define the coordinate encoding by
    \begin{align*}
        \cencodingofat{\exfunction}{\shortcatvariables}
        = \sum_{\shortcatindicesin} \exfunctionat{\shortcatindices} \cdot \onehotmapofat{\shortcatindices}{\shortcatvariables} \, .
    \end{align*}
\end{definition}

In \parref{par:one} and \parref{par:two} we did not distinguish between a real-valued function $\exfunction$ and its coordinate encoding $\hypercoreof{\exfunction}$, in order to abbreviate notation.
Based on coordinate encodings, we now show, that function evaluation can be performed by contractions.

\begin{theorem}[Function evaluation in Coordinate Calculus]
    \label{the:coordinateCalculus}
    Given any real-valued function
    \begin{align*}
        \exfunction \defcols \facstates \rightarrow \rr
    \end{align*}
    and any input state $\shortcatindicesin$, we have
    \begin{align*}
        \exfunctionat{\shortcatindices}
        = \contraction{\cencodingofat{\exfunction}{\shortcatvariables},\onehotmapofat{\shortcatindices}{\shortcatvariables}} \, .
    \end{align*}
\end{theorem}
\begin{proof}
    We use the decomposition in \lemref{lem:tensorBasisDecomposition} and have by linearity of contractions for any index tuple $\shortcatindices\in\facstates$
    \begin{align*}
        \contraction{\cencodingofat{\exfunction}{\shortcatvariables},\onehotmapofat{\shortcatindices}{\shortcatvariables}}
        & = \sum_{\tildeshortcatindices\in\facstates}
        \cencodingofat{\exfunction}{\shortcatvariables=\tildeshortcatindices}
        \cdot \contraction{\onehotmapofat{\tildeshortcatindices}{\shortcatvariables},\onehotmapofat{\shortcatindices}{\shortcatvariables}} \\
        & = \sum_{\tildeshortcatindices\in\facstates}
        \exfunctionat{\tildeshortcatindices}
        \cdot \delta_{\tildeshortcatindices,\shortcatindices} \\
        & = \exfunctionat{\shortcatindices}
    \end{align*}
    where we used that one-hot encodings are orthonormal.
\end{proof}

% Coordinate Calculus
Coordinate calculus is the representation of real-valued functions as tensors, from which its evaluations can be retrieved by the scheme of \theref{the:coordinateCalculus}.
This is in contrast to the basis calculus scheme to be discussed (see \theref{the:basisCalculus}), where the contraction-based evaluations of functions outputs one-hot encodings.

% Retrieval of Coordinates from tensor networks
Tensors of large orders often admit a decomposition by tensor networks.
We in the next theorem show, how such a decomposition can be exploited for efficient contractions and in particular coordinate retrieval.

\begin{theorem}
    \label{the:slicedContractionToCores}
    Given a tensor network $\tnetof{\graph}$ on a hypergraph $\graph=(\nodes,\edges)$, disjoint subsets $\nodesa,\nodesb\subset\nodes$ and $\catindexofin{\nodesb}$, we have
    \begin{align*}
        \contractionof{\tnetof{\graph}}{\catvariableof{\nodesa},\indexedcatvariableof{\nodesb}}
        = \contractionof{
            \{\contractionof{\hypercoreof{\edge}}{\catvariableof{\edge/\nodesb},\indexedcatvariableof{\edge\cap\nodesb}} \, : \, \edge\in\edges \}
        }{\catvariableof{\nodesa}} \, .
    \end{align*}
\end{theorem}
\begin{proof}
    By definition of contractions we have for any $\catindexof{\nodesa}$
    \begin{align*}
        \contractionof{\tnetof{\graph}}{\indexedcatvariableof{\nodesa},\indexedcatvariableof{\nodesb}}
        &= \sum_{\catindexof{\nodes/(\nodesa\cup\nodesb)}\in\nodestatesof{\nodes/(\nodesa\cup\nodesb)}} \prod_{\edge\in\edges} \hypercoreofat{\edge}{\indexedcatvariableof{\edge/\nodesb},\indexedcatvariableof{\edge\cap\nodesb}} \\
        &= \contraction{
            \{\contraction{\hypercoreof{\edge}}{\catvariableof{\edge/(\nodesa\cup\nodesb)},\indexedcatvariableof{\edge\cap\nodesa},\indexedcatvariableof{\edge\cap\nodesb}} \, : \, \edge\in\edges \}
        } \\
        &= \contractionof{
            \{\contraction{\hypercoreof{\edge}}{\catvariableof{\edge/\nodesb},\indexedcatvariableof{\edge\cap\nodesb}} \, : \, \edge\in\edges \}
        }{\indexedcatvariableof{\nodesa}}
    \end{align*}
    and the claim follows.
\end{proof}

% Special case of retrieving single coordinates
If we retrieve a single coordinate of a tensor, we have the situation $\nodesa=\varnothing$, $\nodesb=\nodes$.
In that case, \theref{the:slicedContractionToCores} shows, that the coordinate is the product of the coordinates of the cores. % Thus no contraction required!

\sect{Coordinatewise Transforms}\label{sec:coordinatewiseTransforms}

Let us now discuss a scheme to perform transformations of tensors.
We call them coordinatewise, when the target tensor has the same variables as the input tensors, and each coordinate of the target tensor depends only on the respective coordinates of the input tensors. %Examples for non-coordinatewise transforms are e.g. backward and forward maps

\begin{definition}
    \label{def:coordinatewiseTransform}
    Let $\chainingfunction: \parspace \rightarrow \rr$ be a function.
    Then the coordinatewise transform of tensors $\hypercoreofat{\selindex}{\shortcatvariables}$, where $\selindexin$, under $\exfunction$ is the tensor
    \begin{align*}
        \coordinatetrafowrtofat{\chainingfunction}{\hypercoreof{0},\ldots,\hypercoreof{\seldim-1}}{\shortcatvariables}
    \end{align*}
    with coordinates
    \begin{align*}
        \coordinatetrafowrtofat{\chainingfunction}{\hypercoreof{0},\ldots,\hypercoreof{\seldim-1}}{\indexedshortcatvariables}
        = \chainingfunctionof{\hypercoreofat{0}{\indexedshortcatvariables},\ldots,\hypercoreofat{\seldim-1}{\indexedshortcatvariables}} \, .
    \end{align*}
\end{definition}

% \seldim=1
Coordinatewise transforms in case of $\seldim=1$ have been indicated by ellipses in the diagrammatic depiction of contractions.
We will provide a generic tensor network representation in \charef{cha:basisCalculus}, see \theref{the:tensorFunctionComposition}.


In the following lemma, we state that coordinatewise transforms can be restricted to slices of tensors, when
Although this is an obvious fact, this property can tremendously reduce the computational demand of contractions with coordinatewise transforms of tensors.

\begin{lemma}\label{lem:coordinatewisetrafoSliceReduction}
    For any function $\chainingfunction: \rr \rightarrow \rr$, any tensor $\hypercoreat{\shortcatvariables}$ and index $\catindexof{\variableset}$, where $\variableset\subset[\catorder]$, we have
    \begin{align*}
        \coordinatetrafowrtofat{\chainingfunction}{\hypercoreat{\shortcatvariables}}{\catvariableof{[\catorder]/\variableset},\indexedcatvariableof{\variableset}}
        = \coordinatetrafowrtofat{\chainingfunction}{\catvariableof{[\catorder]/\variableset},\indexedcatvariableof{\variableset}}{\catvariableof{[\catorder]/\variableset}} \, .
    \end{align*}
\end{lemma}
\begin{proof}
    For any state $\catindexof{[\catorder]/\variableset}$ we have that
    \begin{align*}
        \coordinatetrafowrtofat{\chainingfunction}{\hypercoreat{\shortcatvariables}}{\indexedcatvariableof{[\catorder]/\variableset},\indexedcatvariableof{\variableset}}
        &= \chainingfunctionof{\hypercoreat{\indexedshortcatvariables}} \\
        &= \coordinatetrafowrtofat{\chainingfunction}{\catvariableof{[\catorder]/\variableset},\indexedcatvariableof{\variableset}}{\indexedcatvariableof{[\catorder]/\variableset}} \, . \qedhere
    \end{align*}
\end{proof}



% Examples
\begin{example}[Hadamard products as coordinatewise transforms]
    Hadamard products of tensors (see \exaref{exa:hadamard}) are a special way of coordinate calculus, where the transform is the product and thus
    \begin{align*}
        \coordinatetrafowrtofat{\cdot\,}{\hypercoreof{0},\ldots,\hypercoreof{\seldim-1}}{\shortcatvariables}
        = \contractionof{\{\hypercoreofat{\selindex}{\shortcatvariables} \, : \, \selindexin \}}{\shortcatvariables} \, .
    \end{align*}
    These hadamard products are applied in the effective computation of conjunctions, as we will discuss in more detail in \secref{sec:hybridCalculus}.
\end{example}

\begin{example}[Exponentiation of energies]
    In \defref{def:expFamily} we introduced exponential families, based on the exponentiation of energies.
    For a statistic $\sstat$, a base measure $\basemeasure$ and a canonical parameter $\canparam$ we defined
    \begin{align*}
        \stanexpdistof{\canparam} = \frac{
            \contractionof{\expof{\contractionof{\sencsstat,\canparam}{\shortcatvariables},\basemeasureat{\shortcatvariables}}}{\shortcatvariables}
        }{
            \contraction{\expof{\contractionof{\sencsstat,\canparam}{\shortcatvariables},\basemeasureat{\shortcatvariables}}}
        } \, .
    \end{align*}
    Both the nominator and the denominator involve a coordinatewise transform of the energy tensor $\expenergy$ by the exponentiation.
    \theref{the:expFamilyTensorRep} provided a transform-free contraction expression by basis encodings, which is the central tool of basis calculus (see \charef{cha:basisCalculus}).

    Let us note, that \lemref{lem:coordinatewisetrafoSliceReduction} enables the energy-based answering of conditional queries, as has been shown in \theref{the:energyContractionQueries}.
\end{example}




\sect{Directed Tensors}

Directionality as defined in \defref{def:directedTensor} is a constraint on the structure of a tensor, namely that the contraction leaving only incoming variables open trivializes the tensor.
We have motivated such constraints by conditional distributions, see \defref{def:condIndependence}, and referred to Markov Networks (see \defref{def:markovNetwork}) satisfying these by Bayesian Networks (see \defref{def:bayesianNetwork}).
To support our findings therein, we now discuss in more detail the connection between directed hypergraphs and directed tensors.

\begin{definition}[Directed Hypergraph]
    A directed hyperedge is a hyperedge, which node set is split into disjoint sets of incoming and outgoing nodes.
    We say a hypercore $\hypercoreof{\edge}$ decorating a directed hyperedge respects the direction, when it is a conditional probability tensor with respect to the direction of the hyperedge.
    The hypergraph is acyclic, when there is no nonempty cycle of node tuples $(\node_1,\node_2)$, such that $\node_1$ is an incoming node and $\node_2$ an outgoing node of the same hyperedge.
\end{definition}

% Multiple Directions possible
There can be multiple ways to direct a tensor, with an extreme example being Diracs Delta Tensors to be introduced in the next example.
More general examples are basis encodings of invertible functions.

\begin{example}[Dirac Delta Tensors]\label{exa:diracDeltaTensor}
    Given a set of variables $\shortcatvariables=\catvariables$ with identical dimension $\catdim$, Diracs Delta Tensor is the element
    \begin{align*}
        \dirdeltawith \in \bigotimes_{\catenumeratorin} \rr^{\catdim}
    \end{align*}
    with coordinates
    \begin{align}
        \dirdeltaofat{[\catorder],\catdim}{\indexedshortcatvariables} =
        \begin{cases}
            1 \quad & \ifspace \catindexof{0} = \ldots = \catindexof{\catorder-1} \\
            0 & \text{else}
        \end{cases} \, .
    \end{align}
    The contractions with respect to subsets $\secnodes\subset[\catorder]$ are
    \begin{align}
        \contractionof{\dirdeltaof{[\catorder],\catdim}}{\catvariableof{\secnodes}} =
        \begin{cases}
            \catdim & \ifspace \secnodes = \varnothing \\
            \onesat{\catvariableof{\secnodes}} & \ifspace \cardof{\secnodes} = 1\\
            \dirdeltaofat{\secnodes,\catdim}{\catvariableof{\secnodes}} & \text{else}
        \end{cases} \, .
    \end{align}
    Thus are directed for any orientation of the respective edge with exactly one incoming variable.
\end{example}

We can use Diracs Delta Tensors to represent any contraction of a tensor network on a hypergraph by a tensor network on a graph, as we show next.

\begin{lemma}
    \label{lem:deltification}
    Let $\graph=(\nodes,\edges)$ be a hypergraph and $\extnet$ a tensor network on $\graph$.
    We build a graph $\secgraph=(\secnodes,\secedges\cup\Delta^{\graph})$ and a tensor network $\tnetof{\secgraph}$ by % ! See Bethe Cluster Graph definition !
    \begin{itemize}
        \item Recolored Edges $\secedges = \{\tilde{\edge} \, : \, \edge\in \edges\}$ where $\tilde{\edge} = \{\node^{\edge} \, : \, \node\in\edge\}$, which decoration tensor $\hypercoreof{\tilde{\edge}}$ has same coordinates as $\hypercoreof{\edge}$
        \item Nodes $\secnodes = \bigcup_{\edge\in\edges}\tilde{\edge}$ %$\secnodes = \bigcup_{\edge\in\edges}\{\node^{\edge} \, : \, \node\in\edge \}$
        \item Delta Edges $\Delta^{\graph} =  \big\{ \{\node\} \cup \{\node^{\edge} \, : \, \edge\ni\node \} \, : \, \node\in\nodes \big\} $ each of which decorated by a delta tensor $\delta^{\{\node^{\edge} \, : \, \edge\ni\node \}}$
    \end{itemize}
    Then we have
    \begin{align*}
        \contractionof{\extnet}{\catvariableof{\nodes}} =  \contractionof{\tnetof{\secgraph}}{\catvariableof{\nodes}}  \, .
    \end{align*}
\end{lemma}
\begin{proof}
    For any $\catindexof{\nodes}$ we have
    \begin{align*}
        \contractionof{\tnetof{\secgraph}}{\indexedcatvariableof{\nodes}}
        & = \contraction{\{\hypercoreof{\tilde{\edge}}[\catvariableof{\{\node^{\edge} : \node\in\edge\}}] : \edge \in \edges \}\cup
        \{\delta^{\{\node\} \cup \{\node^{\edge} \, : \, \edge\ni\node \}}[\catvariableof{\{\node^{\edge} : \edge\ni\node \}}, \indexedcatvariableof{\node} ]  : \node\in\nodes \}
        } \\
        & =  \contraction{\{\hypercoreof{\tilde{\edge}}[\catvariableof{\{\node^{\edge} : \node\in\edge\}} = \catindexof{\{\node : \node\in\edge\}} ] : \edge \in \edges \}
        } \\
        & = \contractionof{\extnet}{\indexedcatvariableof{\nodes}} \, ,
    \end{align*}
    which establishes the claim.
\end{proof}

\subsect{Normalization}

Normed tensors (see \defref{def:normalization}) are directed and directed tensors invariant under normalization wrt their incoming and outgoing variable, as we show next.

\begin{theorem}
    \label{the:normalizationDirected}
    For any tensor network $\extnet$ on variables $\nodes$ that can be normed with respect to $\innodes$ and $\outnodes$, the normalization is directed with $\innodes$ incoming and $\outnodes$ outgoing.
\end{theorem}
\begin{proof}
    We have for any incoming state ${\atomlegindexof{\innodes}\in\bigtimes_{\node\in\innodes}\catdimof{\node}}$ that
    \begin{align*}
        \contraction{\normalizationofwrt{\extnet}{\innodes}{\outnodes}, \onehotmapof{\atomlegindexof{\innodes}}}
        & =  \frac{
            \contraction{\extnet\cup\{\onehotmapof{\atomlegindexof{\innodes}}\}}
        }{
            \contraction{\extnet\cup\{\onehotmapof{\atomlegindexof{\innodes}}\}}
        } \, .
    \end{align*}
    By \defref{def:directedTensor}, $\normalizationofwrt{\extnet}{\outnodes}{\innodes}$ is thus directed.
\end{proof}

The normalization operation coincides in cases of non-negative tensors with the conditioning of a Markov Network representing a probability distribution.

\subsect{Normalization Equations}

Normalization equations capture certain properties of normalizations of tensors.
We first show that any normalizable tensor is the contraction of its normalization and an accompanying contraction, which generalizes the Bayes \theref{the:bayes} towards more generic normalizable tensors.

\begin{theorem}[normalization as a Contraction Equation]
    \label{the:normalizationContractionEQ}
    For any on $\innodes$ normalizable tensor $\hypercoreat{\catvariableof{\nodes}}$, where $\innodes\dot{\cup}\outnodes=\nodes$, we have
    \begin{align*}
        \contractionof{\hypercore}{\catvariableof{\nodes}}
        = \contractionof{\normalizationofwrt{\hypercore}{\catvariableof{\outnodes}}{\catvariableof{\innodes}},\contractionof{\hypercore}{\catvariableof{\innodes}}}{\catvariableof{\nodes}} \, .
    \end{align*}
\end{theorem}
\begin{proof}
    Let us choose indices $\catindexof{\innodes}$ and $\catindexof{\outnodes}$.
    We have that
    \begin{align*}
        %\contractionof{
        \normalizationofwrt{\hypercore}{\indexedcatvariableof{\innodes}}{\indexedcatvariableof{\outnodes}}
        %}{\indexedcatvariableof{\innodes},\indexedcatvariableof{\outnodes}}
        = \frac{
            \contractionof{\hypercore}{\indexedcatvariableof{\innodes},\indexedcatvariableof{\outnodes}}
        }{
            \contractionof{\hypercore}{\indexedcatvariableof{\innodes}}
        }
    \end{align*}
    and therefor
    \begin{align*}
        \contractionof{\hypercore}{\indexedcatvariableof{\innodes},\indexedcatvariableof{\outnodes}} =
        \normalizationofwrt{\hypercore}{\indexedcatvariableof{\innodes}}{\indexedcatvariableof{\outnodes}}
        \cdot
        \contractionof{\hypercore}{\indexedcatvariableof{\innodes}}
    \end{align*}
    Since the equation holds for arbitrary indices, the claim is established.
\end{proof}

Based on this property, we now show a generic decomposition scheme of tensors, which generalizes the chain rule of \theref{the:chainRule}.

\begin{theorem}[Generic Chain Rule]
    \label{the:genericChainRule}
    For any Tensor $\hypercoreat{\catvariableof{\nodes}}$ and any total order $\prec$ on the nodes $\nodes$ we have % ! CAN DIRECTLY USE [d] when having the order !
    \begin{align*}
        \hypercoreat{\catvariableof{\nodes}} =
        \contractionof{
            \{\normalizationofwrt{\hypercore}{\catvariableof{\node}}{\catvariableof{\prenodes}}  \, : \nodein \}
        }{\catvariableof{\nodes}} \, ,
    \end{align*}
    provided that the normalizations exist.
\end{theorem}
\begin{proof}
    We apply \theref{the:normalizationContractionEQ} on the tensor
    \begin{align*}
        \normalizationofwrt{\hypercore}{
            \catvariableof{\node},\catvariableof{\afternodes}
        }{
            \indexedcatvariableof{\prenodes}
        } \, ,
    \end{align*}
    where $\nodein$ and $\catindexof{\nodes}$ are chosen arbitrarly.
    For any $\nodein$ we get
    \begin{align*}
        \normalizationofwrt{\hypercore}{
            \catvariableof{\node},\catvariableof{\afternodes}
        }{
            \catvariableof{\prenodes}
        }
        = \contractionof{
            \normalizationofwrt{\hypercore}{
                \catvariableof{\afternodes}
            }{
                \catvariableof{\node},\catvariableof{\prenodes}
            },
            \normalizationofwrt{\hypercore}{
                \catvariableof{\node}
            }{
                \catvariableof{\prenodes}
            }
        }{
            \catvariableof{\nodes}
        } \, .
    \end{align*}
    Applying this equation iteratively and making use of the commutation of contractions we get for any $\nodein$
    \begin{align*}
        \normalizationofwrt{\hypercore}{
            \catvariableof{\node},\catvariableof{\afternodes}
        }{
            \catvariableof{\prenodes}
        }
        = \contractionof{
            \normalizationofwrt{\hypercore}{
                \catvariableof{\secnode}
            }{
                \catvariableof{\{\thirdnode : \thirdnode \prec \secnode, \thirdnode\neq\secnode\}}
            }
            \, : \node \prec \secnode
        }{
            \catvariableof{\nodes}
        } \, .
    \end{align*}
    With the maximal node $\node$, that is the $\node$, such that no $\secnode\in\nodes$ with $\node\prec\secnode$ and $\node\neq\secnode$ exists, this is the claim.
\end{proof}


\subsect{Contraction of Directed Tensors}

Let us now investigate, which contractions inherit the directionality of the tensors.

%Next we state that specific contraction of conditional probability tensors are still conditional probability tensors.

%\red{Can be extended to single outgoing legs, by using delta tensors at hyperedges.}

%\begin{theorem}
%	Given a directed acyclic hypergraph, which hyperdedges are decorated by tensor cores respecting the direction.
%	Then the contractions, where all closed nodes appear exactly once as an incoming node and exactly once as an outgoing node, and where all open nodes appear in single hyperedges, are conditional probability tensors
%\end{theorem}
%\begin{proof}
%	It is enough to show this property on the contraction of two hypercores.
%	Since the hypergraph is acyclic, the coinciding nodes are all outgoing on the one and incoming to the other hyperedge.
%	Let the hyperedge with the incoming nodes $\edge_1$ and the one with the outgoing nodes $\edge_2$
%	We need to show that when further contracting the contraction with trivial tensors on the outgoing and basis tensors on the incoming legs we get $1$.
%	For any $\catindex$ and $\seccatindex$ this holds since

%	Here we used in the first equality, that $\hypercoreof{\edge_1}$ is a conditional probability tensor and in the second the same property for $\hypercoreof{\edge_2}$.
%\end{proof}

% Hadamard does not preserve probabilities
%We need to ass as assumption in \theref{the:conditionalContractionPreservation}, that each node is to at most one hyperedge and to at most one hyperedge outgoing.
%This is due to the failure of Hadamard products of probability tensors to be probability tensors themself.


\begin{theorem}
    \label{the:conditionalContractionPreservation}
    Let $\graph=(\nodes,\edges)$ be a directed acyclic hypergraph, such that each node $\node\in\edge$ appears at most in one hyperedge as an outgoing variable and denote $\innodes$ as those nodes, which do not appear as outgoing variables.
    For any tensor network $\extnet$ respecting the direction of $\graph$ we have that
    \[ \contractionof{\extnet}{\catvariableof{\innodes}} = \onesat{\catvariableof{\innodes}} \, , \]
    that is $\contractionof{\extnet}{\catvariableof{\nodes}}$ is a directed tensor with $\innodes$ incoming and $\nodes/\innodes$ outgoing.
\end{theorem}
\begin{proof}
    We show the theorem only for the case of hypergraphs, where variables are appearing at most in two hyperedges.
    If a hypergraph fails to satisfy this assumption, we apply \lemref{lem:deltification} and add delta tensors copying the variables, which are appearing in multiple tensors, and arrive at a tensor network with nodes appearing in at most two hyperedges.

% Approach: Contracting with ones
    We show the theorem over induction on the number $n$ of cores.

    \paragraph{$n=1$:} The claim holds trivially, when $\extnet$ consists of a single core.

    \paragraph{$n-1\rightarrow n$:} Let us assume, that the claim holds for graphs with $n-1$ hyperedges and let $\tnetof{\graph}$ be a tensor network with $n$ hyperedges.
    Since the hypergraph is acyclic, we find an edge $\edge\in\edges$ such that all outgoing nodes of $\edge$ are not appearing as an incoming node in any edge.
    We then apply \theref{the:splittingContractions} and get
    \begin{align*}
        \contractionof{\extnet}{\catvariableof{\innodes}}
        &= \contractionof{
            \tnetof{(\nodes,\edges/\{\edge\})} \cup \{\hypercoreofat{\edge}{\catvariableof{\incomingnodes},\catvariableof{\outgoingnodes}}\}
        }{\catvariableof{\innodes}} \\
        & = \contractionof{
            \tnetof{(\nodes,\edges/\{\edge\})} \cup \{\contractionof{\hypercoreof{\edge}}{\catvariableof{\incomingnodes}} \}
        }{\catvariableof{\innodes}} \\
        & = \contractionof{
            \tnetof{(\nodes,\edges/\{\edge\})} \cup \{\onesat{\catvariableof{\incomingnodes}} \}
        }{\catvariableof{\innodes}} \\
        & \contractionof{
            \tnetof{(\nodes,\edges/\{\edge\})} \}
        }{\catvariableof{\innodes}} \, .
    \end{align*}
    We then notice that the hypergraph $(\nodes,\edges/\{\edge\})$ has $n-1$ hyperedges and each node appears at most once as an incoming and at most once as an outgoing node.
    Thus, we apply the assumption of the induction and get
    \begin{align*}
        \contractionof{\extnet}{\catvariableof{\innodes}} = \contractionof{
            \tnetof{(\nodes,\edges/\{\edge\})} \}
        }{\catvariableof{\innodes}} = \onesat{\catvariableof{\innodes}} \, . & \qedhere
    \end{align*}
\end{proof}


\sect{Proof of Hammersley-Clifford Theorem}\label{sec:proofHCTheorem}

Let us now proof the Hammersley-Clifford theorem formulated in \charef{cha:probRepresentation} as \theref{the:condIndMN}.
Different to the original statement (see \cite{clifford_markov_1971}), we here proof the analogous statement for hypergraphs, where we have to demand the property of clique-capturing defined in \defref{def:ccHypergraph}.
We start with showing the following Lemmata to be exploited in the proof.

\begin{lemma}
    \label{the:contractionFactorization}
    Let $\hypercoreat{\catvariableof{\nodes}}$ be a positive tensor and $\seccatindexof{\nodes}$ an arbitrary index.
    Then we have
    \begin{align*}
        \hypercoreat{\catvariableof{\nodes}}
        = \contractionof{
            \big(\contractionof{\hypercore}{\catvariableof{\nodes/\thirdnodes}, \catvariableof{\thirdnodes} = \seccatindexof{\thirdnodes}}\big)^{(-1)^{\cardof{\secnodes}-\cardof{\thirdnodes}}} \, : \, \thirdnodes \subset \secnodes \subset \nodes
        }{\catvariableof{\nodes}} \, ,
    \end{align*}
    where the exponentiation is performed coordinatewise and positivity of $\hypercore$ ensures the well-definedness.
\end{lemma}
\begin{proof}
    It suffices to show, that for an arbitrary index $\catindexof{\nodes}$ be an arbitrary index we have
    \begin{align*}
        \hypercoreat{\indexedcatvariableof{\nodes}}
        = \prod_{\secnodes\subset\nodes} \prod_{\thirdnodes\subset\secnodes}
        \big(\contractionof{\hypercore}{\indexedcatvariableof{\nodes/\thirdnodes}, \catvariableof{\thirdnodes} = \seccatindexof{\thirdnodes}}\big)^{(-1)^{\cardof{\secnodes}-\cardof{\thirdnodes}}} \, .
    \end{align*}
    We do this by applying a logarithm on the right hand side and grouping the terms by $\thirdnodes$ as
    \begin{align*}
        %\lnof{\hypercoreat{\indexedcatvariableof{\nodes}}}
        & \lnof{\prod_{\secnodes\subset\nodes} \prod_{\thirdnodes\subset\secnodes}
            \contractionof{\hypercore}{\indexedcatvariableof{\nodes/\thirdnodes}, \catvariableof{\thirdnodes} = \seccatindexof{\thirdnodes}}\big)^{(-1)^{\cardof{\secnodes}-\cardof{\thirdnodes}}}} \\
        & = \sum_{\thirdnodes\subset\nodes} \lnof{\contractionof{\hypercore}{\indexedcatvariableof{\nodes/\thirdnodes}, \catvariableof{\thirdnodes} = \seccatindexof{\thirdnodes}}}
        \left( \sum_{\secnodes\subset\nodes \, : \, \thirdnodes\subset \secnodes} (-1)^{\cardof{\secnodes}-\cardof{\thirdnodes}} \right) \\
        & =  \sum_{\thirdnodes\subset\nodes} \lnof{\contractionof{\hypercore}{\indexedcatvariableof{\nodes/\thirdnodes}, \catvariableof{\thirdnodes} = \seccatindexof{\thirdnodes}}}
        \left( \sum_{i \in [\cardof{\nodes}-\cardof{\thirdnodes}]}  (-1)^{i}  \binom{\cardof{\nodes}-\cardof{\thirdnodes}}{i}  \right)
    \end{align*}
    Now, by the generic binomial theorem we have that for $n\in\nn, n \neq 0$
    \[ 0 = (1-1)^n = \sum_{i \in [n]}  (-1)^{i}  \binom{n}{i}   \, . \]
    Therefore, the summands for $\thirdnodes\neq\nodes$ vanish and we have
    \begin{align*}
        & \lnof{ \prod_{\secnodes\subset\nodes} \prod_{\thirdnodes\subset\secnodes}
            \big(\contractionof{\hypercore}{\indexedcatvariableof{\nodes/\thirdnodes}, \catvariableof{\thirdnodes} = \seccatindexof{\thirdnodes}}\big)^{(-1)^{\cardof{\secnodes}-\cardof{\thirdnodes}}} } \\
        & = \lnof{\hypercoreat{\indexedcatvariableof{\nodes}}}
        \left( \sum_{i \in [0]}  (-1)^{i}  \binom{0}{i}  \right) \\
        & = \lnof{\hypercoreat{\indexedcatvariableof{\nodes}}} \, .
    \end{align*}
    Applying the exponential function on both sides establishes the claim.
\end{proof}

\begin{lemma}
    \label{lem:independentContractionFactorization}
    Let $\hypercore$ be a positive tensor, $\secnodes\subset\nodes$ and arbitrary subset and $\catindexof{\secnodes}$ an arbitrary index.
    When there are $\nodesa,\nodesb \in\secnodes$, such that
    \begin{align*}
        \normalizationofwrt{\hypercore}{\catvariableof{\nodesa,\nodesb}}{\catvariableof{\nodes/\{\nodesa,\nodesb\}}}
        = \contractionof{
            \normalizationofwrt{\hypercore}{\catvariableof{\nodesa}}{\catvariableof{\nodes/\{\nodesa,\nodesb\}}},
            \normalizationofwrt{\hypercore}{\catvariableof{\nodesb}}{\catvariableof{\nodes/\{\nodesa,\nodesb\}}}
        }{\catvariableof{\secnodes}}
    \end{align*}
    then
    \begin{align*}
        \prod_{\thirdnodes\subset\secnodes}
        \left(\contractionof{\hypercore}{\indexedcatvariableof{\nodes/\thirdnodes}, \catvariableof{\thirdnodes} = \seccatindexof{\thirdnodes}}\right)^{(-1)^{\cardof{\secnodes}-\cardof{\thirdnodes}}} = 1 \, .
    \end{align*}
\end{lemma}
\begin{proof}
    We abbreviate
    \[ Z_{\thirdnodes} = \contractionof{\hypercore}{\indexedcatvariableof{\nodes/\thirdnodes}, \catvariableof{\thirdnodes} = \seccatindexof{\thirdnodes}} \, .
    \]
    By reorganizing the sum over $\thirdnodes\subset\secnodes$ into  $\thirdnodes\subset\secnodes/\nodesa\cup\nodesb$ we have
    \begin{align}
        \label{eq:indContFacProof}
        \prod_{\thirdnodes\subset\secnodes}
        \left(
        Z_{\thirdnodes}
        \right)^{(-1)^{\cardof{\secnodes}-\cardof{\thirdnodes}}} =
        \prod_{\thirdnodes\subset\secnodes/\{\nodesa,\nodesb\}}
        \left(
        \frac{
            Z_{\thirdnodes} \cdot Z_{\thirdnodes\cup\{\nodesa,\nodesb\}}
        }{
            Z_{\thirdnodes\cup\{\nodesa\}} \cdot Z_{\thirdnodes\cup\{\nodesb\}}
        }
        \right)^{(-1)^{\cardof{\secnodes}-\cardof{\thirdnodes}}} \, .
    \end{align}
    From the independence assumption it follows that for any index $\catindex$
    \begin{align*}
        & \normalizationofwrt{\hypercore}{
            \indexedcatvariableof{\nodesa}
        }{\indexedcatvariableof{\nodes/\thirdnodes\cup\{\nodesa,\nodesb\}},\catvariableof{\thirdnodes}=\seccatindexof{\thirdnodes},  \indexedcatvariableof{\nodesb} }
        \\
        & \quad =
        \normalizationofwrt{\hypercore}{
            \indexedcatvariableof{\nodesa}
        }{\indexedcatvariableof{\nodes/\thirdnodes\cup\{\nodesa,\nodesb\}}, \catvariableof{\thirdnodes}=\seccatindexof{\thirdnodes}} \\
        & \quad  =
        \normalizationofwrt{\hypercore}{
            \indexedcatvariableof{\nodesa}
        }{\indexedcatvariableof{\nodes/\thirdnodes\cup\{\nodesa,\nodesb\}},\catvariableof{\thirdnodes}=\seccatindexof{\thirdnodes},  \catvariableof{\nodesb} = \seccatindexof{\nodesb}}
    \end{align*}
    Applying this in each squares bracket term of \eqref{eq:indContFacProof} we get
    \begin{align*}
        \frac{
            Z_{\thirdnodes}
        }{
            Z_{\thirdnodes\cup\{\nodesa\}}
        }
        & =
        \frac{
            \normalizationofwrt{\hypercore}{
                \indexedcatvariableof{\nodesa}
            }{\indexedcatvariableof{\nodes/\thirdnodes\cup\{\nodesa,\nodesb\}}, \catvariableof{\thirdnodes}=\seccatindexof{\thirdnodes}, \indexedcatvariableof{\nodesb} }
        }{
            \normalizationofwrt{\hypercore}{
                \catvariableof{\nodesa} =\seccatindexof{\nodesa}
            }{\indexedcatvariableof{\nodes/\thirdnodes\cup\{\nodesa,\nodesb\}} , \catvariableof{\thirdnodes}=\seccatindexof{\thirdnodes}, \indexedcatvariableof{\nodesb}}
        } \\
        & =
        \frac{
            \normalizationofwrt{\hypercore}{
                \indexedcatvariableof{\nodesa}
            }{\indexedcatvariableof{\nodes/\thirdnodes\cup\{\nodesa,\nodesb\}}, \catvariableof{\thirdnodes}=\seccatindexof{\thirdnodes}, \catvariableof{\nodesb} = \seccatindexof{\nodesb}}
        }{
            \normalizationofwrt{\hypercore}{
                \catvariableof{\nodesa} =\seccatindexof{\nodesa}
            }{\indexedcatvariableof{\nodes/\thirdnodes\cup\{\nodesa,\nodesb\}}, \catvariableof{\thirdnodes}=\seccatindexof{\thirdnodes},\catvariableof{\nodesb} = \seccatindexof{\nodesb}}
        } \\
        & =
        \frac{
            Z_{\thirdnodes\cup\{\nodesb\}}
        }{
            Z_{\thirdnodes\cup\{\nodesa,\nodesb\}}
        } \, .
    \end{align*}
    Thus, each factor in \eqref{eq:indContFacProof} is trivial, which establishes the claim.
\end{proof}

We are finally ready to proof the Hammersley-Clifford \theref{the:condIndMN} based on the Lemmata above.

%\begin{theorem}[\theref{the:condIndMN}]
%	Let $\probat{\catvariableof{\nodes}}$ be a probability distribution and $\graph$ a clique-capturing hypergraph, such that for $\nodesa$, $\nodesb$, $\nodesc$ we have that $\catvariableof{\nodesa}$ is independent of $\catvariableof{\nodesb}$ conditioned on $\catvariableof{\nodesc}$, when $\nodesc$ separates $\nodesa$ and $\nodesb$ in the hypergraph.
%	Then there is a Markov Network on $\graph$, which distribution is equal to $\probat{\catvariableof{\nodes}}$.
%\end{theorem}
\begin{proof}[Proof of \theref{the:condIndMN}]
    By \lemref{the:contractionFactorization} we have for any index $\catindexof{\nodes}$
    \begin{align*}
        \probat{\indexedcatvariableof{\nodes}} =
        \prod_{\secnodes\subset\nodes} \prod_{\thirdnodes\subset\secnodes}
        \left(
        \probat{\indexedcatvariableof{\thirdnodes},\catvariableof{\nodes/\thirdnodes}=\seccatindexof{\nodes/\thirdnodes}}
        %	\contractionof{\extnet\cup\{\onehotmapof{\catindexof{\nodes/\thirdnodes}}\}}{\catvariableof{\thirdnodes}}
        \right)^{(-1)^{\cardof{\secnodes}-\cardof{\thirdnodes}}}
    \end{align*}
    For any subset $\secnodes\subset\nodes$, which is not contained in a hyperedge, we find $\nodesa,\nodesb \in\secnodes$ such that $\catvariableof{\nodesa}$ is independendent on $\catvariableof{\nodesb}$ conditioned on $\catvariableof{\secnodes/\{\nodesa,\nodesb\}}$.
    If no such nodes $\nodesa,\nodesb \in\secnodes$ exists, $\secnodes$ would be contained in a hyperedge, since the hypergraph is assumed to be clique-capturing.
    By \lemref{lem:independentContractionFactorization} we then have
    \begin{align*}
        \prod_{\thirdnodes\subset\secnodes}
        \left(
        \probat{\indexedcatvariableof{\thirdnodes},\catvariableof{\nodes/\thirdnodes}=\seccatindexof{\nodes/\thirdnodes}}
        \right)^{(-1)^{\cardof{\secnodes}-\cardof{\thirdnodes}}} = 1 \, .
    \end{align*}
    We label by a function
    \begin{align*}
        \alpha: \{\secnodes : \exists\edge\in\edges: \secnodes \subset \edge \} \rightarrow \edges
    \end{align*}
    the remaining node subsets by a hyperedge containing the subset.
    We build the tensor
    \begin{align*}
        \hypercoreofat{\edge}{\catvariableof{\edge}} = \prod_{\secnodes \, : \, \alpha(\secnodes) = \edge} \prod_{\thirdnodes\subset\secnodes}
        \left(
        \probat{\indexedcatvariableof{\thirdnodes},\catvariableof{\nodes/\thirdnodes}=\seccatindexof{\nodes/\thirdnodes}}
        \right)^{(-1)^{\cardof{\secnodes}-\cardof{\thirdnodes}}} \, .
    \end{align*}
    and get, that
    \begin{align*}
        \probat{\catvariableof{\nodes}} & = \contractionof{\extnetasset}{\catvariableof{\nodes}} \\
        & = \normalizationof{\extnetasset}{\catvariableof{\node}} \, .
    \end{align*}
    We have thus constructed a Markov Network with trivial partition function, which contraction coincides with the probability distribution.
\end{proof}

\sect{Differentiation of Contraction}

The structured mean field approaches discussed in \charef{cha:probReasoning} used differentiations of the parametrized tensor networks.
Let us now develop in more detail, how the contraction of tensor networks with variable cores is differentiated.
We capture in additional variables $\seccatvariable$ selecting the coordinates of a tensor, which are varied in a differentiation.

\begin{lemma}
    \label{lem:difMNprob}
    For any tensor network $\extnet$ with positive $\hypercoreof{\edge}$ we have
    \begin{align*}
        \difwrt{\hypercoreofat{\edge}{\seccatvariableof{\edge}}} \extnetdist
        & = \contractionof{
            \identityat{\seccatvariableof{\edge},\edgevariables},
            \frac{\contractionof{\extnet}{\edgevariables}}{\hypercoreofat{\edge}{\edgevariables}},
            \normalizationofwrt{\extnet}{\catvariableof{\nodes/\edge}}{\edgevariables} }{\seccatvariableof{\edge},\nodevariables} \\
        & \quad -  \extnetdist \otimes \contractionof{\frac{\contractionof{\extnet}{\seccatvariableof{\edge}}}{\hypercoreofat{\edge}{\seccatvariableof{\edge}}}
        }{\seccatvariableof{\edge}} \, .
    \end{align*}
\end{lemma}
\begin{proof}
    By multilinearity of tensor network contractions we have
    \begin{align*}
        \difwrt{\hypercoreofat{\edge}{\seccatvariableof{\edge}}} \contractionof{\extnet}{\nodevariables}
        & = \contractionof{\{\identityat{\seccatvariableof{\edge},\edgevariables}\}\cup\{\hypercoreofat{\secedge}{\catvariableof{\secedge}} \, : \, \secedge\neq\edge \}}{\seccatvariableof{\edge},\nodevariables}
    \end{align*}
    and thus
    \begin{align*}
        \difwrt{\hypercoreofat{\edge}{\seccatvariableof{\edge}}} \contraction{\extnet}
        & = \contractionof{\{\identityat{\seccatvariableof{\edge},\edgevariables}\}\cup\{\hypercoreofat{\secedge}{\catvariableof{\secedge}} \, : \, \secedge\neq\edge \}}{\seccatvariableof{\edge}} \, .
    \end{align*}

    Using both we get
    \begin{align}
        \difwrt{\hypercoreofat{\edge}{\seccatvariableof{\edge}}} \extnetdist
        & = \difwrt{\hypercoreofat{\edge}{\seccatvariableof{\edge}}}  \frac{\contractionof{\extnet}{\nodevariables}}{\contraction{\extnet}} \nonumber \\
        & = \frac{ \difwrt{\hypercoreofat{\edge}{\seccatvariableof{\edge}}} \contractionof{\extnet}{\nodevariables}}{\contraction{\extnet}}
        - \frac{ \contractionof{\extnet}{\nodevariables} \difwrt{\hypercoreofat{\edge}{\seccatvariableof{\edge}}} \contraction{\extnet} }{(\contraction{\extnet})^2} \nonumber \\
        & = \frac{ \contractionof{\{\identityat{\seccatvariableof{\edge},\edgevariables}\}\cup\{\hypercoreofat{\secedge}{\catvariableof{\secedge}} \, : \, \secedge\neq\edge \}}{\seccatvariableof{\edge},\nodevariables}}{\contraction{\extnet}} \nonumber \\
        & \quad\quad - \extnetdist \cdot  \frac{\contractionof{\{\identityat{\seccatvariableof{\edge},\edgevariables}\}\cup\{\hypercoreofat{\secedge}{\catvariableof{\secedge}} \, : \, \secedge\neq\edge \}}{\seccatvariableof{\edge}}}{\contraction{\extnet}} \label{eq:differentiatingMNpreresult}
        % = \contractionof{\{\identityat{\seccatvariableof{\edge},\edgevariables}\}\cup\{\hypercoreofat{\secedge}{\catvariableof{\secedge}} \, : \, \secedge\neq\edge \}}{\seccatvariableof{\edge},\nodevariables}
    \end{align}

    For the first term we get with a normalization equation (see \theref{the:normalizationContractionEQ}) that
    \begin{align*}
        \frac{ \contractionof{\{\identityat{\seccatvariableof{\edge},\edgevariables}\}\cup\{\hypercoreofat{\secedge}{\catvariableof{\secedge}} \, : \, \secedge\neq\edge \}}{\seccatvariableof{\edge},\nodevariables}}{\contraction{\extnet}}
        &= \frac{\contractionof{\{\identityat{\seccatvariableof{\edge},\edgevariables}\}\cup\{\hypercoreofat{\secedge}{\catvariableof{\secedge}} \, : \, \secedge\in\edges \}}{\seccatvariableof{\edge},\nodevariables}}{\hypercoreofat{\edge}{\edgevariables}  \cdot \contraction{\extnet}} \\
        &= \frac{
            \contractionof{\identityat{\seccatvariableof{\edge},\edgevariables},\extnetdist}{\seccatvariableof{\edge},\nodevariables}
        }{\hypercoreofat{\edge}{\edgevariables}}  \\
        &= \frac{\contractionof{\identityat{\seccatvariableof{\edge},\edgevariables},
            \normalizationof{\extnet}{\edgevariables},
            \normalizationofwrt{\extnet}{\catvariableof{\nodes/\edge}}{\edgevariables}
        }{\seccatvariableof{\edge},\nodevariables}
        }{\hypercoreofat{\edge}{\edgevariables}}  \, .
    \end{align*}

    Analogously, we have
    \begin{align*}
        \frac{ \contractionof{\{\identityat{\seccatvariableof{\edge},\edgevariables}\}\cup\{\hypercoreofat{\secedge}{\catvariableof{\secedge}} \, : \, \secedge\neq\edge \}}{\seccatvariableof{\edge}}}{\contraction{\extnet}}
        &= \frac{\contractionof{\identityat{\seccatvariableof{\edge},\edgevariables},
            \normalizationof{\extnet}{\edgevariables}%,
        %\normalizationofwrt{\extnet}{\catvariableof{\nodes/\edge}}{\edgevariables}
        }{\seccatvariableof{\edge}}
        }{\hypercoreofat{\edge}{\edgevariables}}  \, .
    \end{align*}

    With \eqref{eq:differentiatingMNpreresult}, we arrive at the claim
    \begin{align*}
        \difwrt{\hypercoreofat{\edge}{\seccatvariableof{\edge}}} \extnetdist
        & = \contractionof{
            \identityat{\seccatvariableof{\edge},\edgevariables},
            \frac{\contractionof{\extnet}{\edgevariables}}{\hypercoreofat{\edge}{\edgevariables}},
            \normalizationofwrt{\extnet}{\catvariableof{\nodes/\edge}}{\edgevariables} }{\seccatvariableof{\edge},\nodevariables} \\
        & \quad -  \extnetdist \otimes \contractionof{\frac{\contractionof{\extnet}{\seccatvariableof{\edge}}}{\hypercoreofat{\edge}{\seccatvariableof{\edge}}}
        }{\seccatvariableof{\edge}} \, . \qedhere
    \end{align*}
\end{proof}


% Could put it into contraction equations?
\begin{lemma}
    \label{lem:difMNExpectation}
    %See Proposition 11.9 in Koller Book.
    For any function $\exfunction(\hypercoreof{\edge})[\nodevariables]$ we have
    \begin{align*}
        \difwrt{\hypercoreofat{\edge}{\seccatvariableof{\edge}}} &
        \contraction{\extnetdist,\exfunction(\hypercoreof{\edge})[\nodevariables]} \\
        = &
        \frac{\normalizationof{\extnet}{\indexedcatvariableof{\edge}}}{\hypercoreofat{\edge}{\indexedcatvariableof{\edge}}}
        \Big( \contraction{\normalizationofwrt{\extnet}{\catvariableof{\nodes/\edge}}{\indexedcatvariableof{\edge}}, \exfunction(\hypercoreof{\edge})[\nodevariables,\seccatvariableof{\edge}]} \\
        & \quad \quad \quad \quad \quad - \contraction{\extnetdist, \exfunction(\hypercoreof{\edge})[\nodevariables]}
        \Big) \\
        & + \contraction{ \extnetdist
        \difofwrt{\exfunction(\hypercoreof{\edge})[\nodevariables]}{\hypercoreofat{\edge}{\seccatvariableof{\edge}}}
        }
    \end{align*}
\end{lemma}
\begin{proof}
    By product rule of differentiation we have
    \begin{align*}
        \difwrt{\hypercoreofat{\edge}{\indexedcatvariableof{\edge}}} \contraction{\extnetdist,\exfunction(\hypercoreof{\edge})[\nodevariables]}
        & =  \contraction{\difwrt{\hypercoreofat{\edge}{\indexedcatvariableof{\edge}}}\extnetdist,\exfunction(\hypercoreof{\edge})[\nodevariables]} \\
        & \quad +  \contraction{\extnetdist,\difwrt{\hypercoreofat{\edge}{\indexedcatvariableof{\edge}}}\exfunction(\hypercoreof{\edge})[\nodevariables]}  \, .
    \end{align*}
    The claim now follows with the application of \lemref{lem:difMNprob} on the first term.
\end{proof}

%\sect{Discussion}
%Representations of linear maps is the typical application of tensors, reason for referring to tensor networks as multilinear algebra.


% Properties of Calculus
\section{Directed Tensor Calculus}\label{cha:directedTC}



We in this chapter investigate the properties of tensors, which where arising in probabilistic and logical reasoning.
We observed already before, that
\begin{itemize}
	\item Conditional probability tensors are directed tensors.
	\item Logical formulas are binary tensors.
\end{itemize}

Thus, the set of tensors, which are both directed and binary is of much interest. 
We will show, that they are equal to the set of relational encodings of functions.




\subsection{Directed Tensors}

Directionality as defined in Definition~\ref{def:directedTensor} represents the constraints on the structure of tensors:
Summing over outgoing trivializes the tensor.
%We will use this property to find efficient algorithms.


\begin{definition}[Directed Hypergraph]
	A directed hyperedge is a hyperedge, which node set is split into disjoint sets of incoming and outgoing nodes.
	We say a hypercore $\hypercoreof{\edge}$ decorating a directed hyperedge respects the direction, when it is a conditional probability tensor with respect to the direction of the hyperedge. 
	The hypergraph is acyclic, when there is no nonempty cycle of node tuples $(\node_1,\node_2)$, such that $\node_1$ is an incoming node and $\node_2$ an outgoing node of the same hyperedge.
\end{definition}

%\begin{definition}
%	A directed Tensor Network is a directed hypergraph, which is decorated by tensors respecting the directionality of the respective edges.
%\end{definition}


% Multiple Directions possible
There can be multiple ways to direct a tensor, which an extreme example being the Dirac Delta Tensors.
Further example are relational encodings of invertible functions.

\begin{example}[Dirac Delta Tensors]
	Given a node set $\edge$ colored with a constant dimension $\catdim$ Diracs Delta Tensors are the tensors
		\[ \dirdeltaof{\edge,\catdim} \in \bigotimes_{\node\in\edge} \rr^{\catdim} \]
	with coordinates
	\begin{align}
		\dirdeltaof{\edge,\catdim}_{\catindices} = 
		\begin{cases}
			1 \quad & \text{if} \quad \catindexof{0} = \ldots = \catindexof{\atomorder-1} \\
			0 & \text{else}
		\end{cases} \, . 
	\end{align}
	The contractions with respect to subsets $\secnodes \subset \edge$ are
	\begin{align}
		\contractionof{\{\dirdeltaof{\edge,\catdim}\}}{\catvariableof{\secnodes}} = 
		\begin{cases}
			\catdim & \text{if} \quad \nodes = \varnothing \\ 
			\ones & \text{if} \quad \cardof{\secnodes} = 1\\ 
			\dirdeltaof{\secnodes,\catdim} & \text{else}
		\end{cases} \, .
	\end{align}
	Thus are directed for any orientation of the respective edge with exactly one incoming variable.
\end{example}




% TRUE?
%% Hypernetworks
% We can use Diracs Delta Tensors to represent a tensor network on a hypergraph by a tensor network on a graph (that is edges contain at most two nodes).


\begin{lemma}\label{lem:deltification}
%	Let $\graph$ be a directed hypergraph, such that each node is at most in one hyperedge appearing in the outgoing nodes.
	Let $\graph=(\nodes,\edges)$ be a hypergraph and $\extnet$ a tensor network on $\graph$.
	We build a graph $\secgraph=(\secnodes,\secedges\cup\Delta^{\graph})$ and a tensor network $\tnetof{\secgraph}$ by % ! See Bethe Cluster Graph definition !
	\begin{itemize}
		\item Recolored Edges $\secedges = \{\tilde{\edge} \, : \, \edge\in \edges\}$ where $\tilde{\edge} = \{\node^{\edge} \, : \, \node\in\edge\}$, which decoration tensor $\hypercoreof{\tilde{\edge}}$ has same coordinates as $\hypercoreof{\edge}$
		\item Nodes $\secnodes = \bigcup_{\edge\in\edges}\tilde{\edge}$ %$\secnodes = \bigcup_{\edge\in\edges}\{\node^{\edge} \, : \, \node\in\edge \}$ 
		\item Delta Edges $\Delta^{\graph} =  \big\{ \{\node\} \cup \{\node^{\edge} \, : \, \edge\ni\node \} \, : \, \node\in\nodes \big\} $ each of which decorated by a delta tensor $\delta^{\{\node^{\edge} \, : \, \edge\ni\node \}}$
	\end{itemize}
	 Then we have
	 	\[ \contractionof{\extnet}{\catvariableof{\nodes}} =  \contractionof{\tnetof{\secgraph}}{\catvariableof{\nodes}}  \, . \]
\end{lemma}
\begin{proof}
	For any $\catindexof{\nodes}$ we have 
	\begin{align*}
		 \contractionof{\tnetof{\secgraph}}{\indexedcatvariableof{\nodes}} 
		 & = \contractionof{\{\hypercoreof{\tilde{\edge}}[\catvariableof{\{\node^{\edge} : \node\in\edge\}}] : \edge \in \edges \}\cup 
		 	\{\delta^{\{\node\} \cup \{\node^{\edge} \, : \, \edge\ni\node \}}[\catvariableof{\{\node^{\edge} : \edge\ni\node \}}, \indexedcatvariableof{\node} ]  : \node\in\nodes \}
		 }{\varnothing} \\
		 & =  \contractionof{\{\hypercoreof{\tilde{\edge}}[\catvariableof{\{\node^{\edge} : \node\in\edge\}} = \catindexof{\{\node : \node\in\edge\}} ] : \edge \in \edges \}
		 }{\varnothing} \\
		 & = \contractionof{\extnet}{\indexedcatvariableof{\nodes}} \, ,
	\end{align*}
	which establishes the claim.
\end{proof}



\subsubsection{Normation}


Normed tensors (see Definition~\ref{def:normation}) are directed and directed tensors invariant under normation wrt their incoming and outgoing variable, as we show next.

\begin{theorem}\label{the:normationDirected}
	For any tensor network $\extnet$ on variables $\nodes$ that can be normed with respect to $\innodes$ and $\outnodes$, the normation is directed with $\innodes$ incoming and $\outnodes$ outgoing.
\end{theorem}
\begin{proof}
	We have for any incoming state ${\atomlegindexof{\innodes}\in\bigtimes_{\node\in\innodes}\catdimof{\node}}$ that
	\begin{align*}
		\contractionof{\{\normationofwrt{\extnet}{\innodes}{\outnodes}, \onehotmapof{\atomlegindexof{\innodes}} \}}{\varnothing} 
		& =  \frac{
		\contractionof{\extnet\cup\{\onehotmapof{\atomlegindexof{\innodes}}\}}{\varnothing}
		}{
		\contractionof{\extnet\cup\{\onehotmapof{\atomlegindexof{\innodes}}\}}{\varnothing}
		} \, .
	\end{align*}
	By Definition~\ref{def:directedTensor}, $\normationofwrt{\extnet}{\outnodes}{\innodes}$ is thus directed.
%
%	We have for any basis tensor $\onehotmapof{\atomlegindexof{\secnodes}}$
%	\begin{align}
%		\contractionof{
%			\{ \normationof{\hypercore}{\secnodes}, \onehotmapof{\atomlegindexof{\secnodes}} \}
%		}{
%			\varnothing
%		}
%		= \frac{
%			\contractionof{\{\hypercore, \onehotmapof{\atomlegindexof{\secnodes}} \}}{\varnothing}
%		}{
%			\contractionof{\{\hypercore, \onehotmapof{\atomlegindexof{\secnodes}} \}}{\varnothing}
%		} 
%		= 1 \, . 
%	\end{align}
%	Thus 
%	\begin{align}
%		\contractionof{
%			\{ \normationof{\hypercore}{\secnodes} \}
%		}{
%			\secnodes 
%		}
%		= \ones
%	\end{align}
%	and by Definition~\ref{def:directedTensor}, $\normationof{\hypercore}{\secnodes}$ is directed with the variables $\secnodes$ incoming.
\end{proof}


The normation operation coincides in cases of non-negative tensors with the conditioning of a Markov Network representing a probability distribution.


\subsubsection{Contraction of Directed Tensors}

Let us now investigate, which contractions inherit the directionality of the tensors.

%Next we state that specific contraction of conditional probability tensors are still conditional probability tensors.

%\red{Can be extended to single outgoing legs, by using delta tensors at hyperedges.}

%\begin{theorem}
%	Given a directed acyclic hypergraph, which hyperdedges are decorated by tensor cores respecting the direction.
%	Then the contractions, where all closed nodes appear exactly once as an incoming node and exactly once as an outgoing node, and where all open nodes appear in single hyperedges, are conditional probability tensors
%\end{theorem}
%\begin{proof}
%	It is enough to show this property on the contraction of two hypercores.
%	Since the hypergraph is acyclic, the coinciding nodes are all outgoing on the one and incoming to the other hyperedge.
%	Let the hyperedge with the incoming nodes $\edge_1$ and the one with the outgoing nodes $\edge_2$
%	We need to show that when further contracting the contraction with trivial tensors on the outgoing and basis tensors on the incoming legs we get $1$.
%	For any $\catindex$ and $\seccatindex$ this holds since

%	Here we used in the first equality, that $\hypercoreof{\edge_1}$ is a conditional probability tensor and in the second the same property for $\hypercoreof{\edge_2}$.
%\end{proof}

% Hadamard does not preserve probabilities
%We need to ass as assumption in Theorem~\ref{the:conditionalContractionPreservation}, that each node is to at most one hyperedge and to at most one hyperedge outgoing.
%This is due to the failure of Hadamard products of probability tensors to be probability tensors themself.


%\begin{lemma}\label{lem:twoDirectedContracted}
%	Given two 
%\end{lemma}





\begin{theorem}\label{the:conditionalContractionPreservation}
	Let $\graph=(\nodes,\edges)$ be a directed acyclic hypergraph, such that each node $\node\in\edge$ appears at most in one hyperedge as an outgoing variable and denote $\innodes$ as those nodes, which do not appear as outgoing variables.
	For any tensor network $\extnet$ respecting the direction of $\graph$ we have that
		\[ \contractionof{\extnet}{\catvariableof{\innodes}} = \onesat{\catvariableof{\innodes}} \, , \]
	that is $\contractionof{\extnet}{\catvariableof{\nodes}}$ is a directed tensor with $\innodes$ incoming and $\nodes/\innodes$ outgoing.
\end{theorem}
\begin{proof}
	We show the theorem only for the case of hypergraphs, where variables are appearing at most in two hyperedges.
	If a hypergraph fails to satisfy this assumption, we apply Lemma~\ref{lem:deltification} and add delta tensors copying the variables, which are appearing in multiple tensors, and arrive at a tensor network with nodes appearing in at most two hyperedges.
	When orienting each edge 
	
	% Approach: Contracting with ones
	We show the theorem over induction on the number $n$ of cores.
	\paragraph{$n=1$:} It holds trivially, when $\extnet$ consists of a single core
	\paragraph{$n\rightarrow n-1$:} Let us assume, that the claim holds for graphs with $n-1$ hyperedges and let $\tnetof{\graph}$ be a tensor network with $n$ hyperedges.
	Since the hypergraph is acyclic, we find an edge $\edge\in\edges$ such that all outgoing nodes of $\edge$ are not appearing as an incoming node in any edge. 
	We then apply Theorem~\ref{the:splittingContractions} and get
	\begin{align*}
		\contractionof{\extnet}{\catvariableof{\innodes}} 
		&= \contractionof{
			\tnetof{(\nodes,\edges/\{\edge\})} \cup \{\hypercoreofat{\edge}{\catvariableof{\incomingnodes},\catvariableof{\outgoingnodes}}\}
			}{\catvariableof{\innodes}} \\
		& = \contractionof{
			\tnetof{(\nodes,\edges/\{\edge\})} \cup \{\sbcontractionof{\hypercoreof{\edge}}{\catvariableof{\incomingnodes}} \}
			}{\catvariableof{\innodes}} \\
		& = \contractionof{
			\tnetof{(\nodes,\edges/\{\edge\})} \cup \{\onesat{\catvariableof{\incomingnodes}} \}
			}{\catvariableof{\innodes}} \\
		& \contractionof{
			\tnetof{(\nodes,\edges/\{\edge\})} \}
			}{\catvariableof{\innodes}} \, . 
	\end{align*}
	We then notice that the hypergraph $(\nodes,\edges/\{\edge\})$ has $n-1$ hyperedges and each node appears at most once as an incoming and at most once as an outgoing node.
	Thus, we apply the assumption of the induction and get
	\begin{align*}
		\contractionof{\extnet}{\catvariableof{\innodes}} = \contractionof{
			\tnetof{(\nodes,\edges/\{\edge\})} \}
			}{\catvariableof{\innodes}} = \onesat{\catvariableof{\innodes}} \, . 
	\end{align*}
	
%	% ALTERNATIVE APPROACH	
%	We then iteratively choose two neighbored tensors and replace them by their contraction, which is by Lemma~\ref{lem:twoDirectedContracted} itself directed and not changing the contraction by  Theorem~\ref{the:splittingContractions}.
%	When no such neighbored tensors are left, we have an tensor product of directed tensors, which are not sharing variables.
%	This outer product is trivially directed.
\end{proof}


% Overwork!
Schematically, the direction conservation property argument is depicted as:
\begin{center}
	\begin{tikzpicture}[scale=0.3, thick] % , baseline = -3.5pt


\draw[dashed] (-22,1) rectangle (-14,3); 
\node[anchor=center] (text) at (-18,2) {\small $\ones$};

\draw[->]  (-21,-1)--(-21,1);% node[midway,left] {\tiny $\node_1$}; 
\draw[->]  (-15,-1)--(-15,1);% node[midway,left] {\tiny $\cdots \quad \edge^{out}_1$}; 
\node[anchor=center] (text) at (-18,0) {\small $\cdots$};


\draw (-18,0) ellipse (4 and 0.5);
\node[anchor=center] (text) at (-23,0) {\tiny $\edge^{out}_1$};

\draw (-22,-1) rectangle (-14,-3);
\node[anchor=center] (text) at (-18,-2) {\small $\hypercoreof{\edge_1}$};
\draw[<-]  (-21,-3)--(-21,-5);% node[midway,left]% {\tiny $\node_1$}; 
\draw[<-]  (-19,-3)--(-19,-5) node[midway,left] {\tiny $\cdots$}; 

\draw[<-]  (-17,-3)--(-17,-7);% node[midway,left] {\tiny $\exrandom$}; 
\draw[<-]  (-15,-3)--(-15,-5) node[midway,left] {\tiny $\cdots$};% node[midway,left] {\tiny $\exrandom$}; 
\draw (-15,-5) -- (-15,-7);

\draw (-18,-4) ellipse (4 and 0.5);
\node[anchor=center] (text) at (-23,-4) {\tiny $\edge^{in}_1$};


\draw[dashed] (-18.5,-5) rectangle (-22,-7); 
\node[anchor=center] (text) at (-20.25,-6) {\small $\onehotmapof{\catindex}$};

\draw (-13.5,-6) ellipse (4 and 0.5);
\node[anchor=center] (text) at (-8.5,-6) {\tiny $\edge^{out}_2$};

\draw[dashed] (-10,-5) rectangle (-13,-3); 
\node[anchor=center] (text) at (-11.5,-4) {\small $\ones$};

\draw[<-]  (-10.5,-5)--(-10.5,-7) node[midway,left] {\tiny $\cdots$}; 
\draw[<-]  (-12.5,-5)--(-12.5,-7);

\draw (-17.5,-7) rectangle (-9.5,-9);
\node[anchor=center] (text) at (-13.5,-8) {\small $\hypercoreof{\edge_2}$};
\draw[->]  (-11.5,-11)--(-11.5,-9) node[midway,left] {\tiny $\cdots$}; 
\draw[->]  (-14.5,-11)--(-14.5,-9);

\draw (-13,-10) ellipse (2 and 0.5);
\node[anchor=center] (text) at (-16,-10) {\tiny $\edge^{in}_2$};

\draw[dashed] (-15.5,-11) rectangle (-10.5,-13); 
\node[anchor=center] (text) at (-13,-12) {\small $\onehotmapof{\seccatindex}$};


\node[anchor=center] (text) at (-6,-2) {${=}$};

\node[anchor=center] (text) at (13,-2) {${=} \quad\quad 1 \, \, . $};



\begin{scope}[shift={(21,0)}]

\draw[dashed] (-22,1) rectangle (-14,3); 
\node[anchor=center] (text) at (-18,2) {\small $\ones$};

\draw[->]  (-21,-1)--(-21,1);% node[midway,left] {\tiny $\node_1$}; 
\draw[->]  (-15,-1)--(-15,1);% node[midway,left] {\tiny $\cdots \quad \edge^{out}_1$}; 
\node[anchor=center] (text) at (-18,0) {\small $\cdots$};


\draw (-18,0) ellipse (4 and 0.5);
\node[anchor=center] (text) at (-23,0) {\tiny $\edge^{out}_1$};

\draw (-22,-1) rectangle (-14,-3);
\node[anchor=center] (text) at (-18,-2) {\small $\hypercoreof{\edge_1}$};
\draw[<-]  (-21,-3)--(-21,-5);% node[midway,left]% {\tiny $\node_1$}; 
\draw[<-]  (-19,-3)--(-19,-5) node[midway,left] {\tiny $\cdots$}; 

\draw[<-]  (-17,-3)--(-17,-5);% node[midway,left] {\tiny $\exrandom$}; 
\draw[<-]  (-15,-3)--(-15,-5) node[midway,left] {\tiny $\cdots$};% node[midway,left] {\tiny $\exrandom$}; 


\draw (-18,-4) ellipse (4 and 0.5);
\node[anchor=center] (text) at (-23,-4) {\tiny $\edge^{in}_1$};


\draw[dashed] (-14,-5) rectangle (-17.5,-7); 
\node[anchor=center] (text) at (-15.75,-6) {\small $\tilde{\probtensor}$};


\draw[dashed] (-18.5,-5) rectangle (-22,-7); 
\node[anchor=center] (text) at (-20.25,-6) {\small $\onehotmapof{\catindex}$};


\end{scope}





\end{tikzpicture}
\end{center}



















%\input{PartIII/binary_tensor_calculus.tex}
\section{Contraction Equations}

We have observed, that many concepts and theorems in probability theory and logics can be understood as contraction equations.
We first provide a summary of the used contraction equations.

\subsection{Contraction equations in logical and probabilistic reasoning}

Let us summarize the application of contractions and normation in the definition of
\begin{itemize}
	\item Marginal probabilities (\defref{def:marginalProbability}, \theref{the:marginalContraction})
		\[ \probof{\exrandom} = \sbcontractionof{\probtensor}{\exrandom} \]
	\item Conditional probabilities (\defref{def:conditionalProbability}, \theref{the:conditionalContraction})
		\[ \condprobof{\exrandom}{\secexrandom} = \sbnormationofwrt{\probtensor}{\exrandom}{\secexrandom} \]
	\item The partition function of a Markov Networks 
		\begin{align*}
			\partitionfunctionof{\extnet} = \contractionof{\extnet}{\varnothing}
		\end{align*}
	\item The probability distribution of a Markov Network is (\defref{def:markovNetwork})
		\begin{align*}
			\probtensor^{\extnet} = \normationofwrt{\extnet}{\nodes}{\varnothing}
		\end{align*}
\end{itemize}


Further the following properties have been defined by contraction equations:
\begin{itemize}
	\item $\exrandom$ and $\secexrandom$ are independent when (\defref{def:independence}, \theref{the:independenceProductCriterion})
		\[  \sbcontractionof{\probtensor}{\exrandom,\secexrandom} 
		=  \sbcontractionof{\probtensor}{\exrandom} 
			\otimes  \sbcontractionof{\probtensor}{\secexrandom} \]
	\item $\exrandom$ and $\secexrandom$ are called independent conditioned on $\thirdexrandom$ when (\defref{def:condIndependence}, \theref{the:condIndependenceProductCriterion})
		\[ \sbnormationofwrt{\probtensor}{\exrandom,\secexrandom}{\thirdexrandom} 
		= \sbnormationofwrt{\probtensor}{\exrandom}{\thirdexrandom} 
		\otimes \sbnormationofwrt{\probtensor}{\secexrandom}{\thirdexrandom} \]
\end{itemize}





\subsection{Normation Equations}

\begin{theorem}[Normation as a Contraction Equation]\label{the:normationContractionEQ}
	For any on $\innodes$ normable tensor $\hypercoreat{\catvariableof{\nodes}}$, where $\innodes\dot{\cup}\outnodes=\nodes$, we have
	\begin{align*}
		\sbcontractionof{\hypercore}{\catvariableof{\nodes}} 
		= \sbcontractionof{\sbnormationofwrt{\hypercore}{\catvariableof{\outnodes}}{\catvariableof{\innodes}},\sbcontractionof{\hypercore}{\catvariableof{\innodes}}}{\catvariableof{\nodes}} \, . 
	\end{align*}
\end{theorem}
\begin{proof}
	Let us choose indices $\catindexof{\innodes}$ and $\catindexof{\outnodes}$.
	We have that
	\begin{align*}
		%\sbcontractionof{
		\sbnormationofwrt{\hypercore}{\indexedcatvariableof{\innodes}}{\indexedcatvariableof{\outnodes}}
		%}{\indexedcatvariableof{\innodes},\indexedcatvariableof{\outnodes}} 
		= \frac{
			\sbcontractionof{\hypercore}{\indexedcatvariableof{\innodes},\indexedcatvariableof{\outnodes}} 	
		}{
			\sbcontractionof{\hypercore}{\indexedcatvariableof{\innodes}} 	
		} 
	\end{align*}
	and therefor
	\begin{align*}
		\sbcontractionof{\hypercore}{\indexedcatvariableof{\innodes},\indexedcatvariableof{\outnodes}} = 
		\sbnormationofwrt{\hypercore}{\indexedcatvariableof{\innodes}}{\indexedcatvariableof{\outnodes}}
		\cdot 
		\sbcontractionof{\hypercore}{\indexedcatvariableof{\innodes}} 	
	\end{align*}
	Since the equation holds for arbitrary indices, the claim is established.
\end{proof}


\begin{theorem}[Generic Chain Rule]\label{the:genericChainRule}
	For any Tensor $\hypercoreat{\catvariableof{\nodes}}$ and any total order $\prec$ on the nodes $\nodes$ we have % ! CAN DIRECTLY USE [d] when having the order !
	\begin{align*}
		\hypercoreat{\catvariableof{\nodes}} = 
		\contractionof{
			\{ \sbnormationofwrt{\hypercore}{\catvariableof{\node}}{\catvariableof{\prenodes}}  \, : \nodein \}
		}{\catvariableof{\nodes}}
	\end{align*}
\end{theorem}
\begin{proof}
	We apply \theref{the:normationContractionEQ} on the tensor
	\begin{align*}
		\sbnormationofwrt{\hypercore}{
			\catvariableof{\node},\catvariableof{\afternodes}
		}{
			\indexedcatvariableof{\prenodes}
		} \, ,
	\end{align*}
	where $\nodein$ and $\catindexof{\nodes}$ are chosen arbitrarly.
	For any $\nodein$ we get
	\begin{align*}
		\sbnormationofwrt{\hypercore}{
			\catvariableof{\node},\catvariableof{\afternodes}
			}{
			\catvariableof{\prenodes}
		} 
		= \sbcontractionof{
			\normationofwrt{\hypercore}{
				\catvariableof{\afternodes}
				}{
				\catvariableof{\node},\catvariableof{\prenodes}
				},
			\normationofwrt{\hypercore}{
				\catvariableof{\node}
				}{
				\catvariableof{\prenodes}
				}
		}{
			\catvariableof{\nodes} 
		} \, .
	\end{align*}
	Applying this equation iteratively and making use of the commutation of contractions we get for any $\nodein$
	\begin{align*}
		\sbnormationofwrt{\hypercore}{
			\catvariableof{\node},\catvariableof{\afternodes}
		}{
			\catvariableof{\prenodes}
		}
		= \sbcontractionof{
			\normationofwrt{\hypercore}{
				\catvariableof{\secnode}
			}{
				\catvariableof{\{\thirdnode : \thirdnode \prec \secnode, \thirdnode\neq\secnode\}}
			} 
			\, : \node \prec \secnode
		}{
			\catvariableof{\nodes} 
		} \, .
	\end{align*}
	With the maximal node $\node$, that is the $\node$, such that no $\secnode\in\nodes$ with $\node\prec\secnode$ and $\node\neq\secnode$ exists, this is the claim.
\end{proof}




\subsection{Proof of Hammersley-Clifford Theorem}\label{sec:proofHCTheorem}

\red{Different to the standard case, we show a version of Hammersley-Clifford for hypergraphs.}


\begin{lemma}\label{the:contractionFactorization}
	Let $\hypercoreat{\catvariableof{\nodes}}$ be a positive tensor and $\seccatindexof{\nodes}$ an arbitrary index.
	Then we have 
	\begin{align*}
		\hypercoreat{\catvariableof{\nodes}}
		= \sbcontractionof{
			\big(\sbcontractionof{\hypercore}{\catvariableof{\nodes/\thirdnodes}, \catvariableof{\thirdnodes} = \seccatindexof{\thirdnodes}}\big)^{(-1)^{\cardof{\secnodes}-\cardof{\thirdnodes}}} \, : \, \thirdnodes \subset \secnodes \subset \nodes
		}{\catvariableof{\nodes}} \, ,
	\end{align*}
	where the exponentiation is performed coordinatewise and positivity of $\hypercore$ ensures the well-definedness.
\end{lemma}
\begin{proof}
	It suffices to show, that for an arbitrary index $\catindexof{\nodes}$ be an arbitrary index we have
	\begin{align*}
		\hypercoreat{\indexedcatvariableof{\nodes}}
		= \prod_{\secnodes\subset\nodes} \prod_{\thirdnodes\subset\secnodes}
			\big(\sbcontractionof{\hypercore}{\indexedcatvariableof{\nodes/\thirdnodes}, \catvariableof{\thirdnodes} = \seccatindexof{\thirdnodes}}\big)^{(-1)^{\cardof{\secnodes}-\cardof{\thirdnodes}}} \, .
	\end{align*}
	We do this by applying a logarithm on the right hand side and grouping the terms by $\thirdnodes$ as
	\begin{align*}
		%\lnof{\hypercoreat{\indexedcatvariableof{\nodes}}} 
		& \lnof{\prod_{\secnodes\subset\nodes} \prod_{\thirdnodes\subset\secnodes}
			\sbcontractionof{\hypercore}{\indexedcatvariableof{\nodes/\thirdnodes}, \catvariableof{\thirdnodes} = \seccatindexof{\thirdnodes}}\big)^{(-1)^{\cardof{\secnodes}-\cardof{\thirdnodes}}}} \\
		& = \sum_{\thirdnodes\subset\nodes} \lnof{\sbcontractionof{\hypercore}{\indexedcatvariableof{\nodes/\thirdnodes}, \catvariableof{\thirdnodes} = \seccatindexof{\thirdnodes}}} 
		\left( \sum_{\secnodes\subset\nodes \, : \, \thirdnodes\subset \secnodes} (-1)^{\cardof{\secnodes}-\cardof{\thirdnodes}} \right) \\
		& =  \sum_{\thirdnodes\subset\nodes} \lnof{\sbcontractionof{\hypercore}{\indexedcatvariableof{\nodes/\thirdnodes}, \catvariableof{\thirdnodes} = \seccatindexof{\thirdnodes}}} 
		\left( \sum_{i \in [\cardof{\nodes}-\cardof{\thirdnodes}]}  (-1)^{i}  \binom{\cardof{\nodes}-\cardof{\thirdnodes}}{i}  \right) 
	\end{align*}
	Now, by the generic binomial theorem we have that for $n\in\nn, n \neq 0$
		\[ 0 = (1-1)^n = \sum_{i \in [n]}  (-1)^{i}  \binom{n}{i}   \, . \]
	Therefore, the summands for $\thirdnodes\neq\nodes$ vanish and we have 
	\begin{align*}
			& \lnof{ \prod_{\secnodes\subset\nodes} \prod_{\thirdnodes\subset\secnodes}
			\big(\sbcontractionof{\hypercore}{\indexedcatvariableof{\nodes/\thirdnodes}, \catvariableof{\thirdnodes} = \seccatindexof{\thirdnodes}}\big)^{(-1)^{\cardof{\secnodes}-\cardof{\thirdnodes}}} } \\
			& = \lnof{\hypercoreat{\indexedcatvariableof{\nodes}}}
			\left( \sum_{i \in [0]}  (-1)^{i}  \binom{0}{i}  \right) \\
			& = \lnof{\hypercoreat{\indexedcatvariableof{\nodes}}} \, . 
	\end{align*}
	Applying the exponential function on both sides establishes the claim.
\end{proof}


%\begin{lemma}\label{the:condIndMN}
%	Let $\extnet$ be a tensor network of positive cores and $\catindexof{\nodes}$ an arbitrary index.
%	Then we have
%	\begin{align*}
%		\contractionof{\extnet}{\catvariableof{\nodes}} =
%		\prod_{\secnodes\subset\nodes} \prod_{\thirdnodes\subset\secnodes} 
%		\left(\contractionof{\extnet\cup\{\onehotmapof{\catindexof{\nodes/\thirdnodes}}\}}{\catvariableof{\thirdnodes}}\right)^{(-1)^{\cardof{\secnodes}-\cardof{\thirdnodes}}}
%	\end{align*}
%\end{lemma}
%\begin{proof}
%	Show, that any factor $\contractionof{\extnet\cup\{\onehotmapof{\catindexof{\nodes/\thirdnodes}}\}}{\catvariableof{\thirdnodes}}$ with $\thirdnodes\neq\nodes$ is cancelled (by the $(1-1)^n$ binominal theorem).
%\end{proof}


\begin{lemma}\label{lem:independentContractionFactorization}
	Let $\hypercore$ be a positive tensor, $\secnodes\subset\nodes$ and arbitrary subset and $\catindexof{\secnodes}$ an arbitrary index.
	When there are $\nodesa,\nodesb \in\secnodes$, such that
	\begin{align*}
		\sbnormationofwrt{\hypercore}{\catvariableof{\nodesa,\nodesb}}{\catvariableof{\nodes/\{\nodesa,\nodesb\}}}
		= \sbcontractionof{ 
		\sbnormationofwrt{\hypercore}{\catvariableof{\nodesa}}{\catvariableof{\nodes/\{\nodesa,\nodesb\}}},
		\sbnormationofwrt{\hypercore}{\catvariableof{\nodesb}}{\catvariableof{\nodes/\{\nodesa,\nodesb\}}}
		}{\catvariableof{\secnodes}}
	\end{align*}
	then
	\begin{align*}
 \prod_{\thirdnodes\subset\secnodes} 
		\left(\sbcontractionof{\hypercore}{\indexedcatvariableof{\nodes/\thirdnodes}, \catvariableof{\thirdnodes} = \seccatindexof{\thirdnodes}}\right)^{(-1)^{\cardof{\secnodes}-\cardof{\thirdnodes}}} = 1 \, .
	\end{align*}
\end{lemma}
\begin{proof}
	We abbreviate
		\[ Z_{\thirdnodes} = \sbcontractionof{\hypercore}{\indexedcatvariableof{\nodes/\thirdnodes}, \catvariableof{\thirdnodes} = \seccatindexof{\thirdnodes}} \, . 
 \]
	By reorganizing the sum over $\thirdnodes\subset\secnodes$ into  $\thirdnodes\subset\secnodes/\nodesa\cup\nodesb$ we have
	\begin{align}\label{eq:indContFacProof}
	 	\prod_{\thirdnodes\subset\secnodes} 
		\left(
			Z_{\thirdnodes}
		\right)^{(-1)^{\cardof{\secnodes}-\cardof{\thirdnodes}}} = 
		 \prod_{\thirdnodes\subset\secnodes/\{\nodesa,\nodesb\}} 
		 \left( 
		 	\frac{
				Z_{\thirdnodes} \cdot Z_{\thirdnodes\cup\{\nodesa,\nodesb\}}
			}{
				Z_{\thirdnodes\cup\{\nodesa\}} \cdot Z_{\thirdnodes\cup\{\nodesb\}}
			}
		 \right)^{(-1)^{\cardof{\secnodes}-\cardof{\thirdnodes}}} \, . 
	\end{align}
	From the independence assumption it follows that for any index $\catindex$
	\begin{align*}
		& \sbnormationofwrt{\hypercore}{
			 	\indexedcatvariableof{\nodesa} 
			 }{\indexedcatvariableof{\nodes/\thirdnodes\cup\{\nodesa,\nodesb\}},\catvariableof{\thirdnodes}=\seccatindexof{\thirdnodes},  \indexedcatvariableof{\nodesb} } 
			 \\
		& \quad =  
		\sbnormationofwrt{\hypercore}{
			 	\indexedcatvariableof{\nodesa} 
			 }{\indexedcatvariableof{\nodes/\thirdnodes\cup\{\nodesa,\nodesb\}}, \catvariableof{\thirdnodes}=\seccatindexof{\thirdnodes}} \\
		 & \quad  = 
		\sbnormationofwrt{\hypercore}{
			 	\indexedcatvariableof{\nodesa} 
			 }{\indexedcatvariableof{\nodes/\thirdnodes\cup\{\nodesa,\nodesb\}},\catvariableof{\thirdnodes}=\seccatindexof{\thirdnodes},  \catvariableof{\nodesb} = \seccatindexof{\nodesb}} 
	\end{align*}
	Applying this in each squares bracket term of \eqref{eq:indContFacProof} we get
	\begin{align*}
		\frac{
			Z_{\thirdnodes}
		}{
			Z_{\thirdnodes\cup\{\nodesa\}}
		} 
		& = 
		\frac{
			 \sbnormationofwrt{\hypercore}{
			 	\indexedcatvariableof{\nodesa} 
			 }{\indexedcatvariableof{\nodes/\thirdnodes\cup\{\nodesa,\nodesb\}}, \catvariableof{\thirdnodes}=\seccatindexof{\thirdnodes}, \indexedcatvariableof{\nodesb} } 
		}{
			 \sbnormationofwrt{\hypercore}{
			 	\catvariableof{\nodesa} = \seccatindexof{\nodesa}
			 }{\indexedcatvariableof{\nodes/\thirdnodes\cup\{\nodesa,\nodesb\}} , \catvariableof{\thirdnodes}=\seccatindexof{\thirdnodes}, \indexedcatvariableof{\nodesb}} 
		} \\
		& = 
		\frac{
			 \sbnormationofwrt{\hypercore}{
			 	\indexedcatvariableof{\nodesa} 
			 }{\indexedcatvariableof{\nodes/\thirdnodes\cup\{\nodesa,\nodesb\}}, \catvariableof{\thirdnodes}=\seccatindexof{\thirdnodes}, \catvariableof{\nodesb} = \seccatindexof{\nodesb}} 
		}{
			 \sbnormationofwrt{\hypercore}{
			 	\catvariableof{\nodesa} = \seccatindexof{\nodesa}
			 }{\indexedcatvariableof{\nodes/\thirdnodes\cup\{\nodesa,\nodesb\}}, \catvariableof{\thirdnodes}=\seccatindexof{\thirdnodes},\catvariableof{\nodesb} = \seccatindexof{\nodesb}} 
		} \\
		& = 
		\frac{
			Z_{\thirdnodes\cup\{\nodesb\}}
		}{
			Z_{\thirdnodes\cup\{\nodesa,\nodesb\}}
		} \, . 
	\end{align*}
	Thus, each factor in \eqref{eq:indContFacProof} is trivial, which establishes the claim.
\end{proof}

We are finally ready to proof \theref{the:condIndMN} based on the Lemmata above.

\begin{theorem}[\theref{the:condIndMN}]
	Let $\probof{\catvariableof{\nodes}}$ be a probability distribution and $\graph$ a clique-capturing hypergraph, such that for $\nodesa$, $\nodesb$, $\nodesc$ we have that $\catvariableof{\nodesa}$ is independent of $\catvariableof{\nodesb}$ conditioned on $\catvariableof{\nodesc}$, when $\nodesc$ separates $\nodesa$ and $\nodesb$ in the hypergraph.
	Then there is a Markov Network on $\graph$, which distribution is equal to $\probof{\catvariableof{\nodes}}$.
\end{theorem}
\begin{proof}[Proof of \theref{the:condIndMN}]
	By \lemref{the:contractionFactorization} we have for any index $\catindexof{\nodes}$
	\begin{align*}
		\probof{\indexedcatvariableof{\nodes}} = 
		\prod_{\secnodes\subset\nodes} \prod_{\thirdnodes\subset\secnodes} 
		\left(
			\probof{\indexedcatvariableof{\thirdnodes},\catvariableof{\nodes/\thirdnodes}=\seccatindexof{\nodes/\thirdnodes}}
		%	\contractionof{\extnet\cup\{\onehotmapof{\catindexof{\nodes/\thirdnodes}}\}}{\catvariableof{\thirdnodes}}
		\right)^{(-1)^{\cardof{\secnodes}-\cardof{\thirdnodes}}}
	\end{align*}
	For any subset $\secnodes\subset\nodes$, which is not contained in a hyperedge, we find $\nodesa,\nodesb \in\secnodes$ such that $\catvariableof{\nodesa}$ is independendent on $\catvariableof{\nodesb}$ conditioned on $\catvariableof{\secnodes/\{\nodesa,\nodesb\}}$.
	If no such nodes $\nodesa,\nodesb \in\secnodes$ exists, $\secnodes$ would be contained in a hyperedge, since the hypergraph is assumed to be clique-capturing.
	By \lemref{lem:independentContractionFactorization} we then have
	\begin{align*}
	 \prod_{\thirdnodes\subset\secnodes} 
		\left(
			\probof{\indexedcatvariableof{\thirdnodes},\catvariableof{\nodes/\thirdnodes}=\seccatindexof{\nodes/\thirdnodes}}
		\right)^{(-1)^{\cardof{\secnodes}-\cardof{\thirdnodes}}} = 1 \, .
	\end{align*}
	We label by a function 
	\begin{align*}
		\alpha: \{\secnodes : \exists\edge\in\edges: \secnodes \subset \edge \} \rightarrow \edges
	\end{align*}	
	the remaining node subsets by a hyperedge containing the subset.
	We build the tensor
	\begin{align*}
		\hypercoreofat{\edge}{\catvariableof{\edge}} = \prod_{\secnodes \, : \, \alpha(\secnodes) = \edge} \prod_{\thirdnodes\subset\secnodes} 
		\left(
			\probof{\indexedcatvariableof{\thirdnodes},\catvariableof{\nodes/\thirdnodes}=\seccatindexof{\nodes/\thirdnodes}}
		\right)^{(-1)^{\cardof{\secnodes}-\cardof{\thirdnodes}}} \, . 
	\end{align*}
	and get, that 
	\begin{align*}
		\probof{\catvariableof{\nodes}} & = \contractionof{\extnetasset}{\catvariableof{\nodes}} \\
		& = \normationofwrt{\extnetasset}{\catvariableof{\node}}{\varnothing} \, .
	\end{align*}
	We have thus constructed a Markov Network with trivial partition function, which contraction coincides with the probability distribution.
\end{proof}





\subsection{Commutation of Contractions}

We show in the next theorem, that a contractions can be performed by contracting a subnetwork first and then further contracting the result with the rest. 

%% OLD Statement
%\begin{theorem}\label{the:splittingContractions}
%	Let $\tnetof{\graph}$ be a tensor network on a hypergraph $\graph=(\nodes,\edges)$.
%	Let us now split the $\graph$ into two graphs $\graph_1=(\nodes_1,\edges_1)$ and $\graph_2=(\nodes_2,\edges_2)$, such that $\edges_1\dot{\cup}\edges_2=\edges$, and let $\secnodes,\secnodes_2\subset\nodes$ be such that $\nodes_2\cap(\nodes_1\cup\secnodes_2) \subset \secnodes$.
%	%$\nodes_1\cup\nodes_2=\nodes$.
%	If $\nodes_2\cup\secnodes \subset \secnodes_2$ then
%		\[ \contractionof{\tnetof{\graph}}{\secnodes} = \contractionof{\tnetof{\graph_1} \cup \{
%			\contractionof{\tnetof{\graph_2}}{\secnodes_2}
%		\}}{\nodes}   \, . \]
%\end{theorem}
%\begin{proof}
%	By an exchange of summations.
%\end{proof}



\begin{theorem}\label{the:splittingContractions}
	Let $\tnetof{\graph}$ be a tensor network on a hypergraph $\graph=(\nodes,\edges)$.
	Let us now split the $\graph$ into two graphs $\graph_1=(\nodes_1,\edges_1)$ and $\graph_2=(\nodes_2,\edges_2)$, such that $\edges_1\dot{\cup}\edges_2=\edges$, $\nodes_1\cup\nodes_2=\nodes$ and all nodes in $\nodes_2$ are contained in an hyperedge of $\edges_2$.
	We then have
		\[ \contractionof{\tnetof{\graph}}{\catvariableof{\secnodes}} 
		= 
		\contractionof{\tnetof{\graph_1} \cup \{
			\contractionof{\tnetof{\graph_2}}{\catvariableof{\nodes_2\cap(\nodes_1\cup\secnodes)}}
		\}}{\catvariableof{\secnodes}}   \, . \]
\end{theorem}
\begin{proof}
	For any index $\catindexof{\secnodes}$ we show that 
			\[ \contractionof{\tnetof{\graph}}{\indexedcatvariableof{\secnodes}} 
		= 
		\contractionof{\tnetof{\graph_1} \cup \{
			\contractionof{\tnetof{\graph_2}}{\catvariableof{\nodes_2\cap(\nodes_1\cup\secnodes)}}
		\}}{\indexedcatvariableof{\secnodes}}   \, . \]
	By definition we have
	\begin{align*}
		\contractionof{\tnetof{\graph}}{\indexedcatvariableof{\secnodes}} 
		& = \sum_{\catindexof{\nodes/\secnodes}} \prod_{\edge\in\edges} \hypercoreofat{\edge}{\indexedcatvariableof{\edge}} \\
		& = \sum_{\catindexof{\nodes/\secnodes}} 
		 	\left( \prod_{\edge\in\edges_1} \hypercoreofat{\edge}{\indexedcatvariableof{\edge}} \right) 
		 	\cdot \left( \prod_{\edge\in\edges_2} \hypercoreofat{\edge}{\indexedcatvariableof{\edge}}  \right) \\
		& =  \sum_{\catindexof{\nodes_1/\secnodes}} \sum_{\catindexof{\nodes_2/(\secnodes\cup\nodes_1)}} 
			\left( \prod_{\edge\in\edges_1} \hypercoreofat{\edge}{\indexedcatvariableof{\edge}} \right) 
		 	\cdot \left( \prod_{\edge\in\edges_2} \hypercoreofat{\edge}{\indexedcatvariableof{\edge}}  \right) \\
		& =  \sum_{\catindexof{\nodes_1/\secnodes}}  
			\left( \prod_{\edge\in\edges_1} \hypercoreofat{\edge}{\indexedcatvariableof{\edge}} \right) 
		 	\cdot \left( \sum_{\catindexof{\nodes_2/(\secnodes\cup\nodes_1)}}  \prod_{\edge\in\edges_2} \hypercoreofat{\edge}{\indexedcatvariableof{\edge}}  \right) \, .
	\end{align*}
	When contracting the variables $\catvariableof{\nodes_2/(\secnodes\cup\nodes_1)}$ on $\tnetof{\graph_2}$, the variables $\catvariableof{\nodes_2\cap(\secnodes\cup\nodes_1)}$ are left open. 
	We therefore have for any $\catindexof{\nodes_2\cap(\secnodes\cup\nodes_1)}$ 
	\begin{align*}
		\sbcontractionof{\tnetof{\graph_2}}{\indexedcatvariableof{\nodes_2\cap(\secnodes\cup\nodes_1)}} =
		 \left( \sum_{\catindexof{\nodes_2/(\secnodes\cup\nodes_1)}}  \prod_{\edge\in\edges_2} \hypercoreofat{\edge}{\indexedcatvariableof{\edge}}  \right) \, . 
	\end{align*}
	It follows with the above, that 
	\begin{align*}
		\contractionof{\tnetof{\graph}}{\indexedcatvariableof{\secnodes}} 
		& =  \sum_{\catindexof{\nodes_1/\secnodes}}  \left( \prod_{\edge\in\edges_1} \hypercoreofat{\edge}{\indexedcatvariableof{\edge}} \right) \cdot \sbcontractionof{\tnetof{\graph_2}}{\indexedcatvariableof{\nodes_2\cap(\secnodes\cup\nodes_1)}} \\
		& = \contractionof{\tnetof{\graph_1} \cup \{
			\contractionof{\tnetof{\graph_2}}{\catvariableof{\nodes_2\cap(\nodes_1\cup\secnodes)}}
		\}}{\indexedcatvariableof{\secnodes}}   \, . 
	\end{align*}
\end{proof}







\subsection{Support of Contractions}\label{sec:supportContractionEquations}



To state the next theorem we introduce the nonzero function $\nonzerofunction: \rr \rightarrow [2]$ by
\begin{align}
	\nonzeroof{x} = \begin{cases}
	0 & \text{if }x=0 \\
	1 & \text{else}
	\end{cases}
\end{align}
Applied coordinatewise on tensors it marks the nonzero coordinates by $1$.

We show that adding binary tensor cores to an contraction orders the results by the partial ordering introduced in \defref{def:partialFTOrder}

\begin{theorem}[Monotonicity of Tensor Contractions]\label{the:monotonicityBinaryContractions}
	Let $\extnet, \secextnet$ be tensor network of non-negative tensors and $\catvariableof{\secnodes}$ an arbitrary set of random variables. %, and $\tilde{\theta}$ another binary tensor. 
	Then we have
		\[ \nonzeroof{\contractionof{\extnet\cup\secextnet}{\catvariableof{\secnodes}}} \prec
		\nonzeroof{\contractionof{\extnet}{\catvariableof{\secnodes}}} \, .  \]
\end{theorem}
\begin{proof}
	It suffices to show that for any $\catindexof{\secnodes}$ with 
		\[ \nonzeroof{\contractionof{\extnet\cup\secextnet}{\indexedcatvariableof{\secnodes}}}=1 \]
	we also have 
		\[ \nonzeroof{\contractionof{\extnet}{\indexedcatvariableof{\secnodes}}}=1 \, . \]
	For any $\catindexof{\secnodes}$ satisfying the first equation we find an extension $\catindexof{\nodes}$ to all variables of the tensor networks such that
		\[ \contractionof{\extnet\cup\secextnet}{\indexedcatvariableof{\nodes}} > 0 \]
	and it follows that
		\[ \contractionof{\extnet}{\indexedcatvariableof{\nodes}} > 0 \quad\text{and}\quad  \contractionof{\secextnet}{\indexedcatvariableof{\nodes}} > 0  \, . \]
	But this already implies, that 
		\[ \nonzeroof{\contractionof{\extnet}{\indexedcatvariableof{\secnodes}}}=1 \, . \]
\end{proof}

Let us now state an equivalence of the contraction, when we add the result of the same contraction 
\begin{theorem}[Invariance under adding subcontractions]\label{the:invarianceAddingSubcontractions}
	Let $\extnet$ be a tensor network of non-negative tensors with variables $\catvariableof{\nodes}$ and let $\secextnet$ be a subset.
	Then we have for any subset $\catvariableof{\secnodes}$ of $\catvariableof{\nodes}$
		\[ \contractionof{\extnet \cup\{
			\nonzeroof{
			\contractionof{\secextnet}{\catvariableof{\secnodes}}
			}
		\}}{\catvariableof{\nodes}} 
		= \contractionof{\extnet}{\catvariableof{\nodes}}
		\, . \]
	
	%For any sets of leg variables $\randomxof{V},\randomxof{\tilde{V}}$ appearing in $\theta$  we have
	%Then we have
	%	\[ \contractionof{\theta\cup
	%	\nonzeroof{\contractionof{\tilde{\theta}}{\randomxof{\tilde{V}}}}
	%			}{\randomxof{V}} = \contractionof{\theta}{\randomxof{V}} \, . \]
\end{theorem}
\begin{proof}
	For any $\catindexof{\nodes}$ with 
		\[ \contractionof{\extnet}{\indexedcatvariableof{\nodes}} = 0 \]
	we also have 
		\[ \contractionof{\extnet \cup\{
			\nonzeroof{
			\contractionof{\secextnet}{\catvariableof{\secnodes}}
			}
		\}}{\indexedcatvariableof{\nodes}} = 0 \, . \]
	For any $\catindexof{\nodes}$ with 
		\[ \contractionof{\extnet}{\indexedcatvariableof{\nodes}} \neq 0 \]
	we have for the reduction $\catindexof{\secnodes}$ of the index $\catindexof{\nodes}$ that
		\[  \contractionof{\secextnet}{\indexedcatvariableof{\secnodes}} \neq 0 \]
	and thus
	\begin{align*}
		\contractionof{\extnet \cup\{
			\nonzeroof{
			\contractionof{\secextnet}{\catvariableof{\secnodes}}
			}
		\}}{\indexedcatvariableof{\nodes}} 
		= \contractionof{\extnet}{\indexedcatvariableof{\nodes}} \cdot \nonzeroof{
			\contractionof{\secextnet}{\catvariableof{\secnodes}}
			}[\indexedcatvariableof{\secnodes}]
		= \contractionof{\extnet}{\indexedcatvariableof{\nodes}} \, . 
	\end{align*}
%	When the subcore transformed by $\nonzeroof{\cdot}$ contains a zero slice, then this
%	 zero slice is also appearing in the rest contraction.
%	Multiplying a zero slice with zero does not affect the contraction, neither does multiplication with one on any slice.
\end{proof}





\begin{remark}
	Similar statements hold, when dropping the non-negativity assumption on the, but demanding that all variables are left open.
\end{remark}






















% Encoding 
\section{Basis Calculus}\label{cha:tensorEncodings}\label{cha:basisCalculus}

\red{
Basis Calculus stores informations in the selection of basis elements, while coordinate calculus uses the coordinates to each index for storage.
While coordinate calculus is more expressive, basis calculus can be exploited in sparse representations of composed functions.
}

\subsection{Encoding of Subsets and Relations}

Based on the concept of one-hot encodings of states we in this chapter develop the construction of encodings to sets, relations and functions.

\begin{figure}[h]
	\begin{center}
		\begin{tikzpicture}[yscale=0.6]
	\draw[dashed] (-10.5,12) rectangle (5.5,2);
	\node[anchor=center] (text) at (-2.5,11) {Tensors with non-negative coordinates};
	
	\draw[red] (-10,10) rectangle (2.5,5); 
	\node[anchor=center,red] (text) at (-5,9) {Directed Tensors: Conditional probability distributions};
	\draw[blue] (-7.5,7.5) rectangle (5,2.5); 
	\node[anchor=center,blue] (text) at (0,3.5) {Binary Tensors: Propositional formulas, Encoding of relations};

	\node[anchor=center] (text) at (-2.5,6.5) {Directed and Binary Tensors: Encoding of functions};
\end{tikzpicture}
	\end{center}
	\caption{Sketch of the tensors with non-negative coordinates. 
	We investigate in this chapter tensors, which are directed and binary.}
\end{figure}


Here we show how we can use binary numbers to encode the truth of set memberships.

\begin{definition}[Subset Encoding]\label{def:subsetEncoding}
	We say that an arbitrary set $\arbset$ is enumerated by a categorical variables $\individualvariableof{\arbset}$ taking values in $[\catdimof{\arbset}]$, when $\catdimof{\arbset}=\absof{\arbset}$ and there is a bijective function
		\[ \indexinterpretationof{} : [\catdimof{\arbset}] \rightarrow \arbset \, . \]
	Given an set $\arbset$ enumerated by the variable $\individualvariableof{\arbset}$, any subset $\arbsubset\subset\arbset$ is encoded by the tensor $\onehotmapto{\arbsubset}[\individualvariable]$ defined for $\catindex\in[\absof{\arbset}]$ as
	%Let there be a set $\arbset$, which elements we enumerate by $x_{\catindex}$ for $\catindex\in[\absof{\arbset}]$, and let there be a categorical variable $\individualvariable$ taking values in $[\absof{\arbset}]$ selecting the members.
	%We define the encoding of a subset $\arbsubset\subset\arbset$ as the tensor $\onehotmapto{\arbsubset}[\individualvariable]$ with coordinates
% \[ \onehotmapto{\arbsubset}[\individualvariable] : \arbset \rightarrow \{0,1\}\]
%	defined for $x\in\arbset$
	\begin{align}
	 	\onehotmapto{\arbsubset}[\individualvariable={\catindex}] = \begin{cases}
		1 & \text{if } \indexinterpretationofat{}{\catindex} \in \arbsubset \\
		0 & \text{else}
		\end{cases} \, . 
	\end{align}
\end{definition}

%	a representation of the set as
%		\[ \arbset = \left\{x_{\catindex} \, : \, \catindex\in[\absof{\arbset}] \right\} \, . \]



% Decomposition
In a one-hot basis decomposition we have
	\[ \onehotmapto{\arbsubset}[\individualvariable] \coloneqq \sum_{\catindex\in[\cardof{\arbset}] \, : \, \indexinterpretationofat{}{\catindex}\in\arbsubset}\onehotmapofat{\catindex}{\individualvariable} \, . \]
%where $\onehotmapof{x}$ denotes the one-hot encoding of an element $x$ with respect to an enumeration of the elements $\arbset$ by $[\catdim]$ or $\facstates$.

% Explanation
Encoding of subsets as vectors: Each coordinate associated with a possible element, $\{0,1\}$ encoding whether in subset.
The encodings is thus a binary tensor.
Any subset encoding is a binary tensor.

\begin{definition}[Relation Encoding]
	Given two finite sets $\inset$, $\outset$, a relation is a subset of their cartesian product
		\[ \exrelation \subset \inset \times \outset \, . \]
	Given an enumeration of $\inset$ and $\outset$ by the categorical variables $\individualvariableof{\insymbol}$ and $\individualvariableof{\outsymbol}$ and interpretation maps 
	$\indexinterpretationof{\insymbol}$, $\indexinterpretationof{\outsymbol}$
	, we define the encoding of this subset as the tensor $\onehotmapto{\exrelation}[\individualvariableof{\insymbol},\individualvariableof{\outsymbol}]$ with the coordinates
	\begin{align}
		\onehotmapto{\exrelation}[\individualvariableof{\insymbol}=\catindexof{\insymbol},\individualvariableof{\outsymbol}=\catindexof{\outsymbol}]
		= \begin{cases}
		1 & \text{if } (\indexinterpretationofat{\insymbol}{\catindexof{\insymbol}},\indexinterpretationofat{\outsymbol}{\catindexof{\outsymbol}}) \in \exrelation \\
		0 & \text{else}
		\end{cases} \, . 
	\end{align}
\end{definition}

% Decomposition
The relation encoding has a decomposition into one-hot encodings as
	\[ \onehotmapto{\exrelation}[\individualvariable_{\insymbol},\individualvariable_{\outsymbol}]
	 = \sum_{\catindexof{\insymbol},\catindexof{\outsymbol} \, : \, (\indexinterpretationofat{\insymbol}{\catindexof{\insymbol}},\indexinterpretationofat{\outsymbol}{\catindexof{\outsymbol}}) \in \exrelation}
	\onehotmapofat{\catindexof{\insymbol}}{\catvariableof{\insymbol}}  \otimes \onehotmapofat{\catindexof{\outsymbol}}{\catvariableof{\outsymbol}}  \, . \]

Relations are subsets of cartesian products and encodings of relations are the encodings of subsets by vectors.
They have a matrix structure by the cartesian product, which can be further folded to tensors, when the sets itself are cartesian products.


\begin{theorem}
	The relational encoding is a bijection between the set of relations and the set of binary tensors.
\end{theorem}
\begin{proof}
	% =>
	By definition, a relational encoding is the encoding of a subset and thus a binary tensor.
	% <= 
	Any matrification of a binary tensor marks by its $1$ coordinates the elements of a relation.
\end{proof}

% Significance
We can thus understand any matrification of a binary tensor as the encoding of a relation and vice versa.



\subsubsection{Higher order relations}


We can extend this contraction to relations of higher order, and arrive at encoding schemes usable for relational databases.

\begin{definition}\label{def:daryRelation}
	Given sets $\arbset^{\atomenumerator}$ for $\atomenumeratorin$, a $\atomorder$-ary relation is a subset of a their cartesian product, that is
		\[ \exrelation \subset  \bigtimes_{\atomenumeratorin} \arbset^{\atomenumerator} \, . \]
	Given an enumeration of each set $\arbset^{\atomenumerator}$ by a variable $\individualvariableof{\atomenumerator}$ and an interpretation map $\indexinterpretationof{\atomenumerator}$, we define the encoding of the relation as the tensor $\onehotmapto{\exrelation}[\individualvariableof{[\atomorder]}]$ with coordinates
	\begin{align}
		\onehotmapto{\exrelation}[\individualvariableof{0}=\catindexof{0},\ldots,\individualvariableof{\atomorder-1}=\catindexof{\atomorder-1}]
		= \begin{cases}
		1 & \text{if } (\indexinterpretationofat{0}{\catindexof{0}},\ldots,\indexinterpretationofat{\atomorder-1}{\catindexof{\atomorder-1}}) \in \exrelation \\
		0 & \text{else}
		\end{cases} \, . 
	\end{align}
\end{definition}


\begin{example}[Propositional Formulas]
	Propositional formulas are equal to the subset encoding of their models.
	The sets $\arbset^{\atomenumerator}$ are all $[2]$ and are interpreted as the possible assignments to the boolean atoms.
\end{example}


\begin{example}[Relational Databases]
	Relational Databases can be encoded as tensors using the relation encoding scheme.
	Each column is thereby understood as a categorical variable, which values form the sets $\arbsetof{\catenumerator}$.
\end{example}

% Sparse Representations
Let us notice, that the dimensionality of the tensor space used for representing a relation is 
	\[ \prod_{\catenumeratorin} \cardof{\arbsetof{\catenumerator}} \]
and therefore growing exponentially with the number of variables.
Relations are however often sparse, in the sense that 
	\[ \cardof{\exrelation} << \prod_{\catenumeratorin} \cardof{\arbsetof{\catenumerator}} \, . \]
It is therefore often benefitially to choose sparse encoding schemes, for example by restricted CP formats (see Chapter~\ref{cha:sparseTC}) to represent $\onehotmapof{\exrelation}$.

\subsection{Encoding of Functions}

Real valued functions are directly tensors by definition.

\subsubsection{Relational Encoding of Functions}

% Reference to first definition
In Definition~\ref{def:functionRepresentation} we have introduced the relational encoding of functions between the states of factored systems.
We now generalize the representation scheme towards maps between arbitrary unstructured sets.

\begin{definition}[Relational Encoding of Maps]\label{def:functionRelationEncoding}
	Any map
		\[ \exfunction : \inset \rightarrow \outset \]
	can be represented by a relation
		\[ \exrelationof{\exfunction} \coloneqq \left\{ (x,\exfunction(x) \, : \, x \in \inset )\right\} \subset \inset \times \outset \, . \]
	Given a enumeration of the sets by $\individualvariableof{\insymbol}$ and $\individualvariableof{\outsymbol}$ we define the relational encoding of $\exfunction$ as the tensor
		\[ \rencodingofat{\exfunction}{\individualvariableof{\insymbol},\individualvariableof{\outsymbol}} = \onehotmapto{\exrelationof{\exfunction}}\left[\individualvariableof{\inset},\individualvariableof{\outset}\right]  \, . \]
\end{definition}

\begin{remark}[Reduction to images]
	% Image enumeration
	When $\exfunction$ maps into a set of infinite cardinality, we restrict $\outset$ to the image of $\exfunction$ and enumerate the image by a variable $\catvariableof{\exfunction}$.
	This scheme is applied, when $\exfunction$ is itself a tensor, i.e. $\outset=\rr$.
	While the variable $\catvariableof{\exfunction}$ can in general be of the same cardinality as the domain set $\inset$, it will be in $[2]$ when considering binary tensors.
\end{remark}

% Characterization of the directed and binary tensors
We notice, that any relational representation of a function is also a directed tensor with incoming variables to the domain and outgoing variables to the image.
It furthermore holds, that the set of directed and binary tensors is characterized by the relational encoding of functions.
This is shown in the next theorem, by the claim that any binary tensor which is directed is the relational representation of a function.

\begin{theorem}\label{the:rencodingDirected}
	Let $\inset,\outset$ be sets and $\exrelation\subset\inset\times\outset$ a relation.
	If and only if there exists a map $\exfunction:\inset\rightarrow\outset$ such that $\exrelation=\exrelationof{\exfunction}$, the relational encoding $\rencodingof{\exfunction}$ is a directed tensor with $\individualvariableof{\insymbol}$ incoming and $\individualvariableof{\outsymbol}$ outgoing.
%	If and only if the relation is a function between domain and image set, its encoding is directed with domain incoming and image outgoing.
\end{theorem}
\begin{proof}
	% =>
	When $\exfunction$ is a function, we have for any $\indindexofin{\insymbol}$
		\[ \sum_{\indindexofin{\outsymbol}} \rencodingofat{\exfunction}{\indexedindvariableof{\insymbol},\indexedindvariableof{\outsymbol}}
		=  \rencodingof{\exfunction}[\indexedindvariableof{\insymbol},\indvariableof{\outsymbol}=\exfunction(\indindexof{\insymbol})] = 1 \, . \]
	% <=
	Conversely let there be a relation $\exrelation$, such that $\rencodingof{\exrelation}$ is directed. %, we construct a map $\exfunction$ with $\exrelation=\exrelationof{\exfunction}$.
	To this end, we observe that for any $\indindexofin{\insymbol}$ the tensor
		\[  \onehotmapofat{\exrelation}{\indexedindvariableof{\insymbol},\indvariableof{\outsymbol}} \]
	is a binary tensor with coordinate sum one and therefore a basis vector.
	It follows that the function $\exfunction : \inset \rightarrow \outset $ 
		\[ \exfunction(\indindexof{\insymbol}) = \invonehotmapof{\onehotmapofat{\exrelation}{\indexedindvariableof{\insymbol},\indvariableof{\outsymbol}} } \]
	is well-defined.
	We then have by construction
	\begin{align*}
		\rencodingof{\exfunction} 
		= \sum_{\indindexofin{\insymbol}} \onehotmapof{\indindexof{\insymbol}} \otimes \onehotmapof{\exfunction(\indindexof{\insymbol})}
		=  \sum_{\indindexofin{\insymbol}} \onehotmapof{\indindexof{\insymbol}} \otimes \onehotmapofat{\exrelation}{\indexedindvariableof{\insymbol},\indvariableof{\outsymbol}} 
		= \onehotmapof{\exrelation}
	\end{align*}
	and therefore $\exrelation = \exrelationof{\exfunction}$.
%	 with a directed encoding, directionality implies that for any $\incatindex$ there is exactly one $\outcatindex$ such that $\onehotmapto{\exrelation}[\indexedcatvariableof{\insymbol}\indexedcatvariableof{\outsymbol}]=1	
%	Then we define a function $\exfunction$ as the mapping of $\incatindex$ to the corresponding $\outcatindex$ with $\onehotmapto{\exrelation}[\incatindex\outcatindex]=1$ and observe $\exrelation=\exrelationof{\exfunction}$.
\end{proof}

% Grid sets
We are specially interested in sets of states of a factored system, which amounts to the case in Definition~\ref{def:functionRepresentation}.
Those state sets have a decomposition into a cartesian product of $\atomorder$ sets
	\[ \arbset = \facstates \, . \]
The most obvious enumeration of the set $\arbset$ is therefore by the collection of state variables $\{\catvariableof{\atomenumerator} \, : \, \atomenumeratorin \}$.
Functions between states of factored systems with $\atomorder_{\insymbol}$ and $\atomorder_{\outsymbol}$ state variables can be represented by $\atomorder_{\insymbol}+\atomorder_{\outsymbol}$-ary relations and Definition~\ref{def:functionRelationEncoding} has an obvious generalization to this case with multiple enumeration variables.


% Conditional 
Since the relational encoding of any map between factored systems is directed, it can be interpreted by a conditional probability tensor, as we state next.

%% Maps
\begin{corollary}%\label{the:condProbFunctionRepresentation}
	The relational encoding $\rencodingof{\exfunction}$ (see Definition~\ref{def:functionRepresentation}) of a function $\exfunction$ between factored systems is a conditional probability tensor, where the legs to the image system are the conditions and the legs to the target system the distribution legs.
\end{corollary}
%\begin{proof}
%	To proof the claim we need to show that contracting the trivial tensor to the target legs results in the trivial tensor of the image legs, that is
%		\[ \onesof{\shortcatvariables} = \contractionof{\{\rencodingof{\exfunction}\}}{\shortcatvariables} \, . \]
%	This can be shown coordinatewise by sums over the target legs as
%	\begin{align*}
%		\sum_{\seccatindices} \ftensorof{\exfunction}_{\seccatindices,:}
%		 	= &  \sum_{\seccatindices} \sum_{\catindices}  \left(\onehotmapof{\exfunction(\catindices)}\right)_{\seccatindices} \otimes \onehotmapof{\catindices} \\
%			= &  \sum_{\catindices} \onehotmapof{\catindices}  \\
%			= &  \onesof{\shortcatvariables}  \, .
%	\end{align*}
%	Here we used in the second equality, that the function $\exfunction$ maps each index pair $(\catindices)$ of the image system to exactly one index pair $(\seccatindices)$ in the target system.
%	In the third equality we further used that the sum of the one-hot encodings of all states is the trivial tensor.
%\end{proof}

%% Deterministic by construction
These are deterministic conditional probability tensors, in the sense that any slice with respect to the input variables is a basis tensor.
Through contractions with distribution tensors (e.g. distributions in domain systems) they get stochastic.
This is for example the case in the empirical distribution, which can be understood as the forwarding of the uniform distribution on the sample enumeration.








\subsection{Calculus of relational encodings}



\subsubsection{Function Evaluation}

\begin{theorem}[Basis Calculus]\label{the:basisCalculus}
	Retrieving the value of the function at a specific state is then the contraction of the tensor representation with the one-hot encoded state.
	For any state indexed by $(\catindices)$ we have
		\[ \onehotmapofat{\exformula(\catindices)}{\catvariableof{\exformula}} 
		= \contractionof{\{\rencodingof{\exformula},\onehotmapof{\catindices}\}}{\catvariableof{\exformula}} \, .\]
	Thus, we can retrieve the function evaluation by the inverse one-hot mapping as
		\[ \exformula(\catindices) 
		= \invonehotmapof{\contractionof{\{\rencodingof{\exformula},\onehotmapof{\catindices}\}}{\catvariableof{\exformula}}} \, . \]
	This generalizes Example~\ref{exa:atomicFunction} to arbitrary maps between factored systems.
\end{theorem}
\begin{proof}
	By doing the summation in tensor product notation.
\end{proof}

%% Usage: Basis Calculus
We can thus use tensor contractions to calculate the values of functions.
Since basis vectors being the one-hot encoding of the domain system are mapped to basis vectors being the encoding of the image system, we call these contraction basis calculus.

\subsubsection{Composition of function}

We have already used, that combination of propositional formulas by connectives can be represented by contractions.
In a more general perspective, any composition of functions between factored systems is in its relational encoding the contraction of the encoded functions.

\begin{theorem}[Composition of Functions]\label{the:compositionByContraction}
	Let there be two maps between factored systems 
		\[ \exfunction : \bigtimes_{\node\in\nodes_1} [\catdimof{\node}] \rightarrow \bigtimes_{\node\in\nodes_2} [\catdimof{\node}] \]
	and 
		\[ \secexfunction : \bigtimes_{\node\in\nodes_2} [\catdimof{\node}] \rightarrow \bigtimes_{\node\in\nodes_3} [\catdimof{\node}] \]
	with the image system of $\exfunction$ is the domain system of $\secexfunction$.
	Then the relational encoding of the composition 
		\[ \secexfunction(\exfunction) :  \bigtimes_{\node\in\nodes_1} [\catdimof{\node}] \rightarrow \bigtimes_{\node\in\nodes_3} [\catdimof{\node}] \]
	satisfies
		\[ \rencodingof{\secexfunction(\exfunction)} = \contractionof{\{\rencodingof{\exfunction},\rencodingof{\secexfunction}\}}{\nodes_1\cup\nodes_3} \, .  \]
\end{theorem}
\begin{proof}
	By definition we have
	\begin{align*}
		\rencodingof{\secexfunction(\exfunction)} = \sum_{i_1 \in \bigtimes_{\node\in\nodes_1} [\catdimof{\node}]} \onehotmapof{i_1} \otimes \onehotmapof{\secexfunction(\exfunction(i_1))} \, ,
	\end{align*}
	where by $i_1$ we denote in a slide abuse of notation the tuple indexing the state of the domain factored systems of $\exfunction$.
	On the other side, we have that
	\begin{align*}
	 	\contractionof{\{\rencodingof{\exfunction},\rencodingof{\secexfunction}\}}{\catvariableof{\nodes_1}\catvariableof{\nodes_3}} 
		& =
		\sum_{\tilde{i}_2 \in \bigtimes_{\node\in\nodes_2}[\catdimof{\node}]} 
		\left( \sum_{i_1 \in \bigtimes_{\node\in\nodes_1}[\catdimof{\node}]}  \onehotmapof{i_1} \cdot \left(\onehotmapof{\exfunction(i_1)}\right)_{\tilde{i}_2} \right)
		\left( \sum_{i_2 \in \bigtimes_{\node\in\nodes_2}[\catdimof{\node}]}  \left(\onehotmapof{i_2}\right)_{\tilde{i}_2} \cdot \onehotmapof{\secexfunction(i_2)} \right) \\
		& = 
		\sum_{\tilde{i}_2 \in \bigtimes_{\node\in\nodes_2}[\catdimof{\node}]} 
		\left( \sum_{i_1 \in \bigtimes_{\node\in\nodes_1}[\catdimof{\node}]}  \onehotmapof{i_1} \cdot \delta_{\exfunction(i_1),\tilde{i}_2} \right)
		\left( \sum_{i_2 \in \bigtimes_{\node\in\nodes_2}[\catdimof{\node}]}  \delta_{i_2 \tilde{i}_2} \cdot \onehotmapof{\secexfunction(i_2)} \right) \\
		& = 
		\sum_{i_1 \in \bigtimes_{\node\in\nodes_1}[\catdimof{\node}]}
		\sum_{\tilde{i}_2 \in \bigtimes_{\node\in\nodes_2}[\catdimof{\node}]}  \delta_{\exfunction(i_1),\tilde{i}_2} \cdot \left(  \onehotmapof{i_1}   \otimes \onehotmapof{\secexfunction(i_2)} \right) \\
		& = 
		\sum_{i_1 \in \bigtimes_{\node\in\nodes_1} [\catdimof{\node}]} 
		\onehotmapof{i_1} \otimes \onehotmapof{\secexfunction(\exfunction(i_1))} \, .
	\end{align*}
	Here we represented the contraction of the variables in $\nodes_2$ by the summation over another index $i_2$.
	In the last equation we used that the delta tensor does not vanish only for $\tilde{i}_2= \exfunction(i_1)$.
	The claim follows, since both expressions are equal.
\end{proof}

% Iterative usage
We can use Theorem~\ref{the:compositionByContraction} iteratively to further decompose the function $\secexfunction$.
In this way, the relational encoding of a function consistent of multiple compositions can be represented as the contractions of all the functions.
The relational encoding of propositional formulas for instance can in this way be represented as a contraction of the encodings of its logical connectives applied on the respective formula spaces. 
%It is thereby of central importance that for any connective a respective variable is added, which then disappears in the contractions.




\subsubsection{Compositions with real functions}

\red{Follows from composition calculus above with the usage of }
	\[ \hypercore = \sbcontractionof{\rencodingof{\hypercore}, \restrictionofto{Id}{\imageof{\hypercore}} }{\shortcatvariables} \, . \]

We here investigate how the composition of a tensor 
	\[ \hypercore : \facstates \rightarrow \rr \]
with arbitrary functions 
	\[ \chainingfunction: \rr \rightarrow \rr \]
can be represented.
This is for example relevant, when representing the distributions of an exponential family.

% Strategy
Our main strategy is in understanding the tensor $\hypercore$ as a map to its finite image, seen as the enumerated states of a categorical variable building a factored system.
We then use the relational encoding $\rencodingof{\hypercore}$ of this map between factored systems. 

By $\restrictionofto{\chainingfunction}{\mathcal{M}}$ we further denote the restriction of a real function $\chainingfunction$ to an enumerated set $\mathcal{M} =\{x_i \, : \, i \in [\cardof{\mathcal{M}}]\} \subset \rr$, i.e. the vector
	\[ \restrictionofto{\chainingfunction}{\mathcal{M}} : [\cardof{\mathcal{M}}] \rightarrow \rr \]
defined for $i \in [\cardof{\mathcal{M}}]$ as
	\[ \restrictionofto{\chainingfunction}{\mathcal{M}}(i) = \chainingfunction(x_i) \, . \]


\begin{theorem}\label{the:tensorFunctionComposition}
	We have for any tensor $\hypercore$ and real function $\chainingfunction$ (see Figure~\ref{fig:tensorFunctionComposition})
		\[ \chainingfunction\circ\hypercore = \contractionof{\{\rencodingof{\hypercore}, \restrictionofto{\chainingfunction}{\imageof{\hypercore}}\}}{\shortcatvariables} \, . \]
\end{theorem}
\begin{proof}
	We enumerate the image $\mathrm{im}(\hypercore)$ by $\{x_i \, : \, i \in [\cardof{\mathrm{im}(\hypercore)}]\}$.
	For arbitrary but fixed $\catindices\in\facstates$ let $\tilde{i}$ be such that $\hypercore(\catindices) = x_i$.
	Then we have for $\randomxof{\hypercore}$ denoting the image variable of $\rencodingof{\hypercore}$ that
		\[ \contractionof{\{\rencodingof{\hypercore},\onehotmapof{\catindices}\}}{\{\randomxof{\hypercore}\}} = \onehotmapof{\tilde{i}} \]
	and
		\[ \contractionof{\{\rencodingof{\hypercore}, \restrictionofto{\chainingfunction}{\imageof{\hypercore}}, \onehotmapof{\catindices}\}}{\varnothing} = 
		\contractionof{\{\restrictionofto{\chainingfunction}{\imageof{\hypercore}}, \onehotmapof{\tilde{i}}\}}{\varnothing} = 
		\chainingfunction(\hypercore(\catindices)) \, . 
		\]
	Since $\catindices$ was chosen arbitrarly form $\facstates$, this shows that $\chainingfunction\circ\hypercore$ and $ \contractionof{\{\rencodingof{\hypercore}, \restrictionofto{\chainingfunction}{\imageof{\hypercore}}\}}{\shortcatvariables}$ coincide on all inputs and are thus equal.
\end{proof}

\begin{figure}[h]
\begin{center}
	\begin{tikzpicture}[scale=0.35] % , baseline = -3.5pt




\begin{scope}[shift={(-15,0)}]

\drawatomcore{-6}{-8}{$\chainingfunction\circ\hypercore$}


	\begin{scope}[shift={(-6,-12)}]
		\draw[] (0,1)--(0,-1) node[midway,left] {\tiny $\catvariableof{0}$}; 
		\draw[] (1.5,1)--(1.5,-1) node[midway,left] {\tiny $\catvariableof{1}$}; 
		\node[anchor=center] (text) at (3,0) {$\cdots$};
		\draw[] (4,1)--(4,-1) node[midway,right] {\tiny $\catvariableof{\atomorder\shortminus1}$}; 
	\end{scope}

\end{scope}



%\node[anchor=center] (text) at (-14.25,-10) {${=}$};



%\begin{scope}[shift={(-12.5,0)}]
%
%\node[anchor=center] (text) at (1,-7.25) {\small $\chainingfunction$};
%\draw (5.5,-7.25) ellipse (6 and 4.5);
%
%
%\drawatomcore{3.5}{-8}{$\ftensorof{\exformula}$}
%\drawatomindices{3.5}{-12}	
%\draw[] (5.5,-9)--(5.5,-7) node[midway,right] {\tiny $\randomxof{\hypercore}$};
%
%
%\draw (4.75,-3.5) rectangle (6.25,-7);
%\node[anchor=center] (text) at (5.5,-5.25) {$\begin{bmatrix} 
%0 \\
%1
%\end{bmatrix}$};
%
%\end{scope}


\begin{scope}[shift={(-12.5,0)}]

\node[anchor=center] (text) at (-0.5,-10) {${=}$};

%\node[anchor=center] (text) at (0.5,-8) {$\mathrm{log}$};

\drawatomcore{3.5}{-8}{$\rencodingof{\hypercore}$}
\drawatomindices{3.5}{-12}	
\draw (5.5,-9)--(5.5,-6) node[midway,right] {\tiny $\randomxof{\hypercore}$};
\draw[->] (5.5,-9) -- (5.5,-7.5);

\draw (3.25,-4) rectangle (7.5,-6);
\node[anchor=center] (text) at (5.5,-5) {$\restrictionofto{\chainingfunction}{\imageof{\hypercore}}$
%\begin{bmatrix} 
%	\chainingfunction(0) \\
%	\chainingfunction(1)
%\end{bmatrix}
};

\end{scope}

\end{tikzpicture}
\end{center}
\caption{Representation of the composition of a tensor $\hypercore$ with a real function $\chainingfunction$.}
\label{fig:tensorFunctionComposition} 
\end{figure}


\begin{corollary}\label{cor:rhoToNormal}
	For any tensor $\hypercoreat{\shortcatvariables}$ we have
		\[ \hypercoreat{\shortcatvariables} = \contractionof{\rencodingof{\hypercore},\idrestrictedto{\imageof{\hypercore}}}{\shortcatvariables} \, . \]
\end{corollary}
\begin{proof}
	Directly by using $\chainingfunction=\mathrm{Id}$.
\end{proof}


\begin{corollary}\label{cor:onesHead}
	For any tensor $\hypercore$, which is directed with $\shortcatvariables$ incoming, we have
		\[ \onesat{\shortcatvariables} = \contractionof{\rencodingof{\hypercore}}{\shortcatvariables} \, . \]
\end{corollary}
\begin{proof}
	Directly by using $\chainingfunction=\ones$.
\end{proof}

% Replacement of Slicing Theorem
\begin{corollary}\label{cor:directedTrafo}
	Let $\basisslices$ be a directed and binary tensor with incoming variables being $\shortcatvariables$, and $\gentensor$ a tensor, which variables are the outgoing variables of $\basisslices$.
	Let further $\coordinatetrafo:\rr\rightarrow\rr$ be any real function.
	Then
		\[ \coordinatetrafo \circ \contractionof{\{\basisslices,\gentensor\}}{\shortcatvariables} = \contractionof{\{\basisslices,\coordinatetrafo\circ \gentensor\}}{\shortcatvariables} \, . \]
\end{corollary}
\begin{proof}
	Since $\basisslices$ is a directed and binary tensor, we find a map
		\[ \exfunction : \facstates \rightarrow \secfacstates \]
	such that $\basisslices=\rencodingof{\exfunction}$ and a map $V$ such that $\gentensor=\restrictionofto{V}{\imageof{\exfunction}}$.
	Then Theorem~\ref{the:tensorFunctionComposition} implies that 
		\[ \contractionof{\{\basisslices,\gentensor\}}{\shortcatvariables} = V \circ \exfunction \, . \]
	It follows that 
	\begin{align*}
		\coordinatetrafo \circ \contractionof{\{\basisslices,\gentensor\}}{\shortcatvariables} = \coordinatetrafo \circ V \circ \exfunction 
	\end{align*}
	and by another application of Theorem~\ref{the:tensorFunctionComposition} that
	\begin{align*}
		\coordinatetrafo \circ V \circ \exfunction
		& = \contractionof{\rencodingof{\exfunction}, \restrictionofto{\coordinatetrafo \circ V}{\imageof{\exfunction}}}{\shortcatvariables} \\
		& = \contractionof{\{\basisslices,\coordinatetrafo\circ\gentensor\}}{\shortcatvariables} \, . 
	\end{align*}
	The claim follows as a combination of both equations.
\end{proof}



\begin{example}[Shannon entropy of empirical distribution]
%\begin{theorem}
	The Shannon entropy of an empirical distribution can be efficiently computed by contraction of the datatensor with itself along the atom indices, then applying a coordinatewise $\ln$ and averaging.
%\end{theorem}

%\begin{proof}
	This follows from commutations of contraction and coordinatewise contraction (see Corollary~\ref{cor:directedTrafo}), using that the datacores are directed. 
	%Theorem~\ref{the:CoordinateTransform}, since the datatensor is a slicewise basis tensor.
	To be more precise, let $\secdatamap$ be a copy of $\datamap$ with identical image space and copied domain space.
	Then $\secdatacoreof{\atomenumerator}$ are tensor cores with identical outgoing legs to $\datacoreof{\atomenumerator}$, but different incoming legs.
	We have that
	\begin{align*}
		\sentropyof{\empdistribution} 
		& = \contractionof{\{\datacoreof{\atomenumerator}\, : \, \atomenumeratorin\} \cup \{\frac{1}{\datanum}\ones \} \cup \{-\ln \contractionof{\{\secdatacoreof{\atomenumerator}\, : \, \atomenumeratorin\} \cup \{\frac{1}{\datanum}\ones \}}{\shortcatvariables} \} }{\varnothing} \\
		& = \contractionof{
		 \left\{\frac{1}{\datanum}\ones, -\ln \contractionof{\{\datacoreof{\atomenumerator},\secdatacoreof{\atomenumerator}\, : \, \atomenumeratorin\} \cup \{\frac{1}{\datanum}\ones \}}{\shortcatvariables} \right\}
		}{\varnothing}
	\end{align*}
	where in the second equation we used Corollary~\ref{cor:directedTrafo}.
%\end{proof}
\end{example}

\subsubsection{Decomposition in case of structured images}

\red{Here the introduction of multiple head variables, i.e. when}
	\[ \outset = \bigtimes_{\catenumeratorin} \arbsetof{\catenumerator}\]
\red{This is the case for the empirical distributions!}


When the image admits a cartesian representation, the relational encoding can be represented by a contraction of relational encodings to each image coordinate.

\begin{theorem}\label{the:functionDecompositionBasisCP}
	Let $\exfunction$ be a function between factored systems
		\[ \exfunction : [\catdim] \rightarrow  \facstates \]
	and denote by
		\[ \exfunction^\atomenumerator : [\catdim] \rightarrow [\catdimof{\atomenumerator}]\]
	the restrictions of $\exfunction$ to axes of $\facstates$.
	We assign the variable $\catvariable$ to the factored system in the domain system of $\exfunction$ and the variables $\catvariableof{\atomenumerator}$ for $\atomenumeratorin$ to the image system of $\exfunction$.
	
	We then have
	\begin{align*}
		\rencodingofat{\exfunction}{\catvariable,\shortcatvariables}  
		= \contractionof{
		\{\rencodingofat{\exfunction^{\atomenumerator}}{\catvariable,\catvariableof{\atomenumerator}} : \atomenumeratorin \} 
		}{\catvariable,\shortcatvariables} \, . 
	\end{align*}
%	and 
%		\[ \baspluscprankof{\rencodingof{\exfunction}} \leq \catdim \, . \]
\end{theorem}
\begin{proof}
	
	We have 
	\begin{align*}
		\rencodingofat{\exfunction}{\catvariable,\shortcatvariables}  
		= \contractionof{
		\{\rencodingofat{\exfunction^{\atomenumerator}}{\catvariable,\catvariableof{\atomenumerator}} : \atomenumeratorin \} 
		}{\catvariable,\shortcatvariables}
	\end{align*}
	since for any $\catindex\in[\catdim]$ 
	\begin{align*}
		\rencodingofat{\exfunction}{\catvariable=\catindex,\shortcatvariables}  
		= \bigotimes_{\atomenumeratorin} \rencodingofat{\exfunction^{\atomenumerator}}{\catvariable=\catindex,\catvariableof{\atomenumerator}}
		= \contractionof{
		\{\rencodingofat{\exfunction^{\atomenumerator}}{\catvariable,\catvariableof{\atomenumerator}} : \atomenumeratorin \} 
		}{\catvariable=\catindex,\shortcatvariables} \, . 
	\end{align*}
	
	To get a representation in the basis CP format, we rename the variable $\catvariable$ in the cores $\rencodingofat{\exfunction^{\atomenumerator}}{\catvariable,\catvariableof{\atomenumerator}}$ by $\decvariable$ and observe that for a trivial scalar core $\onesat{\decvariable}$ 
	\begin{align*}
		\contractionof{
		\{\rencodingofat{\exfunction^{\atomenumerator}}{\catvariable,\catvariableof{\atomenumerator}} : \atomenumeratorin \} 
		}{\catvariable,\shortcatvariables} 
		=
		\contractionof{ 
		\{\onesat{\decvariable}\} \cup \{\rencodingofat{\exfunction^{\atomenumerator}}{\decvariable,\catvariableof{\atomenumerator}} : \atomenumeratorin \} \cup \{\identityat{\catvariable,\decvariable}\}
		}{\catvariable,\shortcatvariables}  \, . 
	\end{align*}
	
%	It suffices to show for any $\catindex\in[\catdim]$ we have
%		\[ \contractionof{\{\ftensorof{\exfunction},\onehotmapof{\catindex}\}}{\shortcatvariables} = 
%		 	\contractionof{\{\ftensorof{\exfunction^\atomenumerator} \, : \, \atomenumeratorin\}\cup\{\onehotmapof{\catindex}\}}{\shortcatvariables} \, . 
%		  \]
%	But this holds, since
%	\begin{align*} 
%		\contractionof{\{\ftensorof{\exfunction},\onehotmapof{\catindex}\}}{\shortcatvariables} 
%			& = \bigotimes_{\atomenumeratorin} \onehotmapof{\exfunction^{\atomenumerator}(\catindex)} \\
%			& = \bigotimes_{\atomenumeratorin} \contractionof{\{\ftensorof{\exfunction^\atomenumerator},\onehotmapof{\catindex}\}}{\randomxof{\atomenumerator}} \\
%			& = \contractionof{\{\ftensorof{\exfunction^\atomenumerator} \, : \, \atomenumeratorin \}\cup\{\onehotmapof{\catindex}\}}{\shortcatvariables} \, .
%	\end{align*}
\end{proof}





\subsection{Effective Calculus}\label{sec:effectiveCalculus} % -> Part III

% Calculus against the direction
For specific functions, slices of the relational encodings with respect to head variables are basis vectors.
In that case, we can perform basis calculus in the inverse direction than suggested by the directions of the tensors.
We examplify this situation in the following theorem for relational encodings of logical conjunctions and negations.

\begin{figure}
\begin{center}
	\begin{tikzpicture}[scale=0.3] 

\node at (-12,2) [above]  {a)};

	\begin{scope}[shift={(-7,0)}]
	  	\draw[]  (-3,2) rectangle (1,4);
		\node at (-1,1.9) [above] {${\exformula\land\secexformula}$};
		\draw (-2,2) -- (-2,-2) node[midway,left] {$\catvariableof{1}$};
		\draw (0,2) -- (0,-2) node[midway,right] {$\catvariableof{2}$};	
	\end{scope}
	\node at (-4.5,1.9) [above] {$=$};	
  	\draw[] (-3,2) rectangle (1,4);
	\node at (-1,1.9) [above] {${\exformula}$};
  	\draw[] (3,2) rectangle (5,4);
	\node at (4,1.9) [above] {${\secexformula}$};
	
	\node at (0,0) [left,] {$\delta$};
	\draw[]  (0,2) -- (0,0);% node[midway,left] {$\placeholderof{1}$};
	\draw[fill,] (0,0) circle (0.25cm);
	\draw[] (0,0) -- (0,-2) node[midway,right] {$\catvariableof{2}$};
	\draw[] (4,2) to[bend left=20] (0,0);
	\draw[] (-2,2) -- (-2,-2) node[midway,left] {$\catvariableof{1}$};

\begin{scope}[shift={(27,0)}]
	\node at (-18,2) [above]  {b)};
  	\begin{scope}[shift={(-14,0)}]
	  	\draw[]  (-3,2) rectangle (1,4);
		\node at (-1,1.9) [above] {${\lnot\exformula}$};
		\draw (-2,2) -- (-2,-2) node[midway,left] {$\catvariableof{1}$};
		\draw (0,2) -- (0,-2) node[midway,right] {$\catvariableof{2}$};	
	\end{scope}
	
    	\begin{scope}[shift={(-2,-2)}]
  		\draw[] (-8,2) rectangle (-4,4);
		\node at (-6,2) [above,] {$\ones$};
		\draw[] (-7,2) -- (-7,0) node[midway,left] {$\catvariableof{1}$};
		\draw[] (-5,2) -- (-5,0) node[midway,right] {$\catvariableof{2}$};
	\end{scope}
	
	\draw[] (-2,2) -- (-2,-2) node[midway,left] {$\catvariableof{1}$};
	\draw[](0,2) -- (0,-2) node[midway,right] {$\catvariableof{2}$};
	\node[] at (-4.5,0) [above] {$-$};



	\node at (-11.5,1.9) [above] {$=$};	
	
  	\draw[]  (-3,2) rectangle (1,4);
	\node at (-1,1.9) [above] {${\exformula}$};	
	
\end{scope}

\end{tikzpicture}
\end{center}
\caption{Decomposition schemes by effecitve calculus. a) Conjunction, b) Negations.}\label{fig:ConNegDecomposition}
\end{figure}

\begin{theorem}\label{the:effectiveConjunction}
	For any formulas $\exformula,\secexformula$ we have
	\begin{align*}
		\sbcontractionof{
			\rencodingofat{\land}{\catvariableof{\exformula\land\secexformula},\catvariableof{\exformula},\catvariableof{\secexformula}},\onehotmapofat{1}{\catvariableof{\exformula\land\secexformula}}
		}{\catvariableof{\exformula},\catvariableof{\secexformula}}
		= \onehotmapofat{1}{\catvariableof{\exformula}} \otimes \onehotmapofat{1}{\catvariableof{\secexformula}} \, . 
	\end{align*}
	In particular, it holds that (see Figure~\ref{fig:ConNegDecomposition}a)
	\begin{align*}
		(\exformula\land\secexformula)[\shortcatvariables] = \sbcontractionof{\exformula,\secexformula}{\shortcatvariables} \, . 
	\end{align*}
\end{theorem}
\begin{proof}
	We decompose 
	\begin{align*}
		\rencodingofat{\land}{\catvariableof{\exformula\land\secexformula},\catvariableof{\exformula},\catvariableof{\secexformula}}
		= \onehotmapofat{1}{\catvariableof{\exformula\land\secexformula}} \otimes \onehotmapofat{1}{\catvariableof{\exformula}} \otimes \onehotmapofat{1}{\catvariableof{\secexformula}}
		+ \onehotmapofat{0}{\catvariableof{\exformula\land\secexformula}} \left( \onesat{\catvariableof{\exformula},\onesat{\catvariableof{\secexformula}}} -  \onehotmapofat{1}{\catvariableof{\exformula}} \otimes \onehotmapofat{1}{\catvariableof{\secexformula}} \right) 
	\end{align*}
	and get the first claim as
	\begin{align*}
		\sbcontractionof{
			\rencodingofat{\land}{\catvariableof{\exformula\land\secexformula},\catvariableof{\exformula},\catvariableof{\secexformula}},\onehotmapofat{1}{\catvariableof{\exformula\land\secexformula}}
		}{\catvariableof{\exformula},\catvariableof{\secexformula}}
		& = \sbcontractionof{
			\onehotmapofat{1}{\catvariableof{\exformula\land\secexformula}} \otimes \onehotmapofat{1}{\catvariableof{\exformula}} \otimes \onehotmapofat{1}{\catvariableof{\secexformula}},\onehotmapofat{1}{\catvariableof{\exformula\land\secexformula}}
		}{\catvariableof{\exformula},\catvariableof{\secexformula}} \\
		& = \onehotmapofat{1}{\catvariableof{\exformula}} \otimes \onehotmapofat{1}{\catvariableof{\secexformula}} \, . 
	\end{align*}
	To show the second claim we use
	\begin{align*}
		(\exformula\land\secexformula)[\shortcatvariables] 
		&= \sbcontractionof{
			\rencodingofat{\exformula}{\catvariableof{\exformula},\shortcatvariables},
			\rencodingofat{\secexformula}{\catvariableof{\secexformula},\shortcatvariables},
			\rencodingofat{\land}{\catvariableof{\exformula\land\secexformula},\catvariableof{\exformula},\catvariableof{\secexformula}},
			\onehotmapofat{1}{\catvariableof{\exformula\land\secexformula}}
			}{\shortcatvariables} \\
		&  = \sbcontractionof{
			\rencodingofat{\exformula}{\catvariableof{\exformula},\shortcatvariables},
			\rencodingofat{\secexformula}{\catvariableof{\secexformula},\shortcatvariables},
			(\onehotmapofat{1}{\catvariableof{\exformula}}\otimes \onehotmapofat{1}{\catvariableof{\secexformula}})
			%\rencodingofat{\land}{\catvariableof{\exformula},\catvariableof{\secexformula},\catvariableof{\exformula\land\secexformula}}
			}{\shortcatvariables} \\
		&= \sbcontractionof{\exformula,\secexformula}{\shortcatvariables} \, . 
	\end{align*}
\end{proof}

A similar decomposition holds for negations, as we show next.

\begin{theorem}
	For any formula $\exformula$ we have
	\begin{align*}
		\sbcontractionof{
			\rencodingofat{\lnot}{\catvariableof{\exformula},\catvariableof{\lnot\exformula}},\onehotmapofat{1}{\catvariableof{\lnot\exformula}}
		}{\catvariableof{\exformula}}
		= \onehotmapofat{0}{\catvariableof{\exformula}} =  \onesat{\catvariableof{\exformula}} - \onehotmapofat{1}{\catvariableof{\exformula}} \, . 
	\end{align*}
	and
	\begin{align*}
		\sbcontractionof{
			\rencodingofat{\lnot}{\catvariableof{\exformula},\catvariableof{\lnot\exformula}},\onehotmapofat{0}{\catvariableof{\lnot\exformula}}
		}{\catvariableof{\exformula}}
		= \onehotmapofat{1}{\catvariableof{\exformula}} \, . 
	\end{align*}
	In particular, it holds that (see Figure~\ref{fig:ConNegDecomposition}b)
	\begin{align*}
		(\lnot\exformula)[\shortcatvariables] = \onesat{\shortcatvariables} - \formulaat{\shortcatvariables}  \, . 
	\end{align*}
\end{theorem}
\begin{proof}
	Using that for two dimensional variables we have $\onesat{\catvariable}=\onehotmapofat{0}{\catvariable}+\onehotmapofat{1}{\catvariable}\, .$
\end{proof}

% Usage
These theorems provide a mean to represent logical formulas by sums of one-hot encodings.
Since any propositional formula can be represented by compositions of negations and conjunctions, they are universal.
We further notice, that the resulting decomposition is a basis+ CP format, as further discussed in Chapter~\ref{cha:sparseTC}.
In Figure~\ref{fig:DecompositionExample} we provide an example of this decomposition.


\begin{figure}
\begin{center}
	\begin{tikzpicture}[scale=0.275] % , baseline = -3.5pt



\begin{scope}[shift={(-19,-1.7)}]
		%\draw[] (-1,2.2) ellipse (4 and 2.5);
	  	\draw  (-3,2) rectangle (1,4);
		\node at (-1,1.9) [above] {${\exformula}$};
		\draw (-2,2) -- (-2,0) node[midway,left] {$\catvariableof{1}$};
		\draw (0,2) -- (0,0) node[midway,right] {$\catvariableof{2}$};	
		\node at (3.5,2.2)[right]  {$=$};
\end{scope}

\draw[] (-4,0.5) ellipse (9 and 3);
   	\begin{scope}[shift={(-2,-2)}]
  		\draw[\skeletoncolor] (-8,2) rectangle (-4,4);
		\node at (-6,2) [above,\skeletoncolor] {$\ones$};
		\draw[\skeletoncolor] (-7,2) -- (-7,0) node[midway,left] {$\catvariableof{1}$};
		\draw[\skeletoncolor] (-5,2) -- (-5,0) node[midway,right] {$\catvariableof{2}$};
	\end{scope}
	
	\draw[\skeletoncolor] (-2,2) -- (-2,-2) node[midway,left] {$\catvariableof{1}$};
	\draw[\skeletoncolor](0,2) -- (0,-2) node[midway,right] {$\catvariableof{2}$};
	\node[\skeletoncolor] at (-4.5,0) [above] {$-$};
\draw[thick,dashed] (-4,-2.5) -- (5,-3.5);
%\draw[] (6,-2.5) -- (1,-2.5);

%% Into negation core
\draw[thick,dashed] (-1,4.7) -- (-1,2);%-4,3.5);

%% Into conjunction core
\draw[thick,dashed] (10,4.7) -- (10,-4);%(6,-2.5);


\begin{scope}[shift={(10,-6)}]
	\draw[] (-4,0.5) ellipse (9 and 3);
	\begin{scope}[shift={(-4,0)}]
		\renewcommand{\skeletoncolor}{red}
	\node at (0,0) [left,\skeletoncolor] {$\delta$};
	\draw[\skeletoncolor]  (0,2) -- (0,0);% node[midway,left] {$\placeholderof{1}$};
	\draw[fill,\skeletoncolor] (0,0) circle (0.25cm);
	\draw[\skeletoncolor] (0,0) -- (0,-2) node[midway,right] {$\catvariableof{2}$};
	\draw[\skeletoncolor] (5,2) to[bend left=20] (0,0);


	\draw[fill,\skeletoncolor] (-2,0.5) circle (0.25cm);
	\draw[\skeletoncolor] (3,2) to[bend left=20] (-2,0.5);
	\draw[\skeletoncolor] (-2,2) -- (-2,-2) node[midway,left] {$\catvariableof{1}$};
	\end{scope}
\end{scope}



\begin{scope}[shift={(0,5)}]
		%\draw[] (-1,2.2) ellipse (4 and 2.5);
	  	\draw  (-3,2) rectangle (1,4);
		\node at (-1,1.9) [above] {${\secexformula^{(1)}}$};
		\draw (-2,2) -- (-2,0) node[midway,left] {$\catvariableof{1}$};
		\draw (0,2) -- (0,0) node[midway,right] {$\catvariableof{2}$};	
\end{scope}

\begin{scope}[shift={(11,5)}]
		%\draw[] (-1,2.2) ellipse (4 and 2.5);
	  	\draw  (-3,2) rectangle (1,4);
		\node at (-1,1.9) [above] {${\secexformula^{(2)}}$};
		\draw (-2,2) -- (-2,0) node[midway,left] {$\catvariableof{1}$};
		\draw (0,2) -- (0,0) node[midway,right] {$\catvariableof{2}$};	
		
\end{scope}




\node at (14,0.5)[right]  {$=$};


\begin{scope}[shift={(29,0)}]

\begin{scope}[shift={(-7.5,-1.7)}]
		%\draw[] (-1,2.2) ellipse (4 and 2.5);
	  	\draw[]  (-3,2) rectangle (1,4);
		\node at (-1,1.9) [above] {${\secexformula^{(2)}}$};
		\draw[] (-2,2) -- (-2,0) node[midway,left] {$\catvariableof{1}$};
		\draw[] (0,2) -- (0,0) node[midway,right] {$\catvariableof{2}$};	
		\node at (2.25,2.2)[right]  {$-$};		
\end{scope}


\begin{scope}[shift={(1,-3.7)}]
%	\renewcommand{\skeletoncolor}{\conjunctioncolor}
	\node at (0,0) [left] {$\delta$};
	\draw[]  (0,2) -- (0,0);% node[midway,left] {$\placeholderof{1}$};
	\draw[fill] (0,0) circle (0.25cm);
	\draw[] (0,0) -- (0,-2) node[midway,right] {$\catvariableof{2}$};
	\draw[] (5,2) to[bend left=20] (0,0);


	\draw[fill] (-2,0.5) circle (0.25cm);
	\draw[] (3,2) to[bend left=20] (-2,0.5);
	\draw[] (-2,2) -- (-2,-2) node[midway,left] {$\catvariableof{1}$};
\end{scope}


\begin{scope}[shift={(1,-1.7)}]
		%\draw[] (-1,2.2) ellipse (4 and 2.5);
	  	\draw  (-3,2) rectangle (1,4);
		\node at (-1,1.9) [above] {${\secexformula^{(1)}}$};
		\draw[] (-2,2) -- (-2,0); % node[midway,left] {$\catvariableof{1}$};
		\draw[] (0,2) -- (0,0); % node[midway,right] {$\catvariableof{2}$};	
	
\end{scope}
	
\begin{scope}[shift={(6,-1.7)}]
		%\draw[] (-1,2.2) ellipse (4 and 2.5);
	  	\draw[]  (-3,2) rectangle (1,4);
		\node at (-1,1.9) [above] {${\secexformula^{(2)}}$};
		\draw[]  (-2,2) -- (-2,0); % node[midway,left] {$\catvariableof{1}$};
		\draw[]  (0,2) -- (0,0); % node[midway,right] {$\catvariableof{2}$};	
	
\end{scope}

\end{scope}



	
	
\end{tikzpicture}
\end{center}
\caption{
	Example of a decomposition by effective calculus of a formula $\exformula(\catvariableof{1},\catvariableof{2}) = \textcolor{blue}{\lnot} \secexformula^{(1)}(\catvariableof{1},\catvariableof{2}) \textcolor{red}{\land}  \secexformula^{(2)}(\catvariableof{1},\catvariableof{2})$ into a sum of contractions.}
	\label{fig:DecompositionExample}
\end{figure}




\subsection{Applications in Machine Learning}

The neural paradigm of Machine Learning describes the relevance of sparse function to be effective models in the sense of learning and approximation.

% Neural Paradigm by Tensor Network Decompositions
Our model of the neural paradigm are tensor network decompositions, seen as decomposition of functions into smaller functions, which take each other as input.
Summations along input axis are avoided, when having directed and binary tensor networks with basis calculus interpretation.

% Basis Calculus
We have already observed in Theorem~\ref{the:basisCalculus}, that the value of discrete maps can be calculated by contractions of the directed binary relation encodings.
This has been framed as Basis Calculus.
What is more, tensor network decompositions into directed binary tensors correspond with representation of functions as compositions of smaller functions.
We can understand each composition as marking a neuron in an architecture and thus have established a neural perspective on binary directed tensor networks.
 % Before sparse tensor calculus!

\section{Contraction Message Passing}\label{cha:localContractions}

In this chapter we introduce local contraction passed along tensor clusters to approximatively calculate global contractions.
These message passing schemes provide tradeoffs between efficiency increases and exactness of the global contraction.
We use the CP Decompositions to investigate the asymptotic behavior of the message passing algorithms.



\subsection{Exact Contractions}

%\red{This is the junction tree algorithm!}

We apply Theorem~\ref{the:splittingContractions} to split a contraction into subcontractions, which are consecutively performed.

% Message Passing
Contractions can be performed partially, and the result passed to the rest of the network as a message.

\subsubsection{Construction of Cluster Graphs}

% Cluster Graphs
\begin{definition}[Cluster Graph]
	Given a tensor network $\extnet$ a cluster partition is a partition of the tensor network into $n$ clusters, by a function
		\[ \alpha : \edges \rightarrow [n] \, . \]
	The clusters are with tensors decorated edge sets $\enc = \{\edge \, : \, \alpha(\edge) = i\}$ with variables $\nodes_i = \bigcup_{\edge \in \enc} \edge$.
	The clusters form a graph where edges between $\enc$ and $C_j$ exist, when the node sets $\nodes_i$ and $\nodes_j$ are not disjoint.
	In this case, we define separation sets $S_{i,j}=\enc\cup C_j$
\end{definition}

\begin{theorem}
	Given a tensor network $\extnet$ and a cluster graph.
	We then define for each cluster the node set
		\[ \tilde{\nodes}_i = \bigcup_{j\neq i} \nodes_j \]
	and have
		\[ \contractionof{\extnet}{\catvariableof{\secnodes}} = 
		\contractionof{
			\{ \contractionof{ \tnetof{\enc} }{\nodes_i \cap (\tilde{\nodes}_i\cup\secnodes)}  : i \in [n]\}
		}{\catvariableof{\secnodes}}  \, . \]
\end{theorem}
\begin{proof}
	By Theorem~\ref{the:splittingContractions} applied for each cluster seen as a subgraph.
\end{proof}



\subsubsection{Message Passing to calculate contractions}

% Cluster Graphs
Having a hypergraph $\graph$, we iteratively apply Theorem~\ref{the:splittingContractions} and call the $\graph_2$ a cluster.
When iterating until $\graph$ is empty, we get a cluster graph, where all tensors are assigned to a cluster.


% Cluster Trees -> Clique Trees in Koller Book
When the cluster are a polytree, that is a union of disjoint trees, we define messages between neighbored clusters $\enc$ and $\secenc$ with $\secenc\prec\enc$ by the contractions

\begin{align}
	\upmes{j}{i} = \contractionof{\{\upmes{\tilde{j}}{j} \, : \,  \thirdenc \prec \secenc\} \cup \tnetof{\secenc}}{\catvariableof{\nodes_i\cap \nodes_j}} \, .
\end{align}
and
\begin{align}
	\downmes{i}{j}  = \contractionof{\{\downmes{\tilde{j}}{i} \, : \,  \enc \prec  \thirdenc\} \cup \tnetof{\enc}}{\catvariableof{\nodes_i\cap \nodes_j}} \, .
\end{align}


We note, that the messages are well defined by these recursive equations, exactly when the cluster graph is a polytree.
%Since messages are recursively defined, we need the tree structure to ensure well-definedness.


\begin{lemma}
	When the cluster graph is a tree, we have for neighbored clusters $\enc$ and $\secenc$ with $\secenc\prec\enc$
		\[ \upmes{\secclusterenumerator}{\clusterenumerator} 
		= \contractionof{\{\tnetof{\thirdenc} \, : \, \thirdenc \prec \secenc \}}{\catvariableof{\nodes_i\cap\nodes_j}}   \]
	and
		\[ \downmes{\clusterenumerator}{\secclusterenumerator}
		= \contractionof{\{\tnetof{\thirdenc} \, : \, \enc \prec \thirdenc \}}{\catvariableof{\nodes_i\cap\nodes_j}}  \, . \]
\end{lemma}
\begin{proof}
	By induction over the cardinality of the preceding clusters.
	\paragraph{$n=1$}: Only a single cluster before, therefore trivial.
	\paragraph{$n+1\rightarrow n$}: Assuming the statement holds for up to $n$ preceding clusters, let there be $n+1$ preceding clusters.
	Then Theorem~\ref{the:splittingContractions} splits contractions into terms, which are by inductive assumption the messages.
\end{proof}


\begin{theorem}
	When the cluster graph is a tree, then we have for each cluster $\enc$ with neighbors $N(\clusterenumerator)$
%	Then for each clique we have the conditional probability of its variables being the contraction of the messages with the cliques cores, that is
	\begin{align}
		\contractionof{\extnet}{\nodes_i} = 
		\contractionof{
			\{ \upmes{\secclusterenumerator}{\clusterenumerator}  \, : \, j \in N(i) , \, \secenc\prec\enc \}  \cup 
			\{ \downmes{\secclusterenumerator}{\clusterenumerator}  \, : \, j \in N(i),  \, \enc\prec\secenc \} \cup \tnetof{\enc}
		}{\nodes_i} \, .
	\end{align}
\end{theorem}
\begin{proof}
	By Theorem~\ref{the:splittingContractions} we split into contractions of the clusters up and down of the respective neighbors and apply the above lemma.
\end{proof}





\subsubsection{Variable Elimination Cluster Graphs}


\begin{remark}[Construction of Cluster Graphs by Variable Elimination]
	% Build a cluster graph
	Following an elimination order of the colors, mark those tensors containing the colors, which have not been marked before, as the cluster.
	% Extension to clique tree
	A clique tree can be constructed by these cluster, when iterating through the clusters and either connect them to previous disconnected clusters or leave the current cluster disconnected.
	Add the disconnected clusters with the current cluster in case there are overlaps of their open colors.
	If the disconnected cluster added has more open colors, 
\end{remark}


\subsubsection{Bethe Cluster Graphs}


\begin{figure}[h]
\begin{center}
	\begin{tikzpicture}[scale=0.35, thick] % , baseline = -3.5pt




\begin{scope}[shift={(23,0)}]

\node[anchor=center] (text) at (-6,8) {$b)$};

\draw (2,8) rectangle (4,6);
\node[anchor=center] (text) at (3,7) {\small $\rencodingof{\lor}$};

\draw (2,5) rectangle (4,3);
\node[anchor=center] (text) at (3,4) {\small $\rencodingof{\land}$};

\draw (2,2) rectangle (4,0);
\node[anchor=center] (text) at (3,1) {\small $\datacoreof{c}$};

\draw (2,-1) rectangle (4,-3);
\node[anchor=center] (text) at (3,-2) {\small $\datacoreof{b}$};

\draw (2,-4) rectangle (4,-6);
\node[anchor=center] (text) at (3,-5) {\small $\datacoreof{a}$};

\draw (2,-7) rectangle (4,-9);
\node[anchor=center] (text) at (3,-8) {\small $\lambda$};


\draw[fill] (-3,-8.5) circle (0.25cm);
\node[anchor=center] (text) at (-4,-8.5) {\tiny $\indexset$};

\draw[] (-3,-8.5) to[bend left=-10]  (2,-8);
\draw[] (-3,-8.5) to[bend left=0]  (2,-5.5);
\draw[] (-3,-8.5) to[bend left=10]  (2,-2.5);
\draw[] (-3,-8.5) to[bend left=20]  (2,0.5);


\draw[fill] (-3,-5.5) circle (0.25cm);
\node[anchor=center] (text) at (-4,-5.5) {\tiny $a$};

\draw[] (-3,-5.5) to[bend left=-10]  (2,-4.5);
\draw[] (-3,-5.5) to[bend left=10]  (2,3.5);

\draw[fill] (-3,-2.5) circle (0.25cm);
\node[anchor=center] (text) at (-4,-2.5) {\tiny $b$};

\draw[] (-3,-2.5) to[bend left=-10]  (2,-1.5);
\draw[] (-3,-2.5) to[bend left=10]  (2,4);

\draw[fill] (-3,0.5) circle (0.25cm);
\node[anchor=center] (text) at (-4,0.5) {\tiny $c$};

\draw[] (-3,0.5) to[bend left=-10]  (2,1.5);
\draw[] (-3,0.5) to[bend left=10]  (2,6.5);

\draw[fill] (-3,3.5) circle (0.25cm);
\node[anchor=center] (text) at (-4.5,3.5) {\tiny $a\land b$};

\draw[] (-3,3.5) to[bend left=-10]  (2,4.5);
\draw[] (-3,3.5) to[bend left=10]  (2,7);

\draw[fill] (-3,6.5) circle (0.25cm);
\node[anchor=center] (text) at (-5.5,6.5) {\tiny $(a\land b)\lor c$};

\draw[] (-3,6.5) to[bend left=10]  (2,7.5);


\draw[dashed] (-0.5,-12) -- (-0.5,8);

\node[right] (text) at (0.5,-11) {$\tilde{\edges}$};
\node[left] (text) at (-1.5,-11) {$\Delta$};

\end{scope}


\node[anchor=center] (text) at (-2,8) {$a)$};

\newcommand{\conposseldec}{3,-5.5}

\draw[fill] (\conposseldec) circle (0.25cm);
\draw (\conposseldec) -- (3,-7.5) node[midway, right] {\tiny ${\indexset}$}; % Unclear, whether this is the best notation!
\draw[] (2,-7.5) rectangle (4, -9.5);
\node[anchor=center] (text) at (3,-8.5) {\small $\lambda$};

\draw[] (0,1) -- (0,-1) node[midway,left] {\tiny $a$};
\draw (-1,-1) rectangle (1, -3);
\node[anchor=center] (text) at (0,-2) {\small $\datacoreof{a}$};
\draw[] (0,-3) to[bend right=20] (\conposseldec);


\draw[] (3,1) -- (3,-1) node[midway,left] {\tiny $b$};
\draw (2,-1) rectangle (4, -3);
\node[anchor=center] (text) at (3,-2) {\small $\datacoreof{b}$};
\draw[] (3,-3) to[bend right=0]  (\conposseldec);


\draw[] (6,5) -- (6,-1) node[midway,left] {\tiny $c$};
\draw (5,-1) rectangle (7, -3);
\node[anchor=center] (text) at (6,-2) {\small $\datacoreof{c}$};
\draw[] (6,-3) to[bend left=20]  (\conposseldec);


\draw[] (1.5,5) -- (1.5,3) node[midway,left] {\tiny $a \land b $};
\draw (-1,3) rectangle (4, 1);
\node[anchor=center] (text) at (1.5,2) {\small $\rencodingof{\land}$};


\draw[] (3.5,9) -- (3.5,7) node[midway,left] {\tiny $(a \land b) \lor c $};
\draw (0,7) rectangle (7, 5);
\node[anchor=center] (text) at (3.5,6) {\small $\rencodingof{\lor}$};

%\draw[] (6,1) to[bend left=20]  (\conposseldec);


		


\end{tikzpicture}
\end{center}
\caption{Example of a Bethe Cluster Graph.
	a) Example of a Tensor Network $\tnetof{\graph}$, which represents the by $\lambda$ averaged evaluation of the formula $(a\land b)\lor c$ on data $\datamap$.
	b) Corresponding Bethe Cluster Hypergraph, which dual is bipartite by the sets $\Delta$ and $\tilde{\edges}$.
	}
\label{fig:betheDataExample} 
\end{figure}

By adding delta tensors to each node $\node\in\nodes$ and defining its leg variables by $\node^{\edge}$ for $\edge\in\edges$.
We mark each such delta tensor by a cluster in $\Delta^{\graph}$, as defined in the following (see also Figure~\ref{fig:betheDataExample}).

\begin{definition}
	Given a tensor network $\tnetof{\graph}$ on a decorated hypergraph $\graph$, we define the Bethe Cluster Hypergraph $\secgraph$ as
	$(\secnodes, \secedges \cup \Delta^{\graph})$ where we have
	\begin{itemize}
		\item Recolored Edges $\secedges = \{\tilde{\edge} \, : \, \edge\in \edges\}$ where $\tilde{\edge} = \{\node^{\edge} \, : \, \node\in\edge\}$, which decoration tensor has same coordinates as $\hypercoreof{\edge}$
		\item Nodes $\secnodes = \bigcup_{\edge\in\edges}\tilde{\edge}$ %$\secnodes = \bigcup_{\edge\in\edges}\{\node^{\edge} \, : \, \node\in\edge \}$ 
		\item Delta Edges $\Delta^{\graph} =  \big\{ \{\node^{\edge} \, : \, \edge\ni\node \} \, : \, \node\in\nodes \big\} $, each of which decorated by a delta tensor $\delta^{\{\node^{\edge} \, : \, \edge\ni\node \}}$
	\end{itemize}
\end{definition}

By Lemma~\ref{lem:deltification} this construction does not change contractions.

% Dual graph
The dual is bipartite, since any variable appears exactly in one cluster in $\secedges$ and in one cluster of $\Delta^{\graph}$.
This further makes the dual of the Bethe Cluster Hypergraph a proper graph (i.e. edges consistent of node pairs). 





\subsubsection{Computational Complexity}

\red{Tree-width here.}

Naive execution of $\contractionof{\tnetof{\graph}}{\secnodes}$: $\prod_{\node\in\nodes} \catdimof{\node}$ many products are built and summed up.
When splitting contractions into local subcontractions, the product can be turned into sums with tremendous decrease in complexity.









\subsection{Approximate Contractions}

We ignore that cluster graphs are not trees and perform contraction message passing along neighbored clusters.
For the contraction of basis tensor networks, this scheme still provides the exact contraction.

\subsubsection{Exact Message Passing for Directed and Binary Contractions}


%% Function Composition Perspective
A Tensor Network of directed and binary cores represents the evaluation of composed functions.
In a Message Passing Perspective each component (let us call them neurons) can be evaluated, when the evaluation of the ancestor neurons are known.

\begin{lemma}\label{lem:diracConBasis}
	\red{Required? Basis vector factorization suffices?}
	For any collection of categorical variables $\shortcatvariables$ with identical dimension and any $\catindexofin{0}$ we have
		\[ \sbcontractionof{\delta[\catvariableof{0},\ldots,\catvariableof{\catorder-1}]}{\indexedcatvariableof{0},\catvariableof{1},\ldots,\catvariableof{\catorder-1}} 
		= \bigotimes_{\catenumerator\in\{1,\ldots,\catorder-1\}} \onehotmapofat{\catindexof{0}}{\catvariableof{\catenumerator}} \, . \]
\end{lemma}
\begin{proof}
	Directly by sum decomposition of delta tensors.
\end{proof}

\begin{theorem}
	Let $\tnetof{\graph}$ be a tensor network on a directed acyclic hypergraph $\graph$, such that each tensor is Boolean and directed, and such that each variable is appearing only once as an outgoing variable of a hyperedge.
	We build a cluster graph by storing each edge as a cluster and use the topological order on $\graph$.
	%We denote the scalar messages on the edges with no outgoing variables a single propagation of messages along the direction of the Bethe Cluster Graph by $\delta^{\edge}$ 
	%Then they coincide with the exact contractions leaving the variables of the edge open.	
	Then
		\[ \upmes{j}{i} = \contractionof{\tnetof{\graph}}{\catvariableof{\nodes_i\cap\nodes_j}} \, . \]
\end{theorem}
\begin{proof}
\red{Lemma above needed?}
	By using that each message is a basis vector (Using Theorem~\ref{the:conditionalContractionPreservation} in an induction argument) and can thus be splitted into the product of multiple copies.

	% Reducing to downcore
	Any hypercore, which is not a precessor to $\hypercoreof{\edge_{\clusterenumerator}}$ can be omitted from the contraction by a root-stripping argument using its directionality.
	Therefore we have
	\begin{align*}
		\contractionof{\tnetof{\graph}}{\catvariableof{\nodes_i\cap\nodes_j}}
		= \contractionof{\{\hypercoreof{\edge_{\thirdclusterenumerator}} \, : \, \edge_{\thirdclusterenumerator} \prec \edge_{i} \}}{\catvariableof{\nodes_i\cap\nodes_j}} \, . 
	\end{align*}

	We then replace each variable which is appearing more than once in outgoing legs by a delta tensor, which does not change the contraction by Lemma~\ref{lem:deltification}.
	%When there are undirected loops in the network beyond the cluster, we apply Lemma~\ref{lem:diracConBasis} to replace the variable by its copies and a delta tensor.
	
	We then follow a leaf stripping argument and apply Theorem~\ref{the:splittingContractions} iteratively on the remaining leaves. 
	Along that line, the leave and its successors are contracted.
	The contraction is a basis vector and can therefore be represented as an outer product of basis vectors.

\end{proof}



When replacing $\onehotmapof{\catindexof{0}}$ by an arbitrary vector in Lemma~\ref{lem:diracConBasis} we have
	\[ \sbcontractionof{\delta[\catvariableof{0},\ldots,\catvariableof{\catorder-1}], V[\catvariableof{0}]}{\catvariableof{1},\ldots,\catvariableof{\catorder-1}} 
		\neq \bigotimes_{\catenumerator\in\{1,\ldots,\catorder-1\}} V[{\catvariableof{\catenumerator}}]  \, . \]
Therefore, the messages will in general differ from the exact contractions.
To provide intuition of what happens in this case, let us take the following cases into account:
\begin{itemize}
	\item $\lambda\cdot\onehotmapof{\catindex}$ sent multiple times: Result gets a factor of $\lambda^{\# \text{copies}}$ compared with the exact contraction.
	\item $\onehotmapof{\catindex}+\onehotmapof{\tilde{\catindex}}$: Result is the exact contraction added by the crossterms of sending $\onehotmapof{\catindex}$ in one and $\onehotmapof{\tilde{\catindex}}$ in the other direction.
\end{itemize}	



\subsubsection{Case of Matrices}

\red{Here a toy example of cycling messages starting with $\ones$.
When normating the messages, the maximal singular vectors will be dominant.}

We investigate the Bethe message passing for a tensor network consisting of a single matrix.

\begin{theorem}
	The stable messages are the linear subspace of the maximal singular values of the fixed core $\hypercore$.
\end{theorem}
\begin{proof}
	Having a Singular Value Decomposition of $\hypercore$ and decompose the messages in the orthonormal system of the respective singular vectors.
\end{proof}


For the propagation of $a$ and $b$ on binary $\exformula(a, b)$ starting with trivial messages of $a$ and $b$ the above theorem implies:
\begin{itemize}
	\item Case of single possible world: Exact 
		$\exformula(a, b) \in \{ a \land b, a \land \lnot b, \lnot a \land b, \lnot a \land \lnot b \}$
		Messages are after first iteration exact
	\item Case of two possible worlds and $\exformula(a, b) \in \{ a \Leftrightarrow b, a \Leftrightarrow \lnot b, \lnot a \Leftrightarrow b, \lnot a \Leftrightarrow \lnot b \}$. 
		In this situation any start message is stable and determines the other.
	\item Case of two possible worlds and $\exformula(a, b) \in \{ a, b, \lnot a, \lnot b\}$.
		In this situation one stable message is determined by the specified atom and the other is always stable.
	\item Case of  three possible worlds: Approximative (exact: $1/3, 2/3$, approximative: $1-golden ratio, golden ratio$
	\item Case of four possible worlds: Exact ($\ones$) 
\end{itemize}


\subsubsection{Case of Tensors}

Let there now be a single tensor of arbitrary order.
When the tensor is not ODECO, we cannot find a CP-Decomposition with leg vectors building an orthonormal system in the respective leg spaces.
This prohibits direct application of the same techniques in the case of a matrix, which is always ODECO.



\subsection{Basis Calculus}

Message Passing of directed and binary message by relational encoding of functions can be interpreted as function evaluation.
This is because any relational encoding of a function, the decomposition
\begin{align*}
	\rencodingof{\exfunction} = \sum_{y \in \imageof{\exfunction}} ( \sum_{i: \exfunction(i)=y}\onehotmapof{i} )  \otimes \onehotmapof{y}
\end{align*}
is a SVD of the matrification of $\rencodingof{\exfunction}$ with respect to incoming and outgoing legs.


Passing a message $\onehotmapof{i}$ in direction thus gives the message $\onehotmapof{\exfunction(i)}$.


%After having established a one-to-one connection between the directed and binary tensors with the encoding of functions, we now interpret contractions as evaluations of the respective functions.
%Applying this insight iteratively on composed functions we show the following theorem.

\begin{remark}[Basis Calculus as Message Passing]
	Given a tensor network of directed and binary tensor cores $\hypercoreof{\edge}$, each representing a function $\exfunction^{\edge}$.
	When there are not directed cycles, we define the compositions of $\exfunction^{\edge}$ to be the function $\exfunction$ from the nodes $\nodes_1$ not appearing as incoming nodes to the nodes $\nodes_2$ not appearing as outgoing nodes in an edge.
	Choosing arbitrary $\catindexof{\node}\in[\catdimof{\node}]$ for $\node\in\nodes_1$ we have
	\begin{align*}
		\contractionof{\{\hypercoreof{\edge} \, : \edge\in\edges\}}{\nodes_2} = \onehotmapof{\exfunction(\catindexof{\node} \, : \, \node\in\nodes_1)}\, . 
	\end{align*}
\end{remark}
%\begin{proof}
%	Use a message passing argument for each function $\exfunction^{\edge}$.
%\end{proof}




\subsection{Applications}

% Application: Dynamic programming
When queries share same parts, can perform their contraction using dynamic programming.
For conditional probability queries, which variables are the clusters of a cluster tree, this results in belief propagation.




% CP sparsity
\section{Sparse Tensor Representations}\label{cha:sparseTC}

We in this chapter investigate, which sparsity notations enable tensors to be representable as contractions of tensor networks.


\subsection{CP Formats}

The CP Decomposition is one way to generalize the ranks of matrices to tensors.
It is oriented on the Singular Value Decomposition of matrices, providing a representation of the matrix as a weighed sum of the tensor product of singular vectors.
Given a tensor of higher order, each such tensor product is over multiple vectors, 

\begin{definition}\label{def:cpFormats}
	A CP Decomposition of rank $\decdim$ of a tensor $\hypercore\in\facspace$ is a collections of tensors $\scalarcoreat{\decvariable}$ and $\legcoreofat{\atomenumerator}{\decvariable,\catvariableof{\atomenumerator}}$ for $\atomenumeratorin$, where $\decvariable$ takes values in $[\decdim]$, such that
		\[  \hypercoreat{\shortcatvariables}
		= \contractionof{
		\{\scalarcoreat{\decvariable}\} \cup \{ \legcoreofat{\atomenumerator}{\decvariable,\catvariableof{\atomenumerator}} \, : \, \atomenumeratorin \}
		}{\shortcatvariables} \, . 
		\]
%	where for each $\decindexin$ and $\atomenumeratorin$ we have $\scalarcoreat{\decindex} \in \rr$ and $\legcoreof{\atomenumerator,\decindex}\in\rr^{\catindexof{\atomenumerator}}$.
	We say that the CP Decomposition is
	\begin{itemize}
		\item directed, when for each $\atomenumerator$ the core $\legcoreof{\atomenumerator}$ is directed with $\decvariable$ incoming and $\catvariableof{\atomenumerator}$ outgoing.
		\item binary, when for each $\atomenumerator$ the core $\legcoreof{\atomenumerator}$ is binary.
		\item basis, where we demand both properties, that is for each $\atomenumeratorin$ and $\decindexin$ 
			\[ \legcoreofat{\atomenumerator}{\inddecvar,\catvariableof{\atomenumerator}}\in \{ \onehotmapofat{[\catindexof{\atomenumerator}]}{\catvariableof{\atomenumerator}} \catindexof{\atomenumerator}\in[\catdimof{\atomenumerator}] \}\, . \]
		\item basis+, when for each $\atomenumeratorin$ and $\decindexin$  %$\legcoreof{\atomenumerator,\decindex}\in\onehotmapof{[\catindexof{\atomenumerator}]}$ or $\legcoreof{\atomenumerator,\decindex}=\ones$.
			\[ \legcoreofat{\atomenumerator}{\inddecvar,\catvariableof{\atomenumerator}}\in \{ \onehotmapofat{[\catindexof{\atomenumerator}]}{\catvariableof{\atomenumerator}} \catindexof{\atomenumerator}\in[\catdimof{\atomenumerator}] \} \cup \{\onesat{\catvariableof{\atomenumerator}}\}\, . \]
	\end{itemize}
	We denote by $\cprankof{\hypercore}$, respectively $\bincprankof{\hypercore}$, $\bascprankof{\hypercore}$ and $\baspluscprankof{\hypercore}$ the minimal cardinality such that $\hypercore$ has a CP Decomposition with directed cores, respectively binary cores, basis cores and basis+ cores.
\end{definition}

% Sum of elementary tensors
We have by definition
	\[ \hypercoreat{\shortcatvariables} = \sum_{\decindexin} \scalarcoreat{\inddecvar} \left( \bigotimes_{\atomenumeratorin} \legcoreofat{\atomenumerator}{\inddecvar,\catvariableof{\atomenumerator}} \right) \, . \]
The right side can be seen as an alternative definition of CP Decompositions by summations of elementary tensors.


\begin{figure}[h]
	\begin{center}
		\begin{tikzpicture}[scale=0.35, thick] % , baseline = -3.5pt

\begin{scope}[shift={(0,2)}]
	\draw[] (0,1)--(0,-1) node[midway,left] {\tiny $\catvariableof{0}$}; 
	\draw[] (1.5,1)--(1.5,-1) node[midway,left] {\tiny $\catvariableof{1}$}; 
	\node[anchor=center] (text) at (3,0) {$\cdots$};
	\draw[] (4,1)--(4,-1) node[midway,right] {\tiny $\catvariableof{\atomorder\shortminus1}$}; 
\end{scope}


\draw (-1,1) rectangle (5,-1);
\node[anchor=center] (text) at (2,0) {\small $\hypercore$};


\node[anchor=center] (text) at (7,0) {${=}$};


\begin{scope}[shift={(10,2)}]

\newcommand{\conposseldec}{4.5,-5.5}

\draw[fill] (\conposseldec) circle (0.25cm);
\draw (\conposseldec) -- (4.5,-7.5) node[midway, right] {\tiny ${\decvariable}$}; % Unclear, whether this is the best notation!
\draw[] (3.5,-7.5) rectangle (5.5, -9.5);
\node[anchor=center] (text) at (4.5,-8.5) {\small $\scalarcore$};

\draw[] (0,1) -- (0,-1) node[midway,left] {\tiny $\randomxof{0}$};
\draw (-1,-1) rectangle (1, -3);
\node[anchor=center] (text) at (0,-2) {\small $\legcoreof{0}$};
\draw[] (0,-3) to[bend right=20] (\conposseldec);


\draw[] (3,1) -- (3,-1) node[midway,left] {\tiny $\randomxof{1}$};
\draw (2,-1) rectangle (4, -3);
\node[anchor=center] (text) at (3,-2) {\small $\legcoreof{1}$};
\draw[] (3,-3) to[bend right=20]  (\conposseldec);

\node[anchor=center] (text) at (6,-2) {$\cdots$};

\draw[] (9,1) -- (9,-1) node[midway,left] {\tiny $\randomxof{\atomorder-1}$};
\draw (7.75,-1) rectangle (10.25, -3);
\node[anchor=center] (text) at (9,-2) {\small $\legcoreof{\atomorder-1}$};
\draw[] (9,-3) to[bend left=20]  (\conposseldec);

\end{scope}

		


\end{tikzpicture}
	\end{center}
	\caption{Tensor Network diagram of a generic CP decomposition (see Definition~\ref{def:cpFormats})}
\end{figure}

We introduce different notions of sparsities based on CP Decomposition with different properties of their leg cores.

\subsubsection{Directed Leg Cores}

This is the canonical CP Decomposition, where the vectors $\legcoreof{\atomenumerator,\decindex}$ are interpreted as generalized singular vectors.
Any CP decomposition can be transformed into a directed CP decomposition without enlarging the index set $\indexset$, simply by diving the vectors by their norms and multiplying it to $\scalarcoreat{\inddecvar}$.

% Directionality
We then have a partially directed Tensor Network representing the decomposed tensor.
The only undirected core is $\scalarcore$, since we do not demand it to be normed.
In many applications applications, however, also the $\scalarcore$ is directed with a single outgoing leg (see for example the empirical distributions as discussed in Section~\ref{sec:empDistribution}).
In that case, also the decomposed tensor is directed with outgoing legs.



\subsubsection{Basis Leg Cores}\label{sec:basisCP}

% From FOL Chapter: Bayesian Network interpretation of Basis CP
%	The basis CP can further be understood as a Bayesian network, where we understand $\dataindex$ as condition and each decomposition core as a conditional probability distribution.
%	We notice that in this interpretation the direction of the dependency is inversed compared with previous representation of grounding tensors in Figure~\ref{fig:groundingCP}. 


Directed and binary leg cores have incoming slices being basis vectors, we thus call them basis CP Decomposition.
This allows the interpretation of the directed and binary CP decomposition in terms of mapping to nonzero coordinates.
We start by defining the number of nonzero coordinates of tensors by the $\ell_0$-norm.

\begin{definition}
	The $\ell_0$-norm counts the nonzero coordinates of a tensor by
		\[ \sparsityof{\hypercore} = \#\big\{ \catindices \, : \, \hypercore_{\catindices }\neq 0 \big\} \, . \]
\end{definition}

The $\ell_0$-norm is not a proper norm itself, but the limit of $\ell_p$-norms (where $p \rightarrow 0$) of the flattened tensor (which are norms for $p\geq1$).

% Interpretation
The $\ell_0$ norm is the number of nonzero coordinates. 
We understand the leg cores as the relational encoding of functions mapping to the slices of these coordinates given an enumeration.
This is consistent with the previous analysis of Chapter~\ref{cha:directedTC}, where we characterized binary and directed cores by the encoding of associated functions.
Based on this idea, we can proof, that any tensor has a directed and binary CP decomposition with rand $\sparsityof{\hypercore}$.


\begin{theorem}\label{the:sparseBasisCP}
	For any tensor $\hypercore$ we have
		\[ \bascprankof{\hypercore} = \sparsityof{\hypercore} \, .  \]	
\end{theorem}
\begin{proof}
	We find a map 
		\[ \exfunction : [\sparsityof{\hypercore}] \rightarrow  \facstates \, , \] 
	which image is the set of nonzero coordinates of $\hypercore$.
	Denoting its image coordinate maps by $\exfunction^{\atomenumerator}$ we have
		\[ \hypercore = \sum_{\dataindexin} \scalarcoreof{\exfunction(\dataindex)} \left( \bigotimes_{\atomenumeratorin} \onehotmapof{\exfunction^\atomenumerator(\dataindex)} \right) \, . \]
	This is a basis CP Decomposition with rank $\sparsityof{\hypercore}$.
	Conversely, any basis CP Decomposition of $\hypercore$ with dimension $r$ would have at most $r$ coordinates different from zero and thus $\sparsityof{\hypercore}\leq r$.
	Thus, there cannot be a CP Decomposition with a dimension $r\leq\sparsityof{\hypercore}$.
\end{proof}

%
The next theorem relates the basis CP decomposition with encodings of $\atomorder$-ary relations (see Definition~\ref{def:daryRelation}).

\begin{theorem}
	If any only if $\hypercore\in\facspace$ has a basis decomposition with slices $\{\catindex_{[\atomorder]}^{\decindex} \, : \, \decindexin \}$ and trivial cores, it coincides with the encoding of the $\atomorder$-ary relation 
		\[ \exrelation = \{\catindex_{[\atomorder]}^{\decindex} \, : \, \decindexin \} \, . \]
%	To each basis CP decompositions with pairwise different slices and trivial scalar cores we find a $d$-ary relation, such that 
\end{theorem}


If in addition $\catdimof{\atomenumerator}=2$, we can interpret basis CP decompositions as propositional formulas.

% Knowledge Bases
\begin{theorem}
	If $\hypercore\in\atomspace$ has a basis decomposition with slices $\{\catindex_{[\atomorder]}^{\decindex} \, : \, \decindexin \}$ and trivial cores, it coincides with the propositional formula
		\[ \formulaat{\shortcatvariables} = 
		\bigvee_{\decindexin} \termof{\catindex_{[\atomorder]}^{\decindex}} \, . \]
\end{theorem}
\begin{proof}
	This is a generalization of Theorem~\ref{the:maximalClausesRepresentation}, which follows from Theorem~\ref{the:tensorToMaxMinTerms}.
\end{proof}


% Storage
The storage demand of any CP decomposition is at most linear in the dimension and the sum of its leg dimension.
When we have a basis CP decomposition, this demand can be further improved.
The basis vectors can be stored by its preimage of the one hot encoding $\onehotmapof{\cdot}$, that is the number of the basis vector in $[\catdim]$.
This reduces the storage demand of each basis vector to the logarithms of the space dimension without the need of storing the full vector.

% Matrix Representation
More precisely, we can store the CP Decomposition by the matrix
	\[ \matrixat{\decvariable,\selvariable} \in \rr^{\datanum \times (\atomorder+1)} \]
defined for $\atomenumeratorin$
	\[ \matrixat{\inddecvar,\selvariable=\atomenumerator} 
	= \invonehotmapof{\legcoreofat{\atomenumerator}{\inddecvar,\catvariableof{\atomenumerator}}}\]
and
	\[ \matrixat{\inddecvar,\selvariable=\atomorder}  
	= \scalarcoreat{\inddecvar} \, . \]
	
This is a typical tabular format to store relational databases.

\subsubsection{Basis+ Leg Cores}

The minimal rank of CP Decompositon is closely related to polynomial sparsity of the map $\hypercore$, which we will define next.

\begin{definition}\label{def:polynomialSparsity}
	A monomial decomposition of a tensor $\hypercore\in\facspace$ is a set $\sliceset$ of tuples $\slicetupleof{}$ where $\slicescalar\in\rr, \variableset\subset[\atomorder]$ and $\catindexof{\variableset}\in\bigtimes_{\atomenumerator\in\variableset} [\catdimof{\atomenumerator}]$ such that
	\begin{align}\label{eq:decIntoMonomials}
		\hypercoreat{\shortcatvariables} = \sum_{\slicetupleof{}\in\sliceset} \slicescalar \cdot \contractionof{\onehotmapof{\catindexof{\variableset}}}{\shortcatvariables} \, .
	\end{align}
	For any tensor $\hypercore\in\facspace$ we define its polynomial sparsity of order $\sliceorder$ as
	\begin{align*}
		\slicerankwrtof{\sliceorder}{\hypercore} =
		 \min \left\{ \cardof{\sliceset} \, : \, 
		 	\hypercoreat{\shortcatvariables} = \sum_{\slicetupleof{}\in\sliceset} \slicescalar \cdot \contractionof{\onehotmapof{\catindexof{\variableset}}}{\shortcatvariables} \, , \, \forall_{\slicetupleof{}\in\sliceset} \cardof{\variableset} \leq \sliceorder \, . 
		 \right\}
	\end{align*}
\end{definition}


% Explanation of monomials
We refer to the terms in a decomposition \eqref{eq:decIntoMonomials} in Definition~\ref{def:polynomialSparsity} as monomials of binary variables, which are enumerated by pairs $(\atomenumerator,\catindexof{\atomenumerator})$ and indicate whether the variable $\catvariableof{\atomenumerator}$ is in state $\catindexof{\atomenumerator}\in[\catdimof{\atomenumerator}]$.
Such indicators are represented by the one-hot encodings
	\[ \onehotmapofat{\catindexof{\atomenumerator}}{\catvariableof{\atomenumerator}} \, . \]
The monomial of multiple such binary variables indicated, whether all variables labelled by a set $\variableset$ are in the state $\catvariableof{\variableset}$, which is represented by
	\[ \onehotmapofat{\catindexof{\variableset}}{\catvariableof{\variableset}} = \bigotimes_{\atomenumerator\in\variableset} \onehotmapofat{\catindexof{\atomenumerator}}{\catvariableof{\atomenumerator}}  \, . \]
The states of the variables labeled by $\atomenumerator\in[\atomorder]/\variableset$ are not specified in the monomial and the monomial is trivially extended to
	\[ \contractionof{\onehotmapof{\catindexof{\variableset}}}{\shortcatvariables}  = \onehotmapofat{\catindexof{\variableset}}{\catvariableof{\variableset}} \otimes \onesat{\catvariableof{[\atomorder]/\variableset}} \, .   \]


% Infinite rank
\red{
For some $\sliceorder<\catorder$ there are tensors $\hypercoreat{\shortcatvariables}$, which do not have a monomial monomial decomposition with $\cardof{\variablesetof{\decindex}} \leq \sliceorder$ for all $\decindex$.
In that case we the minimum is over an empty set and we define $\slicerankwrtof{\sliceorder}{\hypercore}=\infty$.
If $\sliceorder\geq\catorder$, the $\slicerankwrtof{\sliceorder}{\hypercore}$ is always finite, since it coincides with the basis+ CP rank, as we show next.
}


%% Interpretation as Monomial Sparsity
%Each tensor $\lambda \cdot \contractionof{\onehotmapof{\catindexof{\variableset}}}{\shortcatvariables}$ is a by $\lambda\in\rr$ weighted monomial of the variables (or their negations) in $\variableset$.
%The polynomial sparsity is thus the minimal number of monomials that sum up to the function $\hypercore$.
%Given any monomial decomposition of a tensor, we can alternatively write
%	\begin{align*}
%		\hypercore = \sum_{\variableset\subset[\atomorder]} \sum_{\catindexof{\variableset}\in \indexsetof{\variableset}}  
%			\scalarcoreat{\variableset, \catindexof{\variableset}} \left( \prod_{\atomenumeratorin} \catvariableof{\atomenumerator} == (\catindexof{\variableset})_{\atomenumerator} \right) \, . 
%	\end{align*}
%where by $ \catvariableof{\atomenumerator} == (\catindexof{\variableset})_{\atomenumerator}$ we denote the atomic variables, indicating whether the variable $\catvariableof{\atomenumerator}$ is in state $ (\catindexof{\variableset})_{\atomenumerator}$.
%
%% Failing to be directed.
%Note, that the leg cores fail to be directed, when for some $\variableset\neq[\atomorder]$ the set $\indexsetof{\variableset}$ is not empty.


\begin{theorem}
	For any tensor $\hypercore\in\facspace$ we have
		\[ \slicesparsityof{\hypercore} = \baspluscprankof{\hypercore} \, . \]
	When $\catindexof{\atomenumerator}=2$ for all $\atomenumeratorin$, we also have
		\[ \bincprankof{\hypercore} = \slicesparsityof{\hypercore}  \, . \]
\end{theorem}
\begin{proof}
	To proof the first claim, we construct a basis+ CP decomposition given a monomial decomposition and vice versa.
	Let there be a tensor $\hypercoreat{\shortcatvariables}$ with a monomial decomposition by $\sliceset$ with $\cardof{\sliceset}=m$ and let us enumerate the elements in $\sliceset$ by $\slicetupleof{\decindex}$ for $\decindexin$.
	 We define for each $\atomenumeratorin$ the tensors
	 \begin{align*}
		\legcoreofat{\atomenumerator}{\decvariable,\catvariableof{\atomenumerator}}
		 = \left( \sum_{\decindexin \, : \, \atomenumerator\in\variableset} \onehotmapofat{\decindex}{\decvariable} \otimes \onehotmapofat{\catindexof{\atomenumerator}^{\decindex}}{\catvariableof{\atomenumerator}} \right)
		 + \left(\sum_{\decindexin \, : \, \atomenumerator\notin\variableset} \onehotmapofat{\decindex}{\decvariable} \otimes \onesat{\catvariableof{\atomenumerator}} \right)
	\end{align*}
	and 
	\begin{align*}
		\scalarcoreat{\decvariable} = \sum_{\decindexin} \slicescalar^{\decindex} \cdot \onehotmapofat{\decindex}{\decvariable}
	\end{align*}
	 and notice that	 
	\begin{align*}
		\hypercoreat{\shortcatvariables} 
		& = \sum_{\decindexin} \slicescalar^{\decindex} \cdot \contractionof{\onehotmapof{\catindexof{\variableset}^{\decindex}}}{\shortcatvariables} \\
		& = \sum_{\decindexin} \left(  \scalarcoreat{\inddecvar} \cdot \bigotimes_{\atomenumeratorin} \legcoreofat{\atomenumerator}{\inddecvar, \catvariableof{\atomenumerator}} \right) \\
		& = \contractionof{
		\{\scalarcoreat{\decvariable}\} \cup \{\legcoreofat{\atomenumerator}{\decvariable,\catvariableof{\atomenumerator}} \, : \, \atomenumeratorin \}
		}{\shortcatvariables} \, . 
	\end{align*}
	By construction this is a basis+ CP decomposition with rank $\decdim$.
	Since any monomial decomposition can be transformed into a basis+ CP decomposition with same rank we have
	\begin{align*}
		\slicesparsityof{\hypercore} \geq \baspluscprankof{\hypercore} \, . 
	\end{align*}
	
	Let there now be a basis+ CP decomposition we define for each $\decindexin$ 
	\begin{align*}
		\variableset^{\decindex} = \{\atomenumeratorin : \legcoreofat{\atomenumerator}{\inddecvar, \catvariableof{\atomenumerator}} \neq \onesat{\catvariableof{\atomenumerator}} \}
		 \quad \text{and} \quad 
		 \catindexof{\variableset}^{\decindex} = \{\invonehotmapof{\legcoreofat{\atomenumerator}{\inddecvar, \catvariableof{\atomenumerator}} } \, : \atomenumerator\in\variableset\}
	\end{align*}
	where by $\invonehotmapof{\cdot}$ we denote the inverse of the one-hot encoding.
	
	We notice that this is a monomial decomposition of $\hypercoreat{\shortcatvariables}$ to the tuple set
	\begin{align*}
		\sliceset = \{(\scalarcoreat{\inddecvar}, \variableset^{\decindex}, \catindexof{\variableset^{\decindex}}^{\decindex} ) \, : \, \decindexin \} \, . 
	\end{align*}
	It follows from this that
	\begin{align*}
		\slicesparsityof{\hypercore} \leq \baspluscprankof{\hypercore} \, 
	\end{align*}
	and the first claim is shown.
	
	The second claim follows from the observation, that whenever $\catindexof{\atomenumerator}=2$ for all $\atomenumeratorin$ the binary CP decompositions with non-vanishing slices $\legcoreofat{\atomenumerator}{\inddecvar, \catvariableof{\atomenumerator}}$ for $\atomenumeratorin$ and $\decindexin$ are also basis+ CP decompositions and vice versa.
%
%	
%
%	We proof the claim by establishing a one-to-one map between any binary CP decomposition of a a tensor $\hypercore$ and a monomial decomposition of $\hypercore$.
%	% CP Decomposition to monomial decomposition
%	Let there be a binary CP Decomposition of $\hypercore$ with the leg tensors $\{\legcoreof{\atomenumerator}\, :\, \atomenumeratorin\}$ and the scalar core $\scalarcore$.
%	For any index $\decindexin$ and $\atomenumeratorin$ we have
%		\[ \legcoreof{\atomenumerator}_{\decindex} \in \{\onehotmapof{0},\onehotmapof{1},\ones\} \, . \]
%	We define for any $\decindexin$ the sets 
%		\[ \variableset^{\decindex}=\big\{\atomenumerator \, : \, \legcoreof{\atomenumerator}_{\decindex} \in \{\onehotmapof{0},\onehotmapof{1}\}  \big\}\]
%	and an index tuple $\catindexof{\variableset}^\decindex \in \bigotimes_{\atomenumerator\in\variableset}[2]$ by
%	\begin{align*}
%		(\catindexof{\variableset}^\decindex)_\atomenumerator = 
%		\begin{cases} 
%			0 & \text{  if  } \legcoreof{\atomenumerator}_{\decindex} = \onehotmapof{0} \\1 & \text{  if  } \legcoreof{\atomenumerator}_{\decindex} = \onehotmapof{1} 
%		\end{cases} \, . 
%	\end{align*}   
%	Then we have by construction that 
%	\begin{align*}
%		\hypercore = \sum_{\decindexin} \scalarcoreat{\decindex} \cdot \left( \onehotmapof{\catindexof{\variableset}^\decindex} \otimes \onesof{[\atomorder]/\variableset^{\decindex}} \right) \, . 
%	\end{align*}
%	When regrouping the sum over the decomposition index by a sum over possible sets $\variableset\subset[\atomorder]$ and a sum over appearing index tuples $\catindexof{\variableset}$, this is a monomial decomposition of $\hypercore$ with
%		\[ \#\left(\indexset\right) = \# \left( \bigcup_{\variableset\subset[\atomorder]} \indexsetof{\variableset} \right) \, . \]
%	 % Monomial decomposition to CP Decomposition
%	 Vise versa we can construct a binary CP Decomposition given any monomial decomposition of $\hypercore$ 
%	 	\[ \hypercore 
%			= \sum_{\variableset\subset[\atomorder]} \sum_{\catindexof{\variableset}\in \indexsetof{\variableset}}  
%			\scalarcoreat{\variableset, \catindexof{\variableset}} \cdot \left( \onehotmapof{\catindexof{\variableset}} \otimes \onesof{[\atomorder]/\variableset} \right)  \, . \]
%	 To this end, we enumerate the set $\bigcup_{\variableset\subset[\atomorder]} \indexsetof{\variableset}$ by an additional index $\decindexin$ and define leg cores by
%	\begin{align*}
%		\legcoreof{\atomenumerator}_{\decindex} = 
%		\begin{cases} 
%			\onehotmapof{0} & \text{  if  }  \atomenumerator \in \variableset^{\decindex} \text{  and  } (\catindexof{\variableset}^\decindex)_\atomenumerator = 0 \\
%			\onehotmapof{1} & \text{  if  } \atomenumerator \in \variableset^{\decindex} \text{  and  } (\catindexof{\variableset}^\decindex)_\atomenumerator = 1 \\
%			\ones & \text{  if   } \atomenumerator \notin \variableset^{\decindex}
%		\end{cases} 
%	\end{align*}   
%	and a scalar core by coordinates
%		\[ \scalarcoreat{\decindex} = \scalarcoreat{\variableset^{\decindex},\catindexof{\variableset}^\decindex} \, . \]
%	Then, the monomial decomposition coincides with a basis CP Decomposition with the same dimension.
%	Thus, both $\bincprankof{\hypercore}$ and $\slicesparsityof{\hypercore}$ are the minima of identical sets and thus identical.
\end{proof}



\begin{example}[Propositional Formulas]
	When all leg dimensions of a binary tensor $\hypercore$ are $2$, we can interpret $\hypercore$ as a logical formula.
	We can use the binary CP decomposition of any tensor $\sechypercore$ with $\nonzeroof{\sechypercore}=\hypercore$ as a CNF of $\hypercore$.
	Finding the sparsest CNF thus amounts to finding the $\sechypercore$ with minimal $\slicesparsityof{\sechypercore}$ such that $\nonzeroof{\sechypercore}=\hypercore$.
\end{example}






\subsection{Representation involving selection architectures}

The set of slice-sparse tensors coincides with the expressivity of specific selection architecture.
We first define a slice selecting tensor and then show its decomposition into a formula selecting neural network.

\begin{definition}
	Given a set of atomic variables $\shortcatvariables$, a slice selecting tensor of maximal cardinality $\sliceorder$ is the tensor
		\[ \fselectionmapat{\shortcatvariables,\selvariableof{0,0},\ldots,\selvariableof{\sliceorder-1,0},\selvariableof{0,1},\ldots,\selvariableof{\sliceorder-1,1}} \]
	with dimensions
		\[ \seldimof{\selenumerator,0} = 2, \seldimof{\selenumerator,1} = \atomorder \]
	and coordinates
	\begin{align*}
		& \fselectionmapat{\shortcatvariables=\shortcatindices,\indexedselvariableof{0,0},\ldots,\indexedselvariableof{\sliceorder-1,0},\indexedselvariableof{0,1},\ldots,\indexedselvariableof{\sliceorder-1,1}} \\
		& \quad = \begin{cases}
			1 & \text{if} \quad 
			\forall_{\atomenumerator,\selenumerator} : \big(  \selindexof{\selenumerator,1} = \atomenumerator \land \selindexof{\selenumerator,0} \neq 2 \big) \Rightarrow  (\selindexof{\selenumerator,0} = \catindexof{\atomenumerator})  \\
			0
		\end{cases} \, . 
	\end{align*}
\end{definition}

In the next two Lemmata we first show that the defined slice selecting tensors indeed selects slices and then provide a representation as a formula selecting network.

\begin{lemma}\label{lem:sliceFromSliceSelector}
	If all input neurons with same selection index are agreeing on the connective index, the selected formula does not vanish and coincides with a slice to the set
		\[ \variableset = \{ \atomenumerator \, : \, \exists_{\selenumeratorin}: \selindexof{\selenumerator,1} = \atomenumerator \land \selindexof{\selenumerator,0} \neq 2 \} \]
	and for $\atomenumerator\in\variableset$
		\[ \catindexof{\atomenumerator} = \selindexof{\selenumerator,0} \quad \text{if} \quad \selindexof{\selenumerator,1} = \atomenumerator \, . \]
\end{lemma}
\begin{proof}
	We need to show that 
	\begin{align}\label{eq:sliceFromSliceSelector}
	  	\fselectionmapat{\shortcatvariables,\indexedselvariableof{0,0},\ldots,\indexedselvariableof{\sliceorder-1,0},\indexedselvariableof{0,1},\ldots,\indexedselvariableof{\sliceorder-1,1}} = \onehotmapofat{\catindexof{\variableset}}{\shortcatvariables} \, . 
	\end{align}
	If and only if an index $\tilde{\catindex}_{[\atomorder]}$ reduced on $\variableset$ does not coincide with $\catindexof{\variableset}$, we have $\onehotmapofat{\catindexof{\variableset}}{\shortcatvariables=\tilde{\catindex}_{[\atomorder]}}=0$ and otherwise $\onehotmapofat{\catindexof{\variableset}}{\shortcatvariables=\tilde{\catindex}_{[\atomorder]}}=1$ .
	Let us notice, that this condition is equivalent to 
		\[ \forall_{\atomenumerator,\selenumerator} : \big(  \selindexof{\selenumerator,1} = \atomenumerator \land \selindexof{\selenumerator,0} \neq 2 \big) \Rightarrow  (\selindexof{\selenumerator,0} = \catindexof{\atomenumerator}) \]
	and thus \eqref{eq:sliceFromSliceSelector} holds.
\end{proof}


\begin{lemma}\label{lem:fsnnRepresentingSliceSelector}
	The slice selection tensor coincides with a formula selecting neural network with neurons (see Figure~\ref{fig:sliceSelectingNN}):
	\begin{itemize}
		\item unary input neuron enumerated by $\selenumerator$, selecting one of the $\shortcatvariables$ with the variable $\selvariableof{\selenumerator,1}$ and selecting a connective in $\{\lnot, \mathrm{Id}, \mathrm{True}\}$ by $\selvariableof{\selenumerator,0}$
		\item $\sliceorder$-ary output neuron fixed to the $\land$ connective.
	\end{itemize}
\end{lemma}
\begin{proof}
	This can be easily checked on each coordinate.
\end{proof}

It follows, that the expressivity of the slice selecting neural network coincides with the set of tensors with a bound on their slice sparsity, when $\sliceorder\geq\atomorder$.
For arbitrary $\sliceorder$, the following theorem holds.

\begin{theorem}
	Let $\fselectionmap$ be a slice selecting tensor.
	For any parameter tensor $\canparam$ we have
		\[ \baspluscprankof{\contractionof{\fselectionmap,\canparam}{\shortcatvariables}} \leq \bascprankof{\canparam} \, . \]
\end{theorem}
\begin{proof}
	Each non-vanishing coordinate of $\canparam$ represents by Lemma~\ref{lem:fsnnRepresentingSliceSelector} a slice and their weighted sum is thus a monomial decomposition.
%	\red{Use the above lemma for that on each slice.}
%	Show, how a coordinate of $\canparam$ corresponds with a slice determining tuple: $\sliceset$ determined by the selection indices.
\end{proof}


\begin{figure}[h]
	\begin{center}
		\begin{tikzpicture}[thick, scale=0.35] % , baseline = -3.5pt

\drawatomindices{0}{-6}
\draw (-1,1) rectangle (5, -5);
\node[anchor=center] (text) at (2,-2) {$\fselectionmap$};

%\draw[->] (2,-1)--(2,1) node[midway,right] {\tiny ${\atomicformulaof{\parindexof{1}} \land \atomicformulaof{\parindexof{2}}}$}; 

\draw[<-] (5,0.5) -- (7,0.5) node[midway, above] {\tiny $\selvariableof{0,0}$};
\draw[<-] (5,-1)--(7,-1) node[midway,above] {\tiny $\selvariableof{0,1}$}; 
\node[anchor=center] (text) at (6,-1.75) {$\vdots$};
\draw[<-] (5,-3)--(7,-3) node[midway,below] {\tiny $\selvariableof{\sliceorder\shortminus1,0}$}; 
\draw[<-] (5,-4.5)--(7,-4.5) node[midway,below] {\tiny $\selvariableof{\sliceorder\shortminus1,1}$}; 

\draw (7,1) rectangle (9, -5);
\node[anchor=center] (text) at (8,-2) {$\canparam$};


		
\node[anchor=center] (text) at (12,-2) {${=}$};


\begin{scope}[shift={(17,8)}]

	\begin{scope}[shift={(0,-10)}]
	
		\draw (-1,7) rectangle (12, 9);	
		\node[anchor=center] (text) at (5.5,8) {$\land$};
		
		% First leg selector
		\draw[->] (1,5) -- (1,7) node[midway, right] {\tiny $\catvariableof{\lneuron_0}$};	
		\draw (-1,3) rectangle (3, 5);
		\node[anchor=center] (text) at (1,4) {$\rencodingof{\{\notucon,\iducon,\trueucon\}}$};
		
		\draw (3,4) -- (12,4);
		\draw[<-] (12,4) -- (14,4) node[midway, above] {\tiny $\selvariableof{0,0}$};
			
		% SelectorCores
		\draw[->] (1,1) -- (1,3) node[midway, left] {\tiny $\headvariableof{\vselectionsymbol,0}$};
		
		\draw (-1,1) rectangle (3, -1);
		\node[anchor=center] (text) at (1,0) {$\selectorcoreof{0}$};
		\draw (3,0) to[bend left=10]  (12,2.5);
		\draw[<-] (12,2.5) -- (14,2.5) node[midway, above] {\tiny $\selvariableof{0,1}$};
		
		\draw[<-] (0,-1)--(0,-3) node[midway,left] {\tiny $\catvariableof{0}$}; 
		\node[anchor=center] (text) at (1,-2) {$\cdots$};
		\draw[<-] (2,-1)--(2,-3) node[midway,right] {\tiny $\catvariableof{\atomorder\shortminus1}$}; 
		
		\node[anchor=center] (text) at (5.5,0) {$\cdots$};

		% Second leg selector
		\draw (10,5) --(10,7);
		\draw[->] (10,1) -- (10,5) node[midway, right] {\tiny $\catvariableof{\lneuron_{\sliceorder\shortminus 1}}$};	
		\draw (8,-1) rectangle (12, 1);
		\node[anchor=center] (text) at (10,0) {$\rencodingof{\{\notucon,\iducon,\trueucon\}}$};
		\draw[<-] (12,0) -- (14,0) node[midway, below] {\tiny $\selvariableof{\sliceorder\shortminus1,0}$};

%		\draw (9,-1) -- (9,1);
		\draw[->] (10,-3) -- (10,-1) node[midway, left] {\tiny $\headvariableof{\vselectionsymbol,\sliceorder\shortminus1}$};
		\draw (8,-5) rectangle (12, -3);
		\node[anchor=center] (text) at (10,-4) {$\selectorcoreof{\sliceorder-1}$};
		\draw[<-] (12,-4) -- (14,-4) node[midway, below] {\tiny $\selvariableof{\sliceorder\shortminus1,1}$};
		
		
%		\draw[<-] (9,-1)--(9,-3) node[midway,left] {\tiny $\catvariableof{0}$}; 
%		\node[anchor=center] (text) at (1,-2) {$\cdots$};
%		\draw[<-] (11,-1)--(11,-3) node[midway,right] {\tiny $\catvariableof{\atomorder\shortminus1}$}; 
		%\drawatomindices{7}{-4}	
		
		
		
		% ParameterCores
		\draw (14,5) rectangle (16, -5);
		\node[anchor=center] (text) at (15,0) {$\canparam$};
		
		\node[anchor=center] (text) at (13,1.5) {$\vdots$};
		
		
		\drawvariabledot{7.5}{-7}
		\drawvariabledot{3.5}{-7}

		\draw[] (3.5,-7) to[bend left=25] (0,-3);
		\draw[->] (3.5,-7) to[bend right=10] (9,-5);
		
		\draw[] (7.5,-7) to[bend left=10] (2,-3);
		\draw[->] (7.5,-7) to[bend right=25] (11,-5);


		\draw[->] (3.5,-9)--(3.5,-7) node[midway,left] {\tiny $\catvariableof{0}$}; 
		\node[anchor=center] (text) at (5.5,-8) {$\cdots$};
		\draw[->] (7.5,-9)--(7.5,-7) node[midway,right] {\tiny $\catvariableof{\atomorder\shortminus1}$}; 
		
%		\begin{scope}[shift={(-3.5,8)}]
%			\draw[fill] (7.5,-15) circle (0.25cm);
%			\draw[] (7.5,-15) to[bend left=25] (3.5,-13);
%			\draw[] (7.5,-15) to[bend right=25] (10.5,-13);
%
%			%\draw[fill] (9,-15.25) circle (0.25cm);
%			%\draw[] (9,-15.25) to[bend left=25] (5,-13);
%			%\draw[] (9,-15.25) to[bend right=25] (12,-13);
%
%			\draw[fill] (11.5,-15) circle (0.25cm);
%			\draw[] (11.5,-15) to[bend left=25] (7.5,-13);
%			\draw[] (11.5,-15) to[bend right=25] (14.5,-13);
%
%			%\drawatomindices{7.5}{-16}
%
%		\end{scope}
	
	\end{scope}

\end{scope}

\end{tikzpicture}
	\end{center}
	\caption{Representation of a basis+ Tensor by the contraction of a parameter tensor $\canparam$ with a slice selecting architecture $\fselectionmap$, which has a decomposition as a formula selecting neural network (see Lemma~\ref{lem:fsnnRepresentingSliceSelector}).
	The nonzero coordinates of $\canparam$ represent the (see Lemma~\ref{lem:sliceFromSliceSelector}).}
\end{figure}\label{fig:sliceSelectingNN}


\subsubsection{Applications}

One application is as a parametrization scheme in the approximation of a tensor by a slice-sparse tensor, see Chapter~\ref{cha:tensorApproximation}.


\red{
The approximated parameter can then be used as a proxy energy to be maximized.
When choosing $\sliceorder=2$, the approximating tensor contains only quadratic slices, which then poses a QUBO problem.
}

\begin{remark}[Extension to arbitrary CP-formats]
	Select at each input neuron a specific leg.
	For finite number of legs, as it is the case in the binary, basis and basis+ formats, we can enumerate all possibilities by the selection variable.
	For the basis+ format, in case of binary leg dimensions, we here exemplified the approach, by enumerating the three possibilities $\onehotmapof{0},\onehotmapof{1},\onesat{1}$.
	This approach, however, fails as a generic representation of the directed format, since the directed legs are continuous and there therefore are infinite choosable legs.
\end{remark}





\subsection{Constructive Bounds on CP Ranks}

After having defined three CP Decompositions, let us investigate bounds on their ranks which proofs come with constructions of the cores.


\subsubsection{Format Transformations}

%% Case of binary legs
Especially useful, when the leg dimensions are two, where the slice decomposition shows decomposition of the tensor into monomials.


\begin{theorem}\label{the:sliceToCP}
	For any tensor $\hypercoreat{\shortcatvariables}\in\facspace$ we have
		\[ \cprankof{\hypercore} 
		\leq \bincprankof{\hypercore} 
		\leq \baspluscprankof{\hypercore} 
		\leq \bascprankof{\hypercore} \, . \]
\end{theorem}	
\begin{proof}
	% First bound
	Since any CP decomposition into binary leg cores can be normed to a CP decomposition with directed leg cores, the first bound holds.
	% Second bound
	The second bound holds analogously, since any CP decomposition with basis leg cores is also a CP decomposition with binary leg cores.
\end{proof}

	%Taking only $\variableset=[\atomorder]$ and the index set being the nonzero coordinates of $\hypercore$ we get 
	%	\[ \slicesparsityof{\hypercore} \leq \#{\catindices: \hypercore_{\catindices}\neq 0} = \sparsityof{\hypercore} \, . \]

%% Tightness of the Bounds
Consider for example the tensor $\ones$ having maximal $\ell_0$-norm being the dimension of the tensor space, but, since it is elementary, a CP decomposition with rank $1$.


\subsubsection{Summation of CP Decompositions}

\begin{theorem}\label{the:CPrankSumBound}
	For any collections of tensors $\{T^{l}[\catvariableof{\nodes}] : l \in [n]\}$ with identical variables and scalars $\lambda^{l} \in \rr$ for $l\in[n]$  we have
		\[ \cprankof{\sum_{l \in [n]} \lambda^{l} \cdot T^{l}} \leq \sum_{l\in[n]}  \cprankof{T^{l}}  \, . \]
	The bound still holds, when we replace on both sides $\cprankof{\cdot}$ by $\bincprankof{\cdot}$, by $\bascprankof{\cdot}$ or by $\baspluscprankof{\cdot}$.
\end{theorem}
\begin{proof}
	Products with scalars do not change the rank, since they just rescale the core $\scalarcore$.
	The sum of CP Decomposition is just the combination of all slices, thus the rank is at most additive.
\end{proof}

\subsubsection{Contractions of CP Decompositions}

More general, we can bound the sparsity of any contraction by the product of sparsities of affected tensors.

\begin{theorem}\label{the:CPrankContractionBound}
	For any tensor network with variables $\nodes$ and edges $\edges$ we have for any subset $\secnodes\subset\nodes$
		\[ \cprankof{\contractionof{\{\hypercoreof{\edge} : \edge\in\edges \}}{\secnodes}} \leq 
		\prod_{\edge\in\edges \, : \, \secnodes\cap\edge \neq \varnothing} \cprankof{\hypercoreof{\edge}} \, . \]
	The bound still holds, when we replace on both sides $\cprankof{\cdot}$ by $\bincprankof{\cdot}$, by $\bascprankof{\cdot}$ or by $\baspluscprankof{\cdot}$.
\end{theorem}


Remarkably, in Theorem~\ref{the:CPrankContractionBound} the upper bound on the CP rank is build only by the ranks of the tensor cores, which have remaining open edges.
We prepare for its proof by first showing the following Lemmata.

\begin{lemma}\label{lem:sparsityGeneralContraction}
	For any tensors $\hypercoreofat{1}{\catvariableof{\nodes_1}}$ and $\hypercoreofat{2}{\catvariableof{\nodes_2}}$ and any set of variables $\secnodes\subset\nodes_1\cup\nodes_2$ we have
		\[ \cprankof{\contractionof{\{\hypercoreof{1},\hypercoreof{2}\}}{\secnodes}} \leq \cprankof{\hypercoreof{1}} \cdot \cprankof{\hypercoreof{2}} \, . \]
	The bound still holds, when we replace on both sides $\cprankof{\cdot}$ by $\bincprankof{\cdot}$, by $\bascprankof{\cdot}$ or by $\baspluscprankof{\cdot}$.
\end{lemma}
\begin{proof}
	By connecting the cores and restoring the binary or basis properties.
\end{proof}

When one core of the contracted tensor network does not contain variables which are left open, we can drastically sharpen the bound provided by Lemma~\ref{lem:sparsityGeneralContraction} as we show next.

\begin{lemma}\label{lem:sparsityDisjointContraction}
	For any tensor network consistent of two tensors $\hypercoreofat{1}{\catvariableof{\nodes_1}}$ and $\hypercoreofat{2}{\catvariableof{\nodes_2}}$ and any set $\secnodes$ with $\secnodes\cap\nodes_2=\varnothing$ we have
		\[ \cprankof{\contractionof{\{\hypercoreof{1},\hypercoreof{2}\}}{\secnodes}} \leq \cprankof{\hypercoreof{1}} \, . \]
	The bound still holds, when we replace on both sides $\cprankof{\cdot}$ by $\bincprankof{\cdot}$ or by $\bascprankof{\cdot}$.
\end{lemma}
\begin{proof}
	We show the lemma by constructing a CP decomposition of $\cprankof{\contractionof{\{\hypercoreof{1},\hypercoreof{2}\}}{\secnodes}} $ for any CP decomposition of $\hypercoreof{1}$.
	Let therefore take any CP decomposition of $\hypercoreof{1}$ consistent of the leg cores $\{\legcoreof{\node} \, : \, \node \in \nodes_1 \}$ and a scalar core $\scalarcore$.
	Then we define a new $\scalarcore$ by
		\[ \tilde{\scalarcore} = \contractionof{\{\scalarcore\}\cup \{\legcoreof{\node} \, : \, \node \in \nodes_1 , \node \notin \secnodes \} \cup \{\hypercoreof{2}\} }{\decvariable} \, . \]
	Then, the leg cores $\{\legcoreof{\node} \, : \, \node \in \secnodes \}$ build with the scalar core $\tilde{\scalarcore}$ a CP decomposition of $\contractionof{\{\hypercoreof{1},\hypercoreof{2}\}}{\secnodes}$.
	% Binary of basis
	When the CP decomposition of $\hypercoreof{1}$ was binary, basis or basis+, this property is also satisfied by the constructed CP decomposition.
	Thus the bound also holds for the ranks $\bincprankof{\cdot}$ or $\bascprankof{\cdot}$.
\end{proof}

\begin{proof}[Proof of Theorem~\ref{the:CPrankContractionBound}]
	Use delta tensor representation to represent contractions by graphs.
	We then iterate through the cores and contract them to the previously contracted tensor, where we apply Lemma~\ref{lem:sparsityGeneralContraction} when the tensor core has variables left open and Lemma~\ref{lem:sparsityDisjointContraction} if not.
\end{proof}


\begin{example}[Composition of formulas with connectives]
	For any formula $\exformula$ we have $1-\exformula$ = $\lnot\exformula$.
	The CP rank bound brings an increase by at most factor $2$ when taking the contraction with $\concoreof{\lnot}$ which has slice sparsity of $2$.
	This is not optimal, since $\lnot\exformula$ has at most an absolute slice sparsity increase of $1$.
	
	For any formulas $\exformula$ and $\secexformula$ we have $\exformula\cdot\secexformula = \exformula\land\secexformula$.
	Here the CP rank bounds on contractions can also be further tightened.
\end{example}


\begin{example}[Distributions of independent variables]
	Independence means factorization, conditional independence means sum over factorizations.
	Again, the $\ell_0$ norm is bounded by the product of the $\ell_0$ norm of the factors.
\end{example}


\subsubsection{Normations of CP Decompositions}

\red{As a theorem: If any of the above CP Decomposition is normable, the normation has the same CP ranks.
Especially interesting when learning Bayesian Networks, where each core has a CP bound by the number of datapoints.}

\subsubsection{Sparse Encoding of Functions}

%Using the proof idea of Theorem~\ref{the:sparseBasisCP}, we can state a more general CP bound on the encoding of functions.

We now state that the basis CP rank of relational encodings is equal to the cardinality of the domain.
The basis CP format can therefore not provide a sparse representation when the factored system contains many categorical variables.

\begin{theorem}\label{the:rencodingBasCP}
	For any function
		\[ \exfunction : \facstates \rightarrow  \secfacstates \]
	between factored systems we have
		\[ \bascprankof{\rencodingof{\exfunction}} =  \facdim \, . \]
\end{theorem}
\begin{proof}
	With Theorem~\ref{the:sparseBasisCP}, the basis CP rank coincides with the number of not vanishing coordinates, which is the cardinality of the domain of $\exfunction$.
\end{proof}

Allowing for trivial leg vectors can decrease the CP rank, as we show next.

\begin{theorem}
	We have
		\[ \baspluscprankof{\rencodingof{\exfunction}} \leq  \sum_{y \in \imageof{\exfunction}} \baspluscprankof{\ones_{\exfunction == y} } \, , \]
	where by $\ones_{\exfunction == y} $ we denote the indicator, whether the function $\exfunction$ evaluates to $y$.
\end{theorem}
\begin{proof}
	We have
		\[ \rencodingof{\exfunction} = \sum_{y \in \imageof{\exfunction}} \ones_{\exfunction == y}[\catvariable]  \otimes \onehotmapofat{y}{\catvariableof{\exfunction}} \, . \]
	For each $y \in \imageof{\exfunction}$ we represent $\onehotmapofat{y}{\catvariableof{\exfunction}}$ in an basis+ CP format with $\baspluscprankof{\ones_{\exfunction == y} } $ summands and arrive at a basis+ CP decomposition of $\rencodingof{\exfunction}$ with $\sum_{y \in \imageof{\exfunction}} \baspluscprankof{\ones_{\exfunction == y} } $ summands.
\end{proof}

The above claim still holds when replacing $\baspluscprankof{\cdot}$ with the ranks $\bascprankof{\cdot}$ or $\bincprankof{\cdot}$.
For the rank $\bascprankof{\cdot}$ it leads to the bound of Theorem~\ref{the:rencodingBasCP}, since summing the number of non zero coordinators of the indicators is the cardinality of the domain.

\begin{example}{Conjunction of variables}
	For the propositional formula $\exformula = \catvariableof{0} \land \catvariableof{1}$ we have
		\[ \rencodingofat{\exformula}{\catvariableof{0},\catvariableof{1}}
		 = \onehotmapofat{1,1}{\catvariableof{0},\catvariableof{1}} \otimes \onehotmapofat{1}{\catvariableof{\exformula}}
		  +  (\onesat{\catvariableof{0},\catvariableof{1}} - \onehotmapofat{1,1}{\catvariableof{0},\catvariableof{1}}) \otimes \onehotmapofat{0}{\catvariableof{\exformula}}  \]
	and thus 
		\[ \baspluscprankof{\rencodingof{\exformula}} \leq 3\]
	while $\bascprankof{\rencodingof{\exformula}} = 4$.
	\red{Especially useful for $d$-ary conjunctions, see Remark~\ref{rem:naryConnectives}!}
\end{example}




\subsubsection{Construction by averaging the incoming legs}

Basis CP Decompositions can be constructed by understanding the variable $\indvariableof{\insymbol}$ of the relational encoding of a function $\exfunction:\inset \rightarrow \outset$ as the slice selection variable.

\begin{example}{Empirical distributions, see Theorem~\ref{the:empCPRep}}
	Let there be a data map 
		\[ \datamap : [\datanum] \rightarrow \facstates \, . \]
	We can use Theorem~\ref{the:functionDecompositionBasisCP} to find a tensor network representation fo $\rencodingof{\datamap}$ as
	\begin{align*}
		\rencodingofat{\datamap}{\catvariable,\shortcatvariables}  
		= \contractionof{
		\{\rencodingofat{\datamap^{\atomenumerator}}{\catvariable,\catvariableof{\atomenumerator}} : \atomenumeratorin \} 
		}{\catvariable,\shortcatvariables} \, . 
	\end{align*}
	This representation is in the CP format, when adding trivial scalar core and and delta tensor to the data index.
	It is furthermore in a basis CP format, since all $\rencodingof{\datamap^{\catenumerator}}$ are directed and binary tensors.
	Normation to get the empirical distribution amounts to setting a slice core with coordinates $\frac{1}{\datanum}$.
\end{example}





%\subsection{Manipulations of Binary CP Decomposition}\label{sec:BinaryCPManipulation}
%
%Since the coordinates on the legs are binary, operations like contractions, slicing and marginalization are especially efficient given binary CP Decompositions.
%
%
%\begin{example}[Hypertrie Format]
%	Hypertries are another efficient implementation of the slicing operations.
%%	Hypertries make use of the Basic TT Decomposition, given any permutation of the tensor legs.
%%	In addition, they eliminate storage redundancies when representing all permuted TT Decompositions, by referencing to same subnetworks (e.g. when slicing wrt leg 1 and then 2 or slicing wrt to 2 and then leg 1 will leave the same tensors to be further decomposed).
%\end{example}






%%% NEEDED?
%\subsection{Basis Tensor Networks}
%
%
%\begin{definition}
%	We call a tensor network, which cores are directed and binary a basis tensor network. %also acyclic?
%\end{definition}
%
%%\begin{definition}
%%	We call a Tensor Network with open legs $V$ basis, when for any tensor core in the network any slicing of the closed legs is parallel to a basic tensor (that is has $\ell_0$ norm of at most $1$).
%%\end{definition}
%
%
%\subsubsection{Basis elementary decomposition}
%
%Elementary tensors are tensor products of vectors.
%Demanding each vector in the product to be a basis vector leads to basis tensors.
%Thus the tensors which poses a basis elementary decompositions coincide with the basis tensors.
%
%\begin{theorem}
%	Given axis dimensions $\catdimof{\atomenumerator}\in\mathbb{N}$ for $\atomenumeratorin$, the one-hot encoding is a bijection between $\facstates$ and the basis tensors of $\facspace$ with unit norm.
%\end{theorem}
%\begin{proof}
%	The one-hot encoding to the basis tensors is injective, since each state is mapped to a different basis tensor.
%	The one-hot encoding is further surjective, since every basis tensor has a preimage state by the indices of the $1$ coordinate.
%	Therefore the one-hot encoding is a bijection.
%\end{proof}
%
%
%\subsubsection{Basis CP Decomposition}\label{sec:basisCP}
%
%We here provide with the CP Decomposition of binary tensors as ways to overcome the storage overhead of $\ell_0$-sparse (of tensor flattening) tensors.
%The key idea is to enumerate the nonzero coordinates by introducing an additional axis carrying the data index.
%Keeping the such introduced hidden rank as constant then results in an elementary tensor, which has a representation with linear demand.
%We can thus represent the vectors of each such elementary tensor in a matrix and get the cores of the CP decomposition.
%
%
%\subsubsection{Basis TT Decomposition}
%
%Exploiting vanishing slices, thus exploiting a form of block $\ell_0$-sparsity (where full slizes are vanishing).
%
%Can be generated from the basic CP decomposition, by the CP cores contracted with partial $\delta$ tensors.
%The rank will thus not be larger than the rank of the $\cpformat$.
%
%
%
%\subsubsection{Basic HT Decomposition}
%
%The constraint that networks need to be basic just affects the leaf cores.






\subsection{Subspaces of formulas}\label{sec:HT}

\red{
Formula Tensors have Tensor Network decomposition, which are best represented in a $\htformat$ decomposition.
We here describe this perspective and show applications of this formalism in the recovery/learning of formula tensors.
}
The decomposition of formula tensors is basic, since atomic formula tensors being on the leafs consist of basic vectors in the respective legs.


\subsubsection{Formula Subspaces}

Each formula tensor defines the subspace of $\atomspace$
\begin{align}
	\subspaceof{\exformula} = \mathrm{span} \left\{ \lnot\exformula,\exformula \right\} = \mathrm{im}\left(\ftensorof{\exformula} \right)
	%\mathrm{span} \left\{ \braket{\atombasisvector_1,\ftensorof{\exformula}}, \braket{\atombasisvector_0,\ftensorof{\exformula}} \right\}
\end{align}

Let us notice that the spanning vector of the subspace $\subspaceof{\exformula}$ are binary tensors summing up to the tensor of ones.

%\subsubsection{Atomic Tensor Spaces}

For each atom $\atomicformulaof{\atomenumerator}$ we have
	\[ \subspaceof{\atomicformulaof{\atomenumerator}} = \rr^2 \, . \]
The tensor space carrying the factored representation of the worlds is thus
\begin{align}
	\bigotimes_{\atomenumeratorin}\subspaceof{\atomicformulaof{\atomenumerator}} \, .
\end{align}

\subsubsection{Formula Decomposition as a Subspace Choice}

Given a formula $\exformula\exconnective\secexformula$ composed of formulas $\exformula$ and $\secexformula$ containing different atoms we have
\begin{align}
	\subspaceof{\exformula\exconnective\secexformula} 
	\subset \subspaceof{\exformula} \otimes \subspaceof{\secexformula}
\end{align}

A connective $\exconnective$ thus determines the selection of a two-dimensional subspace in the four-dimensional tensor product of subspaces to both subformulas.

%\subsubsection{Approximation problems}

Reconstruction of a formula given its formula tensor amounts to finding the HT Decomposition under the constraints of subspace choices according to the allowed logical connectives.

Given a set of positive and negative examples of a formula poses further an approximation problem of the examples by a HT Decomposition.

Advantages of this perspective are
\begin{itemize}
	\item Given a $\htformat$ the best approximation always exists (Theorem 11.58 in \cite{hackbusch_tensor_2012}), but need to further restrict to cores given by logical connectives 
	\item Apply Approximation algorithms: ALS or HOSVD
\end{itemize}


















%\section{Binary Optimization of Sparse Tensors}

Let us now study the problem of searching for the maximal coordinate in a tensor, when the tensor is represented in a basis+ CP Format. 


%\red{Search for the maximal coordinate in a tensor network is a binary optimization problem, given leg dimensions of 2.
%We explain this perspective in this chapter and connect it to the HUBO/QUBO formalism.}


\subsection{Mode search in exponential families}

Mode search 
\begin{align*}
	\max_{\shortcatindices\in\atomstates} \sbcontraction{\sencsstatat{\indexedshortcatvariables,\selvariable},\canparam} 
	= \max_{\meanparam\in\meanset} \sbcontraction{\meanparamat{\selvariable},\canparamat{\selvariable}}
\end{align*}


% Appearance of mode search
The search for maximal coordinates appears in various reasoning tasks:
\begin{itemize}
	\item MAP query as mode search of MLN: $\hypercore$ is the contraction of evidence with the distribution, leaving the query variables open.
	\item Grafting as mode search of proposal distribution: $\hypercore$ is the contraction of the gradient of the likelihood with the relational encoding of the hypothesis.
\end{itemize}
Both tasks have been formulated as mode search problems in exponential families.



\subsection{Higher-Order Unconstrained Binary Optimization (HUBO)}

\red{
Here binary refers to the leg dimensions $\catdimof{\atomenumerator}$ being 2, not to binary coordinates as often refered to in this work.
}


\begin{definition}
	The binary optimization of a tensor $\hypercoreat{\shortcatvariables}\in\atomstates$ is the problem
	\begin{align}\tag{$\mathrm{P}_{\hypercore}$}\label{prob:HUBO}
		\argmax_{\shortcatindices\in\atomstates} \hypercoreat{\indexedshortcatvariables} 
	\end{align}
	
	We call Problem~\ref{prob:HUBO} a Higher Order Unconstrained Binary Optimization (HUBO) problem of order $\sliceorder$ and sparsity $\slicerankwrtof{\sliceorder}{\hypercore}$, when $\hypercore$ has a monomial decomposition (see Definition~\ref{def:polynomialSparsity}) with $\cardof{\variablesetof{\decindex}}\leq\sliceorder$ for all $\decindexin$, that is when $\slicerankwrtof{\sliceorder}{\hypercore}<\infty$.
	
	
\end{definition}


\begin{remark}[Leg dimensions larger than 2]
% Leg dimension needs to be 2
	We demanded leg dimensions $\catdimof{\atomenumerator}=2$ to have binary valued variables $\catvariableof{\catenumerator}$, which is required to connect with the formalism of binary optimization.
	Categorical variables with larger dimensions can be represented by atomization variables, which are created by contractions with categorical constraint tensors (see Section~\ref{sec:categoricalTN}).
\end{remark}


% Interpretation of sparsity
The sparsity $\slicerankwrtof{\sliceorder}{\hypercore}$ is the minimal number of monomials, for which a weighted sum is equal to $\hypercore$.
Thus we interpret Problem~\ref{prob:HUBO} as searching for the maximum in a polynomial consistent of $\slicerankwrtof{\sliceorder}{\hypercore}$ monomial terms.
\red{Each monomial is also refered to as potential.}



%\begin{remark}[Sparsity]% To sparse Tensor Calculus?
%
%The number $\slicesparsityof{\hypercore}$ of a tensor $\hypercore$ defining a HUBO is of central importance to have an effective solution.
%%Here the number of nonzero coordinates coincides with the number of monomials required to represent the polynomial as a sum.
%
%\red{We investigated the Sparsity as the slice sparsity not the vector sparsity in Chapter~\ref{cha:sparseTC}.}
%%We here investigate, whether the same reasoning assumptions used for sparse representation by tensor networks also lead to $\ell_0$-sparse tensors. 
%
%\end{remark}




\subsection{Quadratic Unconstrained Binary Optimization (QUBO)}

\red{Quadratic Unconstrained Binary Optimization problems are HUBOs of order $\sliceorder=2$.}

We refine the monomial decomposition of tensors (see Definition~\ref{def:polynomialSparsity}) by demanding that monomials consist of at most two variables.

\begin{definition}
	We call a monomial decomposition $\sliceset$ of a tensor $\hypercore\in\atomspace$ a quadratic decomposition, if $\cardof{\variableset}\leq 2$ for all $(\lambda,\variableset,\catindexof{\variableset}) \in \sliceset$.
	We denote the smallest cardinality $\cardof{\sliceset}$ among quadratic decompositions of $\hypercore$ by $\quacprankof{\hypercore}$.

	If a tensor $\hypercore\in\bigotimes_{\atomenumeratorin}\rr^2$ has a quadratic decomposition, we call Problem~\ref{prob:HUBO} a Quadratic Unconstrained Binary Optimization (QUBO) problem of sparsity $\quacprankof{\hypercore}$.
\end{definition}

% CP Decompositions
Analogously to monomial decompositions, quadratic decompositions have an equivalence in a CP decomposition of $\hypercore$.
Beyond being binary tensors, the leg cores are further restricted that for each slice $\decindexin$ at most two of them are basis vectors and the rest trivial vectors $\ones$.

% Existence
We notice, that there are tensors, for which no quadratic decomposition exists.
This is already obvious from the fact, that the tensors with a quadratic decomposition build a $\binom{\atomorder}{2}$ dimensional submanifold in the $2^\atomorder$ dimensional tensor space.
This is in contrast with monomial decompositions, where one can always construct a decomposition.



%% OLD THEOREM: FALSE!
%However, for any non-negative tensor $\hypercore$ the Problem~\ref{prob:HUBO} is equivalent to a QUBO problem of possibly larger order as we state next.
%To turn HUBO problems into QUBO we need the slack variable trick, as described in the next lemma.

%\begin{theorem}\label{the:HUBOtoQUBO}
%	Let there be a tensor $\hypercore\in\in\bigotimes_{\atomenumeratorin}\rr^2$, which has a monomial decomposition with dimension $r$ and non-negative scalar core $\scalarcore$.
%	Then, the HUBO defined by $\hypercore$ is equivalent to a QUBO of order at most $\atomorder+r$ and sparsity at most $\atomorder \cdot r $.
%%	The maximal coordinate problem to any tensor $\hypercore\in\bigotimes_{\atomenumeratorin}\rr^2$ is equivalent to a QUBO with at most $\atomorder+\slicesparsityof{\hypercore}$ variables.
%%	\red{Need positive coordinates!}
%\end{theorem}

%To show the theorem we state the following lemma.


We can transform certain HUBO problems in QUBO problems with the usage of auxiliary variables, as we show in the next lemma.

%% Slack variables
\begin{lemma}\label{lem:monomialToQUBO}
	For any $\atomindices\in[2]$ and $\variableset\subset[\atomorder]$ we have 
		\[ \left( \prod_{\atomenumerator\in\variableset} \atomlegindexof{\atomenumerator } \right)  \left(  \prod_{\atomenumerator\notin\variableset} (1- \atomlegindexof{\atomenumerator }) \right)
		=
		\max_{\slackvariable\in[2]} \slackvariable \cdot 2 \cdot \left( \sum_{\atomenumerator\in\variableset}\atomlegindexof{\atomenumerator}  - \cardof{\variableset} - \sum_{\atomenumerator\notin\variableset}\atomlegindexof{\atomenumerator} + \frac{1}{2} \right) \, . % Alternative: no factor 2, but + 1 instead of +1/2 (->pyqubo)
 		\]
\end{lemma}
\begin{proof} %Proof by case distinction
	Only if $\atomlegindexof{\atomenumerator}=1$ for $\atomenumerator\in\variableset$ and $\atomlegindexof{\atomenumerator}=0$ else we have
		\[ \left( \sum_{\atomenumerator\in\variableset}\atomlegindexof{\atomenumerator}  - \cardof{\variableset} - \sum_{\atomenumerator\notin\variableset}\atomlegindexof{\atomenumerator} + \frac{1}{2} \right) \geq 0 \, . \]
	In this case the maximum is taken for $\slackvariable=1$ and we have
		\[ \max_{\slackvariable\in[2]} \slackvariable \cdot 2 \cdot \left( \sum_{\atomenumerator\in\variableset}\atomlegindexof{\atomenumerator}  - \cardof{\variableset} - \sum_{\atomenumerator\notin\variableset}\atomlegindexof{\atomenumerator} + \frac{1}{2} \right) 
		= 1 = \left( \prod_{\atomenumerator\in\variableset} \atomlegindexof{\atomenumerator } \right)  \left(  \prod_{\atomenumerator\notin\variableset} (1- \atomlegindexof{\atomenumerator }) \right) \, . \]
	In all other cases, the maximum is taken for $\slackvariable=0$ and thus vanishes, that is 
		\[ \max_{\slackvariable\in[2]} \slackvariable \cdot 2 \cdot \left( \sum_{\atomenumerator\in\variableset}\atomlegindexof{\atomenumerator}  - \cardof{\variableset} - \sum_{\atomenumerator\notin\variableset}\atomlegindexof{\atomenumerator} + \frac{1}{2} \right) 
		= 0 = \left( \prod_{\atomenumerator\in\variableset} \atomlegindexof{\atomenumerator } \right)  \left(  \prod_{\atomenumerator\notin\variableset} (1- \atomlegindexof{\atomenumerator }) \right) \, . \]
	Thus, the claim holds in all cases.
\end{proof}	


%\begin{proof}[Proof of Theorem~\ref{the:HUBOtoQUBO}]
%	For each summand in the monomial decomposition apply Lemma~\ref{lem:monomialToQUBO}.
%\end{proof}



\subsection{Integer Linear Programming}

Let us now show how optimization problems can be represented as linear programming problems.



\begin{definition}
	A Binary Integer Linear Program (ILP) is a problem of the form
	\begin{align*}
		\max_{x \in\{0,1\}^n} c^T x \quad \text{subject to } \quad A^{upper} x \leq b^{upper} , A^{lower} x \geq b^{lower} 
	\end{align*}
	where $A^{upper}\in\rr^{n^{upper}\times n}$, $b^{upper}\in\rr^{n^{upper}}$, $A^{lower}\in\rr^{n^{lower}\times n}$, $b^{lower}\in\rr^{n^{lower}}$.
\end{definition}



\begin{theorem}
	Given a monomial decomposition $\sliceset=\enumeratedslices$ of a tensor $\hypercore$ we define an Binary ILP as the maximation of 
	\begin{align*}
		\sum_{\decindexin} \slicescalar^{\decindex} \slackvariable^{\decindex} 
	\end{align*}
	with the constraints for any $\decindex$
	\begin{itemize}
		\item 
		\begin{align*}
			\slackvariable^{\decindex}  \leq \catvariableof{\atomenumerator} \quad \text{for} \quad \atomenumerator\in\variableset^j , \catindexof{\atomenumerator} = 1
		\end{align*}
		\item 
		\begin{align*}
			\slackvariable^{\decindex}  \leq (1-\catvariableof{\atomenumerator}) \quad \text{for} \quad \atomenumerator\in\variableset^j , \catindexof{\atomenumerator} = 0
		\end{align*}
		\item 
		\begin{align*}
			\slackvariable^{\decindex} \geq 1 + \sum_{\atomenumerator\in\variableset^{\decindex} : \catindexof{\atomenumerator} = 1} (\catvariableof{\atomenumerator} -1)
		- \sum_{\atomenumerator\in\variableset^{\decindex} : \catindexof{\atomenumerator} = 0} \catvariableof{\atomenumerator} 
		\end{align*}
	\end{itemize}
	The solution $\catindex^{ILP,\sliceset}$ of this ILP and the solution $\catindex^{HUBO,\sliceset}$ of the HUBO coincide on the variables of hypercore, i.e.
		\[ \catindex^{ILP,\sliceset}|_{[d]} =  \catindex^{HUBO,\sliceset} \, . \]
\end{theorem}
\begin{proof}
	We have to show that the constraints are satisfied if and only if $\slackvariable^{\decindex}=\onehotmapofat{\catvariableof{\variableset^{\decindex}}^{\decindex}}{\indexedcatvariableof{\variableset^{\decindex}}}$.
\end{proof}





% Solution Algorithms
\section{Reasoning by Tensor Approximation}\label{cha:tensorApproximation}

Often reasoning requires the execution of demanding contractions of tensors networks, or combinatorical search of maximum coordinates.
We in this chapter investigate methods, to replace hard to be sampled tensor networks by approximating tensor networks, which then serve as a proxy in inference tasks.


\subsection{Approximation of Energy tensors}

\subsubsection{Direct Approximation}

Direct approximation is the problem
	\[ \argmin_{\canparam\in\Gamma^{\graph}} \|\energytensorat{\shortcatvariables} - \canparamat{\shortcatvariables}\|^2 \, . \]


\subsubsection{Approximation involving Selection Architectures}

Approximation involving a selection architecture $\fselectionmap$ is the problem
	\[ \argmin_{\canparam\in\Gamma^{\graph}} \|\energytensor - \sbcontractionof{\sencodingof{\fselectionmap},\canparam}{\shortcatvariables}\|^2 \, . \]

In a tensor network diagram we depict this as
\begin{center}
    \begin{tikzpicture}[scale=0.3] % , baseline = -3.5pt

	%\drawformulatensorof{\targettensor}


    \draw (-5,1) rectangle (5,-1);
    \node at (0,-1) [above] {$\targettensor$} ;
    \draw (-4,-1)--(-4,-3) node[midway,left] {\tiny $\catvariableof{0}$};
    \draw (4,-1)--(4,-3) node[midway,right] {\tiny $\catvariableof{\catorder\shortminus1}$};
    \node[anchor=center] at (0,-2) {$\cdots$};

		
\begin{scope}[shift={(-19,2)}]

    \draw (0,-5)--(0,-3) node[midway,left] {\tiny $\catvariableof{0}$};
    \draw (4,-5)--(4,-3) node[midway,right] {\tiny $\catvariableof{\catorder\shortminus1}$};
    \node[anchor=center]  at (2,-4) {$\cdots$};

%\drawatomindices{0}{-4}
\draw (-1,1) rectangle (5, -3);
\node[anchor=center] (text) at (2,-1) {$\sencodingof{\formulaset}$};

%\draw[->] (2,-1)--(2,1) node[midway,right] {\tiny ${\atomicformulaof{\parindexof{1}} \land \atomicformulaof{\parindexof{2}}}$}; 

\draw[] (5,0.5) -- (7,0.5) node[midway, above] {\tiny $\selvariableof{\selorder\shortminus1}$};
%\draw[<-] (5,-1)--(7,-1) node[midway,above] {\tiny $\selvariableof{\vselectionsymbol,1}$}; 
\node[anchor=center] (text) at (6,-1) {$\vdots$};
\draw[] (5,-2.5)--(7,-2.5) node[midway,below] {\tiny $\selvariableof{0}$}; 

\draw (7,1) rectangle (9, -3);
\node[anchor=center] (text) at (8,-1) {$\canparam$};


	
%	\draw (-4,5) rectangle (-2,3);
%    	\node at (-3,3.1) [above] {$\varcore{1}$} ;
%			
%	
%	\draw (-5,1) rectangle (-1,-1);
%    	\node at (-3,-1.1) [above] {$\gtensorof{\placeholderof{1}}$} ;
%	\draw (-3,1) -- (-3,3) node [midway,left] {$\atomlegindexof{1}$};
%
%	\node at (0.65,-1.1) [above] {$\cdots$};
%
%	\draw (2,1) rectangle (6,-1);
%    	\node at (4,-1.1) [above] {$\gtensorof{\placeholderof{\atomorder}}$} ;
%	\draw (4,1) -- (4,3) node [midway,left] {$\atomlegindexof{\atomorder}$};
%	\draw (3,5) rectangle (5,3);
%    	\node at (4,3.1) [above] {$\varcore{\atomorder}$} ;
%	\renewcommand{\skeletoncolor}{}
%	\drawskeleton
	
	
	
\end{scope}
	
\node at (-7.5,0) {{$-$}};
	
\draw (-22,3) -- (-22,-3);
\draw (-22.2,3) -- (-22.2,-3);

\draw (6,3) -- (6,-3);
\draw (6.2,3) node[right] {$2$} -- (6.2,-3) ;
	
\node at (-30,0) [right] {{$\argmin_{\canparam\in\Gamma^{\graph}}$}};
	
\end{tikzpicture}
\end{center}


\begin{example}[Approximate based on a slice sparsity selecting architecture]
	Use a term selecting neural network (conjunction neuron on $\atomorder$ unary neurons selecting a variable and $\mathrm{Id},\lnot,\mathrm{True}$ as connective selector.
	Demand the parameter tensor $\canparam$ to be in a basis CP format, then each slice of the parameter tensor corresponds with the slice of the energy.
	The use the approximation for MAP search.
	Same construction possible for probability tensors, but often more involved to instantiate them as tensor network.
\end{example}



\subsection{Transformation of Maximum Search to Risk Minimization}

By the squares risk trick, maximum coordinate searches involving contractions with boolean tensors can be turned into squares risk minimization problems.
This trick can be applied in MAP inference of MLN and the proposal distribution.

\subsubsection{Weighted Squares Loss Trick}

\begin{lemma}
	Let $\hypercore$ be a Boolean tensor, that is $\imageof{\hypercore}\subset\{0,1\}$.
	Then
		\[ \hypercoreat{\shortcatvariables} = \onesat{\shortcatvariables} - \left( \hypercoreat{\shortcatvariables} - \onesat{\shortcatvariables} \right)^2  \]
	where $\ones$ is a tensor with same shape as $\hypercore$ and all coordinates being $1$.
\end{lemma}
\begin{proof}
	Since for each $\shortcatindices\in\facstates$ we have $\hypercore[\shortcatvariables=\shortcatindices]\in\{0,1\}$, it holds that
		\[ \hypercoreat{\shortcatvariables=\shortcatindices} = 1 - (\hypercoreat{\shortcatvariables=\shortcatindices}-1)^2 \]
	and thus in coordinatewise calculus
		\[ \hypercoreat{\shortcatvariables} = \onesat{\shortcatvariables} - \left( \hypercoreat{\shortcatvariables} - \onesat{\shortcatvariables} \right)^2 \, .   \]
\end{proof}

We apply this property to reformulate optimization problems over boolean tensors into weighted least squares problems.

\begin{theorem}[Weighted Squares Loss Trick]\label{the:reweightedLeastSquares}
	Let $\Gamma$ be a set of boolean tensors in $\facspace$ and $\importancetensor\in\facspace$ arbitrary.
	Then we have
	\begin{align}
		\argmax_{\hypercore\in\Gamma} \contraction{\importancetensor,\hypercore} 
		= \argmin_{\hypercore\in\Gamma} \contraction{\importancetensor, (\hypercoreat{\shortcatvariables}-\onesat{\shortcatvariables})^2}
	\end{align} 
\end{theorem}
\begin{proof}
	Using the Lemma above, $\hypercore$ is identical to $\onesat{\shortcatvariables}-(\hypercoreat{\shortcatvariables}-\onesat{\shortcatvariables})^2$ and we get
	\begin{align*}
		 \contraction{\importancetensor,\hypercore} 
		 &=  \contraction{\importancetensor,\onesat{\shortcatvariables}}-\contraction{\importancetensor,(\hypercoreat{\shortcatvariables}-\onesat{\shortcatvariables})^2} 
	\end{align*}
	Since the first term does not depend on $\hypercore$, it can be dropped in the maximization problem.
	The $(-1)$ factor then turns the maximization into a minimization problem.
\end{proof}

% Interpretation and Importance Tensor
Theorem~\ref{the:reweightedLeastSquares} reformulates maximation of binary tensors with respect to an angle to another tensor into minimization of a squares risk.
This squares risk trick is especially useful when combining it with a relaxation of $\Gamma$ to differentiably parametrizable sets, since then common squares risk solvers can be applied.
We will call $\importancetensor$ in the Theorem~\ref{the:reweightedLeastSquares} importance tensor, since it manipulates the relevance of each coordinate in the squares loss.

%
As a result, we interpret the objective
	\[ \contraction{\importancetensor, (\hypercoreat{\shortcatvariables}-\onesat{\shortcatvariables})^2} \]
as a weighted squares loss.

\begin{example}[Proposal distribution maxima]
	The Problem~\ref{prob:steepestAscent} of finding the maximal coordinate can thus be turned into
	\begin{align*}
		\argmax_{\shortselindices} \contractionof{(\empdistribution-\currentdistribution),\fselectionmap}{\shortselvariables=\shortselindices}  
		= \argmin_{\shortselindices} \sbcontraction{(\empdistribution-\currentdistribution),
		\left(\contractionof{\fselectionmap,\onehotmapofat{\shortselindices}{\shortselvariables}}{\shortcatvariables}-\onesat{\shortcatvariables}\right)^2} \, . 
	\end{align*}
\end{example}


\subsubsection{Problem of the trivial tensor}

By the above we motivated least squares problems on the set of one-hot encoded states.
One is tempted to extend this set to $\mnexpfamily$ for efficient solutions by alternating algorithms.

However, for any hypergraph $\graph$ we have $\onesat{\shortcatvariables}\in\mnexpfamily$.
In many situations (e.g. disjoint model sets supported at positive data) the objective is more in favor at the trivial tensor than at the one-hot encoding.
As a result, we do not solve the previously posed one-hot encoding problem, when allowing such an hypothesis embedding.


\begin{example}[Fitting a tensor by a formula tensor]\label{exa:formulaFitting}
	Task: Given a tensor $\hypercore$, find a formula $\exformula\in\formulaset$ such that it coincides with $\hypercore$.

	If $\hypercore$ is a binary tensor, we understand it as a formula and want to find an $\exformula$ such that its number of worlds is maximal, that is solve the problem
		\[ \argmax_{\exformula\in\formulaset}\sbcontraction{\exformula\Leftrightarrow\hypercore}  \, . \]

	We can use the squares risk trick and get an equivalent problem
		\[ \argmin_{\exformula\in\formulaset} \| \sbcontractionof{\exformula\Leftrightarrow\hypercore}{\shortcatvariables}  - \onesat{\shortcatvariables} \|^2 \, . \]
\end{example}

%\begin{remark}{Least Squares Loss by Tensor Fitting}
%	\red{Alternative approach to least squares problems: Tensor Fitting}
%	And, if the target is another formula $y$, such that $\exformula$ conincides with $\tilde{f} \iff y $ we have
%		\[ \left(\polynomialof{\exformula}(\datamap)-1\right)^2 = \left(  \polynomialof{\tilde{f}}(\datamap) - y(\datamap) \right)^2  \]
%	This is exactly the least squares loss, which would appear in a supervised interpretation of the learning.
%\end{remark}




\subsection{Alternating Solution of Least Squares Problems}

When the parameter tensor $\canparam$ is only restricted to have a decomposition as a tensor network on $\graph$, we can iteratively update each core.
The resulting algorithm is called Alternating Least Squares (ALS) (see Algorithm \ref{alg:ALS}).

\begin{algorithm}[hbt!]
\caption{Alternating Least Squares (ALS)}\label{alg:ALS}
\begin{algorithmic}
\For{$\edgein$}
	\State Set $\hypercoreofat{\edge}{\catvariableof{\edge}}$ to a random element in $\rr^{\atomlegdimof{\atomenumerator}}$ 
\EndFor
%\For{$\atomenumeratorin$}
%	\State Set $\varcore{\atomenumerator}$ to a random element in $\rr^{\atomlegdimof{\atomenumerator}}$ 
%\EndFor
\While{Stopping criterion is not met}
\For{$\edgein$}
	\State Set $\hypercoreofat{\edge}{\catvariableof{\edge}}$ to a to a solution of the local problem, that is
	\[ 
	\hypercoreofat{\edge}{\catvariableof{\edge}}
	 \algdefsymbol 
	 \argmin_{\hypercoreofat{\edge}{\catvariableof{\edge}}} 
	 \contraction{\importancetensor, (\contractionof{\fselectionmap,\canparam}{\shortcatvariables} - \targettensor[\shortcatvariables])^2}
	 \]
\EndFor
%\For{$\atomenumeratorin$}
%	\State Set $\varcore{\atomenumerator}$ to a to a solution of 
%	\[ \varcore{\atomenumerator} \algdefsymbol \argmin_{\varcore{\atomenumerator}}  \left\|  \importancetensor \chadamard ( \groundingof{\parametrization(\varcore{1},\ldots,\varcore{\atomorder})} - \targettensor )  \right\|^2 \]
%\EndFor
\EndWhile
\end{algorithmic}
\end{algorithm}


\subsubsection{Choice of Representation Format}

\red{The choice of the hypergraph $\graph$ used for approximation bears a tradeoff between expressivity and complexity in sampling.
Hidden variables, that is variables only present in $\graph$, but not in the sensing matrix, increase the expressivity, especially when assigning large dimensions to them.
When there are no hidden variables, the maximum of $\canparam$ can be found by maximum calibration through a message passing algorithm, since no hidden variable has to marginalized.}


In case of skeleton expressions with many placeholders further decomposition for algorithmic efficiency are required.
\begin{itemize}
	\item Elementary Format ($\elformat$-Format): 
	\item $\cpformat$-Format: Closest to sum of formula tensors (when all vectors are basis, then have a sum).
	\item $\ttformat$-Format: Showed better heuristic performance in optimization
\end{itemize}

For any tensor network decomposition into cores $\canparamof{\parenumerator}$ have the derivative $\frac{\partial}{\partial \canparamof{\parenumerator}} \canparam$ as the tensor network with out the core $\canparamof{\parenumerator}$.

%\begin{remark}[Parametercores being basis tensors]
%	When the parameter core is a basis tensor, the contraction with the parametercore coincides with the respective formula tensor.
%	Thus, we will search for basis tensors optimizing in contractions objectives to specific reasoning tasks, and add them iteratively to the network at hand.
%\end{remark}


%\subsection{Projection onto Basis Tensors}
%\red{This is sampling!}
%We project onto basis tensors to achieve single formulas.


\subsection{Regularization and Compressed Sensing}


When regularizing the least squares problem by enforcing the sparsity of $\canparam$, we arrive at the compressed sensing problem
\begin{align}
	\argmin_{\canparamat{\selvariable}} \sparsityof{\canparam} 
	\quad \text{subject to } \quad 
	\left\| \contractionof{\sencsstat,\canparam}{\shortcatvariables} - \energytensorat{\shortcatvariables} \right\|_2 \leq \eta
\end{align}
Here, the sensing matrix is the selection tensor.


\begin{example}[Formula fitting to an example]
	Choosing the best formula fitting data (see Example~\ref{exa:formulaFitting}) is the problem
	\begin{align}
	\argmin_{\canparamat{\selvariable}\, : \,  \sparsityof{\canparam}=1} \left\| \contractionof{\importancetensor,\sencsstat,\canparam}{\shortcatvariables} - \targettensor \right\|_2 
	\end{align}
	where $\importancetensor$ has nonzero entries at marked coordinates and $\targettensor$ stores in Boolean coordinates whether the marked coordinates are positive or negative examples.
	\red{When the number of positive and negative examples are identical, we can linearly transform the objective to that of a grafting instance, where the current model is the empirical distribution of negative examples and the data consists of the positive examples.}
\end{example}

% Usage as sparse tensor
The sparse tensor solving the problem then has a small number of nonzero coordinates and the selection tensor can be restricted to those.
As a consequence, inference can be performed more efficiently.

% Algorithmic solution
The algorithmic solution of these problems can be done by greedy algorithms, thresholding based algorithms or optimization based algorithms \cite{foucart_mathematical_2013}.

% Guarantees
Guarantees for the success of the algorithms depend on the properties of the sensing matrices.
Here the sensing matrices are deterministic, since constructed as selection tensors, and concentration based approaches towards probabilistic bounds on these properties (see \cite{goesmann_uniform_2021}) are not applicable.





\begin{example}[Propositional Formulas]
	Let there be a set $\formulaset$ of formulas, then we have
	\begin{align*}
		\sbcontractionof{\sencodingofat{\formulaset}{\shortcatvariables,\selvariableof{\insymbol}},\sencodingofat{\formulaset}{\shortcatvariables,\selvariableof{\outsymbol}}}{\indexedselvariableof{\insymbol},\indexedselvariableof{\outsymbol}}
		= \sbcontraction{\formulaof{\selindexof{\insymbol}}, \formulaof{\selindexof{\outsymbol}}} \, . 
	\end{align*}
	If the formulas have disjoint model sets then 
	\begin{align*}
		\sbcontractionof{\sencodingofat{\formulaset}{\shortcatvariables,\selvariableof{\insymbol}},\sencodingofat{\formulaset}{\shortcatvariables,\selvariableof{\insymbol}}}{\indexedselvariableof{\insymbol},\indexedselvariableof{\outsymbol}}
		= \begin{cases}
			\sbcontraction{\formulaof{\insymbol}} & \text{if } \selindexof{\insymbol}=\selindexof{\outsymbol} \\
			0 & \text{else} 
		\end{cases} \, . 
	\end{align*}
\end{example}


\begin{example}[Slice selection networks]
	
	For the slice selection network
	\begin{align*}
		\sbcontractionof{\sliceselectionmapat{\shortcatvariables,\selvariableof{\insymbol}},\sliceselectionmapat{\shortcatvariables,\selvariableof{\outsymbol}}}{\indexedselvariableof{\insymbol},\indexedselvariableof{\outsymbol}}
		= \begin{cases}
			0 & \text{if for a }\seccatenumerator\in\variablesetof{\selindexof{\insymbol}}\cap\variablesetof{\selindexof{\outsymbol}}\text{ we have }\catindexof{\seccatenumerator}^{\selindexof{\insymbol}}\neq\catindexof{\seccatenumerator}^{\selindexof{\outsymbol}} \\
			\prod_{\seccatenumerator\notin\variablesetof{\selindexof{\insymbol}}\cup\variablesetof{\selindexof{\outsymbol}}} \catdimof{\seccatenumerator}& \text{else} 
		\end{cases} \, . 
	\end{align*}

	Given a fixed $\selindexof{\insymbol}$, the maximum value in the respective slice is thus taken at $\selindexof{\insymbol}=\selindexof{\outsymbol}$
\end{example}









% Statistical Analysis
%\section{Concentration of the Expected Sufficient Statistics}
\section{Uniform Concentration of Random Contractions}\label{cha:widthBounds}

We here derive bounds on the uniform concentration of contractions with random tensors.




	
	
	

The width of a vector $\noisetensor$ is the supremum of contractions with respect to a set $\Gamma$ is
\begin{align}
	\widthwrtof{\Gamma}{\noisetensor}
	= \sup_{\theta\in\Gamma} \sbcontraction{\theta,\noisetensor} \, . 
	%= \sup_{\theta\in\Gamma} \contractionof{\{\theta, \noisetensor\}}{\varnothing} \, . 
\end{align}

We are interested in the random vector
	\[ \noisetensor^{\formulaset,\gendistribution,\datamap} = \sbcontractionof{\empdistribution,\formulaset}{\selvariable} -  \sbcontractionof{\gendistribution,\formulaset}{\selvariable}  \,  \]
	%\essdistof{\empdistribution}-\expectationof{\essdistof{\empdistribution}} \]
which is the difference between the mean parameters given the empirical distribution and the underlying generating distribution.

We derive bounds for the hypothesis $\Gamma$ being the set of basis vectors (i.e. for feature search) and being the set of normed vectors (i.e. for feature calibration).





%\subsection{Binomials}
%\subsubsection{Concentration Bounds for Binomials}





\subsection{Naive bounds given binomial coordinates}


We here investigate width bounds on random tensors $\noisetensor$, which coordinates have marginal distributions by Binomials.

% MLN
This is the case for $\noisetensor^{\fselectionmap,\gendistribution,\datanum}$.

Naive means, that we do not exploit the dependencies of the coordinates on each other, as would be the case in more sophisticated chaining approaches.

\subsubsection{Basis Vectors}

We exploit the sub-Gaussian Norm (Def 2.5.6 in CITE Vershynin Book) to state concentration inequalities.

\begin{definition}[Sub-Gaussian Norm]
	The sub-Gaussian norm of a random variable $X$ is defined as
		\[ \sgnormof{X} = \inf \left\{ C > 0 \, : \expectationof{\frac{X^2}{C^2} } \leq 2 \right\} \, .  \]
\end{definition}

\begin{theorem}
	Any coordinate of $\noisetensor$ is sub-Gaussian with 
		\[ \sgnormof{\noisetensor_i} \leq \frac{C}{\sqrt{\ln2}} \frac{1}{\sqrt{m}} \]
\end{theorem}
\begin{proof}
	Centered Bernoulli is bounded and therefore sub-Gaussian.
	Binomial is a sum and we apply Proposition 2.6.1 in CITE Vershynin Book.
\end{proof}


\begin{theorem}\label{the:basisTensorWidthBound}
	Let 
		\[ \Gamma = \{\onehotmapof{i} : i \in [p] \}\]
	and $\noisetensor$ be a random vector in $\rr^p$, which coordinates have a marginal distribution being centered Binomials with a fixed $\datanum\in\nn$.
	Then
		\[ \sgnormof{\widthatof{\Gamma}{\noisetensor}} \leq C \sqrt{\frac{\ln p}{m}} \, . \]
	where $C>0$ is a universal constant.
\end{theorem}
\begin{proof}
	Supremum of Sub-Gaussian variables.
\end{proof}


\subsubsection{Sphere}

We first provide a Chebyshev bound on the width of the sphere.

\begin{theorem}\label{the:sphereBoundVariance}
	Let $\noisetensor$ be a random tensor with marginal coordinate distributions by binomials with parameters $(\fprobof{\catindex},\datanum)$.

	For any $\failprob>0$, $\precision>0$ and $\datanum\in\nn$ with probability at least $1-\failprob$ we have
		\[ \normof{\frac{\noisetensor-\expectationof{\noisetensor}}{\datanum}} \leq   \precision \, \]
	provided that
		\[ \datanum \geq  \frac{ \sum_{\catindex\in\facstates} \fprobof{\catindex}(1-\fprobof{\catindex})}{\precision^2 \failprob} \, . \]
\end{theorem}
\begin{proof}
	%We estimate with the Cauchy Shwartz inequality
	%	\[ \braket{\theta,\noisetensor} \leq \|\noisetensor\| \|\theta\| \, . \]
	Since the squared norm of the noise is the sum of squared centered and averaged Binomials, we have
		\[  \expectationof{\normof{\noisetensor-\expectationof{\noisetensor}}^2}  
		= \sum_{\catindex\in\facstates} \frac{\fprobof{\catindex}(1-\fprobof{\catindex})}{\datanum} \]
	Here we used that the variance of Binomials with parameters $(\fprobof{\catindex},\datanum)$ is $\fprobof{\catindex}(1-\fprobof{\catindex}) \datanum$.
	
	If follows, that 
		\[ \expectationof{\left(\normof{\frac{\noisetensor-\expectationof{\noisetensor}}{\datanum}}\right)^2} =  \frac{ \sum_{\catindex\in\facstates} \fprobof{\catindex}(1-\fprobof{\catindex})}{\datanum} \, . \]
	
	Then we apply a Chebyshev Bound to get for any $\precision>0$
	\begin{align}
		\probof{\normof{\frac{\noisetensor-\expectationof{\noisetensor}}{\datanum}} > \precision} 
		= \probof{\left(\normof{\frac{\noisetensor-\expectationof{\noisetensor}}{\datanum}}\right)^2 > \precision^2} 
		\leq \frac{ \sum_{\catindex\in\facstates} \fprobof{\catindex}(1-\fprobof{\catindex})}{\datanum \cdot \precision^2}
	\end{align} 
	For a $\failprob>0$ we choose any $\datanum$ with
		\[ \datanum \geq  \frac{ \sum_{\catindex\in\facstates} \fprobof{\catindex}(1-\fprobof{\catindex})}{\precision^2 \failprob} \, \]
	and get 
	\begin{align}
		\probof{\normof{\frac{\noisetensor-\expectationof{\noisetensor}}{\datanum}} > \precision} \leq \failprob \, . 
	\end{align} 
	Thus, we have 
	\begin{align}
		\probof{\normof{\frac{\noisetensor-\expectationof{\noisetensor}}{\datanum}} \leq \precision} = 1 - \probof{\normof{\frac{\noisetensor-\expectationof{\noisetensor}}{\datanum}} > \precision}  \geq 1-\failprob \, . 
	\end{align} 
\end{proof}


% Multinomial
For $\sstat=\identity$ the noise tensor is a rescaled and centered multinomial.

\begin{corollary}
	Let there be multinomial variable with parameters $(\fprob,\datanum)$ where $\fprob\in\facspace$ a positive and normed tensor.
	Let $\datamap$ be a set of independent samples
	For any $\failprob>0$, $\precision>0$ and $\datanum\in\nn$ with probability at least $1-\failprob$ we have
		\[ \normof{\frac{\noisetensor-\expectationof{\noisetensor}}{\datanum}} \leq   \precision \, \]
	provided that
		\[ \datanum \geq  \frac{(1-\sbcontraction{(\fprob)^2})}{\precision^2 \failprob} \, . \]
\end{corollary}
\begin{proof}
	\theref{the:sphereBoundVariance} with
		\[ \sum_{\catindex\in\facstates} \fprobof{\catindex}(1-\fprobof{\catindex}) = \sum_{\catindex\in\facstates} \fprobof{\catindex} - \sum_{\catindex\in\facstates} (\fprobof{\catindex})^2 = 1-\sbcontraction{(\fprob)^2} \, . \]
\end{proof}



\subsubsection{Bounds based on the sub-gaussian norm}

\red{Unclear whether this is needed.}

A faster tail decay can be achieved, when bounding sub-gaussian norms.


\begin{theorem}
	Let $\noisetensor$ be a random vector in $\rr^p$, which coordinates have a marginal distribution being centered Binomials with a fixed $\datanum\in\nn$.
	Then
		\[ \sgnormof{\widthatof{\subsphere}{\noisetensor}} \leq C \sqrt{\frac{p}{m}} \, . \]
	where $C>0$ is a universal constant.
\end{theorem}
\begin{proof}
	Using that each coordinate has Sub-gaussian norm of at most $1$.
	\red{Asymptotically, the binomial tends to a gaussian, which has a smaller sg norm. 
	But the binomial has a sub-exponential regime preventing tighter sg bounds.}

	
	Norm of a Sub-gaussian vector, another application of Proposition 2.6.1 in CITE Vershynin Book.
\end{proof}




\subsection{Chaining bounds given binomial coordinates}

To proceed with the uniform concentration investigation, we need a concentration bound on Binomials.

\begin{theorem}
	For any $p\in[0,1]$ and $\datanum\in\mathbb{N}$ any $X \sim \bidistof{p,\datanum}$ satisfies for any $t>0$
		\[ \probof{X-\expectationof{X} > t}  \leq \expof{- \frac{t^2}{2\datanum p + \frac{2t}{3}} } \]
	and
		\[ \probof{\expectationof{X} - X > t}  \leq \expof{- \frac{t^2}{2\datanum p}} \]
	Thus
		\[ \probof{|\expectationof{X} - X| > t} \leq  2 \expof{- \frac{t^2}{2\datanum p + \frac{2t}{3}} }  \]
\end{theorem}
\begin{proof}
	See e.g.
	\href{https://mathweb.ucsd.edu/~fan/wp/concen.pdf}{https://mathweb.ucsd.edu/~fan/wp/concen.pdf}
%	The proof uses the Chernoff bound applying the moment generating function, which is for Binomial variables X and $\lambda\geq0$
%	\begin{align}
%	 	\expectationof{\expof{\lambda (X - \expectationof{X})}} 
%		= & \left( (1-p) \cdot \expof{\lambda -p} + p \cdot \expof{\lambda (1-p)}\right)^\datanum \\
%		= & \expof{-\lambda p \datanum} \left(1 + p(\expof{\lambda}-1) \right)^\datanum \, .
%	\end{align}
%	The Chernoff bound used for any $\lambda>0,t>0$ the Markov inequality
%	\begin{align}
%		\probof{X-\expectationof{X} > t} 
%		= \probof{\expof{\lambda(X-\expectationof{X})} > \expof{\lambda t} }
%		\leq \frac{\expectationof{\expof{\lambda(X-\expectationof{X})}}}{\expof{\lambda t} } 
%	\end{align}
\end{proof}

The binomial thus has a sub-gaussian and a sub-exponential regime.

\begin{theorem}
	For any $p\in[0,1]$ and $\datanum\in\mathbb{N}$ any $X \sim \bidistof{p,\datanum}$ satisfies for any $t>0$
		\[ \probof{|\expectationof{X} - X| > \sqrt{4\datanum p t} + 2 t} \leq  2 \expof{- t }  \]
\end{theorem}
\begin{proof}
	For any s>0 we choose t>0 such that 
		\[ s = - \frac{t^2}{2\datanum p + \frac{t}{3}}  \]
	and observe 
		\[ \min\left( \frac{t^2}{4\datanum p},\frac{t^2}{2t} \right) \]
	and
		\[ t \leq \max(\sqrt{4\datanum p s},2s) \leq  \sqrt{4\datanum p s} + 2s \, . \]
	With the above bound it holds, that
		\[  \probof{|\expectationof{X} - X| >  \sqrt{4\datanum p s} + 2s}
		\leq \probof{|\expectationof{X} - X| > t}
		\leq 2 \expof{- \frac{t^2}{2\datanum p + \frac{t}{3}} } 
		\leq 2 \expof{-s} \, . \]
\end{proof}

We apply this on the variable  $\braket{\ftensor,\noisetensor}$.

\begin{theorem}
	Let $\ftensor = \sum_{\mlnformulain} \weightof{\exformula}\exformula$, then
	\[ \probof{\datanum |\braket{\ftensor,\noisetensor}|\geq \sqrt{4 \datanum} \cdot \left( \sum_{\mlnformulain} |\weightof{\exformula}| \sqrt{\fprobof{\exformula}} \right) \sqrt{t}  + 2 \cdot \left( \sum_{\mlnformulain} |\weightof{\exformula}|  \right) t } \leq 2|\mlnformulaset | \cdot \expof{- t} \]
\end{theorem}
\begin{proof}
	We can not assume independence of the $\braket{\exformula,\noisetensor}$ (in that case we could use a Bernstein inequality) and instead take the naive bound over all formulas in $\mlnformulaset$ 
	\begin{align}
		& \probof{|\braket{\ftensor,\noisetensor}|\geq
		 \sqrt{4 \datanum} \cdot \sum_{\mlnformulain} |\weightof{\exformula}| \sqrt{\fprobof{\exformula}}\sqrt{t}  
		 + 2 \cdot \sum_{\mlnformulain} |\weightof{\exformula}|  t } \\
		& \quad \quad \leq \probof{\exists \mlnformulain \, : \, |\braket{\exformula,\noisetensor}|\geq  \sqrt{4 \datanum}  |\weightof{\exformula}| \sqrt{\fprobof{\exformula}}\sqrt{t}  + 2 |\weightof{\exformula}|  t  }\\
		& \quad \quad \leq \sum_{\mlnformulain} \probof{ |\braket{\exformula,\noisetensor}|\geq  \sqrt{4 \datanum}  |\weightof{\exformula}| \sqrt{\fprobof{\exformula}}\sqrt{t}  + 2 |\weightof{\exformula}|  t  }\\
		& \quad \quad  \leq 2|\mlnformulaset| \cdot  \expof{-t} \, .
	\end{align}
\end{proof}

We can thus proof uniform concentration bounds given covering bounds of a hypothesis set in $\ell_1$ norm (and the reweighted one).

\begin{remark}[Small Formula Probabilities]	
	Directions with small $\fprobof{\exformula}$ will require larger covering sets and thus have large contributions to the bounds.
	Intuitively, they correspond with exceptional special cases, which need many samples to be observed. 
\red{Strange: Should also $1-\fprobof{\exformula}$ small intuitely be an issue?}
\end{remark}


\subsubsection{Generic Width Bounds}

We define the maps
	\[ \nu^p_2(\ftensor) =  \inf_{\mlnparameters : \sum_{\mlnformulain}\weightof{\exformula}\exformula = \ftensor}  \left( \sum_{\mlnformulain} |\weightof{\exformula}| \sqrt{\fprobof{\exformula}} \right) \]
and
	\[ \nu_1(\ftensor) =  \inf_{\mlnparameters : \sum_{\mlnformulain}\weightof{\exformula}\exformula = \ftensor} \left( \sum_{\mlnformulain} |\weightof{\exformula}| \sqrt{\fprobof{\exformula}} \right)  \]
corresponding to the sub-gaussian and sub-exponential regimes.

%This is almost the mixed tail definition of the thesis, just a factor on the probability

\begin{theorem}
	Let $\mlnformulaset$ be a set of formulas and $W\subset\rr^{|\mlnformulaset|}$ a set of weight vectors.
	Then with probability at least $1- |\mlnformulaset| C_1 \expof{-\frac{u^2}{2}} $ we have for the set
		\[ \Gamma = \big\{ \sum_{\mlnformulain}\weightof{\exformula}\exformula \, : \, (\weightof{\exformula})_{\mlnformulain} \in W \big\} \]
	the bound
		\[ \omega_\Gamma(\noisetensor)  \leq \frac{C_2 \gamma_{2}(\Gamma,\nu^p_2)}{\sqrt{\datanum}} + \frac{C_3 \gamma_{1}(\Gamma,\nu_1)}{\datanum} \, . \]
\end{theorem}



%% OLD: These situations are addressed more directly

%\subsubsection{Recovery of atom assignment in skeleton}
%
%We select for each placeholder in the skeleton an atom of $\variableorder$ possible choices.
%The set of formula tensors resulting from these choices is
%	\[ \Gamma^{\skeleton} = \left\{ \exformula \, : \, \exformula = \skeleton(\atomindices), \atomindices \in [\variableorder] \right\} \, .\]
%We can estimate the cardinality by
%	\[ \cardof{\Gamma^{\skeleton}} \leq \variableorder^\atomorder \, . \]
%This is just an inequality, since assignments of atoms to placeholders of the skeleton can result in identical formulas.
%
%When restricting choices by a candidatesdict, the bound can be sharpened by the product of the cardinality at each placeholder.
%
%\begin{theorem}
%	Let $\skeleton$ be a skeleton formula with $\atomorder$ placeholders and $\variableorder$ atoms, which can be selected at each position.
%	Then we have with probability at least $1-C_1\expof{-\frac{u^2}{2}}$
%		\[ \omega_\skeleton(\noisetensor)  \leq 2C_2 \sqrt{\frac{ \atomorder \ln\variableorder}{\datanum}} + 2C_3\frac{\atomorder\ln\variableorder}{\datanum}  \]
%\end{theorem}
%\begin{proof}
%	Application of the generic Dudleys entropy bound.
%\end{proof}
%
%Thus
%	\[ \datanum \sim \atomorder\ln(\variableorder) \]
%is enough for a sharp Kullback Leibler bound of the solution.
%
%
%\subsection{Recovery of weight parameters }




\appendix
\chapter{Implementation in the \tnreason package}\label{cha:implementation}

We here document the implementation of the discussed concepts in the \python package \tnreason.
 
 % Name
\tnreason is an abbreviation of \textbf{t}ensor \textbf{n}etwork \textbf{reason}ing, by which we emphasize the capabilities of this package to represent and answer reasoning tasks by tensor network contractions. 

% Installation
The package can be installed either by cloning \href{https://github.com/EnexaProject/enexa-tensor-reasoning}{https://github.com/EnexaProject/enexa-tensor-reasoning} or by
\begin{lstlisting}
	!pip install tnreason
\end{lstlisting}

\sect{Architecture}

\tnreason is structured in four subpackages and three layers
\begin{itemize}
	\item Layer 1: Storage and numerical manipulations, by subpackage \spengine, "Tensor Networks" -> building "tn" of \tnreason
	\item Layer 2: Specification of workload, subpackage \spencoding specific for storage, subpackage \spalgorithms specific for manipulations
	\item Layer 3: Applications in reasoning, by subpackage \spknowledge, "Reasoning" -> building "reason" of \tnreason
\end{itemize}

We sketch this structure by
\begin{center}
%! Author = alexgoessmann
%! Date = 09.03.25

\begin{tikzpicture}[scale=0.35]
    \draw[dashed] (-30,15) -- (12,15) -- (12,-3) -- (-30,-3) -- (-30,15);

    \draw (-10,10) rectangle (10,14);
    \node [anchor=center] at (0,12) {\spapplication};

    \node [anchor=center] at (-20,12) {\layerthreespec};
    \draw[dashed] (-30,9) -- (12,9);
    \node [anchor=center] at (-20,6) {\layertwospec};

    \draw[->-] (6,10) -- (6,8);
    \draw (2,4) rectangle (10,8);
    \node [anchor=center] at (6,6) {\spreasoning};
    \draw[->-] (6,4) -- (6,2);

    \draw[->-] (-6,10) -- (-6,8);
    \draw (-10,4) rectangle (-2,8);
    \node [anchor=center] at (-6,6) {\sprepresentation};
    \draw[->-] (-6,4) -- (-6,2);

    \draw[dashed] (-30,3) -- (12,3);
    \node [anchor=center] at (-20,0) {\layeronespec};

    \draw (-10,-2) rectangle (10,2);
    \node [anchor=center] at (0,0) {\spengine};
\end{tikzpicture}
\end{center}



\sect{Subpackage \spengine}

The \spengine subpackage is for the storage and numerical manipulation of tensors and tensor networks.

We think of it as the lowest layer, specializing in storage of Tensor Networks and performing the contractions.

\subsect{Contraction Calculus} % -> To engine section?

We have described two main encoding schemes of functions, by a direct interpretation of functions as tensors or a more relational encoding.
Both come with a different calculus scheme, which we have framed coordiante calculus and basis calculus.



\subsect{Cores and Contractions}

\textbf{Cores}

Each Tensor core has attributes
\begin{itemize}
	\item values (array-like): storing the value of the coordinates
	\item colors (list of str): specifying the name of the variables represented by its axes
	\item name (str): to distinguish from other cores
\end{itemize} 
The implemented core types differ in the values argument.
Cores are instantiated by
\begin{lstlisting}
	engine.getCore(coreType)(coreValues, coreColors, coreName)
\end{lstlisting}

\textbf{Polynomial Cores}
Polynomial Cores are implementations of the monomial decomposition or basis+ (see \defref{def:polynomialSparsity}).
Here the each tuple $(\lambda,\variableset,\catvariableof{\variableset})$ is stored as a tuple of the scalar $\lambda$ and a dictionary with $\variableset$ as keys and $\catvariableof{\variableset}$ as values.

\red{The spare cores (Polynomial and Pandas Core) exploit the matrix representation of \remref{rem:matSotrageBasPlus}.}

% Contraction Method List
The supported cores are
\begin{center}
\begin{tabular}{|c|c|c|}
  	\hline
 	\textbf{coreType} & \textbf{Package} & \text{Explanation}  \\
  	\hline
 	\stringof{NumpyTensorCore} 	&  $\mathrm{numpy}$  & Numpy array storing the values\\
  	\hline
 	\stringof{PolynomialCore} 	&  $\mathrm{numpy}$  & Storeing the values in a binary CP Decomposition\\
  	\hline
\end{tabular}
\end{center}


\textbf{Binary CP Decomposition}

Based on the monomial decomposition $\slicesparsityof{\cdot}$ as specified in \defref{def:polynomialSparsity}.
To store the values of a tensor we store the slices of tensors by the indices $\catindexof{\variableset}$. 

% Trick -> To BinaryCP
Contractions can be performed by partially contracting the cores of the decomposition.
In this way, one can avoids coordinatewise storages of high-order tensors, which can be intractable.

\textbf{Tensor Networks}

Tensor networks $\tnetof{\graph}$ are defined by hypergraphs with hyperedges decorated by tensor cores. 
We store them by dictionaries with values being tensor cores and keys coinciding with the name of each tensor core.


\textbf{Contractions}

Reflected in the notation
	\[ \contractionof{\tnetof{\graph}}{\nodevariables} \]
a contraction is defined by
\begin{itemize}
	\item Tensor Network $\tnetof{\graph}$, i.e. a dictionary of tensor cores
	\item Open Variables $\nodes$
\end{itemize}
Contraction calls are done by
\begin{lstlisting}
	engine.contract(contractionMethod, coreDict, openColors, dimensionDict, evidenceColorDict)
\end{lstlisting}
Where
\begin{itemize}
	\item contractionMethod: str, chooses one of the contraction providers
	\item coreDict: Dictionary of TensorCores (of the above formats), representing the Tensor Network $\tnetof{\graph}$ 
	\item openColors: List of str, each str identifying a color, that is a variable to be left open in the contraction
	\item dimensionDict: Dict valued by int and keys by str, storing dimensions to each variable. This is of optional usage, when a color in openColors does not appear in the coreDict.
	\item evidenceColorDict: Dict valued by int and keys by str, indicating sliced variables
\end{itemize}

% Contraction Method List
The supported contraction methods are
\begin{center}
\begin{tabular}{|c|c|c|}
  	\hline
 	\textbf{contractionMethod} (str) & \textbf{Package} & \text{Explanation}  \\
  	\hline
 	\stringof{NumpyEinsum} 	&  $\mathrm{numpy}$  & Einstein summation of $\mathrm{numpy}$ arrays\\
  	\hline
 	\stringof{TensorFlowEinsum} 	&  $\mathrm{tensorflow}$  & Einstein summation of $\mathrm{tensorflow}$ tensors\\
  	\hline
	\stringof{TorchEinsum} 	&  $\mathrm{torch}$  & Einstein summation of $\mathrm{torch}$ tensors\\
  	\hline
	\stringof{TentrisEinsum} 	&  $\mathrm{tentris}$  & Einstein summation of $\mathrm{tentris}$ hypertries\\
  	\hline
	\stringof{PgmpyVariableEliminator} 	&  $\mathrm{pgmpy}$  & Variable Elimination of DiscreteFactors in $\mathrm{pgmpy}$\\
  	\hline
	\stringof{PolynomialContractor} 	&  $\mathrm{numpy}$  & Contraction of CP Decompositions stored in $\mathrm{numpy}$ arrays\\
  	\hline	
\end{tabular}
\end{center}


%\textbf{Einstein Summation}
Contractions represented as Einstein summation, as implemented in:
\begin{itemize}
	\item numpy
	\item tensorflow
	\item pytorch
	\item tentris
\end{itemize}

%\textbf{Variable Elimination}
Contractions can be executed by variable elimination as implemented in:
\begin{itemize}
	\item pgmpy
\end{itemize}

\textbf{Manipulation of Binary CP Decomposition}
Contraction of tensors in Binary CP Decomposition as in \secref{sec:BinaryCPManipulation}.

\textbf{Coordinate Calculus}

Main function
\begin{lstlisting}
engine.coordinate_transform(coresList, transformFunction)
\end{lstlisting}

\textbf{Basis Calculus}

Main function 
\begin{centeredcode}
	engine.relational\_encoding()
\end{centeredcode}
basis calculus then based on contractions





\sect{Subpackage \spencoding}

In the \spencoding subpackage we encode maps 
Here the relational encodings $\rencodingof{\exfunction}$ of various maps $\exfunction$ are created.
%
The maps are either specified by the script language (logical formulas or neuro-symbolic architectures), categorical constraints or data.
Given a specification of a formula $\exformula$ in script language $\synencodingof{\cdot}$, the task amounts to building a semantic representation based on the syntactic specification.

We arrange the \spencoding subpackage into the second layer of the \tnreason architecture, since it specifies tensor cores which formats are specified in \spengine.


\subsect{Script Language}\label{subsec:scriptLanguage} % To encoding?

To specify propositional sentences, neuro-symbolic architectures and Markov Logic Networks, we developed a script language.

\textbf{Propositional Sentences by Nested Lists}

%\textbf{Production Rules}
Are those of Propositional Logics, but instead of brackets we nest the symbols into lists.

% Connectives
\textbf{Connectives} are represented by strings, where the following are supported (see \defref{def:connectives}):
\begin{center}
\begin{tikzpicture}
\node [anchor=center] at (0,0) {
	\begin{tabular}{|c|c|}
  	\hline
 	\textbf{Unary connective $\exconnective$} & \textbf{$\synencodingof{\exconnective}$} \\
  	\hline
 	$\lnot$ 	&  \stringof{not} \\
  	\hline
 	$()$		&  \stringof{id} \\
  	\hline
	\end{tabular}};
\node [anchor=center] at (7,0) {
	\begin{tabular}{|c|c|}
  	\hline
 	\textbf{Binary connective $\exconnective$} & \textbf{$\synencodingof{\exconnective}$} \\
  	\hline
 	$\land$ 		&  \stringof{and} \\
  	\hline
 	$\lor$ 		&  \stringof{or} \\
  	\hline
 	$\Rightarrow$ 	&  \stringof{imp} \\
  	\hline
	 $\oplus$ 		&  \stringof{xor} \\
  	\hline
	 $\Leftrightarrow$ &  \stringof{eq} \\
  	\hline
	\end{tabular}};
\end{tikzpicture}
\end{center}

% WOLFRAM Numbers
Besides these specific connectives we exploit a generic representation scheme of propositional formulas by the so-called Wolfram code orginially designed for the classification of cellular automaton rules \cite{wolfram_statistical_1983} and popularized in the book \cite{wolfram_new_2002}.
Along this, the coordinate encodings of connectives $\exconnective$ with differing arity are flattened and interpreted as a binary number, which is transformed into a decimal number and represented as a string $\synencodingof{\exconnective}$.
We then choose a prefix to encode the arity by
\begin{itemize}
	\item \stringof{u} for unary
	\item \stringof{b} for binary
	\item \stringof{t} for ternary
	\item \stringof{q} for quarternary
\end{itemize}
connectives.
Together, the connective is represented by the string concatenation
	\[  \synencodingof{\exconnective} = \synencodingof{\catorder} + \synencodingof{\exconnective} \, . \]


% Atoms
\textbf{Atomic Formulas} are represented by arbitrary strings, which are not used for the representation of connectives. 
We further avoid the symbols \{\stringof{(}, \stringof{)}, \stringof{\_}\} in the names of atoms, to not confuse them with colors of categorical variables.

% Composed Formulas
\textbf{Composed Formulas} $\exformula_1\exconnective,\exformula_2$ are represented by 
\begin{centeredcode}
	$\synencodingof{\exformula_1\exconnective,\exformula_2}$ = [$\synencodingof{\exconnective}$, $\synencodingof{\exformula_1}$, $\synencodingof{\exformula_2}$]
\end{centeredcode}
where we apply the conventions
\begin{itemize}
	\item Connectives are at the 0th position in each list
	\item Further entries are either atoms as strings or encoded formulas itself
\end{itemize}

% Backus-Naur
The applied grammar in Backus-Naur form is \\
\begin{tabular}{|l|l|}
  	\hline
 	Unary Connective & \stringof{not} | \stringof{id}\\
  	\hline
 	Binary Connective & \stringof{and} | \stringof{or} | \stringof{imp} | \stringof{xor}  | \stringof{eq} \\ 
  	\hline
 	Atomic Formula & Set of strings not in Connectives\\
  	\hline
	Complex Formula & Atomic Formula | [Unary Connective, Complex Formula] | \\
	&  [Binary Connective, Complex Formula, Complex Formula] \\
	\hline
\end{tabular}


\begin{example}[Encoding of the Wet Street example]
For example we have
\begin{itemize}
	\item 
	Atomic variable $\var{Rained}$ by
		\begin{centeredcode}
			$\synencodingof{\var{Rained}}$ 
			= \stringof{Rained}
		\end{centeredcode}
	\item 
	Negative literal $\lnot\var{Rained}$ by
		\begin{centeredcode}
			$\synencodingof{\lnot\var{Rained}}$ 
			= [\stringof{not},\stringof{Rained}]
		\end{centeredcode}
	\item 
	Horn clause $\left(\var{Rained}\Rightarrow\var{Wet}\right)$ by 
		\begin{centeredcode}
			$\synencodingof{\var{Rained}\Rightarrow\var{Wet}}$ 
			= [\stringof{imp},\stringof{Rained},\stringof{Wet}]
		\end{centeredcode}
	\item 
	Knowledge Base
	$(\lnot\var{Rained})\land(\var{Rained}\Rightarrow\var{Wet})$ by
		\begin{centeredcode}
			$\synencodingof{\lnot\var{Rained})\land(\var{Rained}\Rightarrow\var{Wet}}$ 
			=  [\stringof{and}, [\stringof{not}, \stringof{Rained}], [\stringof{imp}, \stringof{Rained}, \stringof{Wet}]]
		\end{centeredcode}
\end{itemize}
\end{example}




\textbf{Knowledge Bases}

% Should distinguish these in knowledge?
We distinguish here formulas, with propositional logic interpretation and formulas which have a soft logic interpretation.
%\textbf{Facts.} 
The formulas with hard interpretation are called facts in a knowledge base $\kb$ and encoded by dictionaries
\begin{centeredcode}
	\{key($\exformula$) : $\synencodingof{\exformula}$ for $\exformula\in\kb$ \}
\end{centeredcode}

\textbf{Markov Logic Networks}

%\textbf{Weighted formulas.} 
The formulas with soft interpretation are called weighted formulas and encoded by $\expof{\weightof{\exformula}\cdot\exformula}$.
We thus require a specification of the weights, which we do by adding $\weightof{\exformula}$ as a $\mathrm{float}$ or an $\mathrm{int}$ to the list $\synencodingof{\exformula}$.
We then store Markov Logic Networks by dictionaries
\begin{centeredcode}
	\{key($\exformula$) : $\synencodingof{\exformula}$ + [$\weightof{\exformula}$] for $\exformula\in\formulaset$\}
\end{centeredcode}

\textbf{Neuro-Symbolic Architecture by Nested Lists}

% Generalizing the script language to specify architectures
To specify neuro-symbolic architectures in terms of formula selecting maps, as has been the subject of \charef{cha:formulaSelection} we further exploit the nested list structure of encoding propositional logics.
We replace, in each hierarchy of the nested structure each entry by a list of possible choices.
In this way, we reinterpret the list index as the choice indices $\selindex$ introduced for connective and formula selections (see \defref{def:connectiveSelector} and \ref{def:formulaSelector}).

% Neurons
A connective selector (see \defref{def:connectiveSelector}) is encoded by the list
	\begin{centeredcode}
			$\synencodingof{\exconnective}$ 
			= [$\synencodingof{\exconnective_{0}}$, $\ldots$, $\synencodingof{\exconnective_{\seldim\shortminus1}}$]
	\end{centeredcode}
and a formula selector (see \defref{def:formulaSelector}) by
	\begin{centeredcode}
			$\synencodingof{\fselectionmap}$ 
			= [$\synencodingof{\exconnective_{0}}$, $\ldots$, $\synencodingof{\exconnective_{\seldim\shortminus1}}$]
	\end{centeredcode}
A logical neuron of order $\selorder$ (see \defref{def:fsNeuron}), defined by a connective selector $\exconnective$, and a formula selector $\fselectionmap_\atomenumerator$ on each argument $\atomenumerator\in[\selorder]$, is encoded by
		\begin{centeredcode}
			$\synencodingof{\lneuron}$ 
			= [$\synencodingof{\exconnective}$, $\synencodingof{\fselectionmap_0}$, $\ldots$,  $\synencodingof{\fselectionmap_{\selorder-1}}$]
		\end{centeredcode}
Only the unary $\selorder=1$ and the $\selorder=2$ cases are supported.


% Confusing?
The resulting nested lists indices have an alternating interpretation at each level compared with the elements of each list.
That is, when $\synencodingof{\lneuron}$ is the encoding of a neuron, then any element $x\in\synencodingof{\lneuron}$ represents a list of choices.
When $x$ is not the first element, then each choice is either the encoding $\synencodingof{\catvariable}$ of an atomic formula, or another neuron. 

% Find another symbol?
A neural architecture $\larchitecture$ is then represented in the dictionary
\begin{centeredcode}
	$\synencodingof{\larchitecture}$ = \{key($\lneuron$) : $\synencodingof{\lneuron}$ for $\exformula\in\larchitecture$\}
\end{centeredcode}
%To store this structure, we choose dictionaries of neuron spe
%\begin{centeredcode}
%	\{key($\lneuron$) : $\synencodingof{\lneuron}$ for $\exformula\in\formulaset$\}
%\end{centeredcode}
where key($\lneuron$) is a string, which can be used in the formula selections of other neurons.

% Important for well-definedness
It is important that the directed graph of neurons induced by the choice possibilities is acyclic, to ensure well-definedness of the architecture.


% Backus-Naur
In order to represent neuro-symbolic architectures, the grammar of $\synencodingof{\cdot}$ in Backus-Naur Form is extended by the production rules \\
\begin{tabular}{|l|l|}
  	\hline
 	Unary Connectives & [Unary Connective] | [Unary Connective] + Unary Connectives \\
  	\hline
 	Binary Connectives & [Unary Connective] | [Binary Connective] + Binary Connectives \\
	%\hline
	%Neuron Name & Any set of strings not used for atoms or connectives \\
  	\hline
 	Dependency Choice & Atomic Formula | Neuron \\ 
  	\hline
	Dependency Choices & [Dependency Choice] | [Dependency Choice] + Dependency Choices \\ 
	\hline
	Neuron & [Unary Connectives, Dependency Choices] | \\
	&  [Binary Connectives, Dependency Choices, Dependency Choices] \\
	\hline
\end{tabular}


\begin{example}[Neuro-Symbolic Architecture for the Wet Street]
	Following the wet street example, we can define a neuron by
	\begin{centeredcode}
		$\synencodingof{\lneuron}$ = [[\stringof{imp},\stringof{eq}],[\stringof{Wet},\stringof{Sprinkler}],[\stringof{Street}]] 
	\end{centeredcode}
	from which the formulas 
	\begin{centeredcode}
		[\stringof{imp}, \stringof{Wet}, \stringof{Street}] \\
		\hspace{0.25cm} [\stringof{eq}, \stringof{Wet}, \stringof{Street}] \\
		\hspace{1cm}[\stringof{imp}, \stringof{Sprinkler}, \stringof{Street}] \\
		\hspace{1cm}[\stringof{eq}, \stringof{Sprinkler}, \stringof{Street}]		
	\end{centeredcode}
	can be chosen.
	Combining this neuron with further neurons, e.g. by the architecture
	\begin{centeredcode}
		$\synencodingof{\larchitecture}$ = \{ \stringof{neur1}: [[\stringof{imp},\stringof{eq}],[\stringof{neur2}],[\stringof{Street}]] , \\
		\hspace{1.8cm}\stringof{neur2}: [[\stringof{lnot},\stringof{id}],[\stringof{Wet},\stringof{Sprinkler}],[\stringof{Street}]] \}
	\end{centeredcode}
	the expressitivity increases.
	In this case, the further neuron provides the flexibility of the first atoms to be replaced by its negation.	
\end{example}





\subsect{Core Nomenclature}

In encoding.suffixes we defined suffixes for the names of cores and colors, which highlight their origin and purpose.

% See in encoding subpackage
Cores are named with suffixes based on their functionality
\begin{itemize}
	\item \comCoreSuf: Computation core: logical connectives (relational encoding of the connective map)
	\item \actCoreSuf: Activation core: two-dimensional vectors representing of the activation core to a formula
\end{itemize}

Exploiting efficient representation tricks we further have:
\begin{itemize}
	\item \atoCoreSuf: Atomization core, for sparse representation of categorical constraints
	\item \vselCoreSuf: Variable selection core: For sparse representation of variable selectors
\end{itemize}

\subsect{Color Nomenclature}

Suffixes in color string denote the type of the variable:
\begin{itemize}
	\item \disVarSuf: Distributed variables $\catvariableof{\cdot}$
	\item \comVarSuf: Computed variables $\headvariableof{\cdot}$
	\item \selVarSuf: Selection variables $\selvariableof{\cdot}$
	\item \terVarSuf: Term variables $\indvariableof{\cdot}$
%	\item %\item Selection variables $\selvariableof{\cdot}$: \stringof{\_sVar} % differing suffixes actVar, selVar! -> unify them
%	\item Term variables $\indvariableof{\cdot}$: tVar % so far without suffix, just atom names, toDo: tVar
\end{itemize}

\subsect{Refinement by infixes}

Both the cores and the colors are further refined by infixes before the suffices to denote specific instantiations.

\begin{itemize}
	\item \selCoreIn: Involving a selection variable
	\item \eviCoreIn: Storing evidence about a variable
	\item \heaIn: Head of a function, typically the variable computed at a activation selector
	\item \funIn: Function selection variables
	\item \posIn+\stringof{i}: Variable selection for argument at position $i$
	\item \datIn: Involving data (data cores and colors)
\end{itemize}

Further infixes are strings denoting atom names and neuron neames.


\subsect{Relational encoding of formulas}

Propositional formulas $\exformula$ are represented in three schemes:
\begin{itemize}
	\item Script language $\synencodingof{\exformula}$ by nested lists (see \secref{subsec:scriptLanguage}).
		Most practical to choose a formula from a neuro-symbolic architecture.
	\item Strings specifying the categorical variables $\catvariableof{\exformula}$.
	\item Representation of formulas by tensor networks being contracted to $\rencodingof{\exformula}$
\end{itemize}

Conversions of the formats:
\begin{itemize}
	\item $\synencodingof{\exformula}$ to color by
		\begin{centeredcode} 
			encoding.get\_formula\_color($\synencodingof{\exformula}$)
		\end{centeredcode}
		Here the nested lists are turned in a string by concatenating all elements of a list with \stringof{\_} and adding \stringof{[} and \stringof{]} at the beginning and end of each list.
	\item  $\synencodingof{\exformula}$ to tensor network 
		\begin{centeredcode}
			encoding.create\_raw\_cores($\synencodingof{\exformula}$)
		\end{centeredcode}
		This creates the connective cores for the semantic representation of $\rencodingof{\exformula}$.
We encode them by
\end{itemize}

When encoding formulas with hard interpretation, we furthermore add a head core of type \stringof{truthEvaluation} since we have
 	\[ {\exformula} = \sbcontractionof{\rencodingof{\exformula},\tbasis}{\catvariableof{\exformula}} \, . \]



\subsect{Representation of MLNs}

\textbf{Computation Cores} are binary cores relating the variables in a predefined way, which is not changing during reasoning.
\begin{itemize}
	\item Logical interpretation: Cores $\rencodingof{\exconnective}$ \red{Structure Cores are those of the Bayesian Propositional Network}
	\item Categorical constraints: Cores $\categoricalcore$
	%\item Data: Cores $\datacore$
\end{itemize}

\textbf{Activation Cores} encode the weights of the formulas in a Markov Logic Network.
%For proper MLN only have unary cores, which we call headCores.
%Head cores with suffix "headCore" in name.

They are modified during reasoning: Selection of activation cores in structure learning, assigning a weight in parameter estimation.



\subsect{Formula Selecting Maps}

Encoding of Neurons according to \defref{def:fsNeuron}:
\begin{itemize}
	\item Activation selection core with suffix \stringof{actCore} in name.
		 Selection by variable with suffix \stringof{actVar}
	\item Selection of neurons as arguments with suffix \stringof{selCore} in name.
		Each argument of each neuron comes with a control variable with suffix \stringof{selVar}.
\end{itemize}

Encoding of Formula Selecting Neural Networks (\defref{def:fsNeuron}) by creating all formula selecting neurons.

Skeleton expression (\defref{def:skeleton}) are stored with placeholderkeys and the candidatelists by dictionaries with the placeholderkeys and values being the possible symbols.



\sect{Subpackage \spalgorithms}

The \spalgorithms subpackage implements basic tensor network algorithms with calls of specific execution in \spengine.
As the \spencoding subpackage it is arranged in the second layer of the \tnreason architecture, since it specifies the manipulation of tensor networks in the \spengine subpackage.


\subsect{Alternating Least Squares}

\begin{itemize}
	\item Tensor Network of Structure Cores
	\item Tensor Network of Parameter Cores
	\item List of importance cores
\end{itemize}

\begin{centeredcode}
	algorithms.ALS
\end{centeredcode}

\subsect{Gibbs Sampling}

\begin{itemize}
	\item Tensor Network of Structure Cores
	\item Parameter cores: Variable tensor network cores representing basis vectors.
	\item List of importance cores
\end{itemize}

\begin{centeredcode}
	algorithms.Gibbs
\end{centeredcode}


\subsect{Knowledge Propagation}

\begin{centeredcode}
	algorithms.ConstraintPropagator
\end{centeredcode}


\subsect{Energy-based Algorithms}

\begin{centeredcode}
	algorithms.NaiveMeanField
\end{centeredcode}

\begin{centeredcode}
	algorithms.GenericMeanField
\end{centeredcode}

\begin{centeredcode}
	algorithms.EnergyBasedGibbs
\end{centeredcode}


\sect{Subpackage \spknowledge}

With the \spknowledge subpackage we provide an interface for reasoning workload.
It builds a third layer, since it used \spencoding to represent knowledge by tensor networks and \spalgorithms in the execution of reasoning tasks.

\subsect{Distributions}

We encode Markov Networks by specifying a set of tensor cores.
Each distribution needs to have a routine
\begin{centeredcode}
	.create\_cores()
\end{centeredcode}
creating the factor cores and 
\begin{centeredcode}
	.get\_partition\_function()
\end{centeredcode}
calculating the partition function.
Although the partition function can be calculated by the contraction of all cores, we separate the method since there are situations where a faster calculation can be performed.


\textbf{Empirical Distributions} are distributions of sample data.
We represent the values as a CP Format of data cores as specified in \secref{sec:empDistribution}
\begin{centeredcode}
	knowledge.EmpiricalDistribution
\end{centeredcode}
Here the partition function is the number of samples used in the creation of the empirical distribution.


\textbf{HybridKnowledgeBases} are probability distributions, which are specified by propositional formulas in the script language.
\begin{centeredcode}
	knowledge.HybridKnowledgeBase
\end{centeredcode}
They are initialized with arguments
\begin{itemize}
	\item facts: Dictionary of propositional formulas stored as $\synencodingof{\exformula}$ representing hard logical constraints
	\item weightedFormulas: Dictionary of propositional formulas stored as $\synencodingof{\exformula}$+$[\weightof{\exformula}]$ representing soft logical constraints
	\item evidence: Dictionary of atomic formulas, where key are the formulas in string representation and values the certainty in $[0,1]$ (float or int) of the atom being true
	\item categoricalConstraints: Dictionary of categorical constrained, which values are lists of atomic formulas stored as strings $\synencodingof{\atomicformula}$
\end{itemize}


\subsect{Inference}

By
\begin{centeredcode}
	knowledge.InferenceProvider
\end{centeredcode}
taking a distribution from the above as argument.

% Probabilistic Queries
Probabilistic queries as specified \defref{def:queries})  by
\begin{centeredcode}
	.query(variableList, evidenceDict)
\end{centeredcode}

MAP queries by
\begin{centeredcode}
	.exact\_map\_query()
\end{centeredcode}
or by
\begin{centeredcode}
	.annealed\_sample()
\end{centeredcode}
using Simulated Annealing (see Remark~\ref{rem:simulatedAnnealing}) to find an approximate maximum.
The second method circumvents the creation of the coordinatewise representation of the distribution and circumvents therefore, at the expense of potentially approximative solutions, a bottleneck in case of many query variables.

% Entailment Queries
Entailment from the distribution (\defref{def:entailment}) is be decided by
\begin{centeredcode}
	.ask(queryFormula, evidenceDict)
\end{centeredcode}
where queryFormula is the formula $\exformula$ to be tested for entailment in the representation $\synencodingof{\exformula}$.

% Sampling
Samples can be drawn by
\begin{centeredcode}
	.draw\_samples(sampleNum, variableList, annealingPattern)
\end{centeredcode}
based on Gibbs sampling, where
\begin{itemize}
	\item sampleNum (int) gives the number of samples to be drawn
	\item variableList (list of str) defines the variables to be represented by the samples (default: all atoms in the distribution)
	\item annealingPattern specifies an annealing pattern 
\end{itemize}


\subsect{Parameter Estimation}

\textbf{EntropyMaximizer} implements Algorithm~\ref{alg:AWO}, which is motivated by the maximum entropy principle (see \secref{sec:maxEntDuality}) to optimize Markov Logic Networks.
The class  
\begin{centeredcode}
	knowledge.EntropyMaximizer
\end{centeredcode}
is initialized with the arguments
\begin{itemize}
	\item expressionsDict: Dictionary of formulas in the format $\synencodingof{\exformula}$ 
	\item satisfactionDict: Dictionary of the satisfaction rates (mean parameters) to be matched by the optimal distribution
\end{itemize}
The optimization is then performed by
\begin{centeredcode}
	.alternating\_optimization(sweepNum, updateKeys)
\end{centeredcode}
method, where the iteration in Algorithm~\ref{alg:AWO} through the updateKeys is performed sweepNum times.

\subsect{Structure Learning}

\textbf{Formula Booster} chooses a formula given a formula selecting map.
\begin{centeredcode}
	knowledge.FormulaBooster
\end{centeredcode}
is initialized with the arguments
\begin{itemize}
	\item knowledgeBase: Distribution representing a current model to be improved
	\item specDict: A neuro-symbolic architecture encoded in a dictionary of neurons
\end{itemize}




\bibliographystyle{plainnat}
\bibliography{literature/tensor_logic}


\end{document}