\documentclass[aps,onecolumn,nofootinbib,pra]{article}

% To use the upper directory for sources
\makeatletter
\def\input@path{{../}}
\makeatother

\usepackage{spec_files/arxiv}
\usepackage{amsmath,amsfonts,amssymb,amsthm,bbm,graphicx,enumerate,times}
\usepackage{mathtools}
\usepackage[usenames,dvipsnames]{color}
\usepackage{hyperref}
\hypersetup{
    breaklinks,
    colorlinks,
    linkcolor=gray,
    citecolor=gray,
    urlcolor=gray,
    pdftitle={The Tensor Network Approach to Efficient and Explainable AI},
    pdfauthor={Alex Goessmann}
}

\usepackage{tikz}
\usepackage{graphicx}
\usepackage{float}
\usepackage{comment}
\usepackage{csquotes}

\usepackage{listings}
\usepackage{verbatim}
\usepackage{etoolbox}
\usepackage{braket}
\usepackage[utf8]{inputenc}
\usepackage[english]{babel}
\usepackage[T1]{fontenc}
\usepackage{amsmath}
\usepackage{amsfonts}
\usepackage{amssymb}
\usepackage{amsthm}
\usepackage{titlesec}
\usepackage{tikz}
\usepackage{mathtools}
\usepackage{fancyhdr}
\usepackage{bbm}
\usepackage{bm}
\usepackage{algpseudocode}
\usepackage{algorithm}
\usepackage{lipsum}

\newtheorem{remark}{Remark}
\newtheorem{theorem}{Theorem}
\newtheorem{lemma}{Lemma}
\newtheorem{corollary}{Corollary}
\newtheorem{definition}{Definition}
\newtheorem{example}{Example}
\newtheorem{assumption}{Assumption}

% To also
\newcommand{\chapter}[1]{\section{#1}}
\newcommand{\sect}[1]{\subsection{#1}}
\newcommand{\subsect}[1]{\subsubsection{#1}}
\newcommand{\subsubsect}[1]{\paragraph{#1}}

% Text Macros
\newcommand{\python}{$\mathrm{python}$ }
\newcommand{\tnreason}{$\mathrm{tnreason}$ }
\newcommand{\spengine}{$\mathrm{engine}$ }
\newcommand{\spencoding}{$\mathrm{encoding}$ }
\newcommand{\spalgorithms}{$\mathrm{algorithms}$ }
\newcommand{\spknowledge}{$\mathrm{knowledge}$ }

\newcommand{\layeronespec}{\textbf{Layer 1}: Storage and manipulations}
\newcommand{\layertwospec}{\textbf{Layer 2}: Specification of workload}
\newcommand{\layerthreespec}{\textbf{Layer 3}: Applications in reasoning}

% Report Chapters
\newcommand{\partonetext}{Classical Approaches}
\newcommand{\chatextprobRepresentation}{Probability Distributions}
\newcommand{\chatextlogicalRepresentation}{Propositional Logic}
\newcommand{\chatextprobReasoning}{Probabilitic Inference}
\newcommand{\chatextlogicalReasoning}{Logical Inference}

\newcommand{\parttwotext}{Neuro-Symbolic Approaches}
\newcommand{\chatextformulaSelection}{Formula Selecting Networks}
\newcommand{\chatextnetworkRepresentation}{Logical Network Representation}
\newcommand{\chatextnetworkReasoning}{Logical Network Inference}
\newcommand{\chatextconcentration}{Probabilistic Guarantees}
\newcommand{\chatextfolModels}{First Order Logic}

\newcommand{\partthreetext}{Contraction Calculus}
\newcommand{\chatextcoordinateCalculus}{Coordinate Calculus}
\newcommand{\chatextbasisCalculus}{Basis Calculus}
\newcommand{\chatextsparseCalculus}{Sparse Calculus}
\newcommand{\chatextapproximation}{Tensor Approximation}
\newcommand{\chatextmessagePassing}{Message Passing}

\newcommand{\focusonespec}{Focus~I: Representation}
\newcommand{\focustwospec}{Foucs~II: Reasoning}

\newcommand{\defref}[1]{Def.~\ref{#1}}
\newcommand{\theref}[1]{The.~\ref{#1}}
\newcommand{\lemref}[1]{Lem.~\ref{#1}}
\newcommand{\algoref}[1]{Algorithm~\ref{#1}}
\newcommand{\probref}[1]{Problem~\ref{#1}}
\newcommand{\exaref}[1]{Example~\ref{#1}}
\newcommand{\parref}[1]{Part~\ref{#1}}
\newcommand{\charef}[1]{Chapter~\ref{#1}}
\newcommand{\secref}[1]{Sect.~\ref{#1}}
\newcommand{\figref}[1]{Figure~\ref{#1}}
\newcommand{\assref}[1]{Assumption~\ref{#1}}
\newcommand{\remref}[1]{Remark~\ref{#1}}

% Part Intro Texts (unused)
\newcommand{\partoneintrotext}{
    The computational automation of reasoning is rooted both in the probabilistic and the logical reasoning tradition.
    Both draw on the same ontological commitment that systems have a factored representation, that is their states are described by assignments to a set of variables.
    Based on this commitment both approaches bear a natural tensor representation of their states and a formalism of the respective reasoning algorithms based on multilinear methods.
}

\newcommand{\parttwointrotext}{
    We now employ tensor networks to define architectures and algorithms for neuro-symbolic reasoning based on the logical and probabilistic foundations.
    Markov Logic Networks will be taken as generative models to be learned from data, using formula selecting tensor networks and likelihood optimization algorithms.
}

\newcommand{\partthreeintrotext}{
    Based on the logical interpretation we often handle tensor calculus with specific tensors.
    Often, they are boolean (that is their coordinates are in $\{0,1\}$ corresponding with a Boolean), and sparse (that is having a decomposition with less storage demand).
    We investigate it in this part in more depth the properties of such tensors, which where exploited in the previous parts.
}
\newcommand{\var}[1]{\text{\emph{#1}}}

\newcommand{\synencodingof}[1]{S\left(#1\right)} % Syntax encoding!
\newcommand{\stringof}[1]{"#1"}

\newcommand{\rdf}{\mathrm{RDF}}
\newcommand{\mathrdftype}{\mathrm{rdf}\mathrm{type}}
\newcommand{\rdftype}{$\mathrm{rdf}:\mathrm{type}$}

\newcommand{\truesymbol}{\mathrm{True}}
\newcommand{\falsesymbol}{\mathrm{False}}
\newcommand{\truthset}{\{\falsesymbol,\truesymbol\}}
\newcommand{\truthstate}{z}
\newcommand{\truthstateof}[1]{\truthstate_{#1}}
\newcommand{\ozset}{\{0,1\}}
\newcommand{\ozbasisset}{\{\fbasisat{\catvariable},\tbasisat{\catvariable}\}}

\newcommand{\uniquantwrtof}[2]{\forall{#1}:{#2}}
\newcommand{\imppremhead}[2]{\left(#1\right)\Rightarrow\left(#2\right)}

\newenvironment{centeredcode}
{\begin{center}
     \begin{algorithmic}
         \hspace{1cm}}
{\end{algorithmic}\end{center}} % Use for tnreason script language, lstlistings for python code!

\newcommand{\algdefsymbol}{\leftarrow}
\newcommand{\proofrightsymbol}{"$\Rightarrow$"}
\newcommand{\proofleftsymbol}{"$\Leftarrow$"}

\newcommand{\distassymbol}{\sim}
\newcommand{\probtagtypeinst}[2]{\mathrm{P}^{#1}_{#2}}

\newcommand{\skeletoncolor}{blue}
\newcommand{\probcolor}{red}
\newcommand{\concolor}{blue}

\newcommand{\conjunctioncolor}{red}
\newcommand{\negationcolor}{blue}
\newcommand{\nodeminsize}{0.8cm}
\newcommand{\nodegrayscale}{gray!50}

% Basic Symbols
\newcommand{\entropysymbol}{\mathbb{H}}
\newcommand{\sentropyof}[1]{\entropysymbol\left[#1\right]}
\newcommand{\centropyof}[2]{\entropysymbol\left[#1,#2\right]}

\newcommand{\subspacedimof}[1]{\mathrm{dim}(#1)}

\newcommand{\subsphere}{\mathcal{S}}
\newcommand{\rr}{\mathbb{R}}
\newcommand{\nn}{\mathbb{N}}

\newcommand{\closureof}[1]{\overline{#1}}
\newcommand{\interiorof}[1]{{#1}^{\circ}}
\newcommand{\sbinteriorof}[1]{{\left(#1\right)}^{\circ}}

\newcommand{\difofwrt}[2]{\frac{\partial #1}{\partial #2}}
\newcommand{\difwrt}[1]{\difofwrt{}{#1}}
\newcommand{\gradwrt}[1]{\nabla_{#1}}
\newcommand{\gradwrtat}[2]{\nabla_{#1}|_{#2}}

\newcommand{\cardof}[1]{\left|#1\right|}
\newcommand{\absof}[1]{\left|#1\right|}

\newcommand{\imageof}[1]{\mathrm{im}\left(#1\right)}

\newcommand{\convhullof}[1]{\mathrm{conv}\left(#1\right)}
\newcommand{\cubeof}[1]{[0,1]^{#1}}
\newcommand{\dimof}[1]{\mathrm{dim}\left(#1\right)}
\newcommand{\spanof}[1]{\mathrm{span}\left(#1\right)}
\newcommand{\subspaceof}[1]{V^{#1}}

\newcommand{\argmin}{\mathrm{argmin}}
\newcommand{\argmax}{\mathrm{argmax}}

% Help functions
\newcommand{\chainingfunction}{h}
\newcommand{\chainingfunctionof}[1]{\chainingfunction\left(#1\right)}

\newcommand{\greaterthanfunction}[1]{\ones_{>#1}}
\newcommand{\greaterthanfunctionof}[2]{\greaterthanfunction{#1}\left(#2\right)}
\newcommand{\existquanttrafo}{\greaterthanfunction{0}}
\newcommand{\universalquanttrafo}{\greaterthanfunction{\inddim-1}}

\newcommand{\greaterzerofunction}{\greaterthanfunction{0}}
\newcommand{\greaterzeroof}[1]{\greaterzerofunction\left(#1\right)}



\newcommand{\nonzerofunction}{\ones_{\neq0}}
\newcommand{\nonzeroof}[1]{\nonzerofunction\left(#1\right)}
\newcommand{\nonzerocirc}{\nonzerofunction\circ}

% Probability distributions
\newcommand{\expof}[1]{\mathrm{exp}\left[#1\right]}
\newcommand{\probtensor}{\mathbb{P}}
\newcommand{\probtensorof}[1]{\probtensor^{#1}}
\newcommand{\probtensorofat}[2]{\probtensor^{#1}\left[#2\right]}
\newcommand{\secprobtensor}{\tilde{\mathbb{P}}}
\newcommand{\secprobat}[1]{\secprobtensor[#1]}

\newcommand{\probtensorset}{\Gamma}
\newcommand{\bmrealprobof}[1]{\expfamilyof{\identity,#1}} % distributions with support coinciding with the base measure in argument

\newcommand{\gendistribution}{\probtensor^*}
\newcommand{\gendistributionat}[1]{\gendistribution\left[#1\right]}
\newcommand{\currentdistribution}{\tilde{\probtensor}}

\newcommand{\probat}[1]{\probtensor\left[#1\right]}
\newcommand{\probof}[1]{\probtensor^{#1}}
\newcommand{\probofat}[2]{\probof{#1}\left[#2\right]}
\newcommand{\probwith}{\probat{\shortcatvariables}}
\newcommand{\probofwrt}[2]{\probtensor_{#1}\left[#2\right]}


\newcommand{\condprobat}[2]{\mathbb{P}\left[#1|#2\right]}
\newcommand{\condprobof}[2]{\condprobat{#1}{#2}}
\newcommand{\condprobwrtof}[3]{\mathbb{P}^{#1}\left[#2|#3\right]}
\newcommand{\margprobat}[1]{\probat{#1}}
\newcommand{\expectationof}[1]{\mathbb{E}\left[#1\right]}
\newcommand{\expectationofwrt}[2]{\mathbb{E}_{#2}\left[#1\right]}
\newcommand{\lnof}[1]{\ln \left[ #1 \right] }
\newcommand{\sgnormof}[1]{\left\|#1\right\|_{\psi_2}} % subgaussian
\newcommand{\normof}[1]{\left\|#1\right\|_{2}}

\newcommand{\distof}[1]{\mathbb{P}^{#1}}

\newcommand{\ones}{\mathbb{I}}
\newcommand{\onesof}[1]{\ones^{#1}}
\newcommand{\onesat}[1]{\ones\left[#1\right]}
\newcommand{\onesofat}[2]{\onesof{#1}\left[#2\right]}
\newcommand{\oneswith}{\onesat{\shortcatvariables}}
\newcommand{\zerosat}[1]{0\left[#1\right]}
\newcommand{\identity}{\delta}
\newcommand{\identityat}[1]{\identity\left[#1\right]}
\newcommand{\dirdeltaof}[1]{\delta^{#1}}
\newcommand{\dirdeltaofat}[2]{\dirdeltaof{#1}\left[#1\right]}

\newcommand{\deltaof}[1]{\delta_{#1}} % used in coordinate calculus proofs


\newcommand{\indicatorofat}[2]{\ones_{#1}\left[#2\right]}

\newcommand{\exmatrix}{M}
\newcommand{\matrixat}[1]{\exmatrix[#1]}
\newcommand{\matrixofat}[2]{\exmatrix^{#1}\left[#2\right]}

\newcommand{\exvector}{V}
\newcommand{\vectorof}[1]{\exvector^{#1}}
\newcommand{\vectorat}[1]{\exvector[#1]}
\newcommand{\vectorofat}[2]{\exvector^{#1}[#2]}

\newcommand{\restrictionofto}[2]{{#1}|_{#2}}
\newcommand{\restrictionoftoat}[3]{\restrictionofto{#1}{#2}\left[#3\right]}

\newcommand{\idsymbol}{\mathrm{Id}} % ! Different to delta tensor in \identity
\newcommand{\idrestrictedto}[1]{\restrictionofto{\idsymbol}{#1}}

%% KNOWLEDGE GRAPH
\newcommand{\kg}{\mathrm{KG}|_{\dataworld}}
\newcommand{\kgat}[1]{\kg\left[#1\right]}
\newcommand{\kgreptensor}{\rencodingof{\kg}}

\newcommand{\exaunaryrelation}{C}
\newcommand{\exabinaryrelation}{R}

\newcommand{\kgtriple}[3]{\braket{#1,#2,#3}}
\newcommand{\exunarytriple}{\kgtriple{\provariable}{\mathrdftype}{\exaunaryrelation}}
\newcommand{\exbinarytriple}{\kgtriple{\provariableof{0}}{\exabinaryrelation}{\provariableof{1}}}

\newcommand{\atomcreator}{\psi}
\newcommand{\atomcreatorofat}[2]{\atomcreator_{#1}\left[#2\right]}
\newcommand{\provariable}{Z}
\newcommand{\provariableof}[1]{\provariable_{#1}}

\newcommand{\sparql}{\mathrm{SPARQL}}
\newcommand{\joinsymbol}{\mathrm{JOIN}}

\newcommand{\subsymbol}{s}
\newcommand{\predsymbol}{p}
\newcommand{\objsymbol}{o}

\newcommand{\sindvariable}{\indvariableof{\subsymbol}}
\newcommand{\pindvariable}{\indvariableof{\predsymbol}}
\newcommand{\oindvariable}{\indvariableof{\objsymbol}}

\newcommand{\invrdftypesymbol}{\mathrm{typ}}



% Propositional Logics: New square bracket notation
\newcommand{\formula}{f}
\newcommand{\formulaof}[1]{\formula_{#1}}
\newcommand{\formulaat}[1]{\formula\left[#1\right]}
\newcommand{\formulaofat}[2]{\formulaof{#1}\left[#2\right]}



\newcommand{\formulavar}{\headvariableof{\formula}}
\newcommand{\formulacc}{\rencodingof{\formula}} % computation core
\newcommand{\formulaccwith}{\rencodingofat{\formula}{\formulavar,\shortcatvariables}}

\newcommand{\enumformula}{\formulaof{\selindex}}
\newcommand{\enumformulaat}[1]{\enumformula\left[#1\right]}

\newcommand{\enumformulavar}{\headvariableof{\selindex}}

\newcommand{\enumformulacc}{\rencodingof{\enumformula}} % computation core
\newcommand{\enumformulaccwith}{\rencodingofat{\enumformula}{\enumformulavar,\shortcatvariables}}
\newcommand{\enumformulaac}{\actcoreof{\enumformula,\canparamat{\indexedselvariable}}}
\newcommand{\enumformulaacwith}{\actcoreofat{\enumformula,\canparamat{\indexedselvariable}}{\headvariableof{\enumformula}}}

\newcommand{\exformula}{\formula}
\newcommand{\exformulavar}{\headvariableof{\exformula}}
\newcommand{\exformulaat}[1]{\exformula\left[#1\right]}

\newcommand{\formulazerocoordinates}{\shortcatindices\,:\,\formulaat{\shortcatindices}=0}
\newcommand{\formulaonecoordinates}{\shortcatindices\,:\,\formulaat{\shortcatindices}=1}

\newcommand{\secexformula}{h} % Since g is atom
\newcommand{\secexformulavar}{\headvariableof{\secexformula}}
\newcommand{\secexformulaat}[1]{\secexformula\left[#1\right]}

\newcommand{\exformulain}{\exformula\in\formulaset}
\newcommand{\exformulaof}[1]{\exformula\left(#1\right)}

\newcommand{\formulasuperset}{\mathcal{H}}

% First order Logics
\newcommand{\folexformula}{q}
 % When representing \folexformula as \importancequery \rightarrow \headfolexformula
\newcommand{\folformulaset}{\mathcal{Q}}
\newcommand{\folexformulain}{\folexformula\in\folformulaset}
\newcommand{\folexformulaof}[1]{\folexformula_{#1}}
\newcommand{\restfolformulaset}{\restrictionofto{\folformulaset}{\worlddomain}}

\newcommand{\enumfolformula}{\folexformulaof{\selindex}}
\newcommand{\enumfolformulaat}[1]{\enumfolformula\left[#1\right]}

\newcommand{\headfolformula}{h}
\newcommand{\headfolexformula}{\headfolformula}
\newcommand{\headfolformulaof}[1]{\headfolformula_{#1}}
\newcommand{\headfolformulaofat}[2]{\headfolformulaof{#1}\left[#2\right]}

\newcommand{\folpredicate}{g}
\newcommand{\folpredicateof}[1]{\folpredicate_{#1}}
\newcommand{\folpredicates}{\folpredicateof{0},\ldots,\folpredicateof{\folpredicateorder-1}}
\newcommand{\folpredicateenumerator}{\atomenumerator} % Due to PL being a special case
\newcommand{\folpredicateorder}{\atomorder}

\newcommand{\worlddomain}{\arbset} % Snce enumerated
\newcommand{\exindividual}{a}
\newcommand{\secindividual}{b}
\newcommand{\exindividualof}[1]{\exindividual_{#1}}

\newcommand{\atombasemeasure}{\nu}

\newcommand{\individuals}{\exindividualof{\indindexof{0}},\ldots,\exindividualof{\indindexof{\individualorder-1}}}
\newcommand{\individualsof}[1]{\exindividualof{0}^{#1},\ldots,\exindividualof{\individualorder-1}^{#1}} % Do not use, index already in individuals


%% Redundant to individual variables
\newcommand{\individualvariable}{\indvariable}
\newcommand{\individualvariableof}[1]{\indvariableof{#1}}
\newcommand{\individualvariables}{\indvariablelist}

\newcommand{\individualorder}{\indorder}
\newcommand{\individualenumerator}{\indenumerator}
\newcommand{\individualenumeratorin}{\indenumeratorin}

\newcommand{\variableindex}{\indindex}
\newcommand{\variableindexof}[1]{\indindexof{#1}}
\newcommand{\variableenumerator}{\indenumerator}
\newcommand{\variableorder}{\indorder}
\newcommand{\variableenumeratorin}{\indenumeratorin}
\newcommand{\variableindices}{\indindexof{0}\ldots\indindexof{\indorder-1}}

\newcommand{\exconnective}{\circ}
\newcommand{\connectiveof}[1]{\exconnective_{#1}}
\newcommand{\connectiveofat}[2]{\connectiveof{#1}\left[#2\right]}

\newcommand{\folworldsymbol}{W}

\newcommand{\dataworld}{\catindexof{\folworldsymbol}}
\newcommand{\dataworldat}[1]{\dataworld[#1]}
\newcommand{\dataworldwith}{\dataworldat{\selvariable,\shortindvariables}}

\newcommand{\randworld}{\catvariableof{\folworldsymbol}}
\newcommand{\indexedrandworld}{\indexedcatvariableof{\folworldsymbol}}


\newcommand{\groundingofatwrt}[3]{{#1}|_{#3} \left[#2\right]}
\newcommand{\groundingofat}[2]{{#1}|_{\dataworld} \left[#2\right]}
\newcommand{\groundingof}[1]{{#1}|_{\dataworld}}
\newcommand{\kggroundingof}[1]{{#1}|_{\dataworld}}
\newcommand{\kggroundingofat}[2]{\kggroundingof{#1}\left[#2\right]}


% Used in FOL Models
%\newcommand{\gtensor}{\rho} % For decompositions
%\newcommand{\gtensorof}[1]{\gtensor^{#1}}

%% For the TCalculus Theorem
\newcommand{\coordinatetrafo}{\chainingfunction}
\newcommand{\gentensor}{T}
\newcommand{\basisslices}{U}

% Parameters 
\newcommand{\candidatelist}{\mathcal{M}}
\newcommand{\candidatelistof}[1]{\candidatelist^{#1}}

% Data Extraction Spec
\newcommand{\impformula}{p}
\newcommand{\fixedimpformula}{\underline{\impformula}}
\newcommand{\fixedimpformulawith}{\underline{\impformula}\left[\indvariableof{\impformula}\right]}
\newcommand{\fixedimpbm}{\basemeasureofat{\fixedimpformula}{\randworld}}
\newcommand{\supportedworlds}{\dataworld \, : \, \groundingofat{\impformula}{\shortindvariables} = \fixedimpformulawith}
\newcommand{\impformulaat}[1]{\impformula\left[#1\right]}

%\newcommand{\decgroundedimpformula}{\groundingof{\impformula}^{\mathrm{enum}}}


\newcommand{\extformula}{q}
\newcommand{\extformulaof}[1]{\extformula_{#1}}
\newcommand{\extformulaofat}[2]{\extformulaof{#1}\left[#2\right]}
\newcommand{\extformulas}{\extformulaof{0},\ldots,\extformulaof{\atomorder-1}}
\newcommand{\shortextformulas}{\extformulaof{[\atomorder]}}

\newcommand{\extractionrelation}{\exrelation}

\newcommand{\variableset}{A} % Still used in monomial decomposition, NOT for object sets!
\newcommand{\variablesetof}[1]{\variableset^{#1}}

\newcommand{\formulaset}{\mathcal{F}}
\newcommand{\formulasetof}[1]{\formulaset_{#1}}

\newcommand{\secformulaset}{\tilde{\formulaset}}

\newcommand{\hardformulaset}{\kb}
\newcommand{\hfbasemeasure}{\basemeasureof{\hardformulaset}}
\newcommand{\hfbasemeasureat}[1]{\hfbasemeasure\left[#1\right]}
\newcommand{\softformulaset}{\formulaset}


% Formula Selecting
\newcommand{\larchitecture}{\mathcal{A}}
\newcommand{\larchitectureat}[1]{\larchitecture\left[#1\right]}

\newcommand{\inneuronset}{\mathcal{A}^{\mathrm{in}}}
\newcommand{\outneuronset}{\mathcal{A}^{\mathrm{out}}}

\newcommand{\lneuron}{\sigma}
\newcommand{\lneuronof}[1]{\lneuron_{#1}}
\newcommand{\lneuronat}[1]{\lneuron\left[#1\right]}
\newcommand{\lneuractivation}{\lneuron^{\larchitecture}}
\newcommand{\lneuractivationat}[1]{\lneuractivation\left[#1\right]}

\newcommand{\fsnn}{\fselectionmapof{\larchitecture}}
\newcommand{\fsnnat}[1]{\fsnn\left[#1\right]}

\newcommand{\sliceselectionmapof}[1]{\fselectionmapof{\land,#1}}
\newcommand{\sliceselectionmapofat}[2]{\fselectionmapofat{\land,#1}{#2}}
\newcommand{\sliceselectionmapat}[1]{\sliceselectionmapofat{\catorder,\sliceorder}{#1}}

\newcommand{\skeleton}{S}
\newcommand{\skeletonof}[1]{\skeleton\left(#1\right)}
\newcommand{\skeletontensor}{\rencodingof{\skeleton}} %OLD! Use skeleton

\newcommand{\skeletoncore}{S}
\newcommand{\skeletoncoreof}[1]{\skeletoncore^{#1}}

\newcommand{\cselectionsymbol}{C}
\newcommand{\vselectionsymbol}{V}
\newcommand{\sselectionsymbol}{S}

\newcommand{\selinputvariable}{\selvariable}
\newcommand{\cselinputvariable}{\selvariableof{\cselectionsymbol}}
\newcommand{\vselinputvariable}{\selvariableof{\vselectionsymbol}}

\newcommand{\fselectionmap}{\mathcal{H}}
\newcommand{\fselectionmapof}[1]{\fselectionmap_{#1}}
\newcommand{\fselectionmapat}[1]{\fselectionmap\left[#1\right]}
\newcommand{\fselectionmapofat}[2]{\fselectionmap_{#1}\left[#2\right]}

\newcommand{\cselectionmap}{\fselectionmapof{\cselectionsymbol}}
\newcommand{\cselectionmapat}[1]{\fselectionmapofat{\cselectionsymbol}{#1}}

\newcommand{\vselectionmap}{\fselectionmapof{\vselectionsymbol}}
\newcommand{\vselectionmapat}[1]{\fselectionmapofat{\vselectionsymbol}{#1}}
\newcommand{\vselectionheadvar}{\headvariableof{\vselectionsymbol}} % Replacing \catvariableof{\vselectionmap}

\newcommand{\sselectionmap}{\fselectionmapof{\sselectionsymbol}}
\newcommand{\sselectionmapat}[1]{\fselectionmapofat{\sselectionsymbol}{#1}}

\newcommand{\vselectionmapof}[1]{\fselectionmapof{\vselectionsymbol,#1}} % tb deleted!

\newcommand{\tranfselectionmap}{\fselectionmap^T}

% Output variables - Following the catvariable convention
\newcommand{\seloutputvariable}{\randomx}
\newcommand{\cseloutputvariable}{\catvariableof{\cselectionsymbol}}
\newcommand{\vseloutputvariable}{\headvariableof{\vselectionsymbol}}

% Tensor Core Representation
\newcommand{\selectorcore}{{\rencodingof{\vselectionsymbol}}}
\newcommand{\selectorcoreof}[1]{\rencodingof{\vselectionmapof{#1}}}

\newcommand{\selectorcomponentof}[1]{\hypercoreof{\vselectionsymbol_{#1}}} % Since not an relational encoding!
\newcommand{\selectorcomponentofat}[2]{\selectorcomponentof{#1}\left[#2\right]}

\newcommand{\parspace}{\bigotimes_{\selenumeratorin}\rr^{\seldimof{\selenumerator}}}
\newcommand{\simpleparspace}{\rr^{\seldim}}

\newcommand{\unitvectoratof}[2]{e^{(#1)}_{#2}}
\newcommand{\parametrizingunittensor}{e_{\atomindices}} % Not required?

\newcommand{\placeholder}{Z} %% When not used in formulas, take the set for it
\newcommand{\placeholderof}[1]{\placeholder^{#1}}

\newcommand{\atomicformula}{\catvariable}
\newcommand{\atomicformulaof}[1]{\catvariableof{#1}}
\newcommand{\atomicformulaofat}[2]{\catvariableof{#1}\left[#2\right]}
\newcommand{\atomicformulas}{\catvariableof{[\atomorder]}} %{\{\atomicformulaof{\atomenumerator} :  \atomenumeratorin \}}
\newcommand{\enumeratedatoms}{\atomicformulaof{0},\ldots,\atomicformulaof{\atomorder-1}}
\newcommand{\atomformulaset}{\formulasetof{\mlnatomsymbol}}

\newcommand{\clause}{Z^{\lor}}
\newcommand{\clauseof}[2]{\clause_{#1,#2}}
\newcommand{\clauseofat}[3]{\clauseof{#1}{#2}\left[#3\right]}
\newcommand{\maxtermof}[1]{\clause_{#1}}
\newcommand{\maxtermformulaset}{\formulasetof{\mlnmaxtermsymbol}}

\newcommand{\term}{Z^{\land}}
\newcommand{\termof}[2]{\term_{#1,#2}}
\newcommand{\termofat}[3]{\termof{#1}{#2}\left[#3\right]}
\newcommand{\mintermof}[1]{\term_{#1}}
\newcommand{\mintermofat}[2]{\mintermof{#1}\left[#2\right]}
\newcommand{\mintermformulaset}{\formulasetof{\mlnmintermsymbol}}

\newcommand{\indexedplaceholderof}[1]{\placeholderof{#1}_{\atomlegindexof{#1}}}
\newcommand{\indexedplaceholders}{\indexedplaceholderof{1},\ldots,\indexedplaceholderof{\atomorder}}

\newcommand{\atomorder}{d}
\newcommand{\secatomorder}{r}
\newcommand{\atomenumerator}{k}
\newcommand{\secatomenumerator}{l}

\newcommand{\atomenumeratorin}{\atomenumerator\in[\atomorder]}
\newcommand{\secatomenumeratorin}{\secatomenumerator\in[\secatomorder]}
\newcommand{\atomlegindex}{\catindex}
\newcommand{\tatomlegindex}{\tilde{\atomlegindex}}
\newcommand{\atomlegindexof}[1]{\atomlegindex_{#1}}
\newcommand{\tatomlegindexof}[1]{\tatomlegindex_{#1}}
\newcommand{\atomindices}{{\atomlegindexof{0},\ldots,\atomlegindexof{\atomorder-1}}}
\newcommand{\atomindicesin}{\atomindices\in\atomstates}

%% OPTIMIZATION
\newcommand{\targettensor}{Y}
\newcommand{\importancetensor}{I}

%% MARKOV LOGIC NETWORK
\newcommand{\loss}{\mathcal{L}_{\datamap}}
\newcommand{\lossof}[1]{\loss\left(#1\right)}
\newcommand{\mlnformulaset}{\mathcal{F}}
\newcommand{\mlnformulain}{\exformula\in\mlnformulaset}
\newcommand{\weight}{\theta}
\newcommand{\weightof}[1]{\weight_{#1}}
\newcommand{\weightat}[1]{\weight[#1]}

\newcommand{\mlnparameters}{\formulaset,\canparam}
%\newcommand{\mlnparameterswithout}{\tilde{\formulaset},\canparamt}
\newcommand{\mlntrueparameters}{(\formulaset^*,\weight^*)}



% Examples
\newcommand{\mlnatomsymbol}{[\catorder]}
\newcommand{\mlnmintermsymbol}{\land}
\newcommand{\mlnmaxtermsymbol}{\lor}

\newcommand{\partitionfunction}{\mathcal{Z}}
\newcommand{\secpartitionfunction}{\tilde{\mathcal{Z}}}
\newcommand{\partitionfunctionof}[1]{\partitionfunction\left(#1\right)}
\newcommand{\secpartitionfunctionof}[1]{\secpartitionfunction\left(#1\right)}

\newcommand{\mlnprob}{\probtensorof{\mlnparameters}}
\newcommand{\mlnprobat}[1]{\expdistofat{\mlnparameters}{#1}}
\newcommand{\mlnenergy}{\energytensorof{\mlnparameters}}

\newcommand{\folmlnparameters}{\restfolformulaset,\canparam,\basemeasureof{\fixedimpformula}}

% For Probabilistic Analysis
\newcommand{\kldivsymbol}{\mathrm{D}_{\mathrm{KL}}}
\newcommand{\kldivof}[2]{\kldivsymbol\left[ #1 || #2 \right]}

\newcommand{\noisetensor}{\eta}
\newcommand{\noiseat}[1]{\noisetensor\left[#1\right]}
\newcommand{\noiseof}[1]{\noisetensor^{#1,\gendistribution,\datamap}}
\newcommand{\sstatnoise}{\noiseof{\sstat}}
\newcommand{\mintermnoise}{\noiseof{\identity}}
\newcommand{\mlnnoise}{\noiseof{\mlnstat}}
\newcommand{\mlnnoiseat}[1]{\mlnnoise\left[#1\right]}

\newcommand{\fprob}{p} % Drop! This is mean parameter
\newcommand{\fprobof}[1]{\fprob^{(#1)}}

\newcommand{\bidistof}[1]{B\left(#1\right)}
\newcommand{\multidistof}[1]{\underline{B}\left(#1\right)}
\newcommand{\widthwrtof}[2]{\omega_{#1}\left(#2\right)}
\newcommand{\widthatof}[2]{\widthwrtof{#1}{#2}}

\newcommand{\selbasisshort}{\Gamma}
\newcommand{\selbasislong}{\{\onehotmapofat{\selindex}{\selvariable} \,:\, \selindexin \}}

\newcommand{\failprob}{\epsilon}
\newcommand{\precision}{\tau}
\newcommand{\maxgap}{\Delta}
\newcommand{\maxgapof}[1]{\maxgap\left(#1\right)}

%% CONTRACTION 
\newcommand{\invtemp}{\beta}

%% Hard Logic
\newcommand{\kb}{\mathcal{KB}}
\newcommand{\kbvar}{\headvariableof{\kb}}
\newcommand{\kbat}[1]{\kb\left[#1\right]}

\newcommand{\seckb}{\tilde{\kb}}

%% Tensor Network Formats
\newcommand{\elformat}{\mathrm{EL}}
\newcommand{\cpformat}{\mathrm{CP}}
\newcommand{\htformat}{\mathrm{HT}}
\newcommand{\ttformat}{\mathrm{TT}}

\newcommand{\extnet}{\mathcal{T}^{\graph}}
\newcommand{\secextnet}{\mathcal{T}^{\tilde{\graph}}}
\newcommand{\extnetat}[1]{\extnet\left[#1\right]}

\newcommand{\objof}[1]{O\left(#1\right)} % Drop!

\newcommand{\nodevariables}{\catvariableof{\nodes}}
\newcommand{\nodevariablesof}[1]{\catvariableof{\nodesof{#1}}}
\newcommand{\indexednodevariables}{\indexedcatvariableof{\nodes}}
\newcommand{\edgevariables}{\catvariableof{\edge}}
\newcommand{\extnetdist}{\normationof{\extnet}{\nodevariables}}

\newcommand{\extnetasset}{\{\hypercoreofat{\edge}{\catvariableof{\edge}}\, : \, \edge\in\edges\}}
\newcommand{\tnetof}[1]{\mathcal{T}^{#1}}
\newcommand{\tnetofat}[2]{\tnetof{#1}\left[#2\right]}

%% Probability Representation
\newcommand{\randomx}{\catvariable}

\newcommand{\exrandom}{\catvariableof{0}}
\newcommand{\secexrandom}{\catvariableof{1}}
\newcommand{\thirdexrandom}{\catvariableof{2}}

\newcommand{\indexedexrandom}{\indexedcatvariableof{0}}
\newcommand{\indexedsecexrandom}{\indexedcatvariableof{1}}
\newcommand{\indexedthirdexrandom}{\thirdexrandom=\thirdexrandind}

\newcommand{\exrandind}{\catindexof{0}}
\newcommand{\exranddim}{\catdimof{0}}
\newcommand{\exrandindin}{\exrandind\in[\exranddim]}

\newcommand{\secexrandind}{\catindexof{1}}
\newcommand{\secexranddim}{\catdimof{1}}
\newcommand{\secexrandindin}{\secexrandind\in[\secexranddim]}

\newcommand{\thirdexrandind}{\catindexof{2}}
\newcommand{\thirdexranddim}{\catdimof{2}}
\newcommand{\thirdexrandindin}{\thirdexrandind\in[\thirdexranddim]}

% Hidden Markov Models
\newcommand{\randomxof}[1]{\randomx_{#1}} % In combination with atomenumerator or tenumerator
\newcommand{\randome}{E}
\newcommand{\randomeof}[1]{\randome_{#1}}
\newcommand{\tenumerator}{t}
\newcommand{\tdim}{T}
\newcommand{\tenumeratorin}{\tenumerator\in[\tdim]}

%% Exponential families
\newcommand{\expdistof}[1]{\probtensorof{#1}}
\newcommand{\expdistofat}[2]{\expdistof{#1}[#2]}
\newcommand{\expdist}{\probtensorof{(\sstat,\canparam,\basemeasure)}}
\newcommand{\expdistwith}{\probtensorofat{(\sstat,\canparam,\basemeasure)}{\shortcatvariables}}
\newcommand{\expdistat}[1]{\expdist\left[#1\right]}
\newcommand{\stanexpdistof}[1]{\expdistof{(\sstat,#1,\basemeasure)}}
\newcommand{\mlnexpdistof}[1]{\expdistof{(\formulaset,#1,\basemeasure)}}

\newcommand{\expfamilyof}[1]{\Gamma^{#1}}
\newcommand{\expfamily}{\expfamilyof{\sstat,\basemeasure}}

\newcommand{\realizabledistsof}[1]{\Lambda^{#1}}
\newcommand{\hlnsetof}[1]{\realizabledistsof{#1,\elformat}}

\newcommand{\mnexpfamily}{\expfamilyof{\graph,\ones}} % The exponential family of Markov Networks on \graph
\newcommand{\mlnexpfamily}{\expfamilyof{\mlnstat,\ones}}

\newcommand{\basemeasure}{\nu}
\newcommand{\basemeasureof}[1]{\basemeasure^{#1}}
\newcommand{\basemeasureofat}[2]{\basemeasure^{#1}\left[#2\right]}
\newcommand{\basemeasureat}[1]{\basemeasure\left[#1\right]}
\newcommand{\basemeasurewith}{\basemeasureat{\shortcatvariables}}

\newcommand{\secbasemeasure}{\tilde{\nu}}
\newcommand{\secbasemeasureat}[1]{\secbasemeasure\left[#1\right]}

\newcommand{\sstat}{\phi}
\newcommand{\sstatat}[1]{\sstat\left(#1\right)}
\newcommand{\sstatof}[1]{\sstat^{#1}}
\newcommand{\secsstat}{\tilde{\sstat}}
\newcommand{\proposalstat}{\fselectionmap^T}
\newcommand{\mlnstat}{\formulaset}
\newcommand{\naivestat}{\identity}

\newcommand{\sstatcoordinateof}[1]{\sstat_{#1}}
\newcommand{\sstatcoordinate}{\sstatcoordinateof{\selindex}}
\newcommand{\sstatcoordinateofat}[2]{\sstat_{#1}\left[#2\right]}

\newcommand{\sstatheadvariables}{\headvariableof{[\seldim]}}

\newcommand{\sstatcc}{\rencodingof{\sstat}}
\newcommand{\sstatccwith}{\rencodingofat{\sstat}{\sstatheadvariables,\shortcatvariables}}
\newcommand{\sstatac}{\actcoreof{\sstatcoordinateof{\selindex},\canparamat{\indexedselvariable}}}
\newcommand{\sstatacwith}{\actcoreofat{\sstatcoordinateof{\selindex},\canparamat{\indexedselvariable}}{\headvariableof{\selindex}}}

\newcommand{\sstatcatof}[1]{\headvariableof{#1}}

\newcommand{\sstatindof}[1]{\catindexof{\sstatcoordinateof{#1}}}
\newcommand{\sencsstat}{\sencodingof{\sstat}}
\newcommand{\sencsstatat}[1]{\sencodingof{\sstat}\left[#1\right]}
\newcommand{\sencsstatwith}{\sencsstatat{\shortcatvariables,\selvariable}}

\newcommand{\sencfset}{\sencodingof{\formulaset}}
\newcommand{\sencfsetat}[1]{\sencfset\left[#1\right]}

\newcommand{\sencmlnstat}{\sencodingof{\mlnstat}}
\newcommand{\sencproposalstat}{\sencodingof{\proposalstat}}

\newcommand{\canparam}{\theta}
\newcommand{\canparamof}[1]{\canparam_{#1}}
\newcommand{\canparamat}[1]{\canparam\left[#1\right]}
\newcommand{\canparamofat}[2]{\canparamof{#1}\left[#2\right]}
\newcommand{\canparamwith}{\canparamat{\selvariable}}
\newcommand{\canparamwithin}{\canparamwith\in\simpleparspace}

\newcommand{\singlecanparam}{\canparam}

\newcommand{\seccanparam}{\tilde{\canparam}}

\newcommand{\canparamwrtat}[2]{\canparamofat{#1}{#2}}
\newcommand{\estcanparam}{\hat{\canparam}}
\newcommand{\naivecanparam}{\tilde{\canparam}}
\newcommand{\naivecanparamat}[1]{\naivecanparam\left[#1\right]}

\newcommand{\datacanparam}{\canparamof{\datamap}}
\newcommand{\datacanparamat}[1]{\canparamofat{\datamap}{#1}}

\newcommand{\gencanparam}{\canparamof{*}}
\newcommand{\gencanparamat}[1]{\canparamofat{*}{#1}}

\newcommand{\canparamhypothesis}{\Gamma}
\newcommand{\canparamin}{\canparam\in\canparamhypothesis}

\newcommand{\expsolution}{\gencanparam}
\newcommand{\empsolution}{\datacanparam}

\newcommand{\meanparam}{\mu}
\newcommand{\secmeanparam}{\tilde{\mu}}
\newcommand{\meanparamof}[1]{\meanparam_{#1}}
\newcommand{\meanparamat}[1]{\meanparam\left[#1\right]}
\newcommand{\meanparamofat}[2]{\meanparamof{#1}\left[#2\right]}
\newcommand{\meanparamwith}{\meanparamat{\selvariable}}

\newcommand{\meanrepprob}{\probtensor^{\meanparam}}

\newcommand{\meanset}{\mathcal{M}}
\newcommand{\meansetof}[1]{\meanset_{#1}}
\newcommand{\genmeanset}{\meanset_{\sstat,\basemeasure}}
\newcommand{\hlnmeanset}{\meanset_{\mlnstat,\basemeasure}}
\newcommand{\propmeanset}{\meanset_{\propstat,\ones}}

\newcommand{\genmeansetargmax}{\argmax_{\meanparam\in\genmeanset}}
\newcommand{\cangenmeansetargmax}{\genmeansetargmax\contraction{\canparamwith,\meanparamwith}}
\newcommand{\cansstatcatindicesargmax}{\argmax_{\shortcatindices}\contraction{\canparam,\sstat(\shortcatindices)}}

\newcommand{\normalvec}{a}
\newcommand{\normalbound}{b}
\newcommand{\normalvecofat}[2]{\normalvec_{#1}\left[#2\right]}
\newcommand{\normalboundof}[1]{\normalbound_{#1}}
\newcommand{\normalboundofat}[2]{\normalbound_{#1}\left[#2\right]}
\newcommand{\halfspaceparams}{\left( (\normalvecofat{i}{\selvariable},\normalboundof{i}) \, : \, i \in[n]\right)}
\newcommand{\facecondset}{\mathcal{I}}
\newcommand{\faceset}{Q}
\newcommand{\genfacesetof}[1]{\faceset^{#1}_{\sstat,\basemeasure}}

\newcommand{\datamean}{\meanparamof{\datamap}}
\newcommand{\datameanat}[1]{\datamean\left[#1\right]}

\newcommand{\genmean}{\meanparam^*}
\newcommand{\genmeanat}[1]{\genmean[#1]}

\newcommand{\currentmean}{\tilde{\meanparam}}

\newcommand{\cumfunctionwrt}[1]{A^{#1}}
\newcommand{\cumfunctionwrtof}[2]{\cumfunctionwrt{#1}\left(#2\right)}
\newcommand{\cumfunction}{\cumfunctionwrt{(\sstat,\basemeasure)}}
\newcommand{\cumfunctionof}[1]{\cumfunction(#1)}
\newcommand{\dualcumfunction}{\big(\cumfunction\big)^*}

\newcommand{\forwardmapwrt}[1]{F^{#1}}
\newcommand{\forwardmap}{\forwardmapwrt{(\sstat,\basemeasure)}}
\newcommand{\forwardmapwrtof}[2]{\forwardmapwrt{#1}(#2)}
\newcommand{\forwardmapof}[1]{\forwardmapwrtof{(\sstat,\basemeasure)}{#1}}

\newcommand{\backwardmapwrt}[1]{B^{#1}}
\newcommand{\backwardmap}{\backwardmapwrt{(\sstat,\basemeasure)}}
\newcommand{\backwardmapwrtof}[2]{\backwardmapwrt{#1}(#2)}
\newcommand{\backwardmapof}[1]{\backwardmapwrtof{(\sstat,\basemeasure)}{#1}}

\newcommand{\energytensor}{E}
\newcommand{\energytensorofat}[2]{\energytensor^{#1}[#2]}
\newcommand{\energytensorof}[1]{\energytensor^{#1}}
\newcommand{\energytensorat}[1]{\energytensor\left[#1\right]}
\newcommand{\expenergy}{\energytensorofat{(\sstat,\canparam,\basemeasure)}{\shortcatvariables}}
\newcommand{\expenergyat}[1]{\energytensorofat{(\sstat,\canparam,\basemeasure)}{#1}}

\newcommand{\energyhypothesis}{\Theta}
\newcommand{\energyhypothesisof}[1]{\energyhypothesis^{#1}}

%% Logical Reasoning
\newcommand{\kcore}{K}
\newcommand{\kcoreof}[1]{\kcore^{#1}}
\newcommand{\kcoreofat}[2]{\kcore^{#1}\left[#2\right]}


\newcommand{\tbasis}{e_1}
\newcommand{\tbasisat}[1]{\tbasis\left[#1\right]}
\newcommand{\fbasis}{e_0}
\newcommand{\fbasisat}[1]{\fbasis\left[#1\right]}
\newcommand{\nbasis}{\ones}

\newcommand{\graph}{\mathcal{G}}
\newcommand{\graphof}[1]{\graph^{#1}}
\newcommand{\secgraph}{\tilde{\graph}}
\newcommand{\nodes}{\mathcal{V}}
\newcommand{\nodesof}[1]{\nodes^{#1}}
\newcommand{\innodes}{\nodesof{\mathrm{in}}}
\newcommand{\outnodes}{\nodesof{\mathrm{out}}}

\newcommand{\domainsymbol}{k}
\newcommand{\domainedges}{\edgesof{\domainsymbol}}

\newcommand{\graphqueue}{\mathcal{Q}}

\newcommand{\elgraph}{\graphof{\elformat}}
\newcommand{\maxgraph}{\graphof{\mathrm{max}}}

\newcommand{\prenodes}{\{\secnode \, : \, \secnode \prec \node, \secnode\neq\node\}}
\newcommand{\afternodes}{\{\secnode \, : \, \node \prec \secnode, \secnode\neq\node\}}

\newcommand{\incomingnodes}{\edge^{\mathrm{in}}}
\newcommand{\outgoingnodes}{\edge^{\mathrm{out}}}

\newcommand{\nodesa}{A}
\newcommand{\nodesb}{B}
\newcommand{\nodesc}{C}

\newcommand{\nodesone}{\nodesof{1}}
\newcommand{\nodestwo}{\nodesof{2}}
\newcommand{\nodesthree}{\nodesof{3}}

\newcommand{\secnodes}{\tilde{\nodes}}
\newcommand{\secnodesof}[1]{\tilde{\nodes}^{#1}}
\newcommand{\thirdnodes}{\bar{\nodes}}

\newcommand{\node}{v}
\newcommand{\nodein}{\node\in\nodes}
\newcommand{\secnode}{\tilde{\node}}
\newcommand{\thirdnode}{\bar{\node}}

\newcommand{\edges}{\mathcal{E}}
\newcommand{\edgesof}[1]{\edges^{#1}}
\newcommand{\secedges}{\tilde{\edges}}

\newcommand{\edge}{e}
\newcommand{\edgeof}[1]{\edge_{#1}}
\newcommand{\secedge}{\tilde{\edge}}
\newcommand{\thirdedge}{\hat{\edge}}
\newcommand{\edgein}{\edge\in\edges}

\newcommand{\parentsof}[1]{\mathrm{Pa}(#1)}
\newcommand{\nondescendantsof}[1]{\mathrm{NonDes}(#1)}

\newcommand{\neighborsof}[1]{\mathrm{N}(#1)}

\newcommand{\bnnodecore}{\hypercoreof{(\parentsof{\node},\{\node\})}}
\newcommand{\bnedges}{\{(\parentsof{\node},\{\node\}) \, : \, \nodein\}}

\newcommand{\hypercore}{T}
\newcommand{\hypercoreat}[1]{\hypercore\left[#1\right]}
\newcommand{\hypercorewith}{\hypercoreat{\shortcatvariables}}
\newcommand{\hypercorewithin}{\hypercoreat{\shortcatvariables}\in\facspace}
\newcommand{\hyperonecoordinates}{\shortcatindices \, : \, \hypercoreat{\indexedshortcatvariables} = 1}
\newcommand{\hyperzerocoordinates}{\shortcatindices \, : \, \hypercoreat{\indexedshortcatvariables} = 0}

\newcommand{\hypercoreof}[1]{\hypercore^{#1}}
\newcommand{\hypercoreofat}[2]{\hypercoreof{#1}\left[#2\right]}
\newcommand{\sechypercore}{\tilde{\hypercore}}
\newcommand{\sechypercoreof}[1]{\sechypercore^{#1}}
\newcommand{\sechypercoreofat}[2]{\sechypercore^{#1}\left[#1\right]}
\newcommand{\sechypercoreat}[1]{\sechypercore\left[#1\right]}

%% Factored System
\newcommand{\onehotmap}{e}
\newcommand{\onehotmapof}[1]{\onehotmap_{#1}}
\newcommand{\onehotmapofat}[2]{\onehotmap_{#1}\left[#2\right]}
\newcommand{\onehotmapto}[1]{\onehotmapof{#1}} % For encoding of sets, relations
\newcommand{\invonehotmapof}[1]{\onehotmap^{-1}(#1)}

\newcommand{\statevectorof}[1]{v_{#1}}
\newcommand{\statevectorofat}[2]{\statevectorof{#1}\left[#2\right]}

% Greedy
\newcommand{\extendedformulaset}{\formulaset\cup\{\formula\}}
\newcommand{\extendedcanparam}{\tilde{\canparam}\cup\{\weightof{\formula}\}}

\newcommand{\exfunction}{f}
\newcommand{\exfunctionof}[1]{\exfunction_{#1}}
\newcommand{\exfunctiontargetspace}{\bigotimes_{l\in[p]}\rr^{\catdimof{l}}}
\newcommand{\exfunctiontargetvariables}{Y_0,\ldots,Y_{p-1}}
\newcommand{\exfunctionimageelement}{y}
\newcommand{\exfunctionat}[1]{\exfunction(#1)}
\newcommand{\secexfunction}{g}
\newcommand{\secexfunctionat}[1]{\secexfunction(#1)}

\newcommand{\compositionof}[2]{{#1}\circ{#2}}
\newcommand{\compositionofat}[3]{(\compositionof{#1}{#2})(#3)}
%% Message Passing
\newcommand{\cluster}{C}
\newcommand{\clusterof}[1]{\cluster_{#1}}
\newcommand{\clusterenumerator}{i}
\newcommand{\secclusterenumerator}{j}
\newcommand{\thirdclusterenumerator}{\tilde{j}}

\newcommand{\enc}{\clusterof{\clusterenumerator}}
\newcommand{\secenc}{\clusterof{\secclusterenumerator}}
\newcommand{\thirdenc}{\clusterof{\thirdclusterenumerator}}

\newcommand{\clusterorder}{n}
\newcommand{\clusterenumeratorin}{\clusterenumerator\in[\clusterorder]}

\newcommand{\mesfromto}[2]{\delta_{#1 \rightarrow #2}}
\newcommand{\mesfromtoat}[3]{\mesfromto{#1}{#2}\left[#3\right]}
\newcommand{\upmes}[2]{\delta_{#1 \rightarrow #2}}
\newcommand{\downmes}[2]{\delta_{#2 \leftarrow #1}}

% Binary connective symbols
\newcommand{\impbincon}{\Rightarrow}
\newcommand{\eqbincon}{\Leftrightarrow}
\newcommand{\lpasbincon}{\triangleleft}

\newcommand{\notucon}{\lnot}
\newcommand{\iducon}{\mathrm{Id}}
\newcommand{\trueucon}{\mathrm{T}}

\newcommand{\indexinterpretation}{I}
\newcommand{\indexinterpretationof}[1]{\indexinterpretation_{#1}}
\newcommand{\indexinterpretationat}[1]{\indexinterpretation(#1)}
\newcommand{\indexinterpretationofat}[2]{\indexinterpretationof{#1}(#2)}

\newcommand{\invindexinterpretation}{I^{-1}}
\newcommand{\invindexinterpretationof}[1]{I_{#1}^{-1}}
\newcommand{\invindexinterpretationat}[1]{\invindexinterpretation(#1)}
\newcommand{\invindexinterpretationofat}[2]{\invindexinterpretationof{#1}(#2)}

%ILP
\newcommand{\objectivesymbol}{c}
\newcommand{\objofat}[2]{\objectivesymbol^{#1}\left[#2\right]}
\newcommand{\rhssymbol}{b}
\newcommand{\rhsofat}[2]{\rhssymbol^{#1}\left[#2\right]}

% Coordinate Calculus
\newcommand{\coordinatetrafowrtof}[2]{{#1}\left(#2\right)}
\newcommand{\coordinatetrafowrtofat}[3]{\coordinatetrafowrtof{#1}{#2}\left[#3\right]}
%% CONTRACTIONS
\newcommand{\contractionof}[2]{\left\langle #1\right\rangle \left[ #2 \right]}

\newcommand{\breakablecontractionof}[2]{\big\langle #1 \big\rangle \big[ #2 \big]}
%\newcommand{\contractionof}[2]{\contractionof{#1}{#2}}
\newcommand{\contraction}[1]{\contractionof{#1}{\varnothing}}
%\newcommand{\contraction}[1]{\contraction{#1}}
\newcommand{\normalizationofwrt}[3]{\left\langle #1\right\rangle \left[ #2 | #3 \right]}
\newcommand{\breakablenormalizationofwrt}[3]{\big\langle #1 \big\rangle \big[ #2 | #3 \big]}
%\newcommand{\normalizationofwrt}[3]{\normalizationofwrt{#1}{#2}{#3}}
\newcommand{\normalizationof}[2]{\normalizationofwrt{#1}{#2}{\varnothing}}
%\newcommand{\normalizationof}[2]{\normalizationofwrt{#1}{#2}{\varnothing}}

\newcommand{\nzcontractionof}[2]{\nonzerocirc\contractionof{#1}{#2}}

%% ENCODING SCHEMES: Coordinate, basis, selection
\newcommand{\cencodingof}[1]{\chi^{#1}}
\newcommand{\cencodingofat}[2]{\cencodingof{#1}\left[#2\right]}
\newcommand{\cencodingwith}{\cencodingofat{\exfunction}{\shortcatvariables}}

\newcommand{\bencodingof}[1]{\beta^{#1}}
\newcommand{\bencodingofat}[2]{\bencodingof{#1}\left[#2\right]}
\newcommand{\bencodingwith}{\bencodingofat{\exfunction}{\headvariableof{\exfunction},\shortcatvariables}}

\newcommand{\sencodingof}[1]{\sigma^{#1}}
\newcommand{\sencodingofat}[2]{\sencodingof{#1}\left[#2\right]}  
\newcommand{\sencodingwith}{\sencodingofat{\exfunction}{\shortcatvariables,\selvariable}}

\newcommand{\brepresentationof}[1]{\tau^{#1}}
\newcommand{\brepresentationofat}[2]{\brepresentationof{#1}\left[#2\right]}


% Further tensors

\newcommand{\actcore}{\alpha} % activation of a formula, typical exp
\newcommand{\actcoreof}[1]{\actcore^{#1}}
\newcommand{\actcoreat}[1]{\actcore\left[#1\right]}
\newcommand{\actcoreofat}[2]{\actcore^{#1}[#2]}
\newcommand{\actcorewith}{\actcoreofat{\selindex,\canparamat{\indexedselvariable}}{\headvariableof{\selindex}}}

\newcommand{\dirdelta}{\delta}
\newcommand{\dirdeltaof}[1]{\dirdelta^{#1}}
\newcommand{\dirdeltaofat}[2]{\dirdeltaof{#1}\left[#2\right]}
\newcommand{\dirdeltawith}{\dirdeltaofat{[\catorder],\catdim}{\shortcatvariables}}

\newcommand{\onehotmap}{\epsilon}
\newcommand{\onehotmapof}[1]{\onehotmap_{#1}}
\newcommand{\onehotmapofat}[2]{\onehotmap_{#1}\left[#2\right]}
\newcommand{\onehotmapto}[1]{\onehotmapof{#1}} % For encoding of sets, relations
\newcommand{\onehotmapwith}{\onehotmapofat{\shortcatindices}{\shortcatvariables}}
\newcommand{\invonehotmapof}[1]{\onehotmap^{-1}(#1)}

\newcommand{\noisetensor}{\eta}
\newcommand{\noiseat}[1]{\noisetensor\left[#1\right]}
\newcommand{\noiseof}[1]{\noisetensor^{#1,\gendistribution,\datamap}}
\newcommand{\sstatnoise}{\noiseof{\sstat}}
\newcommand{\sstatnoisewith}{\noiseof{\sstat}[\selvariable]}

\newcommand{\canparam}{\theta}
\newcommand{\canparamof}[1]{\canparam_{#1}}
\newcommand{\canparamat}[1]{\canparam\left[#1\right]}
\newcommand{\canparamofat}[2]{\canparamof{#1}\left[#2\right]}
\newcommand{\canparamwith}{\canparamat{\selvariable}}
\newcommand{\indexedcanparam}{\canparamat{\indexedselvariable}}
\newcommand{\canparamwithin}{\canparamwith\in\parspace}

\newcommand{\kcore}{\kappa}
\newcommand{\kcoreof}[1]{\kcore^{#1}}
\newcommand{\kcoreat}[1]{\kcore\left[#1\right]}
\newcommand{\kcoreofat}[2]{\kcore^{#1}\left[#2\right]}
\newcommand{\kcorewith}{\kcoreofat{\edge}{\catvariableof{\edge}}}

\newcommand{\scalarcore}{\lambda} % Scalar core in CP decompositons
\newcommand{\scalarcoreof}[1]{\scalarcore[#1]}
\newcommand{\scalarcoreat}[1]{\scalarcore[#1]}
\newcommand{\scalarcoreofat}[2]{\scalarcore^{#1}[#2]}
\newcommand{\scalarcorewith}{\scalarcoreat{\decvariable}}

\newcommand{\meanparam}{\mu}
\newcommand{\meanparamof}[1]{\meanparam_{#1}}
\newcommand{\meanparamat}[1]{\meanparam\left[#1\right]}
\newcommand{\meanparamofat}[2]{\meanparamof{#1}\left[#2\right]}
\newcommand{\indexedmeanparam}{\meanparamat{\indexedselvariable}}
\newcommand{\meanparamwith}{\meanparamat{\selvariable}}

\newcommand{\basemeasure}{\nu}
\newcommand{\basemeasureof}[1]{\basemeasure^{#1}}
\newcommand{\basemeasureofat}[2]{\basemeasure^{#1}\left[#2\right]}
\newcommand{\basemeasureat}[1]{\basemeasure\left[#1\right]}
\newcommand{\basemeasurewith}{\basemeasureat{\shortcatvariables}}

\newcommand{\tnet}{\tau}
\newcommand{\tnetof}[1]{\tnet^{#1}}
\newcommand{\tnetofat}[2]{\tnetof{#1}\left[#2\right]}
\newcommand{\extnet}{\tnetof{\graph}}
\newcommand{\secextnet}{\tnetof{\tilde{\graph}}}
\newcommand{\extnetat}[1]{\extnet\left[#1\right]}
\newcommand{\extnetwith}{\tnetofat{\graph}{\shortcatvariables}}

\newcommand{\legcore}{\rho} % Leg core in CP decompositions
\newcommand{\legcoreof}[1]{\legcore^{#1}}
\newcommand{\legcoreofat}[2]{\legcoreof{#1}\left[#2\right]}
\newcommand{\legcorewith}{\legcoreofat{\atomenumerator}{\catvariableof{\atomenumerator},\decvariable}}

\newcommand{\energytensor}{\phi}
\newcommand{\energytensorofat}[2]{\energytensor^{#1}[#2]}
\newcommand{\energytensorof}[1]{\energytensor^{#1}}
\newcommand{\energytensorat}[1]{\energytensor\left[#1\right]}
\newcommand{\expenergy}{\energytensorofat{(\sstat,\canparam,\basemeasure)}{\shortcatvariables}}
\newcommand{\expenergyat}[1]{\energytensorofat{(\sstat,\canparam,\basemeasure)}{#1}}
\newcommand{\energytensorwith}{\energytensorat{\shortcatvariables}}

%% Sets of tensors
\newcommand{\expfamilyof}[1]{\Gamma^{#1}}
\newcommand{\expfamily}{\genexpfamily}
\newcommand{\genexpfamily}{\expfamilyof{\sstat,\basemeasure}}
\newcommand{\expfamilywith}{\expfamilyof{\sstat,\basemeasure}}

\newcommand{\realizabledistsof}[1]{\Lambda^{#1}}
\newcommand{\maxrealizabledistsof}[1]{\realizabledistsof{#1,\maxgraph}}
\newcommand{\elrealizabledistsof}[1]{\realizabledistsof{#1,\elformat}}
\newcommand{\realizabledistswith}{\realizabledistsof{\sstat,\graph}}
\newcommand{\hlnsetof}[1]{\realizabledistsof{#1,\elformat}}






%% MAIN VARIABLES
\newcommand{\indvariable}{O} 
\newcommand{\inddim}{r}
\newcommand{\indindex}{o} % was s
\newcommand{\indenumerator}{l}
\newcommand{\indorder}{n} % number of variables 

\newcommand{\selvariable}{L} 
\newcommand{\seldim}{p}
\newcommand{\selindex}{l}
\newcommand{\selenumerator}{s}
\newcommand{\selorder}{n}

\newcommand{\catvariable}{X} 
\newcommand{\catdim}{m}
\newcommand{\catindex}{x} % was i
\newcommand{\catenumerator}{\atomenumerator}
\newcommand{\catorder}{\atomorder}

\newcommand{\headvariable}{Y} % head of a basis encoding
\newcommand{\headdim}{n}
\newcommand{\headindex}{y}

\newcommand{\datvariable}{J} % Can be understood as selvariable selecting a datapoint, catvariable as a random datapoint, indvariable as a abstract object representing the sample, also used at indexvariable!
\newcommand{\datdim}{m}
\newcommand{\datindex}{j}

%% Syntactic Sugar
\newcommand{\indvariableof}[1]{\indvariable_{#1}}
\newcommand{\selvariableof}[1]{\selvariable_{#1}}
\newcommand{\catvariableof}[1]{\catvariable_{#1}}
\newcommand{\headvariableof}[1]{\headvariable_{#1}}

\newcommand{\indvariablelist}{\indvariableof{0},\ldots,\indvariableof{\individualorder-1}}
\newcommand{\catvariablelist}{\catvariableof{0},\ldots,\catvariableof{\atomorder-1}}
\newcommand{\selvariablelist}{\selvariableof{0},\ldots,\selvariableof{\selorder-1}}

\newcommand{\shortindvariablelist}{\indvariableof{[\individualorder]}}
\newcommand{\shortcatvariablelist}{\catvariableof{[\atomorder]}}
\newcommand{\shortselvariablelist}{\selvariableof{[\selorder]}}

\newcommand{\selindices}{\selindexof{0},\ldots,\selindexof{\selorder-1}}

\newcommand{\shortindindices}{\indindexof{[\indorder]}}
\newcommand{\shortcatindices}{\catindexof{[\catorder]}}
\newcommand{\shortselindices}{\selindexof{[\selorder]}}

\newcommand{\inddimof}[1]{\inddim_{#1}}
\newcommand{\seldimof}[1]{\seldim_{#1}}
\newcommand{\catdimof}[1]{\catdim_{#1}}
\newcommand{\headdimof}[1]{\headdim_{#1}}

\newcommand{\indindexof}[1]{\indindex_{#1}}
\newcommand{\selindexof}[1]{\selindex_{#1}}
\newcommand{\catindexof}[1]{\catindex_{#1}} 
\newcommand{\headindexof}[1]{\headindex_{#1}}

\newcommand{\indindexin}{\indindex\in[\inddim]}
\newcommand{\selindexin}{\selindex\in[\seldim]}
\newcommand{\catindexin}{\catindex\in[\catdim]}
\newcommand{\datindexin}{\datindex\in[\datdim]}

\newcommand{\indindexofin}[1]{\indindexof{#1}\in[\inddimof{#1}]}
\newcommand{\catindexofin}[1]{\catindexof{#1}\in[\catdimof{#1}]}
\newcommand{\selindexofin}[1]{\selindexof{#1}\in[\seldimof{#1}]}
\newcommand{\headindexofin}[1]{\headindexof{#1}\in[\headdimof{#1}]}

\newcommand{\indindexlist}{\indindexof{0},\ldots,\indindexof{\indorder-1}}
\newcommand{\catindexlist}{\catindexof{0},\ldots,\catindexof{\atomorder-1}}
\newcommand{\selindexlist}{\selindexof{0},\ldots,\selindexof{\selorder-1}}

\newcommand{\indenumeratorin}{\indenumerator\in[\indorder]}
\newcommand{\selenumeratorin}{\selenumerator\in[\selorder]}
\newcommand{\catenumeratorin}{\catenumerator\in[\catorder]}

\newcommand{\indexedindvariableof}[1]{\indvariableof{#1}=\indindexof{#1}}
\newcommand{\indexedcatvariableof}[1]{\catvariableof{#1}=\catindexof{#1}}
\newcommand{\indexedselvariableof}[1]{\selvariableof{#1}=\selindexof{#1}}
\newcommand{\indexedheadvariableof}[1]{\headvariableof{#1}=\headindexof{#1}}

\newcommand{\indexedindvariable}{\indexedindvariableof{}}
\newcommand{\indexedcatvariable}{\indexedcatvariableof{}}
\newcommand{\indexedselvariable}{\indexedselvariableof{}}

\newcommand{\catstatesof}[1]{[\catdimof{#1}]}

\newcommand{\catspaceof}[1]{\rr^{\catdimof{#1}}}

\newcommand{\indspace}{\bigotimes_{\indenumeratorin}\rr^{\inddim}}
\newcommand{\catspace}{\bigotimes_{\atomenumeratorin} \rr^{\catdimof{\atomenumerator}}}

\newcommand{\selstates}{\bigtimes_{\selenumeratorin}[\seldimof{\selenumerator}]}
\newcommand{\selvectorspace}{\rr^{\seldim}}
\newcommand{\selspace}{\bigotimes_{\selenumeratorin}\rr^{\seldimof{\selenumerator}}}
%%

\newcommand{\datanum}{\datdim}

\newcommand{\datain}{\datindex\in[\datanum]}
\newcommand{\data}{\{\datapointof{\datindex}\}_{\datindexin}}
\newcommand{\dataaverage}{\frac{1}{\datanum}\sum_{\datindexin}}

\newcommand{\catvariables}{\catvariablelist}
\newcommand{\shortcatvariables}{\shortcatvariablelist}
\newcommand{\indexedshortcatvariables}{\shortcatvariables=\shortcatindices}
\newcommand{\shortcatindicesin}{\shortcatindices\in\facstates}
\newcommand{\shortatomindicesin}{\shortcatindices\in\atomstates}
\newcommand{\datshortcatvariables}{\shortcatvariables=\shortcatindices^{\datindex}}

\newcommand{\headvariables}{\headvariableof{[\seldim]}}

\newcommand{\shortindvariables}{\shortindvariablelist}
\newcommand{\indexedshortindvariables}{\shortindvariables=\shortindindices}
\newcommand{\datshortindvariables}{\shortindvariables=\shortindindices^{\datindex}}

\newcommand{\selvariables}{\selvariableof{0},\ldots,\selvariableof{\selorder-1}}
\newcommand{\shortselvariables}{\selvariableof{[\selorder]}}
\newcommand{\indexedshortselvariables}{\shortselvariables=\shortselindices}
\newcommand{\secselenumerator}{\tilde{\selenumerator}}
\newcommand{\secselvariable}{\tilde{\selvariable}}
\newcommand{\secselindex}{\tilde{\selindex}}

\newcommand{\nodestatesof}[1]{\bigtimes_{\node\in#1}\catstatesof{\node}}
\newcommand{\atomstates}{\bigtimes_{\atomenumeratorin}[2]}


\newcommand{\symindstates}{\bigtimes_{\indenumeratorin}[\inddim]}

\newcommand{\facstates}{\bigtimes_{\atomenumeratorin}\catstatesof{\atomenumerator}}
\newcommand{\facdim}{\prod_{\atomenumeratorin}\catdimof{\atomenumerator}}
\newcommand{\secfacstates}{\bigtimes_{\secatomenumerator\in[\secatomorder]}\catstatesof{\secatomenumerator}}

\newcommand{\atomspace}{\bigotimes_{\atomenumeratorin}\rr^2}
\newcommand{\facspace}{\catspace}
\newcommand{\secfacspace}{\bigotimes_{\secatomenumerator\in[\seccatorder]} \rr^{\catdimof{\secatomenumerator}}}

\newcommand{\indexedcatvariables}{\indexedcatvariableof{0},\ldots,\indexedcatvariableof{\atomorder-1}} 
\newcommand{\tildeindexedcatvariables}{\catvariableof{0}=\tilde{\catindex}_0,\ldots,\catvariableof{\atomorder-1}=\tilde{\catindex}_{\atomorder-1}} 

\newcommand{\seccatenumerator}{\tilde{\catenumerator}}
\newcommand{\seccatenumeratorin}{\seccatenumerator\in[\catorder]}

\newcommand{\seccatvariable}{Y} % used as differentiation variable
\newcommand{\seccatindex}{y}
\newcommand{\seccatorder}{p} % Has to coincide with seldim for basis encoding def

\newcommand{\seccatvariableof}[1]{\seccatvariable_{#1}}
\newcommand{\indexedseccatvariableof}[1]{\seccatvariableof{#1}=\seccatindexof{#1}}
\newcommand{\seccatvariables}{\seccatvariableof{0},\ldots,\seccatvariableof{\seccatorder\shortminus1}}
\newcommand{\secshortcatvariables}{\seccatvariableof{[\seccatorder]}}
\newcommand{\indexedseccatvariables}{\indexedseccatvariableof{0}\ldots,\indexedseccatvariableof{\seccatorder-1}} 
\newcommand{\indexedsecshortcatvariables}{\indexedseccatvariableof{[\seccatorder]}}

\newcommand{\catvariablesinset}[1]{\catvariableof{#1}}%{\catvariableof{\node} \, : \, \node \in #1}
\newcommand{\seccatindexof}[1]{\seccatindex_{#1}}

\newcommand{\catindices}{\catindexlist}
\newcommand{\tildecatindexof}[1]{\tilde{\catindex}_{#1}}
\newcommand{\tildecatindices}{\tildecatindexof{0},\ldots,\tildecatindexof{\atomorder-1}}
\newcommand{\seccatindices}{{\seccatindexof{0},\ldots,\seccatindexof{\secatomorder-1}}}
\newcommand{\tildeshortcatindices}{\tildecatindexof{[\catorder]}}

\newcommand{\catindicesof}[1]{{\catindexof{0}^{#1},\ldots,\catindexof{\atomorder-1}^{#1}}}

\newcommand{\catzeropositions}{\{\atomenumerator : \catindexof{\atomenumerator}=0\}}
\newcommand{\catonepositions}{\{\atomenumerator : \catindexof{\atomenumerator}=0\}}

%% Cores
\newcommand{\categoricalmap}{Z}
\newcommand{\categoricalmapat}[1]{\categoricalmap\left[#1\right]}
\newcommand{\categoricalmapof}[1]{\categoricalmap^{#1}}
\newcommand{\categoricalmapofat}[2]{\categoricalmap^{#1}\left[#2\right]}

\newcommand{\categoricalcore}{\bencodingof{\categoricalmap}}
\newcommand{\categoricalcoreof}[1]{\bencodingof{\categoricalmapof{#1}}}
\newcommand{\categoricalcoreofat}[2]{\bencodingof{\categoricalmapof{#1}}\left[#2\right]}

\newcommand{\datamap}{D}
\newcommand{\datamapat}[1]{\datamap(#1)}
\newcommand{\datamapof}[1]{\datamap_{#1}}
\newcommand{\datapointof}[1]{\datamapat{#1}}
\newcommand{\datapoint}{\datapointof{\datindex}}
\newcommand{\dataset}{\left((\catindicesof{\datindex})\,:\,\datindexin\right)}

\newcommand{\secdatamap}{\tilde{\datamap}}
\newcommand{\datacore}{\bencodingof{\datamap}}
\newcommand{\datacoreat}[1]{\datacore\left[#1\right]}
\newcommand{\datacoreof}[1]{\bencodingof{\datamap_{#1}}}
\newcommand{\datacoreofat}[2]{\bencodingof{\datamap_{#1}}[#2]}

\newcommand{\secdatacoreof}[1]{\bencodingof{\secdatamap_{#1}}}
\newcommand{\empdistribution}{\probtensor^{\datamap}}
\newcommand{\empdistributionat}[1]{\empdistribution\left[#1\right]}
\newcommand{\empdistributionwith}{\empdistributionat{\shortcatvariables}}

\newcommand{\varcore}[1]{U^{#1}} % For optimization of tensor network
\newcommand{\varspace}[1]{\rr^{p_{#1}}}
\newcommand{\varcollection}{\big\{\varcore{\atomenumerator}\, :\, \atomenumeratorin \big\}}



\newcommand{\conactcore}{\kcore}
\newcommand{\conactcoreof}[1]{\conactcore^{#1}}


% DecompositionIndex 
\newcommand{\decvariable}{I}
\newcommand{\decvariableof}[1]{\decvariable_{#1}}
\newcommand{\decindex}{i} % Needs to be different to datindex!
\newcommand{\decindexof}[1]{\decindex_{#1}}
\newcommand{\indexeddecvariableof}[1]{\decvariableof{#1}=\decindexof{#1}}
\newcommand{\decdim}{n}
\newcommand{\decdimof}[1]{\decdim_{#1}}
\newcommand{\decindexin}{\decindex\in[\decdim]}
\newcommand{\indexeddecvariable}{\decvariable=\decindex}
\newcommand{\inddecvar}{\indexeddecvariable}

\newcommand{\indexeddatvariable}{\datvariable=\datindex}

% Used in poly sparsity
\newcommand{\indexvariable}{\datvariable} % for datacores used
\newcommand{\indexset}{J}
\newcommand{\indexsetof}[1]{\indexset^{#1}}

\newcommand{\slackvariable}{z}
\newcommand{\slackindex}{z}
\newcommand{\slackindexof}[1]{\slackindex_{#1}}

\newcommand{\rankofat}[2]{\mathrm{rank}^{#1}\left(#2\right)}
\newcommand{\cprankof}[1]{\mathrm{rank}\left(#1\right)}
\newcommand{\bincprankof}[1]{\mathrm{rank}^{\mathrm{bin}}\left(#1\right)}
\newcommand{\slicesparsityof}[1]{\slicerankwrtof{\catorder}{#1}} % former {\tilde{\ell} \left(#1\right)}


\newcommand{\dircprankof}[1]{\mathrm{rank}^{\mathrm{dir}}\left(#1\right)}
\newcommand{\bascprankof}[1]{\mathrm{rank}^{\mathrm{bas}}\left(#1\right)}
\newcommand{\baspluscprankof}[1]{\mathrm{rank}^{\mathrm{bas+}}\left(#1\right)}
\newcommand{\quacprankof}[1]{\mathrm{rank}^{\mathrm{qua}}\left(#1\right)}

\newcommand{\sliceset}{\mathcal{M}}
\newcommand{\slicescalar}{\lambda}
\newcommand{\slicescalarof}[1]{\slicescalar^{#1}}
\newcommand{\slicetupleof}[1]{(\slicescalar^{#1}, \variablesetof{#1}, \catindexof{\variablesetof{#1}}^{#1})}
\newcommand{\enumeratedslices}{\{\slicetupleof{\decindex} \, : \, \decindexin\}}

\newcommand{\sliceorder}{r}
\newcommand{\slicerankwrtof}[2]{\mathrm{rank}^{#1}\left(#2\right)}

\usetikzlibrary {arrows.meta} 
\usetikzlibrary{shapes,positioning}
\usetikzlibrary{decorations.markings}
\usetikzlibrary{calc}

\tikzset{
    midarrow/.style={
        postaction={decorate},
        decoration={markings, mark=at position 0.5 with {\arrow{>}}}
    },
    midbackarrow/.style={
        postaction={decorate},
        decoration={markings, mark=at position 0.5 with {\arrow{<}}}
    },
     ->-/.style={midarrow},
     -<-/.style={midbackarrow}
}

\newcommand{\shortminus}{\scalebox{0.4}[1.0]{$-$}}

\newcommand{\drawvariabledot}[2]{
	\draw[fill] (#1,#2) circle (0.15cm);
}

% Draws indices and below the indices the core
\newcommand{\drawatomindices}[2]{
	\begin{scope}[shift={(#1,#2)}]
		\draw[<-] (0,1)--(0,-1) node[midway,left] {\tiny $\catvariableof{0}$}; 
		\draw[<-] (1.5,1)--(1.5,-1) node[midway,left] {\tiny $\catvariableof{1}$}; 
		\node[anchor=center] (text) at (3,0) {$\cdots$};
		\draw[<-] (4,1)--(4,-1) node[midway,right] {\tiny $\catvariableof{\atomorder\shortminus1}$}; 
	\end{scope}
}
\newcommand{\drawundiratomindices}[2]{
	\begin{scope}[shift={(#1,#2)}]
		\draw[] (0,1)--(0,-1) node[midway,left] {\tiny $\catvariableof{0}$}; 
		\draw[] (1.5,1)--(1.5,-1) node[midway,left] {\tiny $\catvariableof{1}$}; 
		\node[anchor=center] (text) at (3,0) {$\cdots$};
		\draw[] (4,1)--(4,-1) node[midway,right] {\tiny $\catvariableof{\atomorder\shortminus1}$}; 
	\end{scope}
}

\newcommand{\drawatomcore}[3]{
	\begin{scope}[shift={(#1,#2)}]
		\draw (-1,-1) rectangle (5,-3);
		\node[anchor=center] (text) at (2,-2) {#3};
	\end{scope}
}
%! Author = alexgoessmann
%! Date = 26.05.25

% Tandem in representation.suffixes, where the macros are strings

% Cores
\newcommand{\comCoreSuf}{\stringof{\_cC}}
\newcommand{\actCoreSuf}{\stringof{\_aC}}

\newcommand{\atoCoreSuf}{\stringof{\_atoC}}
\newcommand{\vselCoreSuf}{\stringof{\_vselC}}

% Colors
\newcommand{\disVarSuf}{\stringof{\_dV}}
\newcommand{\comVarSuf}{\stringof{\_cV}}
\newcommand{\selVarSuf}{\stringof{\_sV}}
\newcommand{\terVarSuf}{\stringof{\_tV}}

% Refiners
\newcommand{\selCoreIn}{\stringof{\_s}}

\newcommand{\eviCoreIn}{\stringof{\_e}}

\newcommand{\heaIn}{\stringof{\_h}}
\newcommand{\funIn}{\stringof{\_f}}
\newcommand{\posIn}{\stringof{\_p}}

\newcommand{\datIn}{\stringof{\_d}}





\pretolerance=500
\tolerance=100
\emergencystretch=10pt

% Bibliography
\DeclareUnicodeCharacter{FB01}{fi}
\usepackage[round]{natbib}


\newcommand{\red}[1]{\textcolor{red}{#1}}

\begin{document}
    \title{The Tensor Network Approach to Efficient and Explainable AI}
    \author{Alex Goessmann, DATEV eG}

    \maketitle
    \date{\today}

    \begin{abstract}
        \chapter{Abstract}

Recent models in artificial intelligence, despite performance breakthroughs in large language models, suffer from limited efficiency and explainability, which prevents them from unlocking their full application potential for economic and trustworthy use.
To train and infer large black-box models, an evolving infrastructure of hardware and software frameworks capable of large-scale parallelizable linear algebra workload has been created.
We in this work develop an approach towards an alternative usage of this infrastructure as processing efficient and explainable models instead of black-box models.
To this end we leverage the mathematical structure of tensor networks, which has been eminent in the logical and probabilistic tradition of artificial intelligence.

While tensors appear naturally in artificial intelligence as factored representations of systems, their decompositions into tensor networks is necessary to avoid the curse of dimensionality. %improve the efficiency and explainability of several approaches.
Since the curse of dimensionality prevents feasible generic representations, logical and probabilistic reasoning approaches trade off efficiency and generality.
While logical approaches focus on models with sparse description in a logical syntax, probabilistic approaches exploit independencies to motivate sparse graphical models.
This work presents a unified treatment of these sparsity mechanisms in the tensor network formalism and formulates feasible reasoning algorithms involving tensor network contractions.

In the first part of this work, we review the classical logical and probabilistic tradition of artificial intelligence in the tensor network formalism.
Exploiting the common framework of tensor networks, the second part describes the integration of these approaches into a neuro-symbolic framework, which we call \HybridLogicNetworks{}.
In the third part we investigate in more detail schemes to exploit tensor network contractions for calculus.

The concepts of this work are implemented in the open-source \python library \tnreason.
This library is based on a modular design and specifies reasoning workload by tensor network operations, which is delegated towards various software frameworks of artificial intelligence.
In this way, \tnreason enables the usage of modern artificial intelligence infrastructure for efficient and explainable reasoning.
    \end{abstract}

    \tableofcontents

    \chapter{Introduction}\label{cha:introduction}

% Explaining the title
Artificial intelligence is a long-standing dream, which has in recent years received enourmous attention, driven by breakthroughs in large language models.
Among the key priorities towards an economic and trustworthy usage remain the creation of efficient and the explainable models.

% Explainability
Instead of post-hoc explainability of a models inference given specific data, our aim in this work is the intrinsic human understandability of a model.
We are motivated by the theory of logic, which formalization of human thoughts serves as an interface between mechanized reasoning on a machine and human understandability.
Having established this advanced form of explainability enables novel forms of human interactions with a model based on verbalizations, manipulations and guarantees on the models inference output.

% Efficiency
The desire of an efficient model originates more from an economic perspective on the realizability of a model and its power consumption.
Tensors appear naturally as representations of a system with multiple variables, both in logical and probabilistic approaches towards artificial intelligence. % avoid factored at this point!
However, already for moderate numbers of variables, the curse of dimensionality prevents a typical machines memory to store a generic representation.
The careful design of representation formats is therefore a necessary task to avoid the exponential increase of storage demands and balance the expressivity and the efficiency of representation formats.

% Tensor Networks
We in this work exploit the formalism of tensor networks in the creation of efficient representation schemes.
The chosen tensor network formats are motivated from explainable learning architectures and provide a synergy between the aims of efficiency and explainability.
Tensor networks appear as the natural numerical structures in probabilistic graphical models and logical knowledge bases.
After presenting the probabilistic and logical approaches based on the tensor network formalism we develop novel applications schemes towards neuro-symbolic artificial intelligence.

\sect{Background}

Before presenting an overview over the contents, we further motivate this work based on the broach approaches towards artificial intelligence and more recent developments.

\subsect{Classical Approached towards AI}

We start with ontological commitments in the description of a system and follow the book \cite{russell_artificial_2021} distinguishing atomic, factored and structured representations.
While in atomic representation, the states of a systems are enumerated and represented in a single variable, factored representations describe a systems state based on a collection of variables.
In the tensor formalism, each state of a system corresponds with a coordinate of a representing tensor.
The order of the tensor coincides therefore with the number of variables in a system.
In an atomic representation, where there is a single coordinate, each state corresponds with a coordinate of the representing vector being a tensor of order one.
Having a factored representation with two variables requires order two tensors or matrices, where a coordinate is specified by a row and a column index.
Given larger numbers of coordinates now extends this representation picture to tensors of larger orders, which have more abstract axes besides rows and columns.
The generalization of the atomic representation to a factored system thus corresponds with the generalization of vectors towards matrices and tensors of larger orders.
Along this line, we can always transform a factored representation of a system to an atomic one, just by enumerating the states of the factored system and interpreting them by a single variable.
This amounts to the flattening of a representing tensor to a vector.
However, by doing so, we would loose much of the structure of the representation, which we would like to exploit in reasoning processes.

% Structured Representations
A more generic representation of systems are structured representation.
Structured representations involve objects of differing numbers and relations between them.
As a consequence the numbers of variables can differ depending on the state of a system.
This poses a challenge to the tensor representation, since a fixed number of variables is required to motivate a tensor space of representations.
There are approaches to circumvent these difficulty by the development of template models such as Markov Logic Networks \cite{richardson_markov_2006}, which are instantiated on systems with differing number of objects.
We will discuss those in \charef{cha:folModels}.

% Continous vs discrete
In this work we treat discrete systems, where the number of states is finite.
One can understand them as a discretization of continuous variables and many results will generalize by the converse limit to the situation of continuous variables.

% Epistemologic
Besides ontological commitments in the choice of a representation scheme, modelling a system also requires epistemologic commitments, by defining what properties are to be reasoned about.
In logical approaches the properties of states are boolean values representing whether a state is consistent with known constraints.
Probabilistic approaches assign to the coordinates of the tensors numbers in $[0,1]$ encoding the probability of a state.
Compared with logical approached to reasoning, probabilistic approaches thus bear a more expressive modelling.

\subsect{Logic and Explainability in AI}

\textbf{Inductive Logic Programming:}
\begin{itemize}
    \item ILP is a classical task \cite{muggleton_inductive_1994}
    \item Amie \cite{galarraga_amie_2013} is a method of learning Horn clauses using a refinement operator.
    \item Class Expression Learning \cite{lehmann_class_2011} is a more recent approach to assist in the design of reasoning capabilities in Knowledge Graphs.
        However, problems arise from the expressivity of description logics and the efficient choice of formulas from exponentially large hypothesis sets.
    \item CEL has therefore recently received further popularity in combination with reinforcement learning \cite{demir_drill-_2021} and neural networks \cite{kouagou_neural_2022, pesquita_neural_2023}, which are methods searching efficienctly in exponentially large spaces of formulas.
\end{itemize}

\textbf{Statistical Relational AI:}
\begin{itemize}
    \item Classical combination of logical and probabilistic approaches to reasoning \cite{getoor_introduction_2019}
\end{itemize}

\textbf{Knowledge Graphs}
\begin{itemize}
    \item The advent of large Knowledge Graphs enables explainable reasoning methods on structured data. \cite{antoniou_semantic_2012,hogan_knowledge_2021}
    \item Knowledge Graphs are stored in a sparse format, i.e. only true atoms instead of all + truth label.
\end{itemize}


\subsect{Tensor Networks in AI}

\textbf{Tensor Network formats}
\begin{itemize}
    \item TT Format \cite{holtz_manifolds_2012,hackbusch_new_2009}
    \item HT Format \cite{hackbusch_tensor_2012}
    \item CP Format
\end{itemize}

\textbf{Tensor Networks as Regressors}
\begin{itemize}
    \item Dynamical Systems learning \cite{gels_multidimensional_2019, goesmann_tensor_2020}
    \item Supervised learning \cite{stoudenmire_supervised_2016}
\end{itemize}

\textbf{Tensor Representation of Logics}
\begin{itemize}
    \item Tensor Networks have been applied in the automatization of logic reasoning \cite{li_linear_2017, sato_linear_2017} apply Matrix multiplication in reasoning.
    \item \cite{nickel_review_2016} review over relational machine learning and latent features via matrix embeddings.
\end{itemize}

\textbf{Tensor Representation of Knowledge Graphs}
\begin{itemize}
    \item Effective representation of queries
    \item Usage of tensor networks in embeddings \cite{yang_embedding_2015} and using complex extensions \cite{trouillon_complex_2017, trouillon_knowledge_2017}
\end{itemize}

\textbf{Tensor Representation of Graphical Models}
\begin{itemize}
    \item Duality of Graphical Models and Tensor Networks: \cite{robeva_duality_2019}
    \item Expressivity studies \cite{glasser_expressive_2019}
\end{itemize}

\subsect{Infrastructure of AI}

The formalism of tensors and their network decompositions and contractions bears the potential of parallel computations exploited in the AI-dedicated soft and hardware.
\begin{itemize}
    \item Hardware: TPUs beyond GPUs
    \item Software: Tensors as basic data structure in TensorFlow, pyTorch etc., storing neural activations and model weights.
\end{itemize}



\sect{Structure of the work}

The chapters are structured into three parts, and two focuses, as sketched by:
\newcommand{\horDistChapter}{6}
\newcommand{\verDistChapter}{2.25}
\newcommand{\parBlockDistance}{0.25}
\newcommand{\drawchapter}[4]{
    \coordinate (logRepStart) at (#1, #2);
    \draw (logRepStart) rectangle ($(logRepStart) + (blockDiagonal)$);
    \node[anchor=center] (text) at ($(logRepStart) + (toTop) +0.5*(blockDiagonal)$) {\small #3};
    \node[anchor=center] (text) at ($(logRepStart) + (toBottom) +0.5*(blockDiagonal)$) {\small #4};
}

\begin{tikzpicture}[scale=0.9]
    \coordinate (blockDiagonal) at (5,1.75);
    \coordinate (toTop) at (0,0.35);
    \coordinate (toBottom) at (0,-0.35);

    % Representation
    \node[anchor=center] (text) at (-0.5, 2*\verDistChapter+0.35) {\focusonespec};
    \draw[dashed]  (-0.25-1*\horDistChapter, 2*\verDistChapter+1) -- (1*\horDistChapter-0.75, 2*\verDistChapter+1) -- (1*\horDistChapter-0.75, -4*\verDistChapter-1)
    -- (-0.25-1*\horDistChapter, -4*\verDistChapter-1) -- (-0.25-1*\horDistChapter, 2*\verDistChapter+1);

    % Reasoning
    \node[anchor=center] (text) at (-0.5+1.5*\horDistChapter, 2*\verDistChapter+0.35) {\focustwospec};
    \draw[dashed]  (-0.25+1*\horDistChapter, 2*\verDistChapter+1) -- (2*\horDistChapter-0.75, 2*\verDistChapter+1) -- (2*\horDistChapter-0.75, -4*\verDistChapter-1)
    -- (-0.25+1*\horDistChapter, -4*\verDistChapter-1) -- (-0.25+1*\horDistChapter, 2*\verDistChapter+1);

    % Part I
    \node[anchor=center] (text) at (-0.5-0.5*\horDistChapter, 1*\verDistChapter-0.25+0.35) {\parref{par:one}:};
    \node[anchor=center] (text) at (-0.5-0.5*\horDistChapter, 1*\verDistChapter-0.25-0.35) {\partonetext};
    \draw (-0.5-1*\horDistChapter,-0.25) -- (2*\horDistChapter-0.5, -0.25) -- (2*\horDistChapter-0.5, 2*\verDistChapter-0.25) --
    (-0.5-1*\horDistChapter, 2*\verDistChapter-0.25) -- (-0.5-1*\horDistChapter,-0.25);
    \foreach \x/\y/\key/\name in {
        0/1/cha:probRepresentation/\chatextprobRepresentation,
        0/0/cha:logicalRepresentation/\chatextlogicalRepresentation,
        1/1/cha:probReasoning/\chatextprobReasoning,
        1/0/cha:logicalReasoning/\chatextlogicalReasoning
    } {
        \drawchapter{\x * \horDistChapter}{\y *\verDistChapter}{\charef{\key}}{\name}
    }

    % Part II
    \node[anchor=center] (text) at (-0.5-0.5*\horDistChapter, -1.5*\verDistChapter-0.5+0.35) {\parref{par:two}:};
    \node[anchor=center] (text) at (-0.5-0.5*\horDistChapter, -1.5*\verDistChapter-0.5-0.35) {\parttwotext};
    \draw (-0.5-1*\horDistChapter,-0.5) -- (2*\horDistChapter-0.5, -0.5) -- (2*\horDistChapter-0.5, -2*\verDistChapter-0.5) --
    (-0.5-1*\horDistChapter, -2*\verDistChapter-0.5) -- (-0.5-1*\horDistChapter,-0.5);
    \foreach \x/\y/\key/\name in {
        -1/-1/cha:formulaSelection/\chatextformulaSelection,
        0/-1/cha:networkRepresentation/\chatextnetworkRepresentation,
        0/-2/cha:folModels/\chatextfolModels,
        1/-1/cha:networkReasoning/\chatextnetworkReasoning,
        1/-2/cha:concentration/\chatextconcentration
    } {
        \drawchapter{\x * \horDistChapter}{\y *\verDistChapter - \parBlockDistance}{\charef{\key}}{\name}
    }

    % Part III
    \node[anchor=center] (text) at (-0.5-0.5*\horDistChapter, -3.5*\verDistChapter-0.5+0.35) {\parref{par:three}:};
    \node[anchor=center] (text) at (-0.5-0.5*\horDistChapter, -3.5*\verDistChapter-0.5-0.35) {\partthreetext};
    \draw (-0.5 - 1*\horDistChapter, -2*\verDistChapter-0.75) -- (2*\horDistChapter-0.5, -2*\verDistChapter-0.75) -- (2*\horDistChapter-0.5, -4*\verDistChapter-0.75) --
    (-0.5-1*\horDistChapter, -4*\verDistChapter-0.75) -- (-0.5-1*\horDistChapter,-2*\verDistChapter-0.75);
    \foreach \x/\y/\key/\name in {
        -1/-3/cha:coordinateCalculus/\chatextcoordinateCalculus,
        0/-3/cha:basisCalculus/\chatextbasisCalculus,
        0/-4/cha:sparseRepresentation/\chatextsparseCalculus,
        1/-3/cha:approximation/\chatextapproximation,
        1/-4/cha:messagePassing/\chatextmessagePassing
    } {
        \drawchapter{\x * \horDistChapter}{\y *\verDistChapter - 2 * \parBlockDistance}{\charef{\key}}{\name}
    }

\end{tikzpicture}

\textbf{\parref{par:one}: \partonetext} \\
\ \\
The probabilistic and logical approaches towards artificial intelligence are reviewed in the tensor network formalism. \\

Tensors appear naturally in
\begin{itemize}
    \item Logics: Boolean tensors indicating models (propositional case) and interpretation tensors in first order logics
    \item Probability theory: Truth tables, which are tensors of probabilities for joint distibutions of categorical variables.
\end{itemize}

% Classical usage of tensor network decompositions
Tensor network decompositions as representation schemes appear in
\begin{itemize}
    \item Logics: Conjunctions of formulas are Hadamard products of the tensor representation of formulas (Coordinate Calculus/ Effective Calculus)
    \item Probability theory: Graphical models are tensor networks of the factors. Further sparsity schemes apply, when placing restrictions on the structure of each factor.
    \item Data bases: Relations encoded by lists as storage of nonvanishing coordinates of a relation encoding
\end{itemize}

% Classical usage of tensor network contractions
Tensor network contractions as reasoning schemes appear in
\begin{itemize}
    \item Logics: Model counts, used for satisfiablility decisions and entailment
    \item Probability theory: Marginal probability distributions, extended to conditional probability distributions through normations
\end{itemize}

\ \\
\textbf{\parref{par:two}: \parttwotext} \\
\ \\
Motivated by the classical approaches we apply the tensor network formalism towards learning and infering neuro-symbolic models. \\

\textbf{Neurosymbolic AI}
\begin{itemize}
    \item Required for more advanced AI \cite{hochreiter_toward_2022}
    \item Add the paradigm of neural computing to logical reasoning
    \item Potential benefits from Statistical Relational AI \cite{marra_statistical_2024}
\end{itemize}

%\subsect{Neuro-Symbolic AI}

\textbf{Tensor Approaches to Neuro-Symbolic AI}
\begin{itemize}
    \item TensorLog \cite{cohen_tensorlog_2020}
    \item \cite{badreddine_logic_2022} representation of logic using tensor networks and automated differentiation to optimize.
    \item \cite{badreddine_logic_2022} representation of logic using tensor networks and automated differentiation to optimize.
\end{itemize}

%% Decomposition of Neural Networks
In Deep Neural Networks, functions between the input layer and the output layer are decomposed into neurons.
Typical neurons are linear transforms with an activation function.

%% Sparsity by fixed architecture
Sparsity means restriction to functions, which are decomposable into a small number of neurons.
Approximations of generic functions (see the universal approximation theorems) would require large amounts of neurons. % CITE!
When restricting to functions based on a fixed architecture, we restrict to a certain set of functions called the inductive bias of the architecture.

\ \\
\textbf{\parref{par:three}: \partthreetext}\\
\ \\
The applied schemes of calculus using tensor network contractions are investigated in more detail.
\ \\
% Representation
\textbf{\focusonespec}\\
\\\
Here we motivate and investigate the efficient representation of tensors based on tensor network decompositions. \\
\ \\
% Reasoning
\textbf{\focustwospec}\\
\ \\
We develop schemes to efficiently perform inductive and deductive reasoning based on information stored in decomposed tensor.


%\subsect{\focusonespec}
%
%\subsect{\focustwospec}
%
%\subsect{\parref{par:one}: \partonetext}
%
%\subsect{\parref{par:two}: \parttwotext}
%
%\subsect{\parref{par:three}: \partthreetext}
    \section{Notation and Basic Concepts}\label{cha:TensorNetworks}

We here provide the fundamental definitions of tensors, which are essentiell for the content in Part~I and Part~II.
In Part~III we will further investigate the properties of tensors focusing on their contractions.

\subsection{Categorical Variables and Representations}

We will in this work investigate systems, which are described by a set of properties, each called a categorical variable. 
This is called an ontological commitment, since it defines what properties a system has.

\begin{definition}
	An atomic representation of a system is described by a categorical variables $\catvariable$ taking values $\catindex$ in a finite set 
		\[  [\catdim]\coloneqq \{0,\ldots, \catdim-1\} \]
	of cardinality $\catdim$.
\end{definition}

% Notation: Large and small literals
We will in this work always notate categorical variables by large literals and indices by small literals, possible with other letters such as $\catvariable,\selvariable,\indvariable,\datvariable$ and corresponding values $\catindex,\selindex,\indindex,\datindex$.

\begin{definition}
	A factored representation of a system is a set of categorical variables $\catvariableof{\atomenumerator}$, where $\atomenumeratorin$, taking values in $[\catdimof{\atomenumerator}]$.
\end{definition}

\subsection{Tensors}

% Gentle introduction sentences
Tensors are multiway arrays and a generalization of vectors and matrices to higher orders.
We will first provide a formal definition as real maps from index sets enumerating the coordinates of vectors, matrices and larger order tensors.

\begin{definition}[Tensor]\label{def:tensor}
	Let there be numbers $\catdimof{\atomenumerator}\in\nn$ for $\atomenumeratorin$ and categorical variables $\catvariableof{\atomenumerator}$ taking their values in $[\catdimof{\atomenumerator}]$.
	We call maps
	\begin{align*}
		\hypercoreat{\catvariables} : \bigtimes_{\atomenumeratorin} [\catdimof{\atomenumerator}] \rightarrow \rr
	\end{align*}
	tensor of order $\atomorder$ and leg dimensions $\catdimof{0},\ldots,\catdimof{\atomorder-1}$.
	Evaluations of these maps at indices $\catindices$ are denoted by
	\begin{align*}
		\hypercoreat{\indexedcatvariables} = \hypercoreat{\catvariables}(\catindices) \, .
	\end{align*}	
%	with coordinates denoted by $\hypercore_{\catindices}$ is called a tensor of order $\atomorder$ and legs with the dimensions $\catdimof{0},\ldots,\catdimof{\atomorder-1}$.
	Tensors $\hypercoreat{\catvariables}$ are elements of the space
	\begin{align*}
		\bigotimes_{\atomenumeratorin} \rr^{\catdimof{\atomenumerator}} \,  
	\end{align*}
	which is, with the operations of coordinatewise summation and scalar multiplication, a linear space called a tensor space.
\end{definition} 

% Non-canonical 
We here introduced tensors in a non-canonical way based on categorical variables assigned to its axis.
While coming as syntactic sugar at this point, this will allow us to define contractions without further specification of axes, based on comparisons of shared categorical variables.
Especially, this eases the implementation of tensor network contractions without the need to further specify a graph (see Appendix~\ref{cha:implementation}).

% Further abbreviations
We abbreviate lists $\catvariables$ of categorical variables by $\shortcatvariables$, that is denote $\hypercoreat{\catvariables}$ by $\hypercoreat{\shortcatvariables}$.
Occasionally, when the categorical variables of a tensor are clear from the context, we will omit the notation of the variables. %further abbreviate $\hypercoreat{\catvariables}$ by $\hypercore$.

\begin{example}[Trivial Tensor]\label{exa:trivialTensor}
	The trivial tensor is defined as the map 
		\[ \onesat{\shortcatvariables} : \facstates \rightarrow \{1\} \subset \rr \]
	with all coordinates being $1$, that is for all $\catindices\in\facstates$
		\[ \onesat{\indexedshortcatvariables} = 1 \, . \]
\end{example}


\subsection{One-hot encodings}

We are now ready to provide the link between tensors and states of systems with factored representations.
To this end, we define the one-hot encoding of a state, which is a bijection between the states and the basis elements of a tensor space.

\begin{definition}[One-hot encodings to Atomic Representations]
	Given an atomic system described by the categorical variable $\catvariable$, we define for each $\catindex\in[\catdim]$ the basis vector $\onehotmapofat{\catindex}{\catvariable}$ by
	\begin{align}
		\onehotmapofat{\catindex}{\catvariable=\tilde{\catindex}} = \begin{cases}
			1 & \text{if} \quad \catindex=\tilde{\catindex} \\
			0 & \text{else} \, .
		\end{cases} 
	\end{align}
	The one-hot encoding of states $\catindex\in[\catdim]$ of the atomic system described by the categorical variable $\catvariable$ is the map
		\[ \onehotmap: [\catdim] \rightarrow \rr^\catdim \]
	which maps $\catindex \in [\catdim]$ to the basis vectors $\onehotmapofat{\catindex}{\catvariable}$.
\end{definition}

% Coordinatewise representation
The basis vectors $\onehotmapofat{\catindex}{\catvariable}$ are tensors of order $1$ and leg dimension $\catdim$ of the structure
\begin{align}
	\onehotmapofat{\catindex}{\catvariable} = \begin{bmatrix}
	0 & \cdots & 0 & 1 &  0 & \cdots & 0
	\end{bmatrix} \, ,
\end{align}
where the $1$ is at the $\catindex$th coordinate of the vector.

% Atomic -> Factored system
We have so far described one-hot representations of the states of a single categorical variable, which would suffice to encode the state of an atomic system.
In a factored system on the other side, we are dealing with multiple categorical variables.

\begin{definition}[One-hot encodings to Factored Representations]
	Let there be a factored system defined by a tuple $(\catvariables)$ of variables taking values in $\facstates$.
	The one-hot encoding of its states is the tensor product of the one-hot encoding to each categorical variables, that is the map
		\[ \onehotmap : \facstates \rightarrow  \facspace \]
	defined by mapping $\catindices=\shortcatindices$ to
	\begin{align*}
		 \onehotmapofat{\shortcatindices}{\shortcatvariables}
		=: \bigotimes_{\atomenumeratorin} \onehotmapofat{\catindexof{\atomenumerator}}{\catvariableof{\atomenumerator}} \, . 
	\end{align*}
	We will call one-hot representations \emph{tensor representations} and depict them as
	\begin{center}
		\begin{tikzpicture}[scale=0.35,thick] % , baseline = -3.5pt


\draw (-12,1) rectangle (-3,3);
\node[anchor=center] (text) at (-7.5,2) {\corelabelsize $\bigotimes_{\atomenumeratorin} \onehotmapof{\catindexof{\atomenumerator}}$};
\draw (-11,1)--(-11,-1) node[midway,right] {\colorlabelsize $\catvariableof{0}$};
\draw (-9.5,1)--(-9.5,-1) node[midway,right] {\colorlabelsize $\catvariableof{1}$};
\node[anchor=center] (text) at (-6.75,0) {$\cdots$};
\draw (-4,1)--(-4,-1) node[midway,right] {\colorlabelsize $\catvariableof{\atomorder\shortminus1}$};


\node[anchor=center] (text) at (-1,2) {\corelabelsize ${=}$};

\draw (1,1) rectangle (3,3);
\node[anchor=center] (text) at (2,2) {\corelabelsize $\onehotmapof{\catindexof{0}}$};
\draw (2,1)--(2,-1) node[midway,right] {\colorlabelsize $\catvariableof{0}$};

\node[anchor=center] (text) at (4.5,2) {$\otimes$};

\draw (6,1) rectangle (8,3);
\node[anchor=center] (text) at (7,2) {\corelabelsize $\onehotmapof{\catindexof{1}}$};
\draw (7,1)--(7,-1) node[midway,right] {\colorlabelsize $\catvariableof{1}$};

\node[anchor=center] (text) at (9.5,2) {$\otimes$};

\node[anchor=center] (text) at (11,2) {$\cdots$};

\node[anchor=center] (text) at (12.5,2) {$\otimes$};


\draw (14,1) rectangle (16,3);
\node[anchor=center] (text) at (15,2) {\corelabelsize $\onehotmapof{\catindexof{\atomorder\shortminus1}}$};
\draw (15,1)--(15,-1) node[midway,right] {\colorlabelsize $\catvariableof{\atomorder\shortminus1}$};





\end{tikzpicture}
	\end{center}
\end{definition}


\begin{remark}[Flattening of Tensors]
	The use the tensor product to represent states of factored systems can be motivated by the reduction to atomic systems by enumeration of the states.
	We have this property reflected in the state encoding of factored systems, since the tensor space $\bigotimes_{\atomenumeratorin}\rr^{\catdimof{\atomenumerator}}$ is isomorphic to the vector spaces $\rr^{\prod_{\atomenumeratorin}\catdimof{\atomenumerator}}$.
	This operation is called flattening (or unfolding) of tensors with many axes to tensors of less axes.
\end{remark}



\subsection{Contractions}

Contractions are the central manipulation operation on sets of tensors.
To introduce them, we will develop a graphical illustration of sets of tensors, which we also call tensor networks.
In Part~III we will further investigate the utility of contractions in representing specific calculations, which demand different encoding schemes.


\subsubsection{Graphical Illustrations}

% Hypergraph as capturing the categorical variable assignment to tensors
Sets of tensor with categorical variables assigned to each legs implicitly carry a notion of a hypergraph.
This perspective is especially useful, when some categorical variables are assigned to axis of multiple tensors, as it will often be the case in the applications considered in this work.
Each variable can then be labeled by a node and each tensor as a hyperedge containing the nodes to its axis variables.
Let us first formally introduce hypergraphs, which are generalizations of graphs allowing edges to be arbitrary nonempty subsets of the nodes, whereas canonical graphs demand a cardinality of two.

\begin{definition}\label{def:hypergraphs}
	A hypergraph is a pair $\graph=(\nodes,\edges)$ of a set of nodes $\nodes$ and a set of edges $\edges$, where each hyperedge $\edge\in\edges$ is a subset of the nodes $\nodes$.
	A directed hypergraph is a pair $\graph=(\nodes,\edges)$, such that each hyperedge $\edge\in\edges$ is the tuple of two disjoint sets $\incomingnodes,\outgoingnodes\subset\nodes$, that is
		\[ \edge = (\incomingnodes,\outgoingnodes)  \, . \]
\end{definition}

% Diagrammatic representation in factor graphs
We will use the standard visualization by factor graphs as a diagrammatic illustration of sets of tensors, where tensors are represented by block nodes and each axis assigned with by a categorical variable $\catvariableof{\atomenumerator}$ represented by a node, see Figure~\ref{fig:tensors}a). 
%We further denote on Each axis of the tensor is represented by a node representing the variable $\catvariableof{\atomenumerator}$ and the tensor $\hypercore$ is associated with the hyperedge $\edge$ connecting all variables.
 %representing the choice of an element in the set $[\catdimof{\atomenumerator}]$
% Hyperedge view
Different simplifications of these factor graph depictions have been evolved in different research fields.
In the tradition of graphical models, which started with the work \cite{pearl_probabilistic_1988}, the categorical variables are highlighted and the tensor blocks just depicted by hyperedges.
To depict dependencies with causal interpretations, the edges are further decorated by directions in the depiction of Bayesian networks, see for example \cite{pearl_causality_2009}.

In the tensor network community on the other hand, a simplification scheme highlighting the tensors as blocks and omitting the depiction of categorical variables has been evolved.
The variables, or sometimes their index or dimension, are then directly assigned to the lines depicting the axes of the tensor blocks.

Both depiction schemes are simplifications of factor graphs, by highlighting the categorical variables in the depiction in Figure~\ref{fig:tensors}b) and the tensors in the depiction in Figure~\ref{fig:tensors}c).
We in this work will prefer the simplification of the tensor network community, depicted in Figure~\ref{fig:tensors}b).

% Duality
In another interpretation (see \cite{robeva_duality_2019}), both simplification schemes are itself interpret as hypergraphs, which are dual to each other.

\begin{figure}[h!]
	\begin{center}
		\begin{tikzpicture}[scale=0.35,thick] % , baseline = -3.5pt


\begin{scope}[shift={(-17,0)}]

\node[anchor=center] (text) at (-3,0) {$a)$};

\draw (-1,1) rectangle (10,-1);
\node[anchor=center] (text) at (4.5,0) {\small $\hypercoreof{\edge}$};

\draw (0,-1)--(0,-3) node[midway,left] {\tiny $\catvariableof{0}$}; 
\draw (3,-1)--(3,-3) node[midway,left] {\tiny $\catvariableof{1}$}; 
\node[anchor=center] (text) at (3,-4) {$\cdots$};
\draw (9,-1)--(9,-3) node[midway,right] {\tiny $\catvariableof{\atomorder\shortminus1}$}; 

\node [circle, draw, thick, fill=gray!50, minimum size = \nodeminsize] (P1) at (0,-4) {\tiny $\catvariableof{0}$};	
\node [circle, draw, thick, fill=gray!50, minimum size = \nodeminsize] (P2) at (3,-4) {\tiny $\catvariableof{1}$};
\node[anchor=center] (text) at (6,-4) {$\cdots$};

\node [circle, draw, thick, fill=gray!50, minimum size = \nodeminsize] (P3) at (9,-4) {};

\node[anchor=center] (text) at (9,-4) {\tiny $\catvariableof{\atomorder-1}$};


\end{scope}


\node[anchor=center] (text) at (-2,0) {$b)$};

\node [circle, draw, thick, fill=gray!50, minimum size = \nodeminsize] (P1) at (0,-3) {\tiny $\catvariableof{0}$};	
\node [circle, draw, thick, fill=gray!50, minimum size = \nodeminsize] (P2) at (3,-3) {\tiny $\catvariableof{1}$};

\node[anchor=center] (text) at (6,-3) {$\cdots$};

\node [circle, draw, thick, fill=gray!50, minimum size = \nodeminsize] (P3) at (9,-3) {};

\node[anchor=center] (text) at (9,-3) {\tiny $\catvariableof{\atomorder-1}$};


\draw (P1) to[bend right=-25] (4.5,0);
\draw (P2) to[bend right=-10] (4.5,0);
\draw (P3) to[bend right=25] (4.5,0);
\node[anchor=center] (text) at (4.5,0.5) {$\edge$};


\begin{scope}[shift={(16,2)}]

\node[anchor=center] (text) at (-3,-2) {$c)$};

\draw (-1,-1) rectangle (5,-3);
\node[anchor=center] (text) at (2,-2) {\small $\hypercoreof{\edge}$};
\draw (0,-3)--(0,-5) node[midway,left] {\tiny $\catvariableof{0}$}; 
\draw (1.5,-3)--(1.5,-5) node[midway,left] {\tiny $\catvariableof{1}$}; 
\node[anchor=center] (text) at (3,-4) {$\cdots$};
\draw (4,-3)--(4,-5) node[midway,right] {\tiny $\catvariableof{\atomorder\shortminus1}$}; 

\end{scope}

%\drawatomcore{3.5}{-8}{$\probtensor$}
%\drawatomindices{3.5}{-12}	
%\draw (5.5,-9)--(5.5,-7) node[midway,right] {\tiny $\catvariableof{\exformula}$};

\end{tikzpicture}
	\end{center}
	\caption{Depiction of Tensors 
	a) As a factor in a factor graph, depicted by a block, and connected to categorical variables assigned to nodes.
	b) Highlighting only the variable dependencies by a hyperedge connecting the variables $\catvariableof{\atomenumerator}$ to each axis $\atomenumeratorin$.
	c) Highlighting the tensor by a blockwise notation with axes denoted by open legs represented by the variables $\catvariableof{\atomenumerator}$.
	}\label{fig:tensors}
\end{figure}


% Diagramatic representation of vectors
To depict vector calculus and its generalizations, we will apply the graphical notation (mainly version b) introduced in Chapter~\ref{cha:TensorNetworks}. 
Along this line, we represent vectors and their generalization to tensors by blocks with legs representing its indices.
The basis vectors being one-hot encodings of states are in this scheme represented by
	\begin{center}
		\begin{tikzpicture}[scale=0.3,thick] % , baseline = -3.5pt

\draw (1,1) rectangle (3,3);
\node[anchor=center] (text) at (2,2) {\small $\onehotmapof{\catindex}$};
\draw (2,-1)--(2,1) node[midway,right] {\tiny $\catvariable$};

\end{tikzpicture}
	\end{center}
where $\tilde{\catindex}$ is an indexed represented by an open leg. 
Assigning $\catindex$ to this index will retrieve the $\catindex$th coordinate (with value $1$), whereas all other assignments will retrieve the coordinate values $0$. 


Drawing on the interpretation of tensors by hyeredges we can continue with the definition of tensor networks.

\begin{definition}\label{def:tensorNetwork}
	Let $\graph=(\nodes,\edges)$ be a hypergraph with nodes decorated by categorical variables $\catvariableof{\node}$ with dimensions
		\[ \catdimof{\node} \in \nn \]	
	and hyperedges $\edge\in\edges$ decorated by core tensors
		\[ \hypercoreofat{\edge}{\catvariableof{\edge}} \in \bigotimes_{\node\in\edge}\rr^{\catdimof{\node}} \, , \]
	where we denote by $\catvariableof{\edge}$ the set of categorical variables $\catvariableof{\node}$ with $\node\in\edge$.
	Then we call the set 
		\[ \tnetofat{\graph}{\catvariableof{\nodes}} = \{\hypercoreofat{\edge}{\catvariableof{\edge}}  \, : \, \edge\in\edges\} \]
	the Tensor Network of the decorated hypergraph $\graph$.
\end{definition}


\begin{figure}
	\begin{center}
		\begin{tikzpicture}[scale=0.35,thick] % , baseline = -3.5pt


\node[anchor=center] (text) at (-2,0) {$a)$};

\node [circle, draw, thick, fill=gray!50, minimum size = \nodeminsize] (P1) at (0,-3) {\tiny $\catvariableof{0}$};	
\node [circle, draw, thick, fill=gray!50, minimum size = \nodeminsize] (P2) at (3,-3) {\tiny $\catvariableof{1}$};
\node [circle, draw, thick, fill=gray!50, minimum size = \nodeminsize] (P3) at (6,-3) {\tiny $\catvariableof{2}$};

\node [circle, draw, thick, fill=gray!50, minimum size = \nodeminsize] (P4) at (9,-3) {\tiny $\catvariableof{3}$};;


\draw (P1) to[bend right=-20] (3,0);
\draw (P2) to[bend right=0] (3,0);
\draw (P3) to[bend right=20] (3,0);
\node[anchor=center] (text) at (3,0.5) {$\edge_0$};

\draw (P2) to[bend right=20] (4.5,-6);
\draw (P3) to[bend right=-20] (4.5,-6);

\node[anchor=center] (text) at (4.5,-6.5) {$\edge_1$};

\draw (P3) to[bend right=20] (7.5,-6);
\draw (P4) to[bend right=-20] (7.5,-6);

\node[anchor=center] (text) at (7.5,-6.5) {$\edge_2$};


\begin{scope}[shift={(25,0)}]

\node[anchor=center] (text) at (-2,0) {$b)$};

\draw (-1,-1) rectangle (5,-3);
\node[anchor=center] (text) at (2,-2) {\small $\hypercoreof{\edge_0}$};
\draw (0,-3)--(0,-5) node[midway,left] {\tiny $\catlegof{0}$}; 
\draw (2,-3)--(2,-5) node[midway,left] {\tiny $\catlegof{1}$}; 
%\draw (3,-3)--(3,-5) node[midway,left] {\tiny $\catlegof{1}$}; 
\draw (4,-3)--(4,-5) node[midway,left] {\tiny $\catlegof{2}$}; 


\draw (6,-1) rectangle (10,-3);
\node[anchor=center] (text) at (8,-2) {\small $\hypercoreof{\edge_2}$};
\draw (7,-3)--(7,-5) node[midway,right] {\tiny $\catlegof{2}$}; 
\draw (9,-3)--(9,-5) node[midway,right] {\tiny $\catlegof{3}$}; 


\draw (1,-7) rectangle (5,-9);
\node[anchor=center] (text) at (3,-8) {\small $\hypercoreof{\edge_1}$};
\draw (2,-5)--(2,-7); % node[midway,left] {\tiny $\catlegof{1}$}; 
\draw (4,-5) to[bend right=20]  (7,-6); % node[midway,left] {\tiny $\catlegof{2}$}; 
\draw (4,-7) to[bend right=-20]  (7,-6); 

\draw[fill] (2,-6) circle (0.25cm);
\draw (2,-6) to[bend right=20] (-1,-8); % node[midway, right]{\tiny $\catlegof{1}$};
\node[anchor=center] (text) at (-2,-8) {\tiny $\catlegof{1}$};

\draw[fill] (7,-6) circle (0.25cm);
\draw (7,-5) -- (7,-6);
\draw (7,-6)--(7,-8) node[midway,right] {\tiny $\catlegof{2}$}; 

\end{scope}


\end{tikzpicture}
	\end{center}
	\caption{
	Example of a tensor network.
	a) Hypergraph with edges $\edge_0=\{\catvariableof{0},\catvariableof{1},\catvariableof{2}\}$, $\edge_1=\{\catvariableof{1},\catvariableof{2}\}$ and $\edge_2=\{\catvariableof{2},\catvariableof{3}\}$ decorated by tensor cores.
	b) Dual tensor network, depicting a contraction with leaving all variables open.
	}\label{fig:network}
\end{figure}

%%
%Diagrammatic notation: Best to do version a) as used in the definition, highlighting that tensors have shared categorical variables with fixed dimensions.




\subsubsection{Tensor Product}

% Diagrams -> Contractions
Let us now exploit the developed graphical representations to define contractions of tensor networks.
The simplest contraction is the tensor product, which maps a pair of two tensors with distinct variables onto a third tensor and has an interpretation by coordinatewise products.
Such a contraction corresponds with a tensor network of two tensors with disjoint variables, depicted as:
\begin{center}
	\begin{tikzpicture}[scale=0.35,thick] % , baseline = -3.5pt


\begin{scope}[shift={(-15,0)}]



\draw (-1,1) rectangle (10,-1);
\node[anchor=center] (text) at (4.5,0) {\corelabelsize $\hypercoreof{\edge_0}$};

\draw (0,-1)--(0,-3) node[midway,left] {\colorlabelsize $\catvariableof{0}$};
\draw (3,-1)--(3,-3) node[midway,left] {\colorlabelsize $\catvariableof{1}$};
\node[anchor=center] (text) at (3,-4) {$\cdots$};
\draw (9,-1)--(9,-3) node[midway,right] {\colorlabelsize $\catvariableof{\atomorder\shortminus1}$};

\node [circle, draw, thick, fill=\nodegrayscale, minimum size = \nodeminsize] (P1) at (0,-4) {\colorlabelsize $\catvariableof{0}$};
\node [circle, draw, thick, fill=\nodegrayscale, minimum size = \nodeminsize] (P2) at (3,-4) {\colorlabelsize $\catvariableof{1}$};
\node[anchor=center] (text) at (6,-4) {$\cdots$};

\node [circle, draw, thick, fill=\nodegrayscale, minimum size = \nodeminsize] (P3) at (9,-4) {};

\node[anchor=center] (text) at (9,-4) {\colorlabelsize $\catvariableof{\atomorder-1}$};



\end{scope}




\draw (-1,1) rectangle (10,-1);
\node[anchor=center] (text) at (4.5,0) {\corelabelsize $\hypercoreof{\edge_1}$};

\draw (0,-1)--(0,-3) node[midway,left] {\colorlabelsize $\seccatvariableof{0}$};
\draw (3,-1)--(3,-3) node[midway,left] {\colorlabelsize $\seccatvariableof{1}$};
\node[anchor=center] (text) at (3,-4) {$\cdots$};
\draw (9,-1)--(9,-3) node[midway,right] {\colorlabelsize $\seccatvariableof{\seccatorder\shortminus1}$};

\node [circle, draw, thick, fill=\nodegrayscale, minimum size = \nodeminsize] (P1) at (0,-4) {\colorlabelsize $\seccatvariableof{0}$};
\node [circle, draw, thick, fill=\nodegrayscale, minimum size = \nodeminsize] (P2) at (3,-4) {\colorlabelsize $\seccatvariableof{1}$};
\node[anchor=center] (text) at (6,-4) {$\cdots$};

\node [circle, draw, thick, fill=\nodegrayscale, minimum size = \nodeminsize] (P3) at (9,-4) {};

\node[anchor=center] (text) at (9,-4) {\colorlabelsize $\seccatvariableof{\seccatorder-1}$};




\end{tikzpicture}
\end{center}

\begin{definition}[Tensor Product]\label{def:tensorProduct}
	Let there be two tensor
	\begin{align*}
		\hypercoreofat{\edge_0}{\shortcatvariables} : \facstates \rightarrow \rr \quad \text{and} \quad  \hypercoreofat{\edge_1}{\secshortcatvariables} : \secfacstates \rightarrow \rr \, 
	\end{align*}
	with different categorical variables assigned to its axes.
	Then there tensor product is the map
	\begin{align*}
		\contractionof{\hypercoreofat{\edge_0}{\shortcatvariables},\hypercoreofat{\edge_1}{\secshortcatvariables}}{\shortcatvariables,\secshortcatvariables} :  \left(\facstates\right) \times \left(\secfacstates\right) \rightarrow \rr
	\end{align*}
	defined for $\catindices\in\facstates$ and $\seccatindices\in\secfacstates$ as
	\begin{align*}
		& \contractionof{\hypercore,\sechypercore}{\indexedcatvariables,\indexedseccatvariables} \\
		&\quad\quad :=  \hypercoreofat{\edge_0}{\indexedcatvariables}\cdot \hypercoreofat{\edge_1}{\indexedseccatvariables} \, .
	\end{align*}
\end{definition}

% Other notations
Other popular standard notations of tensor products (see \cite{kolda_tensor_2009,hackbusch_tensor_2012,cichocki_tensor_2015}) 
	\[ \left(\hypercore \otimes \sechypercore\right) = \left(\hypercore \circ \sechypercore\right)  
	= \contractionof{\hypercoreofat{\edge_0}{\shortcatvariables},\hypercoreofat{\edge_1}{\secshortcatvariables}}{\shortcatvariables,\secshortcatvariables}  \, . \]
We will avoid these notations in this work in favor of a consistent notation capable of depicting generic tensor network contractions.

When the tensor $\hypercoreofat{\edge_1}{\secshortcatvariables}$ coincides with the trivial tensor $\onesat{\secshortcatvariables}$ (see Example~\ref{exa:trivialTensor}), we further make a notation convention to omit that tensor, that is
\begin{align*}
	\contractionof{\hypercoreofat{\edge_0}{\shortcatvariables},\onesat{\secshortcatvariables}}{\shortcatvariables,\secshortcatvariables}  
	= \contractionof{\hypercoreofat{\edge_0}{\shortcatvariables}}{\shortcatvariables,\secshortcatvariables} \, .
\end{align*}


\subsubsection{Generic Contractions}


Contractions of Tensor Networks $\extnet$ are operations to retrieve single tensors by summing products of tensors in a network over common indices.
We will define contractions formally by specifying just the indices not to be summed over.

When some of the variables are not appearing as leg variables, we define the contraction as being a tensor product with the trivial tensor $\ones$ carrying the legs of the missing variables.

\begin{definition}\label{def:contraction}
	Let $\tnetof{\graph}$ be a tensor network on a decorated hypergraph $\graph=(\nodes,\edges)$.
	For any subset $\secnodes\subset\nodes$ we define the contraction  to be the tensor 
	\begin{align}
		\contractionof{\tnetof{\graph}}{\catvariableof{\secnodes}} \in \bigotimes_{\node\in\secnodes} \rr^{\catdimof{\node}}
	\end{align}
	defined coordinatewise by the sum	
	\begin{align}
		\contractionof{\tnetof{\graph}}{\indexedcatvariableof{\secnodes}} =
		\sum_{\catindexof{\nodes/\secnodes} \in\,\nodestatesof{\nodes/\secnodes}}
		\left( \prod_{\edge\in\edges}\hypercoreofat{\edge}{\indexedcatvariableof{\edge}} \right) \, .
	\end{align}
	We call $\catvariableof{\secnodes}$ the open variables of the contraction.
\end{definition}


\begin{figure}
	\begin{center}
		\begin{tikzpicture}[scale=0.35,thick] % , baseline = -3.5pt


%\node[anchor=center] (text) at (-2,0) {$a)$};

\draw (-5,-1) rectangle (9,-3);
\node[anchor=center] (text) at (2,-2) {\small $\contractionof{\{\hypercoreof{\edge_0},\hypercoreof{\edge_1},\hypercoreof{\edge_2}\}}{\catvariableof{1},\catvariableof{3}}$};
\draw (0,-3)--(0,-5) node[midway,left] {\tiny $\catvariableof{1}$}; 
\draw (4,-3)--(4,-5) node[midway,left] {\tiny $\catvariableof{3}$}; 

\node[anchor=center] (text) at (11.5,-2) {${=}$};

\begin{scope}[shift={(15,0)}]

%\node[anchor=center] (text) at (-2,0) {$b)$};

\draw (-1,-1) rectangle (5,-3);
\node[anchor=center] (text) at (2,-2) {\small $\hypercoreof{\edge_0}$};
\draw (0,-3)--(0,-4) node[midway,right] {\tiny $\catvariableof{0}$}; 
\draw (-1,-4) rectangle (1,-6);
\node[anchor=center] (text) at (0,-5) {\small $\ones$};

\draw (2,-3)--(2,-5) node[midway,right] {\tiny $\catvariableof{1}$}; 
\draw (4,-3)--(4,-5) node[midway,right] {\tiny $\catvariableof{2}$}; 


\draw (6,-1) rectangle (10,-3);
\node[anchor=center] (text) at (8,-2) {\small $\hypercoreof{\edge_2}$};
\draw (7,-3)--(7,-5) node[midway,right] {\tiny $\catvariableof{2}$}; 
\draw (9,-3)--(9,-5) node[midway,right] {\tiny $\catvariableof{3}$}; 


\draw (1,-7) rectangle (5,-9);
\node[anchor=center] (text) at (3,-8) {\small $\hypercoreof{\edge_1}$};
\draw (2,-5)--(2,-7); % node[midway,left] {\tiny $\catvariableof{1}$}; 
\draw (4,-5) to[bend right=20]  (7,-6); % node[midway,left] {\tiny $\catvariableof{2}$}; 
\draw (4,-7) to[bend right=-20]  (7,-6); 

\draw[fill] (2,-6) circle (0.15cm);
\draw (2,-6) to[bend right=20] (-1,-8); % node[midway, right]{\tiny $\catvariableof{1}$};
\node[anchor=center] (text) at (-2,-8) {\tiny $\catvariableof{1}$};

\draw[fill] (7,-6) circle (0.15cm);
\draw (7,-5) -- (7,-6);
\draw (7,-6)--(7,-7) node[midway,right] {\tiny $\catvariableof{2}$}; 

\draw (6,-7) rectangle (8,-9);
\node[anchor=center] (text) at (7,-8) {\small $\ones$};

\end{scope}


\end{tikzpicture}
	\end{center}
	\caption{
		Example of a tensor network contraction of all but the variables $\catvariableof{1},\catvariableof{3}$.
		Contraction of variables can always be depicted by closing the open legs with trivial tensors $\ones$ performing index sums.
	}\label{fig:contraction}
\end{figure}

%%
%Diagrammatic notation: Best to do version b), since this is easiest to see how tensors combine to new tensors by contractions.

\begin{remark}[Alternative Notations]
	% Einstein summations
	Contractions can also denoted by the Einstein summations of the indices along connected edges, understood as scalar product in each subspace.
	This is as in Definition~\ref{def:contraction}, just omitting the sums.
	We found it useful in this work to do the diagrammatic representation instead, since it offers a better possibility to depict hierarchical arrangements of shared variables.
\end{remark}


% Mode products
Further notations without usage of axis variables are mode products (see \cite{kolda_tensor_2009,hackbusch_tensor_2012,cichocki_tensor_2015}), often denoted by the operation $\times_n$.
With our more generic variable-based notations, we can capture these more specific contractions by coloring the tensor axes, that is assignment of axis variables.

% Examples
To further gain familiarity with the generic contractions, we show the connection to two more popular examples.

%% Diagrammatic representation of Matrix Vector
\begin{example}{Matrix Vector Products}
	The matrix vector product is a special case of tensor contractions, where a matrix $\matrixat{\exrandom,\secexrandom}$ shares a categorical variable with a vector $\vectorat{\secexrandom}$.
	When leaving the variable unique to the matrix open we get the matrix vector product as
		\[ \contractionof{\matrixat{\exrandom,\secexrandom},\vectorat{\secexrandom}}{\exrandom=\exrandind} = \sum_{\secexrandind\in[\secexranddim]} \matrixat{\exrandom=\exrandind,\secexrandom=\secexrandind} \cdot \vectorat{\secexrandom=\secexrandind} \, .  \]

	Exploiting the diagramatic tensor network visualization we depict matrix vector products by:
	\begin{center}
		\begin{tikzpicture}[scale=0.3,thick,xscale=-1] % , baseline = -3.5pt

\draw (-9,2)--(-7,2) node[midway,above] {\colorlabelsize $\exrandom$};
\draw (-21,1) rectangle (-9,3);
\node[anchor=center] (text) at (-15,2) {\corelabelsize $\contractionof{\matrixat{\exrandom,\secexrandom},\vectorat{\secexrandom}}{\exrandom}$};

\node[anchor=center] (text) at (-5,2) {\corelabelsize ${=}$};

\draw (3,2)--(5,2) node[midway,above] {\colorlabelsize $\exrandom$};
\draw (1,1) rectangle (3,3);
\node[anchor=center] (text) at (2,2) {\corelabelsize $\exmatrix$};
\draw (1,2)--(-1,2) node[midway,above] {\colorlabelsize $\secexrandom$};
\draw (-1,1) rectangle (-3,3);
\node[anchor=center] (text) at (-2,2) {\corelabelsize $\exvector$};

%\node[anchor=center] (text) at (7,1) {$\cdot$};


\end{tikzpicture}
	\end{center}
%	Here the index $j$ is represented by a closed edge, which means that it is eliminated by a sum.
\end{example}


%% Hadamard Product 
\begin{example}{Hadamard Products of Vectors}
	A node appearing in arbitrary many hyperedges denotes a Hadamard product of the axis of the respective decorating tensors.
	To give an example, let $\vectorofat{\catenumerator}{\catvariable}\in\rr^\catdim$ be vectors for $\catenumeratorin$. Their hadamard product is the vector
		\[ \contractionof{\{\vectorofat{\catenumerator}{\catvariable} \, : \, \catenumeratorin\}}{\catvariable}  \in \rr^\catdim \]
	defined by
		\[ \contractionof{\{\vectorofat{\catenumerator}{\catvariable} \, : \, \catenumeratorin\}}{\indexedcatvariable}   
		= \prod_{\atomenumeratorin} \vectorofat{\atomenumerator}{\indexedcatvariable}\, . \]
	In a contraction diagram the Hadamard product is depicted by 
	\begin{center}
		\begin{tikzpicture}[scale=0.3,thick] % , baseline = -3.5pt


\begin{scope}[shift={(-10,0)}]

\draw (-5,1) rectangle (7,3);
\node[anchor=center] (text) at (1,2) {\small $\contractionof{V^{0}[\catvariable],\ldots,V^{\catorder-1}[\catvariable]}{\catvariable}$}; % {\small $\contractionof{\{\vectorofat{\catenumerator}{\catvariable} \, : \, \catenumeratorin\}}{\catvariable}$};
\draw (1,-1)--(1,1) node[midway,right] {\tiny $\catvariable$};

\node[anchor=center] (text) at (9,2) {${=}$};

\end{scope}



\draw (1,1) rectangle (3,3);
\node[anchor=center] (text) at (2,2) {\small $\vectorof{0}$};
\draw (2,-1)--(2,1) node[midway,right] {\tiny $\catvariable$};


\begin{scope}[shift={(5,0)}]

\draw (1,1) rectangle (3,3);
\node[anchor=center] (text) at (2,2) {\small $\vectorof{1}$};
\draw (2,-1)--(2,1) node[midway,right] {\tiny $\catvariable$};

\end{scope}

\node[anchor=center] (text) at (11.5,2) {\small $\cdots$};


\begin{scope}[shift={(15,0)}]

\draw (0.75,1) rectangle (3.25,3);
\node[anchor=center] (text) at (2,2) {\small $\vectorof{\atomorder\shortminus1}$};
\draw (2,-1)--(2,1) node[midway,right] {\tiny $\catvariable$};

\end{scope}


\draw[fill] (9.125,-4.5) circle (0.15cm);

\draw (9.125,-4.5) to[bend right=-20] (2,-1); 
\draw (9.125,-4.5) to[bend right=-20] (7,-1); 
\draw (9.125,-4.5) to[bend right=20] (17,-1); 

\draw (9.125,-4.5) -- (9.125,-6.5) node[midway,right] {\tiny $\catvariable$};; 

\end{tikzpicture}
	\end{center}
\end{example}



\subsubsection{Decompositions}

Tensors can be represented by tensor network decompositions, when the contraction of the network retrieves the tensor.

\begin{definition}\label{def:tnDecomposition}
	%Let $\hypercoreat{\nodevaraibles}$ be a tensor in $\extensorspace$.
	A Tensor Network Decomposition of a tensor $\hypercoreat{\nodevariables}$ is a Tensor Network $\tnetof{\graph}$ such that
		\[ \hypercoreat{\nodevariables}= \contractionof{\tnetof{\graph}}{\nodevariables} \, . \]
	We call the hypergraph $\graph$ the format of the decomposition.
\end{definition}



\subsection{Properties of Tensors}

%% Boolean
We will often encounter situations, where the coordinates of tensors are in $\{0,1\}=[2]$.

\begin{definition}\label{def:booleanTensor} % CALL BOOLEAN INSTEAD?
	We call a tensor $\hypercoreat{\shortcatvariables}$ boolean, when $\imageof{\hypercore}\subset[2]$, i.e. all coordinates are either $0$ or $1$.
\end{definition}

%% Directionality
Directionality represents constraints on the structure of tensors:
Summing over outgoing trivializes the tensor.

\begin{definition}\label{def:directedTensor}
	A Tensor 
		\[ \hypercoreat{\nodevariables} \in \bigotimes_{\nodein}\rr^{\catdimof{\node}} \]
	is said to be directed with incoming variables $\innodes$ and outgoing variables $\outnodes$, where $\nodes=\innodes\dot{\cup}\outnodes$, when
		\[ \sbcontractionof{\hypercore}{\catvariablesinset{\outnodes}} =  \onesat{\catvariablesinset{\innodes}} \]
	where $\onesat{\catvariablesinset{\innodes}}$ denoted the trivial tensor in  $\bigotimes_{\node\in\innodes}\rr^{\catdimof{\node}}$ which coordinates are all $1$.
\end{definition}

While by default all legs are outgoing, we can change the direction by normation.

\begin{definition}\label{def:normation}
	A tensor $\hypercoreat{\nodevariables}$ is said to be normable on $\innodes\subset\nodes$, if for any $\catindexof{\innodes}\in\nodestatesof{\innodes}$ we have
		\[ \sbcontraction{\hypercoreat{\nodevariables},\onehotmapofat{\atomlegindexof{\innodes}}{\catvariableof{\innodes}}} > 0 \, . \]
	The normation of a on $\innodes\subset\nodes$ normable tensor is the tensor
	\begin{align*}
		\sbnormationofwrt{\hypercoreat{\nodevariables}}{\catvariableof{\outnodes}}{\catvariableof{\innodes}} = 
		\sum_{\catindexof{\innodes}\in\nodestatesof{\innodes}} 
		\onehotmapofat{\atomlegindexof{\innodes}}{\catvariableof{\innodes}} \otimes \frac{
		\sbcontractionof{\hypercoreat{\nodevariables},\onehotmapofat{\catindexof{\innodes}}{\catvariableof{\innodes}}}{\catvariableof{\outnodes}}
		}{
		\sbcontraction{\hypercoreat{\nodevariables},\onehotmapofat{\catindexof{\innodes}}{\catvariableof{\innodes}}}
		} 
	\end{align*}
	where $\outnodes = \nodes/\innodes$.
\end{definition}

We will investigate the contractions of directed tensors in Part~III, where we show in Theorem~\ref{the:normationDirected} that normations are directed tensors.


%% Diagrammatic notation
In our graphical tensor notation, we depict directed tensors by directed hyperedges (a), which are decorated by directed tensors (b), for example:
%\red{Draw incoming and outgoing example.}
	\begin{center}
		


\begin{tikzpicture}[scale=0.35,thick] % , baseline = -3.5pt

\node[anchor=center] (text) at (-2,0) {$a)$};

\node [circle, draw, thick, fill=\nodegrayscale, minimum size = \nodeminsize] (P1) at (0,-3) {\colorlabelsize $\catvariableof{0}$};
\node [circle, draw, thick, fill=\nodegrayscale, minimum size = \nodeminsize] (P2) at (3,-3) {\colorlabelsize $\catvariableof{1}$};

%\node[anchor=center] (text) at (6,-3) {$\cdots$};
\node [circle, draw, thick, fill=\nodegrayscale, minimum size = \nodeminsize] (P3) at (6,-3)  {\colorlabelsize $\catvariableof{2}$};

\node [circle, draw, thick, fill=\nodegrayscale, minimum size = \nodeminsize] (P4) at (9,-3)  {\colorlabelsize $\catvariableof{3}$};

\node[anchor=center] (text) at (9,-3) {\colorlabelsize $\catvariableof{3}$};


\draw[midarrow] 
    	(4.5,0) to[bend right=25] (P1);
\draw[midarrow] 
    	(4.5,0) to[bend right=10] (P2);
\draw[midarrow] 
    	(P3) to[bend right=10] (4.5,0);
\draw[midarrow] 
	(P4) to[bend right=25] (4.5,0);
	
\node[anchor=center] (text) at (4.5,0.5) {$\edge$};


\begin{scope}[shift={(20,0)}]

\node[anchor=center] (text) at (-2,0) {$b)$};

\draw (-1,-1) rectangle (7,-3);
\node[anchor=center] (text) at (3,-2) {\corelabelsize $\hypercoreofat{\edge}{\catvariableof{0},\catvariableof{1},\catvariableof{2},\catvariableof{3}}$};
%\draw[->-] (0,-3)--(0,-5) node[midway,left] {\colorlabelsize $\catvariableof{0}$};
%\draw[->-] (1.5,-3)--(1.5,-5) node[midway,left] {\colorlabelsize $\catvariableof{1}$};
%\node[anchor=center] (text) at (3,-4) {$\cdots$};
%\draw[->-] (4,-3)--(4,-5) node[midway,right] {\colorlabelsize $\catvariableof{\atomorder-1}$};


\draw[midarrow]  (0,-3) -- (0,-5) node[midway,left] {\colorlabelsize $\catvariableof{0}$};
\draw[midarrow] 
    (2,-3)--(2,-5) node[midway,left] {\colorlabelsize $\catvariableof{1}$};
\draw[midarrow] 
    (4,-5)--(4,-3) node[midway,left] {\colorlabelsize $\catvariableof{2}$};
\draw[midarrow] 
   (6,-5)--(6,-3) node[midway,right] {\colorlabelsize $\catvariableof{3}$};
\end{scope}



\end{tikzpicture}
	\end{center}



\subsection{Encoding schemes for functions}

Tensors are defined here as real-valued functions on the state set of a system described by categorical variables.
We provide further schemes to represent functions in order to perform sparse calculus and to handle more generic functions.



%
%\subsubsection{Real-valued functions}
%\begin{example}[Uncertainty about States]\label{exa:onehotUncertainty}
%	The uncertainty about the state of a categorical variable $\catvariable$ can be expressed in vectors.
%	For example let there be real numbers $\probof{\catvariable=\catindex} \in [0,1]$ for $\catindex\in[\catdim]$ with $\sum_{\catindex\in[\catdim]}\probof{\catvariable=\catindex}=1$ with the interpretation that $\probof{\catvariable=\catindex}$ is the probability of a system being in state $\catindex$. 
%	We can represent this uncertain state simply by a vector 
%		\[ \probof{\catvariable}\in\rr^{\catdim} \]
%	defined as the sum of one-hot representations weighted by $\probof{\catvariable=\catindex}$
%	\[ \sum_{\catindex\in[\catdim]} \probof{\catvariable=\catindex} \cdot \onehotmapofat{\catindex}{\catvariable} =
%		\begin{bmatrix}
%		\probof{\catvariable=0} & \probof{\catvariable=1} & \cdots & \probof{\catvariable=\catdim-1}
%		\end{bmatrix} \, . 
%	\]
%\end{example}



\subsubsection{Relational encodings}

%We have already observed in Example~\ref{exa:atomicFunction}, that any function of a categorical variable has a representation as a linear function acting on the one-hot encoding of the variable.
Let us now show how we can encode maps between factored systems.
The scheme is described in more generality and detail (encoding of subsets and relations) in Chapter~\ref{cha:tensorEncodings}, see Definition~\ref{def:functionRelationEncoding}.

\begin{definition}[Relation encoding of maps between Factored Systems]\label{def:functionRepresentation}
	Let $\exfunction$ be a function
		\[ \exfunction : \facstates \rightarrow  \secfacstates \]
	which maps the states of a factored system to variables $\catvariables$ to the states of another factored system with variables $\seccatvariables$.
	Then the tensor representation of $\exfunction$ is a tensor
		\[ \rencodingofat{\exformula}{\catvariables,\seccatvariables} \in   \left(\facspace\right) \otimes \left(\secfacspace\right)  \]
	defined by
		\[ \rencodingofat{\exformula}{\catvariables,\seccatvariables}= \sum_{\catindices\in\facstates}  
		  \onehotmapofat{\catindices}{\catvariables} \otimes \onehotmapofat{\exfunction(\catindices)}{\seccatvariables} \, . \]
\end{definition}



% Notation with image categorical variable
%When the categorical variables of the image factored system to a map $\exfunction$ are not specified otherwise, we will denote them by $\catvariableof{\exfunction}$.




\subsubsection{Tensor-valued functions}


%% TO DETAILLED HERE -> Part III?
\begin{definition}[Selection encoding of Maps between Factored Systems]\label{def:selectionEncoding}
	Given a tensor space $\parspace$ described by categorical variables $\selvariables$ and a tensor-valued function
		\[ \exfunction : \facstates \rightarrow \parspace \]
	the selection encoding of $\exfunction$ is a tensor
		\[ \sencodingofat{\exfunction}{\shortcatvariables,\shortselvariables} \in \left(\facspace\right) \otimes \left(\parspace\right) \]
	defined by the basis decomposition
		\[ \sencodingofat{\exfunction}{\shortcatvariables,\shortselvariables} = \sum_{\catindices\in\facstates} \onehotmapofat{\catindices}{\shortcatvariables} \otimes \exfunction(\catindices)[\shortselvariables] \, .  \]
\end{definition}

%%
We call these tensor representation of maps selection encodings, since the coordinate of a function $\exfunction$ to be processed is selected by another argument to $\sencodingof{\exfunction}$.

%\begin{example}[Vector valued functions]\label{exa:atomicFunction} %% CONFUSIN, since already needs selection variables?
%	When using a one-hot representation of the state of a categorical variable, any real valued function has a representation by a real valued matrix acting on the one-hot encoding. 
%	Let there be a vector valued function
%		\[ \exformula : [\catdim] \rightarrow \rr^p \]
%	which maps $\catindex\in[\catdim]$ to the vector
%		\[ \exformula(\catindex)[\selvariable] \in \rr^p \, , \]
%	where we introduced the variable $\selvariable\in[p]$ selecting a coordinate of the image vector.
%	The 
%		\[ \exformula(\catindex)[\selvariable] = 
%		\contractionof{\{\onehotmapof{\catindex}[\catvariable] , \,\concore_{\exformula}[\catvariable,\selvariable]\}}{\selvariable}  \]
%	where $\concore_{\exformula} \in \rr^{\catdim \times p} $ is the matrix defined by the function evaluation vectors of $\exformula$ as
%		\[ \concore_{\exformula}[\catvariable,\selvariable] = \begin{bmatrix}
%			-- & \exformula(0) & -- \\
%			-- & \exformula(1) & -- \\
%			& \vdots &  \\
%			-- & \exformula(\catdim-1) & -- 
%		\end{bmatrix} \, . 
%		\]
%	This can easily be verified, since matrix multiplication with basis vectors amounts to selection of rows (when the basis vector is acting from the left) or columns (when the basis vector is acting from the right).
%	Thus, linear transforms (matrices) acting on the one-hot representation are sufficient to represent any vector valued function of the states of a categorical variable.
%\end{example} 


%% 
We will provide more detail to the tensor representation of functions in Part~III, where we distinguish between embeddings for basis and coordinate calculus. %where we show that domain encodings coincide with selection encodings.







    \part{\partonetext}\label{par:one}
    \chapter{Introduction into Part I}

Within the introduced factored representation of systems we will in \parref{par:one} present the probabilistic and logical approaches to artificial intelligence.
%Both paradigms have developed human interpretable tools into the representation of and reasoning on knowledge.
Both the probabilitic and the logic paradigm provide a human-understandable interface to machine learning.
\begin{itemize}
    \item \textbf{Probability:} Models describe dependencies between variables, which receive a graphical representation.
    \item \textbf{Logics:} Models are formulated in human interpretable logical syntax.
\end{itemize}
As we will describe in \parref{par:two}, they can be combined in one formalism providing efficient reasoning.
We will utilize that tensor network decompositions are in both useful tools of efficient calculus.


\sect{Representation of Factored Systems}

%\subsect{Comparing with probabilistic approaches }
\textbf{Probability} represents the uncertainty of states.
The categorical variables are called random variables and their joint distribution is represented by a probability tensor.
Humans interpret probabilities by Bayesian and frequentist approaches.
Reasoning based on Bayes Theorem has an intuitive interpretation in terms of evidence based update of prior distributions to posterior distributions.
However it is based on interpreting (large amounts) of numbers, which makes it hard for humans to assess the probabilistic reasoning process.

\textbf{Propositional Logics} explains relations between sets of worlds in a human understandable way.
Categorical variables have dimension $2$, where the first is interpreted as indicating a $\falsesymbol$ state and the second as a $\truesymbol$ state.
We mainly restrict to propositional logics, where there are finite sets of such variables called atomic formulas.
Using model-theoretic semantics it defines entailment of sets by other sets, which is understandable as a consequence relation.

\textbf{Tensors} unify both approaches since they are natural numerical structures to represent properties of states in factored systems.
The potential is then based in employing scalable multilinear algorithms to solve reasoning problems.
Further, algorithms formulated in tensor networks have a high parallelization potential, which is why they are of central interest in the development of AI-dedicated software and hardware.

The different areas have developed separated languages to describe similar objects.
Here we want to provide a rough comparison of those in a dictionary.

\begin{tabular}{|p{\fourcolumnwidth}|p{\fourcolumnwidth}|p{\fourcolumnwidth}|p{\fourcolumnwidth}|}
    \hline
    & \textbf{Probability Theory} & \textbf{Propositional Logic} & \textbf{Tensors}   \\
    \hline
    \textit{Atomic System}        & Random Variable             & Atomic Formula               & Vector             \\
    \textit{Factored System}      & Joint Distribution          & Knowledge Base               & Tensor             \\
    \textit{Categorical Variable} & Random Variable             & Atomic Formula               & Axis of the Tensor \\
    \hline
\end{tabular}

While the probability theory lacks to provide an intuition about sets of events, propositional syntax has limited functionality to represent uncertainties.
Tensors on the other side can build a bridge by representing both functionalities and relying on probability theory and logics for respective interpretations.

\sect{Mechanisms of tensor network decompositions}

We investigate two mechanisms to identify tensor network decompositions of probability distributions:
\begin{itemize}
    \item \textbf{Independence approach:} Conditional independence of random variables is a a concept of probability theory.
       The most prominent application of this approach is the motivation of graphical models, which we introduce in \charef{cha:probRepresentation} as tensor networks.
    \item \textbf{Computation approach:} When there are sufficient statistics providing probabilities, we construct tensor networks decompositions by computation of the statistics.
        Whenever the functions to be computed are compositions of functions of lower numbers of arguments, we utilize these representations to construct tensor network decompositions.
        Such decomposition schemes are provided by logical syntax as we will exploit in \charef{cha:logicRepresentation}.
        In probability theory, we will make use of this approach in the efficient representation of sufficient statistics.
\end{itemize}


    \chapter{\chatextprobRepresentation}\label{cha:probRepresentation}

In this chapter we will establish relations between the formalism of tensor networks and basic concepts of probability theory.
We will first understand distributions as tensors and connect their marginalizations and conditionings to the tensor operations of contractions and normalizations.
Then we discuss independence assumptions as examples of contraction equations, which lead to tensor network decompositions known as graphical models.
We then treat more generic exponential families and investigate their representation as tensor networks.

\red{We investigate two mechanisms to identify tensor network decompositions of probability distributions:
\begin{itemize}
    \item Independence approach: Conditional independence implies representations by markov networks
    \item Computation approach: When there are sufficient statistics providing probabilities, we construct tensor networks decompositions by computation of the statistics.
\end{itemize}
}

\sect{Classical Properties of Distributions}

To start, we first relate classical properties of distributions, such as independent variables, with the tensor network formalism.

\subsect{Probability Tensors}

%% Random Variables: Introduction in Bayesian way by uncertainties
After having discussed how to represent states of factored systems by one-hot encodings, let us now take advantage of these representation by associating properties with these states.
Let there be uncertainties of the assignments $\catindexof{\atomenumerator}$ to the categorical variables $\catvariableof{\atomenumerator}$ of a factored system.
We then understand $\catvariableof{\atomenumerator}$ as random variables, which have a joint distribution defined by the uncertainties of the state assignments.
To capture these uncertainties we now make use of the one-hot representation of factored systems.

\begin{definition}[Probability Tensor]
    \label{def:probabilityDistribution} % From the axioms of Kolmogorov!
    Let there be for each $\catenumeratorin$ a categorical variable $\catvariableof{\catenumerator}$ taking values in $[\catdimof{\atomenumerator}]$.
    A joint probability distribution of these categorical variables is a tensor
    \begin{align*}
        \probat{\catvariableof{0},\ldots,\catvariableof{\atomorder-1}} : \facstates \rightarrow [0,1] \subset \rr
    \end{align*}
    such that
    \begin{align*}
        \contraction{\probat{\catvariableof{0},\ldots,\catvariableof{\atomorder-1}}} = 1 \, .
%		\sum_{\catindices\in\facstates} \probat{\indexedcatvariables} = 1 \, .
    \end{align*}
\end{definition}

%% One-hot Decomposition -> Contraction Equivalences
The probability tensor to the distribution is a tensor
\begin{align*}
    \probwith\in\bigotimes_{\catenumeratorin}\rr^{\catdimof{\atomenumerator}} \, ,
\end{align*}
where we again use the abbreviation $\shortcatvariables$ for the list of variables $\catvariableof{0},\ldots,\catvariableof{\catorder-1}$.
The tensor can be decomposed as the sum (see \lemref{lem:tensorBasisDecomposition} in \parref{par:three} for more details)
\begin{align*}
    \probwith = \sum_{\shortcatindices\in\facstates} \probat{\indexedshortcatvariables} \cdot \onehotmapofat{\shortcatindices}{\shortcatvariables} \, ,
\end{align*}
where we understand $\probat{\indexedshortcatvariables}$ as the probability of the categorical variables to take the state $\shortcatindices\in\facstates$.
%% normalization condition by Kolmogorovs second axiom
The normalization condition $1=\contraction{\probwith}$ has a more convenient equivalence by the coordinate sum
\begin{align*}
    1 = \contraction{\probwith}
    =  \sum_{\shortcatindices\in\facstates}\probat{\indexedshortcatvariables} \, ,
\end{align*}
and thus ensures that all probabilities sum to $1$, which is necessary for the probabilistic interpretation.
While the assumptions of non-negative coordinates in \defref{def:probabilityDistribution} reflects the first probability axiom of Kolmogorov, the assumption of contraction $1$ implements the second axiom (see for example \cite{degroot_probability_2016}).
Since probability distributions contract to $1$, they are directed (see \defref{def:directedTensor}) with all distributed variables outgoing and empty incoming variables (see \figref{fig:probabilityTensor}).

\begin{figure}[hbt!]
    \begin{center}
        \begin{tikzpicture}[scale=0.35,thick] % , baseline = -3.5pt

    \node[anchor=center] (text) at (-2,0) {$a)$};

    \node [circle, draw, thick, fill=gray!50, minimum size = \nodeminsize] (P1) at (0,-3) {\tiny $\catvariableof{0}$};
    \node [circle, draw, thick, fill=gray!50, minimum size = \nodeminsize] (P2) at (3,-3) {\tiny $\catvariableof{1}$};

    \node[anchor=center] (text) at (6,-3) {$\cdots$};

    \node [circle, draw, thick, fill=gray!50, minimum size = \nodeminsize] (P3) at (9,-3) {};

    \node[anchor=center] (text) at (9,-3) {\tiny $\catvariableof{\atomorder-1}$};


    \draw[->]
    (4.5,0) to[bend right=25] (P1);
    \draw[->]
    (4.5,0) to[bend right=10] (P2);
    \draw[->]
    (4.5,0) to[bend right=-25] (P3);

    \node[anchor=center] (text) at (4.5,0.5) {$\edge$};


    \begin{scope}
        [shift={(20,0)}]

        \node[anchor=center] (text) at (-2,0) {$b)$};

        \draw (-1,-1) rectangle (5,-3);
        \node[anchor=center] (text) at (2,-2) {\small $\probtensor$};
%\draw[->] (0,-3)--(0,-5) node[midway,left] {\tiny $\catvariableof{0}$};
%\draw[->] (1.5,-3)--(1.5,-5) node[midway,left] {\tiny $\catvariableof{1}$};
        \node[anchor=center] (text) at (3,-4) {$\cdots$};
%\draw[->] (4,-3)--(4,-5) node[midway,right] {\tiny $\catvariableof{\atomorder-1}$};


        \draw[midarrow]  (0,-3) -- (0,-5) node[midway,left] {\tiny $\catvariableof{0}$};
        \draw[midarrow]
        (1.5,-3)--(1.5,-5) node[midway,left] {\tiny $\catvariableof{1}$};
        \draw[midarrow]
        (4,-3)--(4,-5) node[midway,right] {\tiny $\catvariableof{\atomorder-1}$};
    \end{scope}


\end{tikzpicture}
    \end{center}
    \caption{Probability distributions of variables $\catvariableof{0},\ldots,\catvariableof{\atomorder-1}$, sketched
    a) by a directed edge $\edge$ with all variables outgoing, which is decorated b) by a directed tensor $\probwith$.}\label{fig:probabilityTensor}
\end{figure}


\subsect{Base measures}\label{sec:baseMeasure}

From a measure theoretic perspective, probabilities are measurable functions called probability densities, which integrals are $1$ (see for example \cite{degroot_probability_2016}). % Add citations?
In our case of finite dimensional state spaces of factored systems, we implicitly used the trivial tensor $\onesat{\shortcatvariables}$ as a base measure, which measures subsets of states by their cardinality and is therefore refered to as state counting base measure.
The distribution tensors $\probwith$ can then be understood as probability densities with respect to this state counting base measure.
We in this work will also consider more general base measures $\basemeasurewith$, which we restrict to be boolean, that is $\basemeasureat{\indexedshortcatvariables}\in\ozset$ for all states $\shortcatindices$.
When understanding $\probwith$ as a probability density with respect to $\basemeasurewith$, any probabilistic interpretation will be through the contraction $\contractionof{\probwith,\basemeasurewith}{\shortcatvariables}$ and the normalization condition reads as
\begin{align*}
    \contraction{\probwith,\basemeasurewith} = 1 \, .
\end{align*}
Since we restrict to boolean base measures, the contraction effectively manipulates the tensor $\probtensor$ by setting the coordinates $\probat{\indexedshortcatvariables}$ to zero, when $\basemeasureat{\indexedshortcatvariables}=0$.
Therefore, multiple tensors $\probtensor$ will have the same proabilistic interpretation, when $\basemeasurewith\neq\onesat{\shortcatvariables}$.
To avoid this ambiguity, we introduce the notation of representability with respect to a base measure $\basemeasure$, by demanding that such coordinates are zero.

\begin{definition}
    \label{def:representationBaseMeasure}
    We say that a probability distribution $\probtensor$ is representable with respect to a boolean base measure $\basemeasure$, if for all $\shortcatindices$ with $\basemeasureat{\indexedshortcatvariables}=0$ we have $\probat{\indexedshortcatvariables}=0$.
    We denote the set of by $\basemeasure$ representable distributions by $\bmrealprobof{\basemeasure}$.
\end{definition}

When a probability distribution $\probtensor$ is representable with respect to a boolean base measure $\basemeasure$, we have the invariance
\begin{align*}
    \probwith=\contractionof{\probwith,\basemeasurewith}{\shortcatvariables}
\end{align*}
and can therefore safely ignore the base measures.
This enables the characterization of by $\basemeasure$ representable distributions by
\begin{align*}
    \bmrealprobof{\basemeasure}
    = \left\{ \probat{\shortcatvariables} \, : \, \uniquantwrtof{\shortcatindicesin}{\probat{\indexedshortcatvariables}\geq0}, \, \contractionof{\probwith,\basemeasurewith}{\shortcatvariables}
    = \probwith \right\} \, .
\end{align*}

Starting with \charef{cha:logicalRepresentation} we will further investigate boolean tensors and relate them with propositional formulas.
In \charef{cha:logicalReasoning} we will connect the representation and positivity with respect to boolean base measures with the formalism of entailment.
The notation $\bmrealprobof{\basemeasure}$ of by $\basemeasure$ representable distributions will later in \charef{cha:networkRepresentation} relate to minterm exponential families introduced therein.

% Positive distribution
We now investigate, which base measures $\basemeasure$ can be chosen for a probability distribution $\probtensor$, such that $\probtensor$ is representable by $\basemeasure$.
Here we want to find a $\basemeasure$, which is in a sense to be defined minimal amount the base measures, such that $\probtensor$ is representable with respect to them.
For this minimality criterion we will develop in \charef{cha:logicalReasoning} orders based on entailment and show the minimality in \theref{the:minimalRepPosBaseMeasure}.
Here, we just introduce the minimality criterion as positivity of a distribution with respect to a base measure.

\begin{definition}
    \label{def:positivityBaseMeasure}
    We say that a probability distribution $\probwith$ is positive with respect to a boolean base measure $\basemeasurewith$, if the distribution is representable by $\basemeasure$ (i.e. $\contraction{\probtensor,\basemeasure}=1$) and for all $\shortcatindices$ with $\basemeasureat{\indexedcatvariables}=1$ we have $\probat{\indexedcatvariables}>0$.
\end{definition}

%This is a slide abuse of the measure theoretic approach to probability theory, since typically the base measure needs to be defined before considering probability distributions. 


\subsect{Marginal Distribution}

Contractions of probability distributions are related to marginalizations as we introduce next.

\begin{definition}[Marginal Probability]
    \label{def:marginalProbability}
    Given a distribution $\probat{\exrandom,\secexrandom}$ of the categorical variables $\exrandom$ and $\secexrandom$ the marginal distribution of the categorical variable $\exrandom$ is defined for each $\exrandind$ as the tensor
    \begin{align*}
        \probat{\exrandom} : [\exranddim] \rightarrow \rr
    \end{align*}
    defined by the contraction
    \begin{align*}
        \probat{\exrandom}
        = \contractionof{\probat{\exrandom,\secexrandom}}{\exrandom} \, .
%		= \sum_{\secexrandindin} \probat{\indexedexrandom,\indexedsecexrandom} \, .
    \end{align*}
\end{definition}

To connect with a more standard defining equation of marginal distributions, let us notice that for any $\exrandindin$
\begin{align*}
    \probat{\indexedexrandom}
    = \contractionof{\probat{\exrandom,\secexrandom}}{\indexedexrandom}
    = \sum_{\secexrandindin} \probat{\indexedexrandom,\indexedsecexrandom} \, .
\end{align*}
Thus, each coordinate of the marginal distribution is the sum of the joint probability of compatible states.
We say that the variable $\secexrandom$ is marginalized out, when building the marginal distribution $\probat{\exrandom}$ of $\exrandom$.
Let us now justify this terminology and show, that any marginal distribution is a probability distribution as introduced in \defref{def:probabilityDistribution}.

\begin{theorem}
    \label{the:marginalContraction}
    Any marginal distribution is a probability distribution.
\end{theorem}
\begin{proof}
    We further have that any marginal distribution is normed, since by the commutativity of contractions (see for more details \theref{the:splittingContractions} in \parref{par:three})
    \begin{align*}
        \contraction{\probat{\exrandom}} = \contraction{\contractionof{\probat{\exrandom,\secexrandom}}{\exrandom}} = \contraction{\probat{\exrandom,\secexrandom}} = 1 \, .
    \end{align*}
    Further any coordinate is non-negative, since it is a sum of non-negative coordinates.
    It follows from \defref{def:probabilityDistribution}, that any marginal distribution is a probability distribution.
\end{proof}

% Diagrammatic representation
In a tensor network diagram we often represent variables $\secexrandom$ not appearing as open variables of a contraction as contracted with the trivial tensor $\onesat{\secexrandom}$.
Following this notation, we depict the marginal distribution in \defref{def:marginalProbability} by
\begin{center}
    \begin{tikzpicture}[scale=0.3,thick] % , baseline = -3.5pt

\draw (-19,-1) rectangle (-15,-3);
\node[anchor=center] (text) at (-17,-2) {\small $\margprobat{\exrandom}$};
\draw[midarrow]  (-17,-3)--(-17,-5) node[midway,left] {\tiny $\exrandom$}; 

\node[anchor=center] (text) at (-13,-2) {${=}$};

\draw (-11,-1) rectangle (-5,-3);
\node[anchor=center] (text) at (-8,-2) {\small $\probat{\exrandom,\secexrandom}$};
\draw[midarrow]  (-10,-3)--(-10,-5) node[midway,left] {\tiny $\exrandom$}; 
\draw[midarrow]  (-6,-3)--(-6,-5) node[midway,left] {\tiny $\secexrandom$};
\draw (-7,-5) rectangle (-5,-7); 
\node[anchor=center] (text) at (-6,-6) {$\ones$};

\end{tikzpicture}
\end{center}
Since we have shown, that marginal distributions are themself probability distributions, they inherit the outgoing directionality in tensor network diagrams.

% Sets of variables
We notice, that \defref{def:marginalProbability} generalizes to marginalizations of arbitrary sets of variables, when having a distribution $\probat{\shortcatvariables}$ of an arbitrary number of categorical variables.
It suffices for this to interpret $\exrandom$ and $\secexrandom$ as collections of variables, which indices take the states of the respective factored systems.

\subsect{Conditional Probabilities}

Normalizations of probability distributions result in conditional distributions as we define next.

\begin{definition}[Conditional Probability]
    \label{def:conditionalProbability}
    Let $\probat{\exrandom,\secexrandom}$ be a distribution of the categorical variables $\exrandom$ and $\secexrandom$, such that $\probat{\exrandom,\secexrandom}$ is normable on $\{\secexrandom\}$.
    Then the distribution of $\exrandom$ conditioned on $\secexrandom$ is defined by
    \begin{align*}
        \condprobof{\exrandom}{\secexrandom}
        = \normalizationofwrt{\probat{\exrandom,\secexrandom}}{\exrandom}{\secexrandom} \, .
    \end{align*}
\end{definition}

\begin{figure}[hbt!]
    \begin{center}
        \begin{tikzpicture}[scale=0.3,thick] % , baseline = -3.5pt


    \node[anchor=center] (text) at (-2,0) {$a)$};

    \node [circle, draw, thick, fill=gray!50, minimum size = \nodeminsize] (P1) at (0,-3) {\tiny $\exrandom$};
    \node [circle, draw, thick, fill=gray!50, minimum size = \nodeminsize] (P3) at (9,-3) {};

    \node[anchor=center] (text) at (9,-3) {\tiny $\secexrandom$};

    \draw[->]
    (4.5,0) to[bend right=25] (P1);
    \draw[<-]
    (4.5,0) to[bend right=-25] (P3);

    \node[anchor=center] (text) at (4.5,0.5) {$\edge$};

    \begin{scope}
        [shift={(43,0)}]

        \node[anchor=center] (text) at (-23,0) {$b)$};

        \draw (-21,-1) rectangle (-15,-3);
        \node[anchor=center] (text) at (-18,-2) {\small $\condprobof{\exrandom}{\secexrandom}$};
        \draw[->]  (-20,-3)--(-20,-5) node[midway,left] {\tiny $\exrandom$};
        \draw[<-]  (-16,-3)--(-16,-5) node[midway,left] {\tiny $\secexrandom$};

    \end{scope}

\end{tikzpicture}
    \end{center}
    \caption{Depiction of conditional probability distributions a) by an edge with the incoming variable $\secexrandom$ and the outgoing variable $\secexrandom$, which is decorated by b) the directed tensor $\condprobof{\exrandom}{\secexrandom}$. }
    \label{fig:conditionalDistribution}
\end{figure}

Since conditional probabilities are normalizations of probability tensors, they are directed and therefore depicted by directed hyperedges (see \figref{fig:conditionalDistribution}).
For any $\secexrandindin$ we depict the slice $\condprobof{\exrandom}{\indexedsecexrandom}$ defined by a normalization operation as
\begin{center}
    \begin{tikzpicture}[scale=0.3, thick] % , baseline = -3.5pt

    \begin{scope}
        [shift={(-13,0)}]

        \draw (-22,-1) rectangle (-14,-3);
        \node[anchor=center] (text) at (-18,-2) {\corelabelsize $\condprobof{\exrandom}{\indexedsecexrandom}$};
        \draw[->-]  (-18,-3)--(-18,-5) node[midway,left] {\colorlabelsize $\exrandom$};

        \node[anchor=center] (text) at (-12,-2) {${=}$};

    \end{scope}

    \begin{scope}
        [shift={(-1,0)}]

        \draw (-21,-1) rectangle (-15,-3);
        \node[anchor=center] (text) at (-18,-2) {\corelabelsize $\condprobof{\exrandom}{\secexrandom}$};
        \draw[->-]  (-20,-3)--(-20,-5) node[midway,left] {\colorlabelsize $\exrandom$};

        \draw[-<-]  (-16,-3)--(-16,-5) node[midway,left] {\colorlabelsize $\secexrandom$};
        \draw[] (-15,-5) rectangle (-17,-7);
        \node[anchor=center] (text) at (-16,-6) {\corelabelsize $\onehotmapof{\secexrandind}$};

        \node[anchor=center] (text) at (-13,-2) {${=}$};
    \end{scope}

    \begin{scope}
        [shift={(0,6)}]

        \draw (-11,-1) rectangle (-5,-3);
        \node[anchor=center] (text) at (-8,-2) {\corelabelsize $\probat{\exrandom,\secexrandom}$};
        \draw[->-]  (-10,-3)--(-10,-5) node[midway,left] {\colorlabelsize $\exrandom$};
        \draw[->-]  (-6,-3)--(-6,-5) node[midway,left] {\colorlabelsize $\secexrandom$};
        \draw[] (-7,-5) rectangle (-5,-7);
        \node[anchor=center] (text) at (-6,-6) {\corelabelsize $\onehotmapof{\secexrandind}$};

    \end{scope}

    \draw (-12,-2) -- (-4,-2);

    \begin{scope}
        [shift={(0,-2)}]

        \draw (-11,-1) rectangle (-5,-3);
        \node[anchor=center] (text) at (-8,-2) {\corelabelsize $\probat{\exrandom,\secexrandom}$};
        \draw[->-]  (-10,-3)--(-10,-5) node[midway,left] {\colorlabelsize $\exrandom$};
        \draw (-11,-5) rectangle (-9,-7);
        \node[anchor=center] (text) at (-10,-6) {$\ones$};
        \draw[->-]  (-6,-3)--(-6,-5) node[midway,left] {\colorlabelsize $\secexrandom$};
        \draw[] (-7,-5) rectangle (-5,-7);
        \node[anchor=center] (text) at (-6,-6) {\corelabelsize $\onehotmapof{\secexrandind}$};

        \node[anchor=center] (text) at (-4,-6) {$.$};
    \end{scope}


\end{tikzpicture}
\end{center}

As we have done before for marginal distribution, we relate \defref{def:conditionalProbability} with a more convenient coordinatewise definition of conditional probababilities.
For any indices $\exrandindin$ and $\secexrandindin$ we have
\begin{align*}
    \condprobof{\indexedexrandom}{\indexedsecexrandom}
    = \frac{\probat{\indexedexrandom,\indexedsecexrandom}}{\contractionof{\probat{\exrandom,\secexrandom}}{\indexedsecexrandom}}
    = \frac{\probat{\indexedexrandom,\indexedsecexrandom}}{\sum_{\exrandindin} \probat{\indexedexrandom,\indexedsecexrandom}} \, .
\end{align*}
The distribution of $\exrandom$ conditioned on $\secexrandom$ is the normed collection of slice of the probability distribution $\probat{\exrandom,\secexrandom}$.
Each slice of the conditioned distribution with respect to incoming variables is a probability distribution itself, as we show next.

\begin{theorem}
    \label{the:conditionalContraction}
    For any $\secexrandindin$ the tensor $\condprobof{\exrandom}{\indexedsecexrandom}$ is a probability tensor.
\end{theorem}
\begin{proof}
    As a normalization of a non-negative tensor, the conditional probability $\condprobof{\exrandom}{\indexedsecexrandom}$ and any of its slices is also a non-negative tensor.
    Further, we have for any $\secexrandindin$
    \begin{align*}
        \contraction{\condprobof{\exrandom}{\indexedsecexrandom}}
        & = \sum_{\exrandindin} \condprobof{\indexedexrandom}{\indexedsecexrandom} \\
        &= \frac{\sum_{\exrandindin}\probat{\indexedexrandom,\indexedsecexrandom}}{\sum_{\exrandindin}\probat{\indexedexrandom,\indexedsecexrandom}} \\
        &= 1 \, ,
    \end{align*}
    and therefore each slice is normed.
    We can visualize this calculation exploiting our diagramatic notation as
    \begin{center}
        \begin{tikzpicture}[scale=0.3,thick] % , baseline = -3.5pt

\node[anchor=center] (text) at (-30,-2) {\small $\sum_{\atomlegindexof{\exrandom}} \, \condprobof{X=\atomlegindexof{\exrandom}}{Y=\atomlegindexof{\secexrandom}} \quad {=}$};

\draw (-21,-1) rectangle (-15,-3);
\node[anchor=center] (text) at (-18,-2) {\small $\condprobof{X}{Y}$};
\draw[->]  (-20,-3)--(-20,-5) node[midway,left] {\tiny $X$}; 

\draw[<-]  (-16,-3)--(-16,-5) node[midway,left] {\tiny $Y$}; 
\draw[] (-15,-5) rectangle (-17,-7); 
\node[anchor=center] (text) at (-16,-6) {\small $\onehotmapof{\catindexof{Y}}$};

\draw (-21,-5) rectangle (-19,-7); 
\node[anchor=center] (text) at (-20,-6) {$\ones$};

\node[anchor=center] (text) at (-13,-2) {${=}$};


\begin{scope}[shift={(0,6)}]

\draw (-11,-1) rectangle (-5,-3);
\node[anchor=center] (text) at (-8,-2) {\small $\probof{X,Y}$};
\draw[->]  (-10,-3)--(-10,-5) node[midway,left] {\tiny $X$}; 
\draw (-11,-5) rectangle (-9,-7); 
\node[anchor=center] (text) at (-10,-6) {$\ones$};
\draw[->]  (-6,-3)--(-6,-5) node[midway,left] {\tiny $Y$};
\draw[] (-7,-5) rectangle (-5,-7); 
\node[anchor=center] (text) at (-6,-6) {\small $\onehotmapof{\catindexof{Y}}$};

\end{scope}

\draw (-12,-2) -- (-4,-2);

\begin{scope}[shift={(0,-2)}]

\draw (-11,-1) rectangle (-5,-3);
\node[anchor=center] (text) at (-8,-2) {\small $\probof{X,Y}$};
\draw[->]  (-10,-3)--(-10,-5) node[midway,left] {\tiny $X$}; 
\draw (-11,-5) rectangle (-9,-7); 
\node[anchor=center] (text) at (-10,-6) {$\ones$};
\draw[->]  (-6,-3)--(-6,-5) node[midway,left] {\tiny $Y$};
\draw[] (-7,-5) rectangle (-5,-7); 
\node[anchor=center] (text) at (-6,-6) {\small $\onehotmapof{\catindexof{Y}}$};

\end{scope}

%\node[anchor=center] (text) at (-3,-2) {${=}$};
%
%\draw (-1,-3) rectangle (1,-1); 
%\node[anchor=center] (text) at (0,-2) {$\ones$};
%\draw[<-]  (0,-3)--(0,-5) node[midway,left] {\tiny $Y$};
%\draw[] (-1,-5) rectangle (1,-7); 
%\node[anchor=center] (text) at (0,-6) {\small $\onehotmapof{\catindexof{Y}}$};

\node[anchor=center] (text) at (-1,-2) {${=}\quad 1 \, .$};

%\node[anchor=center] (text) at (9,-7) {${.}$};

\end{tikzpicture}
    \end{center}
    Since for any $\secexrandindin$ the slice $\condprobof{\exrandom}{\indexedsecexrandom}$ is non-negative and contracts to $1$, we conclude that it is a probability distribution.
\end{proof}

We further show, that exactly the directed tensors with non-negative coordinates are conditional probability tensors.

\begin{theorem}
    \label{the:conditionalDirected}
    Any tensor with non-negative coordinates is a conditional distribution tensor, if and only if it is directed with the condition variables incoming and the other outgoing.
\end{theorem}
\begin{proof}
    \proofrightsymbol:
    By \theref{the:conditionalContraction} a conditional probability tensor $\condprobof{\exrandom}{\secexrandom}$ is the normalization of a tensor and by \theref{the:normalizationDirected} a directed tensor.
    Since probability tensors have only non-negative coordinates, their contractions with one-hot encodings also have only non-negative coordinates and also their normalizations.

    \proofleftsymbol:
    Conversely, let $\hypercoreat{\nodevariables}$ be a directed tensor with $\innodes$ incoming and $\outnodes$ outgoing and non-negative coordinates.
    Then
    \begin{align}
        \probat{\nodevariables} = \frac{1}{\prod_{\node\in\innodes}\catdimof{\node}} \cdot \hypercoreat{\nodevariables}
    \end{align}
    is a probability tensor, since
    \begin{align*}
        \sum_{\atomlegindexof{\innodes}} \sum_{\atomlegindexof{\outnodes}} \probat{\indexedcatvariableof{\nodes}} =
        \sum_{\atomlegindexof{\innodes}} \sum_{\atomlegindexof{\outnodes}} \frac{1}{\prod_{\node\in\innodes}\catdimof{\node}} \cdot \hypercoreat{\indexedcatvariableof{\nodes}} =
        \sum_{\atomlegindexof{\innodes}} \frac{1}{\prod_{\node\in\innodes}\catdimof{\node}} = 1 \, .
    \end{align*}
    The conditional probability $\condprobof{\catvariableof{\outnodes}}{\catvariableof{\innodes}}$ coincides with $\hypercore$, since
    \begin{align*}
        \condprobof{\catvariableof{\outnodes}}{\indexedcatvariableof{\innodes}}
        =& \frac{
            \probat{\catvariableof{\outnodes},\indexedcatvariableof{\innodes}}
        }{
            \sum_{\catindexof{\outnodes}} \probat{\indexedcatvariableof{\outnodes},\indexedcatvariableof{\innodes}}
        } \\
        =& \frac{
            \hypercoreat{\catvariableof{\outnodes},\indexedcatvariableof{\innodes}}
        }{
            \sum_{\catindexof{\outnodes}} \hypercoreat{\indexedcatvariableof{\outnodes},\indexedcatvariableof{\innodes}}
        }
        = \hypercoreat{\catvariableof{\outnodes},\indexedcatvariableof{\innodes}} \, ,
    \end{align*}
    where in the last equation we used that the denominator is by definition trivial since $\hypercore$ is normed.
\end{proof}

%
\theref{the:conditionalDirected} specifies a broad class of tensors to represent conditional probabilities.
In combination with \theref{the:bencodingDirected}, which states that basis encodings are directed, we get that any basis encoding of a function is a conditional probability tensor.


\subsect{Bayes Theorem and the Chain Rule}

So far, we have connected concepts of probability theory such as marginal and conditional probabilities with contractions and normalizations of tensors.
We will now proceed to show that basic theorems of probability theory translate into more general contraction equations.

\begin{theorem}[Bayes Theorem]
    \label{the:bayes}
    For any probability distribution $\probat{\exrandom, \secexrandom}$ with positive $\probat{\secexrandom}$ we have
    \begin{align*}
        \probat{\exrandom,\secexrandom}
        = \contractionof{\condprobof{\exrandom}{\secexrandom},\probat{\secexrandom}}{\exrandom,\secexrandom} \, .
    \end{align*}
\end{theorem}
\begin{proof}
    This theorem follows from the more generic contraction equation \theref{the:normalizationContractionEQ} to be shown in \charef{cha:coordinateCalculus}.
    We note that by positivity of $\probat{\secexrandom}$, the tensor network $\probtensor$ is normable with respect to $\secexrandom$.
    \theref{the:normalizationContractionEQ} therefore implies choosing $\nodes=\{0,1\}$, $\innodes=\{1\}$ and $\outnodes=\{0\}$, that
    For our tensor
    \begin{align*}
        \probat{\exrandom,\secexrandom}
        &=\contractionof{
            \normalizationofwrt{\probat{\exrandom,\secexrandom}}{\exrandom}{\secexrandom},
            \contractionof{\probat{\exrandom,\secexrandom}}{\secexrandom}
        }{\exrandom,\secexrandom} \\
        &=\contractionof{\condprobof{\exrandom}{\secexrandom},\probat{\secexrandom}}{\exrandom,\secexrandom} \, .
    \end{align*}
\end{proof}

Following the insight of the Bayes \theref{the:bayes}, probability distributions of arbitrary numbers of variables can be decomposed as a contraction of conditional probabilities, as we show in the next theorem.

\begin{theorem}[Chain Rule]
    \label{the:chainRule}
    For any probability distribution $\probwith$ we have
    \begin{align*}
        \probat{\shortcatvariables}
        = \contractionof{\{\margprobat{\catvariableof{0}}\} \cup
        \left\{\condprobof{\catvariableof{\catenumerator}}{\catvariableof{0},\ldots,\catvariableof{\catenumerator-1}}\,:\,\catenumeratorin, \, \catenumerator\geq 1\right\}
        }{\shortcatvariables} \, ,
    \end{align*}
    provided that all conditional probability distributions exist.
%    where for $\catenumerator=0$ we denote by $ \condprobof{\catvariableof{0}}{\catvariableof{0},\ldots,\catvariableof{-1}}$ the marginal distribution $\probat{\catvariableof{0}}$.
\end{theorem}
\begin{proof}
    The claim can be derived by an iterative application of the Bayes \theref{the:bayes} theorem.
    We will proof this statement in more generality in \charef{cha:coordinateCalculus} as \theref{the:genericChainRule}, deducing it from the generalization of the Bayes \theref{the:bayes} by \theref{the:normalizationContractionEQ}.
    The claim here then follows from \theref{the:genericChainRule} using $\nodes=[\catorder]$ and $\hypercoreat{\nodevariables}=\probwith$, since
    \begin{align*}
        \probwith
        &= \contractionof{\{\margprobat{\catvariableof{0}}\} \cup
        \{\normalizationofwrt{\probwith}{\catvariableof{\catenumerator}}{\catvariableof{0},\ldots,\catvariableof{\catenumerator-1}}  \, : \catenumeratorin, \, \catenumerator\geq 1\}
        }{\shortcatvariables} \\
        &  = \contractionof{\{\margprobat{\catvariableof{0}}\} \cup
        \left\{\condprobof{\catvariableof{\catenumerator}}{\catvariableof{0},\ldots,\catvariableof{\catenumerator-1}}\,:\,\catenumeratorin, \, \catenumerator\geq 1\right\}
        }{\shortcatvariables} \, ,
    \end{align*}
\end{proof}

We observe, that the chain rule provides a generic decomposition scheme of probability distributions into conditional distributions.
The conditional distribution to $\atomenumerator=\atomorder-1$, which appears in the chain decomposition, is in the same tensor space as the decomposed distribution $\probat{\shortcatvariables}$.
To achieve our main goal of tensor network decompositions, which is an efficient storage format of the decomposed tensor, we need to further sparsify the appearing conditional probabilities (to be more precise, we aim at basis+ $\cpformat$ decompositions, to be introduced in \charef{cha:sparseCalculus}).
These simplification require additional assumptions on the distribution, which we will introduce in the next section.

\subsect{Independence}

Independence leads to severe sparsifications of conditional probabilities and is therefore the key assumption to gain sparse decompositions of probability distributions.
Before showing such decomposition schemes, we first provide a coordinatewise definition of independent variables.

\begin{definition}[Independence]
    \label{def:independence}
    We say that $\exrandom$ is independent of $\secexrandom$ with respect to a distribution $\probat{\exrandom,\secexrandom}$, if for any values $\exrandindin$ and $\secexrandind$ the distribution satisfies
    \begin{align*}
        \probat{\indexedexrandom,\indexedsecexrandom}
        = \margprobat{\indexedexrandom}\cdot\margprobat{\indexedexrandom} \, .
    \end{align*}
    In this case we denote $\independent{\exrandom}{\secexrandom}$.
\end{definition}

We state next an equivalent independence criterion based on a contraction equation of probability distributions.
%Let us state the most direct equivalent in the next theorem and derive further criteria as corrolaries.
%We give a criterion on independence based on a contraction equation of the probability distribution in the next theorem.

\begin{theorem}[Independence Criterion as a Contraction Equation]
    \label{the:independenceProductCriterion}
    The variable $\exrandom$ is independent from $\secexrandom$ with respect to a probability distribution $\probat{\exrandom,\secexrandom}$, if and only if
    \begin{align*}
        \probat{\exrandom,\secexrandom}
        = \contractionof{\contractionof{\probat{\exrandom,\secexrandom}}{\exrandom},\contractionof{\probat{\exrandom,\secexrandom}}{\secexrandom}}{\exrandom,\secexrandom} \, .
    \end{align*}
\end{theorem}
\begin{proof}
    By \theref{the:marginalContraction} we know that marginal probabilities are equivalent to contracted probability distributions, i.e. $\probat{\exrandom} = \contractionof{\{\probtensor\}}{\exrandom} $.
    By orthogonality of one-hot encodings we have that
    \begin{align*}
        \forall \exrandind, \secexrandind : \quad  \probat{\indexedexrandom,\indexedsecexrandom}
        = \margprobat{\indexedexrandom}
        \cdot
        \margprobat{\indexedexrandom}
    \end{align*}
    is equivalent to
    \begin{align*}
        \sum_{\exrandind}\sum_{\secexrandind} \probat{\indexedexrandom,\indexedsecexrandom} \cdot \onehotmapofat{\exrandind}{\exrandom}\onehotmapofat{\secexrandind}{\secexrandom}
        = \sum_{\exrandind}\sum_{\secexrandind}
        \margprobat{\indexedexrandom}
        \cdot
        \margprobat{\indexedexrandom} \cdot \onehotmapofat{\exrandind}{\exrandom}\onehotmapofat{\secexrandind}{\secexrandom} \, .
    \end{align*}
    We reorder the summations and arrive at
    \begin{align*}
        \sum_{\exrandind,\secexrandind}
        \probat{\indexedexrandom,\indexedsecexrandom} \cdot \onehotmapofat{\exrandind,\secexrandind}{\exrandom, \secexrandom}
        = \left(\sum_{\exrandind}\margprobat{\indexedexrandom} \onehotmapofat{\exrandind}{\exrandom} \right)
        \cdot
        \left( \sum_{\secexrandind}  \margprobat{\indexedexrandom} \cdot \onehotmapofat{\secexrandind}{\secexrandom}  \right)
    \end{align*}
    which is by \lemref{lem:tensorBasisDecomposition} equal to the claim
    \begin{align*}
        \probat{\exrandom,\secexrandom} = \contractionof{\contractionof{\probtensor}{\exrandom},\contractionof{\probtensor}{\secexrandom}}{\exrandom,\secexrandom} \, .
    \end{align*}
\end{proof}

% Usage for tensor decompositions
Two jointly distributed variables are by \theref{the:independenceProductCriterion} independent, if and only if their joint distribution $\probat{\exrandom,\secexrandom}$ is the tensor product of marginal probabilities.
Using tensor network diagrams we depict this property by
\begin{center}
    \begin{tikzpicture}[scale=0.3,thick] % , baseline = -3.5pt


\draw (0,1) rectangle (7,-1);
\node[anchor=center] (text) at (3.5,0) {\small $\probat{\exrandom,\secexrandom}$};
\draw[->] (1,-1) -- (1,-3) node[midway, left] {\tiny $\exrandom$};
\draw[->] (6,-1) -- (6,-3) node[midway, left] {\tiny $\secexrandom$};

\node[anchor=center] (text) at (9,0) {\small ${=}$};


\begin{scope}[shift={(11,0)}]

\draw (0,1) rectangle (7,-1);
\node[anchor=center] (text) at (3.5,0) {\small $\probat{\exrandom,\secexrandom}$};
\draw[->] (1,-1) -- (1,-3) node[midway, left] {\tiny $\exrandom$};
\draw[->] (6,-1) -- (6,-3) node[midway, left] {\tiny $\secexrandom$};
\draw (5,-3) rectangle (7,-5);
\node[anchor=center] (text) at (6,-4) {\small $\ones$};

\end{scope}

\node[anchor=center] (text) at (20,0) {\small $\otimes$};

\begin{scope}[shift={(22,0)}]

\draw (0,1) rectangle (7,-1);
\node[anchor=center] (text) at (3.5,0) {\small $\probat{\exrandom,\secexrandom}$};
\draw[->] (1,-1) -- (1,-3) node[midway, left] {\tiny $\exrandom$};
\draw (0,-3) rectangle (2,-5);
\node[anchor=center] (text) at (1,-4) {\small $\ones$};
\draw[->] (6,-1) -- (6,-3) node[midway, left] {\tiny $\secexrandom$};

\end{scope}

\node[anchor=center] (text) at (31,0) {\small ${=}$};

\begin{scope}[shift={(33,0)}]

\draw (0,1) rectangle (4,-1);
\node[anchor=center] (text) at (2,0) {\small $\margprobat{\exrandom}$};
\draw[->] (2,-1) -- (2,-3) node[midway, left] {\tiny $\exrandom$};

\node[anchor=center] (text) at (6,0) {\small $\otimes$};

\draw (8,1) rectangle (12,-1);
\node[anchor=center] (text) at (10,0) {\small $\margprobat{\secexrandom}$};
\draw[->] (10,-1) -- (10,-3) node[midway, left] {\tiny $\secexrandom$};


\end{scope}

\node[anchor=center] (text) at (46,-3) {\small ${.}$};

\end{tikzpicture} 
\end{center}
Let us notice, that the assumption of independence reduces the degrees of freedom from $\exranddim\cdot\secexranddim-1$ to $(\exranddim-1)+(\secexranddim-1)$.
The decomposition by marginal distributions furthermore exploits this reduced freedom and provides an efficient storage.
Having a joint distribution of multiple variables, which disjoint subsets are independent, we can iteratively apply the decomposition scheme.
As a result, the degrees of freedom scaling exponential in the number of distributed variables would be reduced to a linear scaling, by the assumption of independence.

% Motivation of conditional independence as more realistic property
Independence is, as we observed, a strong assumption, which is often too restrictive.
It is furthermore an undesired property, when in a supervised learning scenario a target variable has to be predicted based on known feature variables.
Conditional independence instead is a less demanding assumption, which still implies efficient tensor network decompositions schemes.
We introduce conditional independence as independence of variables with respect to conditional distributions.

\begin{definition}[Conditional Independence]
    \label{def:condIndependence}
    Given a joint distribution of variables $\exrandom$, $\secexrandom$ and $\thirdexrandom$, such that $\margprobat{\thirdexrandom}$ is positive.
    We say that $\exrandom$ is independent of $\secexrandom$ conditioned on $\thirdexrandom$ if for any states $\exrandindin,\secexrandindin$ and $\thirdexrandindin$
    \begin{align*}
        \condprobof{\indexedexrandom,\indexedsecexrandom}{\indexedthirdexrandom}
        = \condprobof{\indexedexrandom}{\indexedthirdexrandom}
        \cdot \condprobof{\indexedsecexrandom}{\indexedthirdexrandom}   \, .
    \end{align*}
    In this case we denote $\condindependent{\exrandom}{\secexrandom}{\thirdexrandom}$.
\end{definition}

Conditional independence stated in \defref{def:condIndependence} has a close connection with independence stated in \defref{def:independence}.
To be more precise, $\exrandom$ is independent of $\secexrandom$ conditioned on $\thirdexrandom$, if and only if $\exrandom$ is independent of $\secexrandom$ with respect to any slice $\condprobof{\exrandom,\secexrandom}{\thirdexrandom=\thirdexrandind}$ of the conditional distribution $\condprobof{\exrandom,\secexrandom}{\thirdexrandom}$.
Analogously to \theref{the:independenceProductCriterion} for independence, we further find a decomposition criterion for conditional independence.
Since conditional independence can be regarded as a property of conditional probabilities, this decomposition criterion also involves conditional probabilities.

\begin{theorem}[Conditional Independence as a Contraction Equation]
    \label{the:condIndependenceProductCriterion}
    Given a distribution $\probtensor$ of variables $\exrandom$, $\secexrandom$ and $\thirdexrandom$, the variable $\exrandom$ is independent of $\secexrandom$ conditioned on $\thirdexrandom$, if and only if the equation
    \begin{align*}
        \condprobof{\exrandom,\secexrandom}{\thirdexrandom}
        = \contractionof{
            \condprobof{\exrandom}{\thirdexrandom},\condprobof{\secexrandom}{\thirdexrandom}
        }{\exrandom,\secexrandom,\thirdexrandom}
    \end{align*}
    holds.
\end{theorem}
\begin{proof}
    With the same argumentation as in the proof of \theref{the:independenceProductCriterion}, we notice that the contraction equation holds, if and only if for any $\exrandindin$, $\secexrandindin$ and $\thirdexrandindin$
    \begin{align*}
        \condprobof{\indexedexrandom,\indexedsecexrandom}{\indexedthirdexrandom}
        = \condprobof{\indexedexrandom}{\indexedthirdexrandom} \cdot \condprobof{\indexedsecexrandom}{\indexedthirdexrandom} \, .
    \end{align*}
    This is equivalent to conditional independence by \defref{def:condIndependence}.
\end{proof}

We can further exploit conditional independence to find tensor network decompositions of probabilities, as we show as the next corollary.
\begin{figure}[hbt!]
    \begin{center}
        \begin{tikzpicture}[scale=0.3,thick] % , baseline = -3.5pt


\draw (-2,1) rectangle (7,-1);
\node[anchor=center] (text) at (2.5,0) {\corelabelsize $\probat{\exrandom,\secexrandom,\thirdexrandom}$};
\draw[->-] (-1,-1) -- (-1,-3) node[midway, left] {\colorlabelsize $\exrandom$};
\draw[->-] (2.5,-1) -- (2.5,-3) node[midway, left] {\colorlabelsize $\secexrandom$};
\draw[->-] (6,-1) -- (6,-3) node[midway, left] {\colorlabelsize $\thirdexrandom$};

\node[anchor=center] (text) at (9,0) {\corelabelsize ${=}$};

\draw (11,1) rectangle (18,-1);
\node[anchor=center] (text) at (14.5,0) {\corelabelsize $\condprobof{\exrandom}{\thirdexrandom}$};
\draw[->-] (12,-1) -- (12,-3) node[midway, left] {\colorlabelsize $\exrandom$};
\draw[-<-] (17,-1) -- (17,-3) node[midway, left] {\colorlabelsize $\thirdexrandom$};

\draw (21,1) rectangle (25,-1);
\node[anchor=center] (text) at (23,0) {\corelabelsize $\probat{\thirdexrandom}$};
\draw (23,-1) -- (23,-3) node[midway, left] {\colorlabelsize $\thirdexrandom$};

\draw (23,-3) -- (23,-5);
\draw[fill] (23,-5) circle (\dotsize);
\draw[->-] (23,-5) -- (23,-7) node[midway, left] {\colorlabelsize $\thirdexrandom$};
\draw (17,-3) to[bend right=40] (23,-5);
\draw (29,-3) to[bend right=-40] (23,-5);


\draw (28,1) rectangle (35,-1);
\node[anchor=center] (text) at (31.5,0) {\corelabelsize $\condprobof{\secexrandom}{\thirdexrandom}$};
\draw[-<-] (29,-1) -- (29,-3) node[midway, left] {\colorlabelsize $\thirdexrandom$};
\draw[->-] (34,-1) -- (34,-3) node[midway, left] {\colorlabelsize $\secexrandom$};



\end{tikzpicture} 
    \end{center}
    \caption{Diagrammatic visualization of the contraction equation in \corref{cor:secCriterionCondIndepencence}.
    Conditional independence of $\exrandom$ and $\secexrandom$ given $\thirdexrandom$ holds if the contraction on the right ride is equal to the probability tensor on the left side.}
    \label{fig:condIndependenceDecomposition}
\end{figure}


\begin{corollary}
    \label{cor:secCriterionCondIndepencence}
    %Let $\probat{\exrandom,\secexrandom,\thirdexrandom}$ be a distribution, such that
    If and only if $\exrandom$ is independent of $\secexrandom$ conditioned on $\thirdexrandom$ the probability distribution $\probtensor$ satisfies (see \figref{fig:condIndependenceDecomposition})
    \begin{align*}
        \probat{\exrandom,\secexrandom,\thirdexrandom}
        = \contractionof{\condprobof{\exrandom}{\thirdexrandom},\condprobof{\secexrandom}{\thirdexrandom},\margprobat{\thirdexrandom}}{\exrandom,\secexrandom,\thirdexrandom} \, .
    \end{align*}
\end{corollary}
\begin{proof}
    With the Bayes \theref{the:bayes} it holds that
    \begin{align*}
        \probat{\exrandom,\secexrandom,\thirdexrandom}
        = \contractionof{\condprobof{\exrandom,\secexrandom}{\thirdexrandom},\margprobat{\thirdexrandom}}{\exrandom,\secexrandom,\thirdexrandom} \, .
    \end{align*}
    Decomposing the first tensor in the contraction, \theref{the:condIndependenceProductCriterion} implies, that $\exrandom$ is independent of $\secexrandom$ conditioned on $\thirdexrandom$, if and only if
    \begin{align*}
        \probat{\exrandom,\secexrandom,\thirdexrandom}
        = \contractionof{\condprobof{\exrandom}{\thirdexrandom},\condprobof{\secexrandom}{\thirdexrandom},\margprobat{\thirdexrandom}}{\exrandom,\secexrandom,\thirdexrandom} \, .
    \end{align*}
\end{proof}


Let us now recall our motivation of the study of conditional independence, namely to find sparsifications of conditional probabilities as those appearing in chain decompositions \theref{the:chainRule}.
As we state as the next theorem, such sparsifications follow from conditional independence.
%Conditional independence can be exploited in the sparsification of conditional probabilities.

\begin{theorem}
    \label{the:conditionDropping}
    Whenever $\exrandom$ is independent of $\secexrandom$ given $\thirdexrandom$, we have for any $\secexrandindin$
    \begin{align*}
        \condprobof{\exrandom}{\indexedsecexrandom,\thirdexrandom}
        = \condprobof{\exrandom}{\thirdexrandom} \, .
    \end{align*}
\end{theorem}
\begin{proof}
    By the Bayes \theref{the:bayes} we have for any indices to the variables
    \begin{align*}
        \condprobof{\indexedexrandom}{\indexedsecexrandom,\indexedthirdexrandom}
        = \frac{\condprobof{\indexedexrandom,\indexedsecexrandom}{\indexedthirdexrandom}}{
            \contraction{\condprobof{\exrandom,\indexedsecexrandom}{\indexedthirdexrandom}}
        }
    \end{align*}
    If $\exrandom$ is independent of $\secexrandom$ given $\thirdexrandom$ it follows that
    \begin{align*}
        \condprobof{\indexedexrandom}{\indexedsecexrandom,\indexedthirdexrandom}
        & = \frac{\condprobof{\indexedexrandom}{\indexedthirdexrandom} \cdot \condprobof{\indexedsecexrandom}{\indexedthirdexrandom}}{
            \contraction{\condprobof{\exrandom,\indexedsecexrandom}{\indexedthirdexrandom}}
        } \\
        & = \condprobof{\indexedexrandom}{\indexedthirdexrandom} \, .
    \end{align*}
\end{proof}

Following our motivation of sparse decompositions, we now combine this result with the generic chain rule, to show Markov Chain decompositions.
\begin{figure}[h]
    \begin{center}
        \begin{tikzpicture}[scale=0.3,thick] % , baseline = -3.5pt

\node[anchor=center] (text) at (-1,3) {${a)}$};

	\node [circle, draw, thick, fill=gray!50] (T1) at (0,0) {\tiny $\randomxof{0}$};
	\node [circle, draw, thick, fill=gray!50] (T2) at (5,0) {\tiny $\randomxof{1}$};
	\draw[->-] (T1) -- (T2);
	\node [circle, draw, thick, fill=gray!50] (T3) at (10,0) {\tiny $\randomxof{2}$};
	\draw[->-] (T2) -- (T3);
	\node [circle, draw, thick, fill=gray!50] (T4) at (15,0) {\tiny $\randomxof{3}$};
	\draw[->-] (T3) -- (T4);
	\draw[->-] (T4) -- (18,0);

	\node[anchor=center] (text) at (19,0) {$\cdots$};

	%\node [circle, draw, thick, fill=gray!50] (T4) at (17,0) {\tiny $\randomxof{\atomorder}$};
	%\draw[->-] (14,0) -- (T4);
			

\begin{scope}[shift={(25,0)}]

\node[anchor=center] (text) at (-3,3) {${b)}$};

\draw (-3.5,-1) rectangle (0, 1);
\node[anchor=center] (text) at (-1.75,0) {\small $\probat{\randomxof{0}}$};
\draw[->-] (0,0) -- (2,0);
\draw[fill] (1,0) circle (0.15cm);
\draw[->-] (1,0) -- (1,2) node[above] {\tiny $\catvariableof{0}$};
\draw (2,-1) rectangle (7, 1);
\node[anchor=center] (text) at (4.5,0) {\small $\condprobof{\randomxof{1}}{\randomxof{0}}$};
\draw[->-]  (7,0) -- (9,0);
\draw[fill] (8,0) circle (0.15cm);
\draw[->-] (8,0) -- (8,2) node[above] {\tiny $\catvariableof{1}$};
\draw (9,-1) rectangle (14, 1);
\node[anchor=center] (text) at (11.5,0) {\small $\condprobof{\randomxof{2}}{\randomxof{1}}$};
\draw[->-]  (14,0) -- (16,0);
\draw[fill] (15,0) circle (0.15cm);
\draw[->-] (15,0) -- (15,2) node[above] {\tiny $\catvariableof{2}$};
\draw (16,-1) rectangle (21, 1);
\node[anchor=center] (text) at (18.5,0) {\small $\condprobof{\randomxof{3}}{\randomxof{2}}$};
\draw[->-]  (21,0) -- (23,0);
\draw[fill] (22,0) circle (0.15cm);
\draw[->-] (22,0) -- (22,2) node[above] {\tiny $\catvariableof{3}$};
\node[anchor=center] (text) at (24,0) {$\cdots$};


\end{scope}

\end{tikzpicture} 
    \end{center}
    \caption{Depiction of a Markov Chain Decomposition by a
    a) hypergraph with the nodes $\nodes=[\catorder]$ and edges $\edges=\big\{\{0\}\cup\{\{\catenumerator,\catenumerator+1\} \, : \, \catenumeratorin, \, \catenumerator>1\} \big\}$ and
    b) a decorating Tensor Network representing the sparsified conditional probabilities.}
    \label{fig:MC}
\end{figure}

% More of an example?
\begin{theorem}[Markov Chain]
    \label{the:MarkovChain}
    Let there be a set of variables $\catvariableof{\catenumerator}$ where $\catenumeratorin$, and let us denote for $\catenumeratorin$ by $\catvariableof{[\catenumerator]}$ the collection of variables $\catvariableof{0},\ldots,\catvariableof{\catenumerator-1}$.
    Let us assume, that for any $\catenumeratorin$ with $\catenumerator\geq2$ the variable $\catvariableof{\catenumerator}$ is independent of $\catvariableof{[\catenumerator-1]}$ conditioned on $\catvariableof{\catenumerator-1}$, then
    \begin{align*}
        \probwith
        = \contractionof{\{\margprobat{\catvariableof{0}}\} \cup \{\condprobof{\catvariableof{\catenumerator}}{\catvariableof{\catenumerator-1}}\,:\,\catenumeratorin, \, \catenumerator\geq1\}}{\shortcatvariables}
    \end{align*}
%    Here we denote by $\condprobof{\catvariableof{0}}{\catvariableof{-1}}$ the marginal distribution of $\margprobat{\catvariableof{0}}$.
    We depict this decomposition in \figref{fig:MC}.
\end{theorem}
\begin{proof}
    By the chain rule shown in \theref{the:chainRule} we have
    \begin{align*}
        \probwith
        = \contractionof{
            \{ \condprobof{\catvariableof{\catenumerator}}{\catvariableof{[\catenumerator]}} \,:\, \catenumeratorin \}
        }{\shortcatvariables}
    \end{align*}
    Using that $\catvariableof{\catenumerator}$ is conditional independent of $\catvariableof{[\catenumerator-1]}$ conditioned on $\catvariableof{\catenumerator-1}$ we further have by \theref{the:conditionDropping}
    \begin{align*}
        \condprobof{\catvariableof{\catenumerator}}{\catvariableof{[\catenumerator]}}
        = \condprobof{\catvariableof{\catenumerator}}{\catvariableof{\catenumerator-1}} \otimes \onesat{\catvariableof{[\catenumerator-1]}} \, .
    \end{align*}
    Composing both equalities and omitting the trivial tensors shows the claim.
\end{proof}

% Markov Property
The assumption of $\catvariableof{\catenumerator}$ being independent of $\catvariableof{[\catenumerator-1]}$ conditioned on $\catvariableof{\catenumerator-1}$ is called the Markov property and the corresponding collection of random variables is called a Markov Chain.
\theref{the:MarkovChain} states an efficient decomposition of the probability distribution into a concatenated product of matrices representing conditional probability distributions.
Marginal distributions of Markov Chains can therefore consecutively be computed by matrix-vector products, that is for $\catenumeratorin$ with $\catenumerator\geq1$
\begin{align*}
    \margprobat{\catvariableof{\catenumerator}}
    = \contractionof{\condprobat{\catvariableof{\catenumerator}}{\catvariableof{\catenumerator-1}},\margprobat{\catvariableof{\catenumerator-1}}}{\catvariableof{\catenumerator}} \, .
\end{align*}
The conditional probability matrices are therefore called stochastic transition matrices.

We notice, that the decomposition scheme of \theref{the:MarkovChain} hints at an efficient representation of $\probwith$ based on transition matrices.
While $\probwith$ is a tensor in a space of dimension
\begin{align*}
    \prod_{\catenumeratorin} \catdimof{\catenumerator} \, ,
\end{align*}
the sum of the dimension of the transition matrices is
\begin{align*}
    \catdimof{0} + \sum_{\catenumeratorin,\, \catenumerator\geq 1} \catdimof{\catenumerator}\cdot \catdimof{\catenumerator-1} \, .
\end{align*}
We therefore observe a linear increase of the storage demand of the transition matrices in the order $\catorder$, whereas a naive storage of $\probwith$ by its coordinates would have an exponentially demand.

% Generalizing Markov Chain
The Markov Chain serves as a toy example drawing on a restrictive chain arrangement of conditional independencies.
In the following section, we will investigate decomposition schemes, which relax this assumption and draw on more general collections of conditional independencies.
The computation of marginal distribution by consecutive transition matrix multiplications will then be replaced by more general tensor network contractions.



\sect{Sufficient Statistics and Exponential Families}\label{sec:exponentialFamilies}

\red{We have seen, that conditional independence of variables corresponds with decomposition properties of probability tensors.
Another mechanism is through sufficient statistics, which leads to exponential families.
When restricting to graphical models in the next section, we will see that both mechanisms are related through the Hammersley-Clifford theorem.}

\subsect{Sufficient Statistics}

Let us consider a tuple of random variables $\shortcatvariables$, which take values in $\facstates$.
We now understand the probability $\probat{\shortcatvariables}$ as another random variable taking values in $[0,1]$, which has a deterministic dependence on $\shortcatvariables$.

\begin{definition}[Sufficient Statistics]
    Let $\shortcatvariables$ be a tuple of by $\probat{\shortcatvariables}$ jointly distributed random variables and $\sstatat{\shortcatvariables,\selvariable}$ be a tensor.
    We consider the tuple of random variables $(\shortcatvariables,\probat{\shortcatvariables},\sstatat{\shortcatvariables,\selvariable})$, which takes for $\shortcatindices\in\facstates$ with probability $\probat{\indexedshortcatvariables}$ the value
    \begin{align*}
        \left(\shortcatindices,\probat{\indexedshortcatvariables},\sstatat{\indexedshortcatvariables,\selvariable}\right) \, .
    \end{align*}
    We say, that $\sstat$ is a sufficient statistic for $\probtensor$, if this tuple obeys
    \begin{align*}
        \condindependent{\shortcatvariables}{\probat{\shortcatvariables}}{\sstatat{\shortcatvariables,\selvariable}} \, .
    \end{align*}
\end{definition}

%In the notation of this reads as
%\begin{align*}
%    \shortcatvariables\rightarrow\probat{\shortcatvariables} \, .
%\end{align*}

%A sufficient statistic is another random variables $\sstatat{\shortcatvariables,\selvariable}$ taking values in $\rr^{\seldim}$, such that $\shortcatvariables$ is independent of $\probat{\shortcatvariables}$ conditioned on $\sstatat{\shortcatvariables,\selvariable}$.
%We then have a Markov Chain (see \cite{cover_elements_2006})
%\begin{align*}
%    \shortcatvariables \rightarrow \sstatat{\shortcatvariables} \rightarrow \probat{\shortcatvariables} \, .
%\end{align*}


\begin{theorem}\label{the:sufficientStatisticActCoreExistence}
    If and only if $\sstatat{\shortcatvariables,\selvariable}$ is a sufficient statistic, i.e. $\shortcatvariables$ is independent of $\probat{\shortcatvariables}$ conditioned on $\sstatat{\shortcatvariables,\selvariable}$, there is a tensor $\actcoreat{\headvariableof{[\seldim]}}$ with
    \begin{align*}
        \probat{\shortcatvariables} = \contractionof{\bencodingofat{\sstat}{\headvariableof{[\seldim]},\shortcatvariables},\actcoreof{\headvariableof{[\seldim]}}}{\shortcatvariables} \, .
    \end{align*}
\end{theorem}
\begin{proof}
    We exploit conditional entropies, for which definition we refer to Chapter~2 in \cite{cover_elements_2006}. % Can we avoid that and directly refer?
    By the data processing inequality (see e.g. Theorem~2.8.1 in \cite{cover_elements_2006}), we have
    \begin{align*}
        \sentropyof{\probtensor|\sstat} \leq \sentropyof{\probtensor |\shortcatvariables} = 0
    \end{align*}
    and thus $\sentropyof{\probtensor|\sstat}=0$.
    Moreover, $\sentropyof{\probtensor|\sstat}=0$ is equivalent to a straight satisfaction of the data processing inequality, and $\sstat$ being a sufficient statistic for $\probtensor$.
    $\sentropyof{\probtensor|\sstat}=0$ is furhter equivalent to the existance of a function $\exfunction:\rr^{\seldim}\rightarrow[0,1]$, such that for each $\shortcatindices$
    \begin{align*}
        \exfunctionof{\sstatat{\indexedshortcatvariables,\selvariable}} = \probat{\indexedshortcatvariables} \, .
    \end{align*}
    For an index interpretation function $\indexinterpretation$, enumerating $\bigtimes_{\selindexin}\imageof{\sstatcoordinateof{\selindex}}$ using the variables $\headvariableof{[\seldim]}$, we define
    \begin{align*}
        \actcore := \exfunction \circ \indexinterpretation \, .
    \end{align*}
    Using basis calculus (see \charef{cha:basisCalculus}) and in particular \theref{the:tensorFunctionComposition} we have
    \begin{align*}
        \probat{\shortcatvariables} = \contractionof{\bencodingofat{\sstat}{\headvariableof{[\seldim]},\shortcatvariables},\actcoreof{\headvariableof{[\seldim]}}}{\shortcatvariables} \, .
    \end{align*}
\end{proof}

Given a statistic $\sstat$, we can thus characterize the set of probability distributions, for which $\sstat$ is sufficient, as
\begin{align*} % ! First occurance of realizabledists ?
    \realizabledistsof{\sstat,\maxgraph}
    := \left\{\normalizationof{\bencodingofat{\sstat}{\headvariableof{[\seldim]},\shortcatvariables},\actcoreof{\headvariableof{[\seldim]}}}{\shortcatvariables} \right\}
\end{align*}
%where by $\maxgraph$ we refer to the maximal graph to be explained later


% Usage of the selection encoding -> Can also make a theorem out of this

\subsect{Exponential families}

\red{\theref{the:sufficientStatisticActCoreExistence} states the existance of an activation core, once a sufficient vector statistic has been identified.
However, since the dimension of the activation core space is increasing exponential with the number of features (it is the product of the image cardinalities of the features), representation of generic $\actcore$ is not feasible.
We now restrict the activation cores to specific elementary tensors, which correspond with further assumptions on the dependence of $\probtensor$ and $\sstat$ made by exponential families.}

The probability distributions, which are members of an exponential family, share the computation of the probability tensor based on a boolean base measure, marking the support of the distribution, and a statistic function containing features.
They differ only by canonical parameters which weight the features at a given state to calculate the respective probability.
Exponential families consist the most generic distributions investigated in this work and will also serve as a generic framework in the discussion of probabilistic reasoning in \charef{cha:probReasoning}, as well as for neuro-symbolic models in \parref{par:two}.

%where each coordinate is determined by a base measure and a set $\sstat$ of features as
%\[ \probat{\indexedshortcatvariables}  \propto \basemeasure(\catindex) \cdot \expof{\sum_{\selindexin} \sstatcoordinateofat{\selindex}{\indexedshortcatvariables} \cdot \canparamat{\indexedselvariable}} \, . \]


\begin{definition}
    \label{def:expFamily}
    Given a statistic function
    \begin{align*}
        \sstat : \facstates \rightarrow \parspace
    \end{align*}
    and a boolean base measure
    \begin{align*}
        \basemeasure : \facstates \rightarrow \ozset
    \end{align*}
    with $\contraction{\basemeasure}\neq0$, the set $\expfamily=\{\expdist \, : \, \canparamwithin\}$ of probability distributions
    \begin{align*}
        \expdistat{\shortcatvariables} = \normalizationof{\expof{\contractionof{\sencsstatat{\shortcatvariables,\selvariable},\canparamwith}{\shortcatvariables},\basemeasurewith}}{\shortcatvariables}
    \end{align*}
    is called the exponential family to $\sstat$.
    We further define for each member with parameters $\canparam$ the associated energy tensor
    \begin{align*}
        \expenergy = \contractionof{\sencsstat,\canparam}{\shortcatvariables}
    \end{align*}
    and the cumulant function
    \[ \cumfunctionof{\canparam} = \lnof{\contraction{\basemeasure,\expof{\contractionof{\sencsstat,\canparam}{\shortcatvariables} }} } \, .\]
\end{definition}


We used the selection encoding to represent the weighted summation over the statistics, that is the tensor (see \defref{def:selectionEncoding})
\begin{align*}
    \sencsstatat{\shortcatvariables,\selvariable}: \facstates \times [\seldim] \rightarrow \rr
\end{align*}
defined for $\shortcatindices\in\facstates$ and $\selindexin$ as
\begin{align*}
    \sencsstatat{\indexedshortcatvariables,\indexedselvariable} = \sstatcoordinateofat{\selindex}{\indexedshortcatvariables} \, .
\end{align*}
The selection encoding represent the weighted sum of the statistic coordinates by the canonical parameter vector $\canparamwith$ as a contraction
\begin{align*}
    \sum_{\selindexin}\canparamat{\indexedselvariable}\cdot \sstatcoordinateofat{\selindex}{\shortcatvariables}
    = \contractionof{\sencsstatat{\shortcatvariables,\selvariable},\canparamwith}{\shortcatvariables} \, .
\end{align*}
For more details on this representation scheme, we refer to \theref{the:linCompSelEncoding} in \charef{cha:coordinateCalculus}.
Up to normalization, we sketch the probability distribution of any member by the tensor network diagram
\begin{center}
    \begin{tikzpicture}[scale=0.35,thick] % , baseline = -3.5pt

    \begin{scope}
        [shift={(-20,-8)}]

        \draw (-6,1) rectangle (10, 4);
        \node[anchor=center] (text) at (2,2.5) {\small $\contractionof{\expof{\sbcontractionof{\sencsstat,\canparam}{\shortcatvariables},\basemeasureat{\shortcatvariables}}}{\shortcatvariables}$};

        \draw[] (0,1)--(0,-1) node[midway,left] {\tiny $\catvariableof{0}$};
        \draw[] (0,-1)--(0,-1.5);
        \draw[] (1.5,1)--(1.5,-1) node[midway,left] {\tiny $\catvariableof{1}$};
        \draw[] (1.5,-1)--(1.5,-1.5);
        \node[anchor=center] (text) at (3,0) {$\cdots$};
        \draw[] (4,1)--(4,-1) node[midway,right] {\tiny $\catvariableof{\atomorder\shortminus1}$};
        \draw[] (4,-1)--(4,-1.5);

        \node[anchor=center] (text) at (12,1.5) {${=}$};
    \end{scope}

    \begin{scope}
        [shift={(0,-4)}]
        \draw[] (0,1)--(0,-6);
        \node[below] (text) at (0,-6) {\tiny $\catvariableof{\atomorder\shortminus1}$};
        \drawvariabledot{0}{-5}
        \draw[] (1.5,1)--(1.5,-6);
        \node[below] (text) at (1.5,-6) {\tiny $\catvariableof{1}$};
        \drawvariabledot{1.5}{-3}
        \node[anchor=center] (text) at (3,0) {$\cdots$};
        \node[anchor=center] (text) at (3,-5.5) {$\cdots$};
        \draw[] (4,1)--(4,-6);
        \node[below] (text) at (4,-6) {\tiny $\catvariableof{\atomorder\shortminus1}$};
        \drawvariabledot{4}{-2}

        \draw[] (0,-5) -- (6,-5);
        \draw[] (1.5,-3) -- (6,-3);
        \node[anchor=center] (text) at (5,-3.75) {$\vdots$};
        \draw[] (4,-2) -- (6,-2);
        \draw (6,-1) rectangle (9, -6);
        \node[anchor=center] (text) at (7.5,-3.5) {$\basemeasure$};

        \node[anchor=center] (text) at (10,-4.5) {$.$};

    \end{scope}

    \draw (-1,-3) rectangle (5, 0);
    \node[anchor=center] (text) at (2,-1.5) {$\sencsstat$};

    \draw[] (5,-2)--(7,-2) node[midway,below] {\tiny $\selvariable$};
    \draw (7,-3) rectangle (9, -1);
    \node[anchor=center] (text) at (8,-2) {$\canparam$};

    \node[anchor=center] (text) at (-3.5,-2) {\small $\mathrm{exp}$};
    \draw (2,-2) ellipse (8 and 2.75);

\end{tikzpicture}
\end{center}
We here denote by an ellipsis the coordinatewise transformation by the exponential function (see \secref{sec:coordinatewiseTransforms}).
Since such coordinatewise transformation are nonlinear, they are a caveat for efficient contraction of the diagram.

% Diverging partition functions avoided here
Since we restrict the discussion to finite state spaces, the distribution $\expdist$ is well-defined for any $\canparamwith\in\parspace$.
For infinite state space there are sufficient statistics and parameters, such that the partition function $\contraction{\basemeasure,\expof{\contractionof{\sencsstat,\canparam}{\shortcatvariables}}}$ diverges and the normalization $\expdist$ is not well-defined.
In that cases, the canonical parameters need to be chosen from a subset where the partition function is finite \cite{wainwright_graphical_2008}.

% Restriction to boolean base measures
As before, we restrict to boolean base measures, which have to satisfy $\contraction{\basemeasure}\neq0$ for respective distributions to exist.
We notice, that by positivity of the exponential function, any distribution in an exponential family $\expfamily$ is positive with respect to $\basemeasure$ (see \defref{def:positivityBaseMeasure}).
In \charef{cha:networkRepresentation} we will investigate distributions, where the base measures and the sufficient statistics share a common decomposition framework.

% Cumulant representation
\begin{lemma}
    \label{lem:energyCumulantRepresentation}
    For any member of an exponential family $\expfamily$ we have
    \begin{align*}
        \expdistat{\shortcatvariables}
        = \contractionof{\expof{ \expenergy - \cumfunctionof{\canparam}\cdot \onesat{\shortcatvariables}},\basemeasureof{\shortcatvariables}}{\shortcatvariables} \, .
    \end{align*}
\end{lemma}
\begin{proof}
    By definition we have
    \begin{align*}
        \expdistat{\shortcatvariables}
        &= \normalizationof{
            \expof{\contractionof{\sencsstat,\canparam}{\shortcatvariables}},\basemeasurewith
        }{\shortcatvariables} \\
        &= \frac{\contractionof{\expof{\contractionof{\sencsstat,\canparam}{\shortcatvariables},\basemeasurewith}}{\shortcatvariables}
        }{\contraction{\expof{\contractionof{\sencsstat,\canparam    }{\shortcatvariables}},\basemeasurewith}} \\
        &=  \frac{
            \contractionof{\expof{\expenergyat{\shortcatvariables}},\basemeasurewith}{\shortcatvariables}
        }{
            \expof{\cumfunctionof{\canparam}}
        } \\
        & = \contractionof{\expof{ \expenergy - \cumfunctionof{\canparam}\cdot \onesat{\shortcatvariables}},\basemeasureof{\shortcatvariables}}{\shortcatvariables} \, .
    \end{align*}
\end{proof}


% Minimal statistics
A further useful criterion is that of minimality of an exponential family, as we define next.

\begin{definition}[Minimal]
    \label{def:minimalStatistics}
    We say that a statistic $\sstat$ is minimal with respect to a boolean base measure $\basemeasure$, if there is no pair of a non-vanishing vector $\vectorat{\selvariable}$ and a scalar $\lambda\in\rr$ with
    \begin{align*}
        \contractionof{\sencsstatat{\shortcatvariables,\selvariable},\vectorat{\selvariable},\basemeasurewith}{\shortcatvariables} = \lambda\cdot\basemeasurewith \, .
    \end{align*}
\end{definition}

% Making a statistic minimal
If a statistic is not minimal, we can omit coordinates of it without affecting the expressivity $\expfamily$.
As long as we find a non-vanishing vector $\vectorat{\selvariable}$ and $\lambda\in\rr$ as in \defref{def:minimalStatistics}, we can choose a coordinate $\sstatcoordinateof{\selindex}$ such that $\vectorat{\indexedselvariable}\neq0$, conclude that the coordinate is linear dependent on the others and drop it as redundant.


\subsect{Tensor Network Representation}

As we have observed, the selection encoding formalism can efficiently represent the energy tensor to a member of an exponential family, but through coordinatewise transform by the exponential does not provide an efficient decomposition scheme of the probability distribution itself.
We now overcome this problem with usage of the basis encoding formalism to represent members of exponential families by a single contraction without nonlinear transforms.
%The central insight here is a basis encoding of the statistic , which enables representation by tensor network decomposition, when the sufficient statistic is decomposable.

\begin{theorem}[Generic Representation of Exponential Families]
    \label{the:expFamilyTensorRep}
    Given any base measure $\basemeasure$ and a sufficient statistic $\sstat$ we enumerate for each coordinate $\selindexin$ the image $\imageof{\sstatcoordinateof{\selindex}}$ by a variable $\sstatcatof{\selindex}$ taking values in $[\cardof{\imageof{\sstatcoordinateof{\selindex}}}]$ (see for more details on this scheme \charef{cha:basisCalculus}), given an interpretation map
    \begin{align*}
        \indexinterpretationof{\selindex} :
        [\cardof{\imageof{\sstatcoordinateof{\selindex}}}] \rightarrow \imageof{\sstatcoordinateof{\selindex}} \, .
    \end{align*}
    For any canonical parameter vector $\canparamwithin$ we build the activation cores
    \begin{align*}
        \actcoreofat{\selindex,\canparamat{\indexedselvariable}}{\indexedheadvariableof{\selindex}}
        = \expof{\canparamat{\indexedselvariable} \cdot \indexinterpretationofat{\selindex}{\headindex{\selindex}} } \,
    \end{align*}
    and have
    \begin{align*}
        \expdistwith =
        \normalizationof{\{\basemeasurewith\} \cup \{\bencodingofat{\sstatcoordinateof{\selindex}}{\sstatcatof{\selindex},\shortcatvariables} \, : \, \selindexin\}\cup\{\sstatac \, : \, \selindexin\}}{\shortcatvariables} \, .
    \end{align*}
\end{theorem}
\begin{proof}
    We embed the image of $\sstat$ in the cartesian product of the coordinate images  %	which does not modify the statement of Theorem~\ref{the:tensorFunctionComposition} (since extension to cases, which are never met).
    \begin{align*}
        \imageof{\sstat} \subset \bigtimes_{\selindexin} \imageof{\sstatcoordinateof{\selindex}} \,
    \end{align*}
    and design enumerate the embedded image of $\sstat$ by the variables $\sstatcatof{[\seldim]}$.
    \theref{the:tensorFunctionComposition}, to be shown in \charef{cha:basisCalculus}, implies
    \begin{align*}
        \expof{\contractionof{\sencsstat,\canparam}{\shortcatvariables}}
        = \contractionof{
            \bencodingofat{\sstat}{\sstatcatof{[\seldim]},\shortcatvariables},\restrictionofto{\expof{\braket{\cdot,\canparamwith}}
            }{\bigtimes_{\selindexin}\imageof{\sstatcoordinateof{\selindex}}}}{\shortcatvariables} \, .
    \end{align*}
    Here we denote by $\braket{\cdot,\canparamwith}$ the dual function to $\canparamwith$, which assigns to vectors their contraction with $\canparamwith$.
    Its restriction onto the vectors in $\bigtimes_{\selindexin}\imageof{\sstatcoordinateof{\selindex}}$ is the tensor satisfying
    \begin{align*}
        \restrictionoftoat{\expof{\braket{\cdot,\canparam}}}{\imageof{\sstat}}{\sstatcatof{[\seldim]}}
        = \bigotimes_{\selindexin} \restrictionoftoat{\expof{\cdot \canparamat{\indexedselvariable}}}{\imageof{\sstatcoordinateof{\selindex}}}{\sstatcatof{\selindex}}
        = \bigotimes_{\selindexin} \actcoreofat{\selindex,\canparamat{\indexedselvariable}}{\sstatcatof{\selindex}} \, .
    \end{align*}
    We further have (see \theref{the:functionImageDecompositionContraction} in \charef{cha:basisCalculus})
    \begin{align*}
        \bencodingofat{\sstat}{\sstatcatof{[\seldim]},\shortcatvariables}
        = \contractionof{\{\bencodingofat{\sstatcoordinateof{\selindex}}{\sstatcatof{\selindex},\shortcatvariables} \, : \, \selindexin\}}{\sstatcatof{[\seldim]},\shortcatvariables} \, .
    \end{align*}
    Refining the above decomposition of $\expof{\contractionof{\sencsstat,\canparam}{\shortcatvariables}}$ by these further decompositions we arrive at the claim.
\end{proof}


In the proof of \theref{the:expFamilyTensorRep} we have observed, that the basis encoding $\bencodingofat{\sstat}{\sstatcatof{[\seldim]},\shortcatvariables}$ of the statistics decomposed into a tensor network of basis encodings $\bencodingofat{\sstatcoordinateof{\selindex}}{\sstatcatof{\selindex},\shortcatvariables}$ to the coordinate of the statistic.
We can exploit further decomposition mechanisms, which will be discussed in full detail in \charef{cha:basisCalculus}, to find even sparser decompositions.
This is for example the case, when the coordinates of the statistic are compositions of functions depending on small numbers of variables.
When the coorindates of the statistic furthermore share similar parts in their compositions, these parts can be shared in the decomposition.
We will investigate such sparsification mechanisms in more detail in \charef{cha:networkRepresentation}, where the coordinates of the statistic are propositional formulas with a natural decomposition by their syntactical description.
%We used in the proof, that the basis encoding of the statistic decomposes into the basis encoding of the coordinate maps to the statistic as
%\begin{align*}
%    \bencodingofat{\sstat}{\sstatcatof{[\seldim]},\shortcatvariables}
%     = \contractionof{\{\bencodingofat{\sstatcoordinateof{\selindex}}{\sstatcatof{\selindex},\shortcatvariables} \, : \, \selindexin\}}{\sstatcatof{[\seldim]},\shortcatvariables} \, .
%\end{align*}
%One strategy to decompose $\bencodingof{\sstat}$ is thus by the .
%When the coordinate maps are sharing common components, a sparser representation can be derived through encodings of the components shared among the coordinate map encodings.


% Core types
The tensor network representation of an exponential family by \theref{the:expFamilyTensorRep} is a Markov Network consistent of two types of cores.
First, we refer to the basis encodings $\bencodingof{\sstatcoordinateof{\selindex}}$ of the coordinates of a statistic as computation cores.
Our intuition is that they compute the hidden variable $\sstatcatof{\selindex}$, based on Basis Calculus (see \charef{cha:basisCalculus}), which encode the value of the coordinate with respect to the image interpretation map $\indexinterpretationof{\selindex}$.
We notice, that since they are directed with $\sstatcatof{\selindex}$ being the only outgoing variable, they do not influence any contraction with open variables $\shortcatvariables$, unless further tensors sharing the variable $\sstatcatof{\selindex}$ are present in the contraction.
The influence of the contraction is performed by the activation cores $\actcoreofat{\selindex,\canparamat{\indexedselvariable}}{\sstatcatof{\selindex}}$, which exploit the computed statistic variable and provide in combination with the basis encoding a factor
\begin{align*}
    \contractionof{\bencodingofat{\sstatcoordinateof{\selindex}}{\sstatcatof{\selindex},\shortcatvariables},\actcoreofat{\selindex,\canparamat{\indexedselvariable}}{\sstatcatof{\selindex}}}{\shortcatvariables}
\end{align*}
to the Markov Network reduced to the observed variables $\shortcatvariables$.
When the canonical parameter is vanishing at a coordinate, that is $\canparamat{\indexedselvariable}=0$, then this factor is trivial, since $\actcoreofat{\selindex,0}{\sstatcatof{\selindex}}=\onesat{\sstatcatof{\selindex}}$ and as a consequence of the directionality of basis encodings we have
\begin{align*}
    \contractionof{\bencodingofat{\sstatcoordinateof{\selindex}}{\sstatcatof{\selindex},\shortcatvariables},\actcoreofat{\selindex,\canparamat{\indexedselvariable}}{\sstatcatof{\selindex}}}{\shortcatvariables}
    = \contractionof{\bencodingofat{\sstatcoordinateof{\selindex}}{\sstatcatof{\selindex},\shortcatvariables},\onesat{\sstatcatof{\selindex}}}{\shortcatvariables}
    = \onesat{\shortcatvariables} \, .
\end{align*}
In that case both the activation core and the corresponding computation core can be dropped from the network without changing its distribution.

% Interpretation as elementary
By \theref{the:expFamilyTensorRep} any member of an exponential family is represented by the normed contraction of a collection of unary activation cores contracted with the computation network $\bencodingofat{\sstat}{\headvariableof{[\seldim]},\shortcatvariables}$.
We understand these activation cores as a member of a simple Markov Network distributing the head variables $\headvariableof{[\seldim]}$.
This Markov Network has a graph, where the edges contain single variables, that is $\elgraph=([\seldim],\{\{\selindex\} \, : \, \selindexin\})$.
We call this graph the elementary graph, since it also corresponds with elementary tensor network formats consistent of tensor products of vectors.
A straightforward generalization of probability distributions representable by exponential families then allows for arbitrary decomposition formats for activation tensors, as we define next.

% Define sets of realizable distributions
\begin{definition}
    \label{def:realizableStatDistributions}
    Given a statistic $\sstat : \facstates \rightarrow \rr^{\seldim}$, and a hypergraph $\graph=([\seldim],\edges)$ with nodes associated to the coordinates of the statistic, we define the by $\sstat$ and $\graph$ computable family of distributions by
    \begin{align*}
        \realizabledistsof{\sstat,\graph}
        = \left\{ \normalizationof{\{\bencodingofat{\sstat}{\sstatcatof{[\seldim]},\shortcatvariables}\} \cup \{\hypercoreofat{\edge}{\sstatcatof{\edge}}\}}{\shortcatvariables}  \, : \,\hypercoreofat{\edge}{\sstatcatof{\edge}}\in\bigotimes_{\selindex\in\edge}\rr^{\headdimof{\selindex}}, \, \zerosat{\sstatcatof{\edge}}\prec\hypercoreofat{\edge}{\sstatcatof{\edge}} \right\} \, .
    \end{align*}
    Note that we restrict to non-negative activation cores by demanding $\zerosat{\sstatcatof{\edge}}\prec\hypercoreofat{\edge}{\sstatcatof{\edge}}$, a notation which will be introduced in more detail in \charef{cha:logicalReasoning} as partial order of tensors.
    We refer to any member $\probat{\shortcatvariables}\in\realizabledistsof{\sstat,\graph}$ as a by $\sstat$ and $\graph$ computable distribution.
\end{definition}

For unary activation cores, that is for the elementary graph $\elgraph$, any member of $\realizabledistsof{\sstat,\elgraph}$ has up to a normalization factor a tensor network decomposition by the diagram
\begin{center}
    \begin{tikzpicture}[scale=0.35,thick,xscale=1] % , baseline = -3.5pt

    \draw (-1.25,1) rectangle (1.25,3);
    \node[anchor=center] (text) at (0,2) {$\actcoreof{{0}}$};

    \draw (2.75,1) rectangle (5.25,3);
    \node[anchor=center] (text) at (4,2) {$\actcoreof{{\seccatorder\shortminus1}}$};

    \draw[->-] (0,-1)--(0,0);
    \node[left] (text) at (0,0) {\tiny $\headvariableof{0}$};
    \draw[] (0,0)--(0,1);
    \drawvariabledot{0}{0}
    \node[anchor=center] (text) at (2,0) {$\cdots$};

    \draw[->-] (4,-1)--(4,0);
    \node[right] (text) at (4,0) {\tiny $\headvariableof{\seccatorder\shortminus1}$};
    \draw[] (4,0)--(4,1);
    \drawvariabledot{4}{0}

    \draw (-1,-1) rectangle (5,-3);
    \node[anchor=center] (text) at (2,-2) {\small $\bencodingof{\sstat}$};
    \draw[-<-] (0,-3)--(0,-5) node[midway,left] {\tiny $\catvariableof{0}$};
    \draw[-<-] (1.5,-3)--(1.5,-5) node[midway,left] {\tiny $\catvariableof{1}$};
    \node[anchor=center] (text) at (3,-4) {$\cdots$};
    \draw[-<-] (4,-3)--(4,-5) node[midway,right] {\tiny $\catvariableof{\atomorder\shortminus1}$};

    \draw (-1,-1) rectangle (5,-3);
    \node[anchor=center] (text) at (2,-2) {\small $\bencodingof{\sstat}$};
    \draw[-<-] (0,-3)--(0,-5) node[midway,left] {\tiny $\catvariableof{0}$};
    \draw[-<-] (1.5,-3)--(1.5,-5) node[midway,left] {\tiny $\catvariableof{1}$};
    \node[anchor=center] (text) at (3,-4) {$\cdots$};
    \draw[-<-] (4,-3)--(4,-5) node[midway,right] {\tiny $\catvariableof{\atomorder\shortminus1}$};

    \node[anchor=center] (text) at (6,-4.5) {$.$};

%\drawatomcore{3.5}{-8}{$\probtensor$}
%\drawatomindices{3.5}{-12}	
%\draw (5.5,-9)--(5.5,-7) node[midway,right] {\tiny $\catvariableof{\exformula}$};

\end{tikzpicture}
\end{center}
Comparing this representation scheme with \theref{the:expFamilyTensorRep}, we conclude as the next corollary, that any member of an exponential family with trivial base measure can be represented by an elementary activation tensors.

\begin{corollary}[Corollary of \theref{the:expFamilyTensorRep}]
    \label{cor:unaryActivationExpdistRealization}
    For any statistic $\sstat:\facstates\rightarrow[\seldim]$ and trivial base measure $\basemeasurewith=\onesat{\shortcatvariables}$ we have
    \begin{align*}
        \expfamilyof{\sstat,\ones} \subset \realizabledistsof{\sstat,\elgraph} \, .
    \end{align*}
\end{corollary}

For elements of the exponential family with general boolean base measure we have with the activation cores constructed in \theref{the:expFamilyTensorRep}
\begin{center}
    \begin{tikzpicture}[scale=0.35,thick,xscale=1] % , baseline = -3.5pt

    \begin{scope}
        [shift={(-11,0)}]
        \draw (-2,-1) rectangle (6,-3);
        \node[anchor=center] (text) at (2,-2) {\small $\partitionfunctionof{\sstat,\canparam,\basemeasure} \cdot \expdist$};
        \draw[-<-] (0,-3)--(0,-5) node[midway,left] {\tiny $\catvariableof{0}$};
        \draw[-<-] (1.5,-3)--(1.5,-5) node[midway,left] {\tiny $\catvariableof{1}$};
        \node[anchor=center] (text) at (3,-4) {$\cdots$};
        \draw[-<-] (4,-3)--(4,-5) node[midway,right] {\tiny $\catvariableof{\atomorder\shortminus1}$};

        \node[anchor=center] (text) at (8,-2) {${=}$};
    \end{scope}

    \draw (-1.25,1) rectangle (1.25,3);
    \node[anchor=center] (text) at (0,2) {$\actcoreof{{0}}$};

    \draw (2.75,1) rectangle (5.25,3);
    \node[anchor=center] (text) at (4,2) {$\actcoreof{{\seccatorder\shortminus1}}$};

    \draw[->] (0,-1)--(0,0);
    \node[left] (text) at (0,0) {\tiny $\headvariableof{0}$};
    \draw[] (0,0)--(0,1);
    \drawvariabledot{0}{0}
    \node[anchor=center] (text) at (2,0) {$\cdots$};

    \draw[->] (4,-1)--(4,0);
    \node[right] (text) at (4,0) {\tiny $\headvariableof{\seccatorder\shortminus1}$};
    \draw[] (4,0)--(4,1);
    \drawvariabledot{4}{0}

    \draw (-1,-1) rectangle (5,-3);
    \node[anchor=center] (text) at (2,-2) {\small $\bencodingof{\sstat}$};
    \draw[-<-] (0,-3)--(0,-5) node[midway,left] {\tiny $\catvariableof{0}$};
    \draw[-<-] (1.5,-3)--(1.5,-5) node[midway,left] {\tiny $\catvariableof{1}$};
    \node[anchor=center] (text) at (3,-4) {$\cdots$};
    \draw[-<-] (4,-3)--(4,-5) node[midway,right] {\tiny $\catvariableof{\atomorder\shortminus1}$};


    \begin{scope}
        [shift={(0,-4)}]
        \draw[] (0,1)--(0,-6);
        \node[below] (text) at (0,-6) {\tiny $\catvariableof{\atomorder\shortminus1}$};
        \drawvariabledot{0}{-5}
        \draw[] (1.5,1)--(1.5,-6);
        \node[below] (text) at (1.5,-6) {\tiny $\catvariableof{1}$};
        \drawvariabledot{1.5}{-3}
        \node[anchor=center] (text) at (3,0) {$\cdots$};
        \node[anchor=center] (text) at (3,-5.5) {$\cdots$};
        \draw[] (4,1)--(4,-6);
        \node[below] (text) at (4,-6) {\tiny $\catvariableof{\atomorder\shortminus1}$};
        \drawvariabledot{4}{-2}

        \draw[] (0,-5) -- (6,-5);
        \draw[] (1.5,-3) -- (6,-3);
        \node[anchor=center] (text) at (5,-3.75) {$\vdots$};
        \draw[] (4,-2) -- (6,-2);
        \draw (6,-1) rectangle (9, -6);
        \node[anchor=center] (text) at (7.5,-3.5) {$\basemeasure$};

        \node[anchor=center] (text) at (10,-4.5) {$.$};

    \end{scope}

\end{tikzpicture}
\end{center}
where the partition function represents the normalizing contraction of the tensor network.
Let us note, that when choosing activation cores with nontrivial support, we can also prepare boolean base measures and in principle extend \corref{cor:unaryActivationExpdistRealization} to families of nontrivial base measures.
We will investigate such schemes later in \charef{cha:networkRepresentation}, where we call them hybrid logic networks.

% Compare with selection encoding
In comparison with the selection encoding representation of energy tensors, we have prepared a contraction without non-linear transforms, which represents the probability distributions being members of an exponential family.
However, relation encoding come with the expense of introducing more auxiliary variables compared with selection encodings.
To be more precise, while selection encodings bundle the coordinates of the statistic in single selection variables, relation encodings create for each state $\selindexin$ of these selection variable an own auxiliary variable $\sstatcatof{\selindex}$, which enumerated the image of the coordinate and can therefore be of high dimension.
Thus, selection encodings offer in general a more efficient storage format coming at the expense of nonlinear operations in the computation of probabilities.
We later will encounter situations, where selection encodings are feasible while relation encodings are not, when applying the formalism of formula selecting networks (see \charef{cha:formulaSelection}) in neuro-symbolic reasoning (see \charef{cha:networkReasoning}).


% Proper subset
Based on \corref{cor:unaryActivationExpdistRealization} a further natural question is, whether $\expfamilyof{\sstat,\ones}$ is a proper subset of $\realizabledistsof{\sstat,\elgraph}$.
This is the case for most statistics $\sstat$, since members of exponential families are positive with respect to their base measure, which is in the corollaries setting trivial, while in $\realizabledistsof{\sstat,\elgraph}$ we allow also for activation cores with vanishing coordinates, which in general do not produce positive distributions.
The only statistics where $\expfamilyof{\sstat,\ones}$ is not a proper subset of $\realizabledistsof{\sstat,\graph}$ are along this argumentation constant, since then the activation cores are one-dimensional vectors and vanishing coordinates are prohibited by the need for normalizability.
We will follow these intuitions in the discussion of logical reasoning, starting with \charef{cha:logicalReasoning}, and will use the formats $\realizabledistsof{\sstat,\graph}$ as hybrid formats storing probability distributions and logical knowledge bases.

% CP format
While we have restricted our discussion on the elementary decomposition of the activation tensor, further decomposition schemes have interesting interpretations as well.
Given a $\cpformat$ decomposition of the activation tensor (see for more details \charef{cha:sparseCalculus}), the corresponding distributions are weighted mixture distributions built from the elementary decompositions.
In general, the expressivity increases monotonously with the introduction of additional auxiliary variables and hyperedges in the representation format of activation tensors.


%% FALSE STATEMENT?
%We can sum multiples of the trivial tensor on the head cores without changing the distribution as we show next.
%
%\begin{theorem}
%	For any $\selindexin$, the distribution is invariant under replacing $\actcoreofat{\selindex,\canparamat{\indexedselvariable}}{\selvariableof{\selindex}}$ by $\actcoreofat{\selindex,\canparamat{\indexedselvariable}}{\catvariableof{\selindex}}+\lambda\cdot \onesat{\catvariableof{\selindex}}$ where $\lambda\in\rr$
%\end{theorem}
%\begin{proof}
%	Follows from linearity in each head core, trivialization by trivial heads and normalization.
%
%	By linearity we have
%	\begin{align*}
%		\contractionof{\bencodingof{\sstatcoordinateof{\selindex}}, (\actcoreofat{\selindex,\canparamat{\indexedselvariable}}{\catvariableof{\selindex}}+\lambda\cdot \onesat{\catvariableof{\selindex}})}{\shortcatvariables}
%		=
%		\contractionof{\bencodingof{\sstatcoordinateof{\selindex}}, \actcoreofat{\selindex,\canparamat{\indexedselvariable}}{\catvariableof{\selindex}}}{\shortcatvariables}
%		+\lambda\cdot  \contractionof{\bencodingof{\sstatcoordinateof{\selindex}}, \onesat{\catvariableof{\selindex}}}{\shortcatvariables}
%		=  \contractionof{\bencodingof{\sstatcoordinateof{\selindex}}, \actcoreofat{\selindex,\canparamat{\indexedselvariable}}{\catvariableof{\selindex}}}{\shortcatvariables}
%		+ \lambda \cdot \onesat{\shortcatvariables} \, .
%	\end{align*}
%\end{proof}














\sect{Graphical Models}

\red{Specific instances of Exponential families are graphical models \cite{wainwright_graphical_2008, murphy_probabilistic_2022}.
They combine both the independence approach and the computation approach to tensor network representations of probability distributions.}


%We have already depicted conditional dependency assumptions made for Markov Chains in \figref{fig:MC} and discussed the implied decomposition of the dual tensor networks.
Graphical models provide a more generic framework to relate conditional dependency assumptions on a distribution with tensor network decompositions.
Following the tensor network formalism we in this section introduce graphical models based on hypergraphs.
First, we study Markov Networks in most generality and then connect with conditional probabilities in the discussion of Bayesian Networks.

\subsect{Markov Networks}

We now define Markov Networks based on hypergraphs, to establish a direct connection with tensor network decorating the hypergraph.
In a more canonical way, Markov Networks are instead defined by graphs, where instead of the edges the cliques are decorated by factor tensors (see for example \cite{koller_probabilistic_2009}).
%While canonically Markov Networks are defined on graphs, we here define them based on hypergraphs to establish a direct connection to tensor networks defined on the same hypergraph.
%Along that line, Markov Networks are tensor networks with non-negative tensors (see \defref{def:tensorNetwork}), which are interpreted as probability distributions after normalization.

\begin{definition}[Markov Network]
    \label{def:markovNetwork}
    Let $\tnetof{\graph}$ be a tensor network of non-negative tensors decorating a hypergraph $\graph$.
    Then the Markov Network $\probof{\graph}$ to $\tnetof{\graph}$ is the probability distribution of $\catvariableof{\node}$ defined by the tensor
    \begin{align*}
        \probofat{\graph}{\nodevariables} = \frac{
            \contractionof{\{\hypercoreof{\edge} : \edge \in \edges\}}{\nodevariables}
        }{
            \contraction{\{\hypercoreof{\edge} : \edge \in \edges\}}
        } = \normalizationof{\tnetof{\graph}}{\nodevariables} \, .
    \end{align*}
    We call the denominator
    \begin{align*}
        \partitionfunctionof{\tnetof{\graph}} = \contraction{\{\hypercoreof{\edge} : \edge \in \edges\}}
    \end{align*}
    the partition function of the tensor network $\tnetof{\graph}$.
\end{definition}

% Marginalization and Conditioning
%Often, we are only interested in the distribution of a subset of variables, which are called the observable variables, and call the other variables hidden variables.
The marginalization of a Markov Network to $\tnetof{\graph}$ on subsets of variables $\catvariableof{\secnodes}$ is
\begin{align*}
    \probofat{\graph}{\catvariableof{\secnodes}}
    = \normalizationof{\tnetof{\graph}}{\catvariableof{\secnodes}} \, .
\end{align*}
This can be derived from \theref{the:splittingContractions}, which established an equivalence of contractions with sequences of consecutive contractions.

Further, the distribution of $\catvariableof{\secnodes}$ conditioned on $\catvariableof{\thirdnodes}$, where $\secnodes,\thirdnodes$ are disjoint subsets of $\nodes$, is
\begin{align*}
    \probtensor^{\graph}\left[\catvariableof{\secnodes}|\catvariableof{\thirdnodes}\right]
    = \normalizationofwrt{\tnetof{\graph}}{\catvariableof{\secnodes}}{\catvariableof{\thirdnodes}} \, .
\end{align*}

While we have directly defined Markov Networks as decomposed probability distributions, we now want to derive assumptions on a distribution assuring that such decompositions exist.
As we will see, the sets of conditional independencies encoded by a hypergraph are captured by its seperation properties, as we define next.

\begin{definition}[Separation of Hypergraph]
    A path in a hypergraph is a sequence of nodes $\node_{\atomenumerator}$ for $\atomenumeratorin$, such that for any $\atomenumerator\in[\atomorder-1]$ we find a hyperedge $\edge\in\edges$ such that $(\node_{\atomenumerator}, \node_{\atomenumerator+1})\subset \edge$.
    Given disjoint subsets $\nodesa$, $\nodesb$, $\nodesc$ of nodes in a hypergraph $\graph$ we say that $\nodesc$ separates $\nodesa$ and $\nodesb$ with respect to $\graph$, when any path starting at a node in $\nodesa$ and ending in a node in $\nodesb$ contains a node in $\nodesc$.
    %when removing the hyperedges which are contained in $\nodesc$ leads to a hypergraph with no path of hyperedges between a node in $\nodesa$ to a node in $\nodesb$.
\end{definition}

To characterize Markov Networks in terms of conditional independencies we need to further define the property of clique-capturing.
This property of clique-capturing established a correspondence of hyperedges with maximal cliques in the more canonical graph-based definition of Markov Networks \cite{koller_probabilistic_2009}.

\begin{definition}[Clique-Capturing Hypergraph]
    \label{def:ccHypergraph}
    We call a hypergraph $\graph$ clique-capturing, when each subset $\secnodes\subset\nodes$ is contained in a hyperedge, if for any $a,b\in\secnodes$ there is a hyperedge $\edge\in\edges$ with $a,b\in\secnodes$.
\end{definition}

Let us now show a characterization of Markov Networks in terms of conditional independencies, which is analogous to \theref{the:condIndBN}.

% Characterization
\begin{theorem}[Hammersley-Clifford]
    \label{the:condIndMN}
    Given a clique-capturing hypergraph $\graph$, the set of positive Markov Networks on the hypergraph coincides with the set of positive probability distributions, such that each for each disjoint subsets of variables $\nodesa$, $\nodesb$, $\nodesc$ we have $\catvariableof{\nodesa}$ is independent of $\catvariableof{\nodesb}$ conditioned on $\catvariableof{\nodesc}$, when $\nodesc$ separates $\nodesa$ and $\nodesb$ in the hypergraph. % called d-separation
\end{theorem}
\begin{proof}
    \proofrightsymbol:
    %Given any Markov Network, contracting with $\onehotmapof{\atomlegindexof{\nodesc}}$ turns all hyperedges contained in $\nodesc$ to scalar factors (copying possible).
    Let there be a hypergraph $\graph$, a Markov Network $\extnet$ on $\graph$ and nodes $\nodesa,\nodesb,\nodesc \subset \nodes$, such that $\nodesc$ separates $\nodesa$ from $\nodesb$.
    Let us denote by $\nodes_0$ the nodes with paths to $\nodesa$, which do not contain a node in $\nodesc$, and by $\nodes_1$ the nodes with paths to $\nodesb$, which do not contain a node in $\nodesc$.
    Further, we denote by $\edges_0$ the hyperedges which contain a node in $\nodes_0$ and by $\edges_1$ the hyperedges which contain a node in $\nodes_1$.
    By assumption of separability, both sets $\edges_0$ and $\edges_1$ are disjoint and no node in $\nodesa$ is in a hyperedge in $\edges_1$, respectively no node in $\nodesb$ is in a hyperedge in $\edges_0$, .
    We then have
    \begin{align*}
        \normalizationofwrt{\extnetasset}{\catvariableof{\nodesa},\catvariableof{\nodesb}}{\indexedcatvariableof{\nodesc}}
        = & \normalizationof{\extnetasset\cup\{\onehotmapof{\catindexof{\nodesc}}\}}{\catvariableof{\nodesa},\catvariableof{\nodesb}} \\
        = &  \normalizationof{\{\hypercoreof{\edge}\, : \, \edge\in\edges_0\}\cup\{\onehotmapof{\catindexof{\nodesc}}\}}{\catvariableof{\nodesa}} \\
        & \quad \otimes \normalizationof{\{\hypercoreof{\edge}\, : \, \edge\in\edges_1\}\cup\{\onehotmapof{\catindexof{\nodesc}}\}}{\catvariableof{\nodesb}} \, .
    \end{align*}
    By \theref{the:condIndependenceProductCriterion}, it now follows that $\catvariableof{\nodesa}$ is independent of $\catvariableof{\nodesb}$ conditioned on $\catvariableof{\nodesc}$.

    \proofleftsymbol:
    The converse direction, i.e. that positive distributions respecting the conditional indpendence assumptions are representable as Markov Networks, is known as the Hammersley Clifford Theorem (see \cite{clifford_markov_1971}), which we will proof later in \secref{sec:proofHCTheorem} of \charef{cha:coordinateCalculus}.
    %for which proof we refer to Theorem~4.8 in KOLLER.
\end{proof}

% Positivity
From the proof of \theref{the:condIndMN} Markov Networks with zero coordinates still satisfy the conditional independence assumption.
However, the reverse is not true, that is there are distributions with vanishing coordinates, which satisfy the conditional independence assumptions, but cannot be represented as a Markov Network (see Example~4.4 in \cite{koller_probabilistic_2009}).




\subsect{Bayesian Networks}

Compared to Markov Networks, Bayesian Networks impose further conditions on tensor networks representing a distribution.
They assume a directed hypergraph and each tensor decorating the edges to be normed according to the direction.
We will observe, that if the hypergraph is in addition acyclic, then each tensor core coincides with the conditional distribution of the underlying Markov Network.
To introduce Bayesian Networks, we extend \defref{def:hypergraphs} by introducing the property of acyclicity for hypergraphs.

%are described by directed acyclic graphs (DAG).
%The probability distribution is a Hadamard product of conditional probabilities, where each variable has a conditional probability factor conditioned on the parents variables in the graph.
%We introduce Bayesian Networks based on directed hypergraphs (see \defref{def:hypergraphs}) and define further properties.

\begin{definition}
    A directed path is a sequence $\node_{0},\ldots\node_{\secatomorder}$ such that for any $\secatomenumeratorin$ there is an hyperedge $\edge=(\incomingnodes,\outgoingnodes)\in\edges$ such that $\node_{\secatomenumerator}\in\incomingnodes$ and $\node_{\secatomenumerator+1}\in\outgoingnodes$.
    We call the hypergraph $\graph$ acyclic, if there is no path with $\secatomorder>0$ such that $\node_{0}=\node_{\secatomorder}$.
    Given a directed hypergraph $\graph=(\nodes,\edges)$ we define for any node $\nodein$ its parents by
    \[ \parentsof{\node} = \{\secnode \, : \, \exists\edge=(\incomingnodes,\outgoingnodes)\in\edges: \secnode\in\incomingnodes,\node\in\outgoingnodes \} \]
    and its non-descendants $\nondescendantsof{\node}$ as the set of nodes $\secnode$, such that there is no directed path from $\node$ to $\secnode$.
\end{definition}

Based on these additional graphical properties, we now define Bayesian Networks.

\begin{definition}[Bayesian Network]
    \label{def:bayesianNetwork}
    Let $\graph=(\nodes,\edges)$ be a directed acyclic hypergraph with edges of the form
    \[ \edges = \bnedges \, . \]
    A \emph{Bayesian Network} is a decoration of each edge $(\parentsof{\node},\{\node\})$ by a conditional probability distribution
    \[ \condprobof{\catvariableof{\node}}{\catvariableof{\parentsof{\node}}} \]
    which represents the probability distribution
    \begin{align*}
        \probat{\nodevariables} = \contractionof{\{\condprobof{\catvariableof{\node}}{\catvariableof{\parentsof{\node}}} \, : \, \nodein\}}{\nodevariables} \, .
    \end{align*}
\end{definition}

%
By definition each tensor decorating a hyperedge is directed with $\catvariableof{\parentsof{\node}}$ incoming and $\catvariableof{\node}$ outgoing.
Thus, the directionality of the hypergraph is reflected in each tensor decorating a directed hyperedge.
This allows us to verify with \theref{the:conditionalContractionPreservation} that their contraction defines a probability distribution.

% Contraction -> Now in definition!
%By definition we can represent a Bayesian network by the contraction
%\begin{align*}
%	\probtensorof{\graph} = \contractionof{\{ \condprobof{\catvariableof{\node}}{\catvariableof{\parentsof{\node}}} \, : \, \node\in\nodes\}}{\nodes} \, .
%\end{align*}

% Dual
%The dual tensor network consists of conditional probability distributions to each node $\node\in\nodes$ (see Figure~\ref{fig:BayesianFactor}b).

\begin{figure}[h]
    \begin{center}
        \begin{tikzpicture}[scale=0.35,thick] % , baseline = -3.5pt

\node[anchor=center] (text) at (-1,3) {${a)}$};

	\node [circle, draw, thick, fill=gray!50] (H) at (5,0) {\tiny $\randomxof{\node}$};
	\node [circle, draw, thick, fill=gray!50] (P1) at (0,-5) {\tiny $\randomxof{0}$};	
	\node [circle, draw, thick, fill=gray!50] (P2) at (5,-5) {\tiny $\randomxof{1}$};	
	
	\node[anchor=center] (text) at (10,-5) {$\cdots$};
	\node [circle, draw, thick, fill=gray!50] (Pd) at (15,-5) {\tiny $\randomxof{\atomorder\shortminus1}$};
	
	\node [] (E) at (5,-2) {};	
	
	\draw[midarrow] (P1) -- (5,-2) ;	
	\draw[midarrow] (P2) -- (5,-2) ;	
	\draw[midarrow] (Pd) -- (5,-2) ;	
	\draw[midarrow] (5,-2) -- (H) ;	
			

\begin{scope}[shift={(25,0)}]

\node[anchor=center] (text) at (-3,3) {${b)}$};

\draw[->-] (4.5,-1) -- (4.5,1) node[midway, right]{\tiny $\catvariableof{\node}$};
\draw (0,-1) rectangle (9,-4); 
\node[anchor=center] (text) at (4.5,-2.5) {\small $\condprobof{\randomxof{\node}}{\randomxof{[\atomorder]}} $};
\draw[->-] (1,-6) -- (1,-4) node[midway, right]{\tiny $\catvariableof{0}$};
\draw[->-] (2.5,-6) -- (2.5,-4) node[midway, right]{\tiny $\catvariableof{1}$};

\node[anchor=center] (text) at (5.5,-5) {$\cdots$};
	
\draw[->-] (8,-6) -- (8,-4) node[midway, right]{\tiny $\catvariableof{\atomorder\shortminus1}$};

\end{scope}

\end{tikzpicture} 
    \end{center}
    \caption{Example of a Factor of a Bayesian Network to the node $\catvariableof{\node}$ with parents $\catvariableof{0},\ldots,\catvariableof{\catorder-1}$, as an directed edge a) which is decorated by a directed tensor b).}
    \label{fig:BayesianFactor}
\end{figure}


%% Marginalization and Contraction
Marginalization of a Bayesian Network are still Bayesian Networks on a graph where the edges directing to variables, which are not marginalized over, are replaced by directed edges to the children.
Conditioned Bayesian Network do not have a simple Bayesian Network representation, which is why we will treat them as Markov Networks to be introduced next.


\begin{theorem}[Independence Characterization of Bayesian Networks]
    \label{the:condIndBN}
    A probability distribution $\probat{\nodevariables}$ has a representation by a Bayesian Network on a directed acyclic graph $\graph=(\nodes,\edges)$, if and only if for any $\nodein$ the variables $\catvariableof{\node}$ are independent on $\nondescendantsof{\node}$ conditioned on $\parentsof{\node}$.
\end{theorem}
\begin{proof}
    We choose a topological order $\prec$ on the nodes of $\graph$, which exists since $\graph$ is acyclic.

    \proofrightsymbol:
    Let us assume, that the conditional independencies are satisfied and apply the chain rule with respect to that ordering to get
    \begin{align*}
        \probat{\nodevariables} =
        \contractionof{
            \condprobof{\catvariableof{\node}}{\catvariableof{\secnode} : \secnode \prec \node}
        }
        {\nodevariables} \, .
    \end{align*}
    Since $\prec$ is a topological ordering we have
    \[ \parentsof{\node} \subset \{\secnode : \secnode \prec \node\} \]
    We apply the assumed conditional independence with \theref{the:conditionDropping} and get
    \begin{align*}
        \probat{\nodevariables} =
        \contractionof{
            \condprobof{\catvariableof{\node}}{\catvariableof{\parentsof{\node}}}
        }
        {\nodevariables} \, .
    \end{align*}

    \proofleftsymbol:
    To show the converse direction, let there be a Bayesian Network $\probat{\nodevariables}$ on $\graph$.
    To show for any node $\node$, that $\catvariableof{\node}$ is independent of $\nondescendantsof{\node}$ conditioned on $\parentsof{\node}$, we reorder the tensors in the contraction
    %with respect to a set $\node_0$
    \begin{align*}
        & \condprobof{\catvariableof{\node},\catvariableof{\nondescendantsof{\node}}}{\indexedcatvariableof{\parentsof{\node}}} \\
        & \quad\quad = \normalizationofwrt{
            \{\condprobof{\catvariableof{\secnode}}{\catvariableof{\parentsof{\secnode}}} \, : \, \secnode\in\nodes\}
        }
        {\catvariableof{\node},\catvariableof{\nondescendantsof{\node}}}
        {\indexedcatvariableof{\parentsof{\node}}} \\
        & \quad\quad  = \normalizationof{
            \{\condprobof{\catvariableof{\secnode}}{\catvariableof{\parentsof{\secnode}}} \, : \, \secnode\in\nodes\} \cup \{\onehotmapof{\catindexof{\parentsof{\node}}}\}
        }
        {\catvariableof{\node},\catvariableof{\nondescendantsof{\node}}}\\
        &  \quad\quad = \normalizationof{
            \{\condprobof{\catvariableof{\secnode}}{\catvariableof{\parentsof{\secnode}}} \, : \, \secnode\in\nondescendantsof{\node}\} \cup \{\onehotmapof{\catindexof{\parentsof{\node}}}, \condprobof{\catvariableof{\node}}{\catvariableof{\parentsof{\node}}} \}
        }
        {\catvariableof{\node},\catvariableof{\nondescendantsof{\node}}} \\
        &  \quad\quad =  %\contractionof{
        \normalizationof{
            \{\condprobof{\catvariableof{\secnode}}{\catvariableof{\parentsof{\secnode}}} \, : \, \secnode\in\nondescendantsof{\node}\} \cup \{\onehotmapof{\catindexof{\parentsof{\node}}}\}
        }
        {\catvariableof{\nondescendantsof{\node}}} \\
        & \quad\quad  \quad  \cdot \normalizationof{
            \{\condprobof{\catvariableof{\node}}{\catvariableof{\parentsof{\node}}},\onehotmapof{\catindexof{\parentsof{\node}}}\}
        }
        {\catvariableof{\node}} \\
        & \quad\quad  = \contractionof{\{
        \condprobof{\catvariableof{\nondescendantsof{\node}}}{\indexedcatvariableof{\parentsof{\node}}},
            \condprobof{\catvariableof{\node}}{\indexedcatvariableof{\parentsof{\node}}}
            \}}{\catvariableof{\node},\catvariableof{\nondescendantsof{\node}}}
        %}{\catvariableof{\node},\catvariableof{\nondescendantsof{\node}}}
    \end{align*}
    Here we have dropped in the third equation all tensors to the descendants, since their marginalization is trivial (which can be shown by a leaf-stripping argument).
    In the fourth equation we made use of the fact, that any directed path between the non-descendants and the node is through the parents of the node.
    By \theref{the:condIndependenceProductCriterion}, it now follows that $\catvariableof{\node}$ is independent of $\nondescendantsof{\node}$ conditioned on $\parentsof{\node}$.
\end{proof}

\subsect{Bayesian Networks as Markov Networks}

Markov Networks are more flexible compared with Bayesian Networks, since any Bayesian Network is a Markov Network by ignoring the directionality of the hypergraph and understanding the conditional distributions as generic tensor cores.
In the next theorem we provide the conditions for the interpretation of a Markov Network as a Bayesian Network.

\begin{theorem}
    \label{the:MarkovToBayesian}
    Let $\tnetof{\graph}$ be a tensor network on a directed acyclic hypergraph, such that the edges are of the structure
    \[ \edges = \bnedges \]
    and each tensor $\hypercoreof{\edge}$ respects the directionality of the graph, that is each $\hypercoreof{(\parentsof{\node}, \{\node\})}$ is directed with the variables to $\parentsof{\node}$ incoming and $\node$ outgoing.
    Then $\partitionfunctionof{\tnetof{\graph}}=1$ and for each $\node\in\nodes$ we have
    \[ \bnnodecore = \normalizationofwrt{\tnetof{\graph}}{\catvariableof{\node}}{\catvariableof{\parentsof{\node}}} \, . \]
    In particular, $\tnetof{\graph}$ is a Bayesian Network.
\end{theorem}
\begin{proof}
    We show the claim by induction over the cardinality of $\nodes$.

    $\cardof{\nodes}=1$: In this case we find a unique node $\node\in\nodes$ and have $\edges=\{(\varnothing,\{\node\})\}$.
    The tensor $\hypercoreof{(\varnothing,\{\node\})}$ is then normed with no incoming variables and we thus have
    \[ \partitionfunctionof{\tnetof{\graph}} = \contraction{\tnetof{\graph}} = \contraction{\hypercoreof{(\varnothing,\{\node\})}} = 1 \]
    and
    \[ \normalizationof{\tnetof{\graph}}{\catvariableof{\node}} = \hypercoreof{(\varnothing,\{\node\})} \, .  \]

    $\cardof{\nodes}-1 \rightarrow \cardof{\nodes}$: Let there now be a directed hypergraph $\graph=(\nodes,\edges)$ and let us now assume, that the theorem holds for any tensor networks with node cardinality $\cardof{\nodes}-1$.
    Since the hypergraph is acyclic, we find a root $\node\in\nodes$ such that $\node\notin\parentsof{\secnode}$ for $\secnode\in\nodes$.
    We denote $\tnetof{\secgraph}$ the tensor network on the hypergraph $\secgraph=\{\nodes/\{\node\},\edges/\{(\parentsof{\node},\{\node\})\}\}$ with decorations inherited from $\tnetof{\graph}$.
    With Theorem~\ref{the:splittingContractions}, the directionality of $\bnnodecore$ and the induction assumption on $\tnetof{\secgraph}$ we have
    \begin{align*}
        \contraction{\tnetof{\secgraph}\cup\left\{\bnnodecore\right\}}
        = \contraction{\tnetof{\secgraph}\cup\left\{\contractionof{\bnnodecore}{\catvariableof{\parentsof{\node}}}\right\}}
        = \contraction{\tnetof{\secgraph}\cup\left\{\onesat{\catvariableof{\parentsof{\node}}}\right\}}
        = 1
    \end{align*}
    and thus a trivial partition function.
    Since $\node$ does not appear in $\secgraph$, we have for any index $\catindexof{\parentsof{\nodes}}$
    \begin{align*}
        \contractionof{\tnetof{\graph}}{\catvariableof{\node},\indexedcatvariableof{\parentsof{\node}}}
        = \contractionof{\bnnodecore}{\catvariableof{\node},\indexedcatvariableof{\parentsof{\node}}}
        \cdot \contractionof{\tnetof{\secgraph}}{\indexedcatvariableof{\parentsof{\node}}}
    \end{align*}
    and thus, since $\bnnodecore$ is directed, that
    \begin{align*}
        \normalizationofwrt{\tnetof{\graph}}{\catvariableof{\node}}{\catvariableof{\parentsof{\node}}}
        = \bnnodecore \, .
    \end{align*}
\end{proof}

%\begin{theorem}\label{the:BayesianToMarkov}
%	Any Bayesian Network on a directed graph $\graph=(\nodes,\edges)$ is a Markov Network on a hypergraph $\secgraph=(\nodes,\secedges)$ with identical nodes and hyperedges consistent of  a hyeredge to each node with $\node$ being the only outgoing node and
%		\[  \{\tilde{\node} \, : \, (\tilde{\node},\node) \in \edges\} \,  \]
%	being the incoming nodes.
%	Each hyperedge of the Markov Network is decorated with the conditional probability distribution and the partition function is vanishing.
%\end{theorem}
%\begin{proof}
%	Each conditional probability distribution is associated with the hyperedge constructed to the representative node.
%	The contraction of all conditional probability distributions is the Bayesian Network, which corresponds with the constructed Markov Network due to the trivial partition function.
%\end{proof}

%% Bayesian Network richer
Theorem~\ref{the:MarkovToBayesian} states that Bayesian Networks are a subset of Markov Networks.
While Markov Network allow generic tensor cores, Bayesian Networks impose a local directionality condition on each tensor core by demanding it to be a conditional probability tensor.
In our diagrammatic notation, the local normalization of Bayesian Networks is highlighted by the directionality of the hypergraph.
Generic Markov Networks are on undirected hypergraphs, where in general no local directionality condition is assumed.
As a consequence, tasks such as the determination of the partition functions or calculation of conditional distributions involve global contractions.


%% Conditioning
%The representation of Bayesian Networks by Markov Networks is of special interest when representing conditional distributions.
%Bayesian Networks conditioned on evidence are no longer Bayesian Networks on the same graph, but Markov Networks on a hypergraph enriched by the evidence conditioned about.


\subsect{Hidden Markov Models}

Hidden Markov Models are examples of Bayesian Networks, constructed as follows.
Let us recall Markov Chains as investigated in \theref{the:MarkovChain} and extend them by observation variables $\randomeof{\catenumerator}$ for $\catenumeratorin$, representing limited observations of the state variables $\catvariableof{\catenumerator}$.
%Let there be the variables $\catvariableof{\catenumerator}$ (states) and $\randomeof{\catenumerator}$ (observations) with a discrete and finite time $\catenumeratorin$.
To be more precise, we assume the following conditional independencies:
\begin{itemize}
    \item As for Markov Chains, we assume that for $\catenumeratorin$ with $\catenumerator\geq1$ the variable $\catvariableof{\catenumerator}$ is independent of $\catvariableof{[\catenumerator-1}$ and $\randomeof{[\catenumerator-1]}$ conditioned on $\catvariableof{\catenumerator-1}$
    \item In addition, for we assume that for $\catenumeratorin$ the observation variable $\randomeof{\catenumerator}$ is independent of $\catvariableof{[\catenumerator}$ and $\randomeof{[\catenumerator]}$ conditioned on $\catvariableof{\catenumerator}$
\end{itemize}
From this conditional independence assumption, we apply the Chain Rule \theref{the:chainRule} given the order of variables
\begin{align*}
    \catvariableof{0},\randomeof{0},\catvariableof{1},\randomeof{1},\ldots,\catvariableof{\catorder-1},\randomeof{\catorder-1}
\end{align*}
and get
\begin{align*}
    \probat{\catvariableof{[\catorder]},\randomeof{[\catorder]}}
    & = \breakablecontractionof{
        \{\margprobat{\catvariableof{0}},\condprobat{\randomeof{0}}{\catvariableof{0}}\} \\
        & \quad\quad  \cup \{\condprobat{\catvariableof{\catenumerator}}{\catvariableof{[\catenumerator]},\randomeof{[\catorder]}} \,:\, \catenumeratorin\} \\
        & \quad\quad \cup \{\condprobat{\randomeof{\catenumerator}}{\catvariableof{[\catenumerator+1]},\randomeof{[\catorder]}} \,:\, \catenumeratorin\}
    }{
        \catvariableof{[\catorder]},\randomeof{[\catorder]}
    } \, .
\end{align*}
We now apply the conditional independence assumptions to sparsify the appearing conditional distributions by application of \theref{the:conditionDropping}.
This results in the decomposition (see \figref{fig:HMM}b)
\begin{align*}
    \probat{\catvariableof{[\catorder]},\randomeof{[\catorder]}}
    & = \breakablecontractionof{
        \{\margprobat{\catvariableof{0}},\condprobat{\randomeof{0}}{\catvariableof{0}}\} \\
        & \quad\quad  \cup \{\condprobat{\catvariableof{\catenumerator}}{\catvariableof{\catenumerator-1}} \,:\, \catenumeratorin\} \\
        & \quad\quad \cup \{\condprobat{\randomeof{\catenumerator}}{\catvariableof{\catenumerator}} \,:\, \catenumeratorin\}
    }{
        \catvariableof{[\catorder]},\randomeof{[\catorder]}
    } \, .
\end{align*}
In addition to the stochastic transition matrices $\condprobat{\catvariableof{\catenumerator}}{\catvariableof{\catenumerator-1}}$ appearing in Markov Chains, we further have stochastic observation matrices $\condprobat{\randomeof{\catenumerator}}{\catvariableof{\catenumerator}}$ for $\catenumeratorin$.
Their contraction with marginal distribution of the respective state varibles delivers the marginal distribution of the observation matrix by
\begin{align*}
    \margprobat{\randomeof{\catenumerator}}
    = \contractionof{\condprobat{\randomeof{\catenumerator}}{\catvariableof{\catenumerator}},\margprobat{\catvariableof{\catenumerator}}}{\randomeof{\catenumerator}}
\end{align*}
We notice, that this is a Bayesian Network on a directed acyclic hypergraph $\graph$ (see \figref{fig:HMM}a) consistent in nodes $\{\catvariableof{[\catorder]}\}\cup\{\randomeof{[\catorder]}\}$ to each state and observation variables, and the directed hyperedges by
%and directed hyperedges
\begin{itemize}
    \item $(\varnothing,\{\catvariableof{0}\})$, decorated by the intial marginal distribution of $\catvariableof{0}$
    \item $(\{\catvariableof{\catenumerator-1}\}, \{\catvariableof{\catenumerator}\})$ for $\catenumerator\in[\catorder]$ with $\catenumerator\geq1$, decorated by stochastic transition matrices
    \item $(\{\catvariableof{\catenumerator}\}, \{\randomeof{\catenumerator}\})$ for $\catenumeratorin$, decorated stochastic observation matrices
\end{itemize}
While we have derived this directed graph structure directly based on the chain rule decomposition with sparsified conditional distributions, it also follows from the more generic hypergraph characterization of Bayesian Networks through separability by \theref{the:condIndBN}.

\begin{figure}[t!]
    \begin{center}
        \begin{tikzpicture}[scale=0.3,thick] % , baseline = -3.5pt

    \node[anchor=center] (text) at (-1,3) {${a)}$};

    \node [circle, draw, thick, fill=\nodegrayscale, minimum size = \nodeminsize] (T1) at (0,0) {\colorlabelsize $\catvariableof{0}$};
    \node [circle, draw, thick, fill=\nodegrayscale, minimum size = \nodeminsize] (E1) at (0,-4) {\colorlabelsize $\randomeof{0}$};
    \draw[->-] (T1) -- (E1);
    \node [circle, draw, thick, fill=\nodegrayscale, minimum size = \nodeminsize] (T2) at (4,0) {\colorlabelsize $\catvariableof{1}$};
    \node [circle, draw, thick, fill=\nodegrayscale, minimum size = \nodeminsize] (E2) at (4,-4) {\colorlabelsize $\randomeof{1}$};
    \draw[->-] (T2) -- (E2);
    \draw[->-] (T1) -- (T2);
    \node [circle, draw, thick, fill=\nodegrayscale, minimum size = \nodeminsize] (T3) at (8,0) {\colorlabelsize $\catvariableof{2}$};
    \node [circle, draw, thick, fill=\nodegrayscale, minimum size = \nodeminsize] (E3) at (8,-4) {\colorlabelsize $\randomeof{2}$};
    \draw[->-] (T3) -- (E3);
    \draw[->-] (T2) -- (T3);
    \node [circle, draw, thick, fill=\nodegrayscale, minimum size = \nodeminsize] (T4) at (12,0) {\colorlabelsize $\catvariableof{3}$};
    \node [circle, draw, thick, fill=\nodegrayscale, minimum size = \nodeminsize] (E4) at (12,-4) {\colorlabelsize $\randomeof{3}$};
    \draw[->-] (T4) -- (E4);
    \draw[->-] (T3) -- (T4);
    \draw[->-] (T4) -- (15,0);

    \node[anchor=center] (text) at (16,0) {$\cdots$};

    %\node [circle, draw, thick, fill=\nodegrayscale, minimum size = \nodeminsize] (T4) at (17,0) {\colorlabelsize $\catvariableof{\atomorder}$};
    %\draw[->-] (14,0) -- (T4);


    \begin{scope}
        [shift={(22,0)}]

        \node[anchor=center] (text) at (-3,3) {${b)}$};

        \draw (-3.5,-1) rectangle (0, 1);
        \node[anchor=center] (text) at (-1.75,0) {\corelabelsize $\probat{\catvariableof{0}}$};
        \draw[->-] (0,0) -- (2,0);
        \draw[fill] (1,0) circle (\dotsize);
        \draw[->-] (1,0) -- (1,2) node[above] {\colorlabelsize ${\catvariableof{0}}$};
        \draw[->-] (1,0) -- (1,-2);
        \draw (-1.5,-2) rectangle (3.5,-4);
        \node[anchor=center] (text) at (1,-3) {\corelabelsize $\condprobof{\randomeof{0}}{\catvariableof{0}}$};
        \draw[->-] (1,-4) -- (1,-6) node[midway, right]{\colorlabelsize ${\randomeof{0}}$};

        \draw (2,-1) rectangle (7, 1);
        \node[anchor=center] (text) at (4.5,0) {\corelabelsize $\condprobof{\catvariableof{1}}{\catvariableof{0}}$};
        \draw[->-]  (7,0) -- (9,0);
        \draw[fill] (8,0) circle (\dotsize);
        \draw[->-] (8,0) -- (8,2) node[above] {\colorlabelsize ${\catvariableof{1}}$};
        \draw[->-] (8,0) -- (8,-2);
        \draw (5.5,-2) rectangle (10.5,-4);
        \node[anchor=center] (text) at (8,-3) {\corelabelsize $\condprobof{\randomeof{1}}{\catvariableof{1}}$};
        \draw[->-] (8,-4) -- (8,-6) node[midway, right]{\colorlabelsize ${\randomeof{1}}$};


        \draw (9,-1) rectangle (14, 1);
        \node[anchor=center] (text) at (11.5,0) {\corelabelsize $\condprobof{\catvariableof{2}}{\catvariableof{1}}$};
        \draw[->-]  (14,0) -- (16,0);
        \draw[fill] (15,0) circle (\dotsize);
        \draw[->-] (15,0) -- (15,2) node[above] {\colorlabelsize ${\catvariableof{2}}$};
        \draw[->-] (15,0) -- (15,-2);
        \draw (12.5,-2) rectangle (17.5,-4);
        \node[anchor=center] (text) at (15,-3) {\corelabelsize $\condprobof{\randomeof{2}}{\catvariableof{2}}$};
        \draw[->-] (15,-4) -- (15,-6) node[midway, right]{\colorlabelsize ${\randomeof{2}}$};

        \draw (16,-1) rectangle (21, 1);
        \node[anchor=center] (text) at (18.5,0) {\corelabelsize $\condprobof{\catvariableof{3}}{\catvariableof{2}}$};
        \draw[->-]  (21,0) -- (23,0);
        \draw[fill] (22,0) circle (\dotsize);
        \draw[->-] (22,0) -- (22,2) node[above] {\colorlabelsize ${\catvariableof{3}}$};
        \draw[->-] (22,0) -- (22,-2);
        \draw (19.5,-2) rectangle (24.5,-4);
        \node[anchor=center] (text) at (22,-3) {\corelabelsize $\condprobof{\randomeof{3}}{\catvariableof{3}}$};
        \draw[->-] (22,-4) -- (22,-6) node[midway, right]{\colorlabelsize ${\randomeof{3}}$};


        \node[anchor=center] (text) at (24,0) {$\cdots$};


    \end{scope}

\end{tikzpicture} 
    \end{center}
    \caption{Decomposition of a probability distribution in the Hidden Markov Model, consistent of state variables $\catvariableof{\catenumerator}$ and observation variables $\randomeof{\catenumerator}$.
    Given the models conditional independence assumptions, the distribution is a Bayesian Network on the directed hypergraph a).
    The hypergraph is decorated by the network of conditional probability tensors b), which are interpreted as stochastic transition matrices $\condprobat{\catvariableof{\catenumerator}}{\catvariableof{\catenumerator-1}}$ and stochastic observation matrices $\condprobat{\randomeof{\catenumerator}}{\catvariableof{\catenumerator}}$.
    }
    \label{fig:HMM}
\end{figure}




\subsect{Markov Networks as Exponential Families}

% Markov Networks
As we have claimed before, exponential families can be regarded as a generalization of graphical models.
We here show this claim by a construction of exponential families representing Markov Networks on constant hypergraphs.
%Since Markov Networks have been shown to also contain Bayesian Networks
%Given a hypergraph with fixed node decoration, the different decorations of the hyperedges by tensors can be represented by an exponential family, as we show next.

\begin{theorem}[Exponential Representation of Markov Networks]\label{the:markovNetworkExponentialFamilies}
    Let there be a hypergraph $\graph=(\nodes,\edges)$ with a coloring of the nodes by dimensions $\catdimof{\node}$, we define a sufficient statistics
    \begin{align*}
        \sstatof{\graph} \, : \bigtimes_{\node\in\nodes} [\catdimof{\node}] \rightarrow \bigtimes_{\edge\in\edges}\left(\bigtimes_{\node\in\edge}[\catdimof{\node}]\right)
    \end{align*}
    by a cartesian product $\sstatof{\graph} = \bigtimes_{\edge\in\edges} \sstatcoordinateof{\edge}$ of statistics
    \begin{align*}
        \sstatcoordinateof{\edge} : \bigtimes_{\node\in\nodes} [\catdimof{\node}] \rightarrow \bigtimes_{\node\in\edge} [\catdimof{\node}]
    \end{align*}
    defined by the restriction of indices to the respective edge, that is for $\catindexof{\nodes}\in\nodestatesof{\nodes}$
    \begin{align*}
        \sstatcoordinateof{\edge}(\catindexof{\nodes}) = \catindexof{\edge} \, .
    \end{align*}
    Given any Markov Network with positive tensors $\{\hypercoreof{\edge} \, : \, \edge\in\edges\}$ decorating the hyperedges of $\graph$ we define
    \begin{align*}
         \canparamat{\selvariable} = \bigtimes_{\edge\in\edges} \canparamofat{\edge}{\catvariableof{\edge}}
    \end{align*}
    where
    \begin{align*}
        \canparamofat{\edge}{\catvariableof{\edge}} =  \lnof{\hypercoreofat{\edge}{\catvariableof{\edge}}}
    \end{align*}
    and $\selvariable$ enumerates a concatenation of the states of $\catvariableof{\edge}$.
    Then, the Markov Network is the member with canonical parameter $\canparam$ of the exponential family with trivial base measure, statistic $\sstatof{\graph}$, which we denote by $\mnexpfamily$.
\end{theorem}
\begin{proof}
    We have for any $\catindexof{\nodes}$
    \begin{align*}
        \prod_{\edge\in\edges} \hypercoreofat{\edge}{\indexedcatvariableof{\edge}} \\
        &= \expof{\sum_{\edge\in\edges} \lnof{\hypercoreofat{\edge}{\indexedcatvariableof{\edge}}}} \\
        &= \expof{\sum_{\edge\in\edges} \canparamofat{\edge}{\indexedcatvariableof{\edge}}} \\
        &= \expof{\sum_{\edge\in\edges} \contraction{\canparamofat{\edge}{\catvariableof{\edge}},\sstatcoordinateof{\edge}(\catindexof{\nodes})}}  \, .
    \end{align*}
    By contraction, we further have
    \begin{align*}
        \contractionof{\sstatat{\nodevariables,\selvariable},\canparamwith}{\nodevariables}
        = \sum_{\edge\in\edges} \contractionof{\sstatcoordinateofat{\edge}{\nodevariables,\catvariableof{\edge}},\canparamofat{\edge}{\catvariableof{\edge}}}{\nodevariables}
    \end{align*}
    we thus get with the above
    \begin{align}
        \contractionof{\{\hypercoreof{\edge}: \edge\in\edges\}}{\nodevariables}
        = \expof{\contractionof{\sstatat{\nodevariables,\selvariable},\canparamwith}{\nodevariables}} \, .
    \end{align}
    This implies, that the contraction of the tensors in the Markov Network coincides with the exponential of the energy tensor of the constructed member of the exponential family.
    It follows for the normalization, that
    \begin{align}
        \normalizationof{\{\hypercoreof{\edge}: \edge\in\edges\}}{\nodevariables}
        = \normalizationof{\expof{\contractionof{\canparam,\sstat}{\nodevariables}}}{\nodevariables} \, .
    \end{align}
    We thus conclude, that the Markov Network coincides with the constructed member of the exponential family.
\end{proof}

% Mean parameters
The mean parameter of the Markov Network exponential family is the cartesian product of the marginals $\meanparamofat{\edge}{\catvariableof{\edge}}$.
They are often refered to as beliefs in the literature, as introduced by \cite{pearl_probabilistic_1988}.
For Markov Networks on tree hypergraphs, and their embedding into junction tree formats, the corresponding mean parameter polytope can be characterized by local consistency constraints.
More precisely, it can be shown, that for the statistic constructed in \theref{the:markovNetworkExponentialFamilies}, in case of tree hypergraphs, the
\begin{align*}
    \meansetof{\sstatof{\graph}}
    = \Big\{ \meanparamwith=\big(\meanparamofat{\edge}{\catvariableof{\edge}}\big)_{\edge\in\edges} \, : \,
    & \uniquantwrtof{\edge,\secedge\in\edges}{
        \contractionof{\meanparamofat{\edge}{\catvariableof{\edge}}}{\catvariableof{\edge\cap\secedge}} = \contractionof{\meanparamofat{\edge}{\catvariableof{\edge}}}{\catvariableof{\edge\cap\secedge}}}, \\
    & \uniquantwrtof{\edge}{\zerosat{\catvariableof{\edge}} \prec \meanparamofat{\edge}{\catvariableof{\edge}} \land \contraction{\meanparamofat{\edge}{\catvariableof{\edge}}}=1} \Big\} \, .
\end{align*}
That is, the polytope of realizable mean parameters consists of those non-negative and normed beliefs, which are coinciding on the contraction of shared variables.
Capturing these constraints by Lagrange parameters and performing optimization of certain objectives then results in message-passing schemes, as we will discuss in \charef{cha:messagePassing}.
If the hypergraph is not minimally connected, this constructed polytope is only an outer bound of the true mean parameter polytope, but still serves as a motivation of loopy belief propagation schemes (see Chapter~4 in \cite{wainwright_graphical_2008}).

% (see e.g. see for a derivation \cite{wainwright_graphical_2008}).



\subsect{Representation of generic distributions}\label{sec:mintermExpFamily}

\red{We now present a universal exponential family, which contains all positive with respect to a base measure distributions.}

%\red{They are outer bounded by local consistency polytope, which leads to the motivation of message passing algorithms called belief propgations, see for a derivation \cite{wainwright_graphical_2008}.}
% Minterm Exponential Family
%\begin{example}[The minterm exponential family]

The formalism of exponential families can capture any probability distribution, when applying statistic functions of large expressivity.
Taking for the statistic the identity function $\identityat{\shortcatvariables,\selvariableof{[\catorder]}}$ defined as
\begin{align*}
    \identityat{\indexedshortcatvariables,\indexedselvariableof{[\catorder]}}
    = \begin{cases}
          1 & \text{if} \quad \shortcatindices = \selindexof{[\catorder]} \\
          0 & \text{else}
     \end{cases} \, ,
\end{align*}
we can represent any positive probability distribution $\probwith$ as a member of the exponential family $\expfamilyof{\identity,\ones}$.
To see this, it is enough to choose
\begin{align*}
    \canparamat{\selvariableof{[\catorder]}} = \contractionof{\lnof{\probat{\shortcatvariables}},\identityat{\indexedshortcatvariables,\indexedselvariableof{[\catorder]}}}{\selvariableof{[\catorder]}} \, ,
\end{align*}
where the contraction with $\identity$ copies the variables $\shortcatvariables$ to $\selvariableof{[\catorder]}$.
The energy tensor of this member of $\expfamilyof{\identity,\ones}$ is then
\begin{align*}
   \contractionof{\canparamat{\selvariableof{[\catorder]}},\identityat{\indexedshortcatvariables,\indexedselvariableof{[\catorder]}}}{\shortcatvariables}
    = \lnof{\probat{\shortcatvariables}}
\end{align*}
and thus
\begin{align*}
    \expdistofat{\identity,\canparam,\ones}{\shortcatvariables} = \normalizationof{\expof{\lnof{\probat{\shortcatvariables}}}}{\shortcatvariables} = \probat{\shortcatvariables} \, .
\end{align*}
We further note, that the mean parameter of this constructed element of $\expfamilyof{\identity,\ones}$ is
\begin{align*}
    \meanparamat{\selvariableof{[\catorder]}}
    = \contractionof{\expdistofat{\identity,\canparam,\ones}{\shortcatvariables},\identityat{\indexedshortcatvariables,\indexedselvariableof{[\catorder]}}}{\selvariableof{[\catorder]}}
    = \contractionof{\probwith,\identityat{\indexedshortcatvariables,\indexedselvariableof{[\catorder]}}}{\selvariableof{[\catorder]}} \, ,
\end{align*}
and thus coincides with the distribution itself, after a relabelling of the distributed variables.
Let us notice, that this family also correspond with the Markov Network on the maximal hypergraph $\maxgraph=(\nodes,\{\nodes\})$.
We will further revisit this family in \charef{cha:networkRepresentation}, where we will refer to it by the minterm family in order to connect with terminology developed for logical reasoning in \charef{cha:logicalReasoning}.



\sect{Polytopes of mean parameters}

We in this section investigate properties of probability distributions based on their mean parameters.
Given a statistic $\sstat$, we first define a mean parameter to any distribution by the expectation of the statistic.

\begin{definition}
    \label{def:meanPolytope}
    Let there be a statistic $\sstat$ and a boolean base measure $\basemeasurewith$.
    We call the tensor
    \begin{align*}
        \meanparamwith
        = \contractionof{\probat{\shortcatvariables},\sencsstatat{\shortcatvariables,\selvariable}}{\selvariable}
    \end{align*}
    the mean parameter tensor to a distribution $\probat{\shortcatvariables}$.
    The set
    \begin{align*}
        \genmeanset
        = \left\{\contractionof{\probtensor,\sencsstat,\basemeasure}{\selvariable}\wcols\probwith\in\bmrealprobof{\basemeasure} \right\},
    \end{align*}
    is called the polytope of realizable mean parameters.
    Here we denote by $\bmrealprobof{\basemeasure}$ the set of all probability distributions representable with respect to $\basemeasure$ (see \defref{def:representationBaseMeasure}).
\end{definition}

% Convex Hull Characterization Polytope
While introduced here as a property of a distribution, the mean parameters will be central to probabilistic inference in \charef{cha:probReasoning}.
We in the reminder of this section prepare for this application and derive tensor network representations for distributions having sufficient statistics $\sstat$, depending on their corresponding mean parameter in the polytope.


%% Mean parameters are expectation queries
%To prepare for the more detailled discussion of forward and backward inference in exponential families, we in this section investigate the polytope of mean parameters, sketched in \figref{fig:meansetSketchGeneric}.
%Given a pair of a statistic $\sstat:\facstates\rightarrow\parspace$ and a boolean base measure $\basemeasurewith$, the polytope of mean parameters is the set (see \defref{def:meanPolytope})
%\begin{align*}
%    \genmeanset
%    = \convhullof{\contractionof{\sencsstatwith,\probwith}{\selvariable} \, : \, \probwith\in\bmrealprobof{\basemeasure}} \, ,
%\end{align*}
%where by $\bmrealprobof{\basemeasure}$ we note the set of by $\basemeasure$ representable distributions (see \defref{def:representationBaseMeasure}).
%We in this section, we will further characterize the polytope of realizable mean parameters.

\subsect{Representation by convex hulls}

\begin{figure}[t!]
    \begin{center}
        \begin{tikzpicture}[scale=0.35]
    % Define points
    
    \node[below] at (-1,7) {$\sbinteriorof{\genmeanset}$};
    
    \coordinate (A) at (0,0);
    
    \node[below] at (A) {$\meanparamof{1}$};
    \draw[fill] (A) circle (0.15cm);
    
    \coordinate (B) at (12,2.5);
    \path (A) -- (B) coordinate[pos=0.7] (P1);

    \node[below] at (P1) {$\meanparamof{2}$};
    \draw[fill] (P1) circle (0.15cm);

    \coordinate (P2) at (2,10);
    \node[below] at (P2) {$\meanparamof{3}$};
    \draw[fill] (P2) circle (0.15cm);
    
    \coordinate (C) at (7.5,12);
    \path (B) -- (C) coordinate[pos=0.5] (P4);

    \node[right] at (P4) {$\genfacesetof{\canparam}$};
    \coordinate (D) at (-3,12);
    \coordinate (E) at (-10,5);

    \node[below] at ($0.5*(A)+0.5*(E)-(0,1.3)$) {$\genmeanset/\sbinteriorof{\genmeanset}$};

    \draw[thick] (A) -- (B) -- (C) -- (D) -- (E) -- cycle;

    \node[left] at (-9,11) {$\rr^\seldim$};

%    \coordinate (Or) at (-9,11);
%    \node[below] at (Or) {$\zerosat{\selvariable}$};
%    \drawvariabledot{-9}{11}

%    \draw (Or) -- (Or) + (−9.5,−4.5);

%    % Face normal
%    \draw[thick] (Or) -- ($(Or) + -0.6*(-9.5,-4.5)$) node[midway,below] {$\canparamwith$};
%    \draw[->,dashed] (Or) -- ($(Or) + -1.2*(-9.5,-4.5)$);
%    \draw[dashed] (Or) -- ($(Or) + -1.45*(-9.5,-4.5)$);
%    \draw[dashed] (B) -- ($(B)!1.57!(C)$);
%
%
%    \coordinate (int) at ($(Or) + -1.45*(-9.5,-4.5)$);
%    % angle
%    \draw[thick] ($(Or) + -1.22*(-9.5,-4.5)$) arc[start angle=-154.7, end angle=-64.4, radius=2.5cm];
%%    \coordinate (int) at ($(Or) + -1.45*(-9.5,-4.5)$);
%
%    \draw[fill] ($(Or) + -1.22*(-9.5,-4.5) + (1.75,0)$) circle (0.08cm);

    %\draw pic["", draw=black, angle radius=10, angle eccentricity=1.5, right angle symbol={draw}]{right angle=C--Or--B};
\end{tikzpicture}


    \end{center}
    \caption{Sketch of the mean polytope ${\genmeanset}$ to a statistic $\sstat$, which is minimal with respect to $\basemeasure$.
    The mean polytope is a bounded subset of $\parspace$ (here sketched as a 2-dimensional projection). %, where each mean parameter is one of the three cases $\meanparamof{1},\meanparamof{2}$ or $\meanparamof{3}$.
    For any vertex $\meanparamof{1}$ we find $\shortcatindices$ such that $\meanparamofat{1}{\selvariable}=\sencsstatat{\indexedshortcatvariables,\selvariable}$.
    Generic mean parameters $\meanparamof{2}$ outside the interior are on a face $\genfacesetof{\facecondset}$. %not reproduced by a member of the exponential family $\expfamily$, but by a member of the family $\expfamilyof{\sstat,\secbasemeasure}$ where $\secbasemeasure$ is a refined base measure deterimined by \algoref{alg:baseMeasureRefinement}.
%are reproducable by Hybrid Logic Networks with statistic $\mlnstat$ and refined base measure $\secbasemeasure$.
    Interior points $\meanparamof{3}\in\interiorof{\hlnmeanset}$ are exactly those reproducable by positive distributions with respect to $\basemeasure$.  %member of $\expfamily$, which canonical parameter is found by the backward map.
    }\label{fig:meansetSketchGeneric}
\end{figure}

First of all, we provide a simple characterization of the sets of mean parameters as the convex hull of the slices to the selection encoding of the statistic (see \figref{fig:meansetSketchGeneric}).
Convex hulls of finite vectors are called $\mathcal{V}$-polytopes (see Lecture~1 in \cite{ziegler_lectures_2013}).

\begin{theorem}
    \label{the:meanPolytopeConvHull}
    For any statistic $\sstat$ the polytope of mean parameters is the convex hull of the slices of $\sencsstat$ with fixed indices to $\shortcatvariables$, that is
    \begin{align*}
        \genmeanset
        = \convhullof{\sencsstatat{\indexedshortcatvariables,\selvariable}\wcols\shortcatindices\in\facstates\ncond\basemeasureat{\indexedshortcatvariables}=1} \, .
    \end{align*}
\end{theorem}
\begin{proof}
    First we realize that the characterization of by $\basemeasure$ representable distributions is a standard simplex extended by trivial coordinates, that is
    \begin{align*}
        \bmrealprobof{\basemeasure}
        = \convhullof{\onehotmapofat{\shortcatindices}{\shortcatvariables}\wcols\basemeasureat{\indexedshortcatvariables}=1 } \, .
    \end{align*}
    This follows from the fact, that the support of any by $\basemeasure$ representable distribution is contained in the support of $\basemeasure$.
    Further, each representable distribution is contained in the convex hull of the one-hot encoded support elements, since any distribution is normed.

    The polytope of mean parameters is a linear transform of the elements in $\bmrealprobof{\basemeasure}$, since the contraction with $\sencsstat$ is linear.
    It follows that
    \begin{align*}
        \genmeanset
        &= \convhullof{\contractionof{\sencsstatat{\shortcatvariables,\selvariable},\onehotmapofat{\shortcatindices}{\shortcatvariables}}{\selvariable} \, : \, \basemeasureat{\indexedshortcatvariables}=1 } \\
        &= \convhullof{\sencsstatat{\indexedshortcatvariables,\selvariable} \, : \, \basemeasureat{\indexedshortcatvariables}=1} \, .
    \end{align*}
\end{proof}

% Statistic encoding, universal statistic example,
We thus understand the map
\begin{align*}
    \sstatencoding : \facstates\rightarrow\genmeanset \defspace \catindex \rightarrow \sencsstatat{\indexedshortcatvariables,\selvariable}
\end{align*}
as an encoding of states with respect to a statistic $\sstat$.
When the statistic is the universal statistic $\sstat=\universalstat$, this statistic encoding coincides with the one-hot encoding and the mean polytope is the simplex of dimension $\contraction{\basemeasure}-1$ %$(\prod_{\catenumeratorin}\catdimof{\catenumerator})-1$
\begin{align*}
    \meansetof{\universalstat,\basemeasure}
    = \convhullof{\onehotmapofat{\shortcatindices}{\shortcatvariables}\wcols\basemeasureat{\indexedshortcatvariables}=1}\, .
\end{align*}
A generic mean polytope $\genmeanset$ is then the linear transform of the simplex by the contraction with $\sencsstatwith$, where $\selvariable$ is left open.%statistic encoding $\sstatencoding$.


\subsect{Representation as intersecting half-spaces}

For any vector $\normalvec[\selvariable]\in\rr^{\seldim}$ and a scalar $\normalbound\in\rr$, we call the set
\begin{align*}
    \left\{\meanparamwith \, : \, \contraction{\meanparamwith,\normalvec[\selvariable]} \leq \normalbound \right\} \subset \rr^{\seldim}
\end{align*}
a half-space of $\rr^\seldim$.
Bounded intersections of finitely many half-spaces are called $\mathcal{H}$-polytopes \cite{ziegler_lectures_2013}.
% Halfspace Representation
We state next, that the polytope $\genmeanset$ of mean parameters is a $\mathcal{H}$-polytopes.


\begin{theorem}
    \label{the:meanPolytopeHalfspaces}
    The set $\genmeanset$ is for any statistic $\sstat$ and base measure $\basemeasure$ a $\mathcal{H}$-polytope, i.e. there exists a finite collection
    \begin{align*}
        \halfspaceparams
    \end{align*}
    where $a_i[\selvariable]$ a vector and $b_i\in\rr$ for all $i\in[n]$ such that
    \begin{align*}
        \genmeanset
        = \left\{\meanparamwith \, : \, \forall_{i\in[n]} \, \contraction{\meanparamwith,\normalvecofat{i}{\selvariable}}\leq\normalboundof{i} \right\} \, .
    \end{align*}
\end{theorem}
\begin{proof}
    By \theref{the:meanPolytopeConvHull}, the set $\genmeanset$ is the convex hull of a finite set of vectors and is therefore a $\mathcal{V}$-polytope.
    We therefore apply the main theorem for polytopes \cite{motzkin_beitrage_1936}, which states the equivalence of $\mathcal{V}$-polytopes and $\mathcal{H}$-polytope, for which a proof can be found as Theorem~1.1 in \cite{ziegler_lectures_2013}.
    Therefore, $\genmeanset$ is also a $\mathcal{H}$-polytope and has is thus the intersection of finitely many half-spaces.
\end{proof}

The determination of the half-space parametrizing $\halfspaceparams$ is, however, in general difficult and the main reason for the intractability of probabilistic inference (see e.g. \cite{wainwright_graphical_2008}).


\subsect{Characterization of the interior}

The interior of the mean polytope consists of the mean parameters to positive distributions as we show next.

\begin{theorem}
    \label{the:meanPolytopeInteriorCharacterization}
    For any minimal statistics $\sstat$ with respect to a boolean base measure $\basemeasure$ (see \defref{def:minimalStatistics}) and a with respect to $\basemeasure$ positive distribution $\probwith$ we have
%    we have for some $\meanparamwith$ that $\meanparamwith\in\sbinteriorof{\genmeanset}$ if and only if there is a positive distribution with respect to $\basemeasure$ such that
    \begin{align*}
         %\meanparamwith
         %=
         \contractionof{\probwith,\sencsstatwith}{\selvariable} \in \sbinteriorof{\genmeanset} \, .
    \end{align*}
\end{theorem}
\begin{proof}
%    \proofrightsymbol: % LATER!
%    By \theref{the:meanPolytopeInterior} we find a canonical parameter $\canparamat{\selvariable}$ such that
%    \begin{align*}
%        \meanparamwith
%        = \contractionof{\expdistat{\shortcatvariables},\sencsstatwith}{\selvariable} \, .
%    \end{align*}
%    We notice, that $\expdist$ is positive with respect to $\basemeasure$, as is any member of an exponential family with base measure $\basemeasure$.

   % \proofleftsymbol: % Orient on proof of Theorem~3.3
    Since by assumption the statistics is minimal, the convex set $\genmeanset$ is full dimensional (see e.g. Appendix B in \cite{wainwright_graphical_2008}).
    We thus use a well-known property for full-dimensional convex sets (see \cite{rockafellar_convex_1997,hiriart-urruty_convex_1993}), that $\meanparam\in\interiorof{\genmeanset}$ if for any non-vanishing vector $\vectorat{\selvariable}$ there is a  % citations from Wainwright - Appendix B
    there is a $\tilde{\meanparam}[\selvariable]$ with
    \[ \contraction{\vectorat{\selvariable},\meanparamwith} <  \contraction{\vectorat{\selvariable},\tilde{\meanparam}[\selvariable]} \, . \]
    It thus suffices to show for an arbitrary non-vanishing vector $\vectorat{\selvariable}$ the existence of a distribution $\tilde{\probtensor}$, such that
    \begin{align*}
        \contraction{\vectorat{\selvariable},\meanparamwith} < \contraction{\vectorat{\selvariable},\sencsstatwith,\secprobat{\shortcatvariables}} \, .
    \end{align*}
    We define for $\epsilon\in\rr$
    \begin{align*}
        \probofat{\epsilon}{\shortcatvariables}
        = \normalizationof{\probat{\shortcatvariables},\expof{\epsilon\cdot\contractionof{\sencsstatwith,\vectorat{\selvariable}}{\shortcatvariables}}}{\shortcatvariables}
    \end{align*}
    The derivation of this map at $\epsilon=0$ is
    \begin{align*}
        \difwrt{\epsilon}\probofat{\epsilon}{\shortcatvariables}|_{\epsilon=0}
        = \contractionof{\probwith,\sencsstatwith,\vectorat{\selvariable}}{\shortcatvariables} - \contraction{\probwith,\sencsstatwith,\vectorat{\selvariable}} \cdot \probwith
    \end{align*}
    and thus
    \begin{align*}
        \difwrt{\epsilon} \contraction{\probofat{\epsilon}{\shortcatvariables},\sencsstatwith,\vectorat{\selvariable}}|_{\epsilon=0}
        &= \contractionof{\probwith,(\contractionof{\sencsstatwith,\vectorat{\selvariable}})^2}{\shortcatvariables} \\
        & \quad - \left(\contractionof{\probwith,\sencsstatwith,\vectorat{\selvariable}}{\shortcatvariables}\right)^2 \, .
    \end{align*}
    We can interpret this quantity as the variance of the random variable $\contractionof{\sencsstatwith,\vectorat{\selvariable}}{\indexedshortcatvariables}$, where $\shortcatindices$ is drawn from $\probwith$.
    The variance is greater than zero, if this random variable is not constant.
    But from the minimality of $\sstat$ with respect to $\basemeasure$ it follows, that this variable is not constant and we therefore have
    \begin{align*}
        0 < \difwrt{\epsilon} \contraction{\probofat{\epsilon}{\shortcatvariables},\sencsstatwith,\vectorat{\selvariable}}|_{\epsilon=0} \, .
    \end{align*}
    Thus, there is a $\epsilon>0$ with
    \begin{align*}
        \contraction{\vectorat{\selvariable},\meanparamwith} < \contraction{\vectorat{\selvariable},\sencsstatwith,\probofat{\epsilon}{\shortcatvariables}} \, .
    \end{align*}

\end{proof}

%
While \theref{the:meanPolytopeInteriorCharacterization} only states that the mean parameter of each positive distribution is in the interior, we will construct for each interior point a positive distribution in \charef{cha:probReasoning} by a member of the corresponding exponential family.

\subsect{Characterization of the boundary by faces}




Let us now continue with the investigation of the faces of the mean parameter polytope.%, which we define analogously to Def.~2.1 in \cite{ziegler_lectures_2013}. % ! Defined in Ziegler directly with normals

\begin{definition}
    \label{def:meanPolytopeFaces}
    Given a mean parameter polytope $\genmeanset$ in the half space representation of \theref{the:meanPolytopeHalfspaces}, and any subset $\mathcal{I}\subset[n]$ we say that the set
    \begin{align*}
        \genfacesetof{\facecondset}
        = \left\{\meanparamwith\in\genmeanset \wcols \forall_{i\in\mathcal{I}} \, \contraction{\meanparamwith,\normalvecofat{i}{\selvariable}}=\normalboundof{i} \right\}
    \end{align*}
    is the face to the constraints $\mathcal{I}$.
\end{definition}

While all inequalities in a half-space representation are satisfied for any element of the polytope, we defined faces by the additional sharp satisfaction of a subset of the half-space inequalities.
In this way, the faces build the boundary of $\genmeanset$.
This can be easily verified, since for any vector $\meanparamwith\in\genmeanset$, for which no halfspace inequalities hold sharply, also a neighborhood satisfies the halfspace inequalities.
If any halfspace inequality holds sharply, in the other case, the vector is a member of the corresponding face.

% Trivial face containing the whole polytope in case of non-minimal statistics
If $\sstat$ is not minimal with respect to $\basemeasure$, we find a non-vanishing vector $\vectorat{\selvariable}$ and a scalar $\lambda\in\rr$ such that
\begin{align*}
    \contractionof{\sencsstatat{\shortcatvariables,\selvariable},\vectorat{\selvariable},\basemeasurewith}{\shortcatvariables} = \lambda\cdot\basemeasurewith \, .
\end{align*}
This implies, that any probability distribution $\probwith$ representable with $\basemeasure$ satisfies
\begin{align*}
    \contraction{\probwith,\sencsstatat{\shortcatvariables,\selvariable},\vectorat{\selvariable},\basemeasurewith} = \lambda\cdot \contraction{\probwith,\basemeasurewith} = \lambda \, .
\end{align*}
Any $\meanparamwith\in\genmeanset$ then satisfies
\begin{align*}
    \contraction{\meanparamwith,\vectorat{\selvariable}} = \lambda \, .
\end{align*}
Thus, the polytope $\genmeanset$ is contained in an affine linear subspace and has vanishing interior.
We can further understand this equation as two half-space inequalities
\begin{align*}
    \contraction{\meanparamwith,\vectorat{\selvariable}} \leq \lambda \quad \text{and} \quad \contraction{\meanparamwith,\vectorat{\selvariable}} \geq \lambda \, ,
\end{align*}
which can be integrated into any half-space representation.
We conclude, that in the case of non-minimal statistics, the whole polytope $\genmeanset$ is a face itself, since it satisfies these half-space inequalities sharply.


%\subsect{Base measures on faces}

\begin{lemma}\label{lem:faceConvHullPreimage}
    For each face $\genfacesetof{\facecondset}$ we have
    \begin{align*}
        \genfacesetof{\facecondset}
        = \convhullof{\sencsstatat{\indexedshortcatvariables,\selvariable}\wcols\shortcatindices\in(\sstatencoding)^{-1}(\genfacesetof{\facecondset})\ncond\basemeasureat{\indexedshortcatvariables}=1} \, .
    \end{align*}
\end{lemma}
\begin{proof}
    This holds, since each face is the convex hull of the contained vertices (see Proposition~2.2 and 2.3 in \cite{ziegler_lectures_2013}).
    Since the vertices are contained in the image of the statistic encoding $\sstatencoding$, the vertices contained in $\genfacesetof{\facecondset}$ are contained in the set
    \begin{align*}
        \sencsstatat{\indexedshortcatvariables,\selvariable}\wcols\shortcatindices\in(\sstatencoding)^{-1}(\genfacesetof{\facecondset}) \, .
    \end{align*}
\end{proof}

\lemref{lem:faceConvHullPreimage} implies in particular, that faces are mean parameter polytopes with respect to refined base measures.
For reference in later chapters, we define these refined base measures next as face measures.

\begin{definition}
    \label{def:faceMeasure}
    The base measure to the face $\genfacesetof{\facecondset}$ of $\genmeanset$ is the boolean tensor
    \begin{align*}
        \basemeasureofat{\sstat,\facecondset}{\shortcatvariables}
        = \indicatorofat{\sstatencodingof{\shortcatindices}\in\genfacesetof{\facecondset}}{\shortcatvariables} \, .
    \end{align*}
\end{definition}

%
\begin{theorem}\label{the:faceAsRefinedPolytope}
    For any face $\genfacesetof{\facecondset}$ of $\genmeanset$, we have with the refined base measure
    \begin{align*}
        \secbasemeasureat{\shortcatvariables} = \contractionof{\basemeasureat{\shortcatvariables},\basemeasureofat{\sstat,\facecondset}{\shortcatvariables}}{\shortcatvariables}
    \end{align*}
    that
    \begin{align*}
        \genfacesetof{\facecondset} = \meansetof{\sstat,\secbasemeasure} \, .
    \end{align*}
\end{theorem}
\begin{proof}
    We notice that for any $\shortcatindices\in\facstates$, $\shortcatindices\in(\sstatencoding)^{-1}(\genfacesetof{\facecondset})$ is equal to $\basemeasureofat{\sstat,\facecondset}{\indexedshortcatvariables}=1$ and thus
    \begin{align*}
        \left\{\shortcatindices\wcols\shortcatindices\in(\sstatencoding)^{-1}(\genfacesetof{\facecondset})\ncond\basemeasureat{\indexedshortcatvariables}=1\right\}
        = \left\{\shortcatindices\wcols\secbasemeasureat{\indexedshortcatvariables}=1\right\} \, .
    \end{align*}
    In combination with \lemref{lem:faceConvHullPreimage} we then get
    \begin{align*}
        \genfacesetof{\facecondset}
        &= \convhullof{\sencsstatat{\indexedshortcatvariables,\selvariable}\wcols\shortcatindices\in(\sstatencoding)^{-1}(\genfacesetof{\facecondset})\ncond\basemeasureat{\indexedshortcatvariables}=1} \\
        &= \convhullof{\sencsstatat{\indexedshortcatvariables,\selvariable}\wcols\shortcatindices\wcols\secbasemeasureat{\indexedshortcatvariables}=1}
        &= \meansetof{\sstat,\secbasemeasure}
        \, .
    \end{align*}
\end{proof}

Positivity of a distribution with respect to face measures is an equivalent condition for the mean parameter of a distribution to be on a face, as we show next.

\begin{theorem}\label{the:facePolytopeCharacterization}
    If and only if for a distribution $\probwith$ and a face $\facecondset$ we have
    \begin{align*}
        \contractionof{\probwith,\sencsstatwith}{\selvariable}\in\genfacesetof{\facecondset}\, ,
    \end{align*}
    then $\probwith$ is representable with respect to the face measure $\genfacemeasure$.
\end{theorem}
\begin{proof}
    We have
    \begin{align*}
        \meanparamat{\selvariable} = \sum_{\shortcatindices} \probat{\indexedshortcatvariables}\cdot\genstatshortcatencoding \, .
    \end{align*}
    Now, the $\shortcatindices$ with $\genfacemeasureat{\indexedshortcatvariables}=1$ are exactly those, for which the conditions $\facecondset$ hold straight.
    If and only if for a $\shortcatindices$ with $\genfacemeasureat{\indexedshortcatvariables}=0$ we have $\probat{\indexedshortcatvariables}>0$, one of the conditions $\facecondset$ would not hold straight.
    Thus, if and only if $\probwith$ is representable with respect to $\genfacemeasureat{\shortcatvariables}$, we have $\meanparamat{\selvariable}\in\genfacesetof{\facecondset}$.
\end{proof}


Let us now investigate tensor network representations of face measures, based on the basis encoding $\bencodingof{\sstat}$ of a statistic.
% Vertices
Vertices of $\genmeanset$ are faces with single elements, that is $\{\meanparamwith\}$.
By \lemref{lem:faceConvHullPreimage} there must be $\meanparam$ must lie in the image of $\sstatencoding$, since otherwise $\genmeanset$ would be empty.
We denote $\headindex^{\facecondset}_{[\seldim]}$ as the index such that $\indexinterpretationofat{\sstat}{\headindex^{\meanparam}_{[\seldim]}}=\meanparam$.
The vertex measure is then
\begin{align*}
    \basemeasureofat{\sstat,\facecondset}
    = \contractionof{\bencodingofat{\sstat}{\headvariableof{[\seldim]},\shortcatvariables},\onehotmapofat{\headindex^{\meanparam}_{[\seldim]}}{\headvariableof{[\seldim]}}}{\shortcatvariables}
\end{align*}

\begin{theorem}[Face measure representation]\label{the:faceMeasureCharacterization}
    For any face $\genfacesetof{\facecondset}$ of $\meanset$ we have
    \begin{align*}
         \basemeasureofat{\sstat,\facecondset}{\shortcatvariables}
         =\contractionof{\bencodingofat{\sstat}{\headvariableof{[\seldim]},\shortcatvariables},\actcoreat{\headvariableof{[\seldim]}}}{\shortcatvariables}
    \end{align*}
    where
    \begin{align*}
        \actcoreat{\headvariableof{[\seldim]}}
        = \sum_{\meanparam\in\genfacesetof{\facecondset}\cup\imageof{\sstatencoding}} \onehotmapofat{\headindex^{\meanparam}_{[\seldim]}}{\headvariableof{[\seldim]}} \, .
    \end{align*}
\end{theorem}
\begin{proof}
    For any $\meanparam\in\genfacesetof{\facecondset}\cup\imageof{\sstatencoding}$ the tensor
    \begin{align*}
        \hypercoreofat{\meanparam}{\shortcatvariables}
        = \contractionof{\bencodingofat{\sstat}{\headvariableof{[\seldim]},\shortcatvariables},\onehotmapofat{\headindex^{\meanparam}_{[\seldim]}}{\headvariableof{[\seldim]}}}{\shortcatvariables}
    \end{align*}
    is the indicator of the preimage of $\meanparam$ under $\sstatencoding$.
    Since preimages of different $\meanparam$ are disjoint, the support of $\hypercoreofat{\meanparam}{\shortcatvariables}$ is disjoint and their sum
    \begin{align*}
        \sum_{\meanparam\in\genfacesetof{\facecondset}\cup\imageof{\sstatencoding}} \hypercoreofat{\meanparam}{\shortcatvariables}
    \end{align*}
    is the indicator of the preimage of $\genfacesetof{\facecondset}$ under $\sstatencoding$, which is the face measure $\basemeasureofat{\sstat,\facecondset}{\shortcatvariables}$.
    Exploiting linearity of contraction we have
    \begin{align*}
        \basemeasureofat{\sstat,\facecondset}{\shortcatvariables}
        &= \sum_{\meanparam\in\genfacesetof{\facecondset}\cup\imageof{\sstatencoding}} \hypercoreofat{\meanparam}{\shortcatvariables} \\
        &= \contractionof{\bencodingofat{\sstat}{\headvariableof{[\seldim]},\shortcatvariables},\sum_{\meanparam\in\genfacesetof{\facecondset}\cup\imageof{\sstatencoding}}\onehotmapofat{\headindex^{\meanparam}_{[\seldim]}}{\headvariableof{[\seldim]}}}{\shortcatvariables} \\
        &= \contractionof{\bencodingofat{\sstat}{\headvariableof{[\seldim]},\shortcatvariables},\actcoreat{\headvariableof{[\seldim]}}}{\shortcatvariables} \, .
    \end{align*}
\end{proof}

Let us notice, that $\actcoreat{\headvariableof{[\seldim]}}$ in \theref{the:faceMeasureCharacterization} is a sparse tensor with basis $\cpformat$ rank $\cardof{\genfacesetof{\facecondset}\cup\imageof{\sstatencoding}}$ (see \charef{cha:sparseCalculus}).


\sect{Discussion and Outlook}

% Hypergraphs
This chapter has established a foundational treatment of probability distributions by tensors, and motivated tensor network decompositions along classical approaches towards graphical models.
To show this correspondence, we defined both tensor networks and graphical models based on the same hypergraph.
This then enabled us to define Markov Networks simply as the normalizations of tensor networks with non-negative coordinates.
In the literature, tensor networks are, however, often treated as being dual to the graphs defining graphical models (see e.g. \cite{robeva_duality_2019}).
The duality becomes clear, when one interpretes the tensors as nodes and their common variables as edges, as might be natural given the applied notation of wiring diagrams to represent tensor networks.
We in this work avoid the discussion of this ambiguity, and treat tensors as decorations of hyperedges.

% Alternative names
In the literature, the tensors decorating hyperedges are often refered to as "factors" and their coordinatewise logarithm as "features" \cite{koller_probabilistic_2009}.
With the scope of this work, we avoided such further terminology.

Further, graphical models follow a tradition of definition on graphs, instead of hypergraphs.
Tensors, or "factors", are then assigned to maximal cliques.
We observed that the notion of maximality is an important assumption, for example in the proof of the Hammersley-Clifford theorem, and therefore introduced the property of clique capturing hypergraphs (see \defref{def:ccHypergraph}) to connect with this graph-based formalism.

% Arbitrary measureable spaces
While we here restricted our discussion to finite state spaces to each variable, probability distributions can in general be defined for arbitrary measureable spaces.
Joint distributions of these more generic variables still have a tensor structure.
The discussion of them, however, needs to be more careful, since integrals might diverge and tensors therefore not be normable.
By restriction in this work to finite state spaces of factored systems, we where able to exclude such situations.

%\begin{remark}[Alternative definitions of graphical models]
%    Further, we directly use hypergraphs instead of the more canonical association of factors with cliques of a graph.
%    This avoids the discussion of non-maximal cliques as decorated with trivial tensors.
%    Such hypergraphs follow the same line of though compared with factor graphs, which are bipartite graphs with nodes either corresponding with single variables or with a collection of them affected by a factor.
%\end{remark}








    \section{Probabilistic Reasoning}\label{cha:probReasoning} 

We have investigated means to store the knowledge about a system and now turn to the retrieval of information, a process called inference.

% 
Contraction of the relational encoding of a function with a Markov Network gives the statistics over the values of the functions.
When contracting the function directly, we get the expectation.

% Message passing
%Another approximation comes from an approximation of the contractions itself. 
One can increase the efficiency of inference algorithms by using approximative contractions.
Here, message passing schemes can be applied as to be introduced in \charef{cha:localContractions}.


\subsection{Queries}

% Motivation of queries: Avoid distribution instantiation
In the previous chapter, we have derived efficient representation schemes of probability distributions based on tensor network decompositions.
We have argued that one should avoid naive instantiation of these distributions based on an storage of each coordinates.
In the task of reasoning, we want to retrieve information encoded in the probability distribution.
To derive an efficient approach one therefore needs to avoid instantiating the distribution in a coordiantewise manner in an intermediate step.
We thus formalize a basic reasoning scheme by contractions of the decomposed distributions with query tensors.

\subsubsection{Querying by functions}

We can formalize queries by retrieving expectations of functions given a distribution specified by probability tensors. 
We exploit basis calculus in defining categorical variables $\catvariableof{\exfunction}$ to tensors $\exfunction$, which are enumerating the set $\imageof{\exfunction}$.
More details on this scheme are provided in \charef{cha:basisCalculus}, see \defref{def:functionRelationEncoding} therein.

\begin{definition}\label{def:queries}
	The marginal query of a probability distribution $\probof{\shortcatvariables}$ by a tensor 
		\[ \exfunction : \facstates \rightarrow \rr \]
	is the vector $\probof{\catvariableof{\exfunction}} \in \rr^{\cardof{\imageof{\exfunction}}}$ defined as the contraction
	\begin{align*}
		\probof{\catvariableof{\exfunction}} = \contractionof{\probof{\shortcatvariables},\rencodingofat{\exfunction}{\shortcatvariables,\catvariableof{\exfunction}}}{\catvariableof{\exfunction}} \, . 
	\end{align*}
	
	% Used in connection to mean parameters
	The expectation query of $\probtensor$ by $\exfunction$ is 
	\begin{align*}
		\expectationof{\exfunction} = \sbcontraction{\exfunction, \probtensor} \, . 
	\end{align*}
	
	% Used for sampling
	Given another tensor $\secexfunction: \facstates \rightarrow \rr $ the conditional query of the probability distribution $\probof{\shortcatvariables}$ by the tensor $\exfunction$ conditioned on the tensor $\secexfunction$ is the matrix $\condprobof{\catvariableof{\exfunction}}{\catvariableof{\secexfunction}}\in\rr^{\cardof{\imageof{\exfunction}}}\otimes \rr^{\cardof{\imageof{\secexfunction}}}$ defined as the normation
	\begin{align*}
		\condprobof{\catvariableof{\exfunction}}{\catvariableof{\secexfunction}} 
		= \normationofwrt{\{
		\probof{\shortcatvariables},\rencodingofat{\exfunction}{\shortcatvariables,\catvariableof{\exfunction}},\rencodingofat{\secexfunction}{\shortcatvariables,\catvariableof{\secexfunction}}
		\}}{
		\catvariableof{\exfunction}}{\catvariableof{\secexfunction}
		} \, . 
	\end{align*}
\end{definition}

%% Relation of queries and expectation queries
Expectation queries are contractions of marginal queries with identities, that is
	\[ \expectationof{\exfunction} = \sbcontraction{\probof{\catvariableof{\exfunction}} \idrestrictedto{\imageof{\exfunction}}{\catvariableof{\exfunction}} } \, . \]
This will be shown in more detail in \charef{cha:basisCalculus} in Corollary~\ref{cor:rhoToNormal}.

%% Conditional Probabilities and conditional queries
Conditional probabilities are queries, where the tensors $\exfunction$ and $\secexfunction$ are identity mappings in the respective variable state spaces.
Conversely, we can understand the conditional query $\condprobof{\exfunction}{\secexfunction}$ as the conditional probability of $\exfunction$ conditioned on $\secexfunction$, of the underlying Markov Network with cores $\{\probtensor, \rencodingof{\exfunction}, \rencodingof{\secexfunction} \}$ and variables $\catvariableof{\exfunction},\catvariableof{\secexfunction}$ besides the variables distributed by $\probtensor$.

%% Expectations as event queries -> Consistency with $\probof{X=i}$?
We further denote event queries by
	\[  \expectationof{\exfunction=z} = \sbcontraction{\probtensor,\rencodingof{\exfunction},\onehotmapof{z}}\]
where by $\onehotmapof{z}$ be denote the one hot encoding of the state $z$ with respect to some enumeration.
Let us note that they are further contraction of the queries in \defref{def:queries} since by \theref{the:splittingContractions}
\begin{align*}
	 \expectationof{\exfunction=z} 
	& =  \sbcontraction{ \sbcontractionof{\probtensor,\rencodingof{\exfunction}}{\catvariableof{\exfunction}} ,\onehotmapof{z}}\\
	& =  \sbcontraction{ \probof{\exfunction} ,\onehotmapof{z}} \, . 
\end{align*}

%% OLD: Defining queries by 
%\begin{definition}
%	The expectation of functions $\exfunction$ given a probability tensor is the contraction
%		\[ \expectationofwrt{\exfunction(\catvariables)}{\catvariables\sim\probtensor} = 
%			\contractionof{\{\probtensor,\rencodingof{\exfunction}\}}{\{\exfunctiontargetvariables \}} \, . 
%		\]
%\end{definition}
%This is the canonical definition of expectations, since summing function values weighted by the probability of the argument.
%When we have an unnormalized probability distribution $\phi$ the expectation is the quotient
%\begin{align*}
%	\expectationofwrt{\exfunction(\catvariables)}{\catvariables\sim\phi}  = \frac{
%		\contractionof{\{\phi,\ftensorof{\exfunction}\}}{\{\exfunctiontargetvariables \}} 
%	}{
%		\contractionof{\{\phi\}}{\varnothing} 
%	} \, . 
%\end{align*}

%\subsubsection{Conditional Probability Queries}
%
%Typical queries are the computation of an a posteriori distribution given evidence.
%This is just the contraction.
%
%%% As expectation
%The query consists of the one-hot encoding of the evidence and Ids elsewhere.
%The result is then interpreted as another probability distribution, defined as a Markov network and the possible need to normalize with the partition function.
%
%Given evidence, condition the probability tensor on that evidence.






\subsubsection{MAP Queries}

Find the maximal variable of a tensor is a problem, which can be approached by sampling methods as we discuss here.

\begin{definition}
	Given a tensor $\hypercore$ the MAP query is the problem 
	\begin{align}
		\argmax_{\catindices} \hypercoreat{\indexedcatvariables} \, .
	\end{align}
\end{definition}

%Often, the generation of a full (conditioned) probability tensor can be infeasible, if too many variables are queries.
%Having a tensor network decomposition of the probability tensor avoids this generation.

% One hot perspective
By coordinate calculus, we notice that
\begin{align}
	\hypercoreat{\indexedcatvariables} 
	\sbcontraction{\hypercore, \onehotmapof{\catindices}} \, .
\end{align}
Given the image $\Gamma^{\elformat}$ of one-hot encodings, the MAP query problem is equivalent to 
\begin{align}
	\max_{\catindices} \hypercoreat{\indexedcatvariables} 
	= \max_{\theta\in\Gamma^{\elformat}} \sbcontraction{\hypercore, \theta} \, .
\end{align}
We can thus understand MAP queries as a Tensor Network approximation problem, where the approximating tensor are the one-hot encodings of states.

\begin{remark}[MAP queries on energy and probability tensors]
% Usage on energies and probabilities
	Since the exponential function is monotonic, MAP queries on the energy tensor of an exponential family with uniform base measure are equivalent to MAP queries of their energies.
\end{remark}


\subsubsection{Answering queries by energy contractions}

Let us now interpret a probability tensor at hand as a member of an exponential family (see \secref{sec:exponentialFamilies}), which is always possible when taking the naive exponential family.

\begin{lemma}\label{lem:energyContractionQueries} % This is a statement about "full" queries.
	For any probability distribution $\probtensor$ with $\probtensor= \normationof{\expof{\energytensorat{\shortcatvariables}}}{\shortcatvariables}$, disjoint subsets $\nodesa,\nodesb \subset [\catorder]$ with $\nodesa\cup\nodesb=[\catorder]$  and any $\catindexof{\nodesb}$ we have
		\[ \condprobof{\catvariableof{\nodesa}}{\indexedcatvariableof{\nodesb}} 
			= \normationof{
				\expof{\energytensorat{\catvariableof{\nodesa},\indexedcatvariableof{\nodesb}}}
		}{\catvariableof{\nodesa}} \, .\]
\end{lemma}
\begin{proof}
	Since no summation is commuted.
\end{proof}

Thus, it suffices to build the selection encoding of the statistics, and we can avoid the usage of the relational encoding.

% 
We notice, that \lemref{lem:energyContractionQueries} does not generalize to situations, where $\nodesa\cup\nodesb\neq[\catorder]$, since summation over the indices of the variables $[\catorder]/\nodesa\cup\nodesb$ and contraction do not commute.
%\red{In that case, each summed index produces a factor.}


\begin{lemma}  %\red{TRUE?}
	For any probability distribution $\probtensor$ with $\probtensor= \normationof{\expof{\energytensorat{\shortcatvariables}}}{\shortcatvariables}$, disjoint subsets $\nodesa,\nodesb \subset [\catorder]$ and any $\catindexof{\nodesb}$ we have
		\[ \condprobof{\catvariableof{\nodesa}}{\indexedcatvariableof{\nodesb}} 
			=
			\normationof{
			 \sum_{\catindexofin{[\catorder]/\nodesa\cup\nodesb}} 
				 \expof{\energytensorat{\catvariableof{\nodesa},\indexedcatvariableof{\nodesb},\indexedcatvariableof{[\catorder]/\nodesa\cup\nodesb}}}
		}{\catvariableof{\nodesa}} \, .\]
\end{lemma}
\begin{proof}
	By splitting the contraction into terms to $\nodesa\cup\nodesb$. % and using \lemref{lem:energyContractionQueries}.
\end{proof}




\subsection{Sampling based on queries}


Let us here investigate how to draw samples from distributions $\probtensor$, based on queries on $\probtensor$.

%Need to generate the full conditional probability distribution by contraction and then sample from it.
Since there are $\prod_{\node\in\nodes}\catdimof{\node}$ coordinates stored in $\probtensor$, naive methods are often infeasible.
One can instead exploit a representation of $\probtensor$ by a Markov network or the energy term in an exponential family for efficient algorithms and sample from local proxy distributions resulting from contractions and interpreted as marginal and conditional probabilities.

\subsubsection{Exact Methods}

Forward Sampling (see Algorithm~\ref{alg:ForwardSampling}) uses a chain decomposition (see \theref{the:chainRule}) of a probability distribution to iteratively sample the variables.

\begin{algorithm}[hbt!]
\caption{Forward Sampling}\label{alg:ForwardSampling}
\begin{algorithmic}
\For{$\catenumeratorin$}
	\State Draw $\catindexof{\catenumerator}\in[\catdimof{\catenumerator}]$ from the conditional query
		\[ \condprobof{\catvariableof{\catenumerator}}{\indexedcatvariableof{\seccatenumerator} \, : \, \seccatenumerator < \catenumerator} \]
\EndFor
\end{algorithmic}
\end{algorithm}

%
Forward Sampling is especially efficient, when sampling from a Bayesian Network respecting the topological order of its nodes.
The reason for this lies in trivilizations of all conditional distributions, which heads are not included in the evidence of previously sampled variables.
More technically, we can show that
	\[ \condprobof{\catvariableof{\catenumerator}}{\indexedcatvariableof{\seccatenumerator} \, : \, \seccatenumerator < \catenumerator}  
	= \condprobof{\catvariableof{\catenumerator}}{\indexedcatvariableof{\parentsof{\catenumerator}}} \, , \]
which is only involving a single core of a Bayesian network.
\red{This can be shown using Corollary~\ref{cor:onesHead} to be derived in \charef{cha:basisCalculus}.}


%% Comment on rejection Sampling 
%When sampling from conditional probability distributions, one can sample from the conditioned distribution instead.
%However, the conditioning changes the structure of the distribution, and conditioned Bayesian Networks are not Bayesian Networks on the same graph.
%One ways around is rejection sampling, where one samples from the unconditioned distribution and rejects samples not satisfying the event conditioned on.
%When the event conditioned on is of small probability, methods like rejection sampling will come with large runtimes.

\subsubsection{Approximate Methods}

% Problem of many variables
When there are many variables to be sample, the computation of the conditional probability to all variables can be infeasible.
One way to overcome this is Gibbs Sampling: Iteratively resemble single variables given the rest as evidence.

%\subsubsection{Gibbs Sampling}

% Still old: Sample from Marginal
Sample each variable independent from the marginal distribution.
Then, alternate through the variables and sample each variable from the conditional distribution taking the others as evidence.

\begin{algorithm}[hbt!]
\caption{Gibbs Sampling}\label{alg:Gibbs}
\begin{algorithmic}
\For{$\catenumeratorin$}
	\State Draw State for atom $\catenumerator$ from initialization distributions. % In implementation: Initialize with ones and draw -> Avoids zero probability state
\EndFor
\While{Stopping criterion is not met}
\For{$\catenumeratorin$}
	\State Draw $\catindexof{\catenumerator}\in[\catdimof{\catenumerator}]$ from the conditional query
		\[ \condprobof{\catvariableof{\catenumerator}}{\indexedcatvariableof{\seccatenumerator}\, : \seccatenumerator \neq \catenumerator} \]
\EndFor
\EndWhile
\end{algorithmic}
\end{algorithm}


% Energy

Gibbs can be implemented based on the energy tensor $\energytensor$ of the probability tensor, as follows form the \lemref{lem:energyContractionQueries}.



%	\[ \condprobof{\catvariableof{\catenumerator}}{\{\catvariableof{\seccatenumerator}=\catindexof{\seccatenumerator} \, : \seccatenumerator \neq \catenumerator\}} 
%	= \normationofwrt{\expof{\contractionof{\{\energytensor\}\cup\{\onehotmapof{\catindexof{\seccatenumerator}} \, : \seccatenumerator \neq \catenumerator \}}{\catvariableof{\catenumerator}}}}{\catvariableof{\catenumerator}}{\varnothing}  \, .\]
	


\red{This is in contrast with forward sampling, where we need to sum over many coordinates of the exponentiated energy tensor, which amounts to the representation of the probability distribution as a tensor network using relational encodings.
}
%where the operation with energy tensors and selection encodings is not efficient.}




\subsubsection{Simulated Annealing}

\red{MAP queries are approximated by sampling from annealed distributions: Use $\hypercore$ as the energy tensor, e.g. as parameter tensor to the naive exponential family.}

%\begin{remark}\label{rem:simulatedAnnealing}
% Simulated annealing
	\red{Here by the naive exponential family!}
	Simulated annealing manipulates the probability used to sample $\catindexof{\catenumerator}$ in terms of an inverse temperature parameter $\invtemp$, by
		\[ \probtensor \rightarrow \frac{\expof{\invtemp\cdot\lnof{\probtensor}}}{\contraction{\expof{\invtemp\cdot\lnof{\probtensor}}} } \, . \]
	When the temperature is larger than $1$, the probability of states with low probability increases while the probability of states with large probability decreases and for low temperatures the opposite.
	Simulated annealing, that is the decrease of the temperature to $0$ during Gibbs sampling biases the algorithm towards states with large probability.
%	Tuning this parameter can improve the convergence of Gibbs Sampling.

	% On exponential families
	For any exponential family the transformation 
		\[ \energytensor \rightarrow \invtemp \cdot \energytensor  \]
	can be performed by rescaling the canonical parameters as
		\[ \canparam \rightarrow \invtemp \cdot \canparam \, . \]
%\end{remark}







\subsection{Maximum Likelihood Estimation} % Stuff from Parameter Estimation - Problem that Part I is called inference?

Let us now turn to inductive reasoning tasks, where a probabilistic model is trained on given data.

\subsubsection{Likelihood and Loss}

Given a datapoint $\datamapof{\dataindex}$ consisting of the images of the data selecting map $\datamap$ (see \defref{def:dataMap}), the likelihood given a Markov Logic Network is denoted as
	\[ \probat{\shortcatvariables = \datamapof{\dataindex}} \, . \]
	
% Independent assumption
When all $\datamapof{\dataindex}$ are drawn independently from $\probat{\shortcatvariablelist}$, we can factorize into
	\[ \probat{\data}  = \prod_{\dataindexin} \probof{\shortcatvariables=\datamapof{\dataindex}} \, . \]

% Logarithm
It is convenient to apply a logarithm on the objective, which does not influence the optimum when optimizing this quantity.
This is especially useful, when investigating the convergence of the objective for $\datanum\rightarrow\infty$ (see \charef{cha:mlnConcentration}).

\begin{definition}\label{def:loss}
	We define the loss of a distribution $\probtensor$ as
	\begin{align*}%\label{eq:defLikelihoodLossPL}
		\lossof{\probtensor} 
		= \frac{1}{\datanum} \lnof{\probof{\data}} 
	\end{align*}
\end{definition}

We now state the Maximum Likelihood Problem in the form
\begin{align}\tag{$\probtagtypeinst{\loss}{\Gamma,\empdistribution}$}\label{prob:parameterMaxLikelihood}
	\argmin_{\probtensor\in\Gamma} \lossof{\probtensor} \, . % Naive Exponential Family perspective!
\end{align}



\subsubsection{Entropic Interpretation}



\begin{definition}[Shannon entropy]
	The information content or the Shannon entropy of a distribution is defined as
		\[ \sentropyof{\probtensor} 
		:= \expectationofwrt{-\lnof{\probof{\shortcatvariables}}}{\shortcatvariables\sim\probtensor} 
		= \sbcontraction{\probtensor,-\lnof{\probtensor}} \, . \]
	%	= - \sum_{\shortcatindices} \probof{\indexedshortcatvariables} \cdot \lnof{\probof{\indexedshortcatvariables}} \, . \]
	We depict this in a tensor network diagram with an ellipsis denoting a coordinatewise transform (see \charef{cha:coordinateCalculus}) with a natural logarithm $\ln$ as:
	\begin{center}
		\begin{tikzpicture}[scale=0.3,thick] % , baseline = -3.5pt

\node[anchor=center] (text) at (-8,-5) {\corelabelsize $\sentropyof{\probtensor}$};

\node[anchor=center] (text) at (-5,-5) {\corelabelsize ${=}$};

\node[anchor=center] (text) at (-3,-2) {\corelabelsize $\mathrm{ln}$};
\draw (2,-2) ellipse (6 and 2.75);

\draw (-1,-1) rectangle (5,-3);
\node[anchor=center] (text) at (2,-2) {\corelabelsize $\probtensor$};
\draw (-1,-7) rectangle (5,-9);
\node[anchor=center] (text) at (2,-8) {\corelabelsize $\probtensor$};
\draw (0,-5)--(0,-3); 
\draw (0,-5)--(0,-7) node[midway,left] {\colorlabelsize $\catvariableof{0}$};
\draw (1.5,-5)--(1.5,-3); 
\draw (1.5,-5)--(1.5,-7) node[midway,left] {\colorlabelsize $\catvariableof{1}$};
\node[anchor=center] (text) at (3,-4) {$\cdots$};
\draw (4,-5)--(4,-3);
\node[anchor=center] (text) at (3,-6) {$\cdots$};
\draw (4,-5)--(4,-7) node[midway,right] {\colorlabelsize $\catvariableof{\catorder\shortminus1}$};

%\drawatomcore{3.5}{-8}{$\probtensor$}
%\drawatomindices{3.5}{-12}	
%\draw (5.5,-9)--(5.5,-7) node[midway,right] {\colorlabelsize $\catvariableof{\exformula}$};

\end{tikzpicture}
	\end{center}
\end{definition}

\begin{definition}[Cross entropy]\label{def:crossEntropy}
	The cross entropy between two distributions is defined as 
		\[ \centropyof{\probtensor}{\secprobtensor} 
		:=  \expectationofwrt{-\lnof{\secprobtensor[\shortcatvariables]}}{\shortcatvariables\sim\probtensor} 
		= \sbcontraction{\probtensor,-\lnof{\secprobtensor}} \, . \]
		%- \sum_{\catindices}  \probof{\indexedcatvariables} \cdot \lnof{\secprobtensor[\indexedshortcatvariables]}  \, . \]
	We depict this in a tensor network diagram with an ellipsis denoting a coordinatewise transform (here the $\ln$) as :
	\begin{center}
		\begin{tikzpicture}[scale=0.3,thick] % , baseline = -3.5pt

\node[anchor=center] (text) at (-8,-5) {\corelabelsize $\centropyof{\probtensor}{\tilde{\probtensor}}$};

\node[anchor=center] (text) at (-5,-5) {\corelabelsize ${=}$};

\node[anchor=center] (text) at (-3,-2) {\corelabelsize $\mathrm{ln}$};
\draw (2,-2) ellipse (6 and 2.75);

\draw (-1,-1) rectangle (5,-3);
\node[anchor=center] (text) at (2,-2) {\corelabelsize $\tilde{\probtensor}$};
\draw (-1,-7) rectangle (5,-9);
\node[anchor=center] (text) at (2,-8) {\corelabelsize $\probtensor$};
\draw (0,-5)--(0,-3); 
\draw (0,-5)--(0,-7) node[midway,left] {\colorlabelsize $\catvariableof{0}$};
\draw (1.5,-5)--(1.5,-3); 
\draw (1.5,-5)--(1.5,-7) node[midway,left] {\colorlabelsize $\catvariableof{1}$};
\node[anchor=center] (text) at (3,-4) {$\cdots$};
\draw (4,-5)--(4,-3);
\node[anchor=center] (text) at (3,-6) {$\cdots$};
\draw (4,-5)--(4,-7) node[midway,right] {\colorlabelsize $\catvariableof{\catorder\shortminus1}$};

%\drawatomcore{3.5}{-8}{$\probtensor$}
%\drawatomindices{3.5}{-12}	
%\draw (5.5,-9)--(5.5,-7) node[midway,right] {\colorlabelsize $\atomlegindexof{\exformula}$};

\end{tikzpicture}
	\end{center}
\end{definition}

%% Vanishing coordinates case
We here use $\lnof{0}=-\infty$ and $0\cdot \lnof{0} = 0$. 
Then we have $\centropyof{\probtensor}{\secprobtensor} = \infty$ if and only if there is a $\shortcatindices$ such that $\probof{\indexedshortcatvariables}>0$ and $\secprobtensor[\indexedshortcatvariables]=0$.


% KL Divergence
The Gibbs inequality states that
		\[ \centropyof{\probtensor}{\secprobtensor} \geq \sentropyof{\probtensor} \, . \]
The difference between both sides is called the Kullback Leibler Divergence and a useful metric in reasoning, since it vanishes for $\probtensor=\secprobtensor$.

\begin{definition}[Kullback Leibler Divergence]\label{def:KLDivergence}
	The KL divergence between two distributions is defined as 
		\[ \kldivof{\probtensor}{\secprobtensor} = \centropyof{\probtensor}{\secprobtensor} - \sentropyof{\probtensor}  \, . \]
\end{definition}

We are now ready to provide an entropic interpretation of the loss introduced in \defref{def:loss}.

\begin{theorem}\label{the:lossCentropy}
	Given a data selecting map $\datamap$ and a distribution $\probtensor$ we have
	\begin{align}
		\lossof{\probtensor} =  \centropyof{\empdistribution}{\probtensor} \, . % \sbcontraction{\empdistribution,\lnof{\probtensor}} \, . 
	\end{align}
\end{theorem}
\begin{proof}
	We have
	\begin{align*}
		\lossof{\probtensor} 
		& = \frac{1}{\datanum} \lnof{\probof{\data}} 
		= \frac{1}{\datanum} \sum_{\datain} \lnof{\probof{\shortcatvariables =\datamap(\dataindex)}} 
		= \frac{1}{\datanum} \sum_{\datain} \contraction{\{\lnof{\probtensor},\onehotmapof{\datamap(\dataindex)}\}} \\ 
		& = \sbcontraction{\empdistribution,\lnof{\probtensor}} \, .
	\end{align*}
	Comparing with the negative log likelihood we notice that that loss coincides with the cross-entropy between the empirical distribution $\empdistribution$ and $\probtensor$, i.e.
		\[ \lossof{\probtensor} = \centropyof{\empdistribution}{\probtensor} \, . \]
\end{proof}


% Interpretation of MLE as Cross-Entropy Minimization

We can therefore rewrite Problem~\ref{prob:parameterMaxLikelihood} as minimization of cross-entropies and of Kullback Leibler divergences as
\begin{align*}
	\argmin_{\probtensor\in\Gamma} \lossof{\probtensor} 
	= \argmin_{\probtensor\in\Gamma} \centropyof{\empdistribution}{\probtensor} 
	= \argmin_{\probtensor\in\Gamma} \kldivof{\empdistribution}{\probtensor} \, .
\end{align*}
	


% M-Projection -> A projection since P^2 = P, i.e. P applied on the image is id
Most general, the Maximum Likelihood Problem is the M-Projection of a distribution $\gendistribution$ onto a set $\Gamma$ of probability tensors is
\begin{align}\tag{$\mathrm{P}_{\Gamma, \gendistribution}$}\label{prob:mProjection}
	\argmax_{\probtensor\in\Gamma} \centropyof{\gendistribution}{\probtensor} 
\end{align}
where the Maximum Likelihood Estimation is the special case $\gendistribution=\empdistribution$.


\begin{example}[Cross entropy with respect to exponential families]\label{exa:cEntropyExp}
	If $\secprobtensor$ from an exponential family with boolean base measure, have with the representation from \lemref{lem:energyCumulantRepresentation}
	\begin{align*}
		\centropyof{\probtensor}{\expdist} 
		= \sbcontraction{\probtensor,\lnof{\expdist}} 
		= \sbcontraction{\probtensor,\sencsstat} - \cumfunctionof{\canparam} + \sbcontraction{\probtensor,\lnof{\basemeasure}} \, . 
	\end{align*}
	For the trivial base measure we can further exploit the existence of the energy tensor and have the representation
		\[ \centropyof{\probtensor}{\expdist} = \sbcontraction{\probtensor,(\expenergy-\cumfunctionof{\canparam}\cdot \ones)}
		=   \sbcontraction{\probtensor,\expenergy} -\cumfunctionof{\canparam} \, .   \]
\end{example}




\subsection{Forward Mapping in Exponential Families} 


%\red{Integrate: 
%Selection encodings suffice for variational methods, relational encodings of statistics are required for markov network instantiations of exponential families.}


%% Mean parameters are expectation queries
Mean parameter coordinates are expectation queries to $\sstatcoordinateof{\selindex}$, by 
	\[ \meanparamat{\indexedselvariable} = \expectationof{\sstatcoordinateof{\selindex}} \, . \]
	
%% Forward mappings are contractions, variational formulation as an alternative to avoid inefficiencies
Forward mappings have a closed form representation by
	\[ \forwardmapof{\canparam}
	= \sbcontractionof{\sencodingof{\sstat},\normationof{\basemeasure,\expof{\contraction{\sencodingof{\sstat},\canparam}}}{\shortcatvariables}}{\selvariable} \, . \]
% Infeasibility and turn to variational alternatives with selection encodings.
This contraction can, however, be infeasible, since it requires the instantiation of the probability tensor, which can be done by basis encodings of the statistic.
We in this section provide alternative characterization of the forward map and approximations of it, which can be computed based on the selection encoding instead.
Following \cite{wainwright_graphical_2008}, we can characterize the forward mapping to exponential families as a variational problem and provide an alternative characterization to this contraction.



\subsubsection{Variational Formulation}

Besides the direct computation of the mean parameter tensor we can give a variational characterization of the forward mapping.
This is especially useful, when the contraction is intractable, for example because the tensor $\expdist$ is infeasible to create.

\begin{theorem}
	We have
	\begin{align*}
		\forwardmapof{\canparam}
		  = \argmax_{\meanparam\in\genmeanset}  \sbcontraction{\meanparam,\canparam} + \sentropyof{\meanrepprob} 
	\end{align*}
	where by $\meanrepprob$ we denote a probability distribution with respect to a base measure $\basemeasure$, which reproduces the mean parameter $\meanparam$.
\end{theorem}
\begin{proof}
	Theorem~3.4 in \cite{wainwright_graphical_2008}.
\end{proof}

Let us now characterize the image of the forward map, which turns out to be the interior of the mean polytope, if the statistic is minimal (see \defref{def:minimalStatistics}).

\begin{theorem}\label{the:meanPolytopeInterior}
	For any statistics $\sstat$, which is minimal with respect to a base measure $\basemeasure$, the image $\imageof{\forwardmap}$ of the forward map is the interior of the convex polytope $\genmeanset$.
\end{theorem}
\begin{proof}
	Theorem 3.3 in \cite{wainwright_graphical_2008}.
\end{proof}

For the practicle usage of this theorem, we need a characterization of the interior of $\genmeanset$.

\begin{theorem}\label{the:meanPolytopeInteriorCharacterization}
	For any minimal statistics $\sstat$ and boolean base measure $\basemeasure$ we have for some $\meanparamat{\selvariable}$ that $\meanparamat{\selvariable}\in\genmeanset$ if and only if there is a positive distribution with respect to $\basemeasure$ such that
		\[ \meanparamat{\selvariable} = \sbcontractionof{\probtensor,\sencsstat}{\selvariable} \, . \]
%	If $\meanparamat{\selvariable}$ is in an minimal exponential family with boolean base measure $\basemeasure$, then it is in the interior of $\genmeanset$ if and only if it is representable by a positive distribution with respect to $\basemeasure$.
\end{theorem}
\begin{proof} 
	\proofrightsymbol: 
		By \theref{the:meanPolytopeInterior} we find a canonical parameter $\canparamat{\selvariable}$ such that
		\begin{align*}
			\meanparamat{\selvariable} = \sbcontractionof{\expdistat{\shortcatvariables},\sencsstatwith}{\selvariable} \, .
		\end{align*}
		We notice, that $\expdist$ is positive with respect to $\basemeasure$, as is any member of an exponential family with base measure $\basemeasure$.
		
	\proofleftsymbol: % Orient on proof of Theorem~3.3 
		Since by assumption the statistics is minimal, the convex set $\genmeanset$ is full dimensional (see e.g. Appendix B in \cite{wainwright_graphical_2008}). 
		We thus use a well-known property for full-dimensional convex sets (see \cite{rockafellar_convex_1997,hiriart-urruty_convex_1993}), that $\meanparam\in\interiorof{\genmeanset}$ if for any non-vanishing vector $\vectorat{\selvariable}$ there is a  % citations from Wainwright - Appendix B	
		there is a $\tilde{\meanparam}[\selvariable]$ with
			\[ \contraction{\vectorat{\selvariable},\meanparamat{\selvariable}} <  \contraction{\vectorat{\selvariable},\tilde{\meanparam}[\selvariable]} \, . \]
		It thus suffices to show for an arbitrary non-vanishing vector $\vectorat{\selvariable}$ the existence of a distribution $\tilde{\probtensor}$, such that
		\begin{align*}
			\contraction{\vectorat{\selvariable},\meanparamat{\selvariable}} < \contraction{\vectorat{\selvariable},\sencsstatwith,\secprobat{\shortcatvariables}} \, .
		\end{align*}
		We define for $\epsilon\in\rr$
		\begin{align*}
			\probofat{\epsilon}{\shortcatvariables} 
			= \sbnormationof{\probof{\shortcatvariables},\expof{\epsilon\cdot\contractionof{\sencsstatwith,\vectorat{\selvariable}}{\shortcatvariables}}}{\shortcatvariables}
		\end{align*}
		The derivation of this map at $\epsilon=0$ is 
		\begin{align*}
			\difwrt{\epsilon}\probofat{\epsilon}{\shortcatvariables}|_{\epsilon=0}
			= \contractionof{\probwith,\sencsstatwith,\vectorat{\selvariable}}{\shortcatvariables} - \contraction{\probwith,\sencsstatwith,\vectorat{\selvariable}} \cdot \probwith 
		\end{align*}
		and thus
		\begin{align*}
			\difwrt{\epsilon} \contraction{\probofat{\epsilon}{\shortcatvariables},\sencsstatwith,\vectorat{\selvariable}}|_{\epsilon=0}
			&= \contractionof{\probwith,(\contractionof{\sencsstatwith,\vectorat{\selvariable}})^2}{\shortcatvariables} \\
			 & \quad - \left(\contractionof{\probwith,\sencsstatwith,\vectorat{\selvariable}}{\shortcatvariables}\right)^2 \, . 
		\end{align*}
		We can interpret this quantity as the variance of the random variable $\contractionof{\sencsstatwith,\vectorat{\selvariable}}{\indexedshortcatvariables}$, where $\shortcatindices$ is drawn from $\probwith$.
		The variance is greater than zero, if this random variable is not constant.
		But from the minimality of $\sstat$ with respect to $\basemeasure$ it follows, that this variable is not constant and we therefore have
		\begin{align*}
			0 < \difwrt{\epsilon} \contraction{\probofat{\epsilon}{\shortcatvariables},\sencsstatwith,\vectorat{\selvariable}}|_{\epsilon=0} \, . 
		\end{align*}
		Thus, there is a $\epsilon>0$ with 
		\begin{align*}
			\contraction{\vectorat{\selvariable},\meanparamat{\selvariable}} < \contraction{\vectorat{\selvariable},\sencsstatwith,\probofat{\epsilon}{\shortcatvariables}} \, .
		\end{align*}

\end{proof}


\subsubsection{Boundary of convex polytopes}

For mean parameters $\meanparamat{\selvariable}$ outside the interior of $\genmeanset$ we know by \theref{the:meanPolytopeInteriorCharacterization}, that any distribution with mean parameter $\meanparamat{\selvariable}$ is not positive with respect to $\basemeasure$ and is therefore not in the exponential family.
We investigate this situation further and provide here a construction scheme to adapt the base measure such that there are exponential families containing these boundary distributions.

\begin{theorem}\label{the:faceToArgmax}
	Let there be a minimal $\sstat$ with respect to the base measure $\basemeasure$ and $\meanparamat{\selvariable}\notin\interiorof{\genmeanset}$.
	Then there is a $\canparamat{\selvariable}$ with 
		\[ \meanparamat{\selvariable} \in \argmax_{\meanparam\in\genmeanset} \contraction{\canparamat{\selvariable},\meanparamat{\selvariable}} \,  \]
	and all distributions with mean parameter $\meanparamat{\selvariable}$ are representable with respect to the base measure
		\[ \secbasemeasureat{\shortcatvariables} = \contractionof{\basemeasure, \indicatorofat{\arbset}{\shortcatvariables}}{\shortcatvariables} \, , \]
	where the indicator is on the set
		\[ \arbset = \argmax_{\shortcatindices} \contraction{\canparam,\sstat(\shortcatindices)}  \, . \]
\end{theorem}
\begin{proof}
	When $\meanparam\notin\interiorof{\genmeanset}$ we find a face such that $\meanparam\in\genfacesetof{\facecondset}$.
	The existence of $\canparamat{\selvariable}$ follows from \theref{the:faceNormal}, in which also a construction procedure is provided given a half-space representation (see \theref{the:meanPolytopeHalfspaces}).
	
	Now, we have 
	\begin{align*}
		 \meanparamat{\selvariable} \in \argmax_{\meanparam\in\genmeanset} \contraction{\canparamat{\selvariable},\meanparamat{\selvariable}} 
	\end{align*}
	and thus 
	\begin{align*}
		 \meanparamat{\selvariable} \in \convhullof{ \sencsstat{\indexedshortcatvariables,\selvariable} \, : \, 
		 \shortcatindices \in \argmax_{\shortcatindices \, : \, \basemeasureat{\indexedshortcatvariables}=1} \contraction{\canparamat{\selvariable},\sencsstat{\indexedshortcatvariables,\selvariable}} }
	\end{align*}	
	Thus, any distribution reproducing meanparam is a convex combination of the one-hot encodings of the states in $\argmax_{\shortcatindices} \contraction{\canparamat{\selvariable},\sencsstat{\indexedshortcatvariables,\selvariable}}$, and therefore representable with respect to the base measure $\secbasemeasure$.
\end{proof}

Each face of $\genmeanset$ thus defines a refinement of a base measure, which is sufficient to reproduce the mean parameters on that face.

\begin{definition}\label{def:faceBaseMeasure}
	The base measure to the face of $\meanset$ with normal $\canparam$ is
		\[ \basemeasureof{\sstat,\canparam} = \indicatorofat{\argmax_{\shortcatindices} \contraction{\canparam,\sstat(\shortcatindices)}}{\shortcatvariables} \, . \]
\end{definition}

\theref{the:faceToArgmax} therefore states, that when a mean parameter is on a face of $\genmeanset$, then each distribution reproducing the mean parameter has a representation with respect to the refined base measure
\begin{align*}
	\secbasemeasureat{\shortcatvariables} = \contractionof{\basemeasure,\basemeasureof{\sstat,\canparam}}{\shortcatvariables} \, . 
\end{align*}

% Base Measure Refinement algorithm
We now utilize these findings and provide in \algoref{alg:baseMeasureRefinement} a procedure to refine the base measure until the reduced mean parameter is in the open set of a reduced mean parameter polytope.

\begin{algorithm}[h!]
\caption{Base Measure Refinement}\label{alg:baseMeasureRefinement}
\begin{algorithmic}
\State \textbf{Input}: Base measure $\basemeasure$, statistic $\sstat$ and mean parameter $\meanparam\in\genmeanset$
\State \textbf{Output}: Refined base measure $\secbasemeasure$, remaining statistic $\secsstat$ and remaining mean parameter $\secmeanparam$
\hrule
%\State \noindent\rule{\linewidth}{0.4pt}
\While{$\meanparam\notin\sbinteriorof{\genmeanset}$}
	\While{$\sstat$ not minimal with respect to $\basemeasure$ (see \defref{def:minimalStatistics})}
		\State Find non-vanishing vector $\vectorat{\selvariable}$ and scalar $\lambda\in\rr$ such that 
			\[ \contractionof{\sencsstatat{\shortcatvariables,\selvariable},\vectorat{\selvariable},\basemeasureat{\shortcatvariables}}{\shortcatvariables} = \lambda\cdot\basemeasureat{\shortcatvariables} \, . \]
		\State Choose a coordinate $\selindexin$ with $\vectorat{\indexedselvariable}\neq0$ and drop it from $\sstat$ and $\meanparam$
	\EndWhile
	\State Find a non-trivial face (i.e. a non-empty face, which is a proper subset of $\genmeanset$) with normal $\canparam$, such that
		\[ \meanparam\in\genfacesetof{\canparam} \]
	\State Refine base measure
		\[ \basemeasure \algdefsymbol \contractionof{\basemeasure,\basemeasureof{\sstat,\canparam}}{\shortcatvariables} \]
\EndWhile
\State \textbf{return} $\basemeasure, \, \sstat,\,\meanparam$
\end{algorithmic}
\end{algorithm}

\begin{theorem}\label{the:baseMeasureRefinement}
	For arbitrary inputs $\basemeasure,\sstat$ and $\meanparam\in\genmeanset$, \algoref{alg:baseMeasureRefinement} terminates in finite time and outputs a triple of base measure $\secbasemeasure$, statistic $\secsstat$ and mean parameter $\secmeanparam$ such that the following holds.
	Any probability tensor $\probtensor$ reproducing $\meanparam$ is representable with respect to $\secbasemeasure$ and $\secmeanparam\in\sbinteriorof{\meansetof{\secsstat,\secbasemeasure}}$.
%	Thus, there is a member of the exponential family $\expfamilyof{\secsstat,\secbasemeasure}$ reproducing $\meanparam$.
\end{theorem}
\begin{proof}
	Let us first show, that \algoref{alg:baseMeasureRefinement} always terminates.
	The inner while loop of \algoref{alg:baseMeasureRefinement} always terminates, since $\sstat$ has a finite number of coordinates, and in each iteration one of the coordinates is dropped.
	To show that the outer while loop also terminates, it suffices to show, that the non-vanishing coordinates of the refined base measure are a proper subset of the base measure before refinement.
	But if this would not be the case, we would have 
		\[ \basemeasureat{\shortcatvariables} = \contractionof{\basemeasure,\basemeasureof{\sstat,\canparam}}{\shortcatvariables} \]
	and thus $\genfacesetof{\canparam}=\genmeanset$, which is a contradiction with the assumption of a non-trivial face.
	
	The second claim follows from an iterative application of \theref{the:faceToArgmax} and the fact, that a probability distribution reproduces $\meanparam$ in a non-minimal representation, if and only if it reproduces the corresponding reduced $\meanparam$ with respect to the reduced statistics.
\end{proof}


\begin{example}[Faces with normals parallel to one-hot encodings]
	To get some intuition how to represent face base measures, let us consider face normals $\canparam\in\{\lambda\cdot\onehotmapofat{\selindex}{\selvariable} \, : \, \selindexin, \, \lambda\in\rr/\{0\}\}$.
	We use relational encodings of the coordinates $\sstatcoordinateof{\selindex}$ of the statistic $\sstat$, with head variables $\catvariableof{\sstatcoordinateof{\selindex}}$ with dimension $\catdimof{\sstatcoordinateof{\selindex}}$ enumerating the image $\imageof{\sstatcoordinateof{\selindex}}\subset\rr$ in an ascending order.
	If $\canparamat{\selvariable}=\lambda\cdot\onehotmapofat{\selindex}{\selvariable}$ with $\lambda>0$, then $\argmax_{\shortcatindices} \contraction{\canparam,\sstat(\shortcatindices)}$ consists of states $\shortcatindices$ with minimal statistic $\sstatcoordinateofat{\selindex}{\indexedshortcatvariables}$, that is
		\[  \basemeasureofat{\sstat,\lambda\cdot\onehotmapof{\selindex}}{\shortcatvariables}
		 = \contractionof{\rencodingofat{\sstatcoordinateof{\selindex}}{\shortcatvariables,\catvariableof{\sstatcoordinateof{\selindex}}},
		 \onehotmapofat{\catdimof{\sstatcoordinateof{\selindex}}-1}{\catvariableof{\sstatcoordinateof{\selindex}}}}{\shortcatvariables}  \, . \]		
	If $\canparamat{\selvariable}=\lambda\cdot\onehotmapofat{\selindex}{\selvariable}$ with $\lambda<0$, then at the states with minimal statistic $\sstatcoordinateofat{\selindex}{\indexedshortcatvariables}$, that is
		\[  \basemeasureofat{\sstat,\lambda\cdot\onehotmapof{\selindex}}{\shortcatvariables}
		 = \contractionof{\rencodingofat{\sstatcoordinateof{\selindex}}{\shortcatvariables,\catvariableof{\sstatcoordinateof{\selindex}}},
		 \onehotmapofat{0}{\catvariableof{\sstatcoordinateof{\selindex}}}}{\shortcatvariables}  \, . \]
\end{example}


% Define sets of realizable distributions
\begin{theorem}
	For the maximal graph $\maxgraph=([\seldim],\{[\seldim]\})$, which has a single hyperedge containing all head variables we have
	\begin{align*}
		\genmeanset = \left\{ \contractionof{\probat{\shortcatvariables},\sencodingofat{\sstat}{\shortcatvariables,\selvariable}}{\shortcatvariables} \, , \, \probtensor \in \realizabledistsof{\sstat,\maxgraph} \right\}
	\end{align*}
\end{theorem}
\begin{proof}
	It is enough show, that for any output tuples $\secbasemeasure$, $\secsstat$ of the Base Measure Refinement \algoref{alg:baseMeasureRefinement} we have
		\[ \expfamilyof{\secbasemeasure,\secsstat} \subset  \realizabledistsof{\sstat,\maxgraph} \, . \]
	We notice, that the normation of any face base measure is realizable by $\realizabledistsof{\sstat,\maxgraph}$, since the objective in the maximation problem in \defref{def:faceBaseMeasure} depends only on $\sstat$.
	Providing a more technical argument, we have
	\begin{align*}
		\indicatorofat{\argmax_{\shortcatindices} \contraction{\canparam,\sstat(\shortcatindices)}}{\shortcatvariables}
		= \contractionof{
			\sstatcc,
			\sum_{\sstat(\shortcatindices) \, : \, \shortcatindices \in \argmax_{\shortcatindices} \contraction{\canparam,\sstat(\shortcatindices)} }
			\onehotmapofat{\indexinterpretationat{\sstat(\shortcatindices)}}{\sstatheadvariables}
		}{\shortcatvariables} \, .
	\end{align*}
	Since during the execution of \algoref{alg:baseMeasureRefinement}, $\secsstat$ is a subset of $\sstat$, we can find a corresponding $\canparamof{i}$ extending the face normal by vanishing coordinates to $\sstat$.
	We then have, that 
	\begin{align*}
		\secbasemeasure = \contractionof{
			\{\rencodingofat{\sstat}{\sstatheadvariables,\shortcatvariables}\} \cup
			\left\{\sum_{\sstat(\shortcatindices) \, : \, \shortcatindices \in \argmax_{\shortcatindices} \contraction{\canparamof{i},\sstat(\shortcatindices)} } 
			\onehotmapofat{\indexinterpretationat{\sstat(\shortcatindices)}}{\sstatheadvariables}
			: i \in [n] \right\}
			}{\sstatheadvariables} 
	\end{align*}
	represents the output base measure, where $i\in[n]$ label the faces chosen during in the loop of \algoref{alg:baseMeasureRefinement}.
	Now, any member $\expdistof{\secbasemeasure,\canparam,\secsstat}\in\expfamilyof{\secsstat,\secbasemeasure}$ can be represented by a member of  $\realizabledistsof{\sstat,\maxgraph}$, by contracting these base measure representing cores with the activation cores $\bigotimes_{\selindexin}\sstatacwith$.
\end{proof}

\subsubsection{Mode Search by annealing}

%% ANNEALING
Finding the mode of a distribution is related to the forward mapping of $\invtemp\cdot\canparam$: 
$\meanparam$ to a delta distribution (or in the convex hull of multiple maxima) in the limit.

% Annealing effect on the optimization problem
This is because 
\begin{align*}
	\argmax_{\meanparam\in\genmeanset}  \sbcontraction{\meanparam,\canparam}
\end{align*}
is taken at an extreme point in $\genmeanset$ (since linear objective over closed convex set), which is a delta distribution of a set and
\begin{align*}
	\argmax_{\meanparam\in\genmeanset}  \sbcontraction{\meanparam,\invtemp\cdot\canparam}+ \sentropyof{\meanrepprob} 
	= 
	\argmax_{\meanparam\in\genmeanset}  \sbcontraction{\meanparam,\canparam} + \frac{1}{\invtemp} \cdot \sentropyof{\meanrepprob} 	
\end{align*}
thus the entropy term is neglectible for large $\invtemp$.
A more precise argument is using a limit of the maxima and can be found in Theorem~8.1 in \cite{wainwright_graphical_2008}





\subsubsection{Mean Field Method}

We rewrite 
\begin{align*}
	\max_{\meanparam\in\genmeanset}  \sbcontraction{\meanparam,\canparam} + \sentropyof{\meanrepprob} 
	=
	\max_{\probtensor} \sbcontraction{\energytensor,\probtensor} + \sentropyof{\probtensor}
\end{align*}
where
	\[ \energytensor = \sbcontractionof{\sencsstat,\canparam}{\shortcatvariables} \, . \]

We now restrict the distributions in the maximum.
Typically we use the family of independent distributions, also called naive mean field method.
The naive mean field is the approximation by distributions of independent random variables $\legcoreof{\catenumerator}$, that is
\begin{align*}
	\argmax_{\legcoreof{\catenumerator} \, : \, \catenumeratorin} \contraction{\{\energytensor\} \cup \{\legcoreof{\catenumerator} \, : \, \catenumeratorin\}}
	+ \sum_{\catenumeratorin} \sentropyof{\legcoreof{\catenumerator}} \, . 
\end{align*}


\begin{theorem}[Update equations for the mean field approximation]
	Keeping all legs but one constant, the problem
	\begin{align*}
		\argmax_{\legcoreof{\catenumerator}} \contraction{\{\energytensor\} \cup \{\legcoreof{\catenumerator} \, : \, \catenumeratorin\}}
		+ \sum_{\catenumeratorin} \sentropyof{\legcoreof{\catenumerator}} 
	\end{align*}
	is solved at 
		\[ \legcoreofat{\catenumerator}{\catvariableof{\catenumerator}} 
			= \normationof{ \expof{ \contractionof{ \{\energytensor[\shortcatvariables] \} \cup
				\{\legcoreofat{\seccatenumerator}{\catvariableof{\seccatenumerator}} \, : \, \seccatenumerator\neq\catenumerator\} }{\shortcatvariables} }
			}{\catvariableof{\catenumerator}} \, . \]
\end{theorem}
\begin{proof}
	We have
	\begin{align*}
		 \difofwrt{\sentropyof{\legcoreof{\catenumerator}}}{\legcoreof{\catenumerator}}
		=  - \lnof{\legcoreofat{\catenumerator}{\catvariableof{\catenumerator}}}
		+ \onesat{\catvariableof{\catenumerator}}
	\end{align*}
	and by multilinearity of tensor contractions
	\begin{align*}
		\difofwrt{\contraction{\{\energytensor\}\cup\{\legcoreof{\seccatenumerator} \, : \, \seccatenumeratorin \}}}{\legcoreof{\catenumerator}}
		=  \contractionof{\{\energytensor\}\cup\{\legcoreof{\seccatenumerator} \, : \, \seccatenumeratorin ,\, \seccatenumerator\neq\catenumerator \}}{\catvariableof{\catenumerator}} \, . 
	\end{align*}
	Combining both, the condition
	\begin{align*}
		0 = \difofwrt{
			\left( \contraction{\{\energytensor\}\cup\{\legcoreof{\seccatenumerator} \, : \, \seccatenumeratorin \}} + \sum_{\catenumeratorin} \sentropyof{\legcoreof{\catenumerator}} \right)
		}{\legcoreof{\catenumerator}}
	\end{align*}
	is equal to
	\begin{align*}
		\lnof{\legcoreofat{\catenumerator}{\catvariableof{\catenumerator}}} =
		 \onesat{\catvariableof{\catenumerator}} + \contractionof{\{\energytensor\}\cup\{\legcoreof{\seccatenumerator} \, : \, \seccatenumeratorin ,\, \seccatenumerator\neq\catenumerator \}}{\catvariableof{\catenumerator}} \, .
	\end{align*}
	Together with the condition $\sbcontractionof{\legcoreof{\catenumerator}}=1$ this is satisfied at
		\[ \legcoreofat{\catenumerator}{\catvariableof{\catenumerator}} 
			= \normationof{ \expof{ \contractionof{ \{\energytensor\} \cup
				\{\legcoreof{\seccatenumerator} \, : \, \seccatenumerator\neq\catenumerator\} }{\catvariableof{\catenumerator}} }
			}{\catvariableof{\catenumerator}} \, . \]
\end{proof}



Algorithm~\ref{alg:NMF} is the alternation of legwise updates until a stopping criterion is met.

\begin{algorithm}[h!]
\caption{Naive Mean Field Approximation}\label{alg:NMF}
\begin{algorithmic}
\For{$\catenumeratorin$}
	\State 
		\[ \legcoreofat{\catenumerator}{\catvariableof{\catenumerator}} 
		\algdefsymbol \normationof{\ones}{\catvariableof{\catenumerator}}  \]
\EndFor
\While{Stopping criterion is not met}
	\For{$\catenumeratorin$}
		\State 
			\[ \legcoreofat{\catenumerator}{\catvariableof{\catenumerator}} 
			\algdefsymbol \normationof{ \expof{ \contractionof{ \{\energytensor[\shortcatvariables] \} \cup
				\{\legcoreofat{\seccatenumerator}{\catvariableof{\seccatenumerator}} \, : \, \seccatenumerator\neq\catenumerator\} }{\catvariableof{\catenumerator}} }
			}{\catvariableof{\catenumerator}} \]
\EndFor
\EndWhile
\end{algorithmic}
\end{algorithm}


\subsubsection{Structured Variational Approximation}

%% Structured Variational approximation
More generically, we restrict the maximum over the mean parameters of efficiently contractable distributions and get a lower bound.
In this section we use any Markov Network as the approximating family. 

Let $\graph$ be any hypergraph, we define the problem
\begin{align}\tag{$\mathrm{P}_{\mnexpfamily, \probtensor}$}\label{prob:structuredApproximation}
	\argmax_{\probtensor\in \mnexpfamily} \sbcontraction{\energytensor,\probtensor} + \sentropyof{\probtensor}
\end{align}

We approximate the solution of this problem again by an alternating algorithm, which iteratively updates the cores of the approximating Markov Network. 

\begin{theorem}[Update equations for the structured variational approximation]\label{the:updateEquationStructuredVariational}
	The Markov Network $\extnet$ with hypercores $\extnetasset$ is a stationary point for Problem~\ref{prob:structuredApproximation}, if for all $\edgein$
	\begin{align*}
	\hypercoreofat{\edge}{\edgevariables}
	= \lambda\cdot \expof{
	\frac{
		\contractionof{\{\energytensor\}\cup\{
		\hypercoreof{\secedge} : \secedge\neq\edge
		\}}{\edgevariables} 
	}{
		\contractionof{\{
		\hypercoreof{\secedge} : \secedge\neq\edge
		\}}{\edgevariables} 
	}
	- \sum_{\thirdedge\neq\edge} 
		\frac{
		\contractionof{\{\lnof{\hypercoreof{\thirdedge}}\}\cup\{
		\hypercoreof{\secedge} : \secedge\neq\thirdedge
		\}}{\edgevariables} 
	}{
		\contractionof{\{
		\hypercoreof{\secedge} : \secedge\neq\thirdedge
		\}}{\edgevariables} 
	}
	}
	\end{align*}
	for any $\lambda>0$ (e.g. by the norm).
	Here, the quotient denotes the coordinatewise quotient.
\end{theorem}
\begin{proof}%[Proof of \theref{the:updateEquationStructuredVariational}]
	We proof the theorem by first order condition on the objective $\objof{\extnet} = \sbcontraction{\energytensor,\extnetdist} + \sentropyof{\extnetdist}$.
	
	To proof the theorem, we use \lemref{lem:difMNExpectation}, which shows a characterization of the derivative of functions
	
	%% Energy Contraction Term
	We have %for $\probtensor\in\mnexpfamily$
	\begin{align*}
		\sbcontraction{\energytensor,\normationof{\extnet}{\shortcatvariables}} 
		=  \frac{
			\contraction{\{\energytensor\}\cup\extnet} 
		}{
			\contraction{\extnet} 			
		} \, . 
	\end{align*}
	
	%% Entropy Term Decomposition
	Further we have
	\begin{align*}
		\sentropyof{\normationof{\extnet}{\shortcatvariables}}
		= \left(\sum_{\secedge\in\edges} \contraction{-\lnof{\hypercoreof{\secedge}},\normationof{\extnet}{\shortcatvariables}} \right)
		+ \lnof{\contraction{\extnet}}	
	\end{align*}
	
	We define the tensor
		\[ \sechypercore[\catvariableof{\nodes}] = \energytensorat{\catvariableof{\nodes}} 
		- \sum_{\secedge\neq\edge} \lnof{\hypercoreofat{\secedge}{\catvariableof{\secedge}}} \otimes \onesat{\catvariableof{\nodes/\secedge}} \]
	and notice, that $\sechypercore$ does not depend on $\hypercoreof{\edge}$.	

	The objective has then a representation as
	\begin{align*}
		\objof{\extnet} = \sbcontraction{\sechypercore[\catvariableof{\nodes}], \extnetdist} - \sbcontraction{ \lnof{\hypercoreof{\edge}}, \extnetdist} +  \lnof{\contraction{\extnet}}
	\end{align*}
	
	Let us now differentiate all terms.
	With \lemref{lem:difMNExpectation} we now get
	\begin{align*}
		\difwrt{\hypercoreofat{\edge}{\seccatvariableof{\edge}}} \sbcontraction{\sechypercore[\catvariableof{\nodes}], \extnetdist}
		& = \sbcontractionof{\sechypercoreat{\nodevariables},
	 	\identityat{\seccatvariableof{\edge},\edgevariables}, 
		\frac{\contractionof{\extnet}{\edgevariables}}{\hypercoreofat{\edge}{\edgevariables}}, 
		\normationofwrt{\extnet}{\catvariableof{\nodes/\edge}}{\edgevariables} }{\seccatvariableof{\edge},\nodevariables} \\
		& \quad -  \contraction{\sechypercoreat{\nodevariables},\extnetdist}
		 \otimes \sbcontractionof{\frac{\contractionof{\extnet}{\seccatvariableof{\edge}}}{\hypercoreofat{\edge}{\seccatvariableof{\edge}}}
		}{\seccatvariableof{\edge}} \, .
	\end{align*}
	
	Further we have
	\begin{align*}
		\difwrt{\hypercoreofat{\edge}{\seccatvariableof{\edge}}} \sbcontraction{ \lnof{\hypercoreof{\edge}}, \extnetdist} 
		& = \sbcontractionof{\lnof{\hypercoreofat{\edge}{\edgevariables}},
	 	\identityat{\seccatvariableof{\edge},\edgevariables}, 
		\frac{\contractionof{\extnet}{\edgevariables}}{\hypercoreofat{\edge}{\edgevariables}}, 
		\normationofwrt{\extnet}{\catvariableof{\nodes/\edge}}{\edgevariables} }{\seccatvariableof{\edge},\nodevariables} \\
		& \quad -  \contraction{\lnof{\hypercoreofat{\edge}{\edgevariables}},\extnetdist}
		 \otimes \sbcontractionof{\frac{\contractionof{\extnet}{\seccatvariableof{\edge}}}{\hypercoreofat{\edge}{\seccatvariableof{\edge}}}
		}{\seccatvariableof{\edge}} \\
		& \quad\quad - \sbcontraction{ \frac{1}{\hypercoreofat{\edge}{\edgevariables}}, \extnetdist}
	\end{align*}
	and (see Proof of \ref{lem:difMNprob})
	\begin{align*}
		\difwrt{\hypercoreofat{\edge}{\seccatvariableof{\edge}}} \lnof{\contraction{\extnet}}
		 = \frac{\difwrt{\hypercoreofat{\edge}{\seccatvariableof{\edge}}} \contraction{\extnet}}{\contraction{\extnet}} 		
		 = \frac{\contractionof{\extnet}{\seccatvariableof{\edge}}}{\hypercoreofat{\edge}{\seccatvariableof{\edge}}} \, .
	\end{align*}
	
	Together, the first order condition
	\begin{align*}
		0 = \difwrt{\hypercoreofat{\edge}{\seccatvariableof{\edge}}} \objof{\extnet}
	\end{align*}
	is equal to all $\seccatindexof{\edge}$ satisfying% here drop seccatvariable to catvariable by slicing 
	\begin{align*}
		0 & = \frac{\contractionof{\extnet}{\indexedseccatvariableof{\edge}}}{\hypercoreofat{\edge}{\indexedseccatvariableof{\edge}}}
		 \Big(
		 	\sbcontraction{\sechypercoreat{\catvariableof{\nodes/\edge},\catvariableof{\edge}=\seccatindexof{\edge}}, \normationofwrt{\extnet}{\catvariableof{\nodes/\edge}}{\catvariableof{\edge}=\seccatindexof{\edge}}} \\
			&\quad \quad - \sbcontraction{\sechypercoreat{\nodevariables}, \extnetdist}  \\
			&\quad \quad - \sbcontraction{\lnof{\hypercoreofat{\edge}{\edgevariables=\seccatindexof{\edge}}}, \normationofwrt{\extnet}{\catvariableof{\nodes/\edge}}{\catvariableof{\edge}=\seccatindexof{\edge}}} \\
			&\quad \quad + \sbcontraction{\lnof{\hypercoreofat{\edge}{\edgevariables}}, \extnetdist} 
		 \Big) \, . 
	\end{align*}
	
	We notice, that by normation
		\[ \sbcontraction{\lnof{\hypercoreofat{\edge}{\edgevariables=\seccatindexof{\edge}}}, \normationofwrt{\extnet}{\catvariableof{\nodes/\edge}}{\catvariableof{\edge}=\seccatindexof{\edge}}} =  \lnof{\hypercoreofat{\edge}{\edgevariables=\seccatindexof{\edge}}} \]
	and that the scalar
		\[ \lambda_1 = \sbcontraction{\sechypercoreat{\nodevariables},\normationof{\extnet}{\catvariableof{\nodes}}}	
		- \sbcontraction{\lnof{\hypercoreofat{\edge}{\edgevariables}},\normationof{\extnet}{\catvariableof{\nodes}}}	\]
	is the constant for all $\seccatindexof{\edge}$.
	
	The first order condition is therefore equal to the existence of a $\lambda_1\in\rr$ such that for all $\seccatindexof{\edge}$ 
	\begin{align*}
		\lnof{\hypercoreofat{\edge}{\catvariableof{\edge}=\seccatindexof{\edge}}}
		= 	\sbcontraction{\sechypercoreat{\catvariableof{\nodes/\edge},\catvariableof{\edge}=\seccatindexof{\edge}}, 
		\normationofwrt{\extnet}{\catvariableof{\nodes/\edge}}{\catvariableof{\edge}=\seccatindexof{\edge}}} + \lambda_1 \, . 
	\end{align*}
	The claim follows when applying the exponential on both sides and with the observation, that 
	\begin{align*}
	\sbcontraction{\sechypercoreat{\catvariableof{\nodes/\edge},\catvariableof{\edge}=\seccatindexof{\edge}}, 
		\normationofwrt{\extnet}{\catvariableof{\nodes/\edge}}{\catvariableof{\edge}=\seccatindexof{\edge}}}
		= 
		\frac{\contractionof{\{\sechypercore\}\cup\{\hypercoreof{\secedge} \, : \, \secedge\neq \edge\}}{\catvariableof{\edge}=\seccatindexof{\edge}} }{
		\contractionof{\{\hypercoreof{\secedge} \, : \, \secedge\neq \edge\}}{\catvariableof{\edge}=\seccatindexof{\edge}} 
		}
	\end{align*}
	and reparametrization of $\lambda_1$ to
		\[ \lambda = \expof{\lambda_1} \, . \]
\end{proof}

%% KL Divergence
The mean field method corresponds with minimization of the KL Divergence to the efficiently contractable family, i.e. the I-projection onto the family.

\begin{theorem}
	For any hypergraph $\graph$ and energy tensor $\energytensor$ we have 
	\begin{align*}
		\argmax_{\probtensor\in \mnexpfamily} \sbcontraction{\energytensor, \probtensor}+ \sentropyof{\probtensor}
		= \argmax_{\probtensor\in \mnexpfamily} \kldivof{\expdistof{(\graph,\canparam)}}{\normationof{\expof{\energytensor}}{\shortcatvariables}}
	\end{align*}
	Problem~\ref{prob:structuredApproximation} is thus the I-projection onto the exponential family $\mnexpfamily$.
\end{theorem}
\begin{proof}
%	This follows from the fact, that the objective is the cross-entropy and the position of the maximum is invariant under substracting $\sentropyof{\probtensor}$.
	By rearranging the objective to the KL divergence.
\end{proof}








\subsection{Backward Mapping in Exponential Families}

%% FROM NETWORK LEARNING
The parameters optimizing the likelihood, will be shown to coincide with the backward mapping evaluated on the expectation of the sufficient statistics (see \theref{the:parEstToBackwardMap}).
This is in most generality true for the parameters of the M-projection of any distribution onto the exponential family.
We therefore investigate methods to compute the backward mapping, in most generality by alternating algorithms and in the special case of Markov Logic Networks by closed form representations.




%\begin{theorem}[Moment Matching Criteria]\label{the:MM}
	We have that $\canparam$ is a solution of the backward problem at $\genmean$, if and only if 
		\[ \sbcontractionof{\expdist,\sencsstat}{\selvariable} = \genmeanat{\selvariable} \, . \]
%\end{theorem}

This contraction equation is called moment matching, since the moment of the empirical distribution is matched by the moment of the fitting distribution.

We find one backward mapping as the dual problem to the forward mapping.


\subsubsection{Variational Formulation}

The backward mapping to $\datameanat{\selvariable} = \sbcontractionof{\empdistribution,\sencsstat}{\selvariable}$ is Maximum Likelihood estimation and the solution of the maximum entropy problem.

\begin{theorem}\label{the:varBackward}
	Let there be a sufficient statistic $\sstat$.
	The map $\backwardmap: \rr^{\seldim}\rightarrow \rr^{\seldim}$ defined as
	\begin{align*}
		\backwardmapof{\meanparam}
		= \argmax_{\canparam\in\rr^{\seldim}}  \sbcontraction{\meanparam,\canparam} - \cumfunctionof{\canparam} \, . 
	\end{align*}
	is a backward mapping.
\end{theorem}
\begin{proof}
	%\red{From duality, see Theorem~3.4 in \cite{wainwright_graphical_2008}.}
	We show the claim can be shown by the first order condition on the objective.	
	It holds that
	\begin{align*}
		\difwrt{\canparamat{\selvariable}}  \cumfunctionof{\canparam}  
		 & = \difwrt{\canparamat{\selvariable}}  \lnof{\contraction{\expof{\contractionof{\sencsstat,\canparam}{\shortcatvariables}}}} \\
		 & = \difwrt{\canparamat{\selvariable}} \frac{\contraction{\sencsstat[\selvariable],\expof{\contractionof{\sencsstat,\canparam}{\shortcatvariables}}}}{\contraction{\expof{\contractionof{\sencsstat,\canparam}{\shortcatvariables}}}}   \\
		 & = \forwardmapof{\canparam}[\selvariable]
	\end{align*}
	and thus
	\begin{align*}
		\difwrt{\canparamat{\selvariable}} \left( \sbcontraction{\meanparam,\canparam} - \cumfunctionof{\canparam}  \right) 
		= \meanparamat{\selvariable} -  \forwardmapof{\canparam}[\selvariable] \, . 
	\end{align*}
	
	The first order condition is therefore 
		\[ \meanparamat{\selvariable} =  \forwardmapof{\canparam}[\selvariable] \]
	and any $\canparam$ satisfies this condition exactly when $\canparam=\backwardmapof{\meanparam}$ for a backward map.
\end{proof}


\subsubsection{Interpretation by Maximum Likelihood Estimation}

% Backward mapping
Backward mapping coincides with the Maximum Likelihood Estimation Problem \eqref{prob:parameterMaxLikelihood}, when we take $\Gamma$ to the distributions in an exponential family $\expfamily$ for a sufficient statistic $\sstat$.

% Cross entropy
The loss is the cross entropy between a distribution with $\meanparam$ and the distribution $\expdistof{(\sstat,\canparam,\basemeasure)}$.


\begin{theorem}
	Let there be any exponential family, a mean parameter vector $\genmean\in\imageof{\forwardmap}$ and a backward map $\backwardmap$.
	Then $\estcanparam=\backwardmapof{\genmean}$ is the parameter of the M-projection (Problem~\ref{prob:mProjection}) of any $\gendistribution$ with $\sbcontractionof{\sencsstat,\gendistribution}{\selvariable}=\genmeanat{\selvariable}$ on to $\expfamily$, that is
		\[ \expdistof{(\sstat,\estcanparam,\basemeasure)} \in \argmax_{\probtensor\in\expfamily} \centropyof{\gendistribution}{\probtensor}  \, . \]
	In particular, if $\meanparam=\datamean$ for a data map $\datamap$, the backward map is a maximum likelihood estimator.
\end{theorem}
\begin{proof}
	We exploit the variational characterization of the backward map by \theref{the:varBackward}, and first show that the objective coincides with the cross entropy between the distribution $\gendistribution$ and the respective member of the exponential family.
	For any $\gendistribution$ and $\canparam$ we have with Example~\ref{exa:cEntropyExp}
	\begin{align*}
		\centropyof{\gendistribution}{\expdistof{(\sstat,\canparam,\basemeasure)}} 
		=   \sbcontraction{\gendistribution,\sencsstat,\canparam} -\cumfunctionof{\canparam} \, .  
	\end{align*}
	We use that by assumption $\sbcontractionof{\gendistribution,\sencsstat}{\selvariable}=\genmeanat{\selvariable}$ and thus
	\begin{align*}
		\centropyof{\gendistribution}{\expdistof{(\sstat,\canparam,\basemeasure)}} 
		=   \sbcontraction{\genmean,\canparam} -\cumfunctionof{\canparam} \, .  
	\end{align*}
	This shows, that the backward map coincides with the M-projection onto $\Gamma=\expfamily$.

	Further, if $\meanparam=\datamean$ for a data map $\datamap$, we have that the corresponding empirical distribution $\empdistribution$ satisfies $\sbcontractionof{\sencsstat,\empdistribution}{\selvariable}=\meanparamat{\selvariable}$.
	The backward map of $\meanparam$ is therefore the M-projection of $\empdistribution$, which is with \theref{the:lossCentropy} the maximum likelihood estimator.
\end{proof}


%\begin{lemma}
%	Let $\sstat\in\facspace\otimes\rr^{\seldim}$ be a sufficient statistic and $\gendistribution\in\facspace$ a probability distribution.
%	For any member $\expdist\in\expfamily$ we have
%		\[ \centropyof{\gendistribution}{\expdist} = \sbcontraction{\canparam,\genmean} - \cumfunctionof{\canparam} \]
%	where 
%		\[ \genmean = \sbcontractionof{\gendistribution,\sencsstat}{\selvariableof{\sstat}} \,  \]
%	and 
%		\[ \cumfunctionof{\canparam} = \lnof{\contraction{\expof{\expenergy}}} \, . \]
%	The M-projection of $\gendistribution$ onto $\expfamily$ is  $\expdistof{(\sstat,\estcanparam,\basemeasure)}$ for
%		\[ \estcanparam\in \argmax_{\canparam}  \sbcontraction{\canparam,\genmean} - \cumfunctionof{\canparam} \, .  \]
%\end{lemma}
%\begin{proof}
%	By decomposing 
%	\begin{align*}
%		\expdist 	& = \normationof{\expof{\sbcontractionof{\sencsstat,\canparam}{\shortcatvariables}}}{\shortcatvariables} \\
%				& = \frac{\expof{\expenergy}}{\sbcontraction{\expof{\expenergy}}}
%	\end{align*}
%	we get
%	\begin{align*}
%		\lnof{\expdist} & = \lnof{\expof{\expenergy}} - \onesat{\shortcatvariables} \cdot \sbcontraction{\expof{\expenergy}} \\ 
%		& = \expenergy - \cumfunction(\canparam) \cdot \onesat{\shortcatvariables}  \, .
%	\end{align*}
%	If follows that
%	\begin{align*}
%		\centropyof{\gendistribution}{\expdist} 
%		&=  \sbcontraction{\gendistribution,\lnof{\expdist}} \\
%		&=  \sbcontraction{\gendistribution,\expenergy} - \cumfunction(\canparam) \cdot \sbcontraction{\gendistribution}   \\
%		&= \sbcontraction{\canparam, \genmean} - \cumfunction(\canparam) \, . 
%	\end{align*}
%\end{proof}




%%\subsection{Maximum Likelihood and Maximum Entropy for Exponential Families}
%
%Parameter Estimation is the M-Projection of a distribution onto the exponential family.
%

%% DONE BEFORE!
%\begin{theorem}[\cite{wainwright_graphical_2008}]\label{the:parEstToBackwardMap}
%	Given any probability distribution $\probof{\shortcatvariables}$ and a exponential family defined by the sufficient statistic $\sstat$, the M-Projection onto the family is the distribution $\probtensorof{(\sstat,\estcanparam,\basemeasure)}$ where
%	\begin{align*}
%		\estcanparam = \backwardmapof{\contractionof{\probtensor,\sencsstat}{\selvariable}} \, .
%	\end{align*}
%\end{theorem}
%\begin{proof}
%	$\contractionof{\probtensor,\sencsstat}{\selvariable}$ is in $\imageof{\forwardmap}$ and MLE has a variational characterization with maximum at the dual $\estcanparam$, see \cite{wainwright_graphical_2008}.
%\end{proof}





\subsubsection{Connection with Maximum Entropy}\label{sec:maxEntDuality}


The Maximum entropy problem with respect to matching expected statistics $\genmean\in\genmeanset$ 
\begin{align}\tag{$\probtagtypeinst{\entropysymbol}{\sstat,\basemeasure,\genmean}$}\label{prob:maxEntropy}
	\argmax_{\probtensor\in\Gamma^{\basemeasure}} \sentropyof{\probtensor} \quad \text{subject to} \quad 
	 \sbcontractionof{\probtensor,\sencsstat}{\selvariable} =  \genmeanat{\selvariable}
\end{align}
where the optimization is over all the distributions $\Gamma^{\basemeasure}$, which are representable with respect to the base measure $\basemeasure$.

\begin{theorem}\label{the:maxEntInterior}
	Let $\sstat$ be a statistic and $\basemeasure$ a base measure.
 	For any $\genmean\in\sbinteriorof{\genmeanset}$ the solution of \probref{prob:maxEntropy} is the distribution $\expdistof{(\secsstat,\estcanparam,\secbasemeasure)}$, where $\estcanparam=\backwardmapwrtof{\secsstat,\secbasemeasure}{\secmeanparam}$.
\end{theorem}
\begin{proof}
	Since $\genmean\in\sbinteriorof{\genmeanset}$, \theref{the:meanPolytopeInteriorCharacterization} implies the existence of $\estcanparam$ such that 
		\[ \genmeanat{\selvariable} = \sbcontractionof{\expdistof{(\sstat,\estcanparam,\basemeasure)},\sencsstat}{\selvariable}   \, . \]
	We now follow the argumentation of the proof of Theorem~20.2 in \cite{koller_probabilistic_2009}.
	Let $\secprobtensor$ further be an arbitrary distribution, possibly different from $\expdistof{(\sstat,\estcanparam,\basemeasure)}$, such that
		\[ \genmeanat{\selvariable} = \sbcontractionof{\secprobtensor,\sencsstat}{\selvariable}  \, . \]
	We then have
	\begin{align*}
		\sentropyof{\expdistof{(\sstat,\estcanparam,\basemeasure)}}
		= \centropyof{\secprobtensor}{\expdistof{(\sstat,\estcanparam,\basemeasure)}}
	\end{align*}
	
	With the Gibbs inequality we have if $\secprobtensor\neq\expdistof{(\sstat,\estcanparam,\basemeasure)}$
	\begin{align*}
		\sentropyof{\expdistof{(\sstat,\estcanparam,\basemeasure)}} - \sentropyof{\secprobtensor}
		= \centropyof{\secprobtensor}{\expdistof{(\sstat,\estcanparam,\basemeasure)}} - \sentropyof{\secprobtensor} > 0 \, . 
	\end{align*}	
	
	Therefore, if $\secprobtensor$ does not coincide with$\expdistof{(\sstat,\estcanparam,\basemeasure)}$, it is not a solution of Problem~\ref{prob:maxEntropy}.
	%Classical result based on duality of maximum entropy and maximum likelihood, shown e.g. in Koller Book.
\end{proof}

% Interpretation
Let us highlight the fact, that in \probref{prob:maxEntropy} we did not restrict to distributions in an exponential family and only demanded representability with respect to the base measure.
When choosing the trivial base measure, this does not pose a restriction on the distributions.
\theref{the:maxEntInterior} states, that when the maximum entropy problem has a solution (i.e. $\genmean\in\genmeanset$), then the solution is in the exponential family to the statistic $\sstat$.

% Generalization
When $\genmean\notin\sbinteriorof{\genmeanset}$, the mean paramater is by \theref{the:meanPolytopeInteriorCharacterization} not reproducable by a member of the exponential family $\expfamilyof{\sstat,\basemeasure}$. 
Instead, in combination with the base measure refinement \algoref{alg:baseMeasureRefinement}, we show that the solution is in a refined exponential family.
% dropping the assumption that the mean parameters are in the interior of the mean parameter polytope.

\begin{theorem}\label{the:maxEntMaxLikeDuality} 
	Let $\sstat$ be a statistic and $\basemeasure$ a base measure.
	For any $\genmean\in\genmeanset$, let $\secsstat,\secbasemeasure$ and $\secmeanparam$ be the outputs of \algoref{alg:baseMeasureRefinement} when passing $\sstat,\basemeasure$ and $\genmean$ as input.
	Then, the distribution $\expdistof{(\secsstat,\estcanparam,\secbasemeasure)}$, where $\estcanparam=\backwardmapwrtof{\secsstat,\secbasemeasure}{\secmeanparam}$, solves \probref{prob:maxEntropy}.
\end{theorem}
\begin{proof}
	\theref{the:baseMeasureRefinement} and the above Lemma.
\end{proof}

% Minimality of the refined base measure
\theref{the:maxEntMaxLikeDuality} further implies, that the base measure $\secbasemeasure$ identified by \algoref{alg:baseMeasureRefinement} is minimal for the maximum entropy problem, in the sense that the solving distribution is positive with respect to it and all feasible distributions have to be representable by it.
This highlights the fact, that the maximum entropy distribution does not vanish beyond those states, which are necessary by \theref{the:baseMeasureRefinement}.


%\begin{theorem}\label{the:maxEntMaxLikeDuality} % In Koller Book, Theorem 20.2
%	If $\genmean\in\imageof{\forwardmap}$, we have that any distribution solving Problem~\ref{prob:maxEntropy} has a representation by $\expdistof{(\sstat,\estcanparam,\basemeasure)}$, 
%	where $\estcanparam=\backwardmapof{\genmean}$ for any backward map of the exponential family. 
%	%where $\estcanparam$ is the Maximum Likelihood Estimate with respect to any $\probtensor$ with $\sbcontractionof{\secprobtensor,\sencsstat}{\selvariable} =\genmean$.
%%
%%	Let $\sstat$ be a map and $\gendistribution$ be any distribution of $\atomstates$ and define
%%		\[ \genmeanat{\selvariable} = \sbcontractionof{\gendistribution,\sencsstat}{\selvariable} \, .  \]
%%	Then the solution of \ref{prob:maxEntropy} coincides with the member $\expdistof{(\sstat,\estcanparam,\basemeasure)}$ of the exponential family $\expfamily$ where
%%		\[ \estcanparam = \backwardmapof{\genmean} \]
%%	for a backward map $\backwardmap$ of $\expfamily$.
%\end{theorem}
%\begin{proof}
%	Since $\genmean\in\imageof{\forwardmap}$, there is a parameter $\estcanparam$ such that 
%		\[ \genmeanat{\selvariable} = \sbcontractionof{\expdistof{(\sstat,\estcanparam,\basemeasure)},\sencsstat}{\selvariable}   \, . \]
%	Let $\secprobtensor$ further be an arbitrary distribution such that
%		\[ \genmeanat{\selvariable} = \sbcontractionof{\secprobtensor,\sencsstat}{\selvariable}  \, . \]
%	We then have
%	\begin{align*}
%		\sentropyof{\expdistof{(\sstat,\estcanparam,\basemeasure)}}
%		= \centropyof{\secprobtensor}{\expdistof{(\sstat,\estcanparam,\basemeasure)}}
%	\end{align*}
%	
%	With the Gibbs inequality we have if $\secprobtensor\neq\expdistof{(\sstat,\estcanparam,\basemeasure)}$
%	\begin{align*}
%		\sentropyof{\expdistof{(\sstat,\estcanparam,\basemeasure)}} - \sentropyof{\secprobtensor}
%		= \centropyof{\secprobtensor}{\expdistof{(\sstat,\estcanparam,\basemeasure)}} - \sentropyof{\secprobtensor} > 0 \, . 
%	\end{align*}	
%	
%	Therefore, if $\secprobtensor$ does not coincide with$\expdistof{(\sstat,\estcanparam,\basemeasure)}$, it is not a solution of Problem~\ref{prob:maxEntropy}.
%	%Classical result based on duality of maximum entropy and maximum likelihood, shown e.g. in Koller Book.
%\end{proof}




\subsubsection{Alternating Algorithms to Approximate the Backward Map}\label{sec:alternatingBackwardMap}


\red{While the forward map always has a representation in closed form by contraction of the probability tensor, the backward map in general fails to have a closed form representation.
Computation of the Backward map can instead be performed by alternating algorithms, as we show here.} % Are these fixpoint iterations?


Alternate through the coordinates of the statistics and adjust $\canparamat{\indexedselvariable}$ to a minimum of the likelihood, i.e. where for any $\selindexin$
\begin{align*}
	0 = \frac{\partial}{\partial \canparamat{\indexedselvariable}} \lossof{\expdist} \, . 
\end{align*}

% Moment matching
This condition is equal to the collection of moment matching equations % (see \theref{the:mm})
\begin{align*}
	\sbcontractionof{\expdist,\sencsstat}{\indexedselvariable} = \sbcontraction{\empdistribution,\sencsstat}{\indexedselvariable} \, . 
\end{align*}


\begin{lemma}\label{lem:mmContractionEquation}
	For any sufficient statistic $\sstat$ a parameter vector $\canparam$ and a $\selindexin$ we define
	\begin{align*}
	 	\hypercoreat{\catvariableof{\sstatcoordinateof{\selindex}}} 
		= \contractionof{\{\sstatcc\}\cup\{\headcoreof{\tilde{\selindex}} : \tilde{\selindex} \in [\seldim], \tilde{\selindex}\neq\selindex\}}{\catvariableof{\sstatcoordinateof{\selindex}}} \, . 
	\end{align*}
	Then the moment matching condition for $\sstatcoordinateof{\selindex}$ relative to $\canparam$ and $\meanparam$ is satisfied for any $\canparamat{\indexedselvariable}$ with
	\begin{align*}
		\sbcontraction{\headcoreof{\selindex}, \idrestrictedto{\imageof{\sstatcoordinateof{\selindex}}}, \hypercoreat{\selvariable_\sstat}}
		= \sbcontraction{\headcoreof{\selindex}, \hypercoreat{\selvariable_\sstat}} \cdot \meanparamat{\indexedselvariable} \, . 
	\end{align*}
\end{lemma}
\begin{proof}
	We have
	\begin{align*}
		\expdist = \frac{
			\sbcontractionof{\headcoreof{\selindex}, \hypercore}{\shortcatvariables}
		}{
			\sbcontraction{\headcoreof{\selindex}, \hypercore}
		}
	\end{align*}
	and 
	\begin{align*}
		\sbcontraction{\expdist, \sstatcoordinateof{\selindex}}
		= \frac{
			\sbcontractionof{\headcoreof{\selindex}, \idrestrictedto{\imageof{\sstatcoordinateof{\selindex}}}, \hypercore}{\shortcatvariables}
		}{
			\sbcontraction{\headcoreof{\selindex}, \hypercore}
		} \, . 
	\end{align*}
	Here we used
		\[ \sstatcoordinateof{\selindex} = \sbcontractionof{\headcoreof{\selindex}, \idrestrictedto{\imageof{\sstatcoordinateof{\selindex}}}}{\shortcatvariables} \]
	and redundancies of copies of relational encodings.
	It follows that 
	\begin{align*}
		\sbcontraction{\expdist,\sstatcoordinateof{\selindex}} = \contraction{\empdistribution,\sstatcoordinateof{\selindex}}
	\end{align*}
	is equal to
	\begin{align*}
		\sbcontraction{\headcoreof{\selindex}, \idrestrictedto{\imageof{\sstatcoordinateof{\selindex}}}, \hypercoreat{\catvariableof{\sstatcoordinateof{\selindex}}}}
		= \sbcontraction{\headcoreof{\selindex},\hypercoreat{\catvariableof{\sstatcoordinateof{\selindex}}}} \cdot \meanparamat{\indexedselvariable} \, . 
	\end{align*}	
\end{proof}

% Alternation necessary
The steps have to be alternated until sufficient convergence, since matching the moment to $\selindex$ by modifying $\canparamat{\indexedselvariable}$ will in general change other moments, which will have to be refit.


%Coordinate descent
An alternating optimization is the coordinate descent of the negative likelihood, seen as a function of the coordinates of $\canparam$, see Algorithm~\ref{alg:AMM}.
Since the log likelihood is concave, the algorithm converges to a global minimum.



\begin{algorithm}[h!]
\caption{Alternating Moment Matching}\label{alg:AMM}
\begin{algorithmic}
\State Set $\canparamat{\selvariable}=0$
\State Compute $\datameanat{\selvariable}= \sbcontractionof{\empdistribution,\sencsstat}{\selvariable}$
%\For{$\selindexin$}
%	\State Set $\canparamat{\indexedselvariable}=0$ 
%	\State Compute $\meanparamat{\indexedselvariable}^{\datamap} = \contractionof{\{\empdistribution,\sstatcoordinateof{\selindex}\}}{\varnothing} $ % Or give those as input!
%\EndFor
\While{Stopping criterion is not met}
\For{$\selindexin$}
	\State Compute 
		\begin{align*}
			\hypercoreofat{\selindex}{\catvariableof{\sstatcoordinateof{\selindex}}} 
			\algdefsymbol \contractionof{\{\sstatcc\}\cup\{\headcoreof{\tilde{\selindex}} : \tilde{\selindex} \in [\seldim], \tilde{\selindex}\neq\selindex\}}{\catvariableof{\sstatcoordinateof{\selindex}}} 
		\end{align*}
	\State Set $\canparamat{\indexedselvariable}$ to a solution of 
	\begin{align*}
		\sbcontraction{\headcoreof{\selindex},\idrestrictedto{\imageof{\sstatcoordinateof{\selindex}}},\hypercoreof{\selindex}}
		\algdefsymbol \sbcontraction{\headcoreof{\selindex},\hypercoreof{\selindex}} \cdot \datameanat{\indexedselvariable} \, . 
	\end{align*}
\EndFor
\EndWhile
\end{algorithmic}
\end{algorithm}


% 
In general, if $\imageof{\sstatcoordinateof{\selindex}}$ contains more than two elements, there exists no closed form solutions.
We will investigate the case of binary images, where there are closed form expressions, later in \secref{sec:alternatingParEstMLN}.


%
The computation of $\hypercoreof{\selindex}$ in Algorithm~\ref{alg:AMM} can be intractable and be replaced by an approximative procedure based on message passing schemes.

\subsection{Discussion}

% Forward mapping as gradient of A
Further in \cite{wainwright_graphical_2008}: Convex Duality.
Forward mapping coincides with gradient, i.e. $\meanparam = \nabla \cumfunction(\canparam)$.

% Gradient property of the backward mapping
In \cite{wainwright_graphical_2008}:
The objective is the conjugate dual $\dualcumfunction$ of $\cumfunction$, and backward mapping has an expression by the gradient, i.e. $\canparam = \nabla \dualcumfunction(\meanparam)$.




    \section{Propositional Logics}\label{cha:FormulaTensors}

Propositional logics describes systems with $\atomorder$ binary categorical variables, which are called atoms and denoted by $\atomicformulaof{\atomenumerator}$ for $\atomenumeratorin$.
Indices $\atomlegindexof{\atomenumerator}\in[2]$ to the atoms $\atomenumeratorin$ enumerate the $2^\atomorder$ states of these systems, which are called worlds.
In each world indexed by $\atomindices$ the indices $\atomicformulaof{\atomenumerator}$ encode whether the corresponding variable is $\truesymbol$. 

% Structure
We here choose a semantic centric approach to propositional logic, by defining formulas as binary tensors.
Then we investigate the corresponding syntax of formulas as specification of a tensor network decomposition of the relational encoding of formulas.


\subsection{Encoding of Booleans}

Propositional logic amounts to reason about Boolean variables, which are categorical variables with $2$ possible values.
Before applying this insight in the representation of propositional formulas, we first investigate how Boolean calculus can be represented by multilinear operations.

\subsubsection{Booleans as categorical variables}


%\begin{remark}[Boolean Calculus to Binary Calculus] % This is Coordinate Calculus, Happening on the Coordinates during Binary Tensor Contractions
	%We use an embedding of truth assignments to $\{0,1\}=[2]$ and store in large vectors, restructured as tensors, truths to sets of formulas.
	To represent Booleans by categorical variables $\catvariable$ with two states we use the following standard group homomorphism % (this is standard, also build in python)
		\[ \big(\{\truesymbol,\falsesymbol\},\land\big) \quad \text{and} \quad \big(\{ 1,0\},\cdot\big) \]
	by the map
    		\[ [\cdot]:\{\truesymbol,\falsesymbol\} \rightarrow \{1,0\} \]
	defined as
	    	\[ [\truesymbol] = 1 \quad , \quad [\falsesymbol] = 0 \, . \]
	The multiplication is performed in the binary tensor contractions and can thus be interpreted as the $\land$ connective performed on Boolean coordinates.
%\end{remark}


% Expressivity Issues
While the conjunction of is in this embedding performed by multiplications, operations like the negation
	\[ [\lnot X] = 1 - [X]  \]
are affine linear.
Direct applications of these affine linear operations to perform logical calculus will be discussed later in Section~\ref{sec:effectiveGroundingCalculus}.

However, in this chapter we will circumvent the problems arising with affine linearity by using the one-hot encoding of Booleans.


\subsubsection{One-hot Encoding} % This is what Basis Calculus does! Refer to that here?

Booleans are categorical variables with $\catdim=2$ states, where we interpret the states $[2]=\{0,1\}$ by $\{\truesymbol,\falsesymbol\}$.
The one-hot encoding of Booleans
	\[ \onehotmap: [2] \rightarrow \{\fbasis[\catvariable] ,\tbasis[\catvariable] \}  \subset \rr^2 \]
is thus understood as an encoding of truth values.

%% Expressivity
As discussed before, the one-hot encoding is rich enough to represent any function of the state by a linear function on the encoding.
The truth of formulas is a function of the truth of atomic formulas, and thus representable by linear functions of the one-hot encodings.




\subsection{Semantics of Propositional Formulas}

The epistemological commitments are whether the state is $\truesymbol$ or $\falsesymbol$ reflected by the coordinate of the one-hot encoding being $1$ or $0$.
Intuitively this describes, whether a specific world can be the state of a factored system.

\subsubsection{Formulas}

%% Intro of formulas
Logics is especially strong in interpreting binary tensors representing Propositional Knowledge Bases, based on connections with abstract human thinking.
To make this more precise, we associate each such tensor is associated with a formula $\exformula$ being a composition of the atomic variables with logical connectives as we proof next.


\begin{definition}\label{def:formulas}
	A propositional formula $\formulaat{\catvariables}$ depending on $\atomorder$ atoms $\catvariableof{\atomenumerator}$ is a tensor
		\[ \formulaat{\catvariables} : \atomstates \rightarrow [2] \subset \rr \, . \]
	We call $\atomindices \in \atomstates$ a model of a propositional formula $\formula$, if 
		\[ \formulaat{\indexedcatvariables}=1 \].
	If there is a model to a propositional formula, say the formula is satisfiable.
\end{definition}

% Binary Tensors
The propositional formulas coincide therefore with the binary tensors (see Definition~\ref{def:binaryTensor}).


% Decomposition into model sums
Since propositional formulas are binary valued tensors, the generic decomposition of Lemma~\ref{lem:tensorBasisDecomposition} simplifies to
\begin{align}\label{eq:formulaModelDecomposition}
	\formulaat{\catvariables} = \sum_{\catindices\in\atomstates} \formulaat{\indexedcatvariables} \cdot \onehotmapofat{\catindices}{\catvariables} \\
	= \sum_{\catindices\in\atomstates \, : \, \formulaat{\indexedcatvariables}=1}  \onehotmapofat{\catindices}{\catvariables} \, .
\end{align}
Thus, any propositional formula is the sum over the one-hot encodings of its models.
This is equal to the encoding of the set of models, which will be introduced in Chapter~\ref{cha:tensorEncodings} (see Definition~\ref{def:subsetEncoding}).

We depict this decomposition in the diagrammatic notation by
\begin{center}
	\begin{tikzpicture}[scale=0.35, thick]

    \draw (-1,-1) rectangle (5,-3);
    \node[anchor=center] (text) at (2,-2) {\corelabelsize ${\exformula}$};
    \draw[] (0,-3)--(0,-5) node[midway,left] {\colorlabelsize $\catvariableof{0}$};
    \draw[] (1.5,-3)--(1.5,-5) node[midway,left] {\colorlabelsize $\catvariableof{1}$};
    \node[anchor=center] (text) at (3,-4) {$\cdots$};
    \draw[] (4,-3)--(4,-5) node[midway,right] {\colorlabelsize $\catvariableof{\atomorder\shortminus1}$};


    \node[anchor=center] (text) at (7,-2) {${=}$};

    \node[anchor=center] (text) at (12,-2.5) {${\sum\limits_{\atomindices\in\atomstates}}$};
    \node[anchor=center] (text) at (12,-4) {\colorlabelsize $\exformula(\atomindices)=1$};

    \begin{scope}
        [shift={(19.5,1)}]

        \draw (-3,-2) rectangle (-1,-4);
        \node[anchor=center] (text) at (-2,-3) {\corelabelsize $\onehotmapof{\atomlegindexof{0}}$};
        \draw[->-] (-2,-4)--(-2,-6) node[midway,right] {\colorlabelsize $\catvariableof{0}$};

        \node[anchor=center] (text) at (1,-3) {\corelabelsize $\cdots$};

        \draw (3,-2) rectangle (5,-4);
        \node[anchor=center] (text) at (4,-3) {\corelabelsize $\onehotmapof{\atomlegindexof{\atomorder\shortminus1}}$};
        \draw[->-] (4,-4)--(4,-6) node[midway,right] {\colorlabelsize $\catvariableof{\atomorder\shortminus1}$};

    \end{scope}

\end{tikzpicture}
\end{center}




% Maps to multiple formulas -> Later?
%We can extend the map to factored systems of multiple formulas, by using Definition~\ref{def:formulas} as coordinate maps.
%This is exactly what we will study by Bayesian Propositional Networks.
%We will make use of redundancies in the maps to get an efficient representation based on decompositions.





%% Semantic approach
We here chose a semantic approach to propositional logic in contrary to the standard syntactical approach.
Instead of defining formulas by connectives acting on atomic formulas, we define them here as binary valued functions of the states of a factored system.
They are interpreted by marking possible states as models, given the knowledge of $\exformula$.
The syntactical side will then be introduced later by studying decompositions of formulas.


%\begin{figure}[h]
%\begin{center}
%	\begin{tikzpicture}[scale=0.35, thick]

    \draw (-1,-1) rectangle (5,-3);
    \node[anchor=center] (text) at (2,-2) {\corelabelsize ${\exformula}$};
    \draw[] (0,-3)--(0,-5) node[midway,left] {\colorlabelsize $\catvariableof{0}$};
    \draw[] (1.5,-3)--(1.5,-5) node[midway,left] {\colorlabelsize $\catvariableof{1}$};
    \node[anchor=center] (text) at (3,-4) {$\cdots$};
    \draw[] (4,-3)--(4,-5) node[midway,right] {\colorlabelsize $\catvariableof{\atomorder\shortminus1}$};


    \node[anchor=center] (text) at (7,-2) {${=}$};

    \node[anchor=center] (text) at (12,-2.5) {${\sum\limits_{\atomindices\in\atomstates}}$};
    \node[anchor=center] (text) at (12,-4) {\colorlabelsize $\exformula(\atomindices)=1$};

    \begin{scope}
        [shift={(19.5,1)}]

        \draw (-3,-2) rectangle (-1,-4);
        \node[anchor=center] (text) at (-2,-3) {\corelabelsize $\onehotmapof{\atomlegindexof{0}}$};
        \draw[->-] (-2,-4)--(-2,-6) node[midway,right] {\colorlabelsize $\catvariableof{0}$};

        \node[anchor=center] (text) at (1,-3) {\corelabelsize $\cdots$};

        \draw (3,-2) rectangle (5,-4);
        \node[anchor=center] (text) at (4,-3) {\corelabelsize $\onehotmapof{\atomlegindexof{\atomorder\shortminus1}}$};
        \draw[->-] (4,-4)--(4,-6) node[midway,right] {\colorlabelsize $\catvariableof{\atomorder\shortminus1}$};

    \end{scope}

\end{tikzpicture}
%\end{center}
%\caption{Direct interpretation of a propositional formula $\exformula$ as a tensor.
%	The tensor is the sum of the one hot encodings of its models.
%	While the one hot encodings are directed, their sum is not.}
%\label{fig:formulaDirect} 
%\end{figure}

%% Intro of connectives
%Logical connectives are basic building blocks of such formulas and can be understood by simple computations represented in truth tables.
% Here truth tables?
%We call each combination of atomic formulas with connectives a formula.

\subsubsection{Relational encoding of formulas}


%% Direct and Relational interpretation of $\exformula$
There are two ways to represent formulas by tensors.
One way is to understand $[2]$ as subset of $\rr$ and interpreting the formula directly as a tensor (as in Definition~\ref{def:formulas}).
Another way is to understand $[2]$ as the possible values of a categorical variable.
% Maps between factored systems
Following this second perspective, formulas are maps between factored systems, where the image system is the factored systems of atoms and the target system the atomic system defined by a variable $\catvariableof{\formula}$ representing the formula satisfaction.
%Following this perspective, formulas are maps between the factored systems of atoms and the atomic system of the formula.
We can then build the relational encoding (Definition~\ref{def:functionRepresentation}) of that map to represent the formula (see Figure~\ref{fig:formulaRencoding}).

\begin{definition}[Relation Encoding of Formulas] % Own definition, since a reinterpretation of the formula
	Given a factored system with $\atomorder$ atoms $\catvariables$ and a propositional formula $\formula$, we define the relational encoding of $\formula$ (see Definition~\ref{def:functionRepresentation}) 
		\[ \rencodingofat{\formula}{\catvariables,\catvariableof{\formula}} \in  \left(\bigotimes_{\atomenumeratorin} \rr^2\right) \otimes \rr^2 \]
	by 
	\begin{align} 
		\rencodingofat{\formula}{\catvariables,\catvariableof{\formula}} 
		= & \sum_{\atomindices\in\atomstates}  \onehotmapofat{\atomindices}{\catvariables} \otimes \onehotmapofat{\exformula(\atomindices)}{\catvariableof{\formula}} \, . 
	\end{align}
\end{definition}

%% More general relational encodings
We can build relational encodings more generally of any tensors, where we identify the image of the tensor with the states of a categorical variable.
Exactly for propositional formulas, this construction will lead to Boolean image variables.


\begin{lemma}\label{lem:formulaEncodingDecomposition}
	For any formula $\formula$ we have
		\[ \rencodingofat{\formula}{\shortcatvariables,\catvariableof{\formula}} 
		= \formulaat{\shortcatvariables} \otimes \onehotmapofat{1}{\catvariableof{\formula}} 
		+ \lnot\formulaat{\shortcatvariables} \otimes  \onehotmapofat{0}{\catvariableof{\exformula}} \, . 
		 \]
	In particular
		\[ \formulaat{\shortcatvariables} = \contractionof{\{
		\rencodingofat{\formula}{\shortcatvariables,\catvariableof{\formula}} , \onehotmapofat{1}{\catvariableof{\formula}}
		\}}{\shortcatvariables} \, . \]
\end{lemma}
\begin{proof}
%% Decomposition
	We can decompose relational encodings of formulas into the sum (see Figure~\ref{fig:formulaRencoding}) % ! Not a tensor network decomposition !
	\begin{align} 
		\rencodingof{\exformula} = & \fbasis \otimes \left( \sum_{\atomindices\, : \, \exformula(\atomindices) = 0}  \onehotmapof{\atomindices} \right) \\
		 + & \tbasis \otimes \left( \sum_{\atomindices\, : \,  \exformula(\atomindices) = 1}  \onehotmapof{\atomindices} \right)
	\end{align}
	where the second term sums up the models of $\exformula$ and the first one the models of $\lnot\exformula$.
\end{proof}


% Comparison with direct interpretation
Compared with the direct interpretation of a formula as a tensor and the decomposition into models in Equation~\ref{eq:formulaModelDecomposition}, we notice that the relational encoding also represents encoding of worlds where the formula is not satisfied.
This representation is required to represent arbitrary propositional formulas by contracted tensor networks of its components, as will be investigated in the following sections.


%% Coordinatewise 
The relational decomposition $\rencodingof{\exformula}$ has coordinates 
\begin{align}
		\contractionof{\{\rencodingof{\exformula},\onehotmapof{\atomindices}\}}{\catvariableof{\exformula}} 
		= \begin{cases}
		\tbasis & \text{if the world where $\atomicformulaof{\atomenumerator}=\atomlegindexof{\atomenumerator}$ is a model of $\exformula$}  \\
		\fbasis & \text{else}\, .
		\end{cases}
\end{align} 
The contractions of the relational encoding therefore calculate whether an assignment of atoms is a model of the formula, using basis calculus (see Theorem~\ref{the:basisCalculus}).

\begin{figure}[h]
\begin{center}
	\begin{tikzpicture}[scale=0.35, thick] % , baseline = -3.5pt

\draw[->] (2,-1)--(2,1) node[midway,right] {\tiny $\formulavar$};
\draw (-1,-1) rectangle (5,-3);
\node[anchor=center] (text) at (2,-2) {\small $\rencodingof{\exformula}$};
\draw[<-] (0,-3)--(0,-5) node[midway,left] {\tiny $\randomxof{0}$}; 
\draw[<-] (1.5,-3)--(1.5,-5) node[midway,left] {\tiny $\randomxof{1}$}; 
\node[anchor=center] (text) at (3,-4) {$\cdots$};
\draw[<-] (4,-3)--(4,-5) node[midway,right] {\tiny $\randomxof{\atomorder\shortminus1}$}; 


\node[anchor=center] (text) at (7,-2) {${=}$};

\node[anchor=center] (text) at (12,-2.5) {${\sum\limits_{\atomindices\in\atomstates}}$};

\begin{scope}[shift={(19,-0.5)}]

\draw (-2,1) rectangle (4,-1);
\node[anchor=center] (text) at (1,0) {\small $\onehotmapof{\exformula(\atomindices)}$};
\draw[->] (1,1)--(1,3) node[midway,right] {\tiny $\formulavar$}; 

\draw (-3,-2) rectangle (-1,-4);
\node[anchor=center] (text) at (-2,-3) {\small $\onehotmapof{\atomlegindexof{0}}$};
\draw[->] (-2,-4)--(-2,-6) node[midway,right] {\tiny $\catvariableof{0}$}; 

\node[anchor=center] (text) at (1,-3) {\small $\cdots$};

\draw (3,-2) rectangle (5,-4);
\node[anchor=center] (text) at (4,-3) {\small $\onehotmapof{\atomlegindexof{\atomorder\shortminus1}}$};
\draw[->] (4,-4)--(4,-6) node[midway,right] {\tiny $\catvariableof{\atomorder\shortminus1}$}; 

\end{scope}

\end{tikzpicture}
\end{center}
\caption{Relational encoding of a propositional formula. 
The encoding is a sum of the one hot encodings of all states of the factored system in a tensor product with basis vectors, which encode whether the state is a model of the formula.
The tensor is directed, since any contraction with an encoded state results in the basis vector evaluating the formula, which we called basis calculus.
}
\label{fig:formulaRencoding} 
\end{figure}










\subsection{Syntax of Propositional Formulas}

Relational encodings of propositional formulas are especially useful when representing function compositions by the representation of their components (see Theorem~\ref{the:compositionByContraction}). 
In propositional logics, the syntax of defining propositional formulas is oriented on compositions of formulas by connectives. % Quantifications will be studied in the FOL Chapter.
We in this section investigate the decomposition schemes of relational encodings into tensor networks of component encodings for binary tensors following propositional logic syntax.

\subsubsection{Atomic Formulas}

We call atomic formulas the most granular formulas, which are not splitted into compositions of other formulas.
Our syntactic decomposition of propositional formulas will then investigate, how any propositional formula can be represented by these.

\begin{definition}
	The tensors $\formulaofat{\atomenumerator}{\catvariables}$ defined for $\catindices$ as
		\[ \formulaofat{\atomenumerator}{\indexedcatvariables} = \atomlegindexof{\atomenumerator} \]
	are called atomic formulas.
\end{definition}

\begin{theorem}\label{the:AtomicFTensor}
	The relational encoding of any atomic formula $\formulaofat{\atomenumerator}{\catvariables}$ has a tensor decomposition by
		\[ \rencodingofat{\atomicformulaof{\atomenumerator}}{\catvariables,\catvariableof{\formulaof{\atomenumerator}}}
		= \contractionof{\{\identityat{\catvariableof{\atomenumerator},\catvariableof{\formulaof{\atomenumerator}}}\}}{\catvariables,\catvariableof{\formulaof{\atomenumerator}}} \, . \]
	The decomposition is depicted in a network diagram as
	\begin{center}
		\begin{tikzpicture}[scale=0.35,thick] % , baseline = -3.5pt

\drawatomcore{3.5}{-8}{$\bencodingof{\formulaof{\atomenumerator}}$}
\drawatomindices{3.5}{-12}	
\draw[->-] (5.5,-9)--(5.5,-7) node[midway,right] {\colorlabelsize $\headvariableof{\atomenumerator}$};

\node[anchor=center] (text) at (10,-10) {${=}$};

\draw (12,-9) rectangle (15,-11); 
\node[anchor=center] (text) at (13.5,-10) {\corelabelsize $\ones$};
\draw[-<-] (12.5,-11)--(12.5,-13) node[midway,left] {\colorlabelsize $\catvariableof{0}$};
\node[anchor=center] (text) at  (13.5,-12) {$\cdots$};
\draw[-<-] (14.5,-11)--(14.5,-13) node[midway,right] {\colorlabelsize $\catvariableof{\atomenumerator\shortminus1}$};

\node[anchor=center] (text) at (16.25,-10) {\corelabelsize $\otimes$};

\draw[->-] (18.5,-9)--(18.5,-7) node[midway,right] {\colorlabelsize $\headvariableof{\atomenumerator}$};
\draw (17.5,-9) rectangle (19.5,-11);
\node[anchor=center] (text) at (18.5,-10) {\corelabelsize $\delta$};
\draw[-<-]  (18.5,-11)--(18.5,-13) node[midway,right] {\colorlabelsize $\catvariableof{\atomenumerator}$};

\node[anchor=center] (text) at (20.75,-10) {\corelabelsize $\otimes$};

\begin{scope}[shift={(10,0)}]

\draw (12,-9) rectangle (15,-11); 
\node[anchor=center] (text) at (13.5,-10) {\corelabelsize $\ones$};
\draw[-<-]  (12.5,-11)--(12.5,-13) node[midway,left] {\colorlabelsize $\catvariableof{\atomenumerator+1}$};
\node[anchor=center] (text) at  (13.5,-12) {$\cdots$};
\draw[-<-]  (14.5,-11)--(14.5,-13) node[midway,right] {\colorlabelsize $\catvariableof{\atomorder\shortminus1}$};

\node[anchor=center] (text) at  (16.5,-13) {$.$};

\end{scope}

\end{tikzpicture}
	\end{center}
\end{theorem}
\begin{proof}
	We have by definition
	\begin{align*}
		\rencodingofat{\atomicformulaof{\atomenumerator}}{\catvariables,\catvariableof{\formulaof{\atomenumerator}}}
		=& \sum_{\catindices\in\atomstates} \onehotmapofat{\catindices}{\catvariables} \otimes \onehotmapofat{\formulaofat{\atomenumerator}{\indexedcatvariables}}{\catvariableof{\formulaof{\atomenumerator}}} \\
		=& \left( \onehotmapofat{0,0}{\catvariableof{\atomenumerator},\catvariableof{\formulaof{\atomenumerator}}} +
		\onehotmapofat{1,1}{\catvariableof{\atomenumerator},\catvariableof{\formulaof{\atomenumerator}}} \right) \otimes \onesat{\catvariableof{\secatomenumerator}\, : \, \secatomenumerator \neq \atomenumerator} \\
		=& \contractionof{\{\identityat{\catvariableof{\atomenumerator},\catvariableof{\formulaof{\atomenumerator}}}\}}{\catvariables,\catvariableof{\formulaof{\atomenumerator}}} \, .
%		\ftensorof{\exformula}_{1,\atomindices} = \begin{cases}
%		1 & \text{if $\atomlegindexof{\atomenumerator}=1$}  \\
%		0 & \text{else} \, .
	%\end{cases}
	\end{align*} 
%	Since all atom indices $\secatomenumerator\neq\atomenumerator$ are irrelevant, the formula tensor decomposed into factors with the constant vector $\onesof{\secatomenumerator}$.
\end{proof}


%\begin{figure}[h]
%\caption{Representation of an atomic formula tensor $\atomicformulaof{\atomenumerator}$.}
%\label{fig:FormulaChain} 
%\end{figure}

\begin{remark}[Representation of atomic formula tensors for connective action]
 	Need to represent this as $\braket{\delta \otimes \ones, \truevectorat{\atomenumerator}}$, where the bracket indicates contraction along the $\atomenumerator$th axis.
	The core $\truevectorat{\atomenumerator}$ can be replaced by further operations based on logical connectives.
	The axis $\atomenumerator$ is changed from an axis associated with an atom truth to an axis associated with an formula truth.
\end{remark}


\subsubsection{Syntactical combination of formulas}

Formula tensors are elements of tensor spaces with $\atomorder+1$ axis. 
The number of coordinates thus grows exponentially with the number of atoms, which is
	\[ \dim\left[ \rr^2 \otimes \bigotimes_{\atomenumeratorin}\rr^{2} \right] = 2^{\atomorder +1} \, . \]
When the number of atoms in a theory is large, the naive representation of formula tensors will be thus intractable.
In contrast, most logical formulas appearing in practical knowledge bases are sparse in the sense that they have short representations in a logical syntax.
Motivated by this consideration we now discuss propositional syntax and investigate the sparse decomposition of formula tensors along their formula structure to avoid the curse of dimensionality.

%% Propositional Syntax
In logical syntax formulas are described by atomic formulas recursively connected via connectives. 
We show, that representations of logical connectives $\circ \in \{\lnot, \land, \lor, \oplus, \Rightarrow, \Leftrightarrow\}$ can be represented by feasible tensor cores $\concoreof{\circ}$ contracted along a tensor network.


%More general: Theorem~\ref{the:compositionByContraction} shows that any composition of functions can be expressed by contractions of relational encodings.


\begin{example}\label{exa:connectives}
%	Given formulas $\exformula$ and $\secexformula$ we define $\lnot \exformula$ and $\exformula \exconnective \secexformula$ for $\exconnective \in \{\lnot, \land, \lor, \oplus, \Rightarrow, \Leftrightarrow\}$ by their formula tensors
%	\begin{align}
%		\rencodingof{\lnot\exformula} = \contractionof{\{\concoreof{\lnot},\rencodingof{\exformula}\}}{\atomicformulas\cup\{\catvariableof{\lnot\exformula}\}}
%	\end{align}
%	and 
%	\begin{align}
%		\rencodingof{\exformula\exconnective\secexformula} 
%		= \contractionof{\{\concoreof{\exconnective},\rencodingof{\exformula},\rencodingof{\secexformula}\}}{\atomicformulas\cup\{\catvariableof{\exformula\exconnective\secexformula}\}}
%	\end{align}
	We use the following connectives:
	\begin{itemize}
	\item negation $\lnot: [2]\rightarrow [2]$ by the vector
	\begin{align}
		{\lnot}[\catvariableof{\exformula}] = \begin{bmatrix}
		0  \\
		1  
		\end{bmatrix} 
	\end{align}
%	\begin{align}
%		\rencodingof{\lnot} = \begin{bmatrix}
%		0 & 1 \\
%		1 & 0 
%		\end{bmatrix} 
%	\end{align}
%	and by $\rencodingof{\exconnective}$ the order $3$ tensors
	\item conjunctions $\land:  [2]\times[2] \rightarrow[2]$
		\begin{align}
			\land[\catvariableof{\exformula},\catvariableof{\secexformula}]
			 = \begin{bmatrix}
			0 & 0 \\
			0 & 1 
			\end{bmatrix}
		\end{align}
	\item disjunctions $\lor : [2]\times[2] \rightarrow[2]$
		\begin{align}
			\lor[\catvariableof{\exformula},\catvariableof{\secexformula}]
			 = \begin{bmatrix}
			0 & 1 \\
			1 & 1 
			\end{bmatrix}
		\end{align}
	\item exact disjunction $\oplus:  [2]\times[2] \rightarrow[2]$	
		\begin{align}
			\oplus[\catvariableof{\exformula},\catvariableof{\secexformula}]
			 = \begin{bmatrix}
			0 & 1 \\
			1 & 0 
			\end{bmatrix}
		\end{align}
	\item implications $\Rightarrow:  [2]\times[2] \rightarrow[2]$ 
		\begin{align}
			\Rightarrow[\catvariableof{\exformula},\catvariableof{\secexformula}]
			 = \begin{bmatrix}
			1 & 1 \\
			0 & 1 
			\end{bmatrix}
		\end{align}
	\item biimplication $\Leftrightarrow:  [2]\times[2] \rightarrow[2]$ 
		\begin{align}
			\Leftrightarrow[\catvariableof{\exformula},\catvariableof{\secexformula}]
			 = \begin{bmatrix}
			1 & 0 \\
			0 & 1 
			\end{bmatrix}
		\end{align}
	\end{itemize}
%	\item conjunctions $\land$
%		\begin{align}
%			\rencodingof{\land}_{1,:,:} 
%			 = \begin{bmatrix}
%			0 & 0 \\
%			0 & 1 
%			\end{bmatrix}
%			\quad,\quad			
%			\rencodingof{\land}_{0,:,:} 
%			 = \begin{bmatrix}
%			1 & 1 \\
%			1 & 0 
%			\end{bmatrix} \, .
%		\end{align}
%	\item disjunctions $\lor$
%		\begin{align}
%			\rencodingof{\lor}_{1,:,:} 
%			 = \begin{bmatrix}
%			0 & 1 \\
%			1 & 1 
%			\end{bmatrix}
%			\quad, \quad \rencodingof{\lor}_{0,:,:} 
%			 = \begin{bmatrix}
%			1 & 0 \\
%			0 & 0 
%			\end{bmatrix}
%		\end{align}
%	\item exact disjunction $\oplus$	
%		\begin{align}
%			\rencodingof{\oplus}_{1,:,:} 
%			 = \begin{bmatrix}
%			0 & 1 \\
%			1 & 0 
%			\end{bmatrix}
%			\quad, \quad \rencodingof{\oplus}_{0,:,:} 
%			 = \begin{bmatrix}
%			1 & 0 \\
%			0 & 1 
%			\end{bmatrix}
%		\end{align}
%	\item implications $\Rightarrow$ 
%		\begin{align}
%			\rencodingof{\Rightarrow}_{1,:,:} 
%			 = \begin{bmatrix}
%			1 & 1 \\
%			0 & 1 
%			\end{bmatrix}
%			\quad, \quad \rencodingof{\Rightarrow}_{0,:,:} 
%			 = \begin{bmatrix}
%			0 & 0 \\
%			1 & 0 
%			\end{bmatrix}
%		\end{align}
%	\item biconditionals $\Leftrightarrow$ 
%		\begin{align}
%			\rencodingof{\Leftrightarrow}_{1,:,:} 
%			 = \begin{bmatrix}
%			1 & 0 \\
%			0 & 1 
%			\end{bmatrix}
%			\quad, \quad \rencodingof{\Leftrightarrow}_{0,:,:} 
%			 = \begin{bmatrix}
%			0 & 1 \\
%			1 & 0 
%			\end{bmatrix}
%		\end{align}
%	\end{itemize}
\end{example}


\begin{lemma}\label{lem:compositionByContraction}
	Let there be formulas $\exformula$ and $\secexformula$ depending on categorical variables $\shortcatvariables=\catvariables$ and a map 
		\[ \exconnective: [2]\times[2] \rightarrow[2] \, . \]
	Then we have
	\begin{align*}
		\rencodingofat{\exformula\exconnective\secexformula}{\shortcatvariables,\catvariableof{\exformula\exconnective\secexformula}}
		= \contractionof{\{
		\rencodingofat{\exconnective}{\catvariableof{\exformula},\catvariableof{\secexformula},\catvariableof{\exformula\exconnective\secexformula}},
		\rencodingofat{\exformula}{\shortcatvariables,\catvariableof{\exformula}},
		\rencodingofat{\secexformula}{\shortcatvariables,\catvariableof{\secexformula}} 
		\}}{
		\shortcatvariables,\catvariableof{\exformula\exconnective\secexformula}
		}
	\end{align*}
	and for any map $\exconnective: [2] \rightarrow[2]$
	\begin{align*}
		\rencodingofat{\exconnective\exformula}{\shortcatvariables,\catvariableof{\exconnective\exformula}}
		= \contractionof{\{
		\rencodingofat{\exconnective}{\catvariableof{\exformula},\catvariableof{\exconnective\exformula}},
		\rencodingofat{\exformula}{\shortcatvariables,\catvariableof{\exformula}}
		\}}{
		\shortcatvariables,\catvariableof{\exconnective\exformula}
		} \, . 
	\end{align*}
\end{lemma}
\begin{proof}
	This follows from Theorem~\ref{the:compositionByContraction} to be shown in Chapter~\ref{cha:tensorEncodings}.
\end{proof}


\begin{theorem}[Composition of Formulas]\label{the:compositionByContraction}
	Let there be a set of binary variables $\catvariableof{\nodes}$ including atoms $\catvariables$ and image variables to some formulas.
	For any formula $\formulaat{\catvariables}$, which has a syntactical composition into connectives $\{\exconnective_{l}[\catvariableof{\{\nodes_l \}}] : l \in [p]\}$ taking their inputs by variables $\catvariableof{\{\nodes_l \}}\subset \catvariableof{\nodes}$ and output by a variable $\catvariableof{\exconnective_l}$
	we have that
	\begin{align*}
		\rencodingofat{\formula}{\catvariables,\catvariableof{\formula}} =
		\contractionof{\left\{
		\rencodingofat{\exconnective_l}{\catvariableof{\{\nodes_l \}}, \catvariableof{\exconnective_l} : l \in [p] }
		\right\} }
		{\catvariables,\catvariableof{\formula}} \, . 
	\end{align*}
\end{theorem}
\begin{proof}
	When a variable in $\catvariableof{\nodes}$ appears multiple times as input to connectives, we replace it by a set of copies (which wont change the contraction, since all tensors are binary and Theorem~\ref{the:invarianceAddingSubcontractions} can be applied).
	The claim follows then from iterative application of Lemma~\ref{lem:compositionByContraction}.
\end{proof}

\begin{figure}[h]
\begin{center}
	\begin{tikzpicture}[scale=0.35, thick] % , baseline = -3.5pt

\node[anchor=center] (text) at (2,-4) {$a)$};

\draw[->] (5.5,-5)--(5.5,-3) node[midway,right] {\tiny $\headvariableof{\neg\exformula}$};

\node[anchor=center] (text) at (5.5,-6) {$\rencodingof{\lnot}$};
\draw (4.5,-7) rectangle (6.5,-5);

\draw[->] (5.5,-9)--(5.5,-7) node[midway,right] {\tiny $\formulavar$};


\drawatomcore{3.5}{-8}{$\rencodingof{\exformula}$}
\drawatomindices{3.5}{-12}	




\begin{scope}[shift={(15,0)}]

\node[anchor=center] (text) at (2,-4) {$b)$};

\draw[->] (9.5,-5)--(9.5,-3) node[midway,right] {\tiny $\headvariableof{\exformula\circ\secexformula}$};

\node[anchor=center] (text) at (9.5,-6) {$\rencodingof{\circ}$};
\draw (4.5,-7) rectangle (14.5,-5);

\draw[->] (5.5,-9)--(5.5,-7) node[midway,right] {\tiny $\formulavar$};

\drawatomcore{3.5}{-8}{$\rencodingof{\exformula}$}
\drawatomindices{3.5}{-12}	

\begin{scope}[shift={(8,0)}]

	\draw[->] (5.5,-9)--(5.5,-7) node[midway,right] {\tiny $\secexformulavar$};

	\drawatomcore{3.5}{-8}{$\rencodingof{\secexformula}$}
	\drawatomindices{3.5}{-12}	

\end{scope}

\draw[fill] (7.5,-15) circle (0.25cm);
\draw[] (7.5,-15) to[bend left=25] (3.5,-13);
\draw[] (7.5,-15) to[bend right=25] (11.5,-13);

\draw[fill] (9,-15.25) circle (0.25cm);
\draw[] (9,-15.25) to[bend left=25] (5,-13);
\draw[] (9,-15.25) to[bend right=25] (13,-13);

\draw[fill] (11.5,-15) circle (0.25cm);
\draw[] (11.5,-15) to[bend left=25] (7.5,-13);
\draw[] (11.5,-15) to[bend right=25] (15.5,-13);



\draw[] (7.5,-15)--(7.5,-17) node[midway,left] {\tiny $\catvariableof{0}$}; 
\draw[] (9,-15.25)--(9,-17) node[midway,left] {\tiny $\catvariableof{1}$}; 
\node[anchor=center] (text) at (10.5,-16.5) {$\cdots$};
\draw[] (11.5,-15)--(11.5,-17) node[midway,right] {\tiny $\catvariableof{\atomorder-1}$}; 

\end{scope}

\end{tikzpicture} 
\end{center}
\caption{a) Relational encoding of a negated formula $\exformula$ as a tensor network of the encoded formula and the encoded connective $\lnot$.
b) Relational encoding of a composition of formulas $\exformula, \secexformula$ by a connective $\circ\in\{\land,\lor,\oplus,\Rightarrow,\Leftrightarrow\}$. 
The encoding is a contraction of encodings to  $\exformula, \secexformula$ and $\circ$.}
\label{fig:NegatedFormulaTensor} 
\end{figure}

%\begin{remark}[Universality of connectives]
%	The negation $\lnot$ is the only nontrivial unary connective in propositional logics.
%	In combination with $\land$ it already suffices to represent any other $\atomorder$-connective.
%	This fact can be derived from CNF decompositions and De Morgans rules.
%\end{remark}

% To Logical Inference!
%\begin{remark}[Conditional Probability Interpretation]
%	For all these symbol tensors we have the zeroth component defined by
%		\[ \concoreof{\circ}_{0,:,:}  = \ones - \concoreof{\circ}_{1,:,:} \, .\]
%	This again results from the conditional probability distribution interpretation of each $\concoreof{\circ}$.
%\end{remark}

\begin{remark}[$\atomorder$-ary connectives such as $\land$ and $\lor$]\label{rem:naryConnectives}
	Since the decomposition of relational encoding can be applied to generic function compositions (see Theorem~\ref{the:compositionByContraction}), we can also allow for $\atomorder$-ary connectives
		\[ \exconnective : \bigtimes_{\atomenumeratorin} [2] \rightarrow [2] \,  \]
	in Theorem~\ref{the:compositionByContraction}
	The connectives $\land$ and $\lor$ satisfy associativity and have thus straightforward generalizations to the $\atomorder$-ary case.
	This is because associativity can be exploited to represent the relational encoding by any tree-structured composition of binary $\land$ and $\lor$ connectives.
\end{remark}

%% Maps perspective
In general, any $\atomorder$ary logical connective is a map

%In Example~\ref{exa:connectives} we defined unary ($\atomorder=1$) and binary ($\atomorder=2$) connectives.
%Propositional Syntax describes generic formulas $\exformula$ based on the composition of these maps starting with atomic formulas.
%We can thus apply Theorem~\ref{the:compositionByContraction} recursively to decompose the formula tensors $\ftensorof{\exformula}$ into cores $\concoreof{\exconnective}$.



%% Construction from atomic formula tensors
Propositional syntax consists in the application of connectives on atomic formulas, and recursively on the results of such constructions.
When passed towards connective cores, atomic formula tensors act trivial on the legs and just identify the corresponding atomic formula index $\atomlegindexof{\atomicformulaof{\atomenumerator}}$ with $\atomlegindexof{\atomenumerator}$.
This is due to the fact, that the Hadamard product with the trivial tensor $\ones$ leaves any tensor invariant, and the contraction with the elementary matrix $\delta$ identifies indices with each other.
We can thus savely ignore the atomic formula tensors appearing in the decomposition of formula tensors to non-atomic formulas.
An example of such a decomposition is depicted in Figure~\ref{fig:FTDecomposition}.





\begin{figure}[h]
\begin{center}
	\begin{tikzpicture}[scale=0.35, yscale=-1, thick] % , baseline = -3.5pt

    \begin{scope}
        [shift={(-15,0)}]

        \node[anchor=center] (text) at (-3,-6) {${a)}$};

        \node [circle, draw, thick, fill=\nodegrayscale, minimum size = \nodeminsize] (T1) at (0,2) {\colorlabelsize $\catvariableof{a}$};
        \node [circle, draw, thick, fill=\nodegrayscale, minimum size = \nodeminsize] (T2) at (3,2) {\colorlabelsize $\catvariableof{b}$};
        \node [circle, draw, thick, fill=\nodegrayscale, minimum size = \nodeminsize] (T3) at (6,2) {\colorlabelsize $\catvariableof{c}$};

        \node [circle, draw, thick, fill=\nodegrayscale, minimum size = \nodeminsize] (and) at (1.5,-3) {\colorlabelsize $\headvariableof{a\land b}$};
        \coordinate (lowandcenter) at (1.5, -0.5);
        \node [circle, draw, thick, fill=\nodegrayscale, minimum size = \nodeminsize] (not) at (6,-3) {\colorlabelsize $\headvariableof{\lnot c}$};
        \coordinate (lnotcenter) at (6, -0.5);

        \draw [->-] (T1) -- (lowandcenter);
        \draw [->-] (T2) -- (lowandcenter);
        \draw [->-] (lowandcenter) -- (and);

        \draw [->-] (T3) -- (lnotcenter);
        \draw [->-] (lnotcenter) -- (not);

        \node [circle, draw, thick, fill=\nodegrayscale, minimum size = \nodeminsize] (head) at (3.25,-8) {\colorlabelsize $\headvariableof{a\land b \land \lnot c}$};
        \coordinate (highandcenter) at (3.28, -5.5);

        \draw [->-] (and) -- (highandcenter);
        \draw [->-] (not) -- (highandcenter);
        \draw [->-] (highandcenter) -- (head);
    \end{scope}

    \node[anchor=center] (text) at (-3,-6) {${b)}$};

    \draw[->-] (0,1)--(0,-1) node[midway,left] {\colorlabelsize $\catvariableof{a}$};
    \draw[->-] (1.5,1)--(1.5,-1) node[midway,right] {\colorlabelsize $\catvariableof{b}$};
    \draw[->-] (3,1)--(3,-1) node[midway,right] {\colorlabelsize $\catvariableof{c}$};
    \draw (-1,-1) rectangle (4, -3);
    \node[anchor=center] (text) at (1.5,-2) {\corelabelsize $\bencodingof{a \land b \land \lnot c}$};
    \draw[->-] (1.5,-3)--(1.5,-5) node[midway,right] {\colorlabelsize $\headvariableof{a \land b \land \lnot c}$};

    \node[anchor=center] (text) at (5,-2) {${=}$};


    \begin{scope}
        [shift={(7,0)}]

        \draw[->-] (0,1)--(0,-1) node[midway,left] {\colorlabelsize $\catvariableof{a}$};
        \draw[->-] (3,1)--(3,-1) node[midway,right] {\colorlabelsize $\catvariableof{b}$};
        \draw[->-] (6,1)--(6,-1) node[midway,right] {\colorlabelsize $\catvariableof{c}$};

        \draw (-1,-1) rectangle (4, -3);
        \node[anchor=center] (text) at (1.5,-2) {\corelabelsize $\bencodingof{\land}$};

        \draw[->-] (1.5,-3) --(1.5,-5) node[midway,right]{\colorlabelsize $\headvariableof{a \land b}$};

        \draw (5,-1) rectangle (7, -3);
        \node[anchor=center] (text) at (6,-2) {\corelabelsize $\bencodingof{\lnot}$};

        \draw[->-] (6,-3) --(6,-5) node[midway,right]{\colorlabelsize $\headvariableof{\lnot c}$};

        \draw (0.5,-5) rectangle (6.5,-7);
        \node[anchor=center] (text) at (3.5,-6) {\corelabelsize $\bencodingof{\land}$};

        \draw[->-] (4,-7) -- (4,-9) node[midway,right] {\colorlabelsize $\headvariableof{a \land b \land \lnot c}$};

%\draw (3,-9) rectangle (5,-11);
%\node[anchor=center] (text) at (4,-10) {$\truevectorat{}$};

    \end{scope}

\end{tikzpicture}
\end{center}
\caption{Decomposition of the formula tensor to $\exformula = a \land b \land \lnot c$ into unary (matrix) and binary (third order tensor) cores.
	a) Visualization of $\exformula$ as a graph. %(\red{Exploiting the duality between tensor networks on hypergraphs and graphical models \cite{robeva_duality_2019} })
	b) Dual Tensor Network decomposition of $\exformula$.
	We can make use of the invariance of a Hadamard product with a constant tensor $\ones$ and thus not draw axis to atoms not affected by a formula.}
\label{fig:FTDecomposition}
\end{figure}





\begin{remark}[$\htformat$ Interpretation of Formula Tensor Networks]\label{rem:HTDecomFT}
	The sketched decomposition of the formula tensor into a network is a hierarchical tree decomposition of the formula tensor, which we will describe in more detail in Section~\ref{sec:HT}.
	At each decomposition of a formula into subformulas, two subspaces spanned by the respective atomic spaces are selected. 
\end{remark}


\subsubsection{Syntactical decomposition of formulas}\label{sec:termClauseDecomposition}

% Decomposition in case of missing 
We have seen how the decomposition of complex formulas into connectives acting on the component formulas can be exploited to find effective representations of the semantics by tensor networks.
Here the question arises here, how to perform such decompositions in case of a missing syntactical representation of a formula.
By Definition~\ref{def:formulas} any binary tensor is a formula.
We show in the following, how we can find a syntactic specification of a formula given its tensor.

%
%Let us now show that any formula tensor can be decomposed into a network of these connective symbols and atomic formula tensors.


\begin{definition}[Terms and Clauses]\label{def:clauses}
	Given two disjoint subsets $\nodes_0$ and $\nodes_1$ of the $[\atomorder]$, the corresponding term is the formula defined on the indices $\catindices\in\atomstates$ by
		\[ \termof{\nodes_0}{\nodes_1}
		=\left( \bigwedge_{\atomenumerator\in\nodes_0} \lnot\formulaof{\atomenumerator} \right)  \land \left( \bigwedge_{\atomenumerator\in\nodes_1} \formulaof{\atomenumerator} \right)  \]
	and the corresponding clause is the formula defined on the indices $\catindices\in\atomstates$ by
		\[ \clauseof{\nodes_0}{\nodes_1}
		=\left( \bigvee_{\atomenumerator\in\nodes_0} \formulaof{\atomenumerator} \right)  \lor \left( \bigvee_{\atomenumerator\in\nodes_1} \lnot\formulaof{\atomenumerator} \right)  \, , \]
	where by $\land_{\atomenumerator\in\nodes}$ and $\lor_{\atomenumerator\in\nodes}$ we refer to the $n$-ary connectives $\land$ and $\lor$.
	%We call the clause a minterm, if $\nodes_0\cup\nodes_1 = [\atomorder]$.
	We call the term a minterm and the clause a maxterm, if $\nodes_0\cup\nodes_1 = [\atomorder]$.
\end{definition}

%% 
Terms and Clauses have for any index tuple $\catindexof{[\atomorder]}$ a polynomial representation by
		\[ \termof{\nodes_0}{\nodes_1}[\shortcatvariables=\catindexof{[\atomorder]}] 
		= \left( \prod_{\atomenumerator \in \nodes_0} (1-\catindexof{\atomenumerator}) \right)
		\left(  \prod_{\atomenumerator \in \nodes_1} \catindexof{\atomenumerator} \right) \]
and
		\[ \clauseof{\nodes_0}{\nodes_1}[\shortcatvariables=\catindexof{[\atomorder]}] 
		= 1 - \left( \prod_{\atomenumerator \in \nodes_0} (1-\catindexof{\atomenumerator})\right)
		\left(  \prod_{\atomenumerator \in \nodes_1} \catindexof{\atomenumerator} \right) \, . \]


\begin{lemma}\label{lem:termClauseOneHot}
	Terms are contractions of one-hot encodings, that is for any disjoint subsets $\nodes_0,\nodes_1\subset[\atomorder]$ we have
		\[ \termof{\nodes_0}{\nodes_1}[\shortcatvariables] = \contractionof{\onehotmapof{\{\atomlegindexof{k}=0 : k \in \nodes_0 \} \cup \{\atomlegindexof{k}=1 : k \in \nodes_1\}}}{\shortcatvariables} \, . \]
	Clauses are substractions of one-hot encodings from the trivial tensor, that is for any disjoint subsets $\nodes_0,\nodes_1\subset[\atomorder]$ we have
		\[ \clauseof{\nodes_0}{\nodes_1}[\shortcatvariables] = 
		\onesat{\shortcatvariables} -
		\contractionof{\onehotmapof{\{\atomlegindexof{k}=0 : k \in \nodes_0 \} \cup \{\atomlegindexof{k}=1 : k \in \nodes_1\}}}{\shortcatvariables} \, . \]
\end{lemma}


	
%
The reference of the formulas in the case $\nodes_0\dot{\cup}\nodes_1 = [\atomorder]$ as minterms and maxterms is due to the fact, that minterms are formulas with unique models and maxterms are formulas with a unique world not satisfying the formula.
% Enumeration by $\atomstates$
We use this insight and enumerate maxterms and minterms by the index $\catindex\in\atomstates$ of the unique world where the minterm is satisfied, respectively the maxterm is not satisfied.
For any $\nodes_0\dot{\cup}\nodes_1 = [\atomorder]$ we take the index tuple $\catindices$ where $\catindexof{\atomenumerator}=0$ if $\atomenumerator\in\nodes_0$ and $\catindexof{\atomenumerator}=1$ if $\atomenumerator\in\nodes_1$ and define
\begin{align*}
	\maxtermof{\catindices} = \clauseof{\nodes_0}{\nodes_1} \quad \text{and} \quad \mintermof{\catindices} = \termof{\nodes_0}{\nodes_1} \, .
\end{align*}


\begin{corollary}
	Minterms are basis elements of the tensor space, that is for any $\catindices\in\atomstates$ we have
	\begin{align*}
		\mintermof{\catindices} = \onehotmapof{\catindices}
	\end{align*}
	Maxterms are substraction of basis elements from the trivial tensor, that is for any $\catindices\in\atomstates$ we have
	\begin{align*}
		\maxtermof{\catindices} = \onesat{\shortcatvariables} - \onehotmapof{\catindices} \, .
	\end{align*}
\end{corollary}
\begin{proof}
	Follows from Lemma~\ref{lem:termClauseOneHot}, since when $\nodes_0\cup\nodes_1 = [\atomorder]$ the contractions of the one-hot encodings coincides with the one-hot encoding of a fully specified state.
\end{proof}


Based on this insight, we can decompose any propositional formula into a conjunction of maxterms or a disjunction of minterms as we show next.

\begin{theorem}\label{the:tensorToMaxMinTerms}
	For any binary tensor $\hypercoreat{\shortcatvariables}\in\bigotimes_{\atomenumeratorin}\rr^2$ with two-dimensional axes we have
	\begin{align*}
		\hypercoreat{\shortcatvariables} = \left( \bigvee_{\catindices : \hypercoreat{\shortcatvariables=\catindexof{[\atomorder]}}=1} 
		\termof{
		\{\atomenumerator : \catindexof{\atomenumerator}=0\}
		}{
		\{\atomenumerator: \catindexof{\atomenumerator}=1\}
		} 
		\right)
		[\shortcatvariables] 
	\end{align*}
	and
	\begin{align*}
		\hypercoreat{\shortcatvariables} = \left( \bigwedge_{\catindices : \hypercoreat{\shortcatvariables=\catindexof{[\atomorder]}}=0} 
		\clauseof{
		\{\atomenumerator : \catindexof{\atomenumerator}=0\}
		}{
		\{\atomenumerator: \catindexof{\atomenumerator}=1\}
		} 
		\right)
		[\shortcatvariables] \, .
	\end{align*}
\end{theorem}
\begin{proof}
	To show the representation by minterms we use the decomposition
	\begin{align*}
		\hypercoreat{\shortcatvariables}  = \sum_{\catindexof{[\atomorder]} : \hypercoreat{\catvariableof{[\atomorder]}=\catindexof{[\atomorder]}} = 1} \onehotmapofat{\catindexof{[\atomorder]}}{\shortcatvariables}
	\end{align*}
	and notice that each term in the disjunction modifies the formula by adding respective world $\catindexof{[\atomorder]}$ to the models of the formula.
	To show the representation by maxterms we use the decomposition
	\begin{align*}
		\hypercoreat{\shortcatvariables}  = \onesat{\shortcatvariables} \quad - \sum_{\catindexof{[\atomorder]} : \hypercoreat{\catvariableof{[\atomorder]}=\catindexof{[\atomorder]}} = 0} \onehotmapofat{\catindexof{[\atomorder]}}{\shortcatvariables}
	\end{align*}
	and notice that each term in the conjunction modifies the formula by removing the respective world $\catindexof{[\atomorder]}$ from the models of the formula.	
	Thus, both decompositions are propositional formulas with the same set of models as the formula $\hypercore$ and are thus identical to $\hypercore$.
\end{proof}


% Canonical normal forms
The decompositions found in Theorem~\ref{the:tensorToMaxMinTerms} are also called canonical normal forms to propositional formulas $\hypercoreat{\shortcatvariables}$.

%\begin{theorem}\label{the:FormulaToTensor}
%	For any binary tensor $\hypercore\in\bigotimes_{\atomenumeratorin}\rr^2$ with two-dimensional axes we find a formula $\exformula$ in propositional syntax such that
%		\[ \hypercore = \exformula \, . \]
%\end{theorem}
%\begin{proof}
%	For any such $\hypercore$ we construct the formula
%		\[ \exformula^{\hypercore} = \lor_{\atomindices : \hypercore_{\atomindices}=1} \clauseof{\{\atomenumerator : \atomlegindexof{\atomenumerator=0}\}}{\{\atomenumerator : \atomlegindexof{\atomenumerator=1}\}}  \, . \]
%	Since the clauses are basis vectors and $\hypercore$ is binary we get
%		\[ \exformula^{\hypercore} = \sum_{\atomindices : \hypercore_{\atomindices}=1} \onehotmapof{\atomindices} = \hypercore \, .   \] 
%%	Take for any $\ftensor$ the formula
%%		\[ \exformula = \lor_{\atomindices : \ftensor_{:,\atomindices} = \tbasis}
%%		 \left( \land_{\atomenumerator : \atomlegindexof{\atomenumerator} =1} \atomicformulaof{\atomenumerator}\right)
%%		\left( \land_{\atomenumerator : \atomlegindexof{\atomenumerator} =0} \lnot \atomicformulaof{\atomenumerator} \right) \]
%%	to show the claim.
%\end{proof}

%
%In the proof of Theorem~\ref{the:FormulaToTensor} we only applied the connectives $\land,\lor$ and $\lnot$ in the construction of syntactical specifications of formulas. 
%These connectives thus universal in the sense, that their combinations are representing any binary tensor as in Theorem~\ref{the:FormulaToTensor}.
%This observation can be connected with the theory of normal forms, where arbitrary formulas can be syntactically represented by these connectives only (often called logical equivalent, which in our case means identical formula tensor). 




%% Universality of representations
\begin{remark}[Efficient Representation in Propositional Syntax]
	% Relation with binary CP
	The decomposition in Theorem~\ref{the:tensorToMaxMinTerms} is a basis CP decomposition of the binary tensor and will further be investigated in Chapter~\ref{cha:sparseTC}. 
	The formulas constructed in the proof of Theorem~\ref{the:tensorToMaxMinTerms} are however just one possibility to represent a formula tensor in propositional syntax.
	Typically there are much sparser representations for many formula tensors, in the sense that less connectives and atomic symbols are required.
	Having such a sparser syntactical description of a propositional formula can be exploited to find a shorter conjunctive normal form of the formula and construct a sparse polynomial based on similar ideas as in Theorem~\ref{the:tensorToMaxMinTerms}.
	%One way to eliminate syntactical redundancies are through schemes for decompositions called normal forms, for example the Conjunctive Normal Form (CNF) or the Disjunctive Normal Form (DNF).
	We will provide such constructions in Chapter~\ref{cha:sparseTC}, where we show that dropping the demand of directionality and investigating binary CP Decompositions will improve the sparsity of the polynomial formula representation.
\end{remark}


\subsection{Discussion and Outlook}

Further study of representing Knowledge Bases based on Tensor Networks of its formulas in Section~\ref{sec:hardNetworks} (see Theorem~\ref{the:conDecKB}).




%\subsection{Knowledge Bases as Tensor Networks}
%
%%We here aim at a representation of the semantics of a Knowledge Base, whereas traditional systems store the Knowledge Base exploiting the syntax (i.e.storing the known formulas in the propositional syntax).
%% In this formalism a Knowledge Base is represented by its models, i.e. worlds where it is true.
%
%% Representation of multiple formuals
%Let us investigate how we to store a Knowledge Base of formulas $\exformula\in\formulaset$
%\begin{align}
%	\kb = \land_{\exformulain}\exformula \, .
%\end{align}
%
%One obvious way is to use the scheme of Theorem~\ref{the:compositionByContraction} and contract the relational encoded formulas with a conjunction encoding, that is
%\begin{align}
%	\rencodingofat{\kb}{\shortcatvariables,\catvariableof{\kb}}
%	= \contractionof{
%	\{\rencodingofat{\land}{\catvariableof{\formulaset}, \catvariableof{\kb}}\}
%	\cup\{\rencodingofat{\exformula}{\shortcatvariables,\catvariableof{\exformula}}  : \exformula \in \formulaset \}}{\shortcatvariables,\catvariableof{\kb}}
%\end{align}
%Here we denote by $\land$ the $\cardof{\formulaset}$-ary conjunction, which is well-defined by Remark~\ref{rem:naryConnectives}.
%
%% Simpler by effective calculus
%It is however possible to execute the $\cardof{\formulaset}$-ary conjunction by effective calculus (see Section~\ref{sec:effectiveGroundingCalculus}) and we have
%\begin{align}
%	\kb[\shortcatvariables]= \contractionof{\{\formulaat{\shortcatvariables} : \exformulain \}}{\shortcatvariables} \, .
%\end{align}











    \chapter{\chatextlogicalReasoning}\label{cha:logicalReasoning}

We approach logical inference by defining probability distributions based on propositional formulas and then apply the methodology introduced in the more generic situation of probabilistic inference.
Logical approaches pay here special attention to situations of certainty, where a state of a variable has probability $1$.
In this situation, we say that the corresponding formula is entailed.
%Such situations are called entailment, and we will investigate how we can find these by contractions.

% From Probabilistic 
We start the discussion with the derivation of contraction criteria for logical entailment.
We interpret formulas by distributions and extend logical entailment towards probabilistic reasoning.
%This enables us to define logical entailment based on the resulting conditional distributions.

\sect{Entailment in Propositional Logics}

Entailment is the central consequence relation among logical formulas.
Let us define this relation first based on the models of a knowledge base and a test formula.

\begin{definition}[Entailment of propositional formulas]
    \label{def:logicalEntailment}
    Given two propositional formulas $\kb$ and $\exformula$ we say that $\kb$ entails $\exformula$, denoted by $\kb\models\exformula$, if any model of $\kb$ is also a model of $\exformula$, that is
    \begin{align*}
        \uniquantwrtof{\shortatomindicesin}{\imppremhead{\kbat{\indexedshortcatvariables}=1}{\formulaat{\indexedshortcatvariables}=1}} \, .
    \end{align*}
    If $\kb\models\lnot\exformula$ holds, we say that $\kb$ contradicts $\exformula$.
\end{definition}

% Connection with tensor formalism
To use the tensor network formalism for the decision of entailment, we will in the following develop three equivalent criteria for entailment. %, based on contractions.

\subsect{Deciding Entailment by Contractions}

First of all, we can decide entailment based on vanishing contractions with the negated test formula.

\begin{theorem}[Contraction Criterion of Entailment]
    \label{the:contCriterionLogEntailment}
    We have $\kb\models\exformula$ if and only if
    \begin{align*}
        \contraction{\kb,\lnot\exformula} = 0 \, .
    \end{align*}
\end{theorem}
\begin{proof}
    \proofleftsymbol:
    If for a $\shortatomindicesin$ we have $\kbat{\indexedshortcatvariables}=1$ but not $\big(\exformulaat{\indexedshortcatvariables}=1\big)$, we would have $\big(\lnot\exformulaat{\indexedshortcatvariables}=1\big)$ and
    \begin{align*}
        \contraction{\kb,\lnot\exformula} =
        \sum_{\shortatomindicesin} \kbat{\indexedshortcatvariables} \cdot \exformulaat{\indexedshortcatvariables} > 1 \, .
    \end{align*}
    Thus, whenever the contraction vanishes, we have
    \begin{align*}
        \uniquantwrtof{\shortatomindicesin}{\imppremhead{\kbat{\indexedshortcatvariables}=1}{\formulaat{\indexedshortcatvariables}=1}} \, .
    \end{align*}
    %for all $\shortatomindicesin$ that $\big(\kbat{\indexedshortcatvariables}=1\big) \rightarrow \big(\formulaat{\indexedshortcatvariables}=1\big)$.

    \proofrightsymbol:
    Conversely, if the contraction $\contraction{\kb,\lnot\exformula}$ does not vanish, we would find $\shortatomindicesin$ with $\kbat{\indexedshortcatvariables}=1$ and $\lnot\formulaat{\indexedshortcatvariables}=1$, therefore $\formulaat{\indexedshortcatvariables}=0$.
    It follows that $\kb\models\exformula$ does not hold.
\end{proof}


% Can use basis encoding
The contraction criterion can be extended to the decision of contradiction as well, since $\kb\models\lnot\exformula$ is equivalent to $\contraction{\kb,\exformula}=0$.
Therefore, entailment and contradiction can be decided simultaneously by a single contraction, as we state next.
%To decide whether a formula is entailed, or its negation is entailed (in which case one says that the formula is contradicted) by a single contraction, one can perform the contraction
%\begin{align*}
%	\hypercore = \contractionof{\kbat{\shortcatvariables},\formulaat{\shortcatvariables,\exformulavar}}{\exformulavar}
%\end{align*}

\begin{theorem}
    \label{the:entailmentContradictionContraction}
    Given propositional formulas $\kb$ and $\exformula$ we build
    \begin{align*}
        \hypercoreat{\exformulavar}
        = \contractionof{\kbat{\shortcatvariables},\formulaccwith}{\exformulavar} \, .
    \end{align*}
    Then $\kb\models\exformula$ is equivalent to $\hypercoreat{\exformulavar=0}=0$, and $\kb\models\lnot\exformula$ is equivalent to $\hypercoreat{\exformulavar=1}=0$ \, .
\end{theorem}
\begin{proof}
    This follows from \theref{the:contCriterionLogEntailment} using that
    \begin{align*}
        \contraction{\kb,\lnot\exformula} = \hypercoreat{\exformulavar=0}
    \end{align*}
    and
    \begin{align*}
        \contraction{\kb,\exformula} = \hypercoreat{\exformulavar=1} \, . & \qedhere
    \end{align*}
\end{proof}





\subsect{Deciding Entailment by Partial Ordering}

% Subset relation
Logical entailment can be understood by subset relations of the models of the respective formulas.
This perspective can be applied with subset encodings in \charef{cha:basisCalculus}.
The subset relation corresponds with partial ordering of its encoded tensors, as will be shown in \theref{the:subsetRelationSubsetEncoding}.
For two propositional formulas, we denote to this end $\exformula\prec\secexformula$ (see \defref{def:partialOrder}), if and only if for all $\shortatomindicesin$
\begin{align*}
    \exformulaat{\indexedshortcatvariables} \leq \secexformulaat{\shortcatindices}  \, .
\end{align*}

\begin{theorem}[Partial Ordering Criterion of Entailment]
    \label{the:orderingEntailmentCriterion}
    We have $\kb\models\exformula$ if and only if $\kbat{\shortcatvariables}\prec\exformulaat{\shortcatvariables}$.
\end{theorem}
\begin{proof}
    Since both $\kb$ and $\exformula$ are boolean tensors, we have for any $\shortatomindicesin$ that
    \begin{align*}
        \kbat{\indexedshortcatvariables},\formulaat{\indexedshortcatvariables}\in\ozset \, .
    \end{align*}
    Thus,
    \begin{align*}
        \uniquantwrtof{\shortatomindicesin}{\kbat{\indexedshortcatvariables}\leq\formulaat{\indexedshortcatvariables}}
    \end{align*}
    is equivalent to
    \begin{align*}
        \uniquantwrtof{\shortatomindicesin}{\imppremhead{\kbat{\indexedshortcatvariables}=1}{\formulaat{\indexedshortcatvariables}=1}} \, .
    \end{align*}
    This states that $\kbat{\shortcatvariables}\prec\exformulaat{\shortcatvariables}$ is equivalent to $\kb\models\exformula$.
\end{proof}



\subsect{Redundancy of Entailed Formulas}

Another interpretation of entailment is by redundancy of a formula in a Knowledge Base.
This is especially interesting for the sparse representation of Knowledge Bases.
%Towards getting insights on this we first show that entailed formulas can be dropped from the Knowledge Base.

\begin{theorem}[Redundancy Criterion of Entailment]
    \label{the:ReduncancyOfEntailed}
    If and only if $\kb\models\exformula$ we have
    \begin{align*}
        \kbat{\shortcatvariables}= \contractionof{\kb,\exformula}{\shortcatvariables}  \, .
    \end{align*}
\end{theorem}
\begin{proof}
    For any formula $\exformula$ we have
    \begin{align*}
        \onesat{\shortcatvariables} = \exformulaat{\shortcatvariables} + \lnot\exformulaat{\shortcatvariables}
    \end{align*}
    and thus
    \begin{align*}
        \kbat{\shortcatvariables}
        & = \contractionof{\kbat{\shortcatvariables},\onesat{\shortcatvariables}}{\shortcatvariables} \\
        & = \contractionof{\kbat{\shortcatvariables},\exformulaat{\shortcatvariables}}{\shortcatvariables} +  \contractionof{\kbat{\shortcatvariables},\lnot\exformulaat{\shortcatvariables}}{\shortcatvariables} \, .
    \end{align*}
    Now, by \theref{the:contCriterionLogEntailment} we have $\kb\models\exformula$, if and only if $\contractionof{\kbat{\shortcatvariables},\lnot\exformulaat{\shortcatvariables}}{\shortcatvariables}=0$, which is thus equal to
    \begin{align*}
        \kbat{\shortcatvariables}
        = \contractionof{\kbat{\shortcatvariables},\exformulaat{\shortcatvariables}}{\shortcatvariables} \, . & \qedhere
    \end{align*}
\end{proof}

%This provides us with another interpretation of the entailment relation, in terms of redundant formulas.
%\begin{remark}[Sparsest Description of a Knowledge Base]
%	Given a set of worlds indexed by $\hypercore$, find the sparsest set of formulas $\kb$ such that
%		\[ \hypercore = {\kb} \]
%	would be benefitial for small computational complexity.
%	Since the formula tensors are invariant under entailment, we can drop entailed formulas.
%\end{remark}

\subsect{Contraction Knowledge Base}

We exploit the contraction and redundancy criteria of entailment to sketch an implementation of a propositional Knowledge Base in \algoref{alg:contractionKB}.
Here the function $\mathrm{ASK}(\exformula)$ returns, whether a formula $\exformula$ is entailed or contradicted by a Knowledge Base.
If the formula is neither entailed or contradicted, we say it is contingent.
If it is both, we have $\kbat{\shortcatvariables}=0$ and thus an inconsistent Knowledge Base.
Exploiting \theref{the:entailmentContradictionContraction} we decide these situations based on a single contraction.

The function $\mathrm{TELL}(\exformula)$ incorporates an additional formula $\exformula$ into a Knowledge Base $\kb$.
Here we exploit \theref{the:ReduncancyOfEntailed} and do not add a formula, which is entailed in order to maintain a sparse representation. %(in which case it returns Incons).
The function further refuses to add a formula, which would make the Knowledge Base inconsistent (returns Refused) and only changes the Knowledge Base in case of a contingent formula (returns Added).


\begin{figure}
    \begin{center}
        \begin{tikzpicture}[scale=0.35]

    \node[anchor=center] (Or) at (-0.4,0){
        \begin{tabular}{|c|c|}
            \hline
            \rule{0pt}{1.3em} \stringof{Contingent}    & \stringof{Entailed}     \\
            \hline
            \rule{0pt}{1.3em} \stringof{Contracticted} & \stringof{Inconsistent} \\
            \hline
        \end{tabular}
    };

    \node[above] (Dec) at (0,4) {$\kb\models\exformula$};
    \node[above] (Dec) at (-0.5,2) {$\falsesymbol$ \hspace {1.7cm} $\truesymbol$};
    \draw (0,2.2) -- (0,4);

    \begin{scope}[shift={(1.5,0)}]
        \node[left] (Dec) at (-13.25,0) {$\kb\models\lnot\exformula$};
        \node[left] (Dec) at (-10,1) {$\falsesymbol$};
        \draw (-10,0) -- (-13,0);
        \node[left] (Dec) at (-10,-1) {$\truesymbol$};
    \end{scope}

\end{tikzpicture}
    \end{center}
    \caption{Table of possible logical relations between a knowledge base $\kb$ and $\formula$, based on whether the knowledge base entails the formula ($\kb\models\formula$) and its negation ($\kb\models\lnot\formula$).}\label{fig:askDecisionTable}
\end{figure}


% Works also for Markov Networks!
\begin{algorithm}[hbt!]
    \caption{Contraction Knowledge Base with operations ASK and TELL}\label{alg:contractionKB}
    \vspace{0.3cm} \textbf{ASK}($\kb$, $\exformula$)
    \begin{algorithmic}
    \iosepline
    \Require Knowledge base $\kb$, query formula $\exformula$
    \Ensure Decision which relation between $\kb$ and $\exformula$ holds (see \figref{fig:askDecisionTable})
    \iosepline
        \State{$\hypercoreat{\formulavar} \algdefsymbol \contractionof{\{\secexformulaat{\shortcatvariables} \wcols \secexformula\in\kb\},\bencodingofat{\exformula}{\formulavar,\shortcatvariables}}{\formulavar}$}
        \If{$\hypercoreat{\formulavar=0}=0$ and $\hypercoreat{\formulavar=1}=0$ }
            \State \Return \stringof{Inconsistent}
        \ElsIf{$\hypercoreat{\formulavar=0}=0$}
            \State \Return \stringof{Entailed}
        \ElsIf{$\hypercoreat{\formulavar=1}=0$}
            \State \Return \stringof{Contradicted}
        \Else
            \State \Return \stringof{Contingent}
        \EndIf
    \vspace{0.1cm}
    \hrule
    \end{algorithmic}
    \vspace{0.3cm} \textbf{TELL}($\kb$, $\exformula$)
    \begin{algorithmic}
        \iosepline
    \Require Knowledge base $\kb$, query formula $\exformula$
    \Ensure Decision whether a formula is added to the knowledge base (\stringof{Added}), or an exception in (\stringof{Inconsistent}, \stringof{Redundant} or \stringof{Refused}) is raised.
    \iosepline
        \State{answer $\algdefsymbol$ ASK($\exformula$)}
        \If{answer is \stringof{Inconsistent}:}
            \State \Return \stringof{Inconsistent}
        \ElsIf{answer is \stringof{Entailed}:}
            \State \Return \stringof{Redundant}
        \ElsIf{answer is \stringof{Contradicted}:}
            \State \Return \stringof{Refused}
        \ElsIf{answer is \stringof{Contingent}:}
            \State $\kb \algdefsymbol \kb\cup\{\exformula\}$
            \State \Return \stringof{Added}
        \EndIf
    \end{algorithmic}

\end{algorithm}



\sect{Formulas as Random Variables}

In order to present logical entailment as extreme cases of more generic probabilistic reasoning, we now provide probabilistic interpretations of propositional formulas.
% Interpretation of basis encoding as conditional probabilities
In the next sections, we will investigate two ways of interpreting basis encodings of formulas as conditional probabilities.
The atom centric one, which understands the atomic legs as conditions and calculates the truth of the formula, leads to a direct interpretation of $\bencodingof{\exformula}$ as a conditional probability distribution.
When instead taking the formula itself centric, we get uniform distributions of its models and the complement, when conditioning on the satisfaction of the formula.

\subsect{Probabilistic Queries by Formulas}

Let $\probat{\shortcatvariables}$ be a joint distribution of atomic variables $\catvariableof{\atomenumerator}$, where $\atomenumeratorin$, taking variables in $\catdimof{\atomenumerator}=2$.
Let us then ask a query in the formalism of \defref{def:queries}, where the query function is assumed to be a propositional formula.
The joint distribution can be extended to a variable $\exformulavar$ representing the satisfaction of a formula $\exformula$ given an assignment to the atoms, by adding its basis encoding as
\begin{align*}
    \probat{\exformulavar,\shortcatvariables}
    = \contractionof{\bencodingofat{\exformula}{\exformulavar,\shortcatvariables},\probat{\shortcatvariables}}{\shortcatvariables} \, .
\end{align*}
Let us note, that this is a normalized probability distribution, since $\contractionof{\bencodingofat{\exformula}{\exformulavar,\shortcatvariables}}{\shortcatvariables}=\onesat{\shortcatvariables}$ and $\probat{\shortcatvariables}$ is normalized.

Conditioning this probability distribution on the atoms, we get
\begin{align*}
    \condprobof{\formulavar}{\shortcatvariables}
    = \bencodingofat{\exformula}{\shortcatvariables} \, .
\end{align*}
We thus interpret the basis encoding of a formula as a conditional probability of $\exformula$ given the assignments to the atoms $\shortcatvariables$ and depict this by
\begin{center}
    \begin{tikzpicture}[scale=0.35,thick] % , baseline = -3.5pt



\draw[->-] (2,-1)--(2,1) node[midway,right] {\tiny $\formulavar$};

\draw (-3,-1) rectangle (7,-3);
\node[anchor=center] (text) at (2,-2) {\small $\condprobof{\formulavar}{\atomicformulas}$};
\draw[-<-] (0,-3)--(0,-5) node[midway,left] {\tiny $\catvariableof{0}$};
\draw[-<-] (1.5,-3)--(1.5,-5) node[midway,left] {\tiny $\catvariableof{1}$};
\node[anchor=center] (text) at (3,-4) {$\cdots$};
\draw[-<-] (4,-3)--(4,-5) node[midway,right] {\tiny $\catvariableof{\atomorder\shortminus1}$};


\node[anchor=center] (text) at (9,-2) {${=}$};


\begin{scope}[shift={(12,0)}]

\draw[->-] (2,-1)--(2,1) node[midway,right] {\tiny $\formulavar$};
\draw (-1,-1) rectangle (5,-3);
\node[anchor=center] (text) at (2,-2) {\small $\rencodingof{\exformula}$};
\draw[-<-] (0,-3)--(0,-5) node[midway,left] {\tiny $\catvariableof{0}$};
\draw[-<-] (1.5,-3)--(1.5,-5) node[midway,left] {\tiny $\catvariableof{1}$};
\node[anchor=center] (text) at (3,-4) {$\cdots$};
\draw[-<-] (4,-3)--(4,-5) node[midway,right] {\tiny $\catvariableof{\atomorder\shortminus1}$};

\node[anchor=center] (text) at (7,-5) {${\cdot}$};

\end{scope}


\end{tikzpicture}
\end{center}
To be more precise, we have for any $\shortcatindices$
\begin{align*}
    \condprobof{\formulavar}{\indexedshortcatvariables} =
    \begin{cases}
        \fbasisat{\formulavar} & \ifspace \exformulaat{\indexedshortcatvariables} = 0 \text{, i.e. $\shortcatindices$ is not a model of $\exformula$} \\
        \tbasisat{\formulavar} & \ifspace \exformulaat{\indexedshortcatvariables} = 1 \text{, i.e. $\shortcatindices$ is a model of $\exformula$}
    \end{cases} \, .
\end{align*}

% Interpretation of directionality as
Since the conditional query $\condprobof{\formulavar}{\shortcatvariables}$ provides an interpretation of $\bencodingof{\exformula}$ as a conditional probability, we interpret $\probat{\exformulavar}$ as a marginal distribution inherited by $\probat{\shortcatvariables}$.
This is also reflected in the fact that both $\condprobof{\formulavar}{\shortcatvariables}$ and $\bencodingofat{\exformula}{\exformulavar,\shortcatvariables}$ are directed, since the first is a normalization by \defref{def:queries} and the second an basis encoding of a formula.
Probabilistic queries (see \defref{def:queries}), which functions are propositional formulas are thus answered by the satisfaction rate of a propositional formula given a joint distribution of the corresponding atoms.


%This trivialization is depicted as:
%\begin{center}
%	\begin{tikzpicture}[scale=0.35,thick] % , baseline = -3.5pt



\draw (1,1) rectangle (3,3);
\node[anchor=center] (text) at (2,2) {\small $\ones$};
\draw[->] (2,-1)--(2,1) node[midway,right] {\tiny $\formulavar$};

\draw (-1,-1) rectangle (5,-3);
\node[anchor=center] (text) at (2,-2) {\small $\rencodingof{\exformula}$};
\draw[<-] (0,-3)--(0,-5) node[midway,left] {\tiny $\catvariableof{0}$}; 
\draw[<-] (1.5,-3)--(1.5,-5) node[midway,left] {\tiny $\catvariableof{1}$}; 
\node[anchor=center] (text) at (3,-4) {$\cdots$};
\draw[<-] (4,-3)--(4,-5) node[midway,right] {\tiny $\catvariableof{\atomorder-1}$}; 


\node[anchor=center] (text) at (7,-2) {${=}$};

\begin{scope}[shift={(10,0)}]

\draw (1,1) rectangle (3,3);
\node[anchor=center] (text) at (2,2) {\small $\ones$};

\draw[->] (2,-1)--(2,1) node[midway,right] {\tiny $\formulavar$};
\draw (-1,-1) rectangle (5,-3);
\node[anchor=center] (text) at (2,-2) {\small $\condprobof{\formulavar}{\atomicformulaof{\atomenumerator}}$};
\draw[<-] (0,-3)--(0,-5) node[midway,left] {\tiny $\catvariableof{0}$}; 
\draw[<-] (1.5,-3)--(1.5,-5) node[midway,left] {\tiny $\catvariableof{1}$}; 
\node[anchor=center] (text) at (3,-4) {$\cdots$};
\draw[<-] (4,-3)--(4,-5) node[midway,right] {\tiny $\catvariableof{\atomorder\shortminus1}$}; 

\end{scope}

\node[anchor=center] (text) at (17,-2) {${=}$};

\begin{scope}[shift={(20,0)}]

\draw (-1,-1) rectangle (5,-3);
\node[anchor=center] (text) at (2,-2) {\small $\ones$};
\draw[<-] (0,-3)--(0,-5) node[midway,left] {\tiny $\catvariableof{0}$}; 
\draw[<-] (1.5,-3)--(1.5,-5) node[midway,left] {\tiny $\catvariableof{1}$}; 
\node[anchor=center] (text) at (3,-4) {$\cdots$};
\draw[<-] (4,-3)--(4,-5) node[midway,right] {\tiny $\catvariableof{\atomorder\shortminus1}$}; 

\node[anchor=center] (text) at (6,-5) {$.$};

\end{scope}


\end{tikzpicture}
%\end{center}

%% Conditional interpretation -> Formulas as conditional probability ("local")
%Our main interpretation understands each tuple of indices $\shortcatindices$ as conditions of a probability tensor.
%Since the satisfaction of a formula is determined by $\shortcatindices$,
%Given a truth assignment to the atomic variables $\atomicformulaof{\atomenumerator}$, that is a choice of indices $\atomlegindexof{\atomenumerator}$, the truth of the formula.


%\begin{theorem}\label{the:conditionByAtoms}
%	Given any distribution $\probat{\shortcatvariables}$ of atomic variables, we extend the joint distribution to a formula variable $\exformulavar$ by contraction with the basis encoding $\bencodingofat{\exformula}{\exformulavar,\shortcatvariables}$.
%	Then the basis encoding $\bencodingofat{\exformula}{\exformulavar,\shortcatvariables}$ coincides with the conditional probability of that formula conditioned on the atoms, that is
%	\begin{align*}
%		\bencodingofat{\exformula}{\shortcatvariables}
%		= \condprobof{\formulavar}{\shortcatvariables} \, .
%	\end{align*}
%	We depict this by
%	\begin{center}
%		\begin{tikzpicture}[scale=0.35,thick] % , baseline = -3.5pt



\draw[->] (2,-1)--(2,1) node[midway,right] {\tiny $\formulavar$};

\draw (-3,-1) rectangle (7,-3);
\node[anchor=center] (text) at (2,-2) {\small $\condprobof{\formulavar}{\atomicformulas}$};
\draw[<-] (0,-3)--(0,-5) node[midway,left] {\tiny $\catvariableof{0}$}; 
\draw[<-] (1.5,-3)--(1.5,-5) node[midway,left] {\tiny $\catvariableof{1}$}; 
\node[anchor=center] (text) at (3,-4) {$\cdots$};
\draw[<-] (4,-3)--(4,-5) node[midway,right] {\tiny $\catvariableof{\atomorder\shortminus1}$}; 


\node[anchor=center] (text) at (9,-2) {${=}$};


\begin{scope}[shift={(12,0)}]

\draw[->] (2,-1)--(2,1) node[midway,right] {\tiny $\formulavar$};
\draw (-1,-1) rectangle (5,-3);
\node[anchor=center] (text) at (2,-2) {\small $\rencodingof{\exformula}$};
\draw[<-] (0,-3)--(0,-5) node[midway,left] {\tiny $\catvariableof{0}$}; 
\draw[<-] (1.5,-3)--(1.5,-5) node[midway,left] {\tiny $\catvariableof{1}$}; 
\node[anchor=center] (text) at (3,-4) {$\cdots$};
\draw[<-] (4,-3)--(4,-5) node[midway,right] {\tiny $\catvariableof{\atomorder\shortminus1}$}; 

\node[anchor=center] (text) at (7,-5) {${\cdot}$};

\end{scope}


\end{tikzpicture}
%	\end{center}
%\end{theorem}
%\begin{proof}
%	The distribution $\probtensor$ does not influence the conditional query, since the normalization acts on any state.
%
%\end{proof}

\subsect{Uniform Distributions of the Models}

% Defining probability distribution by formulas
Let us now converse the order of conditioning from $\condprobof{\exformulavar}{\shortcatvariables}$ to $\condprobof{\shortcatvariables}{\exformulavar}$.
In this way, we understand a propositional formula as a definition of a joint probability distributions of the atoms, instead of a formulation of a probabilistic query against a joint distribution.
To this end, we define by the single tensor core $\{\bencodingofat{\exformula}{\exformulavar,\shortcatvariables}\}$ a Markov Network $\probofat{\{\exformula\}\cup[\catorder]}{\exformulavar,\shortcatvariables}$.
%Given a Markov Network $\probtensor$ with a single core $\bencodingof{\exformula}$ for a propositional formula $\exformula$.
By definition we have
\begin{align*}
    \condprobwrtof{\{\exformula\}\cup[\catorder]}{\shortcatvariables}{\exformulavar}
    = \normalizationofwrt{\bencodingof{\exformula}}{\shortcatvariables}{\exformulavar} \, .
\end{align*}
We depict this construction by:
\begin{center}
    \begin{tikzpicture}[scale=0.35,thick] % , baseline = -3.5pt


    \begin{scope}
        [shift={(-2,0)}]


        \draw[-<-] (2,-1)--(2,1) node[midway,right] {\colorlabelsize $\formulavar$};

        \draw (-1,-1) rectangle (5,-3);
        \node[anchor=center] (text) at (2,-2) {\corelabelsize $\condprobof{\atomicformulas}{\formulavar}$};
        \draw[->-] (0,-3)--(0,-5) node[midway,left] {\colorlabelsize $\catvariableof{0}$};
        \draw[->-] (1.5,-3)--(1.5,-5) node[midway,left] {\colorlabelsize $\catvariableof{1}$};
        \node[anchor=center] (text) at (3,-4) {$\cdots$};
        \draw[->-] (4,-3)--(4,-5) node[midway,right] {\colorlabelsize $\catvariableof{\atomorder\shortminus1}$};


        \node[anchor=center] (text) at (7,-2) {${=}$};

    \end{scope}


    \node[anchor=center] (text) at (8,-2.25) {$\sum\limits_{\headindexof{\exformula}\in[2]}$};

    \draw[] (10,-1) rectangle (12,-3);
    \node[anchor=center] (text) at (11,-2) {\corelabelsize $\onehotmapof{\headindexof{\exformula}}$};

    \draw[->-] (11,-3)--(11,-5) node[midway,right] {\colorlabelsize $\formulavar$};



    \begin{scope}
        [shift={(15,0)}]

        \draw[] (1,1) rectangle (3,3);
        \node[anchor=center] (text) at (2,2) {\corelabelsize $\onehotmapof{\headindexof{\exformula}}$};

        \draw[->-] (2,-1)--(2,0.5) node[right] {\colorlabelsize $\formulavar$};
        \drawvariabledot{2}{0}
        \draw (2,0.5) -- (2,1);
        \draw (-1,-1) rectangle (5,-3);
        \node[anchor=center] (text) at (2,-2) {\corelabelsize $\bencodingof{\exformula}$};
        \draw[-<-] (0,-3)--(0,-5) node[midway,left] {\colorlabelsize $\catvariableof{0}$};
        \draw[-<-] (1.5,-3)--(1.5,-5) node[midway,left] {\colorlabelsize $\catvariableof{1}$};
        \node[anchor=center] (text) at (3,-4) {$\cdots$};
        \draw[-<-] (4,-3)--(4,-5) node[midway,right] {\colorlabelsize $\catvariableof{\atomorder\shortminus1}$};


    \end{scope}


    \begin{scope}
        [shift={(25,0)}]

        \draw (-5,-7) -- (0,3);

        \draw[] (1,1) rectangle (3,3);
        \node[anchor=center] (text) at (2,2) {\corelabelsize $\onehotmapof{\headindexof{\exformula}}$};

        \draw[->-] (2,-1)--(2,0.5) node[right] {\colorlabelsize $\formulavar$};
        \drawvariabledot{2}{0}
        \draw (2,0.5) -- (2,1);
        \draw (-1,-1) rectangle (5,-3);
        \node[anchor=center] (text) at (2,-2) {\corelabelsize $\bencodingof{\exformula}$};
        \draw[-<-] (0,-3)--(0,-5) node[midway,left] {\colorlabelsize $\catvariableof{0}$};
        \draw[-<-] (1.5,-3)--(1.5,-5) node[midway,left] {\colorlabelsize $\catvariableof{1}$};
        \node[anchor=center] (text) at (3,-4) {$\cdots$};
        \draw[-<-] (4,-3)--(4,-5) node[midway,right] {\colorlabelsize $\catvariableof{\atomorder\shortminus1}$};

        \draw (-1,-5) rectangle (5,-7);
        \node[anchor=center] (text) at (2,-6) {\corelabelsize $\ones$};

    \end{scope}

    \node[anchor=center] (text) at (32,-5) {$.$};

\end{tikzpicture}
\end{center}

% Conditioning on the formula being true
Let us further investigate the slices of $\condprobof{\shortcatvariables}{\exformula}$ with respect to $\exformula$, which define distributions of the states of the factored system.
To this end, let us condition on the event of $\exformula=1$, for which we have the distribution
\begin{align}
    \label{eq:eventFormulaProb}
    \condprobof{\shortcatvariables}{\formulavar=1} = \frac{1}{\contraction{\exformula}}
    \sum_{\shortatomindicesin \wcols \formulaat{\indexedshortcatvariables}=1} \onehotmapofat{\shortcatindices}{\shortcatvariables} \, .
\end{align}
With $\contraction{\exformula}$ being the number of models of $\exformula$, this is the uniform distribution among the models of $\exformula$.
Conversely, when conditioning on the event $\formulavar=0$ we get a uniform distribution of the models of $\lnot\exformula$.

% 
The probability distribution in Equation~\eqref{eq:eventFormulaProb} is well defined except for the case that $\contraction{\exformula}=0$.
In that case we would have $\exformulaat{\shortcatvariables}=\zerosat{\shortcatvariables}$ and call $\exformula$ unsatisfiable, since it has no models.

%% Uniform interpretation -> KB as probability distribution over its models ("global")
From an epistemological point of view, probability theory is a generalization of logics, since we allow for probability values in the interval $[0,1]$.
The set of distributions being constructed by conditioning on propositional formulas as in Equation~\eqref{eq:eventFormulaProb} correspond within the set of probability distributions with those being constant on their support.
% More specific
While the distributions build a $2^\atomorder-1$-dimensional manifold, the formulas parametrize by this construction $2^{\left(2^\atomorder\right)}$.%, most of which having vanishing coordinates.




\subsect{Probability of a Formula given a Knowledge Base}

% Both directions for entailment
We now combine the ideas of the previous two subsections and define probabilities of formulas $\exformula$ given the satisfaction of another formula $\kb$, which we call a knowledge base.
We have % by \theref{the:conditionByAtoms} % Again, Markov Network with bencoding of \exformula, \kb build the precise \probtensor
\begin{align*}
    \condprobof{\formulavar}{\kbvar}
    & = \contractionof{
        \condprobof{\formulavar}{\atomicformulas}, \condprobof{\atomicformulas}{\kbvar}
    }{\formulavar,\kbvar} \\
    & = \normalizationofwrt{\bencodingof{\exformula},\bencodingof{\kb}}{\formulavar}{\kbvar} \, .
\end{align*}
We notice that we have to assume a satisfiable knowledge base $\kb$ for this construction to be well-defined.

% 
Of special interest is the conditional probability of $\formulavar$ given that $\kbvar$ is satisfied, that is
\begin{align*}
    \condprobof{\formulavar}{\kbvar=1}
    & = \normalizationof{\{\bencodingof{\exformula} ,\kb\}}{\formulavar}\\
    & = \frac{\contractionof{\{\bencodingof{\exformula},\kb\}}{\formulavar}}{\contraction{\{\kb\}}} \, .
\end{align*}
This conditional probability establishes a connection with the entailment relation of propositional formulas, as we show next.

%
\begin{theorem}
    \label{the:probEntailment}
    Given a satisfiable formula $\kb$, we have $\kb\models\exformula$, if and only if
    \begin{align*}
        \condprobof{\formulavar=0}{\kbvar=1} = 0 \, .
    \end{align*}
\end{theorem}
\begin{proof}
    Since $\kb$ is satisfiable, we have $\contraction{\kb}>0$ and
    \[ \condprobof{\formulavar=0}{\kbvar=1} = \frac{\contraction{\lnot\exformula,\kb}}{\contraction{\kb}} \, .  \]
    This term vanishes if and only if $\contraction{\lnot\exformula,\kb}$ vanish.
    Now, by \theref{the:contCriterionLogEntailment} we have $\kb\models\exformula$ if and only if $\contraction{\kb,\lnot\exformula}=0$, which is therefore equal to $\condprobof{\formulavar=0}{\kbvar=1}=0$.
\end{proof}

Since any conditional distribution is directed, we have
\begin{align}
    \condprobof{\formulavar}{\kbvar=1} = %\begin{cases}
    \fbasisat{\formulavar}  & \ifspace \kb\models\lnot\exformula \\
    \tbasisat{\formulavar}  & \ifspace \kb\models\exformula \\
    %\tbasis & \text{if }\kb \models \exformula \\
    \notin \{\fbasisat{\formulavar},\tbasisat{\formulavar}\} & \text{else}
    \, .
\end{align}
We depict the case of entailment $\kb\models\exformula$ by the contraction diagram
%It suffices to check, whether the contraction with the normalized Knowledge Base is the basis vector $\tbasis$, respectively $\fbasis$, that is
\begin{center}
    \begin{tikzpicture}[scale=0.35,thick]

\draw[->-] (2,-1)--(2,1) node[midway,right] {\tiny $\exformulavar$};
\draw (-1,-1) rectangle (5,-3);
\node[anchor=center] (text) at (2,-2) {\small $\bencodingof{\exformula}$};
\draw[-<-] (0,-3)--(0,-5) node[midway,left] {\tiny $\catvariableof{0}$};
\draw[-<-] (1.5,-3)--(1.5,-5) node[midway,left] {\tiny $\catvariableof{1}$};
\node[anchor=center] (text) at (3,-4) {$\cdots$};
\draw[-<-] (4,-3)--(4,-5) node[midway,right] {\tiny $\catvariableof{\atomorder\shortminus1}$};

\draw (-1.5,-5) rectangle (5.5,-7);
\node[anchor=center] (text) at (2,-6) {\small $\normalizationof{\kb}{\shortcatvariables}$};

\node[anchor=center] (text) at (9,-4) {\small ${=}$};

\draw[->-] (13,-3)--(13,-1) node[midway,right] {\tiny $\exformulavar$};
\draw (12,-5) rectangle (14,-3);
\node[anchor=center] (text) at (13,-4) {\small $\tbasis$};

\node[anchor=center] (text) at (16,-6) {$\cdot$};

\end{tikzpicture}
\end{center}

We can further omit the normalization by $\contraction{\kb}$ when deciding entailment, and thus drop the assumption of satisfiability of $\kb$, as we state next.

% Not needed?
\begin{theorem}
    Given a formula $\kb$, we have $\kb\models\exformula$ (respectively $\kb\models\lnot\exformula$), if and only if
    \begin{align*}
        \contractionof{\kb,\bencodingof{\exformula}}{\formulavar=0} = 0
        \quad \text{( respectively }
        \contractionof{\kb,\bencodingof{\exformula}}{\formulavar=1} = 0 \, .
    \end{align*}
\end{theorem}
\begin{proof}
    This follows from \theref{the:contCriterionLogEntailment} using that
    \begin{align*}
        \bencodingofat{\exformula}{\exformulavar=0,\shortcatvariables} = \lnot\exformulaat{\shortcatvariables} \andspace
        \bencodingofat{\exformula}{\exformulavar=1,\shortcatvariables} = \exformulaat{\shortcatvariables} \, . & \qedhere
    \end{align*}
\end{proof}


%
Relating entailment to probability distributions motivates an extension of the entailment as provided by \defref{def:logicalEntailment} to arbitrary probability distributions.

\begin{definition}
    \label{def:probEntailment}
    For any propositional formula $\exformulaat{\shortcatvariables}$ we say that a probability distribution $\probof{\shortcatvariables}$ probabilistically entails $\exformula$, denoted as $\probtensor\models\exformula$, if
    \begin{align*}
        \contractionof{\probat{\shortcatvariables},\bencodingofat{\exformula}{\exformulavar,\shortcatvariables}}{\exformulavar=0} = 0 .
    \end{align*}
    If $\probtensor\models\lnot\exformula$, that is $\contractionof{\probat{\shortcatvariables},\bencodingofat{\exformula}{\exformulavar,\shortcatvariables}}{\exformulavar=1} = 0$, we say that $\probtensor$ probabilistically contradicts $\exformula$.
\end{definition}

%
We note, that when choosing for a formula $\kb$ the uniform distribution
\begin{align*}
    \probat{\shortcatvariables}=\normalizationof{\shortcatvariables}{\kbvar=1}
\end{align*}
among its models, then probabilistic entailment $\probtensor\models\exformula$ of a propositional formula $\exformula$ is by \theref{the:probEntailment} equivalent to $\kb\models\exformula$.

\subsect{Knowledge Bases as Base Measures for Probability Distributions}

% Generic Probability Tensors
Let us now further relate the probabilistic entailment provided by \defref{def:probEntailment} with logical entailment, by constructing a corresponding propositional formula to an arbitrary distribution.
Given a generic probability distribution $\probtensor$ we can build a Knowledge Base by
\begin{align*}
    \kb^{\probtensor} = \nonzerofunction \circ \probtensor \, ,
\end{align*}
where $\nonzerofunction:\rr\rightarrow \rr$ denotes the indicator function of the support defined as
\begin{align}
    \nonzeroof{x}
    = \begin{cases}
          0 & \ifspace x=0 \\
          1 & \text{else}
    \end{cases} \, .
\end{align}

Probabilistic entailment with respect to $\probtensor$ is then equivalent to entailment with respect to $\kb^{\probtensor}$, as we show next.

% Generic case of distributions
\begin{theorem}
    \label{the:entailmentProbToLogical}
    Any probability distribution $\probat{\shortcatvariables}$ probabilistically entails a formula $\exformulaat{\shortcatvariables}$, if and only if $\kb^{\probtensor}\models\exformula$.
\end{theorem}
\begin{proof}
    Whenever $\probtensor$ does not entail $\exformula$ probabilistically we find a state $\shortatomindicesin$ such that
    \begin{align*}
        \probat{\indexedshortcatvariables} >0 \quad\text{and} \quad \formulaat{\indexedshortcatvariables} = 0 \, .
    \end{align*}
    We further have $\probat{\indexedshortcatvariables}>0$ if and only if $\kb^{\probtensor}[\indexedshortcatvariables]=1$.
    Therefore the statement
    \begin{align*}
        \imppremhead{\kb^{\probtensor}[\indexedshortcatvariables]=1}{\formulaat{\indexedshortcatvariables}=1}
    \end{align*}
    is not satisfied.
    Together, $\probtensor\models\exformula$ does not holds if and only if
    \begin{align*}
        \uniquantwrtof{\shortatomindicesin}{\imppremhead{\kb^{\probtensor}[\indexedshortcatvariables]=1}{\formulaat{\indexedshortcatvariables}=1}}
    \end{align*}
    is not satisfied.
    Therefore, probabilistic entailment of $\exformula$ by $\probtensor$ is equivalent to logical entailment of $\exformula$ by $\kb^{\probtensor}$.
\end{proof}

Let us use this to connect the entailment formalism with the representability (see \defref{def:representationBaseMeasure}) and positivity (see \defref{def:positivityBaseMeasure}) of distributions with respect to boolean base measures.

\begin{theorem}
    \label{the:minimalRepPosBaseMeasure}
    Let $\probtensor$ be a distribution of boolean variables and let $\basemeasure$ be a boolean base measure.
    Then, $\probtensor$ is representable with respect to $\basemeasure$, if and only if $\nonzerocirc\probtensor\models\basemeasure$.
    Further, $\probtensor$ is positive with respect to $\basemeasure$, if and only if $\basemeasure=\nonzerocirc\probtensor$.
\end{theorem}
\begin{proof}
    To show the first claim, let $\probtensor$ be a distribution and $\basemeasure$ be a base measure.
    With \defref{def:representationBaseMeasure}, $\probtensor$ is representable with respect to $\basemeasure$, if and only if
    \begin{align*}
        \uniquantwrtof{\shortatomindicesin}{\imppremhead{\basemeasureat{\indexedshortcatvariables}=0}{\probat{\indexedshortcatvariables}=0}}
    \end{align*}
    This is equal to
    \begin{align*}
        \uniquantwrtof{\shortatomindicesin}{\imppremhead{\nonzerocirc\probat{\indexedshortcatvariables}=1}{\basemeasureat{\indexedshortcatvariables}=1}}
    \end{align*}
    and by definition \defref{def:logicalEntailment} equal to $\basemeasure\models\nonzerocirc\probtensor$.

    To prove the second claim, we show that when $\probtensor$ is in addition positive with respect to $\basemeasure$, then also $\basemeasure\models\nonzerocirc\probtensor$ and thus $\basemeasure=\nonzerocirc\probtensor$.
    Let $\probtensor$ be a distribution, which is representable with respect to $\basemeasure$.
    Then $\probtensor$ is positive with respect to $\basemeasure$, if and only if
    \begin{align*}
        \uniquantwrtof{\shortatomindicesin}{\imppremhead{\basemeasureat{\indexedshortcatvariables}=1}{\probat{\indexedshortcatvariables}>0}}
    \end{align*}
    This is equal to
    \begin{align*}
        \uniquantwrtof{\shortatomindicesin}{\imppremhead{\basemeasureat{\indexedshortcatvariables}=1}{\nonzerocirc\probat{\indexedshortcatvariables}=1}}
    \end{align*}
    and thus $\basemeasure\models\nonzerocirc\probtensor$.
\end{proof}

%\begin{remark}[Propositional Formulas as Computation Activation Networks]
%    Uniform distributions of the models of a propositional formula $\exformula$ have $\exformula$ as a sufficient statistic.
%    The computation tensor network is any representation of $\formulaccwith$.
%    The activation tensor $\tbasisat{\headvariable}$ represent $\exformula$, $\fbasisat{\headvariable}$ represent $\lnot\exformula$ and $\onesat{\headvariable}$ represents the tautology $\onesat{\shortcatvariables}$.
%\end{remark}

\sect{Entropy Optimization in Logics}

Since normalizations of propositional formulas are probability distributions, we can apply the concepts of probabilistic reasoning.
To develop results on entropy optimization, let us characterize the relative entropy between formulas using the entailment formalism.

\begin{lemma}\label{lem:relEntropyFormulas}
    Given two satisfiable formulas $\exformula,\secexformula$ we have
    \begin{align*}
        \kldivof{\normalizationof{\exformula}{\shortcatvariables}}{\normalizationof{\secexformula}{\shortcatvariables}}
        = \begin{cases}
              \lnof{\contraction{\secexformula}} - \lnof{\contraction{\exformula}} & \ifspace \exformula\models\secexformula \\
              \infty & \ifspace \exformula\not\models\secexformula
        \end{cases} \, .
    \end{align*}
\end{lemma}
\begin{proof}
    The relative entropy (see \defref{def:crossEntropy}) decomposed as
    \begin{align*}
        \kldivof{\normalizationof{\exformula}{\shortcatvariables}}{\normalizationof{\secexformula}{\shortcatvariables}}
        = \centropyof{\normalizationof{\exformula}{\shortcatvariables}}{\normalizationof{\secexformula}{\shortcatvariables}}
        - \sentropyof{\normalizationof{\exformula}{\shortcatvariables}}
    \end{align*}

    The entropy term is
    \begin{align*}
        \sentropyof{\normalizationof{\exformula}{\shortcatvariables}}
        &= \contraction{\frac{\exformulaat{\shortcatvariables}}{\contraction{\exformula}},-\lnof{\frac{\exformulaat{\shortcatvariables}}{\contraction{\exformula}}}} \\
        &= \contraction{\frac{\exformulaat{\shortcatvariables}}{\contraction{\exformula}},-\lnof{\exformulaat{\shortcatvariables}}}
        + \lnof{\contraction{\exformula}} \cdot \contraction{\frac{\exformulaat{\shortcatvariables}}{\contraction{\exformula}}}  \\
        &= \lnof{\contraction{\exformula}} \, .
    \end{align*}
    In the last equation we use the convention $0\cdot\lnof{0}=0$.

    The cross entropy term is
    \begin{align*}
        \centropyof{\normalizationof{\exformula}{\shortcatvariables}}{\normalizationof{\secexformula}{\shortcatvariables}}
        =\contraction{\frac{\exformulaat{\shortcatvariables}}{\contraction{\exformula}},-\lnof{\frac{\secexformulaat{\shortcatvariables}}{\contraction{\secexformula}}}}
    \end{align*}
    If and only if $\exformula\not\models\secexformula$ there is a state $\shortcatindicesin$ such that $\secexformulaat{\indexedshortcatvariables}=0$ and $\exformulaat{\indexedshortcatvariables}$, which contributes to the contraction the summand
    \begin{align*}
        (-1) \cdot \exformulaat{\indexedshortcatvariables} \cdot \lnof{\secexformulaat{\indexedshortcatvariables}} = \infty \, .
    \end{align*}
    If and only if $\exformula\not\models\secexformula$ we therefore have
    \begin{align*}
        \kldivof{\normalizationof{\exformula}{\shortcatvariables}}{\normalizationof{\secexformula}{\shortcatvariables}} = \infty \, .
    \end{align*}

    In the case $\exformula\models\secexformula$ we get 
    \begin{align*}
        \contraction{\frac{\exformulaat{\shortcatvariables}}{\contraction{\exformula}},-\lnof{\frac{\secexformulaat{\shortcatvariables}}{\contraction{\secexformula}}}}
        &= \contraction{\frac{\exformulaat{\shortcatvariables}}{\contraction{\exformula}},-\lnof{\secexformulaat{\shortcatvariables}}}
        +\lnof{\contraction{\secexformula}} \cdot \contraction{\frac{\exformulaat{\shortcatvariables}}{\contraction{\exformula}}}  \\
        &= \lnof{\contraction{\secexformula}} 
    \end{align*}
    and with the above 
    \begin{align*}
        \kldivof{\normalizationof{\exformula}{\shortcatvariables}}{\normalizationof{\secexformula}{\shortcatvariables}}
        = \lnof{\contraction{\secexformula}} - \lnof{\contraction{\exformula}} \, . & \qedhere
    \end{align*}
\end{proof}

We use \lemref{lem:relEntropyFormulas} to characterize the logical analogue of the M-projection and the I-projection.

\begin{theorem}
    Given a set of formulas $\greedyhypothesis$ we get for the logical M-projection
    \begin{align*}
        \min_{\exformula\in\greedyhypothesis} \kldivof{\normalizationof{\secexformula}{\shortcatvariables}}{\normalizationof{\exformula}{\shortcatvariables}}
        = \min_{\exformula\in\greedyhypothesis\wcols\secexformula\models\exformula} \lnof{\contraction{\exformula}} - \lnof{\contraction{\secexformula}} 
    \end{align*}
    and for the logical I-projection
    \begin{align*}
        \min_{\exformula\in\greedyhypothesis} \kldivof{\normalizationof{\exformula}{\shortcatvariables}}{\normalizationof{\secexformula}{\shortcatvariables}}
        = \min_{\exformula\in\greedyhypothesis\wcols\exformula\models\secexformula} \lnof{\contraction{\secexformula}} - \lnof{\contraction{\exformula}} \, .
    \end{align*}
\end{theorem}
\begin{proof}
    We use the characterization of the relative entropy from \lemref{lem:relEntropyFormulas} and restrict the feasible set of the minimizations to the respective cases, where the cost is finite.
    Note that if $\secexformula\models\exformula$ (respectively $\exformula\models\secexformula$) is not satisfied for $\exformula\in\greedyhypothesis$, then the feasible sets of the right sight are empty and we use the convention that the minimum is $\infty$.
\end{proof}

%% TO DO: DRAW THE VENN DIAGRAM INTERPRETING THE M- and I-projection


\sect{Constraint Satisfaction Problems}

Let us now explore a more general class of logical inference problems and discuss probabilistic entailment within that class.
We then provide further examples based on categorical constraints.
Following Chapter~5 in \cite{russell_artificial_2021}, we now define Constraint Satisfaction Problems.

\begin{definition}\label{def:csp}
    Let there be a hypergraph $\graph=(\nodes,\edges)$ and $\extnet$ be a tensor network of boolean constraint tensors $\hypercoreofat{\edge}{\catvariableof{\edge}}$ to each $\edgein$, that is
    \begin{align*}
        \extnet = \{ \hypercoreofat{\edge}{\catvariableof{\edge}} \wcols \edge\in\edges \} \, .
    \end{align*}
    The Constraint Satisfaction Problem (CSP) to $\extnet$ is the decision whether there is a state $\catindexof{\nodes}$ such that
    \begin{align*}
        \contractionof{\extnet}{\indexedcatvariableof{\nodes}} = 1 \, .
    \end{align*}
    We say the CSP is satisfiable, when there is such a state, and unsatisfiable if not.
\end{definition}

\subsect{Deciding Entailment on Markov Networks}

Deciding entailment on Markov Networks is a general class of contraint satisfaction problems.
Here, any factor tensor in the Markov Networks produces a constraint tensor in the respective CSP.

\begin{theorem}
    \label{the:factorReduction}
    Let $\probof{\graph}$ be a Markov Network to the Tensor Network $\extnet=\extnetasset$ on a hypergraph $\graph=(\nodes,\edges)$. % $\secnodes\subset\nodes$ be a subset and
%	\begin{align*}
%		\probat{\catvariableof{\secnodes}} = \normalizationof{\{\hypercoreat{\edge} \wcols \edge\in\edges \}}{\catvariableof{\secnodes}}
%	\end{align*}
    For each $\edge\in\edges$ we build the factor constraint cores
    \begin{align*}
        \sechypercoreofat{\edge}{\catvariableof{\edge}} = \nonzerocirc\hypercoreofat{\edge}{\catvariableof{\edge}} \, .
    \end{align*}
    Let further $\exformulaat{\catvariableof{\secnodes}}$ be a formula depending on the variables $\secnodes$, and build $\secgraph=(\nodes,\edges\cup\{\secnodes\})$.
    Then we have that $\probof{\graph}\models\exformula$ if and only if the constraint satisfaction problem of $\secgraph$ to the constraint tensors
    \begin{align*}
        \{\sechypercoreof{\edge} \wcols \edge\in\edges \} \cup \{\lnot\exformula \}
    \end{align*}
    is unsatisfiable.
    %and
    %	\[ \tilde{\probtensor}[\catvariableof{\secnodes}] = \normalizationof{\{\nonzerofunction \circ \hypercoreat{\edge} \wcols \edge\in\edges \}}{\catvariableof{\secnodes}} \]
    %Then we have for any $\exformula$ that $\probtensor\models\exformula$ if and only if $\tilde{\probtensor}\models\exformula$.
\end{theorem}
\begin{proof}
    We first show, that
    \begin{align*}
        \nonzerocirc\probofat{\graph}{\nodevariables} =
        \contractionof{\{\sechypercoreof{\edge} \wcols \edge\in\edges \}}{\nodevariables} \, . %\nonzerocirc\tilde{\probtensor} \, .
    \end{align*}
    To this end, let $\catindexof{\nodes}\in\nodestatesof{\nodes}$ be arbitrary.
    We have $\probofat{\graph}{\indexednodevariables}=0$ if and only if at there is an edge $\edge\in\edges$ with $\hypercoreofat{\edge}{\indexedcatvariableof{\edge}}$.
    But this is equivalent to
    \begin{align*}
        \contractionof{\{\sechypercoreof{\edge} \wcols \edge\in\edges \}}{\nodevariables} \, .
    \end{align*}
    We thus have for any $\catindexof{\nodes}\in\nodestatesof{\nodes}$
    \begin{align*}
        \nonzerocirc\probofat{\graph}{\indexednodevariables} =
        \contractionof{\{\sechypercoreof{\edge} \wcols \edge\in\edges \}}{\indexednodevariables} \, . %\nonzerocirc\tilde{\probtensor} \, .
    \end{align*}

    To continue, we have $\probof{\graph}\models\exformula$ if and only if
    \begin{align*}
        \contraction{\extnetat{\shortcatvariables},\lnot\exformulaat{\shortcatvariables}} = 0
    \end{align*}
    which is equal to
    \begin{align*}
        \contraction{\nonzerocirc\extnetat{\shortcatvariables},\lnot\exformulaat{\shortcatvariables}} = 0 \, .
    \end{align*}
    We notice that this is the unsatisfiability of the claimed Constraint Satisfaction Problem.

%	We first show
%	\begin{align}\label{eq:proofFacReduction}
%		 \nonzerocirc\probtensor = \nonzerocirc\tilde{\probtensor} \, .
%	\end{align}
%	The claim follows then from \theref{the:entailmentProbToLogical}.
%	To show \eqref{eq:proofFacReduction} let there be $\indexedcatvariableof{\secnodes}$ such that $\probtensor[\indexedcatvariableof{\secnodes}]=0$.
%	Then for any $\indexedcatvariableof{\nodes}$ extending  $\indexedcatvariableof{\secnodes}$ we have $\contractionof{\{\hypercoreat{\edge} \wcols \edge\in\edges \}}{\indexedcatvariableof{\nodes}} = 0$ and thus also $\contractionof{\{\nonzerocirc\hypercoreat{\edge} \wcols \edge\in\edges \}}{\indexedcatvariableof{\nodes}} = 0$ and $\tilde{\probtensor}[\indexedcatvariableof{\secnodes}]=0$.
%	One can similarly show, that when $\tilde{\probtensor}[\indexedcatvariableof{\secnodes}]=0$ then also ${\probtensor}[\indexedcatvariableof{\secnodes}]=0$.
%	The support of the distributions $\probtensor$ and $\tilde{\probtensor}$ is thus identical and \eqref{eq:proofFacReduction} holds.
\end{proof}

% Consequence: Reduction of probabilitic entailment to logical entailment.
For any positive tensor $\hypercore$ we have
\[ \nonzerocirc\hypercoreat{\catvariableof{\edge}} = \onesat{\catvariableof{\edge}} \, , \]
which does not influence the distribution and can be omitted from the Markov Network.
By \theref{the:factorReduction}, when deciding entailment, we can reduce all tensors of a Markov Network to their support and omit those with full support.
Since the support indicating tensors $\nonzerocirc\hypercoreat{\catvariableof{\edge}}$ are boolean, each is a propositional formula and the Markov Network is turned into a Knowledge Base of their conjunctions.
Deciding probabilistic entailment is thus traced back to logical entailment.

%Exponential families
Exponential families have a tensor network representation by a Markov Network (see \theref{the:expFamilyTensorRep}).
However, all factors corresponding with a coordinate of the statistic $\sstat$ have a trivial support, and therefore do not influence the support of the distribution.
The only tensors with non-trivial support are those to the boolean base measure $\basemeasure$.


\subsect{Categorical Constraints}\label{sec:categoricalTN}

%% Categorical variables with more possibilities
We so far in this chapter made the assumption that all categorical variables in factored systems to be represented by propositional logics take binary values (i.e. $\catdim=2$).
In cases where a categorical variable $\catvariable$ takes multiple values we define for each $\catindex$ an atomic formula $\catvariableof{\catindex}$ representing whether $\catvariable$ is assigned by $\catindex$ in a specific state.
%\[ \catvariableof{\catindex} =  (\catvariable = \catindex \, . \] Confusing notation
Following this construction we have the constraint that exactly one of the atoms $\catvariableof{\catindex}$ is $1$ at each state.

%% Capture constraint
%To capture the constraints resulting from this construction we introduce auxiliary parts. % of Bayesian Propositional Networks.
%Such constraints can also be expressed by a formula but would result in an unnecessary large tensor network.


%% Categorical selection map
\begin{definition}[Categorical Constraint and Atomization Variables]\label{def:catConAtomVar}
    Given a list $\catvariableof{0},\ldots,\catvariableof{\catdim-1}$ of boolean variables and a categorical variable $\catvariable$ with dimension $\catdim$ a categorical constraint is a tensor $\categoricalcoreat{\catvariableof{[\catdim]},\catvariable}$ with coordinates
    \begin{align*}
        \categoricalcoreat{\indexedcatvariableof{[\catdim]},\indexedcatvariable}
%		 \categoricalmap(\catindex,\catindexof{\variableset})
        = \begin{cases}
              1 & \ifspace \shortcatindices = \onehotmapof{\catindex} \quad \Big(\text{i.e. }\forall \catenumerator\in[\catdim]: \big(\catindex=\catenumerator\big) \Leftrightarrow \big(\catindexof{\catenumerator}=1\big) \Big)\\
              0 & \text{else} \, .
        \end{cases}
    \end{align*}
    We then call the variables  $\catvariableof{0},\ldots,\catvariableof{\catdim-1}$ the atomization variables to the categorical variable $\catvariable$.
\end{definition}

%% Decomposition
%We notice that the categorical constraint tensor is the basis encoding of the map $\categoricalmap \defcols [\atomorder] \rightarrow \bigtimes_{\atomenumeratorin} [2]$ defined as
%\begin{align*}
%    \categoricalmapat{\atomenumerator} = \onehotmapof{\atomenumerator} \, .
%\end{align*}
We notice that the categorical constraint tensor is the basis encoding of the one-hot map on $[\catdim]$.
With \theref{the:functionDecompositionBasisCP} the basis encoding $\categoricalcore$ decomposes in a basis $\cpformat$ format (see \figref{fig:CategoricalDecomposition}b) of if its coordinate maps $\categoricalmapof{\catenumerator}$, where $\catenumerator\in[\catdim]$, defined as
\begin{align*}
    \categoricalmapofat{\atomenumerator}{\secatomenumerator}
    = \begin{cases}
          1 & \ifspace \seccatenumerator=\atomenumerator \\
          0 & \text{else} \, .
      \end{cases}
\end{align*}
%
%\begin{align*}
%    \categoricalmapofat{\catenumerator}{\indexedcatvariableof{\catenumerator},\indexedcatvariable}
%    = \begin{cases}
%          1 & \ifspace \catindex=\catenumerator \andspace \catindexof{\catenumerator} = 1 \\
%          1 & \ifspace \catindex\neq\catenumerator \andspace \catindexof{\catenumerator} = 0 \\
%          0 & \text{else} \, .
%    \end{cases}
%\end{align*}
Their basis encoding are decomposed as
\begin{align}
    \categoricalcoreofat{\catenumerator}{\catvariableof{\catenumerator},\catvariable}
    = \tbasisat{\catvariableof{\catenumerator}} \otimes \onehotmapofat{\catenumerator}{\catvariable}
    + \onehotmapofat{0}{\catvariableof{\catenumerator}} \otimes \big(\onesat{\catvariable}- \onehotmapofat{\catenumerator}{\catvariable}\big) \, .
\end{align}
We thus have by \theref{the:functionDecompositionBasisCP}
\begin{align*}
    \bencodingofat{\categoricalmap}{\catvariableof{[\catdim]},\catvariable}
    = \contractionof{\left\{\categoricalcoreofat{\catenumerator}{\catvariableof{\catenumerator},\catvariable} \wcols \catindex\in[\catdim]\right\}}{\catvariable, \catvariableof{0}, \ldots, \catvariableof{\catdim-1}} \, .
\end{align*}


In the next theorem we show how a categorical constraint can be enforced in a tensor network by adding the tensor $\categoricalmap$ to a contraction.

\begin{theorem}
    For any tensor $\hypercoreat{\shortcatvariables}$ and a categorical constraint defined by an ordered subset $\catvariableof{\variableset}\subset\shortcatvariables$, a variable $\catvariable\in\shortcatvariables$ we have
    \begin{align*}
        \contractionof{\hypercoreat{\shortcatvariables},\categoricalcoreat{\catvariableof{\variableset},\catvariable}}{\indexedcatvariables}
        = \begin{cases}
              \hypercoreat{\indexedcatvariables} & \ifspace \exists \catindex : \catindexof{\variableset} = \onehotmapof{\catindex} \\
              0 & \text{else} \, .
        \end{cases}
    \end{align*}
    Here by $\catindexof{\variableset}$ we denote the restriction of $\shortcatindices$ on the set $\variableset$.
\end{theorem}
\begin{proof}
    For any $\catindexof{[\atomorder]}$ we have
    \[ \contractionof{\{\hypercoreat{\shortcatvariables},\categoricalcoreat{\catvariableof{\variableset},\catvariable}\}}{\indexedcatvariables}  =
    \hypercoreat{\shortcatindices} \cdot \categoricalcoreat{\indexedcatvariableof{\variableset},\indexedcatvariable} \, .
    \]
    If $\catindexof{\variableset} = \onehotmapof{\catindex}$ we have $\categoricalmap[\indexedcatvariableof{\variableset},\indexedcatvariable] = 1$ and thus
    \[ \contractionof{\hypercoreat{\shortcatvariables},\categoricalcoreat{\catvariableof{\variableset},\catvariable}}{\indexedcatvariables}  =  \hypercoreat{\shortcatindices}  \, . \]
    If this is not the case then $\categoricalcoreat{\indexedcatvariableof{\variableset},\indexedcatvariable} = 0$ and
    \begin{align*}
        \contractionof{\hypercoreat{\shortcatvariables},\categoricalcoreat{\catvariableof{\variableset},\catvariable}}{\indexedcatvariables}  = 0 \, . & \qedhere
    \end{align*}
\end{proof}

\begin{figure}[t]
    \begin{center}
        \begin{tikzpicture}[scale=0.35, thick] % , baseline = -3.5pt

\begin{scope}[shift={(-15,2)}]

\node[anchor=center] (text) at (-1,3) {${a)}$};


\node [circle, draw, thick, fill=gray!50] (T1) at (0,0) {\tiny $\randomxof{0}$};	
\node [circle, draw, thick, fill=gray!50] (T2) at (3,0) {\tiny $\randomxof{1}$};	
\node[anchor=center] (text) at (6,0) {\small $\cdots$};
\node [circle, draw, thick, fill=gray!50] (T3) at (9,0) {\tiny $\randomxof{\atomorder-1}$};	

\node [circle, draw, thick, fill=gray!50] (C) at (4.5,-5) {\tiny $\randomxof{\categoricalmap}$};	

\draw[->] (C) -- (T1);
\draw[->] (C) -- (T2);
\draw[->] (C) -- (T3);

\end{scope}

\node[anchor=center] (text) at (-1,5) {${b)}$};


\drawatomindices{0}{2}
\draw (-1,1) rectangle (5,-1);
\node[anchor=center] (text) at (2,0) {\small $\categoricalcore$};
\draw[->] (2,-3) -- (2,-1) node[midway,left] {\tiny $\randomxof{\categoricalmap}$};

\node[anchor=center] (text) at (7,0) {${=}$};


\begin{scope}[shift={(10,2)}]

\newcommand{\conposseldec}{4.5,-5.5}

\draw[fill] (\conposseldec) circle (0.25cm);
\draw (\conposseldec) -- (4.5,-7.5) node[midway, right] {\tiny $\randomxof{\categoricalmap}$};
%!TEX encoding = UTF-8 Unicode\draw[dashed] (3.5,-7.5) rectangle (5.5, -9.5);
%\node[anchor=center] (text) at (4.5,-8.5) {\small $\ones$};

\draw[<-] (0,1) -- (0,-1) node[midway,left] {\tiny $\randomxof{0}$};
\draw (-1,-1) rectangle (1, -3);
\node[anchor=center] (text) at (0,-2) {\small $\categoricalcoreof{0}$};
\draw[<-] (0,-3) to[bend right=20] (\conposseldec);


\draw[<-] (3,1) -- (3,-1) node[midway,left] {\tiny $\randomxof{1}$};
\draw (2,-1) rectangle (4, -3);
\node[anchor=center] (text) at (3,-2) {\small $\categoricalcoreof{1}$};
\draw[<-] (3,-3) to[bend right=20]  (\conposseldec);

\node[anchor=center] (text) at (6,-2) {$\cdots$};

\draw[<-] (9,1) -- (9,-1) node[midway,left] {\tiny $\randomxof{\atomorder-1}$};
\draw (7.75,-1) rectangle (10.25, -3);
\node[anchor=center] (text) at (9,-2) {\small $\categoricalcoreof{\atomorder-1}$};
\draw[<-] (9,-3) to[bend left=20]  (\conposseldec);




\end{scope}

		


\end{tikzpicture}
    \end{center}
    \caption{Representation of a categorical constraint in a $\cpformat$ Format tensor network.
    a) Representation of the dependency of the graphical model.
    b) Tensor Representation with further network decomposition.
    }
    \label{fig:CategoricalDecomposition}
\end{figure}

% Drop?
\begin{remark}[Constraint Satisfaction Problems of Categorical Constraints]
    We can define CSPs by collection of categorical constraints.
    An example, where the corresponding Constraint Satisfaction Problem is unsatisfiable are the categorical constraints to the three sets
    \[ \{\catvariableof{0},\catvariableof{1},\catvariableof{2},\catvariableof{3}\} \, , \, \{\catvariableof{0},\catvariableof{1}\}\, ,\,\{\catvariableof{2},\catvariableof{3}\} \, . \]
%	Besides the categorical cores also the datacores have a similar bayesian network affecting the atoms by another hidden variable.
%	Combining both is well-defined, only when all datapoints satisfy the categorical constraints (that is only one of the atoms in each constraint is active).
\end{remark}


\begin{example}[Sudoku]
    \label{exa:sudoku}
    An interesting example, where categorical constraints are combined is Sudoku, the game of assigning numbers to a grid (see for example Section~5.2.6 in \cite{russell_artificial_2021}).
    The basic variables therein are $\catvariableof{i,j}$, with $\catdimof{i,j}=n^2$ and $i,j\in[n^2]$.
    By understanding $i$ as a line index and $j$ as a column index, they are ordered in a grid as sketched in \figref{fig:sudokuGrid} in the case $n=3$.

    For a $n\in\nn$ we further define the atomization variables $\catvariableof{i,j,k}$ where $i,j,k\in[n^2]$ and $\catdimof{i,j,k}=2$.
    These $n^6$ variables are the booleans indicating whether a specific position has a specific number assigned.
    The consistency of the atomization variables to the basic variables is then for each $i,j\in[n^2]$ ensured by the categorical constraints on the sets
    \[ \{\catvariableof{i,j,k} \wcols k\in [n^2] \} \, . \]

    We further have $3\cdot n^2$ constraints by the
    \begin{itemize}
        \item Row constraints: Each number $k$ appears exactly once in each row $i\in[n^2]$, captured by the constraints
        \[ \{\catvariableof{i,j,k}  \wcols j \in [n^2] \} \, . \]
        \item Column constraints: Each number $k$ appears exactly once in each column $j\in[n^2]$, captured by the constraints
        \[ \{\catvariableof{i,j,k}  \wcols i \in [n^2] \} \, . \]
        \item Square constraints: Each number appears exactly once in each square $s,r\in[n]$, captured by the constraints
        \[ \{\catvariableof{i+n\cdot s,j+n\cdot r,k}  \wcols i,j \in [n] \} \, . \]
    \end{itemize}

    In total we have $3\cdot n^2 + n^4$ constraints for $n^6$ variables.

    Deciding whether a Sudoku has a solution is a Constraint Satisfaction Problem \cite{simonis_sudoku_2005}, which is NP-hard \cite{agerbeck_multi-agent_2008}.
    Let us notice, that due to this large number of variables and constraints, direct solution of the problem by a global contraction is not feasible.
    For efficient algorithmic solutions, we instead refer to \secref{subsec:LocalEntailment}.

    \begin{figure}
        \begin{center}
            \begin{tikzpicture}[scale=0.9]
% Draw the main grid

%\node[anchor=center] (text) at (0,9) {${a)}$};

\draw[very thick] (0,0) rectangle (9,9); % Outer border
\foreach \x in {1,2,...,8} {
    \draw[thin] (\x,0) -- (\x,9); % Vertical lines
    \draw[thin] (0,\x) -- (9,\x); % Horizontal lines
}
% Thicker lines for 3x3 subgrids
\foreach \x in {3,6} {
    \draw[very thick] (\x,0) -- (\x,9); % Vertical thick lines
    \draw[very thick] (0,\x) -- (9,\x); % Horizontal thick lines
}

% Add variables in the middle of each square
\foreach \i in {0,1,...,8} {
    \foreach \j in {0,1,...,8} {
        \node[circle, draw, thick, fill=gray!50, inner sep = 0.5pt, minimum size=0.6cm, align=center] 
        at (\j+0.5,8-\i+0.5) {$X_{\i,\j}$};
    }
}

%\begin{scope}[shift={(2,0)}]
%
%\node[anchor=center] (text) at (9,9) {${b)}$};
%	
%% Draw a line of variables (horizontal)
%\foreach \k in {0,1,...,8} {
%    \node[circle, draw, thick,  fill=gray!50, inner sep=0pt, 
%    minimum size=0.6cm, align=center] 
%    at (10+\k,8.5) {$X_{i,\k}$};
%}
%
%\node[anchor=center] (text) at (9,7) {${c)}$};
%
%% Draw a column of variables (vertical)
%\foreach \l in {0,1,...,8} {
%    \node[circle, draw, thick, fill=gray!50, inner sep=0pt, 
%    minimum size=0.6cm, align=center] 
%    at (10,-\l+6) {$X_{\l,j}$};
%}
%
%\node[anchor=center] (text) at (9,7) {${d)}$};
%
%% Draw a square of variables (3x3)
%\foreach \i in {0,1,2} {
%    \foreach \j in {0,1,2} {
%        \node[circle, draw, thick, fill=gray!50, inner sep=0pt, 
%        minimum size=0.6cm, align=center] 
%        at (12+\j,6-\i) {$X_{\i,\j}$};
%    }
%}
%
%\end{scope}

\end{tikzpicture}

        \end{center}
        \caption{
            Sudoku grid of basic categorical variables $\catvariableof{i,j}$, here drawn in the standard case of $n=3$, each with dimension $\catdim=n^2=9$.
            Each basic categorical variables has $n^2$ corresponding atomization variables, which are further atomization variables to the row, column and squares constraints.
            Instead of depicting those constraints by hyperedges in a variable dependency graph, we here just indicate their existence through row, column and squares blocks.
        }\label{fig:sudokuGrid}
    \end{figure}
\end{example}




\sect{Deciding Entailment by Local Contractions}\label{subsec:LocalEntailment}

When having a Constraint Satisfaction Problem on a large number of variables, which are densely connected by constraint tensors, direct exploitation of the global entailment criterion in \theref{the:contCriterionLogEntailment} will be infeasible.
An alternative to deciding entailment by global operations is the use of local operations.
Here we interpret a part of the network (for example a single core) as an own knowledge base (with atomic formulas being the roots of the directed subgraph, that is potentially differing with the atoms in the global perspective) and perform entailment with respect to that.

\subsect{Monotonicity of Entailment}

Vanishing local contractions provide sufficient but not necessary criterion to decide entailment, as we show in the next theorem.

\begin{theorem}[Monotonicity of Entailment]
    \label{the:monotonEntailment}
    For any Markov Network $\probtensor^{\graph}$ on the decorated hypergraph $\graph$ and any subgraph $\secgraph$, we have for any formula that $\probtensor^{\graph}\models\exformula$ if $\probtensor^{\secgraph}\models\exformula$.
\end{theorem}

%To show this theorem, we show the following lemma, that whether a contraction of non-negative tensors vanishes, a vanishing contraction of a subset of these tensors is a sufficient criterion.
To prove the theorem, we first establish the following lemma that states if a contraction of non-negative tensors vanishes, the vanishing of a contraction over a subset of these tensors is a sufficient criterion.

\begin{lemma}
    \label{lem:monotocityOfVanishingContractions}
    For any non-negative tensor network $\extnet$ on $\graph$ and $\secedges\subset\edges$ we have the following.
    For $\secgraph=(\secnodes,\secedges)$ with $\secnodes=\cup_{\edge\in\secedges}\edge$ and the tensor network $\tnetof{\secgraph}$ with tensors coinciding on $\secedges$ with those in $\extnet$ we have
    \begin{align*}
        \contraction{\extnet} = 0
    \end{align*}
    if $\contraction{\tnetof{\secgraph}}=0$.
\end{lemma}
\begin{proof}
    Since the tensor network $\tnetof{\secgraph}$ is non-negative, we have whenever $\contraction{\tnetof{\secgraph}}=0$ that
    \begin{align*}
        \contractionof{\tnetof{\secgraph}}{\catvariableof{\secnodes}} = \zerosat{\catvariableof{\secnodes}} \, .
    \end{align*}
    It follows with the commutation of contractions (see \theref{the:splittingContractions} in \charef{cha:messagePassing}), that
    \begin{align*}
        \contractionof{\extnet}{\catvariableof{\nodes}}
        &= \contractionof{
            \{\hypercoreof{\edge} \wcols \edge\in\edges/\secedges\}
            \cup \{\contractionof{\tnetof{\secgraph}}{\catvariableof{\secnodes}}\}
        }{\catvariableof{\nodes}} \\
        &=    \contractionof{
            \{\hypercoreof{\edge} \wcols \edge\in\edges/\secedges\}
            \cup \{\zerosat{\catvariableof{\secnodes}}\}
        }{\catvariableof{\nodes}} \\
        &= 0
    \end{align*}
    Thus, also the contraction of $\extnet$ vanishes in this case.
\end{proof}

\begin{proof}[Proof of \theref{the:monotonEntailment}]
    We use \lemref{lem:monotocityOfVanishingContractions} on the subset $\tnetof{\secgraph}$ of the cores $\extnet$ to the Markov Network $\probof{\graph}$, which itself defines the Markov Network $\probof{\secgraph}$.
    Whenever $\probof{\secgraph}\models\exformula$ for a formula $\exformula$, then we have by \theref{the:contCriterionLogEntailment}
    \begin{align*}
        \contraction{\tnetof{\secgraph} \cup\{\lnot\exformula\}} = 0 \, .
    \end{align*}
    It follows with \lemref{lem:monotocityOfVanishingContractions} that also
    \begin{align*}
        \contraction{\tnetof{\graph} \cup\{\lnot\exformula\}} = 0 \, .
    \end{align*}
    and therefore $\probof{\graph}\models\exformula$.
\end{proof}


\begin{remark}
    To make use of \theref{the:monotonEntailment} we can exploit any entailment criterion.
    However, there is no general statement about entailment possible, when the local entailment does not hold.
    \theref{the:monotonEntailment} therefore just provides a sufficient but not necessary criterion of entailment with respect to $\probtensor^{\graph}$.
\end{remark}

\subsect{Knowledge Cores}

To store preliminary conclusions, we define auxiliary knowledge cores storing constraints on variables $\edge\subset\nodes$.
They are understood as logical formulas to the atomization variables $\left(\catvariableof{\edge}=\catindexof{\edge}\right)$ of the respective formulas
\begin{align*}
    \kcoreofat{\edge}{\catvariableof{\edge}}
    = \bigvee_{\catindexof{\edge} \wcols \kcoreofat{\edge}{\indexedcatvariableof{\edge}}} \bigwedge_{\node\in\edge} \left(\catvariableof{\edge}=\catindexof{\edge}\right) \, .
\end{align*}

\begin{definition}\label{def:knowledgeCoreSoundComplete}
    Let $\extnet$ be a constraint satisfaction problem. % and $$ be a knowledge core. % \left(\catvariableof{\edge}=\catindexof{\edge}\right)
    We say that a knowledge core $\kcoreofat{\edge}{\catvariableof{\edge}}$ is sound for $\extnet$, if
    \begin{align*}
        \nonzerocirc\contractionof{\extnet}{\catvariableof{\edge}}  \prec \kcoreofat{\edge}{\catvariableof{\edge}}
    \end{align*}
    and complete for $\extnet$ if in addition
    \begin{align*}
        \nonzerocirc\contractionof{\extnet}{\catvariableof{\edge}}=\kcoreofat{\edge}{\catvariableof{\edge}} \, .
    \end{align*}
\end{definition}


\subsect{Knowledge Propagation}

We now provide a solution algorithm for constraint satisfaction problems by propagating local contractions.
The dynamic programming paradigm is implemented by the storage of partial entailment results in Knowledge Cores.
We then iterate over local entailment checks, where we recursively add further entailment checks to be redone due to additional knowledge.
This local entailment scheme is called Knowledge Propagation and described in a generic way in \algoref{alg:knowledgePropagation}.

\begin{algorithm}[hbt!]
    \caption{Knowledge Propagation}\label{alg:knowledgePropagation}
    \begin{algorithmic}
        \Require Boolean Tensor Network $\extnet$ on $\graph=(\nodes,\edges)$, domain edges $\domainedges$ and a set $\arbset$ of subsets of $\edges$ for local propagation
        \Ensure Knowledge cores $\kcoreofat{\edge}{\catvariableof{\edge}}$ for $\edge\in\domainedges$ with $\contractionof{\extnet}{\catvariableof{\edge}}\prec\kcoreofat{\edge}{\catvariableof{\edge}}$
        \iosepline
        \State
        \State Initialize for all $\edge\in\domainedges$:
        \begin{align*}
            \kcoreofat{\edge}{\catvariableof{\edge}}=\onesat{\catvariableof{\edge}}
        \end{align*}
        \State Initialize a queue
        \begin{align*}
            \graphqueue = \arbset
        \end{align*}
        \While{$\graphqueue$ is not empty}
            \State Choose a set of edges from the queue
            \begin{align*}
                \secedges \algdefsymbol \graphqueue\mathrm{.pop()}
            \end{align*}
            \ForAll{$\edge\in\domainedges$ with $\edge\cap\bigcup_{\secedge\in\secedges}\secedge\neq\varnothing$}
                \State Contract
                \begin{align*}
                    \hypercoreat{\catvariableof{\edge}}
                    = \nonzerocirc\contractionof{
                        \{\hypercoreofat{\secedge}{\catvariableof{\secedge}}\wcols\secedge\in\secedges\}
                        \cup \{\kcoreofat{\edge}{\catvariableof{\edge}}\wcols\edge \in \domainedges , \, \edge\cap\bigcup_{\secedge\in\secedges}\secedge\neq\varnothing \}
                    }{\catvariableof{\edge}}
                \end{align*}
                \If{$\hypercoreat{\catvariableof{\edge}}\neq\kcoreofat{\edge}{\catvariableof{\edge}}$}
                    \begin{align*}
                        \kcoreofat{\edge}{\catvariableof{\edge}} \algdefsymbol \hypercoreat{\catvariableof{\edge}}
                    \end{align*}
                    \ForAll{$\secedges\in\arbset$ with $\edge\cap\bigcup_{\secedge\in\secedges}\secedge\neq\varnothing$}
                        \begin{align*}
                            \graphqueue\mathrm{.push(}\secedges\mathrm{)}
                        \end{align*}
                    \EndFor
                \EndIf
            \EndFor
        \EndWhile
        \State \Return $\{\kcoreofat{\edge}{\catvariableof{\edge}}\,:\,\edge\in\domainedges\}$
    \end{algorithmic}
\end{algorithm}

% Interpretation
Each chosen subset $\secedges\in\arbset$ is understood as a local knowledge base, which is then applied for local entailment.
The knowledge cores are understood as messages, which propagate information from different regions of a tensor network (see \charef{cha:messagePassing}).

%Implementation
There are different ways of implementing \algoref{alg:knowledgePropagation}, by choosing the set $\arbset$ of constraint sets $\secedges$ and domain $\domainedges$. % and a stopping criterion.
The AC-3 algorithm (see \cite{mackworth_consistency_1977}) is a specific instance, where knowledge cores are assigned to single variables and propagation is performed on single constraint cores.

%The central property used in knowledge propagation is that any subcontraction can be added to the constraint network without changing it.
%\begin{theorem}\label{the:booleanContractionInvariance}
%	Given a boolean tensor network $\extnet$ on $\graph$ and $\secedges\subset\edges$.
%	For $\secgraph=(\secnodes,\secedges)$ with $\nodes=\cup_{\edge\in\secedges}\edge$ and the tensor network $\tnetof{\secgraph}$ with boolean tensors coinciding on $\secedges$ with those in $\extnet$ we have
%	\begin{align*}
%		\contractionof{\extnet}{\catvariableof{\nodes}} =
%		\contractionof{\extnet\cup\{\nonzerocirc\contractionof{\tnetof{\secgraph}}{\thirdnodes}\}}{\catvariableof{\nodes}} \, ,
%	\end{align*}
%	where $\thirdnodes\subset\nodes$ is arbitrary.
%\end{theorem}
%\begin{proof}
%	We will proof this statement later in \charef{cha:messagePassing}, see \theref{the:invarianceAddingSubcontractions}.
%\end{proof}

\begin{theorem}
    \label{the:soundnessKnowledgePropagation}
    At any state of the Knowledge Propagation \algoref{alg:knowledgePropagation}, we have that each knowledge core $\kcoreof{\secedge}$ is sound for $\extnet$.
    %\begin{align*}
    %    \contractionof{\extnet}{\nodevariables}
    %    = \contractionof{\extnet \cup \{\kcoreof{\edge} \wcols \edge\in\domainedges\}}{\nodevariables} \, .
    %\end{align*}
    After each update in \algoref{alg:knowledgePropagation}, $\kcoreof{\secedge}$ is further monotonically decreasing with respect to the partial ordering.
    %\begin{align*}
    %    \nonzerocirc\contractionof{\extnet}{\catvariableof{\secedge}} \prec \kcoreofat{\secedge}{\catvariableof{\secedge}} \, .
    %\end{align*}
\end{theorem}
\begin{proof}
    We show the first claim by induction over the update steps in \algoref{alg:knowledgePropagation}.
    At the start, where $\kcoreofat{\edge}{\edgevariables} = \onesat{\edgevariables}$, we trivially have
    \begin{align*}
        \contractionof{\extnet \cup \{\kcoreofat{\edge}{\edgevariables} \wcols \edge\in\domainedges\}}{\nodevariables}
        = \contractionof{\extnet \cup \{\onesat{\edgevariables} \wcols \edge\in\domainedges\}}{\nodevariables}
        = \contractionof{\extnet}{\nodevariables} \, .
    \end{align*}
    Let us now assume, that for a state of cores $\{\kcoreof{\edge} \wcols \edge\in\domainedges\}$ the first claim holds and let $\secedges\subset\edges$ be chosen for the update of $\kcoreof{\secedge}$.
    By the invariance under adding the support of subcontractions, which we will proof in more detail as \theref{the:monotonicityBinaryContractions} in \charef{cha:messagePassing}, we have for the update
    \begin{align*}
        \tilde{\kcore}^{\secedge}[\catvariableof{\secedge}]
        = \nonzerocirc\contractionof{
            \{\hypercoreof{\edge} \wcols \edge\in\secedges\} \cup \{\kcoreof{\edge} \wcols \edge\in\domainedges, \, \edge\cap\bigcup_{\secedge\in\secedges}\secedge\neq\varnothing  \}
        }{\catvariableof{\secedge}}
    \end{align*}
    that
    \begin{align*}
        \contractionof{\extnet \cup \{\kcoreofat{\edge}{\edgevariables} \wcols \edge\in\domainedges\}}{\nodevariables}
        = \contractionof{\extnet \cup \{\kcoreofat{\edge}{\edgevariables} \wcols \edge\in\domainedges\} \cup \{\tilde{\kcore}^{\secedge}[\catvariableof{\secedge}]\}
        }{\nodevariables}  \, .
    \end{align*}
    Thus, the first claim holds also after the update of the core to $\secedge$.

    % Since the kcore is in the contraction
    We further have with the monotonicity of boolean contraction (see \theref{the:monotonicityBinaryContractions}) that for any update of $\kcoreof{\secedge}$ by $\tilde{\kcore}^{\secedge}$
    \begin{align*}
        \tilde{\kcore}^{\secedge}[\catvariableof{\secedge}]
        = \nonzerocirc\contractionof{
            \{\hypercoreof{\edge} \wcols \edge\in\secedges\}
            \cup \{\kcoreof{\edge} \wcols \edge\in\domainedges, \, \edge\cap\bigcup_{\edge\in\secedges}\edge\neq\varnothing\}
        }{\catvariableof{\secedge}}
        \prec \kcoreofat{\secedge}{\catvariableof{\secedge}} \, .
    \end{align*}
    Thus, each Knowledge Core is monotonously decreasing at each update, with respect to the partial tensor ordering.

    From the first claim we further have for any $\secedge\in\secedges$
    \begin{align*}
        \contractionof{\{\kcoreofat{\secedge}{\catvariableof{\secedge}}\}\cup\extnet \cup \left\{\kcoreof{\edge} \wcols \edge\in\domainedges/\{\secedge\}\right\}}{\catvariableof{\secedge}}
        = \contractionof{\extnet}{\catvariableof{\secedge}}
    \end{align*}
    And thus in combination with the monotonicity of boolean contraction (see \theref{the:monotonicityBinaryContractions}) that
    \begin{align*}
        \nonzeroof{\contractionof{\extnet}{\catvariableof{\secedge}}} \prec \kcoreofat{\secedge}{\catvariableof{\secedge}}  \, . & \qedhere
    \end{align*}
%	We deduce the theorem from generic properties of the support of contractions, see \secref{sec:supportContractionEquations}.
%	Monotonic decreasing follows from montonocity of boolean tensor contractions, see \theref{the:monotonicityBinaryContractions}.
%	By \theref{the:invarianceAddingSubcontractions} we have during any state of the algorithm
%		\[ \nonzerocirc\contractionof{\extnet}{\catvariableof{\nodes}}  =
%		\nonzerocirc\contractionof{\extnet\cup\{\kcoreof{\edge} : \edge\in\edges\}}{\catvariableof{\nodes}}  \, .
%		\]
%	If follows that
%		\[ \nonzeroof{\contractionof{\extnet}{\edgevariables}} =  \nonzerocirc\contractionof{\extnet\cup\{\kcoreof{\edge} : \edge\in\edges\}}{\catvariableof{\edge}} \]
%	and by \theref{the:monotonicityBinaryContractions}
%		\[  \tilde{\kcoreof{\edge}}  \prec \kcoreof{\edge} \, . \]
\end{proof}

Let us now show that Knowledge Propagation always terminates.
We can further characterize the knowledge cores at termination.

\begin{definition}
    We say that a set of knowledge cores $\{\kcoreof{\edge} \wcols \edge\in\domainedges\}$ is consistent with a set $\{\hypercoreof{\edge} \wcols \edge\in\secedges\}$, if for any $\edge\in\domainedges$
    \begin{align*}
        \kcoreofat{\edge}{\catvariableof{\edge}}
        = \nzcontractionof{\{\hypercoreof{\edge} \wcols \edge\in\secedges\} \cup \{\kcoreof{\edge} \wcols \edge\in\domainedges\}}{\catvariableof{\edge}} \, .
    \end{align*}
\end{definition}

This property is similar to the completeness of a knowledge core, when interpreting the other knowledge cores and the constraints $\{\hypercoreof{\edge} \wcols \edge\in\secedges\}$ as posing a Constraint Satisfaction Problem.

\begin{theorem}
    Knowledge Propagation \algoref{alg:knowledgePropagation} always terminates.
%    At termination, each knowledge core $\kcoreof{}$
    At termination we further for each $\secedges\in\arbset$ and $\edge\in\domainedges$ with $\edge\cap\bigcup_{\edge\in\secedges}\edge\neq\varnothing$, that the knowledge cores $\{\kcoreof{\edge} \wcols \edge\in\domainedges, \, \edge\cap\bigcup_{\edge\in\secedges}\edge\neq\varnothing\}$ are consistent with $\{\hypercoreof{\edge} \wcols \edge\in\secedges\}$.
\end{theorem}
\begin{proof}
    For each knowledge core, there are finitely many boolean tensor precessing it with respect to the partial order.
    Therefore, since they are monotonously decreasing, each knowledge core can only be varied finitely many times during the algorithm.
    In total the algorithm can run only finitely many times in the second for loop, where new sets of edges are pushed into the queue.
    Therefore the while loop will always terminate.

    When after a single pass through the while loop with chosen $\secedges\in\arbset$, the set $\secedges$ is not pushed back into $\graphqueue$, we have for any $\edge\in\domainedges$ with $\edge\cap\bigcup_{\edge\in\secedges}\edge\neq\varnothing$ that
%    At termination, we have for any $\secedges\in\arbset$
    \begin{align*}
        \kcoreofat{\edge}{\catvariableof{\edge}}
        = \nzcontractionof{\{\hypercoreof{\edge} \wcols \edge\in\secedges\} \cup \{\kcoreof{\edge} \wcols \edge\in\domainedges, \, \edge\cap\bigcup_{\edge\in\secedges}\edge\neq\varnothing\}}{\catvariableof{\edge}} \, .
    \end{align*}
    Whenever the contraction on the right hand side changes during the algorithm, the set $\secedges$ is pushed into $\graphqueue$.
    At termination of the algorithm, $\graphqueue$ is empty, and the claimed consistency therefore has to hold.
%    since otherwise the queue $\graphqueue$ would at least contain $\secedges$ and the while loop would not terminate.
\end{proof}

%A direct consequence of this theorem is that when running Knowledge Propagation with all edges, then
%\begin{corollary}
%    When we choose $\arbset=\{\edges\}$ for an CSP $\extnet$, then the knowledge cores returned by \algoref{alg:knowledgePropagation} are complete.
%\end{corollary}


We can exploit the Knowledge Propagation \algoref{alg:knowledgePropagation} for the solution of Constraint Satisfaction Problems, by taking $\extnet$ as the tensor network of constraint tensors.
Whenever a knowledge core vanishes, we can conclude that the Constraint Satisfaction Problems is not satisfiable, as we show next.

\begin{corollary}
    Let us for a Constraint Satisfaction Problem encoded by $\extnet$ run Knowledge Propagation \algoref{alg:knowledgePropagation}.
    Whenever for any $\secedge\in\edges$ we have $\kcoreofat{\secedge}{\catvariableof{\secedge}}=\zerosat{\catvariableof{\secedge}}$, then the Constraint Satisfaction Problem is not satisfiable.
\end{corollary}
\begin{proof}
    Whenever $\kcoreofat{\secedge}{\catvariableof{\secedge}}=\zerosat{\catvariableof{\secedge}}$, then we have by \theref{the:soundnessKnowledgePropagation}
    \begin{align*}
        \nonzeroof{\contractionof{\extnet}{\catvariableof{\secedge}}} \prec \zerosat{\catvariableof{\secedge}}
    \end{align*}
    and therefore
    \begin{align*}
        \contraction{\extnet} = 0 \, . & \qedhere
    \end{align*}
\end{proof}

% Combination with backtracking search
When the Knowledge Propagation \algoref{alg:knowledgePropagation} converges in a given implementation and no knowledge core vanishes, we can however not conclude that the Constraint Satisfaction Problem is not satisfiable.
However, for any index tuple $\catindexof{\nodes}$ to be a solution of the CSP to $\extnet$, we have the necessary condition
\begin{align*}
    \uniquantwrtof{\secedge\in\secedges}{\kcoreofat{\edge}{\catvariableof{\secedge}=\restrictionofto{\catindexof{\nodes}}{\secedges}}=1} \, ,
\end{align*}
where by $\restrictionofto{\catindexof{\nodes}}{\secedges}$ we denote the restriction of the index tuple $\catindexof{\nodes}$ to the variables included in $\secedges$.
One can use this insight as a starting point for backtracking search, where the assignments to variables $\catvariableof{\secnodes}$ are iteratively guessed, based on the restriction that each constraint is locally satisfialbe, i.e. .
\begin{align*}
    \uniquantwrtof{\secedge\in\secedges}{
        \contraction{\kcoreofat{\edge}{
            \catvariableof{\secedge\cap\secnodes}=\restrictionofto{\catindexof{\nodes}}{\secedge\cap\secnodes}
            ,\catvariableof{\secedge/\secnodes}}} \neq 0
    } \, .
\end{align*}
One can understand the guess of an assignment $\catindexof{\node}$ to a variable $\catvariableof{\node}$, as it is done during backtracking search, as an inclusion of a constraint
\begin{align*}
    \kcoreofat{\{\node\}}{\catvariableof{\node}}
    = \onehotmapofat{\catindexof{\node}}{\catvariableof{\node}} \, .
\end{align*}
Therefore, Knowledge Propagation \algoref{alg:knowledgePropagation} can be integrated with backtracking search, with iterations between propagations of knowledge and guessing of additional variables.
%The knowledge cores $\kcoreof{\edge}$ are subset encoding of possible local choices, according to which variables can be assigned.



\subsect{Applications}

Let us exemplify the usage of Knowledge Propagation on Constraint Satisfaction Problems posed by entailment queries on Markov Networks.

\begin{corollary}
    \label{cor:knowledgePropagationMarkovNetworks}
    Let \algoref{alg:knowledgePropagation} be run on the cores $\tnetof{\graph}\cup\{\bencodingof{\exformula}\}$ with an arbitrary design of $\secedges$.
    Whenever for a formula $\formulaat{\catvariableof{\secnodes}}$ and a $\kcoreof{\edge}$ we have
    \[ \contractionof{\kcoreof{\edge},\bencodingof{\exformula}}{\exformulavar=0} =0  \]
    then the Markov Network $\extnet$ probabilistically entails $\exformula$.
    If on the contrary
    \[ \contractionof{\kcoreof{\edge},\bencodingof{\exformula}}{\exformulavar=1} =0  \]
    then the Markov Network $\extnet$ probabilistically entails $\lnot\exformula$, that is probabilistically contradicts $\exformula$.
\end{corollary}
\begin{proof}
    This follows from \theref{the:soundnessKnowledgePropagation} ensuring the soundness of Knowledge Propagation and the sufficiency of local entailment.
\end{proof}

\begin{example}[Batch decision of entailment]
    Let $\formulaset$ be a set of formulas and $\probofat{\graph}{\shortcatvariables}$ a Markov Network, for which it shall be decided, which formulas in $\formulaset$ are entailed, contradicted or contingent.
    We can in addition to the cores of the Markov Network create the cores $\{\bencodingofat{\exformula}{\formulavar,\shortcatvariables} \wcols \exformula\in\formulaset\}$ and prepare the knowledge cores
    \begin{align*}
        \kcoreofat{\{\formula\}}{\formulavar} \, .
    \end{align*}
    To decide entailment batchwise, Knowledge Propagation \algoref{alg:knowledgePropagation} can be run.
    Whenever during the algorithm we have that for a $\formula$, then \corref{cor:knowledgePropagationMarkovNetworks} implies that if
    \begin{align*}
        \kcoreofat{\{\formula\}}{\formulavar} =
        \begin{cases}
            \tbasisat{\formulavar} & \text{then, the formula is entailed by }\probof{\graph} \, . \\
            \fbasisat{\formulavar} & \text{then, the formula is contradicted by }\probof{\graph} \, .\\
            \onesat{\formulavar} & \text{then no conclusion can be drawn.}
        \end{cases}
    \end{align*}
    Note, that $\kcoreofat{\{\formula\}}{\formulavar} = \zerosat{\formulavar}$ can not happen, since this would mean that $\nonzeroof{\probof{\graph}}$ is inconsistent.
    Thus, at any stage of \algoref{alg:knowledgePropagation}, one of the three holds.
    % Inference rules such as modus ponens can be mimicked.
\end{example}


\subsect{Mimicking Inference Rules by Propagation}

While so far we have discussed semantic based entailment, there are inference rules exploiting only logical syntax to infer entailed statements.
We here show, that they can be captures by the knowledge propagation scheme, if the sets $\arbset$ and $\domainedges$ are chosen properly.

Whenever
\begin{align*}
    \bigvee_{\exformula\in\formulaset} \exformula \models \secexformula
\end{align*}
then
\begin{align*}
    \tbasisat{\secexformulavar} =
    \contractionof{\{\formulaat{\shortcatvariables} \wcols \formula\in\formulaset\} \cup \{\bencodingofat{\secexformula}{\secexformulavar,\shortcatvariables}\}}{\secexformulavar} \, ,
\end{align*}
that is the inference rule can be performed in when $\formulaset$ are in $\arbset$.

\begin{example}{Modus Ponens}
    For example, when for two formulas $\exformula,\secexformula\in\formulaset$ we have $\exformula\models\secexformula$, then when $\kcoreofat{\{\exformula\}}{\exformulavar} = \tbasisat{\formulavar}$ we have
    \begin{align*}
        \tbasisat{\secexformulavar}
        = \contractionof{(\exformula\Rightarrow\secexformula)[\exformulavar,\secexformulavar],\kcoreofat{\{\exformula\}}{\exformulavar}}{\secexformulavar} \, ,
    \end{align*}
    that is entailment of $\secexformula$ can be concluded using a single update.

    When we have a Knowledge Base of horn clauses, we run Knowledge Propagation with each horn clause being a constraint core and a knowledge core for any variable.
    \algoref{alg:knowledgePropagation} therefore resembles the forward chaining algorithm of propositional logics (see Figure~7.15 in \cite{russell_artificial_2021}).
    It is known, that forward chaining is complete for Horn Logic.
    Thus, the knowledge cores returned in that case by \algoref{alg:knowledgePropagation} are complete for the Knowledge Base as a CSP.
\end{example}


%\subsubsect{Resolution}
%Requires additional knowledge core.

% Comment
%We notice that a careful design of \algoref{alg:knowledgePropagation} can increase the efficiency of the batchwise entailment.

%\sect{Discussion}
%
%%% Where to put? -> already in part intro
%\begin{remark}[Interpretation of Contractions in Logical Reasoning]
%    The coordinates of contracted boolean tensor networks describe whether the by the coordinate indexed world is a model of the Knowledge Base at hand.
%    Contractions, which only leave a part variables open, store the counts of the world respecting conditions given by the choice of slices.
%    When contracting without open variables, we thus get the total world count.
%
%    This is consistent with the probabilistic interpretation of contractions, when applying the frequentist interpretation of probability and defining normalized worldcounts as probabilities.
%\end{remark}

%\begin{remark}{Tradeoff between generality and efficiency}
%    While generic entailment decision algorithms (those by the full network) can decide any entailment, local algorithms as presented here can only perform some, but therefore more effectively as operating batchwise (dynamically deciding entailment for many leg variables).
%    This is a typical phenomenon in logical reasoning and related to decidability.
%\end{remark}
%
%Local contraction approaches in inference, especially when orchestrated by a Knowledge Propagation algorithm, mimik inference rules in syntax-based prove approaches.


    \part{\parttwotext}\label{par:two}
    \chapter{Introduction into \parref{par:two}}

\begin{highlight}
	Research [...] should bring together logic`s aptitude for handling the visible and probability`s ability to summarize the invisible. - \text{Judea Pearl \cite{pearl_probabilistic_1988}}
\end{highlight}

After having explored the tensor approach in the logical and probabilistic foundations of artificial intelligence in \parref{par:one}, we now investigate \HybridLogicNetworks{}, which unify the two approaches.

\sect{\HybridLogicNetworks{} as Computation Activation Networks}

%We unify the discussion on exponential families and propositional formulas, by constructing sufficient statistics and base measures based on propositional formulas.
%This restricts the sufficient statistics to boolean valued functions, which can be decomposed into connectives.

% Computation Activation Networks in part \parref{par:one}:
The unification is along the framework of Computation Activation Networks (see \defref{def:realizableStatDistributions}), which is common to the tensor network representations in probabilistic and logical models:
\begin{itemize}
    \item Probability distributions, which have a sufficient statistics can be represented by a computation network of the statistic and an activation tensor.
    We are especially interested in distributions with elementary activation tensor, which is the case for exponential families.
    \item Propositional formulas can be computed based on their syntactical decompositions (see \charef{cha:logicalRepresentation}).
    The uniform distributions over their models has the formula itself as a sufficient statistic and can be instantiated by a boolean activation cores.
\end{itemize}
By allowing for arbitrary elementary activation tensors, we can unify both approaches, when focusing on boolean valued statistics.
The resulting computation activation networks are called \HybridLogicNetworks{}, since they unify logical and probabilistic models.
In \charef{cha:networkRepresentation} we focus on the representation of such networks and introduce the following (see \figref{fig:elementaryComputableSketch}):
\begin{itemize}
    \item \textbf{\MarkovLogicNetworks{}:} By demanding positive and elementary activation tensors, we parametrize positive distributions interpreted as uncertainty-tolerant soft logical knowledge bases.
    \item \textbf{\HardLogicNetworks{}:} By demanding boolean and elementary activation tensors, we parametrize propositional formulas interpreted by hard logical knowledge bases.
    \item \textbf{\HybridLogicNetworks{}:} Both the soft and the hard parametrizations are unified when admitting generic elementary activation tensors.
\end{itemize}
In \charef{cha:networkReasoning} we characterize these networks as maximum entropy distributions and then focus on the inference properties of such networks.

Regarding the representation of \HybridLogicNetworks{}, we investigate sparsity mechanisms to result in efficient tensor network representations:
\begin{itemize}
	\item \textbf{\DecompositionSparsity{}:} When the sufficient statistics is decomposable into logical connectives, we find tensor network representation by basis encodings of the component functions.
    This mechanism has been exploited in \charef{cha:logicalRepresentation} for a single propositional formula and will in \charef{cha:networkRepresentation} be extended to sets of propositional formulas.
    \item \textbf{\SelectionSparsity{}:} We will investigate representation schemes for sets of formulas, which share a common structure, in \charef{cha:formulaSelection}.
%    To capture sparse mechanisms we define formula selecting networks in \charef{cha:formulaSelection}.
%    This allows for sparse representation of weighted sums of formulas, which appear in energy tensors.
    \item \textbf{\PolynomialSparsity{}:} Tensors are decomposed into sums of restricted elementary tensors, which are interpreted by monomials.
    We will describe such decompositions in \secref{sec:HLNpolyRepresentation} and relate it later in \charef{cha:sparseRepresentation} to restricted $\cpformat$ decompositions, which we call basis+.
\end{itemize}

\begin{figure}[t]
    \begin{center}
        \begin{tikzpicture}[yscale=0.5, xscale=0.9]
	\node[anchor=center] (text) at (-2.25,13) {Distributions with sufficient statistic $\formulaset$: $\realizabledistsof{\formulaset,\maxgraph}$};
	\draw[dashed] (-10.5,14) rectangle (5,6);

	\draw[dashed] (-10,12) rectangle (4.5,6.5);
	\node[anchor=center] (text) at (-2.25,11) {Hybrid Logic Networks: $\realizabledistsof{\formulaset,\elgraph}$};
	
	\draw[\probcolor] (-9.5,10) rectangle (-3,7);
	\node[\probcolor] (text) at (-6,9) {Markov Logic Networks: };
	\node[\probcolor] (text) at (-6,7.9) {Positive activation cores};

	\draw[\concolor] (-2.5,10) rectangle (4,7);
	\node[\concolor] (text) at (1,9) {Hard Logic Networks:};
	\node[\concolor] (text) at (1,7.9) {Boolean activation cores};

\end{tikzpicture}
    \end{center}
    \caption{Sketch of distributions with sufficient statistics by a set of propositional formulas $\mlnstat$.
    In most generality, any distribution with a sufficient statistics can be represented by an arbitrary activation tensor and is therefore in $\realizabledistsof{\mlnstat,\maxgraph}$.
    The \HybridLogicNetworks{} are those with elementary activation tensor, that is the elements of $\realizabledistsof{\mlnstat,\elgraph}$.
    By further demanding positive activation tensors we characterize the \MarkovLogicNetworks{} $\mlnstat$ and by demanding boolean activation tensors we the \HardLogicNetworks{} $\mlnstat$.
    }
    \label{fig:elementaryComputableSketch}
\end{figure}

\sect{Extensions towards First-Order Logics}

While so-far the focus had been on propositional logic as an explainable framework in machine learning, we show in \charef{cha:folModels} extenstions towards more expressive first-order logics.
We therein encounter two mechanisms in logic introducing tensor structures:
\begin{itemize}
    \item \textbf{\SubstitutionStructure{}:} Assigning objects to variables results in a vector of possible substitutions. When there are formulas with multiple variables, these lists of substitutions carry a tensor structure.
    \item \textbf{\SemanticStructure{}:} Mapping all possible worlds defined by $\atomorder$ atoms requires $2^{\atomorder}$ dimensions, which are at best represented by a space of order $\atomorder$ tensors.
    This is analogous to the factored representations of a system.
\end{itemize}

\sect{Probabilistic guarantees}

Besides that, probabilistic guarantees on the success of the learning problems are derived in \charef{cha:concentration}.
Here we also focus on boolean statistics, which coordinates are Bernoulli variables.
Due to their boundedness, they and their averages are sub-Gaussian variables with favorable concentration properties.

    \chapter{\chatextformulaSelection}\label{cha:formulaSelection}

In this chapter we will investigate efficient schemes to represent collections of formulas with similar structure in one tensor network.

% basis encoding of the selection map
\begin{definition}
	Given a set of $\seldim$ formulas $\{\formulaof{\selindex} : \selindexin\}$, we define the formula selecting map as
		\[  \fselectionmapat{\shortcatvariables,\selvariable} : \atomstates \times [\seldim] \rightarrow [2] \]
	defined for $\selindexin$ by
		\[ \fselectionmapat{\shortcatvariables=\atomindices,\selvariable=\selindex} =  \formulaofat{\selindex}{\shortcatvariables=\atomindices} \, . \]
\end{definition}

% Selection Variables
We introduce a selection variable $\selvariable$ and depict the formula selection in Figure~\ref{fig:formulaSelectionMap}.

% Depiction
\begin{figure}[h]
\begin{center}
	\begin{tikzpicture}[scale=0.35, thick] % , baseline = -3.5pt

    \begin{scope}
        [shift={(-20,0)}]
        \node[anchor=center] (text) at (-1,3) {${a)}$};

        \node [circle, draw, thick, fill=gray!50, minimum size = \nodeminsize] (T1) at (0,0) {\tiny $\catvariableof{0}$};
        \node [circle, draw, thick, fill=gray!50, minimum size = \nodeminsize] (T2) at (3,0) {\tiny $\catvariableof{1}$};
        \node[anchor=center] (text) at (6,0) {${\cdots}$};
        \node [circle, draw, thick, fill=gray!50, minimum size = \nodeminsize] (T3) at (9,0) {};
        \node[anchor=center] (text) at (9,0) {\tiny $\catvariableof{\seldim\shortminus1}$};

        \node [circle, draw, thick, fill=gray!50, minimum size = \nodeminsize] (T4) at (12,3) {};
        \node[anchor=center] (text) at (12,3) {\tiny $\selvariable$};

        \draw[->-] (6,3) -- (6,6);

        \node [circle, draw, thick, fill=gray!50, minimum size = \nodeminsize] (S) at (6,6) {};
        \node[anchor=center] (text) at (6,6) {\tiny $\headvariableof{\fselectionmap}$};

        \draw[->-] (T1) -- (6,3);
        \draw[->-] (T2) -- (6,3);
        \draw[->-] (T3) -- (6,3);
        \draw[->-] (T4) -- (6,3);

    \end{scope}


    \node[anchor=center] (text) at (-1,3) {${b)}$};


    \begin{scope}
        [shift={(0,-2)}]
        \draw[-<-] (0,1)--(0,-1) node[midway,left] {\tiny $\catvariableof{0}$};
        \draw[-<-] (1.5,1)--(1.5,-1) node[midway,left] {\tiny $\catvariableof{1}$};
        \node[anchor=center] (text) at (3,0) {$\cdots$};
        \draw[-<-] (4,1)--(4,-1) node[midway,right] {\tiny $\catvariableof{\seldim\shortminus1}$};
    \end{scope}

    \draw (-1,1) rectangle (5,-1);
    \node[anchor=center] (text) at (2,0) {$\bencodingof{\fselectionmap}$};
    \draw[->-] (2,1) -- (2,3) node[midway, right]  {\tiny $\headvariableof{\fselectionmap}$};
    \draw[-<-] (5,0) -- (7,0) node[midway, above] {\tiny $\selvariable$};


\end{tikzpicture}
\end{center}
\caption{Representation of the Formula Selecting map as a 
a) Graphical Model with a selection variable $\fselectionmap$.
b) Decorating Tensor Core with selection variable corresponding with an additional axis.}
\label{fig:formulaSelectionMap}
\end{figure}


% Decomposition
A naive representation of the formula selecting map is as a sum
	\[ \fselectionmap = \sum_{\selindexin} \formulaofat{\selindex}{\shortcatvariables}  \otimes \onehotmapofat{\selindex}{\selvariable} \, . \]
Such a representation scheme requires linear resources in the number of formulas.
We will show in the following, that we can exploit common structure in formulas to drastically reduce this resource consumption.



\sect{Construction schemes}

% Naturality of folding
Let us now investigate efficient schemes to define sets of formulas to be used in the definition of $\fselectionmap$.
We will motivate the folding of the selection variable into multiple selection variables by compositions of selection maps.


\subsect{Connective Selecting Tensors}

We represent choices over connectives with a fixed number of arguments by adding a selection variable to the cores and defining each slice by a candidate connective.

% Formal map
\begin{definition}\label{def:connectiveSelector}
	Let $\{\connectiveof{0},\ldots,\connectiveof{\seldimof{\cselectionsymbol}-1}\}$ be a set of connectives with $\atomorder$ arguments.
	The associated connective selection map is
		\[ \cselectionmapat{\shortcatvariables,\selvariableof{\cselectionsymbol}}
		: \atomstates \times [\seldimof{\cselectionsymbol}] \rightarrow [2] \]
	defined for each $\selindexofin{\cselectionsymbol}$ and $\shortcatindices\in\atomstates$ by 
		\[ \cselectionmapat{\shortcatvariables=\shortcatindices,\indexedselvariableof{\cselectionsymbol}} 
		= \connectiveofat{\selindexof{\cselectionsymbol}}{\shortcatvariables=\shortcatindices}  \, . \]
\end{definition}

We depict the basis encoding of connective selection maps in Figure~\ref{fig:connectiveSelector}.

\begin{figure}[h]
\begin{center}
	\begin{tikzpicture}[scale=0.35, thick] % , baseline = -3.5pt

\begin{scope}[shift={(-20,0)}]
	\node[anchor=center] (text) at (-1,3) {${a)}$};

	\node [circle, draw, thick, fill=gray!50, minimum size = \nodeminsize] (T1) at (0,0) {\tiny $\catvariableof{0}$};
	\node [circle, draw, thick, fill=gray!50, minimum size = \nodeminsize] (T2) at (3,0) {\tiny $\catvariableof{1}$};
	\node[anchor=center] (text) at (6,0) {${\cdots}$};
	\node [circle, draw, thick, fill=gray!50, minimum size = \nodeminsize] (T3) at (9,0) {};
	\node[anchor=center] (text) at (9,0) {\tiny $\catvariableof{\atomorder\shortminus1}$};
	
	\node [circle, draw, thick, fill=gray!50, minimum size = \nodeminsize] (T4) at (12,0) {};
	\node[anchor=center] (text) at (12,0) {\tiny $\cselinputvariable$};

	
	\node [circle, draw, thick, fill=gray!50, minimum size = \nodeminsize] (S2) at (6,4.5) {};
	\node[anchor=center] (text) at (6,4.5) {\tiny $\catvariableof{\cselectionsymbol}$};
	
	\coordinate (S) at (6,2.5);
	\draw[->] (S) -- (S2);

	\draw[->] (T1) -- (S);
	\draw[->] (T2) -- (S);
	\draw[->] (T3) -- (S);
	\draw[->] (T4) -- (S);
	
\end{scope}


\node[anchor=center] (text) at (-1,3) {${b)}$};
	
	
\begin{scope}[shift={(5,2)}]

	\begin{scope}[shift={(0,-2)}]
		\draw[<-] (0,1)--(0,-1) node[midway,left] {\tiny $\catvariableof{0}$}; 
		\draw[<-] (1.5,1)--(1.5,-1) node[midway,left] {\tiny $\catvariableof{1}$}; 
		\node[anchor=center] (text) at (3,0) {$\cdots$};
		\draw[<-] (4,1)--(4,-1) node[midway,right] {\tiny $\catvariableof{\atomorder\shortminus1}$}; 
	\end{scope}
	
\draw (-1,1) rectangle (5,-1);
\node[anchor=center] (text) at (2,0) {$\bencodingof{\cselectionmap}$};
\draw[->] (2,1) -- (2,3) node[midway, right]  {\tiny $\catvariableof{\cselectionsymbol}$};
\draw[<-] (5,0) -- (7,0) node[midway, above] {\tiny $\cselinputvariable$};

\end{scope}
	

\end{tikzpicture}
\end{center}
\caption{Connective Selector.}
\label{fig:connectiveSelector}
\end{figure}

%Following a different perspective: skeleton+atomindices at atomic expression level, atomindices at complex expression level!
%Having an parametrization of binary connectives by $\circ_{\selindex}$ we can define the corresponding connective selector tensor by
%	\[ \rencodingof{\circ}_{\selindex,:,:} = \rencodingof{\circ_{\selindex}}_{:,:} \, . \]

%\begin{remark}[$\htformat$ Interpretation of Superposed Formula Tensor Networks]\label{rem:HTDecomSFT}
	%Continuing Remark ~\ref{rem:HTDecomFT}:
%	Superposed Formula Tensors have a decomposition into a $\htformat$ as sketched here, where we distinguish between formula selection subspaces (indices $\selindexof{\selenumerator}$) and atomic subspaces (indices $\atomlegindexof{\atomenumerator})$.
%	At each formula selection we thus have a decomposition into three subspaces, two of atomic formulas and one for the formula selection.
%\end{remark}




\subsect{Variable Selecting Tensor Network}

%\red{Works also for categorical variables! -> Into Contraction Calculus?}

%% Definition
\begin{definition}\label{def:variableSelector}
	The selection of one out of $\seldim$ variables in a list $\catvariableof{[\seldim]}$ is done by variable selecting maps
	\begin{align}
		\vselectionmapat{\catvariableof{[\seldim]},\selvariableof{\vselectionsymbol}}:  \left(\bigtimes_{\selindex\in[\seldim]}[2]\right) \times [\seldim]  \rightarrow [2]
	\end{align}
	are defined coordinatewise by
	\begin{align}
		\vselectionmapat{\indexedcatvariableof{0},\ldots,\indexedcatvariableof{\seldim-1},\indexedselvariableof{\vselectionsymbol}} = \catindexof{\selindex} \, .
	\end{align}
\end{definition}
	
% Interpretation as multiplex gate
Variable selecting maps appear in the literature as multiplex gates (see Definition 5.3 in \cite{koller_probabilistic_2009}).

The basis encoding of the variable selection map has a decomposition
\begin{align*}
	\rencodingofat{\vselectionmap}{\vselectionheadvar,\catvariableof{[\seldimof{\vselectionsymbol}]}}
	= \sum_{\selindexofin{\vselectionsymbol}} 
	\rencodingofat{\atomicformulaof{\selindexof{\vselectionsymbol}}}{\vselectionheadvar,\catvariableof{\selindexof{\vselectionsymbol}}} \otimes  \onehotmapofat{\selindexof{\vselectionsymbol}}{\selvariableof{\vselectionsymbol}} \, . 
\end{align*}
This structure is exploited in the next theorem to derive a tensor network decomposition of $\rencodingof{\vselectionmap}$.

\begin{theorem}[Decomposition of Variable Selecting Maps]\label{the:varSelectorDecomposition}
	Given a list $\catvariableof{[\seldimof{\vselectionsymbol}]}$ of variables, we define for each $\selindexofin{\vselectionsymbol}$ the tensors
		\[ \selectorcomponentofat{\selindexof{\vselectionsymbol}}{\catvariableof{\selindexof{\vselectionsymbol}},\selvariableof{\vselectionsymbol}} 
		= \identityat{\vselectionheadvar,\catvariableof{\selindexof{\vselectionsymbol}}} \otimes \onehotmapofat{\selindexof{\vselectionsymbol}}{\selvariableof{\vselectionsymbol}} 
		+ \onesat{\vselectionheadvar,\catvariableof{\selindexof{\vselectionsymbol}}} \otimes \left(\onesat{\selvariableof{\vselectionsymbol}} - \onehotmapofat{\selindexof{\vselectionsymbol}}{\selvariableof{\vselectionsymbol}} \right) \, . 
		\]
	Then we have (see Figure~\ref{fig:SelectorDecomposition})
		\[ \rencodingofat{\vselectionmap}{\vselectionheadvar,\catvariableof{[\seldim]},\selvariableof{\vselectionsymbol}}
		= \contractionof{
			\{\selectorcomponentofat{\selindexof{\vselectionsymbol}}{\vselectionheadvar,\catvariableof{\selindexof{\vselectionsymbol}},\selvariableof{\vselectionsymbol}} \, : \, \selindexofin{\vselectionsymbol}\}
		}{\vselectionheadvar,\catvariableof{[\seldim]},\selvariableof{\vselectionsymbol}} \, .
		\]
\end{theorem}
\begin{proof}
	We show the equivalence of the tensors on an arbitrary coordinates.
	For $\tilde{\selindex}_{\vselectionsymbol}\in[\seldimof{\vselectionsymbol}]$, $\vselectionheadvar\in[2]$ and $\catindexof{[\seldimof{\vselectionsymbol}]}\in\bigtimes_{\catenumerator\in[\seldimof{\vselectionsymbol}]}[2]$ we have
	\begin{align*}
		& \contractionof{
			\{\selectorcomponentofat{\selindexof{\vselectionsymbol}}{\vselectionheadvar,\catvariableof{\selindexof{\vselectionsymbol}},\selvariableof{\vselectionsymbol}} \, : \, \selindexofin{\vselectionsymbol}\}
		}{\indexedheadvariableof{\vselectionsymbol},\indexedcatvariableof{[\seldim]},\selvariableof{\vselectionsymbol} = \tilde{\selindex}_{\vselectionsymbol}} \\
		& \quad = 
		\prod_{\selindexofin{\vselectionsymbol}} \selectorcomponentofat{\selindexof{\vselectionsymbol}}{
			\indexedheadvariableof{\vselectionsymbol},\indexedcatvariableof{\selindexof{\vselectionsymbol}},\selvariableof{\vselectionsymbol}=\tilde{\selindex}_{\vselectionsymbol}
			} \\
		& \quad = \selectorcomponentofat{\tilde{\selindex}_{\vselectionsymbol}}{
			\indexedheadvariableof{\vselectionsymbol},\indexedcatvariableof{\selindexof{\vselectionsymbol}},\selvariableof{\vselectionsymbol}=\tilde{\selindex}_{\vselectionsymbol}
		} \\
		& \quad = 
		\begin{cases}
		 	1 & \text{if} \quad \headindexof{\vselectionsymbol} = \catindexof{\selindexof{\vselectionsymbol}} \\
		 	0 & \text{else}  
		 \end{cases} \\
		 & = \rencodingofat{\vselectionmap}{\indexedheadvariableof{\vselectionsymbol},\indexedcatvariableof{[\seldim]},\selvariableof{\vselectionsymbol}=\tilde{\selindex}_{\vselectionsymbol}}
	\end{align*}
	In the second equality, we used that the tensor $\selectorcomponentof{\selindexof{\vselectionsymbol}}$ have coordinates $1$ whenever $\tilde{\selindex}_{\vselectionsymbol}\neq\selindexof{\vselectionsymbol}$.
\end{proof}


The decomposition provided by \theref{the:varSelectorDecomposition} is in a CP format, as will be further discussed in \charef{cha:sparseCalculus}.
The introduced tensors $\selectorcomponentof{\selindexof{\vselectionsymbol}}$ are Boolean, but not directed and therefore encodings of relations but not functions (see \charef{cha:basisCalculus}).

%%% Decomposition
%% ! THIS IS NOT \theref{the:functionDecompositionBasisCP}, but works on slice sparsity!

%Using that the encoding $\rencodingof{\atomicformulaof{\selindex}}$ of atomic formulas admits and elementary decomposition (see \theref{the:AtomicFTensor}) we notice that Equation~\ref{eq:selectorDecomposition} describes a so-called monomial decomposition, which will be introduced in \defref{def:polynomialSparsity}.
%We can apply \theref{the:sliceToCP} to find a decomposition of $\selectorcore$ in a CP format consisting of cores
%\begin{align}
%	\selectorcoreof{\selindex} = \dirdeltaof{\randomxof{\selindex},\selvariable} \otimes \onehotmapof{\selindex}
%	+ \sum_{\tilde{\selindex}\in[\seldim] \, , \, \tilde{\selindex}\neq\selindex} \onesof{\randomxof{\selindex},\selvariable} \otimes \onehotmapof{\tilde{\selindex}} \, .
%\end{align}	
%The CP decomposition is depicted in Figure~\ref{fig:SelectorDecomposition}.
%
%% Selectorcores are non-functional basis encodings
%We notice, that the selector cores $\selectorcoreof{\selindex}$ are encodings of a relation, which is not a function.
%Therefore, they are binary but not directed tensors.
%Their contraction 
%\begin{align}\label{eq:selectorDecomposition}
% 	\selectorcore = 
%	\contractionof{\{\selectorcomponentof{\selindex}\, : \, \selindexin\}}
%	{\{\randomxof{0},\ldots,\randomxof{\seldim-1},\selvariable\}}
%\end{align}
%is the basis encoding of the function $\vselectionmap$ and thus binary and directed.


%% Interpretation
%The selectorcores $\selectorcoreof{\selindexof{1}}$ are contracted with the parameter cores and select the respective atom when contracted with truth vector tensormultiplied by constant cores (as placeholder for the other possible atoms).
%Decomposed into disconnected strands for each atomkey, which connect on the selection axis and on the atom truth axis.


\begin{figure}[h]
\begin{center}
	\begin{tikzpicture}[scale=0.35, thick] % , baseline = -3.5pt

%\begin{scope}[shift={(-20,0)}]
%	\node[anchor=center] (text) at (-1,3) {${a)}$};
%
%	\node [circle, draw, thick, fill=gray!50, minimum size = \nodeminsize] (T1) at (0,0) {\tiny $\catvariableof{0}$};
%	\node [circle, draw, thick, fill=gray!50, minimum size = \nodeminsize] (T2) at (3,0) {\tiny $\catvariableof{1}$};
%	\node[anchor=center] (text) at (6,0) {${\cdots}$};
%	\node [circle, draw, thick, fill=gray!50, minimum size = \nodeminsize] (T3) at (9,0) {};
%	\node[anchor=center] (text) at (9,0) {\tiny $\catvariableof{\parlegdim\shortminus1}$};
%	
%	\node [circle, draw, thick, fill=gray!50, minimum size = \nodeminsize] (T4) at (12,3) {};
%	\node[anchor=center] (text) at (12,3) {\tiny $\vselectionvariable$};
%
%	
%	\node [circle, draw, thick, fill=gray!50, minimum size = \nodeminsize] (S) at (6,3) {};
%	\node[anchor=center] (text) at (6,3) {\tiny $\vselectionmap$};
%	
%	\draw[->] (T1) -- (S);
%	\draw[->] (T2) -- (S);
%	\draw[->] (T3) -- (S);
%	\draw[->] (T4) -- (S);
%	
%\end{scope}
%
%
%\node[anchor=center] (text) at (-1,3) {${b)}$};
	

	\begin{scope}[shift={(0,-2)}]
		\draw[<-] (0,1)--(0,-1) node[below] {\tiny $\catvariableof{0}$}; 
		\draw[<-] (1.5,1)--(1.5,-1) node[below] {\tiny $\catvariableof{1}$}; 
		\node[anchor=center] (text) at (3,0) {$\cdots$};
		\draw[<-] (4,1)--(4,-1) node[below] {\tiny $\catvariableof{\seldimof{\vselectionsymbol}\shortminus1}$}; 
	\end{scope}
	
\draw (-1,1) rectangle (5,-1);
\node[anchor=center] (text) at (2,0) {$\rencodingof{\vselectionmap}$};
\draw[->] (2,1) -- (2,3) node[midway, right]  {\tiny $\vselectionheadvar$};
\draw[<-] (5,0) -- (7,0) node[midway, above] {\tiny $\selvariableof{\vselectionsymbol}$};


\node[anchor=center] (text) at (9,0) {${=}$};


\begin{scope}[shift={(12,2)}]

\newcommand{\conposseldec}{4.5,1}

\draw[fill] (\conposseldec) circle (0.25cm);
\draw[->] (\conposseldec) -- (4.5,3) node[midway, right]{\tiny $\vselectionheadvar$};

\draw[<-]  (0,-3) -- (0,-5);
\draw (0,-5) -- (0,-7) node[midway,left] {\tiny $\catvariableof{0}$};
\draw (-1,-1) rectangle (1, -3);
\node[anchor=center] (text) at (0,-2) {\small $\selectorcomponentof{0}$};
\draw[] (0,-1) to[bend right=-20] (\conposseldec);
\draw[] (1,-1.5) -- (12,-1.5) ; 

\draw[<-]  (3,-4) -- (3,-5);
\draw[] (3,-5) -- (3,-7) node[midway,left] {\tiny $\catvariableof{1}$};
\draw (2,-2) rectangle (4, -4);
\node[anchor=center] (text) at (3,-3) {\small $\selectorcomponentof{1}$};
\draw[] (3,-2) to[bend right=-20]  (\conposseldec);
\draw[] (4,-3) to[bend right=3]  (12,-1.5);


\node[anchor=center] (text) at (6,-3.5) {$\cdots$};

\draw[<-]  (9,-5) -- (9,-7) node[midway,left] {\tiny $\catvariableof{\seldimof{\vselectionsymbol}\shortminus1}$};
\draw (7.55,-3) rectangle (10.45, -5);
\node[anchor=center] (text) at (9,-4) {\small $\selectorcomponentof{\seldimof{\vselectionsymbol}\shortminus1}$};
\draw[] (9,-3) to[bend left=-20]  (\conposseldec);
\draw[] (10.45,-4) to[bend right=5]  (12,-1.5);

\draw[fill] (12,-1.5) circle (0.25cm);
\draw[<-] (12.25,-1.5) -- (14,-1.5) node[midway,above] {\tiny $\selvariableof{\vselectionsymbol}$};
\end{scope}

		


\end{tikzpicture}
\end{center}
\caption{Decomposition of the basis encoding of a variable selecting tensor into a network of tensors defined in \theref{the:varSelectorDecomposition}.
	The decomposition is in a $\cpformat$-Format (see \charef{cha:sparseCalculus}. %, when grouping the indices  $\selindexof{\selenumerator}$ and $\atomlegindexof{\atomicformulaof{\selindexof{\selenumerator}}}$).
	%To ease the notation, we here use $\rencodingof{\selenumerator}$ to denote $\rencodingof{\rencodingof{\selenumerator}}$.
}
\label{fig:SelectorDecomposition}
\end{figure}




\sect{State Selecting Tensors}

As an alternative, one can select a state of a categorical variable $\catvariable$.

\begin{definition}
	Given a categorical variable $\catvariableof{\sselectionsymbol}$ with dimension $\catdimof{\sselectionsymbol}$ and a selection variable $\selvariableof{\sselectionsymbol}$ with dimension $\seldimof{\sselectionsymbol}=\catdimof{\sselectionsymbol}$ the state selecting tensor 
		\[ \sselectionmapat{\catvariableof{\sselectionsymbol},\selvariableof{\sselectionsymbol}} : [\catdimof{\sselectionsymbol}] \times [\seldimof{\sselectionsymbol}] \rightarrow [2] \]
	is defined on $\catindexofin{\sselectionsymbol}$ and $\selindexofin{\sselectionsymbol}$ by
	\begin{align*}
		\sselectionmapat{\indexedcatvariableof{},\indexedselvariableof{\sselectionsymbol}} = 
		\begin{cases}
			1 & \text{if} \quad \catindex = \selindexof{\sselectionsymbol} \\
			0 & \text{else}
		\end{cases} \, . 
	\end{align*}
\end{definition}

% Comment: Alternative based on categorical constraints to be introduced later
State selecting tensors can also be realized by variable selecting tensors.
In \secref{sec:categoricalTN} we will describe methods to build atomic variables indicating the states of a categorical variable.
This would, however, increase the number of variables in a tensor network and can thus lead to an exponential overhead of dimensions.
State selecting tensors can therefore be seen as a mean to avoid such dimension increases.

\red{State Selectors can be integrated in Variable Selection framework. In this perspective, variable selection networks are the specific case to $X=1$. }


%% OLD Alternative
%Such categorical variable cores have the advantage of avoiding a full atomization of the categorical variable, which is the creation of atoms reproducing the values
%	\[ \catvariable==\catindexof{\catvariable} \, . \]
%By representing categorical variable choice, one can thus avoid an increase of the order of the encoded tensors, which avoids intractabilities.
%Categorical selection cores can further be integrated in the decomposition scheme \eqref{eq:selectorDecomposition}. 







\sect{Composition of formula selecting maps}
%\sect{Folding of the Selection Variable}

We will now parametrize the sets $\formulaset$ with additional indices and define formula selector maps subsuming all formulas.
To handle large sets of formulas, we further fold the selection variable into tuples of selection variables.

\begin{definition}\label{def:formulaSelector}
	Let there be a formula $\formulaof{\selindexlist}$ for each index tuple in $\selindexlist\in\selstates$, where $\selorder,\seldimof{0},\ldots,\seldimof{\selorder-1}\in\nn$.
	The folded formula selector map (see Figure~\ref{fig:foldedSelector}) is the map 
		\[ \fselectionmapat{\shortcatvariables,\shortselvariables} : \left(\atomstates\right) \times \left(\selstates\right) \rightarrow [2] \]
	with the coordinates at the indices $\shortcatindices\in\atomstates$, $\shortselindices\in\selstates$
		\[  \fselectionmapat{\shortcatvariables=\shortcatindices,\shortselvariables=\shortselindices} 
		= \formulaofat{\shortselindices}{\shortcatvariables=\shortcatindices} \, . \]
\end{definition}

% Formula Section based on skeleton expressions
We will find formula selector maps by composition variables selector maps (\defref{def:variableSelector}) and connective selector maps (\defref{def:connectiveSelector}).
This is especially useful to provide efficient decompositions of basis encodings.

\begin{figure}[h]
\begin{center}
	\begin{tikzpicture}[scale=0.35,thick] % , baseline = -3.5pt

\begin{scope}[shift={(-25,-4)}]
	\node[anchor=center] (text) at (0,7) {${a)}$};

	\node [circle, draw, thick, fill=\nodegrayscale, minimum size = \nodeminsize] (T1) at (0,0) {\colorlabelsize $\catvariableof{0}$};
	\node [circle, draw, thick, fill=\nodegrayscale, minimum size = \nodeminsize] (T2) at (3,0) {\colorlabelsize $\catvariableof{1}$};
	\node[anchor=center] (text) at (6,0) {${\cdots}$};
	\node [circle, draw, thick, fill=\nodegrayscale, minimum size = \nodeminsize] (T3) at (9,0) {};
	\node[anchor=center] (text) at (9,0) {\colorlabelsize $\catvariableof{\atomorder-1}$};
	



	\node [circle, draw, thick, fill=\nodegrayscale, minimum size = \nodeminsize] (S2) at (6,6) {};
	\node[anchor=center] (text) at (6,6) {\colorlabelsize $\headvariableof{\fselectionmap}$};
	
	\node [circle, draw, thick, fill=\nodegrayscale, minimum size = \nodeminsize] (T4) at (12,6) {};
	\node[anchor=center] (text) at (12,6) {\colorlabelsize $\selvariableof{\selorder-1}$};
	
	%\node [circle, draw, thick, fill=\nodegrayscale, minimum size = \nodeminsize] (S) at (6,3) {};
	\node[anchor=center] (text) at (12,3.5) {$\vdots$};
	
	\node [circle, draw, thick, fill=\nodegrayscale, minimum size = \nodeminsize] (T5) at (12,1) {};
	\node[anchor=center] (text) at (12,1) {\colorlabelsize $\selvariableof{0}$};
	
	\coordinate (S) at (6,2.5);
	\draw[->-] (S) -- (S2);
	
	\draw[->-] (T1) -- (S);
	\draw[->-] (T2) -- (S);
	\draw[->-] (T3) -- (S);
	\draw[->-] (T4) -- (S);
	\draw[->-] (T5) -- (S);
	
\end{scope}



\node[anchor=center] (text) at (-3,3) {${b)}$};


\drawatomindices{0}{-4}
\draw (-1,3) rectangle (5, -3);
\node[anchor=center] (text) at (2,0) {$\bencodingof{\fselectionmap}$};

\draw[->-] (2,3)--(2,5) node[midway,right] {\colorlabelsize $\headvariableof{\fselectionmap}$};

\draw[-<-] (5,-2)--(7,-2) node[midway,below] {\colorlabelsize $\selvariableof{0}$};
\draw[-<-] (5,-0.5)--(7,-0.5) node[midway,below] {\colorlabelsize $\selvariableof{1}$};
\node[anchor=center] (text) at (6,0.75) {$\vdots$};
\draw[-<-] (5,2)--(7,2) node[midway,above] {\colorlabelsize $\selvariableof{\selorder\shortminus1}$};


%\draw (7,3) rectangle (9, -3);
%\node[anchor=center] (text) at (8,0) {$\canparam$};


\end{tikzpicture}
\end{center}
\caption{Basis encoding of the folded map $\fselectionmap$.}
\label{fig:foldedSelector}
\end{figure}




\subsect{Formula Selecting Neuron}


% Motivating foldings by composition
The folding of the selection variable is motivated by the composition of selection maps.
We call the composition of a connective selection with variable selection maps for each argument a formula selecting neuron.


\begin{definition}\label{def:fsNeuron}
	Given an order $\selorder\in\nn$ let there be a connective selector $\selvariable_{\exconnective}$ selecting connectives of order $\selorder$ and let $\vselectionmapof{0},\ldots,\vselectionmapof{\selorder-1}$ be a collection of variable selectors.
	The corresponding logical neuron is the map
	\begin{align*}
		\lneuronat{\shortcatvariablelist,\shortselvariablelist} 
		: \left(\atomstates\right) \times [\seldimof{\cselectionsymbol}] \times \left( \bigtimes_{\selenumeratorin} [\seldimof{\selenumerator}]\right) \rightarrow [2] 
	\end{align*}
	defined for $\shortcatindices\in\atomstates$, $\selindexof{\cselectionsymbol}\in[\seldimof{\cselectionsymbol}]$ and
	$\selindices\in \bigtimes_{\selenumeratorin} [\seldimof{\selenumerator}]$ by
	\begin{align*}
		\lneuron(\atomindices, \selindexof{\cselectionsymbol}, \selindices) =
		\cselectionmap(\vselectionmapof{0}(\atomindices, \selindexof{0}),\ldots,\vselectionmapof{\selorder-1}(\atomindices,\selindexof{\selorder-1}), \selindexof{\cselectionsymbol}) \, .
	\end{align*}
\end{definition}

% Tensor Network Decomposition
Each neuron has a tensor network decomposition by a connective selector tensor and a variable selector tensor network for each argument, as we state in the next theorem.

\begin{theorem}{Decomposition of formula selecting neurons}\label{the:neuronDecomposition}
	Let $\lneuron$ a logical neuron, defined for a connective selector $\selvariable_{\exconnective}$ and variable selectors $\vselectionmapof{0},\ldots,\vselectionmapof{\selorder-1}$.
	Then we have (see Figure~\ref{fig:neuronDecomposition} for the example of $\selorder=2$):
	\begin{align*}
		&\rencodingofat{\lneuron}{\headvariableof{\lneuron},\shortcatvariables,\selvariableof{\cselectionsymbol},\selvariableof{\vselectionsymbol,0},\ldots,\selvariableof{\vselectionsymbol,\selorder-1}} \\
		&\quad = \langle\{\rencodingofat{\cselectionmap}{
				\headvariableof{\lneuron},\headvariableof{\vselectionsymbol,0},\ldots,\headvariableof{\vselectionsymbol,\selorder-1}}, \\
		& \quad\quad\quad\rencodingofat{\vselectionmapof{0}}{
				\headvariableof{\vselectionsymbol,0},\shortcatvariables,\selvariableof{\vselectionsymbol,0}},\ldots,
				\rencodingofat{\vselectionmapof{\selorder-1}}{
					\headvariableof{\vselectionsymbol,\selorder-1},\shortcatvariables,\selvariableof{\vselectionsymbol,\selorder-1}}
				\} \rangle
		\left[\headvariableof{\lneuron},\shortcatvariables, \selvariableof{\cselectionsymbol},\selvariableof{\vselectionsymbol,0},\ldots,\selvariableof{\vselectionsymbol,\selorder-1}\right] \, .
	\end{align*}
\end{theorem}
\begin{proof}
	By composition \theref{the:compositionByContraction}.
\end{proof}


%% Example of a FSNN: A skeleton expression, where only the atoms are varied.
%Given a skeleton expression and a set of candidates at each placeholder, we parameterize a set of formulas by the assignment of candidate atoms to each placeholder position.
%Let us denote the set of formulas, which are generated through choosing atoms from $\candidatelistof{\selenumerator}$ for the skeleton formula $\skeleton$ by
%		\[ \formulasetof{\skeleton} \coloneqq
%	 \left\{ \skeletonof{\placeholderof{1},\ldots,\placeholderof{\atomorder}} \, : \, \placeholderof{\atomenumerator} \in \candidatelistof{\atomenumerator} \right\} \]

%We now enumerate at each position $\selenumerator$ the list of candidates $\candidatelistof{\selenumerator}$ using an index $\selindexof{\selenumerator}\in[\seldimof{\selenumerator}]$ and parametrize the choice of the $\selindexof{\selenumerator}$ for the placeholder $\placeholderof{\selenumerator}$ by unit vectors
%	\[ \unitvectoratof{\selenumerator}{\selindexof{\selenumerator}} \in \rr^{\seldimof{\selenumerator}} \, . \]
%We thus have a parameter space $\rr^{\seldim}$ parametrizing the possible assignments to the skeleton in its basis vectors.

\begin{figure}[h]
\begin{center}
	\begin{tikzpicture}[scale=0.45, yscale=1.2, thick] % , baseline = -3.5pt

\begin{scope}[shift={(-14,0)}]

\node[anchor=center] (text) at (-3,6) {${a)}$};

	\node [circle, draw, thick, fill=gray!50] (T1) at (0,0) {\tiny $\atomicformulaof{0}$};
	\node [circle, draw, thick, fill=gray!50] (T2) at (3,0) {\tiny $\atomicformulaof{1}$};
	\node [circle, draw, thick, fill=gray!50] (T3) at (6,0) {\tiny $\atomicformulaof{2}$};
	\node [circle, draw, thick, fill=gray!50] (T4) at (9,0) {\tiny $\atomicformulaof{3}$};
	
	\node [circle, draw, thick, fill=gray!50] (ph) at (1.5,3) {\tiny $\headvariableof{\vselectionsymbol,0}$};%{\tiny $\placeholderof{0}$};
	\node [circle, draw, thick, fill=gray!50] (sel) at (-1.5,3) {\tiny $\selvariableof{\vselectionsymbol,0}$};
	
	\node [circle, draw, thick, fill=gray!50] (ph2) at (6,3) {\tiny  $\headvariableof{\vselectionsymbol,1}$};
	\node [circle, draw, thick, fill=gray!50] (sel2) at (9,3) {\tiny $\selvariableof{\vselectionsymbol,1}$};	
	

	\node [circle, draw, thick, fill=gray!50] (sel3) at (0.25,6) {\tiny $\selvariableof{\cselectionsymbol}$};
	\node [circle, draw, thick, fill=gray!50] (head) at (3.25,6) {\tiny $\headvariableof{\lneuron}$};

	\coordinate (S3) at (3.25,4.5);
	\draw[->] (S3) -- (head);
	\draw [->] (sel3) -- (S3);
	\draw [->] (ph2) -- (S3);
	\draw [->] (ph) -- (S3);
	
	\coordinate (S1) at (1.5,1.5);
	\draw[->] (S1) -- (ph);
	\draw [->] (T1) -- (S1);
	\draw [->] (T2) -- (S1);
	\draw [->] (T3) -- (S1);	
	
	\coordinate (S2) at (6,1.5);
	\draw[->] (S2) -- (ph2);
	\draw [->] (sel) -- (S1);		
	\draw [->] (sel2) -- (S2);

	\draw [->] (T2) -- (S2);
	\draw [->] (T3) -- (S2);	

	\draw [->] (T4) -- (S2);

%	\draw [->] (ph) -- (head);			
\end{scope}


\node[anchor=center] (text) at (-1,6) {${b)}$};

\draw (-1,1) rectangle (4, 4.5);
\node[anchor=center] (text) at (1.5,2.75) {\small $\rencodingof{\lneuron}$}; %{\small $\rencodingof{\placeholderof{0} \placeholderof{1} \placeholderof{2}}$};
\draw[->] (1.5,4.5)--(1.5,6) node[midway,right] {\tiny $\headvariableof{\lneuron}$}; %{\tiny $\catvariableof{\placeholderof{0} \placeholderof{1} \placeholderof{2}}$};

\draw[<-] (4,4)--(5.5,4) node[midway,above] {\tiny $\selvariableof{\vselectionsymbol,0}$}; 
\draw[<-] (4,2.75)--(5.5,2.75) node[midway,above] {\tiny $\selvariableof{\vselectionsymbol,1}$}; 
\draw[<-] (4,1.5)--(5.5,1.5) node[midway,above] {\tiny $\selvariableof{\cselectionsymbol}$}; 

\draw[<-] (-0.5,1)--(-0.5,-0.5) node[midway,left] {\tiny $\catvariableof{0}$}; 
\draw[<-] (0.75,1)--(0.75,-0.5) node[midway,left] {\tiny $\catvariableof{1}$}; 
\draw[<-] (2,1)--(2,-0.5) node[midway,left] {\tiny $\catvariableof{2}$}; 
\draw[<-] (3.25,1)--(3.25,-0.5) node[midway,left] {\tiny $\catvariableof{3}$}; 


\node[anchor=center] (text) at (6.5,2.75) {${=}$};


\draw (7.5,1) rectangle (10.5,2.5);
\node[anchor=center] (text) at (9,1.75) {\small $\selectorcoreof{0}$};
\draw[->] (9,2.5)--(9,3.5) node[midway,left] {\tiny $\headvariableof{\vselectionsymbol,0}$};

\draw[<-] (7.75,1)--(7.75,-0.5) node[midway,left] {\tiny $\catvariableof{0}$}; 
\draw[<-] (9,1)--(9,-0.5) node[midway,left] {\tiny $\catvariableof{1}$}; 
\draw[<-] (10.25,1)--(10.25,-0.5) node[midway,left] {\tiny $\catvariableof{2}$}; 
\draw[<-] (10.5,1.75)--(11.5,1.75) node[midway,above] {\tiny $\selvariableof{\vselectionsymbol,0}$}; 

\draw (12.5,1) rectangle (15.5,2.5);
\node[anchor=center] (text) at (14,1.75) {\small $\selectorcoreof{1}$};
\draw[->] (14,2.5)--(14,3.5) node[midway,left] {\tiny $\headvariableof{\vselectionsymbol,1}$};

\draw[<-] (12.75,1)--(12.75,-0.5) node[midway,left] {\tiny $\catvariableof{1}$}; 
\draw[<-] (14,1)--(14,-0.5) node[midway,left] {\tiny $\catvariableof{2}$}; 
\draw[<-] (15.25,1)--(15.25,-0.5) node[midway,left] {\tiny $\catvariableof{3}$}; 
\draw[<-] (15.5,1.75)--(16.5,1.75) node[midway,above] {\tiny $\selvariableof{\vselectionsymbol,1}$}; 

\draw (8.5,5) rectangle (14.5,3.5);
\node[anchor=center] (text) at (11.5,4.25) {\small $\rencodingof{\cselectionmap}$};
\draw[->] (11.5,5)--(11.5,6) node[midway,left] {\tiny $\headvariableof{\lneuron}$};  %{\tiny $\catvariableof{\placeholderof{0} \placeholderof{1} \placeholderof{2}}$};
\draw[<-] (14.5,4.25)--(15.5,4.25) node[midway,above] {\tiny $\selvariableof{\cselectionsymbol}$}; 

\end{tikzpicture}
\end{center}
\caption{Example of a logical neuron $\lneuron$ of order $\selorder=2$.
	a) Selection and categorical variables and their interdependencies visualized in a hypergraph.
	b) Basis encoding of the logical neuron and tensor network decomposition into variable selecting and connective selecting tensors.
}
\label{fig:neuronDecomposition}
\end{figure}


\subsect{Formula Selecting Neural Network}

% Enhancement of the Expressivity
Single neurons have a limited expressivity, since for each choice of the selection variables they can just express single connectives acting on atomic variables.
The expressivity is extended to all propositional formulas, when allowing for networks of neurons, which can select each others as input arguments.


\begin{definition}\label{def:fsNeuralNetwork}

%	We call a graph consistent of nodes decorated by formula selecting neurons and directed edges representing the argument dependencies of the neuron on other neurons, an architecture graph.
%	An acyclic architecture graph is called a formula selecting neural network.	
%	Formula selecting neurons, which are not included by other formula selecting neurons are called output neurons and collected in the variables $\catvariableof{\larchitecture}$. 
%	A logical neural network is a collection of logical neurons, such that the network graph (nodes: neurons, edges: directed representing argument dependencies) is acyclic (a DAG).
	
	An architecture graph $\graphof{\larchitecture}=(\nodesof{\larchitecture},\edgesof{\larchitecture})$ is an acyclic directed hypergraph with nodes appearing at most once as outgoing nodes.
	Nodes appearing only as outgoing nodes are input neurons and are labeled by $\inneuronset$ and nodes not appearing as outgoing nodes are the output neurons in the set $\outneuronset$ (see Figure~\ref{fig:architectureGraph} for an example).

	Given an architecture graph $\graphof{\larchitecture}=(\nodesof{\larchitecture},\edgesof{\larchitecture})$, a \emph{formula selecting neural network} $\fsnn$ is a tensor network of logical neurons at each $\lneuron\in\nodesof{\larchitecture}/\inneuronset$, such that each neuron depends on variables $\catvariableof{\parentsof{\lneuron}}$ and on selection variables $\selvariableof{\lneuron}$.
	The collection of all selection variable is notated by $\selvariableof{\larchitecture}$.

	The activation tensor of each neuron $\lneuron\in\nodesof{\larchitecture}/\inneuronset$ is
	\begin{align*}
		\lneuractivationat{\catvariableof{\inneuronset},\selvariableof{\larchitecture}} 
		= \contractionof{
			\{\rencodingof{\tilde{\lneuron}} \, : \, \tilde{\lneuron}\in\nodesof{\larchitecture}/\inneuronset \} \cup \{\onehotmapofat{1}{\headvariableof{\lneuron}}\}
		}{\catvariableof{\inneuronset},\selvariableof{\larchitecture}} \, . 
	\end{align*}
		
	The activation tensor of the formula selecting neural network is the contraction
	\begin{align*}
		\fsnnat{\catvariableof{\inneuronset},\selvariableof{\larchitecture}} 
		= \contractionof{
			\{\rencodingofat{\lneuractivation}{\headvariableof{\lneuron},\catvariableof{\parentsof{\lneuron}},\selvariableof{\larchitecture}} \, : \, \lneuron\in\nodesof{\larchitecture}/\inneuronset \} \cup \{\onehotmapofat{1}{\headvariableof{\lneuron}} \, : \, \lneuron\in\outneuronset\}
		}{\catvariableof{\inneuronset},\selvariableof{\larchitecture}} \, . 
	\end{align*}
	
	The expressivity of a formula selecting neural network $\fsnn$ is the formula set
	\begin{align*}
		\formulasetof{\larchitecture} = \left\{ \larchitectureat{\catvariableof{\inneuronset},\indexedselvariableof{\larchitecture}}  : \selindexof{\larchitecture}\in\selstates \right\} \, . 
	\end{align*}
	
\end{definition}

% ? Extend by activation cone stuff
The activation tensor of each neuron depends in general on the activation tensor of its ancestor neurons with respect to the directed graph $\graphof{\larchitecture}$, and thus inherits the selection variables.

% Architecture graph -> Tensor Network
We notice that the architecture graph is a scheme to construct the variable dependency graph of the tensor network $\formulasetof{\larchitecture}$.
To this end, we replace each neuron $\lneuron\in\nodesof{\larchitecture}/\inneuronset$ by an output variable $\headvariableof{\lneuron}$ and further add selection variables $\selvariableof{\lneuron}$ to the directed edges, that is to each directed hyperedge $(\{\lneuron\}, \parentsof{\lneuron})\in\edgesof{\larchitecture}$ we construct a directed hyperedge $(\{\headvariableof{\lneuron}\}, \catvariableof{\parentsof{\lneuron}}\cup\selvariableof{\lneuron})$.

\begin{figure}[h]
\begin{center}
	\begin{tikzpicture}[scale=0.45, yscale=1.2, thick] % , baseline = -3.5pt


	\node [circle, draw, thick, fill=gray!50] (T1) at (0,0) {\tiny $\lneuronof{0}$};
	\node [circle, draw, thick, fill=gray!50] (T2) at (3,0) {\tiny $\lneuronof{1}$};
	\node [circle, draw, thick, fill=gray!50] (T3) at (6,0) {\tiny $\lneuronof{2}$};
	\node [circle, draw, thick, fill=gray!50] (T4) at (9,0) {\tiny $\lneuronof{3}$};
	
	\node [circle, draw, thick, fill=gray!50] (ph) at (1.5,3.5) {\tiny $\lneuronof{4}$};
	\node [circle, draw, thick, fill=gray!50] (ph2) at (6,2.5) {\tiny  $\lneuronof{5}$};	
	
	\node [circle, draw, thick, fill=gray!50] (head) at (3.25,6) {\tiny $\lneuronof{6}$};
	\node [circle, draw, thick, fill=gray!50] (head2) at (0,6) {\tiny $\lneuronof{7}$};

	\draw[->] (ph) -- (head2);

	\coordinate (S3) at (3.25,4.5);
	\draw[->] (S3) -- (head);
	%\draw [->] (sel3) -- (S3);
	\draw [->] (ph2) -- (S3);
	\draw [->] (ph) -- (S3);
	
	\coordinate (S1) at (1.5,1.5);
	\draw[->] (S1) -- (ph);
	\draw [->] (T1) -- (S1);
	\draw [->] (T2) -- (S1);
	\draw [->] (T3) -- (S1);	
	\draw [->] (ph2) -- (S1);	
	
	\coordinate (S2) at (6,1.5);
	\draw[->] (S2) -- (ph2);
	%\draw [->] (sel) -- (S1);		
	%\draw [->] (sel2) -- (S2);

	\draw [->] (T2) -- (S2);
	\draw [->] (T3) -- (S2);	

	\draw [->] (T4) -- (S2);



\end{tikzpicture}
\end{center}
\caption{Example of an architecture graph $\graphof{\larchitecture}$ with input neurons $\inneuronset=\{\lneuronof{0},\lneuronof{1},\lneuronof{2},\lneuronof{3}\}$ and output neurons $\outneuronset=\{\lneuronof{6},\lneuronof{7}\}$
}
\label{fig:architectureGraph}
\end{figure}


\begin{theorem}
	Given fixed selection variables $\selvariableof{\larchitecture}$, the formula selecting neural network is the conjunction of output neurons, that is
	\begin{align*}
		\fsnnat{\catvariableof{\inneuronset},\selvariableof{\larchitecture}} = \bigwedge_{\lneuron\in\outneuronset} \lneuronat{\catvariableof{\inneuronset},\selvariableof{\larchitecture}} \, . 
	\end{align*}
\end{theorem}
\begin{proof}
	By effective calculus (see \theref{the:effectiveConjunction}), we have
		\[ \sbcontractionof{\rencodingofat{\land}{\catvariableof{\land},\shortcatvariables},\onehotmapofat{1}{\catvariableof{\land}}}{\shortcatvariables} = \bigotimes_{\catenumeratorin} \onehotmapofat{1}{\catvariableof{\catenumerator}} \]
	and thus
	\begin{align*}
		\fsnnat{\catvariableof{\inneuronset},\selvariableof{\larchitecture}}
		= \contractionof{
			\{\rencodingof{\lneuron} \, : \, \lneuron\in\nodesof{\larchitecture}/\inneuronset \} \cup \{\rencodingofat{\land}{\catvariableof{\land},\headvariableof{\lneuron}  \, : \, \lneuron\in\outneuronset}, \onehotmapofat{1}{\catvariableof{\land}}\}
		}{\catvariableof{\inneuronset},\selvariableof{\larchitecture}} \, . 
	\end{align*}
\end{proof}


% Combination of decompositions
By the commutation of contractions, we can further use \theref{the:neuronDecomposition} to decompose each tensor $\rencodingof{\lneuron}$ into connective and variable selecting components to get a sparse representation of a formula selecting neural network $\fsnn$.

%% Now as the definition!
%\begin{theorem}{Decomposition of formula selecting neural networks}\label{the:architectureDecomposition}
%	We have
%		\[ \rencodingof{\larchitecture} = \contractionof{\{\rencodingof{\lneuron} \, : \, \lneuron \in \larchitecture\}}{\catvariableof{\larchitecture},\shortcatvariables,\selvariableof{\larchitecture}} \]
%\end{theorem}
%\begin{proof}
%	By composition \theref{the:compositionByContraction}.
%	%\red{In addition: $X_{\larchitecture}$ specifying the headneurons! }
%\end{proof}

%% Now as the definition!
%% Relation between $\lneuron$ and $\rencodingof{\larchitecture}$
%Another useful property of encoded formula selecting architecture, is that we can retrieve any neuron by a simple contraction, as we show next.
%
%\begin{theorem}\label{the:formulaRetrieval}
%	Any neuron $\lneuron\in\larchitecture$ is retrieved by the contraction 
%		\[ \lneuron = \contractionof{\rencodingof{\larchitecture},\onehotmapof{1}[X_{\lneuron}]}{X\cup Z} \, . \]
%\end{theorem}
%\begin{proof}
%	First use the head neutralization property (Corollary~\ref{cor:onesHead}) in a parent stripping argument.
%	Then we are left with an architecture with $\lneuron$ being the only output neuron and use Corollary~\ref{cor:rhoToNormal} (we have $\restrictionofto{\mathrm{Id}}{[2]}=\onehotmapof{1}$).
%\end{proof}

% Alternative: Headneuron retrieval
%In case of multiple output neurons, the retrieval needs to be performed separately as in \theref{the:formulaRetrieval}, since contracting basis vectors $\onehotmapof{1}$ at multiple output neurons will retrieve the conjunction of those output neurons.




%\subsect{Skeleton Expressions}
%
%When only allowing for argument selections at the leaf level of the network, we get a skeleton expression.
%
%\begin{definition}\label{def:skeleton}
%	A skeleton expression
%		\[ \skeleton(\placeholderof{0},\ldots,\placeholderof{\selorder-1}) \]
%	is a composition of atom and connective selector maps, which are denoted by placeholders $\placeholderof{\selenumerator}$, where $\selenumerator\in[\selorder]$..
%	Each placeholder has a by $\selindexof{\selenumerator}$ enumerated list $\candidatelistof{\selenumerator}$ with cardinality $\seldimof{\selenumerator}= \cardof{\candidatelistof{\selenumerator}}$ of possible symbols denoting atoms or connectives to be placed in at this position.
%	This defines a map
%		\[ \skeleton : \left(\facstates\right) \times \left(\secfacstates\right) \rightarrow \{0,1\} \]
%	where $\skeleton(\atomindices,\selindices)$ denotes the formula given the selection of placeholders by $\selindices$, which is evaluated at the atoms $\atomindices$.
%\end{definition}

%\begin{definition}
%	A skeleton expression is a formula
%		\[ \skeleton(\placeholderof{0},\ldots,\placeholderof{\selorder-1}) \]
%	where instead of atoms and connectives there are placeholders $\placeholderof{\selenumerator}$, where $\selenumerator\in[\selorder]$.
%	Each skeleton has for each placeholder $\placeholderof{\selenumerator}$ a set $\candidatelistof{\selenumerator}$ of candidate atoms to be plugged in the placeholders.
%	We denote its cardinality to be $\seldimof{\selenumerator}= \cardof{\candidatelistof{\selenumerator}}$ and enumerate the elements $\placeholderof{\selenumerator}_{\selindexof{\selenumerator}}$ in each candidates list by an index $\selindexof{\selenumerator}\in[\seldimof{\selenumerator}]$.
%\end{definition}






%% ANOTHER EXAMPLE:





%\begin{definition}
%	Given a skeleton expression, the skeleton tensor is the map from the parameter space to the space of formula tensors, defined by
%	\begin{align}
%		 \skeletontensor : \rr^{\seldim} \rightarrow  \modelspace \quad , \quad
%		 \skeletontensor\left( \bigotimes_{\selenumeratorin}\unitvectoratof{\selenumerator}{\selindexof{\selenumerator}} \right) = \rencodingof{\skeletonof{\placeholderof{\selenumerator}_{\selindexof{\selenumerator}}\, : \, \selenumeratorin}}
%	\end{align}
%\end{definition}
%
%Using the canonical duality of tensors as maps and elements of tensor spaces, we can reinterpret is as a tensor \red{Domain representation of skeleton map}
%	\[ \skeletontensor \in \bigotimes_{\selenumerator\in[\selorder]} \rr^{\seldimof{\selenumerator}} \otimes  \modelspace  \, , \]
%which is the superposed formula tensor to a skeleton based parametrization.
%In the following, we investigate how to efficiently represent the skeleton tensor $\rencodingof{\skeleton}$ as a tensor network.









\sect{Application of Formula Selecting Networks}

There are two main applications of formula selecting networks.
First, when contracting the selection variables with a weight tensor we get a weighted sum of the parametrized formulas.
Second, when contracting the categorical variables with a distribution or a knowledge base, we get a tensor storing the satisfaction rates respectively the world counts of the parametrized formulas.

\subsect{Representation of selection encodings}

The main application of formula selecting networks in this work is the efficient representation of selection encodings.
This will be exploited in the sparse representation of exponential families by energies and in structure learning.
In the next lemma we will show the correspondence of formula selecting networks and selection encodings.

\begin{lemma}\label{lem:relToSelFSN}
	Given a set $\{\formulaof{\selindexlist} : \selindexlist\in\selstates\}$ of propositional formulas we define the statistic
		 \[ \formulaset : \catindices \rightarrow (\formulaof{\selindexlist}(\catindices))_{\selindexlist} \, . \]
	and the formula selecting map
		\[ \fselectionmap: \catindices , \selindexlist \rightarrow \formulaof{\selindexlist} (\catindices) \, . \]
	Then 
		\[ \sencodingofat{\formulaset}{\shortcatvariablelist, \shortselvariablelist} = \fselectionmap\left[\shortcatvariablelist, \shortselvariablelist \right] \, .  \]
\end{lemma}
\begin{proof}
	For any indices $\shortselindices\in\selstates$ and $\shortcatindices\in\atomstates$ we have
	\begin{align*}
		\sencodingofat{\formulaset}{\shortcatvariablelist=\shortcatindices, \shortselvariablelist=\shortselindices}
		=  \formulaof{\selindexlist}(\catindices) =  \fselectionmap\left[\shortcatvariablelist=\shortcatindices, \shortselvariablelist=\shortselindices \right] \, . 
	\end{align*}
\end{proof}

%% Reason for basis encodings and selection encodings.
Technically, basis encodings have been exploited to derive decompositions based on basis calculus.
Selection encodings on the other hand enable the application of formula selecting networks as superpositions of formulas.



\subsect{Efficient Representation of Formulas}

% Exponentially many formulas represented by linear demand
Formula Selecting Neural Networks are means to represent exponentially many formulas with linear (in sum of candidates list lengths) storage.
Their contraction with probability tensor networks, is thus a batchwise evaluation of exponentially many formulas.
This is possible due to redundancies in logical calculus due to modular combinations of subformulas.

% Retrieve functions
We can retrieve specific formulas by slicing the selection variables, i.e. for $\selindices$ we have
	\[ \exformula_{\selindices}[\shortcatvariables] = \fselectionmapat{\shortcatvariables,\selvariable=\selindices} \, .  \]

In a tensor network diagram we depict this by
\begin{center}
	\begin{tikzpicture}[thick, scale=0.35] % , baseline = -3.5pt

\node[anchor=east] (text) at (-3,0) {${\exformula_{\selindices}} \quad\quad {=}$};

\drawatomindices{0}{-4}
\draw (-1,3) rectangle (5, -3);
\node[anchor=center] (text) at (2,0) {$\bencodingof{\fselectionmap}$};

\draw[->] (2,3)--(2,5) node[midway,right] {\tiny $\headvariableof{\fselectionmap}$};
\draw (1,5) rectangle (3,7);
\node[anchor=center] (text) at (2,6) {$\tbasis$};

\draw[<-] (5,-2.25)--(7,-2.25) node[midway,below] {\tiny $\selvariableof{0}$}; 
\draw[<-] (5,-0.5)--(7,-0.5) node[midway,below] {\tiny $\selvariableof{1}$}; 
\node[anchor=center] (text) at (6,1) {$\vdots$};
\draw[<-] (5,2.25)--(7,2.25) node[midway,above] {\tiny $\selvariableof{\selorder\shortminus1}$};

\draw (7,1.5) rectangle (9,3);
\node[anchor=center] (text) at (8,2.25) {$\onehotmapof{\selindexof{\selorder\shortminus1}}$};

\draw (7,0.25) rectangle (9,-1.25);
\node[anchor=center] (text) at (8,-0.5) {$\onehotmapof{\selindexof{1}}$};

\draw (7,-1.5) rectangle (9,-3);
\node[anchor=center] (text) at (8,-2.25) {$\onehotmapof{\selindexof{0}}$};

\end{tikzpicture}
\end{center}

% Interpretation by dynamic programming
Another perspective on the efficient formula evaluation by selection tensor networks is dynamic computing.
Evaluating a formula requires evaluations of its subformulas, which are done by subcontractions and saved for different subformulas due to the additional selection legs.

% Storage problem of solutions
However, we need to avoid contracting the tensor with leaving all selection legs open, since this would require exponential storage demand.
% Sparse algorithm
We can avoid this storage bottleneck by contraction of parameter cores $\canparam$ with efficient network decompositions along the selection variables. %extending the contractions by additional cores leaving less variable legs open.

% Gibbs sampling
In Gibbs Sampling (\algoref{alg:Gibbs}), one can use the energy-based approach to queries \theref{the:energyContractionQueries}, and contract basis vectors on all but one selection variables.

%\red{This is the case when contracting gradients of the parameter tensor networks in alternating least squares approaches.
%Other methods avoiding the bottleneck can be constructed by MCMC sampling, for example Gibbs Sampling.
%Here we only need to vary local components of the formula reflected in keeping only single variable legs open.}



\subsect{Batch contraction of parametrized formulas}

Given a set $\formulaset$ of formulas, we build a formula selecting network parametrizing the formulas.
The contraction 
\begin{align*}
	\contractionof{\extnet,\fselectionmap}{\shortselvariables} 
\end{align*}
is a tensor containing the contractions of the formulas $\formulaof{\shortselindices}$ with an arbitrary tensor network $\extnet$ as
\begin{align*}
	\sbcontraction{\extnet,\formulaof{\shortselindices}} = \sbcontractionof{\extnet,\fselectionmap}{\shortselvariables=\shortselindices} \, . 
\end{align*}


\subsect{Average contraction of parametrized formulas}

We show in the next two examples, how a full contraction of the formula selecting map with a probability distribution or a knowledge base can be interpreted.

\begin{example}[Average satisfaction of formulas]
	The average of the formula satisfactions in $\formulaset$ giben a probability tensor $\probtensor$ is 
		\[ \frac{1}{\prod_{\selenumeratorin}\seldimof{\selenumerator}} \cdot \sbcontraction{\probtensor,\sencodingof{\formulaset}} \, . \]
\end{example}


\begin{example}[Deciding whether any formula is not contradicted]
	For example: We want to decide, whether there is a formula in $\formulaset$ not contradicted by a Knowledge base $\kb$.
	This is the case if and only if 
		\[ \sbcontraction{\kb,\sencodingof{\formulaset}} = 0 \, .  \]
	We use \lemref{lem:relToSelFSN} to get that $\sencodingof{\formulaset}=\fselectionmap$.
	When the formulas are representable in a folded scheme, we find tensor network decompositions of $\fselectionmap$ and exploit them along efficient representations of $\kb$ in an efficient calculation of $\sbcontraction{\kb,\sencodingof{\formulaset}} $.
	This is further equal to 
		\[ \kb \models \lnot \left( \bigvee_{\exformula\in\formulaset} \exformula\right) \, . \]
\end{example}



%\subsect{Neuro-Symbolic Architectures}
%
%%% Neuro-Symbolic Architecture
%We understand selector tensor networks as a neuro-symbolic architecture, where the selector variables are understood as parameters and the processed variables as neural activation variables.
%The orientation of the tensor network organizes the variables in layers.




\sect{Examples of formula selecting neural networks}




\subsect{Correlation}


For example (see Figure \ref{fig:AndSupFTDecomposition}) consider the logical neuron with single activation candidate $\{\land\}$ and two variable selectors selecting $\catorder$ atomic variables $\shortcatvariables$.
The expressivity of this network is the set of all conjunctions of the atoms
	\[ \{\catvariableof{\atomenumerator} \land \catvariableof{\secatomenumerator} \, : \, \atomenumerator,\secatomenumerator\in[\atomorder] \} \]


% Covariance measure
Contracting with a probability distribution, we use the tensor
	\[ \hypercoreat{\selvariableof{\vselectionsymbol,0},\selvariableof{\vselectionsymbol,1}} = \sbcontractionof{\fsnn}{\selvariableof{\vselectionsymbol,0},\selvariableof{\vselectionsymbol,1}} \]
to read of covariances as
	\[ \mathrm{Cov}(\catvariableof{\atomenumerator},\catvariableof{\secatomenumerator}) = \hypercoreat{\selvariableof{\vselectionsymbol,0}=\atomenumerator,\selvariableof{\vselectionsymbol,1}=\secatomenumerator}  - 
	\hypercoreat{\selvariableof{\vselectionsymbol,0}=\atomenumerator,\selvariableof{\vselectionsymbol,1}=\atomenumerator}  \cdot \hypercoreat{\selvariableof{\vselectionsymbol,0}=\secatomenumerator,\selvariableof{\vselectionsymbol,1}=\secatomenumerator} \, .  \]
	
	
	

%	\[ \skeleton = \placeholderof{1} \land \placeholderof{2} \]
%with the candidates for each placeholder being a set of $\atomorder$ atoms.

\begin{figure}[h]
\begin{center}
	\begin{tikzpicture}[thick, scale=0.35] % , baseline = -3.5pt

\drawatomindices{0}{-4}
\draw (-1,-1) rectangle (5, -3);
\node[anchor=center] (text) at (2,-2) {$\bencodingof{\atomicformulaof{\selindexof{1}} \land \atomicformulaof{\selindexof{2}}}$};

\draw[->] (2,-1)--(2,1) node[midway,right] {\tiny ${\headvariableof{\selindexof{1}} \land \headvariableof{\selindexof{2}}}$};

\draw[<-] (5,-2.5)--(7,-2.5) node[midway,below] {\tiny $\selvariable_{1}$};
\draw[<-] (5,-1.5)--(7,-1.5) node[midway,above] {\tiny $\selvariable_{0}$};

\draw (7,-1) rectangle (9, -3);
\node[anchor=center] (text) at (8,-2) {$\canparam$};


		
\node[anchor=center] (text) at (12,-2) {${=}$};


\begin{scope}[shift={(17,8)}]
	\begin{scope}[shift={(0,-10)}]

		\draw[->] (5.5,5) -- (5.5,7) node[midway, right] {\tiny ${\headvariableof{\selindexof{1}} \land \headvariableof{\selindexof{2}}}$};
		\draw (1,3) rectangle (10, 5);
		\node[anchor=center] (text) at (5.5,4) {$\bencodingof{\land}$};

			
		% SelectorCores
		\draw[->] (2,1) -- (2,3) node[midway, left] {\tiny ${\headvariableof{\selindexof{2}}}$};
		\draw (-1,1) rectangle (5, -1);
		\node[anchor=center] (text) at (2,0) {$\selectorcoreof{2}$};
		\draw (5,0) -- (12,0);
		\draw[<-] (12,0) -- (14,0) node[midway, above] {\tiny $\selvariable_{1}$};
		\begin{scope}[shift={(0,-2)}]
			\draw[<-] (0,1)--(0,-3) node[midway,left] {\tiny $\catvariableof{0}$}; 
			\draw[<-] (1.5,1)--(1.5,-3) node[midway,left] {\tiny $\catvariableof{1}$}; 
			\node[anchor=center] (text) at (3,0) {$\cdots$};
			\draw[<-] (4,1)--(4,-3) node[midway,right] {\tiny $\catvariableof{\atomorder\shortminus1}$}; 
		\end{scope}


		\draw (9,-1) -- (9,1);
		\draw[->] (9,1) -- (9,3) node[midway, left] {\tiny ${\headvariableof{\selindexof{1}}}$};
		\draw (6,-3) rectangle (12, -1);
		\node[anchor=center] (text) at (9,-2) {$\selectorcoreof{1}$};
		\draw[<-] (12,-2) -- (14,-2) node[midway, above] {\tiny $\selvariable_{0}$};
		\drawatomindices{7}{-4}	
		
		% ParameterCores
		\draw (14,1) rectangle (16, -3);
		\node[anchor=center] (text) at (15,-1) {$\canparam$};
		
		\begin{scope}[shift={(-3.5,8)}]
			\draw[fill] (7.5,-15) circle (0.15cm);
			\draw[] (7.5,-15) to[bend left=25] (3.5,-13);
			\draw[] (7.5,-15) to[bend right=25] (10.5,-13);

			\draw[fill] (9,-15.25) circle (0.15cm);
			\draw[] (9,-15.25) to[bend left=25] (5,-13);
			\draw[] (9,-15.25) to[bend right=25] (12,-13);

			\draw[fill] (11.5,-15) circle (0.15cm);
			\draw[] (11.5,-15) to[bend left=25] (7.5,-13);
			\draw[] (11.5,-15) to[bend right=25] (14.5,-13);

			\drawatomindices{7.5}{-16}

		\end{scope}

	\end{scope}
\end{scope}

\end{tikzpicture}
\end{center}
\caption{Superposition of the encoded formulas $\rencodingof{\atomicformulaof{\selindexof{1}} \land \atomicformulaof{\selindexof{2}}}$ with weight $\canparam_{\selindexof{1} \selindexof{2}}$}
\label{fig:AndSupFTDecomposition}
\end{figure}



\subsect{Conjunctive and Disjunctive Normal Forms}%\label{sec:CNFasFormulaSelection}

% Architecture
\red{
We can represent any propositional knowledge base by the following scheme:
Literal selecting neurons are logical neurons with connective identity/negation (selecting positive/negative literal) and selecting neurons select for each an atom.
The single output neuron represents the disjunction, respectively the conjunction, combining the literal selecting neurons.
The number of neurons defined by the maximal clause size plus one.
Smaller clauses can be covered when adding False as a possible choice (The respective neuron has to choose the identity, otherwise the full clause will be trivial).
This architecture will be discussed in more detail in \charef{cha:approximation} as $\cpformat$ selecting networks.
% Parameter
The parameter core is in the basis $\cpformat$ format and each slice selects a clause of the knowledge base.
In combination with polynomial decompositions, which will be provided in \charef{cha:networkRepresentation}, one can exploit this architecture to find sparse formula decompositions.
%When taking the slice values to infinity (e.g. by an annealing procedure), the represented member of the exponential family converges to the uniform distribution of the models of the knowledge base.
}

% Representation by selection tensor networks
\begin{remark}[Minterms and Maxterms]
	All minterms and maxterms can be represented by a two layer selection tensor networks without variable selection in two layers.
	The bottom layer has an $\lnot/\mathrm{Id}$ connective selection neuron to each atom and the upper layer consists of a single $\atomorder$ary conjunction.
\end{remark}




\sect{Extension to variables of larger dimension}

While we here restricted on boolean variables, formula selecting networks can be extended to variables of larger cardinality.
\begin{itemize}
	\item Connective selecting tensors: Can encode arbitrary functions $h_{\selindex}$ of discrete variables, but need $\catvariableof{\cselectionmap}$ to be an enumeration of the states, in particular to be of dimension
	\[ \catdimof{\cselectionmap} = \cardof{ \cup_{\selindexin} \imageof{h_{\selindex}} } \, . \]
	\item Variable selecting tensors can be understood as specific cases of connective selecting tensors and can thus also be generalized in a straight forward manner by
	\[ \catdimof{\cselectionmap} = \cardof{ \cup_{\selindexin} \imageof{h_{\selindex}} } \, .  \]
	\item State selecting tensors are directly defined for larger dimensions
\end{itemize}


An example of such a more generic usage is a discretization scheme for continuous neurons.

\begin{example}[Discretization of a continuous neuron]
	Let there be a neuron by a map of weight vectors and input vectors to $\rr$, that is
		\[ \sigma( w, x) : \rr^{\catorder} \times \rr^{\catorder} \rightarrow \rr \, .\]
%	When $w \in \arbsetof{weight}\subset \rr^{\catorder}$ and $x \in \arbsetof{x}\subset \rr^{\catorder}$ have
	We restrict the weights to a subset $\arbsetof{weight}\subset\rr^{\catorder}$ and the input vectors to $\arbsetof{x}\subset\rr^{\catorder}$,
	If follows that
		\[ \cardof{\imageof{\restrictionofto{\sigma}{\arbsetof{weight}\times\arbsetof{x}}}} \leq \cardof{\arbsetof{weight}} \cdot \cardof{\arbsetof{x}} \, . \]
	To discretize the neuron, we use the subset encoding scheme of \defref{def:subsetEncoding} and define enumeration variables $\indvariableof{weight}$, $\indvariableof{x}$ and $\indvariableof{\sigma}$ enumerating $\arbsetof{weight}$, $\arbsetof{x}$ and $\imageof{\restrictionofto{\sigma}{\arbsetof{weight}\times\arbsetof{x}}}$, which are accompanied by respective index interpretation functions.
	Then the basis encoding of the discretized neuron is
	\begin{align*}
        \rencodingofat{\sigma}{\indvariableof{\sigma},\indvariableof{weight},\indvariableof{x}} \, .
        = \sum_{\indindexofin{weight},\indindexofin{x}}
		\onehotmapof{\invindexinterpretationofat{\sigma(\indexinterpretationofat{weight}{\indindexof{weight}},\indexinterpretationofat{x}{\indindexof{x}})}{\sigma}}{\indvariableof{\sigma}}
		\otimes \onehotmapofat{\indindexof{weight}}{\indvariableof{weight}}
		\otimes \onehotmapofat{\indindexof{x}}{\indvariableof{x}} \, .
    \end{align*}
	If the neuron is of the form
		\[ \sigma(w,x) = \psi(\sum_i w_i \cdot x_i)\]
	a decomposition into multiplication at each coordinate and summation of the results, with basis encodings for each, can be done.
	\red{Here the index interpretation variables are split into a selection enumerated by $i$ and each variable gets assigned to single cores in the decomposition.}
\end{example}


    \chapter{\chatextnetworkRepresentation}\label{cha:networkRepresentation}

Logic networks are graphical models with an interpretation by propositional logics.
We first distinguish between Markov Logic Networks, which are an approach to soft logics in the framework of exponential families, and Hard Logic Networks, which correspond with propositional knowledge bases.
Then we exploit non-trivial boolean base measures to unify both approaches by Hybrid Logic Networks, which are itself in exponential families.




%Markov Logic Networks are probability functions of truth assignments to logical functions.
%They respect propositional logic as hard constraints, but have beyond that freedom to shape probability distributions on possible situations.
%To capture these properties, we define them as graphical models with structure cores representing propositional logics and activation cores representing the specification of probability distributions.
% We in this part employ them to combine the probabilistic and the logical paradigm.


\sect{Markov Logic Networks}

Markov Logic Networks exploit the efficiency and interpretability of logical calculus as well as the expressivity of graphical models. 

\subsect{Markov Logic Networks as Exponential Families}

We introduce Markov Logic Networks in the formalism of exponential families (see \secref{sec:exponentialFamilies}).

\begin{definition}[Markov Logic Networks]
	Markov Logic Networks are exponential families $\mlnexpfamily$ with sufficient statistics by functions
		\[ \mlnstat : \atomstates \rightarrow \bigtimes_{\exformulain}[2] \subset \rr^{\cardof{\formulaset}} \]
	defined coordinatewise by propositional formulas $\exformulain$.
\end{definition}

% Binary Statistics as propositional formula


% Characterization of MLNs among exponential families: When choosing binary features
Since the image of each coordinate $\sstatcoordinateof{\selindex}$ is contained in $\ozset$, each is a propositional formulas (see \defref{def:formulas}).
%Conversely, any boolean feature $\sstatcoordinateof{\selindex}$ of an exponential family defines a propositional formula (see \defref{def:formulas}).
Thus, any exponential family of distributions of $\atomstates$, such that $\imageof{\sstatcoordinateof{\selindex}}\subset\ozset$ for all $\selindexin$ is a set of Markov Logic Networks with fixed formulas.

The sufficient statistics consistent in a map $\formulaset$ of formulas brings the following advantages:
\begin{itemize}
	\item Numerical Advantage: The sufficient statistics is decomposable into logical connectives. 
	If the formulas are sparse (in the sense of limited number of connectives necessary in their representation), this gives rise to efficient tensor network decompositions of the basis encoding.
	\item Statistical Advantage: Since each formula is Boolean valued, the coordinates of the sufficient statistic are Bernoulli variables. 
	Due to their boundedness, they and their averages (by Hoeffdings inequality) are sub-Gaussian variables with favorable concentration properties (absence of heavy tails).
\end{itemize}

\begin{remark}[Alternative Definitions]
	We here defined MLNs on propositional logic, while originally they are defined in FOL.
	The relation of both frameworks will be discussed further in \charef{cha:folModels}.
\end{remark}



\subsect{Tensor Network Representation}

Based on the previous discussion on the representation of exponential families by tensor networks in \secref{sec:exponentialFamilies} we now derive a representation for Markov Logic Networks.

\subsubsect{Basis encodings for distributions}

\begin{theorem}[Basis encodings for Markov Logic Networks]\label{the:mlnTensorRep}
	A Markov Logic Network to a set of formulas $\formulaset = \{\enumformula \, : \, \selindexin\}$ is represented as
	\begin{align*}
		\mlnprobat{\shortcatvariables} = 
		\normalizationof{\{\enumformulaccwith : \selindexin \} \cup \{\enumformulaacwith : \selindexin \}
		}{\shortcatvariables}
	\end{align*}
	where we denote for each $\selindexin$ an activation core
	\begin{align*}
		\enumformulaac\left[\indexedheadvariableof{\selindex}\right]
		= \begin{cases}
			1 & \text{for} \quad \headindexof{\selindex} = 0 \\
			\expof{\canparamat{\indexedselvariable}} & \text{for} \quad \headindexof{\selindex}  = 1
		\end{cases}  \, .
	\end{align*}
%	\begin{align*}
%		\enumformulaacwith
%		= \begin{bmatrix} 1 \\
%		 \expof{\canparamat{\indexedselvariable}}
%		 \end{bmatrix}[\enumformulavar] \, .
%	\end{align*}
\end{theorem}
\begin{proof}
	Markov Logic Networks are exponential families, which base measure is trivial and which statistic consist of boolean features.
	We apply the tensor network decomposition of more generic exponential families \theref{the:expFamilyTensorRep} to this case and get
	\begin{align*}
        \mlnprobat{\shortcatvariables} =
        \normalizationof{\{\onesat{\shortcatvariables}\}
		\cup \{\bencodingofat{\sstatcoordinateof{\selindex}}{\sstatcatof{\selindex},\shortcatvariables} \, : \, \selindexin\}
		\cup\{\enumformulaacwith\, : \, \selindexin\}}{\shortcatvariables} \, .
    \end{align*}
	While the base measure tensor is trivial, it can be ignored in the contraction.
	Since the image of each feature $\enumformula$ is in $[2]$, we choose the index interpretation function by the identity $\indexinterpretation : [2] \rightarrow \ozset$ and get
%	We further choose the standard index interpretation for booleans (see \secref{sec:booleanEncoding}) and get
	\begin{align*}
		\enumformulaac\left[\indexedheadvariableof{\selindex}\right]
		&= \expof{\canparamat{\indexedselvariable} \cdot \indexinterpretationofat{\selindex}{\headindexof{\selindex}} }
		= \expof{\canparamat{\indexedselvariable} \cdot \headindexof{\selindex}} \\
		&= \begin{cases}
			1 & \text{for} \quad \headindexof{\selindex} = 0 \\
			\expof{\canparamat{\indexedselvariable}} & \text{for} \quad \headindexof{\selindex}  = 1
		\end{cases}
	\end{align*}
\end{proof}

\begin{figure}[t!]
\begin{center}
	\begin{tikzpicture}[thick, scale=0.35] % , baseline = -3.5pt

\drawundiratomindices{0}{-4}
\draw (-2,-1) rectangle (6, -3);
\node[anchor=center] (text) at (2,-2) {$\expof{\canparamat{\indexedselvariable}\cdot\enumformula}$};

		
\node[anchor=center] (text) at (10,-2) {${=}$};


\begin{scope}[shift={(15,-2)}]

		\draw (-0.5,3) rectangle (4.5, 5);
		\node[anchor=center] (text) at (2,4) {$\enumformulaac$};

		\drawvariabledot{2}{2.25}
		\draw[] (2,2.25) -- (2,3);
		\draw[->] (2,1) -- (2,2.5) node[midway, left] {\tiny $\headvariableof{\selindex}$};
		
		\draw (-1,1) rectangle (5, -1);
		\node[anchor=center] (text) at (2,0) {$\bencodingof{\enumformula}$};

		\drawatomindices{0}{-2}

\end{scope}

\end{tikzpicture}
\end{center}
\caption{Factor of a Markov Logic Network to a formula $\enumformula$, represented as the contraction of a computation core $\enumformulacc$ and an activation core $\enumformulaac$.
	While the computation core $\enumformulacc$ prepares based on basis calculus a categorical variable representing the value of the statistic formula $\enumformula$ dependent on assignments to the distributed variables, the activation core multiplies an exponential weight to coordinates satisfying the formula.
}
\label{fig:mlnFactor}
\end{figure}

% Graphical model representation
\theref{the:mlnTensorRep} provides a decomposition of markov logic networks by a tensor network of computation cores $\bencodingof{\enumformula}$ and accompanying activation cores $\enumformulaac$.
Since the head variable $\headvariableof{\selindex}$ appears exclusively in these pairs, we can contract each computation core with the corresponding activation core to get a factor, see \figref{fig:mlnFactor}.
With this we get the decomposition
\begin{align*}
	\mlnprobat{\shortcatvariables}
	= \normalizationof{\{\expof{\canparamat{\indexedselvariable}\cdot\enumformulaat{\shortcatvariables}} \, : \, \selindexin\}}{\shortcatvariables} \, .
\end{align*}
More precisely, this transformation of the decomposition holds by \theref{the:splittingContractions} to be shown in \charef{cha:messagePassing}, stating that the contraction of computation and activation cores can be performed before the global contraction of the result. 

% Sparsification by trivial variables
While in the decomposition of \theref{the:mlnTensorRep} the basis encodings of the features carry all distributed variables $\shortcatvariables$, we now seek towards sparser decompositions.
To each $\selindexin$ we denote by $\nodesof{\selindex}$ the maximal subset of $[\catorder]$ such that there is a reduced function
$\tilde{\formula}_{\selindex} : \bigtimes_{} \rightarrow [2]$
with
\begin{align*}
	\enumformulaat{\shortcatvariables}
	= \contractionof{\tilde{\formula}_{\selindex}[\catvariableof{\nodesof{\selindex}}]}{\shortcatvariables} \, .
\end{align*}
We often account for such situations of sparse formulas, when $\enumformula$ has a syntactical decomposition involving only the atomic variables $\nodesof{\selindex}$.
As a consequence we have
\begin{align*}
	\bencodingofat{\enumformula}{\shortcatvariables}
	= \bencodingofat{\tilde{\formula}_{\selindex}}{\catvariableof{\nodesof{\selindex}}}
\end{align*}
and
\begin{align*}
	\mlnprobat{\shortcatvariables}
	= \normalizationof{\{\expof{\canparamat{\indexedselvariable}\cdot\tilde{\formula}_{\selindex}[\catvariableof{\nodesof{\selindex}}]} \, : \, \selindexin\}}{\shortcatvariables} \, .
\end{align*}
Thus, any markov logic network has a sparse representation by a markov network on the graph
\begin{align*}
	\graphof{\formulaset} = ([\catorder],\{\nodesof{\selindex} \, : \, \selindexin\}) \, .
\end{align*}
This sparsity inducing mechanism is analogous to the decomposition of probability distributions based on conditional independence assumptions, when understanding each formula in the markov logic network as an introduced dependency among the affected variables $\nodesof{\selindex}$.


\begin{figure}[t]
\begin{center}
	\begin{tikzpicture}[scale=0.35, thick, yscale=-1] % , baseline = -3.5pt

\draw (-5,-3) rectangle (1, -5);
\node[anchor=center] (text) at (-2,-4) {$\expof{\canparamat{0}\cdot\formulaof{0}}$};

\draw (3,-3) rectangle (9, -5);
\node[anchor=center] (text) at (6,-4) {$\expof{\canparamat{1}\cdot\formulaof{1}}$};


\draw[->] (0,1)--(0,-1)  node[midway,left] {\tiny $\catvariableof{a}$}; 
\drawvariabledot{0}{-1}
\draw[->] (0,-1) to[bend right=25] (-4,-3);
\draw[->] (0,-1) to[bend left=25] (4,-3);

\draw[->]  (2,1)--(2,-1) node[midway,right] {\tiny $\catvariableof{b}$}; 
\drawvariabledot{2}{-1}
\draw[->] (2,-1) to[bend right=25] (-2,-3);
\draw[->] (2,-1) to[bend left=25] (6,-3);

\draw[->] (4,1)--(4,-1) node[midway,right] {\tiny $\catvariableof{c}$}; 
\drawvariabledot{4}{-1}
\draw[->] (4,-1) to[bend right=25] (0,-3);
\draw[->] (4,-1) to[bend left=25] (8,-3);


\node[anchor=center] (text) at (12.5,-2) {${=}$};


\begin{scope}[shift={(20,0)}]

\draw[->] (0,1)--(0,-1)  node[midway,left] {\tiny $\catvariableof{a}$}; 
\draw[->]  (3,1)--(3,-1) node[midway,right] {\tiny $\catvariableof{b}$}; 
\draw[->] (6,1)--(6,-1) node[midway,right] {\tiny $\catvariableof{c}$}; 

	
\draw (-1,-1) rectangle (4, -3);
\node[anchor=center] (text) at (1.5,-2) {$\rencodingof{\lor}$};

\draw[->] (1.5,-3) --(1.5,-4.5) node[midway,right]{\tiny $\catvariableof{a \lor b}$};
\drawvariabledot{1.5}{-4.5}
\draw[->] (1.5,-4.5) --(1.5,-6) ;

\draw[] (1.5,-4.5) -- (0,-4.5);

\draw[]  (-7, -3.5) rectangle (0, -6.5);
\node[anchor=center,] (text) at (-3.5,-5) {$\begin{bmatrix} 
1 \\
\expof{\canparamat{0}}
\end{bmatrix}$};


\draw (5,-1) rectangle (7, -3);
\node[anchor=center] (text) at (6,-2) {$\rencodingof{\lnot}$};

\draw[->] (6,-3) --(6,-4.5) node[midway,right]{\tiny $\catvariableof{\lnot c}$};
\drawvariabledot{6}{-4.5}
\draw[->] (6,-4.5) --(6,-6);	
	
	
\draw (0.5,-6) rectangle (6.5,-8);
\node[anchor=center] (text) at (3.5,-7) {$\rencodingof{\lor}$};
	
\draw[<-] (4,-9.5) -- (4,-8) node[midway,right] {\tiny $\catvariableof{a \lor b \lor \lnot c}$};

\drawvariabledot{4}{-9.5}
\draw[] (4,-9.5) -- (2.75,-9.5);

\draw[]  (-4.25, -8.5) rectangle (2.75, -11.5);
\node[anchor=center,] (text) at (-0.75,-10) {$\begin{bmatrix} 
1 \\
\expof{\canparamat{1}}
\end{bmatrix}$};



\end{scope}

\end{tikzpicture}
\end{center}
\caption{Example of a decomposed Markov Network representation of a Markov Logic Network with formulas $\{\formulaof{0} = a\lor b, \formulaof{1} = a \lor b \lor \lnot c\}$.
	Since both formulas share the subformula $a\lor b$, their contracted factors have a representation by a connected tensor network.}
% Where $\actcoreofat{\enumformula}{\enumformulavar} =\begin{bmatrix} 1 & \expof{\weightof{\exformula}} \end{bmatrix}[\formulavar] $}
\label{fig:mlnDecRep}
\end{figure}


% Sparsity by decomposition
A further sparsity introducing mechanism is through exploiting redundancy in the computation of $\enumformula$, when a decomposition of the feature is known.
For the propositional formulas $\enumformula$ this amounts to a syntactic representation of the formula as a composition of logical connectives is available (see \figref{fig:mlnDecRep}). % Make a precise definition in logicRepresentation and link it!
In this case, we exploit the representation by tensor networks of the basis encodings (shown as \theref{the:compositionByContraction} in \charef{cha:basisCalculus})
Note, that this decomposition scheme introduces further auxiliary variables $\headvariable$ with deterministic dependence on the distributed variables $\shortcatvariables$.
Such variables are often refered to as hidden.

% Shared formulas
We can further exploit common syntactical structure in the formulas $\enumformula\in\formulaset$ to reduce the number of basis encodings of connectives.
This is the case, when the syntax graph of two or more formulas share a subgraph.
In that case, the respective syntax graph needs to be represented only once an can be incorporated into the decomposition of all formulas, which share this subgraph.
For an example see \figref{fig:mlnDecRep}, where the syntactical representation of the formula $\formulaof{0}$ is a subgraph of the syntactical representation of $\formulaof{1}$.


%Since any member of an exponential family is a Markov Network with tensors to each coordinate of the statistic, also Markov Logic Networks are Markov Networks.

%\begin{corollary}\label{cor:MLNasMN}
%	Given a set $\formulaset$ of formulas on atomic variables $\catvariableof{\nodes}$, we construct a $\graph=(\nodes,\edges)$, where $\nodes$ are decorated by the atoms and
%		\[ \edges = \{ \nodesof{\formula}: \formula\in\formulaset \} \, , \]
%	where by $\nodesof{\formula}$ we denote the minimal set such that there exists a tensor $\hypercoreat{\catvariableof{\nodesof{\formula}}}$ with
%		\[ \formulaat{\catvariableof{\nodes}} = \hypercoreat{\catvariableof{\nodesof{\formula}}} \otimes \onesat{\catvariableof{\nodes/\nodesof{\formula}}} \, . \]
%	Any Markov Logic Network $\mlnprobat{\shortcatvariables}$ is then a Markov Network given the graph $\graphof{\formulaset}$
%	$\{\expof{\canparamat{\indexedselvariable}\cdot\enumformula}
%\, :\,\selindexin\}$.
%\end{corollary}


% MLN as graphical models
%Markov Logic Networks are Markov Networks with the factors given in a restricted form from the weighted truth of a formula.
%Each formula is seen as a factor of the graphical model.

To summarize, there are two sparsity mechanisms, originating from graphical models and propositional syntax, providing sparse representations of markov logic network:
\begin{itemize}
	\item \textbf{Dependence Sparsity:} Formulas depend only on subsets of atoms.
		This exploits the main sparsity mechanism in graphical models, where factors in sparse representations depend only on a subset of variables.
%		The underlying assumptions of conditional independence loss generality.
	\item \textbf{Computation Sparsity:}
		When the features of an exponential family are compositions of smaller formulas, the computation core is decomposed into a tensor network of their basis encodings.
		This can be regarded as the main sparsity mechanism of propositional logics, where syntactical decompositions of formulas are exploited.
		Further, when the structure of the smaller formulas is shared among different features, the respective basis encodings need to be instantiated only once.
\end{itemize}


%%
%\red{
%	We can extend the set of variables, by including the hidden formulas, and get a Markov Network of the basis encodings of connectives and headcores.
%Here hidden variables are additional variables facilitating the decomposition, but not appearing in open variables of contractions when doing reasoning.
%One can then exploit redundancies and make sure that every subresult is computed just once, by dropping basis encodings with identical head functions.
%}




%\begin{theorem}[Selection encodings for Energy representation]
%	\red{More the definition of exponential families.}
%	The energy of Markov Logic Networks is the contraction
%		\[ \mlnenergy = \contractionof{\sencodingof{\formulaset},\canparam}{\shortcatvariables} \, . \]
%\end{theorem}


\subsubsect{Selection encodings for energy tensors}

%% Tensor Representation of MLN

As for generic exponential families, we can represent markov logic networks in terms of their energy tensors
%The energy tensor of an markov logic network is the contraction
\begin{align}
	\mlnenergy\left[\shortcatvariables\right]
	= \sum_{\selindexin} \canparamat{\indexedselvariable} \cdot \enumformulaat{\shortcatvariables} 
	= \contractionof{\sencodingofat{\formulaset}{\shortcatvariables,\selvariable},\canparamat{\selvariable}}{\shortcatvariables} \, .
\end{align}
The energy tensor provides an exponential representation of the distribution by
\begin{align}
	\mlnprobat{\shortcatvariables} = \normalizationof{\expof{\mlnenergy}}{\shortcatvariables} \, .
\end{align}

In case of a common structure of the formulas in a Markov Logic Network, formula selecting networks (see \charef{cha:formulaSelection}) can be applied to represent their energies.
% Energy representation
%The weighted sum of formulas is then the energy of the Markov Logic Network.
We represent the superposition of formulas as a contraction with a parameter tensor.
Given a factored parametrization of formulas $\exformula_{\selindices}$ with indices $\selindexof{\selenumerator}$ we have the superposition by the network representation:
\begin{center}
	\begin{tikzpicture}[thick, scale=0.35] % , baseline = -3.5pt

\node[anchor=east] (text) at (-3,0) {$\sum_{\parindexof{[\parorder]}\in\parstates} \canparamat{\selvariableof{[\parorder]}=\parindexof{[\parorder]}} {\exformula_{\parindexof{[\parorder]}}} \quad {=}$};

%\node[anchor=center] (text) at (0.5,-8) {$\mathrm{log}$};

%\drawatomcore{3.5}{-8}{$\rencodingof{\fselectionmap}$}
%\drawatomindices{3.5}{-12}	
%
%
%\drawatomcore{3.5}{-4}{$\canparam$}
%\drawparindices{3.5}{-8}	



\drawatomindices{0}{-4}
\draw (-1,3) rectangle (5, -3);
\node[anchor=center] (text) at (2,0) {$\rencodingof{\fselectionmap}$};

\draw[->] (2,3)--(2,5) node[midway,right] {\tiny $\catvariableof{\fselectionmap}$}; 
\draw (1,5) rectangle (3,7);
\node[anchor=center] (text) at (2,6) {$\tbasis$};

\draw[<-] (5,-2)--(7,-2) node[midway,below] {\tiny $\selvariableof{0}$}; 
\draw[<-] (5,-0.5)--(7,-0.5) node[midway,below] {\tiny $\selvariableof{1}$}; 
\node[anchor=center] (text) at (6,0.75) {$\vdots$};
\draw[<-] (5,2)--(7,2) node[midway,above] {\tiny $\selvariableof{\parorder\shortminus1}$}; 

\draw (7,3) rectangle (9,-3);
\node[anchor=center] (text) at (8,0) {$\canparam$};

\end{tikzpicture}
\end{center}


% Representation 
If the number of atoms and parameters gets large, it is important to represent the tensor ${\exformula_{\selindices}}$ efficiently in tensor network format and avoid contractions.
To avoid inefficiency issues, we also have to represent the parameter tensor $\canparam$ in a tensor network format to improve the variance of estimations (see \charef{cha:concentration}) and provide efficient numerical algorithms.

% Fail of full probability representation
However, when required to instantiate the probability distribution of a Markov Logic Network as a tensor network, we need to exponentiate and normate the energy tensor, a task for which basis encodings are required.
For such tasks, contractions of formula selecting networks are not sufficient and each formula with a nonvanishing weight needs to be instantiated as a factor tensor of a Markov Network. 






\subsect{Expressivity}\label{sec:MLNMaxMintermRep}

Based on Markov Logic Networks containing only maxterms and minterms (see \defref{def:clauses}), we now show that any positive probability distribution has a representation by a markov logic network.

\begin{theorem}\label{the:maximalClausesRepresentation}\label{the:mintermExpressivityMLN}
	Let there be a positive probability distribution
		 \[ \probat{\shortcatvariables} \in \bigotimes_{\atomenumeratorin}\rr^2 \, . \]
	Then the Markov Logic Network of minterms (see \defref{def:clauses})
		\[ \mintermformulaset = \{\mintermof{\atomindices} \, : \, \atomindices\in\atomstates \}\]
	with parameters %with nonzero weights at the maxterms indexed by $\atomindicesin$
		\[ \canparamat{\selvariableof{0}=\catindexof{0},\ldots,\selvariableof{\atomorder-1}=\catindexof{\atomorder-1}}% \weightof{\mintermof{\atomindices}} 
		= \ln \probat{\indexedcatvariables} \]
	coincides with $\probat{\shortcatvariables}$.

	Further, the Markov Logic Network of maxterms
		\[ \maxtermformulaset = \{\maxtermof{\atomindices} \, : \, \atomindices\in\atomstates \}\]
	with wparameters
		\[ \canparamat{\selvariableof{0}=\catindexof{0},\ldots,\selvariableof{\atomorder-1}=\catindexof{\atomorder-1}} %\weightof{\maxtermof{\atomindices}} 
		= - \ln\probat{\indexedcatvariables} \]
	coincides with $\probat{\shortcatvariables}$.
\end{theorem}
\begin{proof}
	It suffices to show, that in both cases of choosing $\formulaset$ by minterms or maxterms with the respective parameters
		\[ \mlnenergy =  \ln\probat{\shortcatvariables} \]
	and therefore
		\[ \mlnprobat{\shortcatvariables} 
		= \normalizationof{\expof{\mlnenergy}}{\shortcatvariables}
		=  \contractionof{\expof{\mlnenergy}}{\shortcatvariables}
		= \probat{\shortcatvariables}\, . \]
	
	In the case of minterms, we notice that for any $\atomindicesin$
		\[ \mintermof{\atomindices}[\shortcatvariables] = \onehotmapofat{\atomindices}{\shortcatvariables} \]
	and thus with the weights in the claim
		\[ \sum_{\atomindicesin} 
		\left( \ln \probat{\indexedcatvariables} \right) \cdot \mintermof{\atomindices}[\shortcatvariables]
		= \ln\probat{\shortcatvariables} \, .
		 \]

	For the maxterms we have analogously
		\[ \maxtermof{\atomindices}[\shortcatvariables] = \onesat{\shortcatvariables} - \onehotmapofat{\catindices}{\shortcatvariables} \]
	and thus that the maximal clauses coincide with the one-hot encodings of respective states.
	We thus have
	\begin{align*}
		& \sum_{\atomindicesin} 
		\left( - \ln \probat{\indexedcatvariables} \right) \cdot \maxtermof{\atomindices}[\shortcatvariables] \\
		& =
		\left(  \sum_{\nodes_0\subset [\atomorder]} 
		\left( - \ln \probat{\indexedcatvariables} \right) \cdot \onesat{\shortcatvariables} \right) \\
		& \quad + 
		\left(  \sum_{\nodes_0\subset [\atomorder]} 
		\left(  \ln \probat{\indexedcatvariables} \right) \cdot
		\onehotmapofat{\catindices}{\shortcatvariables} 
		\right) 
		 \\
		 & = \ln\probat{\shortcatvariables} + \lambda \cdot  \onesat{\shortcatvariables}\,,
	\end{align*}
	where $\lambda$ is a constant.
\end{proof}

We note, that there are $2^{\atomorder}$ maxterms and $2^{\atomorder}$ minterms, which would have to be instantiated by basis encodings to get a tensor network decomposition.
This large number of features originates from the generality of the representation scheme.
As a fundamental tradeoff, efficient representations come at the expense of a smaller expressivity of the representation scheme.


% Redundant parametrization
%In general, this representation is redundant, since any offset of the weight by $\lambda\cdot\ones$ results in the same distribution.
%However, the only $\bar{\canparam}$ are multiples of $\onesat{\shortcatvariables}$.

% Comparison with previous schemes
Theorem~\ref{the:maximalClausesRepresentation} is the analogue in Markov Logic to Theorem~\ref{the:tensorToMaxMinTerms}, which shows that any binary tensor has a representation by a logical formula, to probability tensors.
Here we require positive distributions for well-defined energy tensors.

% Markov Networks
Sparser representation formats based on the same principle as used in \theref{the:maximalClausesRepresentation} can be constructed to represent markov networks by markov logic networks.
Here, we can separately instantiate the factors by combinations of terms and clauses only involving the variables containted in the factor.

%\begin{remark}[Representation of Markov Networks]
% Composition of Markov Networks
%	If a probability distribution is representable as a Markov Network, we only need to activate clauses and terms, which variables are contained in factors.%
%	\red{Make a theorem out of that?}
%\end{remark}
%\subsect{Examples}

\subsect{Examples}

Let us now provide examples of markov logic networks.

\subsubsect{Distribution of independent variables}

We show next, the independent positive distributions are representable by tuning the $\atomorder$ weights of the atomic formulas and keeping all other weights zero.

\begin{theorem}\label{the:independentAtomicMLN}
	Let $\probat{\shortcatvariables}$ be a positive probability distribution, such that atomic formulas are independent from each other.
	Then $\probat{\shortcatvariables}$ is the Markov Logic Network of atomic formulas
		\[ \atomformulaset = \{\atomicformulaof{\catenumerator} \, : \, \catenumeratorin \} \]
	and parameters
		\[ \canparamat{\selvariable=\catenumerator} 
		= \lnof{\frac{
		\contractionof{\probtensor}{\catvariableof{\catenumerator}=1}
		}{
		\contractionof{\probtensor}{\catvariableof{\catenumerator}=0}
		}} \]
%	Any distribution such that the atom satisfaction is independent from each other is reproducable by a MLN with nonzero weights only for the atomic formulas.
\end{theorem}
\begin{proof}
%	Using the independent assumptions, the probability tensor factorizes into normed vectors to each atom, with are transformed atomic formulas (leaving out the neutral ones tensors).
%	We then find a weight to each atom such that the vector is reproduced by the contraction with the activation core.
	
	By Theorem~\ref{the:independenceProductCriterion} we get a decomposition 
		\[ \probat{\shortcatvariables} = \bigotimes_{\catenumeratorin} \probofat{\catenumerator}{\catvariableof{\catenumerator}} \,  \]
	where 
		\[ \probofat{\catenumerator}{\catvariableof{\catenumerator}} = \contractionof{\probtensor}{\catvariableof{\catenumerator}} \, . \]
	
	By assumption of positivity, the vector $\probofat{\catenumerator}{\catvariableof{\catenumerator}}$ is positive for each $\catenumeratorin$ and the parameter
		\[ \canparamat{\selvariable=\catenumerator} 
		= \lnof{\frac{
		\probofat{\catenumerator}{\catvariableof{\catenumerator}=1}
		}{
		\probofat{\catenumerator}{\catvariableof{\catenumerator}=0}
		}} \]
	well-defined.
	
	We then notice, that 
		\[ \expdistofat{(\{\atomicformulaof{\catenumerator}\},\canparamat{\selvariable=\catenumerator})}{\catvariableof{\catenumerator}} 
		= \probofat{\catenumerator}{\catvariableof{\catenumerator}}\]
	and therefore with the parameter vector of dimension $\seldim=\catorder$ defined as
		\[ \canparamat{\selvariable} = \sum_{\catenumeratorin} \canparamat{\selvariable=\catenumerator} \cdot \onehotmapofat{\catenumerator}{\selvariable}  \]
	we have
	\begin{align*}
	 	 \expdistofat{(\{\atomicformulaof{\catenumerator} \, : \, \catenumeratorin\},\canparam)}{\shortcatvariables} 
		& = \bigotimes_{\catenumeratorin} \expdistofat{(\{\atomicformulaof{\catenumerator}\},\canparamat{\selvariable=\catenumerator})}{\catvariableof{\catenumerator}} \\
		& = \bigotimes_{\catenumeratorin} \probofat{\catenumerator}{\catvariableof{\catenumerator}} \\
		& = \probat{\shortcatvariables} \, . 
	\end{align*}
\end{proof}

%In general, the statistic to an atomic formula measures the marginal distribution. -> To Parameter Estimation

% Failing to be positive -> Hybrid networks
In Theorem~\ref{the:independentAtomicMLN} we made the assumption of positive distributions.
If the distribution fails to be positive, we still get a decomposition into distributions of each variable, but there is at least one factor failing to be positive.
Such factors need to be treated by hybrid logic networks, that is they are base measure for an exponential family coinciding with a logical literal (see \secref{sec:hardNetworks}).

% Energy representation
All atomic formulas can be selected by a single variable selecting tensor, that is
	\[ \energytensorofat{(\{\atomicformulaof{\catenumerator} \, : \, \catenumeratorin\},\canparam)}{\shortcatvariables}
	= \contractionof{\vselectionmapat{\shortcatvariables,\selvariable},\canparamat{\selvariable}}{\shortcatvariables} \, .
	\]
	
% Holds also more generally for any formula! -> Place it earlier?
In case of negative coordiantes $\canparamat{\selvariable=\catenumerator}$ it is convenient to replace $\atomicformulaof{\catenumerator}$ by $\lnot\atomicformulaof{\catenumerator}$, in order to facilitate the interpretation.
The probability distribution is left invariant, when also replacing $\canparamat{\selvariable=\catenumerator}$ by $-\canparamat{\selvariable=\catenumerator}$.



\subsubsect{Boltzmann machines}

A Boltzmann machine is a distribution of boolean variables $\shortcatvariables$ depending on weight tensors %(see e.g. Chapter 43 in \cite{mackay_information_2003})
\begin{align*}
	W[\selvariableof{\vselectionsymbol,0},\selvariableof{\vselectionsymbol,1}] \in\rr^{\catorder}\otimes\rr^{\catorder} \quad \text{and} \quad b[\selvariableof{\vselectionsymbol,0}] \in\rr^{\catorder} \, .
\end{align*}
Its distribution is
\begin{align*}
	\probofat{W,b}{\shortcatvariables} = \normalizationof{\expof{\energytensorofat{W,b}{\shortcatvariables}}}{\shortcatvariables}
\end{align*}
where its energy tensor is
	\[ \energytensorofat{W,b}{\indexedshortcatvariables} =
	\sum_{\atomenumerator,\secatomenumerator \in [\atomorder]} 
		W[\selvariableof{\vselectionsymbol,0}=\atomenumerator, \selvariableof{\vselectionsymbol,1}=\secatomenumerator] \cdot \catindexof{\atomenumerator} \cdot \catindexof{\secatomenumerator} 
	+ \sum_{\atomenumerator\in[\atomorder]} b[\selvariableof{\vselectionsymbol,0}=\atomenumerator] \cdot \catindexof{\atomenumerator}\, . \]
We notice, that this tensor coincides with the energy tensor of a markov logic network with formula set
	\[ \formulaset = \{ \catvariableof{\atomenumerator} \Leftrightarrow \catvariableof{\secatomenumerator} \, : \, \atomenumerator,\secatomenumerator \in[\atomorder] \} 
	\cup \{ \catvariableof{\atomenumerator}\, : \, \atomenumeratorin \} \, \]
with cardinality $\atomorder^2+\atomorder$.
Each formula is in the expressivity of an architecture consisting of a single binary logical neuron selecting any variable of $\shortcatvariables$ in each argument and selecting connectives $\{\eqbincon,\lpasbincon\}$, where by $\lpasbincon$ we refer to a connective passing the first argument, defined for $\catindexofin{0}, \catindexofin{1}$ as 
	\[ \lpasbincon[\indexedcatvariableof{0},\indexedcatvariableof{1}] = \vselectionmapat{\indexedcatvariableof{0},\indexedcatvariableof{1},\selvariableof{\vselectionsymbol}=0} \, . \]
When we choose the canonical parameter as
	\[ \canparamat{\selvariableof{\cselectionsymbol},\selvariableof{\vselectionsymbol,0},\selvariableof{\vselectionsymbol,1}}
	= \onehotmapofat{0}{\selvariableof{\cselectionsymbol}} \otimes W[\selvariableof{\vselectionsymbol,0},\selvariableof{\vselectionsymbol,1}]
	+ \onehotmapofat{1}{\selvariableof{\cselectionsymbol}} \otimes b[\selvariableof{\vselectionsymbol,0}] \otimes  \onehotmapofat{0}{\selvariableof{\vselectionsymbol,1}} \, .
	\]
we have (see \figref{fig:boltzmannEnergy})
	\[ \energytensorofat{W,b}{\shortcatvariables} = 
	\contractionof{\fsnnat{\shortcatvariables,\selvariableof{\cselectionsymbol},\selvariableof{\vselectionsymbol,0},\selvariableof{\vselectionsymbol,1}},
		\canparamat{\selvariableof{\cselectionsymbol},\selvariableof{\vselectionsymbol,0},\selvariableof{\vselectionsymbol,1}}}{\shortcatvariables} \, . \]

Therefore, Boltzmann machines are specific markov logic networks with the statistic being biimplications between atoms and atoms itself.
Generic markov logic networks are more expressive than Boltzmann machines, by the flexibility to create further features by propositional formulas.
%We can use any binary logical connective and have an associated formula where we can put a weight on.

\begin{figure}[t]
\begin{center}
	\begin{tikzpicture}[thick, scale=0.35] % , baseline = -3.5pt

\drawatomindices{0}{-4}
\draw (-1,1) rectangle (5, -3);
\node[anchor=center] (text) at (2,-1) {$\sencodingof{\formulaset}$};

%\draw[->] (2,-1)--(2,1) node[midway,right] {\tiny ${\atomicformulaof{\parindexof{1}} \land \atomicformulaof{\parindexof{2}}}$}; 

\draw[<-] (5,0.5) -- (7,0.5) node[midway, above] {\tiny $\selvariableof{\cselectionsymbol}$};
\draw[<-] (5,-1)--(7,-1) node[midway,above] {\tiny $\selvariableof{\vselectionsymbol,1}$}; 
\draw[<-] (5,-2.5)--(7,-2.5) node[midway,below] {\tiny $\selvariableof{\vselectionsymbol,0}$}; 

\draw (7,1) rectangle (9, -3);
\node[anchor=center] (text) at (8,-1) {$\canparam$};


		
\node[anchor=center] (text) at (12,-2) {${=}$};


\begin{scope}[shift={(17,8)}]

% SkeletonCores

%% Would be required to match the formula tensor example, but would get messy!

%\drawatomindices{0}{0}
%\draw (-1,-1) rectangle (5, -3);
%\node[anchor=center] (text) at (1.5,-2) {$\skeleton^{\land}$};
%\drawatomindices{0}{-4}


%\drawatomindices{7}{0}
%\draw (6,-1) rectangle (12, -3);
%\node[anchor=center] (text) at (9,-2) {$\skeleton^{\lnot}$};
%\drawatomindices{7}{-4}

%\drawatomindices{3.5}{-4}	
%\draw (-1,-5) rectangle (12,-7);
%\node[anchor=center] (text) at (5.5,-6) {$\skeletoncoreof{\land}$};
%\drawatomindices{0}{-8}	
%\drawatomindices{7}{-8}	


	\begin{scope}[shift={(0,-10)}]
	
		\draw (4.5,7) rectangle (6.5, 9);	
		\node[anchor=center] (text) at (5.5,8) {$\onehotmapof{1}$};
		
		\draw[->] (5.5,5) -- (5.5,7) node[midway, right] {\tiny $\headvariableof{\lneuron}$};
		\draw (1,3) rectangle (10, 5);
		\node[anchor=center] (text) at (5.5,4) {$\rencodingof{\{\eqbincon,\lpasbincon \}}$};
		\draw (10,4) -- (12,4);
		\draw[<-] (12,4) -- (14,4) node[midway, above] {\tiny $\selvariableof{\cselectionsymbol}$};
			
		% SelectorCores
		\draw[->] (2,1) -- (2,3) node[midway, left] {\tiny $\headvariableof{\vselectionsymbol,0}$};
		\draw (-1,1) rectangle (5, -1);
		\node[anchor=center] (text) at (2,0) {$\selectorcoreof{1}$};
		\draw (5,0) -- (12,0);
		\draw[<-] (12,0) -- (14,0) node[midway, above] {\tiny $\selvariableof{\vselectionsymbol,1}$};
		\begin{scope}[shift={(0,-2)}]
			\draw[<-] (0,1)--(0,-3) node[midway,left] {\tiny $\catvariableof{0}$}; 
			\draw[<-] (1.5,1)--(1.5,-3) node[midway,left] {\tiny $\catvariableof{1}$}; 
			\node[anchor=center] (text) at (3,0) {$\cdots$};
			\draw[<-] (4,1)--(4,-3) node[midway,right] {\tiny $\catvariableof{\atomorder\shortminus1}$}; 
		\end{scope}


		\draw (9,-1) -- (9,1);
		\draw[->] (9,1) -- (9,3) node[midway, left] {\tiny $\headvariableof{\vselectionsymbol,1}$};
		\draw (6,-3) rectangle (12, -1);
		\node[anchor=center] (text) at (9,-2) {$\selectorcoreof{0}$};
		\draw[<-] (12,-2) -- (14,-2) node[midway, above] {\tiny $\selvariableof{\vselectionsymbol,0}$};
		\drawatomindices{7}{-4}	
		
		% ParameterCores
		\draw (14,5) rectangle (16, -3);
		\node[anchor=center] (text) at (15,1) {$\canparam$};
		
		\begin{scope}[shift={(-3.5,8)}]
			\draw[fill] (7.5,-15) circle (0.25cm);
			\draw[] (7.5,-15) to[bend left=25] (3.5,-13);
			\draw[] (7.5,-15) to[bend right=25] (10.5,-13);

			\draw[fill] (9,-15.25) circle (0.25cm);
			\draw[] (9,-15.25) to[bend left=25] (5,-13);
			\draw[] (9,-15.25) to[bend right=25] (12,-13);

			\draw[fill] (11.5,-15) circle (0.25cm);
			\draw[] (11.5,-15) to[bend left=25] (7.5,-13);
			\draw[] (11.5,-15) to[bend right=25] (14.5,-13);

			\drawatomindices{7.5}{-16}

		\end{scope}
	
	\end{scope}

\end{scope}

\end{tikzpicture}
\end{center}
\caption{Tensor network representation of the energy of a Boltzmann machine}
\label{fig:boltzmannEnergy}
\end{figure}


%where by $(\cdot,\cdot)|_{0}$
%To connect with the formalism of Boltzmann machines, let us identify the visible units of a Boltzmann machines with the atoms in a propositional theory.

%Boltzmann machines are then reproduced by taking $\atomorder^2+\atomorder$ many formulas, namely those measuring the correlations and the marginal distributions.
%To be more precise, the correlation between atom $\atomicformulaof{\atomenumerator}$ and $\atomicformulaof{\secatomenumerator}$ is measured by the satisfaction rate of the formula 
%	\[ \exformula_{\atomenumerator,\secatomenumerator} = \atomicformulaof{\atomenumerator} \leftrightarrow \atomicformulaof{\secatomenumerator}\]

%\begin{theorem}
%	Any Boltzmann machine over $\atomorder$ units with interaction matrix $U\in\rr^{\atomorder\times\atomorder}$ and potential term $b\in\rr^{\atomorder}$ (MacKay Book notation) is a MLN where the only nonzero weights are 
%		\[ \weightof{\atomicformulaof{\atomenumerator} } = b_{\atomenumerator} \quad, \quad \atomenumeratorin \]
%	and 
%		\[ \weightof{ \exformula_{\atomenumerator,\secatomenumerator} } = U_{\atomenumerator, \secatomenumerator} \quad , \quad \atomenumerator,\secatomenumerator \in [\atomorder]\] 
%\end{theorem}



















\sect{Hard Logic Networks}\label{sec:hardNetworks} % To be dropped in the unification with the MLN chapter

% Hard logic vs markov logic
While exponential families are positive distributions, in logics probability distributions can assign states zero probability.
As a consequence, Markov Logic Networks have a soft logic interpretation in the sense that violation of activated formulas have nonzero probability.
We here discuss their hard logic counterparts, where worlds not satisfying activated formulas have zero probability.

\subsect{The limit of hard logic}\label{sec:hardLogicLimit} % To be merged with the above

The probability function of Markov Logic Networks with positive weights mimiks the tensor network representation of the knowledge base, which is the conjunction of the formulas. 
The maxima of the probability function coincide with the models of the corresponding knowledge base, if the latter is satisfiable.
However, since the Markov Logic Network is defined as a normed exponentiation of the weighted formula sum, it is a positive distribution whereas uniform distributions among the models of a knowledge base assign zero probability to world failing to be a model.
Since both distributions are tensors in the same space to a factored system, we can take the limits of large weights and observe, that Markov Logic Networks indeed converge to normed knowledge bases.


\begin{lemma}
	For any satisfiable formula $\formulaat{\shortcatvariables}$ and a variable weight $\canparam\in\rr$, we have for $\canparam\rightarrow\infty$
	\begin{align*}
		\normalizationof{\expof{\canparam\cdot\formulaat{\shortcatvariables}}}{\shortcatvariables} \rightarrow \normalizationof{\exformula}{\shortcatvariables}
	\end{align*}
	and for $\canparam\rightarrow-\infty$
	\begin{align*}
		\normalizationof{\expof{\canparam\cdot\formulaat{\shortcatvariables}}}{\shortcatvariables} \rightarrow \normalizationof{\lnot\exformula}{\shortcatvariables} \, .
	\end{align*}
	Here we denote the understand the convergence of tensors as a convergence of each coordinate.
\end{lemma}
\begin{proof}
	We have 
	\begin{align*}
		\partitionfunctionof{\mlnparameters} = \left(\prod_{\atomenumeratorin} \catdimof{\atomenumerator}\right) - \contraction{\exformula} + \contraction{\exformula} \cdot \expof{\canparam}
	\end{align*}
	and therefore for any $\shortcatindices\in\atomstates$ with $\formulaat{\indexedshortcatvariables}=1$
	\begin{align*}
		\normalizationof{\expof{\canparam\cdot \exformula}}{\indexedshortcatvariables}
		&= \frac{
			\expof{\canparam}
			}{
			\left(\prod_{\atomenumeratorin} \catdimof{\atomenumerator} \right) - \contraction{\exformula} + \contraction{\exformula} \cdot \expof{\canparam}
			} \\
		& \rightarrow \frac{1}{\contraction{\exformula}} 
		= \normalizationof{\exformula}{\indexedcatvariables} \, .
	\end{align*}
	For any $\atomindices\in\atomstates$ with $\formulaat{\indexedshortcatvariables}=0$ we have on the other side
	\begin{align*}
		\normalizationof{\expof{\canparam\cdot \exformula}}{\indexedcatvariables}
		&= \frac{
			1
			}{
			\left(\prod_{\atomenumeratorin} \catdimof{\atomenumerator}\right) - \contraction{\exformula} + \contraction{\exformula} \cdot \expof{\canparam}
			} \\
		& \rightarrow 0
		= \normalizationof{\exformula}{\indexedcatvariables} \, .
	\end{align*}
\end{proof}



% Limit on the activation core
We can by the above Lemma represent both the situation of non-asymptotic weights and the limit for diverging weights by the same computation core $\formulaccwith$, with different activation cores, since
\begin{align*}
	\normalizationof{\expof{\canparam\cdot\formulaat{\shortcatvariables}}}{\shortcatvariables}
	= \contractionof{\formulaccwith,\actcoreof{\formula,\canparam}}{\shortcatvariables}
\end{align*}
and 
\begin{align*}
	\normalizationof{\formula}{\shortcatvariables}
	= \contractionof{\formulaccwith,\tbasisat{\formulavar}}{\shortcatvariables}
\end{align*}
respectively
\begin{align*}
	\normalizationof{\lnot\formula}{\shortcatvariables}
	= \contractionof{\formulaccwith,\fbasisat{\formulavar}}{\shortcatvariables} \, . 
\end{align*}



\begin{theorem}
	Let $\formulaset$ be a formula set and $\canparam$ a positive parameter vector.
	If the formula
		\[ \kb = \bigwedge_{\exformulain} \exformula \]
	is satisfiable we have in the limit $\invtemp\rightarrow\infty$ the coordinatewise convergence
		\[ \expdistofat{(\formulaset,\invtemp\cdot\canparam)}{\shortcatvariables} \rightarrow \normalizationofwrt{\kb}{\shortcatvariables} \, . \]
\end{theorem}
\begin{proof}
	Since $\kb$ is satisfiable we find $\catindices\in\atomstates$ with
		\[  \contractionof{\expof{\sum_{\exformulain}\invtemp\cdot \weightof{\exformula} \cdot \exformula}}{\indexedcatvariables} = \expof{\invtemp \cdot \sum_{\exformulain}\weightof{\exformula}}  \]
	and the partition function obeys
		\[ \contractionof{\expof{\sum_{\exformulain}\invtemp\cdot \weightof{\exformula} \cdot \exformula}}{\varnothing} \geq  \expof{\invtemp \cdot \sum_{\exformulain}\weightof{\exformula}}  \, . \]
	For any state $\catindices\in\atomstates$ with $\kb(\catindices)=0$ we find $\secexformula\in\formulaset$ with
	\begin{align*}
		\secexformula(\catindices)=0
	\end{align*}
	and have
	\begin{align*}
	 	\frac{
		\contractionof{\expof{\sum_{\exformulain}\invtemp\cdot \weightof{\exformula} \cdot \exformula}}{\indexedcatvariables}
		}{
		\contractionof{\expof{\sum_{\exformulain}\invtemp\cdot \weightof{\exformula} \cdot \exformula}}{\varnothing}
		} 
		\leq  
	 	\frac{
		\expof{\invtemp\cdot \sum_{\exformulain : \exformula\neq \secexformula}\weightof{\exformula}}
		}{
		\expof{\invtemp\cdot \sum_{\exformulain}\weightof{\exformula}}
		} 
		= \expof{\invtemp \cdot \weightof{\secexformula}} \rightarrow 0 \, . 
	\end{align*}
	The limit of the distribution has thus support only on the models of $\kb$. 
	Since any model of $\kb$ has same energy at any $\invtemp$ the limit is a uniform distribution and coincides therefor with
		\[ \normalizationof{\kb}{\shortcatvariables} \, . \]
\end{proof}

% Generalization to face base measures
We here assumed, that the conjunction $\kb$ of the formulas in $\formulaset$ is satisfiable, and showed, that the markov logic network converges in the limit of large weights to the uniform distribution of the models of $\kb$.
In \charef{cha:networkReasoning} we will drop this assumption and show that the limit is the face base measure associated with the corresponding face of the mean parameter polytope.
The face base measure coincides thereby with $\kb$, if $\kb$ is satisfiable.

%\begin{remark}[More generic situation of simulated annealing]
%	The process of taking $\invtemp\rightarrow\infty$ is known as simulated annealing, see \charef{cha:probReasoning}.
%	From the discussion there we have the more general statement, that the limiting distribution is the uniform distribution among the maxima of $\expdistofat{(\formulaset,\canparam)}{\shortcatvariables}$.
%	If the formula $\kb$ is not satisfiable the normalization $\normalizationofwrt{\kb}{\shortcatvariables}{\varnothing}$ does not exist and the limit distribution has another syntactical representation, to be gained e.g. by minterm or maxterm representation (see Theorem~\ref{the:tensorToMaxMinTerms}).
%\end{remark}


\subsect{Tensor Network Representation}

Hard Logic Network coincide with Knowledge Bases and are thus representable by contractions of formulas (which can be interpreted as a hybrid calculus scheme, see \secref{sec:hybridCalculus}).


\begin{theorem}[Conjunction Decomposition of Knowledge Bases]\label{the:conDecKB}
	For a Knowledge Base
		\[ \kb = \bigwedge_{\exformula\in\formulaset} \exformula \]
	we have
		\[ \kbat{\shortcatvariables} = \contractionof{\formulaat{\shortcatvariables}}{\shortcatvariables}   \]
	and
		\[ \kbat{\shortcatvariables} = \contractionof{\{\formulaccwith \, : \, \formula\in\formulaset\} \cup \{\tbasisat{\formulavar} \, : \, \formula\in\formulaset\} }{\shortcatvariables} \, .  \]
\end{theorem}
\begin{proof}
	This follows from the representation of conjunctions by contraction (see \secref{sec:hybridCalculus}) and
%	By the $\land$-symmetry, see effective calculus and
		\[ \formulaat{\shortcatvariables} =  \contractionof{\formulaccwith,\tbasisat{\formulavar}}{\shortcatvariables} \, .\]
\end{proof}

We call this representation scheme the $\land$-symmetry, since we can either represent $\kb$ by instantiation of $\bencodingof{\kb}$, which involves a basis encoding of the conjunction $\land$, or by instantiations of a collection of $\bencodingof{\formula}$.
We use the $\land$ symmetry to represent them as a contraction of the formulas building the Knowledge Base as conjunction.

\begin{remark}{$\land$ symmetry does not generalize to Markov Logic Networks} % Strange -> Drop?
	% Comparison to Markov Logic
	In Markov Logic Networks, similar decompositions are not possible.
	For example, consider a MLN with a single formula $\atomicformulaof{0}\land\atomicformulaof{1}$ and nonvanishing weight $\canparam$.
	This does not coincide with the distribution of a MLN of two formulas $\atomicformulaof{0}$ and $\atomicformulaof{1}$.
	To see this, we notice that with respect to the distribution of the first MLN, both variables are not independent, while for any MLN constructed by the two atomic formulas they are.
	% Can also be understood based on non-elementary contraction of $\land$ !
\end{remark}


%\begin{theorem}[$\land$-symmetry]\label{the:landSymmetry}
%	We observe that the contraction of an $\land$ core with $\tbasis$  is equivalent with $\tbasis$ cores on all the connected subformulas.
%\end{theorem}
%\begin{proof}
%	By equality of the Knowledge Base contraction in both ways: The missing subformulas behave the same if they are activated, since they then are contrained to the same subnetworks somewhere else. 
%	%\red{Find better arguments for missing subformulas when having the larger core.}
%\end{proof}
%
%\begin{theorem}[$\lnot$-symmetry]
%	Similarly the contraction of an $\lnot$ core with $\tbasis$ or $\tbasis$ has the same result as with $\tbasis$ or $\tbasis$ on the subformula.
%\end{theorem}
%
%We call the application of these in changing the Knowledge Cores without changing the contracted network as the representation symmetry.


%\subsect{Conjunctive Normal Representation}

%One tensor representation of a Knowledge Base is the association of the Knowledge Core $\tbasis$ at the formula being the Knowledge Base itself.
%We can use the $\land$ symmetry (Theorem~\ref{the:landSymmetry}) to propagate $\tbasis$ to all clause cores and get an alternative representation.
%Those are especially interesting when using Modus Ponens/Resolution as local sub-KB reasoners (see \secref{subsec:LocalEntailment}).

\sect{Hybrid Logic Network}\label{sec:hybridNetworks}

Markov Logic Networks are by definition positive distributions.
In contrary, Hard Logic Networks model uniform distributions over model sets of the respective Knowledge Base and therefore have vanishing coordinates.
We now show how to combine both approaches by defining Hybrid Logic Networks, when understanding Hard Logic Networks as base measures.
This trick is known to the field of variational inference, see for Example~3.6 in \cite{wainwright_graphical_2008}. 

\begin{definition}\label{def:hln}
	Given a set of formulas $\softformulaset$ with weights $\canparam$ and set $\hardformulaset$ of formulas, which conjunction is satisfiable, the hybrid logic network is the probability distribution
	\begin{align*}
		\probtensorof{(\softformulaset,\canparam,\hfbasemeasure)}[\shortcatvariables] 
		= \normalizationof{
		\{\exformula : \exformula\in\hardformulaset\} \cup \{\expof{\weightof{\exformula}\cdot\exformula} : \exformula\in\softformulaset\}
		}{\shortcatvariables} \, ,
	\end{align*}
	which is the member of the exponential family with statistic by $\softformulaset$ and the base measure
		\[ \hfbasemeasure[\shortcatvariables] = \contractionof{\{\formula : \formula \in \hardformulaset\}}{\shortcatvariables} \, .\]
		
	Given a set of formulas $\formulaset$, we define the set of hybrid logic networks realizable with $\formulaset$ and elementary activation cores as
	\begin{align*}
		\hlnsetof{\formulaset} 
		= \bigcup_{\secformulaset\subset\formulaset \, , \, \meanparam\in\{0,1\}^{\cardof{\formulaset}}}
		\expfamilyof{\formulaset/\secformulaset,\basemeasureof{\secformulaset,\meanparam}}
	\end{align*}
	where we denote base measures
	\begin{align*}
		\basemeasureofat{\secformulaset,\meanparam}{\shortcatvariables}
		= \bigwedge_{\enumformula\in\secformulaset} \lnot^{(1-\meanparamat{\indexedselvariable})} \enumformulaat{\shortcatvariables} \, . 
	\end{align*}
\end{definition}

The assumption of a satisfiable set $\hardformulaset$ is necessary, as we show next.

\begin{theorem}
	If any only if $\bigwedge_{\formula\in\hardformulaset}\formula$ is satisfiable, the tensor 
		\[  \contractionof{
		\{\exformula : \exformula\in\hardformulaset\} \cup \{\expof{\weightof{\exformula}\cdot\exformula} : \exformula\in\softformulaset\}
		}{\shortcatvariables} \]
	is normable.
\end{theorem}
\begin{proof}
	We need to show that
	\begin{align}\label{eq:tbsWellDefinedHLN}
		\contraction{\{\exformula : \exformula\in\hardformulaset\} \cup \{\expof{\weightof{\exformula}\cdot\exformula} : \exformula\in\softformulaset\}} > 0 \, . 
	\end{align}
	Since the conjunction of $\hardformulaset$ is satisfiable we find a $\shortcatindices$ with $\formulaat{\indexedcatvariableof{[\catorder]}}=1$ for all $\exformula\in\hardformulaset$.
	Then 
	\begin{align*}
		\contractionof{\{\exformula : \exformula\in\hardformulaset\} \cup \{\expof{\weightof{\exformula}\cdot\exformula} : \exformula\in\softformulaset\}}{\indexedcatvariableof{[\catorder]}}
		 & \quad = \left( \prod_{\exformula\in\hardformulaset}\formulaat{\indexedcatvariableof{[\catorder]}} \right) \\
		 \cdot \left( \prod_{\exformula\in\softformulaset}\expof{\weightof{\exformula}\cdot\exformula}[\indexedcatvariableof{[\catorder]}] \right) \\
		 & \quad =  \left( \prod_{\exformula\in\softformulaset}\expof{\weightof{\exformula}\cdot\exformula}[\indexedcatvariableof{[\catorder]}] \right) \\
		 & \quad > 0 \, .
	\end{align*}
	Condition \eqref{eq:tbsWellDefinedHLN} follows from this and the Hybrid Logic Network is well-defined.
\end{proof}


\subsect{Tensor Network Representation}

We can employ the formula decompositions to represent both probabilistic facts of the MLN and hard facts (seen as the limit of large weights).

\begin{theorem}\label{the:hybridNetworkRepresentation}
	For any hybrid logic network we have
	\begin{align*}
		\probtensorof{(\softformulaset,\canparam,\hardformulaset)}[\shortcatvariables] 
		= \breakablenormalizationofwrt{
		&\{\formulaccwith : \exformula\in\softformulaset\cup\hardformulaset \} \\
		&\cup \{\tbasisat{\formulavar} : \exformula\in\hardformulaset \}
		\cup \{\actcoreofat{\exformula}{\formulavar} : \exformula\in\softformulaset \}
		}{\shortcatvariables}{\varnothing} \, .
	\end{align*}
\end{theorem}
\begin{proof}
	By \lemref{lem:formulaEncodingDecomposition}.
\end{proof}

% Discussion
While the statistics computing cores in the basis encoding are shared to compute the soft and the hard logic formulas, their activation cores differ.
While probabilistic soft formulas get activation cores (see Theorem~\ref{the:mlnTensorRep})
\begin{align*}
	\actcoreofat{\exformula}{\formulavar} 
	= \begin{bmatrix} 1 \\
		 \expof{\canparamat{\indexedselvariableof{}}} 
		 \end{bmatrix}[\enumformulavar] \,
\end{align*}
the hard formulas get activation cores by unit vectors
\begin{align*}
	\tbasisat{\formulavar} 
	= \begin{bmatrix} 0 \\
		 1
	 \end{bmatrix}[\enumformulavar]  \, .
\end{align*}
As shown in \secref{sec:hardLogicLimit}, the soft activation cores converge to these hard activation cores in the limit of large parameters, when imposing a local normalization.
%
We further notice, that the probabilistic activation cores are trivial tensors if and only if the corresponding parameter coordinate vanishes.


%The reason for this is the Slicing Theorem, enabling the operations by both (exponentiation and selection of one slice) by the activation cores.
For an example see Figure~\ref{fig:ActivatedHeads}.

\begin{figure}[h]
\begin{center}
	\begin{tikzpicture}[scale=0.35, thick, yscale=-1] % , baseline = -3.5pt


\draw[<-] (0,-1)--(0,1) node[midway,left] {\tiny $\catvariableof{a}$}; 
\draw[<-] (1.5,-1)--(1.5,1) node[midway,right] {\tiny $\catvariableof{b}$}; 
\draw[<-] (3,-1)--(3,1) node[midway,right] {\tiny $\catvariableof{c}$}; 
\draw (-1,-1) rectangle (4, -3);
\node[anchor=center] (text) at (1.5,-2) {$\partitionfunction \cdot \probtensor$};



\node[anchor=center] (text) at (5.5,-2) {${=}$};


\begin{scope}[shift={(11,0)}]

\draw[->] (0,1)--(0,-1)  node[midway,left] {\tiny $\catvariableof{a}$}; 
\draw[->]  (3,1)--(3,-1) node[midway,right] {\tiny $\catvariableof{b}$}; 
\draw[->] (6,1)--(6,-1) node[midway,right] {\tiny $\catvariableof{c}$}; 
	
\draw (-1,-1) rectangle (4, -3);
\node[anchor=center] (text) at (1.5,-2) {$\bencodingof{\lor}$};

\draw[->] (1.5,-3) --(1.5,-4.5) node[midway,right]{\tiny $\catvariableof{a \lor b}$};
\drawvariabledot{1.5}{-4.5}
\draw[->] (1.5,-4.5) --(1.5,-6) ;


\draw[fill, \probcolor] (1.5,-4.5) circle (0.15cm);
\draw[\probcolor] (1.5,-4.5) -- (-0.25,-4.5);
\draw[\probcolor]  (-6.75, -3.5) rectangle (-0.25, -6.5);
\node[anchor=center,\probcolor] (text) at (-3.5,-5) {$\begin{bmatrix} 
1 \\
\expof{\canparam}
\end{bmatrix}$};

\draw (5,-1) rectangle (7, -3);
\node[anchor=center] (text) at (6,-2) {$\bencodingof{\lnot}$};

\draw[->] (6,-3) --(6,-4.5) node[midway,right]{\tiny $\catvariableof{\lnot c}$};
\drawvariabledot{6}{-4.5}
\draw[->] (6,-4.5) --(6,-6);	


	
\draw (0.5,-6) rectangle (6.5,-8);
\node[anchor=center] (text) at (3.5,-7) {$\bencodingof{\lor}$};
	
\draw[<-] (4,-9.5) -- (4,-8) node[midway,right] {\tiny $\catvariableof{a \lor b \lor \lnot c}$};

\draw[fill,\concolor] (4,-9.5) circle (0.15cm);
\draw[\concolor] (4,-9.5) -- (2.25,-9.5);
\draw[\concolor]  (0.25, -8.5) rectangle (2.25, -11.5);
\node[anchor=center,\concolor] (text) at (1.25,-10) {$\begin{bmatrix} 
0 \\
1
\end{bmatrix}$};

%\draw (3,-9) rectangle (5,-11);
%\node[anchor=center] (text) at (4,-10) {$\truevectorat{}$};

\end{scope}

\end{tikzpicture}
\end{center}
\caption{Diagram of a formula tensor with activated heads, containing \textcolor{\concolor}{hard constraint cores} and \textcolor{\probcolor}{probabilistic weight cores} .} %along \textcolor{\inactivecolor}{inactive cores}.}
\label{fig:ActivatedHeads} 
\end{figure}



\begin{remark}{Probability interpretation using the Partition function}
	The tensor networks here represent unnormalized probability distributions.
	The probability distribution can be normed by the quotient with the naive contraction of the network, the partition function.
\end{remark}


\subsect{Reasoning Properties}

Deciding probabilistic entailment (see \defref{def:probEntailment}) with respect to Hybrid Logic Networks can be reduced to the hard logic parts of the network.

\begin{theorem}\label{the:hlnEntailmentReduction}
	Let $(\softformulaset,\canparam,\hardformulaset)$ define a Hybrid Logic Network.
	Given a query formula $\exformula$ we have that 
		\[ \probtensorof{(\softformulaset,\canparam,\hardformulaset)} \models \exformula \]
	if and only if
		\[ \hardformulaset \models \exformula \, . \]
\end{theorem}
\begin{proof}
	This follows from Theorem~\ref{the:factorReduction} on the representation of Hybrid Logic Networks as Markov Networks in Theorem~\ref{the:hybridNetworkRepresentation}.
\end{proof}


Formulas in $\softformulaset$, which are entailed or contradicted by $\hardformulaset$ are redundant, as we show next.

\begin{theorem}%\label{the:hlnRepRedundancy}
	If for a formula $\exformula$ and $\hardformulaset$ we have 
		\[ \hardformulaset \models \exformula \, \quad \text{or} \quad \hardformulaset \models \lnot\exformula \]
	then for any $(\softformulaset,\canparam,\hardformulaset)$
		\[ \probofat{(\softformulaset/\{\exformula\},\tilde{\canparam},\hardformulaset)}{\shortcatvariables} =  \probofat{(\softformulaset,\canparam,\hardformulaset)}{\shortcatvariables}  \, , \]
	where $\tilde{\canparam}$ denotes the tensor $\canparam$, where the coordinate to $\exformula$ is dropped, if $\exformula\in\softformulaset$.
\end{theorem}
\begin{proof}
	Isolate the factor to the hard formula, which is constant for all situations.
\end{proof}

%% Now in the 
A similar statement holds for the hard formulas itself, as shown in Theorem~\ref{the:ReduncancyOfEntailed}.
However, notice that if $\hardformulaset/\{\exformula\}\models\lnot\exformula$, then $\hardformulaset\cup\{\exformula\}$ is not satisfiable and a hybrid logic network cannot be defined for $\hardformulaset\cup\{\exformula\}$ as hard logic formulas.

%If the conjunction of $\hardformulaset/\{\exformula\}$ entails $\exformula$, we can erase $\exformula$ from $\hardformulaset$ without changing the contraction, therefore without changing the base measure of the Hybrid Logic Network.

% Utility in Contraction KB implementation
These results are especially interesting for the efficient implementation of \algoref{alg:contractionKB}, which has been introduced in \charef{cha:logicalReasoning}.
By Theorem~\ref{the:hlnEntailmentReduction} only the hard logic parts of a Hybrid Logic Network are required in the ASK operation.
%Theorem~\ref{the:hlnRepRedundancy} for the TELL operation, but already discussed in more generality

\subsect{Expressivity}

Hybrid Logic Networks extend the expressivity result of Theorem~\ref{the:mintermExpressivityMLN} to arbitrary probability tensors, dropping the positivity constraints for Markov Logic Networks.

\begin{theorem}\label{the:mintermExpressivityHLN}
	Let $\probat{\shortcatvariables}$ a possibly not positive probability tensor we build a base measure
		\[ \hfbasemeasure = \nonzeroof{\probat{\shortcatvariables}} \]
	and a parameter tensor
	\begin{align*}
		\canparamat{\selvariableof{[\catorder]}=\shortcatindices}
		= \begin{cases}
			0 & \text{if} \quad \probat{\indexedshortcatvariables} = 0  \\
			\lnof{\probat{\indexedshortcatvariables}} & \text{else} 
		\end{cases} \, . 
	\end{align*}
	Then the probability tensor is the member of the minterm exponential family with base measure $\hardformulaset$ and parameter $\canparam$, that is
		\[ \probat{(\mintermformulaset,\canparam,\hfbasemeasure)}\]
\end{theorem}
\begin{proof}
	It suffices to show that 
		\[ \contractionof{\hfbasemeasure, \expof{\contractionof{
		\sencodingof{\mintermformulaset}\canparam
		}{
		\shortcatvariables
		}}}{\shortcatvariables} = \probat{\shortcatvariables} \, . \]
	For indices $\shortcatindices$ with $\probat{\indexedshortcatvariables}=0$ we have $\hfbasemeasureat{\indexedshortcatvariables}=0$ and thus also 
		\[ \contractionof{\hfbasemeasure, \expof{\contractionof{
		\sencodingof{\mintermformulaset}\canparam
		}{
		\shortcatvariables
		}}}{\indexedshortcatvariables} = 0 \, . \]
	For indices $\shortcatindices$ with $\probat{\indexedshortcatvariables}>0$ we have $\hfbasemeasureat{\indexedshortcatvariables}=1$ and
	\begin{align*}
		 \contractionof{\hfbasemeasure, \expof{\contractionof{
		\sencodingof{\mintermformulaset},\canparam
		}{
		\shortcatvariables
		}}}{\indexedshortcatvariables} 
		&= \prod_{\selindexof{[\catorder]}} \expof{\canparamat{\selvariableof{[\catorder]}=\selindexof{[\catorder]}} \cdot \mintermofat{\selindexof{[\catorder]}}{\indexedshortcatvariables}} \\
		&=  \expof{\canparamat{\selvariableof{[\catorder]}=\shortcatindices}} \\
		&=  \probat{\indexedshortcatvariables} \, .
	\end{align*}
\end{proof}



\sect{Polynomial Representation}

%We now apply the representation symmetries to represent a propositional Knowledge Base in conjunctive normal form.
We now sparse representation formats for the introduced logic networks, namely the basis+ $\cpformat$ format introduced in \charef{cha:sparseRepresentation} which are understood as polynomial decompositions.
First of all, we establish a sparsity result for terms and clauses (see \defref{def:clauses}).

\begin{lemma}\label{lem:clauseTermBasisPlus}
	Any term is representable by a single monomial and any clause is representable by a sum of at most two monomials. %, any term of basis+ with rank 1. %Use also \baspluscprankof{}
\end{lemma}
\begin{proof}
	Let $\nodes_0$ and $\nodes_1$ be disjoint subsets of $\nodes$, then we have
	\begin{align*}
		\termof{\nodes_0}{\nodes_1} = \onehotmapofat{
			\{\catindexof{\atomenumerator} = 0 : \atomenumerator\in\nodes_0\} \cup \{\catindexof{\atomenumerator} = 1 : \atomenumerator\in\nodes_1\}
		}{\catvariableof{\nodes_0\cup\nodes_1}} \otimes \onesat{\catvariableof{\nodes/(\nodes_0\cup\nodes_1)}}
	\end{align*}
	and
	\begin{align*}
		\clauseof{\nodes_0}{\nodes_1} = \onesat{\catvariableof{\nodes}} - \onehotmapofat{
			\{\catindexof{\atomenumerator} = 0 : \atomenumerator\in\nodes_0\} \cup \{\catindexof{\atomenumerator} = 1 : \atomenumerator\in\nodes_1\}
		}{\catvariableof{\nodes_0\cup\nodes_1}}
		\otimes \onesat{\catvariableof{\nodes/(\nodes_0\cup\nodes_1)}} \, .
	\end{align*}
	We notice, that any tensors $\ones$ and $\onehotmapof{\catindex}\otimes \ones$ habe basis+-rank of $1$ and therefore $\termof{\nodes_0}{\nodes_1}$ of $1$ and $\clauseof{\nodes_0}{\nodes_1}$ of at most $2$.
\end{proof}

A formula in conjunctive normal form is a conjunction of clauses, where clauses are disjunctions of literals being atoms (positive literals) or negated atoms (negative literals).
Based on these normal forms, we show representations of formulas as sparse polynomials. %, which will be discussed in more detail in \charef{cha:sparseRepresentation} (see \defref{def:polynomialSparsity}).
We apply \lemref{lem:clauseTermBasisPlus} to show the following sparsity bound. % on the energy tensor of Markov Logic Networks.

\begin{theorem}\label{the:formulaSlicePolynomialDecomposition}
	Any formula $\exformula$ with a conjunctive normal form of $\clausedim$ clauses satisfies
		\[ \slicesparsityof{\exformula} \leq 2^{\clausedim} \, . \]
\end{theorem}
\begin{proof}
	Let $\exformula$ have a conjunctive normal form with clauses indexed by $\clauseenumeratorin$ and each clause represented by subsets $\nodes_0^\clauseenumerator, \nodes_1^\clauseenumerator$, that is
		\[ \exformula = \bigwedge_{\clauseenumeratorin} \clauseof{\nodes_0^\clauseenumerator}{\nodes_1^\clauseenumerator} \, . \]
	We now use the rank bound of \theref{the:CPrankContractionBound} and \lemref{lem:clauseTermBasisPlus} to get
	\begin{align*}
		\slicesparsityof{\exformula} \leq \prod_{\clauseenumeratorin} \slicesparsityof{\clauseof{\nodes_0^\clauseenumerator}{\nodes_1^\clauseenumerator}} \leq 2^{\clausedim} \, .
	\end{align*}
\end{proof}

We apply this result on the sparse representation of a single formula to derive sparse representations for Hard Logic Networks and the energy tensor of Hybrid Logic Networks.
Both results use in addition to \theref{the:formulaSlicePolynomialDecomposition} sparsity bounds, which are shown by explicit representation construction in \charef{cha:sparseRepresentation}.

\begin{corollary}
	Any Hard Logic Network $\hardformulaset$ obeys
	\begin{align*}
		\slicesparsityof{\hardformulaset} \leq \prod_{\exformula\in\hardformulaset} 2^{\clausedimof{\exformula}}
\end{align*}
\end{corollary}
\begin{proof}
	We apply the contraction bound \theref{the:CPrankContractionBound} for the decomposition
	\begin{align*}
		\kbat{\shortcatvariables} = \contractionof{\{\formulaat{\shortcatvariables} \, : \, \formula\in\hardformulaset\}}{\shortcatvariables}
	\end{align*}
	and get
	\begin{align*}
		\slicesparsityof{\kb} \leq \prod_{\formula\in\hardformulaset} \slicesparsityof{\formula} \, .
	\end{align*}
	The claimed bound follows with \theref{the:formulaSlicePolynomialDecomposition}.
\end{proof}

\begin{corollary}
	The energy tensor of a Hybrid Logic Network with statistic $\mlnstat$
	\begin{align*}
		\slicesparsityof{\contractionof{\sencmlnstatwith,\canparamwith}{\shortcatvariables}} \leq \sum_{\selindexin\,:\,\canparamat{\indexedselvariable}\neq0} 2^{\clausedimof{\enumformula}} \, .
	\end{align*}
	where $\clausedimof{\enumformula}$ denotes the number of clauses in a conjunctive normal form of $\enumformula$.
\end{corollary}
\begin{proof}
	We decompse the energy into the sum
	\begin{align*}
		\contractionof{\sencmlnstatwith,\canparamwith}{\shortcatvariables}
		= \sum_{\selindexin\,:\,\canparamat{\indexedselvariable}\neq0} \canparamat{\indexedselvariable} \cdot \enumformulawith
	\end{align*}
	and apply \theref{the:CPrankSumBound} to get
	\begin{align*}
		\slicesparsityof{\contractionof{\sencmlnstatwith,\canparamwith}{\shortcatvariables}}
		\leq \sum_{\selindexin\,:\,\canparamat{\indexedselvariable}\neq0} \slicesparsityof{\enumformulawith}
		\leq \sum_{\selindexin\,:\,\canparamat{\indexedselvariable}\neq0}2^{\clausedimof{\enumformula}} \, .
	\end{align*}
\end{proof}

\sect{Applications}

Hybrid Logic Networks as neuro-symbolic architectures:
\begin{itemize}
	\item Neural Paradigm here by decompositions of logical formulas into their connectives.
		In more generality by decompositions of sufficient statistics into composed functions, using Basis Calculus.
		Deeper nodes as carrying correlations of lower nodes.
	\item Symbolic Paradigm by interpretability of propositional logics.
\end{itemize}


Hybrid Logic Networks as trainable Machine Learning models:
\begin{itemize}
	\item Expressivity: Can represent any positive distribution, as shown by Theorem~\ref{the:maximalClausesRepresentation}, with $2^d$ formulas.
	\item Efficiency: Can only handle small subsets of possible formulas, since their possible number is huge.
		Tensor networks provide means to efficiently represent formulas depending on many variables and reason based on contractions.
	\item Differentiability: Distributions are differentiable functions of their weights, see Parameter Estimation Chapter. 
		The log-likelihood of data is therefore also differentiable function of their weights and we can exploit first-order methods in their optimization.
	\item Structure Learning: We need to find differentiable parametrizations of logical formulas respecting a chosen architecture.
		In \charef{cha:formulaSelection} such representations are described based on Selector Tensor Networks.
\end{itemize}
Differentiability and structure learning will be investigated in more detail in the next chapter.

\red{When understanding atoms as observed variables, and the computed as hidden, Hybrid Logic Networks are deep higher-order boltzmann machines: 
More generic correlations can be captured by a logical connective, calculated by a basis encoding and activated by an activation core.}

\red{Hybrid Logic Networks as bridging soft and hard logics within the formalism of exponential families.}

%\begin{remark}[Hopfield networks]
%	Also interesting for MLNs is a Hopfield perspective.
%	Having an initialization the update can be interpreted as a Gibbs sampling step at temperature $0$ (since deterministic update).
%\end{remark}


% CSP
A more general class of problems, which have natural representations by Hard Logic Networks are Constraint Satisfaction Problems (see Chapter~5 in \cite{russell_artificial_2021}).
Solving such problems is then equivalent to sampling from the worlds in a logical interpretation, and can be approached by the methods of \charef{cha:logicalReasoning}.
Among these classed, we have only discussed the Sudoku game in Example~\ref{exa:sudoku}.
Extensions by Hybrid Logic Networks can be interpreted as implementations of preferences among possible solutions by probabilities.
    \chapter{\chatextnetworkReasoning}\label{cha:networkReasoning}

In this chapter we investigate the inference properties of Hybrid Logic Networks starting with characterizations of its mean parameter polytopes.
We investigate unconstrained parameter estimated for Markov Logic Networks and Hybrid Logic Networks, which are special cases of the backward maps introduced in \charef{cha:probRepresentation}.
We then motivate structure learning based on sparsity constraints on the parameters on the minterm exponential family and present heuristic strategies leading to efficient structure learning algorithms.



\sect{Entropic Motivation of unconstrained Parameter Estimation} \label{sec:parameterEstimation} % Check for redundancy with the mln introduction chapter!

% Repetition and result transfering
Markov Logic Networks are exponential families with statistics by a set $\formulaset$ of propositional formulas.
We furthermore allow for propositional formulas as base measures, to also include the discussion of Hybrid Logic Networks.
Based on this, we apply the theory of probabilistic inference, developed in \charef{cha:probReasoning} for the generic exponential families.

\subsect{Maximum Likelihood in Hybrid Logic Networks}

% Special example: MLE
The Maximum Likelihood Problem on a hybrid logic network family is the moment projection
\begin{align*}
	\argmax_{\canparamat{\selvariable}\in\rr^{\seldim}} \quad
	\centropyof{\probtensor}{\expdistof{(\sstat,\canparam,\basemeasure)}}
\end{align*}
in the case $\probtensor=\empdistribution$ for a sample selector map $\datamap$.

% Backward map
The moment projection coincides, after dropping constant terms in case of non-trivial base measure, with the backward map
\begin{align*}
	\argmax_{\canparamat{\selvariable}\in\rr^{\seldim}} \quad
	\contraction{\canparamat{\selvariable},\meanparamwith} - \cumfunctionof{\canparamat{\selvariable}}
\end{align*}
where
\begin{align*}
	\meanparamwith
	= \contractionof{\sencmlnstat,\probtensor}{\selvariable}
	\quad \text{and} \quad
	\cumfunctionof{\canparamat{\selvariable}}
	= \contraction{\expof{ \contractionof{\sencodingofat{\formulaset}{\shortcatvariables,\selvariable},\canparamat{\selvariable}}{\shortcatvariables} }, \basemeasure} \, .
\end{align*}

% Extension to HLN
We now extend to Hybrid Logic Networks
\begin{align*}
	\argmax_{\secprobtensor\in\hlnsetof{\formulaset}} \quad
	\centropyof{\probtensor}{\secprobtensor}
\end{align*}



\begin{corollary}
	Let $\meanparamwith = \contractionof{\probtensor,\sencfset}{\selvariable}$ and
		\[ \secformulaset = \{\enumformula \, : \, \meanparamat{\indexedselvariable} \in \{0,1\}\} \quad , \quad
		\basemeasureofat{\secformulaset,\meanparam}{\shortcatvariables}
		= \bigwedge_{\enumformula\in\secformulaset} \lnot^{(1-\meanparamat{\indexedselvariable})} \enumformulaat{\shortcatvariables}
		\, . \]
	If $\meanparamwith$ is reproduceable by a positive distribution with respect to $\basemeasureofat{\secformulaset,\meanparam}{\shortcatvariables} $, then the solution of the M-projection of $\probtensor$ onto the set of hybrid logic networks is representable by $\formulaset$ then coincides with the projection of $\probtensor$ onto $\expfamilyof{\formulaset/\secformulaset,\basemeasureof{\secformulaset,\meanparam}}$.
\end{corollary}
%\begin{proof}
%	Work by weight cutoff, in the limit of hard logics?
%\end{proof}

\subsect{Maximum Entropy in Hybrid Logic Networks}


% Special example: MaxEnt
The Maximum Entropy Problem for a statistic $\mlnstat$ is %Markov Logic Networks is
\begin{align}
	\argmax_{\probtensor} \quad \sentropyof{\probtensor}
	\quad \text{subject to} \quad
	\contractionof{\probtensor,\sencmlnstat}{\selvariable}
	 =  \meanparamwith
\end{align}



\begin{corollary}% THEOREM DOES NOT EXIST! [of \theref{the:maxEntMaxLikeDuality}]
	%\red{Works only for $\meanparamat{\indexedselvariable}\in\interiorof{\hlnmeanset}$}
	Let $\empdistribution$ be a distribution such that there is a positive distribution $\probtensor$ with $\contractionof{\probtensor,\sencmlnstat}{\selvariable} = \contractionof{\empdistribution,\sencmlnstat}{\selvariable}$.
	Among all positive distributions $\probtensor$ of $\atomstates$ satisfying this moment matching condition, the Markov Logic Network with formulas $\formulaset$ and weights $\canparam$ being the solution of the maximum likelihood problem has minimal entropy.
\end{corollary}

% Unique property of MLN
We notice, that the solution of the maximum entropy problem is thus a Markov Logic Network.
This is remarkable, because this motivates our restriction to Markov Logic Networks as those distributions with maximal entropy given satisfaction rates of formulas in $\formulaset$.


% Extension to HLN
When now extend to the situations $\meanparamat{\indexedselvariable}\in\{0,1\}$ can appear.
It that case the formula is entailed or contradicted by the facts, and dropping should be considered in both cases.

The max entropy - max likelihood duality still holds for hybrid logic networks as we show in the next theorem.

\begin{theorem}
	Given a set of formulas $\tilde{\formulaset}$ and $\tilde{\meanparam}$, with coordinates $\tilde{\meanparam}_\selindex\in[0,1]$ in the closed interval $[0,1]$.
	If the corresponding maximum entropy problem is feasible, its solution is a hybrid logic network with
	\begin{itemize}
		\item $\hardformulaset= \{\enumformula : \selindexin, \meanparamat{\indexedselvariable} = 1\} \cup \{\lnot\enumformula : \selindexin, \meanparamat{\indexedselvariable} = 0\} $
		\item $\softformulaset = \{\enumformula : \selindexin, \meanparamat{\indexedselvariable} \in (0,1)\}$
		\item $\canparam$ being the backward map evaluated at the vector $\meanparam$ consisting of the coordinates of $\tilde{\meanparam}$ not in $\{0,1\}$
	\end{itemize}
\end{theorem}
\begin{proof}
	Feasible distributions have a density with base measure by $\hardformulaset$, we therefore reduce the set of distributions in the argmax to those with density to the base measure.
	The max entropy is a max entropy problem with respect to that base measure, where we only keep the constraints to the mean parameters different from $\{0,1\}$ (those are trivially satisfied).
	The statement then follows from the generic property (see Sec3.1 in \cite{wainwright_graphical_2008}).
\end{proof}






\sect{Alternating Algorithms to Approximate the Backward Map}\label{sec:alternatingParEstMLN}

Let us now introduce an implementation of the Alternating Moment Matching Algorithm~\ref{alg:AMM} in case of Markov Logic Networks.
To solve the moment matching condition at a formula $\enumformula$ we refine \lemref{lem:mmContractionEquation} in the following.

\begin{lemma}\label{ref:lemMMinMLN}
	Let there be a base measure $\basemeasure$, a formula selecting map $\formulaset=\{\enumformula \, : \, \selindexin\}$ and weights $\canparam$, and choose $\selindexin$ such that $\enumformula  \notin \{\onesat{\shortcatvariables},\zerosat{\shortcatvariables}\}$.
	The moment matching condition relative to $\canparam$, $\selindexin$ and $\datameanat{\indexedselvariable}\in(0,1)$ is then satisfied, if
	\begin{align} \label{sol:momentMatchingExformula}
	 	\weightat{\indexedselvariable} = \lnof{
		\frac{\datameanat{\indexedselvariable}}{(1-\datameanat{\indexedselvariable})}
		\cdot \frac{\hypercoreat{\catvariableof{\enumformula }=0}}{\hypercoreat{\catvariableof{\enumformula }=1}}
		}
	\end{align}
	where by $\hypercoreat{\catvariableof{\enumformula }}$ we denote the contraction
	\begin{align*}
	 	\hypercoreat{\catvariableof{\enumformula}}
		= \contractionof{\{\bencodingof{\enumformula} \, : \, \selindexin\}
		\cup\{\actcoreof{\tilde{\selindex}} : \tilde{\selindex} \in [\seldim], \tilde{\selindex}\neq\selindex\}
		\cup\{\basemeasure\}}{\catvariableof{\enumformula}} \, .
	\end{align*}
\end{lemma}
\begin{proof}
	Since $\imageof{\enumformula}\subset[2]$ we have
	\begin{align*}
		\idrestrictedto{\imageof{\enumformula}} = \onehotmapofat{1}{\catvariableof{\enumformula}}
	\end{align*}
	and the moment matching condition is by \lemref{lem:mmContractionEquation} satisfied if
	\begin{align*}
		\contraction{\actcoreof{\selindex}, \onehotmapof{1}, \hypercore}
			= \contraction{\actcoreof{\selindex},\hypercore} \cdot \datameanat{\indexedselvariable} \, .
	\end{align*}
	This is equal to
	\begin{align*}
		\expof{\canparamat{\indexedselvariable}} \cdot \hypercoreat{\catvariableof{\enumformula}=1}
		= \left( \expof{\canparamat{\indexedselvariable}} \cdot \hypercoreat{\catvariableof{\enumformula}=1} + \hypercoreat{\catvariableof{\enumformula}=0} \right) \cdot \datameanat{\indexedselvariable} \, .
	\end{align*}
	Rearranging the equations this is equal to
	\begin{align*}
	 	\hypercoreat{\catvariableof{\enumformula}}
		= \contractionof{\{\bencodingof{\enumformula}\}
		\cup\{\actcoreof{\tilde{\selindex}} : \tilde{\selindex} \in [\seldim], \tilde{\selindex}\neq\selindex\}
		\cup\{\basemeasure\}}{\selvariable} \, .
	\end{align*}
	We notice that the right side is well defined, since we have by assumption $\datameanat{\indexedselvariable}, (1- \datameanat{\indexedselvariable}) \neq 0$ and $\hypercoreat{\catvariableof{\enumformula}=0}, \hypercoreat{\catvariableof{\enumformula}=1} \neq 0$ since Markov Logic networks are positive distributions and $\enumformula \notin \{\onesat{\shortcatvariables},\zerosat{\shortcatvariables}\}$.
\end{proof}


%% Hard network reference!
In the case $\datameanat{\indexedselvariable}\in\{0,1\}$ the moment matching conditions are not satisfiable for $\canparamat{\indexedselvariable}\in\rr$.
But, we notice, that in the limit $\canparamat{\indexedselvariable}\rightarrow \infty $ (respectively $-\infty$) we have
	\[ \meanparamat{\indexedselvariable} \rightarrow  1 \quad \text{(respectively $0$)}\, ,  \]
and the moment matching can be satisfied up to arbitrary precision.
In \secref{sec:hardNetworks} we will allow infinite weights and interpret the corresponding factors by logical formulas.
As a consequence, we will able to fit graphical models, which we will call hybrid networks on arbitrary satisfiable mean parameters.

%
The cases $\hypercoreat{\catvariableof{\enumformula}=1}=0$, respectively $\hypercoreat{\catvariableof{\enumformula}=1}=0$ only appear for nontrivial formulas when the distribution is not positive.
This is not the case for Markov Logic Networks, but will happen when formulas are added as cores of a Markov Network.
This situation will has been investigated in \secref{sec:hardNetworks}.


% Concave likelihood 
Since the likelihood is concave (see \cite{koller_probabilistic_2009}), there are not local maxima the coordinate descent could run into and coordinate descent will give a monotonic improvement of the likelihood.

We suggest an alternating optimization by Algorithm~\ref{alg:AWO}, solving the moment matching equation iteratively for all formulas $\exformulain$ and repeat the optimization until a convergence criterion is met.
This is an coordinate ascent algorithm, when interpreted the loss $\lossof{\expdist}$ as an objective depending on the vector $\canparam$.

\begin{algorithm}[hbt!]
\caption{Alternating Weight Optimization (AWO)}\label{alg:AWO}
\begin{algorithmic}
%% INPUT: Numerated formula set, mean parameter $\datameanat{\selvariable}$
%\For{$\exformula\in\formulaset$}
	\Require Empirical distribution $\empdistribution$, boolean features $\mlnstat$ %and base measure $\basemeasure$
	\Ensure Canonical parameter $\canparamwith$, such that $\expdist$ is the (approximative) moment projection of $\empdistribution$ onto $\expfamily$
	\hrule
    \State Compute $\datameanat{\selvariable}= \contractionof{\empdistribution,\sencsstat}{\selvariable}$
	\State $\kb = \ones$, $\secnodes=\varnothing$
\For{$\selindexin$}
	\If{$\meanparamat{\indexedselvariable}=1$}
		\[ \hardformulaset \algdefsymbol \hardformulaset \cup \{\enumformula\}\]
	\ElsIf{$\meanparamat{\indexedselvariable}=0$}
		\[ \hardformulaset \algdefsymbol \hardformulaset \cup \{\lnot\enumformula\}\]
	\Else
		\State $\secnodes\algdefsymbol \secnodes \cup \{\selindex\}$
	\EndIf
\EndFor
\For{$\selindex\in\secnodes$}
		\State Compute
		\[ \hypercoreat{\catvariableof{\enumformula}}
		\algdefsymbol \contractionof{\bencodingof{\enumformula}}{\catvariableof{\enumformula}} \]
		\State Set
		\begin{align*}
	 		\canparamat{\indexedselvariable}
			\algdefsymbol \lnof{
			\frac{\datameanat{\indexedselvariable}}{(1-\datameanat{\indexedselvariable})}
			\cdot \frac{\hypercoreat{\catvariableof{\enumformula}=0}}{\hypercoreat{\catvariableof{\enumformula}=1}}
			}
		\end{align*}
\EndFor
\If {$\contraction{\kb}=0$}
	 \State \textbf{raise} "Inconsistent Knowledge Base"
\EndIf
\While{Convergence criterion is not met}
\For{$\selindex\in\secnodes$}
	\State Compute
	\begin{align*}
	 	\hypercoreat{\catvariableof{\enumformula}}
		= \contractionof{\{\bencodingof{\enumformula} \, : \, \selindexin\}
		\cup\{\actcoreof{\tilde{\selindex}} : \tilde{\selindex} \in [\seldim], \tilde{\selindex}\neq\selindex\}
		\cup\{\basemeasure\}}{\catvariableof{\enumformula}}
	\end{align*}
	\State Set
	\begin{align*}
	 	\canparamat{\indexedselvariable} = \lnof{
		\frac{\datameanat{\indexedselvariable}}{(1-\datameanat{\indexedselvariable})}
		\cdot \frac{\hypercoreat{\catvariableof{\enumformula}=0}}{\hypercoreat{\catvariableof{\enumformula}=1}}
		}
	\end{align*}
\EndFor
\EndWhile
	\State \Return $\canparamwith$
\end{algorithmic}
\end{algorithm}


% Independent formulas
In the initialization phase of Algorithm~\ref{alg:AWO}, each parameters is initialized relative to a uniform distribution.
The algorithm would be finished, if the variables $\catvariableof{\exformula}$ are independent.
This would be the case, if the Markov Logic Network consists of atomic formulas only.
When they fail to be independent, the adjustment of the weights influence the marginal distribution of other formulas and we need an alternating optimization.
% 
This situation corresponds with couplings of the weights by a partition contraction, which does not factorize into terms to each formula.


% Inference
\red{
Solving Equation~\ref{sol:momentMatchingExformula} requires inference of a current model by answering a query.}
This can be a bottleneck and circumvented by approximative inference, see e.g. CAMEL \cite{ganapathi_constrained_2008}.



\begin{remark}[Grouping of coordinates with trivial sum]
	When having a set of coordinates, such that the coordinate functions are binary and sum to the trivial tensor, one can find simultaneous updates to the canonical parameters, such that the partition function is staying invariant.
	Given a parameter $\canparam^t$ we compute
		\[ \meanparam^t = \contractionof{\expdistof{(\sstat,\canparam^t)}, \sstat}{\selvariable} \]
	and build the update
		\[ \canparam^{t+1} = \canparam^t + \lnof{\meanparam^{\datamap}}{\meanparam^t} \, . \]
	Then, $\canparam^{t+1}$ satisfies the moment matching equations for all coordinates in the set.


	The assumptions are met when taking all features to any hyperedge in a Markov Network seen as an exponential family.
	In that case, the update algorithm is refered to as  Iterative Proportional Fitting \cite{wainwright_graphical_2008}.
	Further, when activating both $\exformula$ and $\lnot\exformula$.
\end{remark}


\sect{Forward and backward mappings in closed form}

% Closed form availability
We recall from \charef{cha:probReasoning}, that while forward mappings are always in closed form by contractions, backward mapping in general do not have a closed form representation.
Instead, the backward map is in general implicitly characterized by a maximum entropy problem constrained to matching expected sufficient statistics.
We investigate in this section specific examples, where closed forms are available for both.
In these cases, parameter estimation can thus be solved by application of the inverse on the expected sufficient statistics with respect to the empirical distribution, and iterative algorithms can be avoided.

% Usage
%When the backward map $\backwardmap$ is available in closed form, we directly get optimal parameters by the inversion acting on the satisfaction rate and can avoid iterative algorithms of parameter estimation.

\subsect{Maxterms and Minterms}

Minterms (respectively maxterms) are ways in propositional logics to get a syntactical formula representation based on a formula to each world which is a model (respectively fails to be a model).
We have already studied in \secref{sec:MLNMaxMintermRep} how to represent any distribution as a MLN of maxterms (respectively minterms), see \theref{the:maximalClausesRepresentation}.

We use the tuple enumeration of the maxterms and minterms by $\atomstates$ introduced in \secref{sec:termClauseDecomposition}.
With respect to this enumeration the canonical parameters and mean parameters are tensors in $\bigotimes_{\atomenumeratorin}\rr^2$.
%% Interpretation of the mean parameters
Since the statistic of the minterm family is the identity, the mean parameters for the minterm family are
	\[ \meanparamat{\selvariableof{[\atomorder]}=\catindexof{[\atomorder]}}
	= \probat{\catindexof{[\atomorder]}}
	\]
and therefore after a relabeling of categorical variables to selection variables $\meanparam=\probtensor$.
For maxterms we have analogously
	\[ \meanparamat{\selvariableof{[\atomorder]}=\catindexof{[\atomorder]}}
	= 1-\probat{\catindexof{[\atomorder]}}
	\]
and $\meanparam = \onesat{}-\probtensor$.
We can use these insights to provide a characterization of the forward and backward maps of the minterm and maxterm family.

\begin{theorem}
	Given the Markov Logic Networks to the formula sets
		\[ \mintermformulaset := \{ \mintermof{\atomindices} \, : \, \atomindicesin\} \quad \text{and} \quad
		\maxtermformulaset := \{ \maxtermof{\atomindices} \, : \, \atomindicesin\}  \]
	of all minterms, respectively of all mapterms, the forward mapping are
		%\[ \forwardmapwrt{\mintermformulaset}: \bigotimes_{\atomenumeratorin}\rr^{2} \rightarrow \bigotimes_{\atomenumeratorin}\rr^{2} \]
		\[ \forwardmapwrt{\mlnmintermsymbol}(\canparam) = \normalizationofwrt{\expof{\canparam}}{\shortcatvariables}{\varnothing}
		\quad \text{and} \quad
		 \forwardmapwrt{\mlnmaxtermsymbol}(\canparam) = \normalizationofwrt{\expof{-\canparam}}{\shortcatvariables}{\varnothing} \, , \]
	where in a slight abuse of notation we assigned the variables $\shortcatvariables$ to the canonical parameters $\canparam$.

	Possible choices of the backward mappings are
		%\[ \backwardmapwrt{\mlnmintermsymbol}: \bigotimes_{\atomenumeratorin}\rr^{2} \rightarrow \bigotimes_{\atomenumeratorin}\rr^{2} \]
		\[ \backwardmapwrt{\mlnmintermsymbol}(\meanparam) = \lnof{\meanparam}
			\quad \text{and} \quad
			\backwardmapwrt{\maxtermformulaset}(\meanparam) = -\lnof{\meanparam} \, .
		 \]
\end{theorem}
\begin{proof}
	For the minterms we use that
		\[ \mintermformulaset[\shortcatvariables,\catvariableof{\mintermformulaset}]  = \identityat{\shortcatvariables,\catvariableof{\maxtermformulaset}}\]
	and get
		\[ \forwardmapwrt{\mlnmintermsymbol}(\canparam)
		= \normalizationof{
		\expof{\contractionof{\{\mintermformulaset, \canparam\}}{\shortcatvariables}}
		}{\shortcatvariables}
		=
		\normalizationof{\expof{\canparam}}{\shortcatvariables} \, .
		\]

	We notice that for any $\meanparam$ in the image of the forward map we have
		\[ \forwardmapwrt{\mlnmintermsymbol}(\backwardmapwrt{\mlnmintermsymbol}(\meanparam)) = \meanparam \]
	Therefore, $\backwardmapwrt{\mintermformulaset}$ is indeed a backward mapping to the exponential family of minterms.

	For the maxterms we use that
		\[ \maxtermformulaset[\shortcatvariables,\catvariableof{\maxtermformulaset}] = \onesat{\shortcatvariables,\catvariableof{\maxtermformulaset}}-\identityat{\shortcatvariables,\catvariableof{\maxtermformulaset}} \]
	and get
	\begin{align*}
		\forwardmapwrt{\mlnmaxtermsymbol}(\canparam)
		& = \normalizationof{
		\expof{\contractionof{\{\mintermformulaset, \canparam\}}{\shortcatvariables}}
		}{\shortcatvariables} \\
		& = \normalizationof{\{
		\expof{\contractionof{\{\ones, \canparam\}}{\shortcatvariables}},
		\expof{-\contractionof{\canparam}{\shortcatvariables}} \}
		}{\shortcatvariables} \\
		& = \normalizationof{
		\expof{-\canparam}
		}{\shortcatvariables}
	\end{align*}
	where we used, that $\expof{\contractionof{\{\ones, \canparam\}}{\shortcatvariables}}$ is a multiple of $\onesat{\shortcatvariables}$ and is thus eliminated in the normalization.
	For any $\meanparam\in\imageof{\forwardmapwrt{\mlnmaxtermsymbol}}$ we have
		\[ \forwardmapwrt{\mlnmaxtermsymbol}(\backwardmapwrt{\mlnmaxtermsymbol}(\meanparam) )
		= \meanparam
		%-\lnof{\expof{-\canparam}} + \contractionof{\expof{-\canparam}}{\varnothing} \cdot \onesat{\shortcatvariables}
		%= \canparam + \contractionof{\expof{-\canparam}}{\varnothing} \cdot \onesat{\shortcatvariables}
		\]
	and $\backwardmapwrt{\mlnmintermsymbol}$ is thus a backward map for the exponential family of maxterms.
\end{proof}

% Fitting arbitrary distributions
Any positive probability distribution can thus be fitted by minterms when we choose $\canparam=\lnof{\probtensor}$, respectively by maxterms when we choose $\canparam=\ones-\lnof{\probtensor}$.
Thus, we have identified a subset of $2^{\atomorder}$ formulas, which is rich enough to fit any distribution.





\subsect{Atomic formulas}

% Repeat atomic formulas
Let us now derive a closed form backward mapping for the statistic
	\[ \atomformulaset := \{\atomicformulaof{\atomenumerator}: \atomenumeratorin\} \, . \]

The mean parameters coincide with the queries on the atomic formulas, that is the marginal
	\[ \meanparamat{\selvariable=\atomenumerator} = \probat{\catvariableof{\atomenumerator}=1}  \, . \]

\begin{theorem}
	Given a Markov Logic Network with the statistic $\atomformulaset$ of atomic formulas, the forward mapping from canonical parameters to mean parameters is the coordinatewise sigmoid, that is
		\[ \forwardmapwrtof{\mlnatomsymbol}{\canparamat{\selvariable}} = \frac{\expof{\canparamat{\selvariable}}}{\onesat{\selvariable}+\expof{\canparamat{\selvariable}}}   \]
	where the quotient is performed coordinatewise.

	A backward mapping is the coordinatewise logit, that is
		\[ \backwardmapwrt{\mlnatomsymbol}(\meanparamwith)
		= \lnof{\frac{
			\meanparamwith
			}{
			\onesat{\selvariable}-\meanparamwith
			}}  \, . \]
\end{theorem}
\begin{proof}
	We have for any $\canparamat{\selvariable}\in\rr^{\atomorder}$
		\[ \probofat{(\atomformulaset,\canparam)}{\shortcatvariables}
		= \bigotimes_{\atomenumeratorin} \normalizationof{\expof{\canparamat{\selvariable=\atomenumerator}\cdot \atomicformulaof{\atomenumerator}}}{\catvariableof{\atomenumerator}}  \, . \]


	For any $\atomenumeratorin$ it therefore holds, that
	\begin{align*}
		\forwardmapwrtof{\mlnatomsymbol}{\canparamat{\selvariable}}[\selvariable=\atomenumerator]
		&=\contraction{\atomicformulaof{\atomenumerator},  \probofat{(\atomformulaset,\canparam)}{\shortcatvariables}} \\
		&=\contraction{\atomicformulaof{\atomenumerator},  \normalizationof{\expof{\canparamat{\selvariable=\atomenumerator}\cdot \atomicformulaof{\atomenumerator}}}{\catvariableof{\atomenumerator}}} \\
		& = \frac{\expof{\canparamat{\selvariable=\atomenumerator}}}{1+\expof{\canparamat{\selvariable=\atomenumerator}}} \, .
	\end{align*}

	Since the coordinatewise logit is the inverse function of the coordinatewise sigmoid the map
	\begin{align*}
		\backwardmapwrtof{\mlnatomsymbol}{\meanparamwith}[\selvariable=\atomenumerator]
		& = \lnof{\frac{\meanparamat{\selvariable=\atomenumerator}}{1- \meanparamat{\selvariable=\atomenumerator}}}
	\end{align*}
	satisfies for any $\meanparam$ in the image of the forward map
	\begin{align*}
		\forwardmapwrt{\mlnatomsymbol}(\backwardmapwrt{\mlnatomsymbol}(\meanparam)) = \meanparam
	\end{align*}
	and is therefore a backward map.
\end{proof}


% Representation by selection tensor networks
In a selection tensor networks they are represented by a single neuron with identity connective and variable selection to all atoms.
We will investigate such examples in more detail in \charef{cha:sparseRepresentation}, where atomic formulas Markov Logic Networks are specific cases of monomial decomposition of order 1.

% Interpretation of the result as independence approximation
The maximum likelihood estimator of a positive probability distribution by the MLN of atomic formulas is therefore the tensor product of the marginal distributions.
The Kullback-Leibler divergence between the distribution and its projection is the mutual information of the atoms, see for example Chapter~8 in \cite{mackay_information_2003}.

\begin{remark}[Decomposition into systems of atomic networks]
	\red{By Independence Decomposition we reduce to a system of atomic MLN.
	The minterms of such MLNs are the literals.
	By redundancy (literals sum up to $\ones$), it suffices to take only the positive or the negative literal.
	}
%	We set the weights of $\weightof{\lnot\atomicformulaof{\atomenumerator}}=0$ (corresponding with a gauge normalization of the energy offset symmetry). % Not needed!
\end{remark}






\sect{Constrained parameter estimation in the minterm family}

% Naive exponential family
We approach structure learning as constrained parameter estimation in the naive exponential family (see \secref{sec:mintermExpFamily}), which coincides with the minterm family $\formulasetof{\mlnmintermsymbol}$.
The minterm family is defined by the statistic $\sstat = \identityat{\shortcatvariables, \selvariableof{[\catorder]}}$ and has energy tensors coinciding with the canonical parameters.

% Convex polytope characterization
\red{For the minterm family, we have as mean parameter set the convex hull of one-hot encodings.
Each basis vector is an extreme point is an extreme point.
}


By \theref{the:mintermExpressivityMLN} all positive distributions are member of the minterm markov logic network family.
This expressivity result was generalized to arbitrary distributions, when allowing for formulas as basemeasures by \theref{the:mintermExpressivityHLN}.

Finding the distribution maximizing the likelihood of data would then be the empirical distribution.
In this case we would have $\datameanat{\selvariableof{[\catorder]}=\shortcatindices} = \empdistributionat{\shortcatvariables=\shortcatindices}$ and the maximum likelihood distribution is found by the problem
\begin{align*}
	\argmax_{\canparam\in\facspace}  \contraction{\canparam,\empdistribution} - \cumfunctionof{\canparam} \,
\end{align*}
which is solved at $\canparam=\lnof{\empdistribution}$ with $\probtensorof{(\identity,\lnof{\empdistribution})}= \empdistribution$.
This follows from $\lossof{\probtensorof{(\identity,\canparam)}}=\kldivof{\empdistribution}{\probtensorof{(\identity,\canparam)}}$, which is by Gibbs inequality minimized at $\probtensorof{(\identity,\canparam)}=\empdistribution$, which is the case for $\canparam = \lnof{\empdistribution}$.

We here allow for $\lnof{0}=-\infty$, with the convention of $\expof{-\infty}=0$, to handle datasets where specific worlds are not represented.
\red{Better: Use \theref{the:mintermExpressivityHLN} with basemeasure dropping non appearing data.}


% Regularization
To avoid this overfitting situation, we regularize by restricting the parameter to be a set $\energyhypothesis\subset\facspace$ and state
\begin{align}\tag{$\mathrm{P}_{\energyhypothesis, \empdistribution}$}\label{prob:restrictedNaiveMLE}
	\argmax_{\canparam\in\energyhypothesis}  \contraction{\canparam,\empdistribution} - \cumfunctionof{\canparam} \, .
\end{align}

Problem~\ref{prob:restrictedNaiveMLE} has two important types of instantiation, which we discuss in the next sections.

\subsect{Parameter Estimation}

% Parameter Estimation
\red{Projecting onto the markov logic family to the statistic $\formulaset$ is the instance of Problem~\ref{prob:restricedNaiveMLE} with the hypothesis choice}
%When the $\formulaset$ is known we take $\energyhypothesis$ as the linear hull 
	\[ \energyhypothesisof{\formulaset} = \spanof{\{\formula : \formula\in\formulaset \}} \, . \]
Then, the problem is the parameter estimation problem studied in \secref{sec:parameterEstimation}.
To see this, we reparametrize by the coefficient vectors of the elements in the span, which are then understood as the canonical parameter of the respective distribution in the markov logic family to $\formulaset$.


\begin{remark}[Overparametrization]
	Taking $\formulaset$ to consist of all propositional formulas, we get a massive overparametrization:
	The essential statistics maps to a $2^{\left(2^\atomorder \right)}$ dimensional real vector space.
	All possible distributions of the $\atomorder$ atomic variables are mapped to an $2^\atomorder-1$ dimensional submanifold, where also the essential statistics maps to.

	Thus, to identify probabilistic knowledge bases, we need to drastically restrict the shape of formulas allowed.
	It is in principle impossible to decide which formulas to be activated, based only on statistics and not on prior assumptions.

	%The nodes of a Markov Propositional Network are all formulas in a propositional theory and the hyperedges all possible decompositons.
	When having $\atomorder$ atoms, there are $2^{\atomorder}$ states in the factored system.
	Since each state can either be a model of a formula or not, there are
		\[ \cardof{\formulaset} = 2^{\big(2^\atomorder \big)} \]
	formulas.
	Having, for example, $\atomorder=10$, then $\cardof{\formulaset}>10^{308}$.


	% Regularization by sparsity
	One regularization is by allowing only a small number of formulas to be active.
	This corresponds with regularization with $\sparsityof{\canparam}$.
	The problem is then non-convex.


	% Regularization by formula size
	A further regularization strategy is the restriction of the size of the possible formulas to maintain interpretability.
	Thus, we choose small formula selection networks.
\end{remark}




\subsect{Structure Learning}

% Structure Learning
The problem of structure learning arises, when the set of parameters in Problem~\ref{prob:restricedNaiveMLE} is choosen as
	\[ \energyhypothesisof{\formulasuperset}= \bigcup_{\formulaset\in\formulasuperset} \spanof{\formulaset} \, .  \] %\energyhypothesisof{\formulaset}\, .
In this case, the problem in general fails to be convex.

% Subspace instuition
Each formula set $\formulaset$ represents a subspace in the parameters of the minterm family, which is spanned by the propositional formulas $\exformula\in\formulaset$.

%\red{Intuition by subspaces in the minterm parameters, which are selected by a nonlinear objective, to distinguish from compressed sensing.}







\sect{Greedy Structure Learning}


%Motivation 
It can be impracticle to learn all formulas at once, since the set $\formulasuperset$ often grows combinatorically, for example when choosing as a powerset of formulas.
\red{Further, we need to avoid overfitting and carefully choose a hypothesis.}
To avoid intractabilities and overfitting, one can choose a greedy approach and learn in addition formulas $\exformula$ when already having learned a set $\formulaset$ of formulas.
We in this section assume a current model $\currentdistribution$, which is a generic positive distribution not necessarily a Markov Logic Network. % or Hybrid Logic Network.

% 
We will use the effective selection tensor network representation of exponentially many formulas described in \charef{cha:formulaSelection} and select from them a small subset.

%\red{Alternative discussion: Can use current distribution as base measure and apply moment matching as first order condition.}


\subsect{Greedy formula inclusions}

Having a current set of formulas $\formulaset$ we want to choose the best $\formula\in\fselectionmap$ to extend the set of formulas to $\formulaset\cup\{\formula\}$ in a way minimizing the cross entropy.
Given this, add each step we solve the greedy cross entropy minimization
\begin{align}\label{prob:perfectGreedy}\tag{$\mathrm{P}_{\datamap,\formulaset,\fselectionmap}$}
	\argmin_{\formula\in\fselectionmap} \argmin_{\canparam\in\rr^{\cardof{\formulaset}+1}}
	\centropyof{\empdistribution}{\expdistof{(\formulaset\cup\{\formula\},\canparam,\basemeasure)}} \, .
\end{align}


A brute force solution would require parameter estimation for each candidate in $\fselectionmap$.
We provide two more efficient approximative heuristics in the following (see Chapter~20 in \cite{koller_probabilistic_2009}).


\subsect{Gain Heuristic}

In the gain heuristic, only the parameters of the new formula are optimized and the others left unchanged.
This amounts to
\begin{align}\label{prob:greedyGain}\tag{$\mathrm{P}^{\mathrm{gain}}_{\datamap,\formulaset,\fselectionmap}$}
	\argmin_{\formula\in\fselectionmap} \left ( \min_{\canparamat{\cardof{\formulaset}}\in\rr}
	\centropyof{\empdistribution}{\expdistof{(\formulaset\cup\{\formula\},\canparam,\basemeasure)}} \right) \, .
\end{align}
Here we denote by $\canparam$ the first $\cardof{\formulaset}$ coordinates of the M-projection $\currentdistribution$  of $\empdistribution$ onto $\formulaset$ and the variable new coordinate at position $\canparamat{\cardof{\formulaset}}$.

\begin{lemma}
	The gain heuristic objective is an upper bound on the true greedy objective.
\end{lemma}
\begin{proof}
Since
\begin{align*}
	&\argmin_{\formula\in\fselectionmap} \left( \argmin_{\canparam\in\rr^{\cardof{\formulaset}+1}}
	\centropyof{\empdistribution}{\expdistof{(\formulaset\cup\{\formula\},\canparam,\basemeasure)}} \right) \\
	&\quad \leq 	\argmin_{\formula\in\fselectionmap} \left ( \argmin_{\canparamat{\cardof{\formulaset}}\in\rr}
	\centropyof{\empdistribution}{\expdistof{(\formulaset\cup\{\formula\},\canparam,\basemeasure)}} \right) \, .
\end{align*}
\end{proof}


% Minterm family interpretation
Further, this is \probref{prob:restrictedNaiveMLE} in the case
\begin{align*}
	\energyhypothesis = \lnof{\currentdistribution} + \cup_{\formula\in\formulaset} \spanof{\formula} \, .
\end{align*}



% For single formula
Let us choose a formula $\formula\in\formulaset$ and consider Problem~\ref{prob:restrictedNaiveMLE}  in the case
\begin{align*}
	\energyhypothesisof{\formula} = \lnof{\currentdistribution} + \spanof{\formula} \, .
\end{align*}
This is parameter estimation on the exponential family with the single feature $\formula$ and the base measure $\currentdistribution$.
Therefore we can apply the theory of \charef{cha:probReasoning} and characterize the solution by the $\weight$ satisfying the moment matching condition
\begin{align*}
	\contraction{\currentdistribution, \normalizationof{\expof{\weight}}{\shortcatvariables} } = \contraction{\empdistribution, \formula} \, .
\end{align*}
We state the solution of this condition in the next theorem.

\begin{theorem}
	Problem~\eqref{prob:greedyGain} is solved at any
	\begin{align*}
		\hat{\canparam} = \weightof{\hat{\formula}} \cdot \hat{\formula}
	\end{align*}
	where the formula $\hat{\formula}$ is in
	\begin{align*}
		\hat{\formula} \in \argmax_{\formula\in\formulaset} \kldivof{\contraction{\empdistribution,\formula}}{\contraction{\currentdistribution,\formula}}
	\end{align*}
	and $\weightof{\hat{\formula}}$ is the weight of $\hat{\formula}$ in the solution of Problem~\ref{prob:restrictedNaiveMLE} with $\Gamma = \currentdistribution + \mathrm{span}(\exformula)$.
	Here we denote by $\kldivof{p_1}{p_2}$ the Kullback-Leibler divergence between Bernoulli distributions with parameters $p_1,p_2\in[0,1]$, that is
		\[ \kldivof{p_1}{p_2} = p_1 \cdot \lnof{\frac{p_1}{p_2}} + (1-p_1) \cdot \lnof{\frac{(1-p_1)}{(1-p_2)}}  \]
\end{theorem}
\begin{proof}
	% Solution of the problem restricted to
	For any formula $\formula$, the inner minimum of Problem~\eqref{prob:greedyGain} is by \lemref{ref:lemMMinMLN} taken at
		\[ \weightof{\formula} = \lnof{\frac{\datamean}{(1-\datamean)}\cdot \frac{(1-\currentmean)}{\currentmean}}  \]
	where
		\[ \currentmean = \contraction{\currentdistribution,\formula} \]
	and
		\[ \datamean = \contraction{\empdistribution,\formula} \, . \]

	The difference of the likelihood at the current distribution and the optimum is
	\begin{align*}
		\centropyof{\empdistribution}{\currentdistribution}
		- \centropyof{\empdistribution}{\expdistof{(\extendedformulaset,\extendedcanparam,\basemeasure)}}
		= \datamean \cdot \weightof{\formula} - \cumfunctionwrtof{\extendedformulaset,\basemeasure}{\extendedcanparam} \, .
	\end{align*}

	% Loss gain at optimum
	We use the representation scheme of Theorem~\ref{the:hybridNetworkRepresentation} and get
	\begin{align*}
		\contraction{\currentdistribution, \expof{\weightof{\formula} \cdot \formula}}
		& = \contraction{\currentdistribution, \bencodingofat{\formula}{\catvariableof{\formula}}, \actcoreofat{\formula}{\catvariableof{\formula}}} \\
		& = (1-\currentmean) + \currentmean\cdot \expof{\weightof{\formula}} \\
		& = (1 - \currentmean) + \frac{\datamean \cdot (1-\currentmean)}{(1-\datamean)} \\
		& = (1-\currentmean) \cdot \frac{1}{(1-\datamean)} \, .
	\end{align*}
	% Refining the cumulant term
	It follows, that
	\begin{align*}
		\cumfunctionwrtof{\extendedformulaset,\basemeasure}{\extendedcanparam}
		& = \lnof{\contraction{\currentdistribution, \expof{\weightof{\formula} \cdot \formula}}} \\
		& = \lnof{1-\currentmean} - \lnof{1-\datamean} \, .
	\end{align*}
	% Refining the mean product term
	We further have
	\begin{align*}
		\datamean \cdot \weightof{\formula}
		= \datamean \cdot \left[ \lnof{\frac{\datamean}{(1-\datamean)}\cdot \frac{(1-\currentmean)}{\currentmean}}  \right]
		= \datamean \lnof{\datamean} - \datamean \lnof{1-\datamean} + \datamean \lnof{1-\currentmean} - \datamean \lnof{\currentmean}
	\end{align*}
	and arrive at
	\begin{align*}
		& \centropyof{\empdistribution}{\currentdistribution}
		- \centropyof{\empdistribution}{\expdistof{(\exformula,\weightof{\formula},\currentdistribution)}} \\
		& \quad =  \datamean \lnof{\datamean} - \datamean \lnof{1-\datamean} + \datamean \lnof{1-\currentmean} - \datamean \lnof{\currentmean}
		-  \lnof{1-\currentmean} - \lnof{1-\datamean} \\
		& \quad = \left( -\datamean \lnof{\currentmean} - (1-\datamean) \lnof{1-\currentmean} \right)  - \left( -\datamean \lnof{\datamean} - (1-\datamean) \lnof{1-\datamean} \right) \, .
	\end{align*}
	By definition, this is the Kullback-Leibler divergence between Bernoulli distributions with parameters $\datamean$ and $\currentmean$.
	%
	Since the gain in the likelihood loss when restricting to $\energyhypothesis = \spanof{\formula}$ is thus given by $\kldivof{\contraction{\empdistribution,\formula}}{\contraction{\currentdistribution,\formula}}$, we have that Problem~\ref{prob:restrictedNaiveCE}  in the case $\energyhypothesis = \bigcup_{\formula\in\formulaset}\spanof{\formula}$ is solved at $\estcanparam = \weightof{\hat{\formula}}\cdot \hat{\formula}$ where
		\[ \hat{\formula} = \kldivof{\contraction{\empdistribution,\formula}}{\contraction{\currentdistribution,\formula}} \, . \]
\end{proof}

\red{Thus, we solve the grain heuristic with a coordinatewise transform of the mean parameter tensors to $\empdistribution$ and $\currentdistribution$, using the Bernoulli Kullback-Leibler divergence as transform function.}


% Interpretation
One therefore takes the formula, which marginal distribution in the current model and the targeted distribution are differing at most, measured in the KL divergence.

% Optimization method
One optimization method would thus be the computation of the mean parameters to both distribution, building the coordinatewise KL divergence and choosing the maximum.
Since we need to evaluate each coordinate, this can be intractable for large sets of formulas.


% Further weight optimization
Further improvement of the model can be achieved by iteratively optimizing the other weights as well, since their corresponding moment matching conditions might be violated after the integration of a new formula.
This would require the computation of backward mappings for each candidate formula, for which we only have an alternating approach in general.



\subsect{Gradient heuristic and the proposal distribution}

\red{Advantage: Might avoid formulawise calculus, when sampling from proposal distribution.
Brute force solution of gain heuristic require formulawise approach.}

We now derive a heuristic of choosing features based on the maximal coordinate of the gradient when differentiating the canonical parameter in the minterm family.
To prepare for this, we build the gradient of the loss
%For the naive exponential family 
\begin{align*}
	\lossof{\expdistof{(\naivestat, \naivecanparam)}}
	%= \frac{1}{\datanum} \sum_{\datindexin}\lnof{\expdistofat{(\naivestat, \naivecanparam)}{\shortcatvariables=\datamapat{\datindex}}}
	= \contraction{\empdistribution, \sencodingof{\naivestat}, \naivecanparam} - \lnof{\contraction{\expof{\contractionof{\sencodingof{\naivestat}, \naivecanparam}{\shortcatvariables}}}}
\end{align*}
as
\begin{align*}
	\gradwrt{\naivecanparamat{\selvariable}} \lossof{\expdistof{(\naivestat, \naivecanparam)}}
	&= \contractionof{\sencodingof{\naivestat},\empdistribution}{\selvariable} - \contractionof{\sencodingof{\naivestat},\expdistof{(\naivestat, \naivecanparam)}}{\selvariable} \\
	&= \empdistribution - \expdistof{(\naivestat, \naivecanparam)} \, .
\end{align*}

%% Single feature
%Given a feature $\exfunction[\shortcatvariables]$ we vary the naive parameters by a function on $\canparam\in\rr$ by
%\begin{align*}
%	 \naivestat(\canparam) %=  \mlntensor + \weight_{\selindices} \bencodingof{\exformula_{\selindices}}
%	= \naivestat(0) + \canparam\cdot\exfunction
%\end{align*}
%and get a likelihood gradient of
%\begin{align*}
%	 \frac{\partial \lossof{\expdistof{(\naivestat(\canparam), \naivecanparam)}}}{\partial\canparam} 
%	 &= \contraction{
%	 	\frac{\partial\lossof{\expdistof{(\naivestat, \naivecanparam)}}}{\partial\naivecanparam}|_{\naivecanparam(0)},
%		\frac{\partial\naivecanparam(\canparam)}{\partial\canparam} 
%	 }  \\
%	 &= \contraction{\empdistribution,\exfunction} -   \contraction{\expdistof{(\naivestat, \naivecanparam)},\exfunction} \, .
%\end{align*}


%% Positive and Negative Search
The gradient shows the typical decomposition into a positive and a negative phase.
While the positive phase comes from the data term and prefers directions of large data support, the negative phase originates in the partition function and draws the gradient away from directions already supported by the current model $\expdistof{(\naivestat, \naivecanparam)}$.
%% Regularization functionality
The negative phase is a regularization, by comparing with what has already been learned.
When nothing has been learned so far, we can take the current model to be the uniform distribution, which is the naive exponential family with vanishing canonical parameters.



%% Collection of features by selection
Given a set $\fselectionmap$ of features we vary $\naivecanparam$ by the function
\begin{align*}
	 \exfunction(\canparam) = \naivecanparam + \contractionof{\canparam,\sencodingof{\fselectionmap}}{\shortcatvariables} \, .
\end{align*}
At $\canparam=0$ we have the gradient of the loss of the parametrized formula by
\begin{align*}
	 \gradwrtat{\canparam}{0}
	 \lossof{\expdistof{(\naivestat,\exfunction(\canparam),\basemeasure)}}
	 &= \contraction{
	 	 \gradwrtat{\exfunction(\canparam)}{\naivecanparam}  \lossof{\expdistof{(\naivestat,\exfunction(\canparam),\basemeasure)}},
		 \gradwrtat{\canparam}{0}  \exfunction(\canparam)
	 }  \\
	 &= \contractionof{\empdistribution,\sencodingof{\sstat}}{\selvariable} -   \contractionof{\expdistof{(\naivestat, \naivecanparam, \basemeasure)},\sencodingof{\sstat}}{\selvariable} \, .
\end{align*}


%% Grafting
We want to choose the formula, which is best aligned with the gradient of the log-likelihood, that is using a formula selecting map $\fselectionmap$
\begin{align} \label{prob:greedyGrad} \tag{$\mathrm{P}^{\mathrm{grad}}_{\datamap,\formulaset,\fselectionmap}$}
	\argmax_{\selindex\in[\seldim]} \contractionof{\empdistribution,\fselectionmap}{\indexedselvariable}
	- \contractionof{\expdistof{(\naivestat, \naivecanparam, \basemeasure)},\fselectionmap}{\indexedselvariable} \, .
\end{align}
This method is known as the gradient heuristic or grafting.
% Mean parameter interpretation
The objective of Problem~\eqref{prob:greedyGrad} has another interpretation by the difference of the mean parameter $\datamean$ and $\currentmean$ of the projections of the empirical and current distributions on the family to $\fselectionmap$. % ! NOT the proposal family, those have transposed statistic

%% Formula alignment perspective
Problem~\eqref{prob:greedyGrad} is further equivalent to the formula alignment
\begin{align*}
	\argmax_{\formula\in\fselectionmap} \contraction{\formula,\empdistribution-\currentdistribution} \, .
\end{align*}
The objective can be interpreted as the difference of the satisfaction probability of the formula with respect to the empirical distribution and the current distribution.
%We can choose selection architectures to efficiently parametrize the formulas in the hypothesis $\fselectionmap$ and rewrite the problem as
%\begin{align*}
%	\argmax_{\selindexin} \contractionof{ \gradwrtat{\canparam}{\canparam=0} \lossof{\expdist}}{\indexedselvariable}
%\end{align*}
%This is thus equivalent to the problem \ref{prob:greedyGrad}, when taking all formulas selectable by $\formulaset$ as the hypothesis $\Gamma$.












\subsect{Iterations}

Let us now iterate the search for a best formula at a current model with the optimization of weights after each step.
The result is Algorithm~\ref{alg:greedyStructureLearning}, which is a greedy algorithm adding iteratively the currently best feature.

\begin{algorithm}[hbt!]
\caption{Greedy Structure Learning}\label{alg:greedyStructureLearning}
\begin{algorithmic}
	\Require Empirical distribution $\empdistribution$, hypothesis $\fselectionmap$ of formulas
	\Ensure Distribution $\expdist$ approximating $\empdistribution$
	\hrule
	\State Initialize
		\[ \currentdistribution \algdefsymbol \frac{1}{\prod_{\catenumeratorin}\catdimof{\atomenumerator}} \cdot \onesat{\shortcatvariables} \quad, \quad \formulaset = \varnothing \]
	\While{Stopping criterion is not met}
		% REFINE! Work in data
		\State \textbf{Structure Learning:} Compute a (approximative) solution $\hat{\formula}$ to Problem~\ref{prob:restrictedNaiveMLE} and add the formula to $\formulaset$, i.e.
				\[ \formulaset \algdefsymbol \formulaset \cup\{\hat{\formula}\} \]
			Extend dimension of $\selvariable$ by one, by $\formulaof{\seldim}=\hat{\formula}$ and $\canparamat{\seldim}=0$
		\State \textbf{Weight Estimation:} Estimate the best weights for the added formula and recalibrate the weights of the previous formulas, by calling Algorithm~\ref{alg:AWO}.
				\[ \currentdistribution \algdefsymbol \expdistof{\formulaset, \canparam} \]
\EndWhile
	\State \Return $\formulaset$, $\canparam$ %, $\kb$
\end{algorithmic}
\end{algorithm}



%% Energy Storage -> Useful after learning for energy-based inference
When having used the same learning architecture multiple times, the energy of the corresponding formulas are all representable by a formula selecting architecture.
Their energy term is therefore a contraction of the selecting tensor with a parameter tensor $\canparam$ in a basis CP decomposition with rank by the number of learned formulas.
When mutiple selection architectures have been used, the energy is a sum of such contractions.
% 
Let us note, that this representation is useful after learning, when performing energy-based inference algorithms on the result.
During learning, one needs to instantiate the proposal distribution, which requires instantiation of the probability tensor.
\red{However, one could alternate data energy-based and use this as a particle-based proxy for the probability tensor.}


\begin{remark}[Sparsification by Thresholding]
	To maintain a small set of active formulas, one could combine greedy learning approaches with thresholding on the coordinates of $\canparam$.
	This is a standard procedure in Iterative Hard Thresholding algorithms of Compressed Sensing, but note that here we do not have a linear in $\canparam$ objective.
\end{remark}




\sect{Proposal distribution}


% Proposal distribution
Let us now understand the likelihood gradient as the energy tensor of a probability distribution, which we call the proposal distribution.

\begin{definition}[Proposal Distribution]
	Let there be a base distribution $\currentdistribution$, a targeted distribution $\empdistribution$ and a formula selecting map $\fselectionmap[\shortcatvariables, \selvariable]$.
	The proposal distribution at inverse temperature $\invtemp>0$ is the distribution of $\selvariable$ defined by
	\begin{align*}
		\normalizationof{\expof{\contractionof{\invtemp\cdot(\empdistribution-\currentdistribution),\fselectionmap}{\selvariable}} }{\selvariable} \, .
	\end{align*}
	The proposal distribution is the member of the exponential family with statistics $\fselectionmap$ and parameter $\invtemp\cdot(\empdistribution-\currentdistribution)$.
\end{definition}


%. Exponential family
The proposal distribution is in the exponential family with sufficient statistic by the formula selecting map $\fselectionmap$, namely the member with the canonical parameters $\canparam=\empdistribution-\currentdistribution$.
Of further interest are tempered proposal distributions, which are in the same exponential family with canonical parameters $\invtemp\cdot(\empdistribution-\currentdistribution)$ where $\invtemp>0$ is the inverse temperature parameter.

% MLN
As Markov Logic Networks, the proposal distributions are in exponential families with the sufficient statistic defined in terms of formula selecting maps.
While Markov Logic Networks contract the maps on the selection variables $\selvariable$, the proposal distributions contract them along the categorical variables $\catvariable$ to define energy tensors.

% Methods to solve mode search
The grafting Problem~\eqref{prob:greedyGrad} is the search for the mode of the proposal distribution.
To solve grafting, we thus need to answer a mode query, for which we can apply the methods introduced in \charef{cha:probReasoning}, such as Gibbs Sampling or Mean Field Approximations in combination with annealing.


\subsect{Mean parameter polytope}

The mean parameter polytope of the proposal distribution with statistic $\proposalstat$ is the convex hull of the formulas in $\formulaset$, that is
\begin{align*}
	\meansetof{\proposalstat}
	= \convhullof{\sencodingof{\proposalstat}{\indexedselvariable,\shortcatvariables} \, : \, \selindexin{}}
	= \convhullof{\formulaat{\shortcatvariables} \, : \, \formula\in\fselectionmap}
\end{align*}


% 0/1
As it was the case for Markov Logic Networks, the mean parameter polytopes are instances of a $0/1$-polytopes \cite{ziegler_lectures_2000,gillmann_01-polytopes_2007}.

% Interpretation as formulas
The extreme points are the formulas selectable by the formula selecting map $\fselectionmap$.


\sect{Discussion}

\begin{remark}[Bayesian approach]
	We only treated the estimation of a single resulting distribution by the data, while in a Bayesian approach one typically considers an uncertainty over possible distributions.
	% MAP
	\red{When treating $\canparam$ as a random tensor, which prior distribution is given and posteriori distribution wanted, we have a more involved Bayesian approach.}
	When having a prior $\probat{\mlnparameters}$ over the Markov Logic Networks we alternatively want to find the parameters $\mlnparameters$ solving the maximum a posteriori problem
	\begin{align}
		\argmax_{\mlnparameters} \mlnprobat{\data}\cdot \probat{\mlnparameters}\, .
	\end{align}
\end{remark}

To summarize some insights on the mean polytopes $\hlnmeanset$:
\begin{itemize}
	\item If and only íf all coordinates are in $\{0,1\}$ then an extreme points, then $\meanparam$ is reproduced by a hard logic network.
	\item If some mean params in $\{0,1\}$, then not in the interior, and not reproduced by a markov logic network.
		Back direction not correct: There are interior points where no coordinate in $\{0,1\}$.
	\item If not in the interior, we can identify with base measure refinement a base measure, such that reproducable by a distribution representable by the base measure.
\end{itemize}


% Polytopes - MLN 
The polytopes of mean parameters to hybrid logic networks and proposal distributions are an interesting connection between the fields of combinatorical optimization and the study of expressivity of tensor networks.
% Minimal Connectivity: Local consitency - Hierarchical Tucker
This is of special interest, when the computation cores of a hybrid logic network are minimally connected, the mean parameters are captured by local consistencies.
Similar investigations have been made in the field of tensor networks, where minimal connected tensor networks are refered to by Hierarchical Tucker formats (HT).
Minimal connection is exploited in the tensor network community to show numerical properties of the format, such as closedness and existence of best approximators.
















    \chapter{\chatextconcentration}\label{cha:concentration}

When drawing data independently from a random distribution, we are limited by random effects.
We in this chapter derive guarantees, that the learning methods introduced in \charef{cha:probReasoning} and \charef{cha:networkReasoning} are robust against such effects.

\sect{Fluctuations of random data}

A random tensor is a random element of a tensor space $\facspace$, drawn from a probability distribution on $\facspace.$
In contrast to the discrete distributions investigated previously in this work, the random tensors are in most generality continuous distributions. % However, when drawing data they are 

\subsect{Fluctuation of the empirical distribution}

% Random one hot encodings
When drawing random states $\datapoint\in\facstates$ by a distribution $\gendistribution$, we use the one-hot encoding to forward each random state to the random tensor
\[ \onehotmapofat{\datapoint}{\shortcatvariables} \, . \]
The expectation of this random tensor is
\begin{align*}
    \expectationof{\onehotmapof{\datapoint}}
    = \sum_{\shortcatindices\in\facstates} \gendistributionat{\indexedshortcatvariables} \onehotmapofat{\shortcatindices}{\shortcatvariables}
    = \gendistributionat{\shortcatvariables} \, .
\end{align*}

The empirical distribution is then the average of independent random one-hot encodings, namely the random tensor
\[ \empdistribution = \frac{1}{\datanum} \sum_{\datindexin}  \onehotmapofat{\datapoint}{\shortcatvariables} \, . \]
To avoid confusion let us strengthen, that in this chapter we interpret $\empdistribution$ as a random tensor taking values in $\facspace$, whereas each supported value of $\empdistribution$ is an empirical distribution taking values in $\facstates$.
The forwarding of $\facstates$ under the one-hot encoding is a multinomial random variable, see \theref{the:multinomialEmpdistFluctuation}.


% Expectation -> Does not make use of independence here!
When the marginal of each datapoint is $\gendistribution$, the expectation of the empirical distribution is
\begin{align*}
    \expectationof{\empdistribution}
    = \frac{1}{\datanum} \sum_{\datindexin}  \expectationof{\onehotmapof{\datapoint}}
    = \gendistribution \, .
\end{align*}

% Law of large numbers
From the law of large numbers it follows, that in the limit of $\datanum\rightarrow\infty$ at any coordinate $\catindex\in\facstates$ almost everywhere
\[ \empdistributionat{\indexedshortcatvariables} \rightarrow \expectationof{\empdistributionat{\indexedshortcatvariables}} =  \gendistributionat{\indexedshortcatvariables} \, . \]

% Fluctuation
At finite $\datanum$ the empirical distribution differs from the by the difference
\[ \empdistribution - \gendistribution \]
which we call a fluctuation tensor.

\subsect{Mean parameter of the empirical distribution}

We now investigate the empirical mean parameter
\[
    \datameanat{\selvariable} = \contractionof{\sencsstatwith,\empdistributionat{\shortcatvariables}}{\selvariable} \, .
\]

Each coordinate of $\datamean$ is decomposed as
\[ \datameanat{\indexedselvariable} = \frac{1}{\datanum}\sum_{\datindexin} \sstatcoordinateofat{\selindex}{\datapointof{\datindex}} \]
and thus stores the empirical average of the feature $\sstatcoordinateof{\selindex}$ on the dataset $\data$.

% Expectation of the empirical mean
Since the mean parameter depends linearly on the corresponding distribution, we can show the following correspondence between the empirical and the expected mean parameter.

\begin{theorem}
    \label{the:expectedMeanParameter}
    When drawing data independently from $\gendistribution$, we have $\expectationof{\datameanat{\selvariable}}=\genmeanat{\selvariable}$, where we call
    \[
        \genmeanat{\selvariable} = \contractionof{\sencsstatwith,\empdistributionat{\shortcatvariables}}{\selvariable} \,
    \]
    the expected mean parameter.
\end{theorem}
\begin{proof}
    Since the expectation commutes with linear functions.
%    Since the mean parameter of a distribution depends linearly on the distribution.
\end{proof}


% Convergence by Law of Large Numbers and issues
For each $\selindexin$ the law of large numbers guarantees that $\genmeanat{\indexedselvariable}$ converges almost surely against $\genmeanat{\indexedselvariable}$ when $\datanum\rightarrow\infty$.
To utilize these we need to approach the following issues:
\begin{itemize}
    \item We need non-asymptotic convergence bounds, since one has access to finite data when learning
    \item The convergence has to happen uniformly for all $\selindexin$
    \item Guarantees on the result of an estimated model are more accessible when provided for quantities like the canonical parameter and KL-divergences of the learning result.
    Those, however, depend nonlinearly on $\datameanat{\selvariable}$ and therefore require further investigation.
\end{itemize}

\subsect{Noise tensor and its width}

% Definition of noise tensors
Motivated by \theref{the:expectedMeanParameter}, we build our derivation of probabilistic guarantees on non-asymptotic and uniform convergence bounds for $\datameanat{\selvariable}$.
Let us first define the fluctuations of the empirical mean parameter, when drawing the data independently from a random distribution, as the noise tensor.

\begin{definition}
    \label{def:noiseTensor}
    Given a statistic $\sstat$, $\datanum\in\nn$ and a distribution $\gendistribution$, we call
    \begin{align*}
        \sstatnoise = \contractionof{(\empdistribution-\gendistribution),\sencsstat}{\selvariable}
    \end{align*}
    the \emph{noise tensor}, where $\datamap$ is a collection of $\datanum$ independent samples of $\gendistribution$.
\end{definition}

% Minterm
The fluctuation of the empirical distribution around the generating distribution corresponds in this notation with the minterm exponential family, taking the identity as statistics.
% Appearances
Besides this, fluctuation tensors appears in \MarkovLogicNetworks{} as fluctuations of random mean parameters and in proposal distributions as fluctuation of random energy tensor.
We will discuss these examples in the following sections.


% Fluctuation of mean parameter
We notice, that the fluctuation tensor $\sstatnoise$ is the centered mean parameter to the empirical distribution, that is
\begin{align*}
    \datamean - \expectationof{\datamean} =  \contractionof{\sencsstat,\empdistribution-\gendistribution}{\selvariable} \, .
\end{align*}

% Widths
In the following we will use the supremum of contractions with random tensors in the derivation of success guarantees for learning problems.
Such quantities are called widths.

\begin{definition}
    \label{def:width}
    Given a set $\canparamhypothesis\subset\facspace$ and $\sstatnoise$ a random tensor taking values in $\facspace$ we define the width as the random variable
    \[ \widthwrtof{\canparamhypothesis}{\sstatnoise} = \sup_{\canparamin} \absof{\contraction{\canparam,\sstatnoise}} \, . \]
\end{definition}

% Uniform concentration events
Bounds on the widths are also called uniform concentration bounds \cite{goesmann_uniform_2021} and generic probabilistic bounds will be provided in \secref{sec:widthBounds}.

\sect{Error bounds based on the noise width}

We now derive error bounds for parameter estimation and structure learning, as introduced in \charef{cha:networkReasoning}.
When combined with probabilistic bounds on the noise width, they are probabilistic success guarantees.

\subsect{Parameter Estimation}

\red{We in this section always assume, that $\empdistribution$ is representable by the base measure $\basemeasure$ of the respective exponential families.}

Parameter Estimation is the M-projection of the empirical distribution onto an exponential family.
In \charef{cha:probReasoning} we have characterized those by the backward map acting on the mean parameter.
Thus, while we are interested in the expected canonical parameter
\[
    \gencanparamat{\selvariable} = \backwardmapof{\genmeanat{\selvariable}}
\]
we get an estimation by the empirical canonical parameter
\[
    \datacanparamat{\selvariable}  = \backwardmapof{\datameanat{\selvariable}} \, .
\]

% Nonlinearity
Unfortunately, since the backward map is not linear, we in general do not have that $\expectationof{\backwardmapof{\datamean}}$ coincides with $\backwardmapof{\genmean}$.
To build intuition on the concentration we recall the expression of the backward map as
% Concentration
\begin{align*}
    \backwardmapof{\meanparam}
    = \argmax_{\canparam} -\centropyof{\meanrepprob}{\stanexpdistof{\canparam}}
\end{align*}
where $\meanrepprob$ is any distribution reproducing the mean parameter.
We want to compare the solutions $\backwardmapof{\datamean}$ and $\backwardmapof{\genmean}$, in which case $\meanrepprob$ can be chosen as $\empdistribution$ and $\gendistribution$.
It is common to call the objectives $\centropyof{\empdistribution}{\stanexpdistof{\canparam}}$ and $\centropyof{\gendistribution}{\stanexpdistof{\canparam}}$ empirical and expected risk \cite{shalev-schwartz_shai_understanding_2014}
Since the empirical risk has a linear dependence on $\datamean$, we have at each $\canparam$
\begin{align*}
    \expectationof{\centropyof{\empdistribution}{\stanexpdistof{\canparam}}}
    &= \expectationof{\contraction{\datamean,\canparam} - \cumfunctionof{\canparam}} \\
    &= \contraction{\expectationof{\datamean},\canparam} - \cumfunctionof{\canparam} \\
    &= \centropyof{\gendistribution}{\stanexpdistof{\canparam}}
\end{align*}
By the law of large numbers, in the limit $\datanum\rightarrow\infty$ we thus have at each $\canparam$ a convergence of the empirical risk to the expected risk.
However, since the backward map is defined by the minima of these risks, we need a uniform and non-asymptotical concentration guarantee to get more useful bounds.
To this end, we now consider constrained parameter estimation and relate the supremum on the differences between expected and empirical risks with the width of the noise tensor.

\begin{lemma}
    \label{lem:centropyWidthCharacterization}
    For any $\canparamhypothesis$ and $\datamap$ we have
    \begin{align*}
        \widthwrtof{\canparamhypothesis}{\sstatnoise}
        = \sup_{\canparamin} \absof{\centropyof{\empdistribution}{\stanexpdistof{\canparam}} - \centropyof{\gendistribution}{\stanexpdistof{\canparam}}} \, .
    \end{align*}
\end{lemma}
\begin{proof}
    For any $\canparam\in\canparamhypothesis$ and by $\meanrepprob$ realizable mean parameter $\meanparam$ we have
    \begin{align*}
        \centropyof{\meanrepprob}{\stanexpdistof{\canparam}}
        = - \contraction{\meanparam,\canparam} + \cumfunctionof{\canparam} \, .
    \end{align*}
    It follows that
    \begin{align*}
        \centropyof{\empdistribution}{\stanexpdistof{\canparam}} - \centropyof{\gendistribution}{\stanexpdistof{\canparam}}
        = -\contraction{(\datamean-\genmean),\canparam}
    \end{align*}
    and the claim follows from comparison with \defref{def:noiseTensor} and \defref{def:width}.
\end{proof}

% This is just useful for Constrained Parameter Estimation!
As a direct consequence, we have at any $\canparam\in\canparamhypothesis$
\begin{align*}
    \absof{\centropyof{\empdistribution}{\stanexpdistof{\canparam}} - \centropyof{\gendistribution}{\stanexpdistof{\canparam}}}
    \leq \widthwrtof{\canparamhypothesis}{\sstatnoise} \, .
\end{align*}
Thus, the absolute difference of the expected risk and the empirical risk is bounded by the width of the noise tensor.
This is especially useful for the solution $\datamean$ of the empirical risk minimization, where we can state
\begin{align*}
    \centropyof{\gendistribution}{\stanexpdistof{\datacanparam}}
    \leq \centropyof{\empdistribution}{\stanexpdistof{\datacanparam}} + \widthwrtof{\canparamhypothesis}{\sstatnoise} \, .
\end{align*}
At the solution of a empirical risk minimization problem over $\canparamhypothesis$, the expected risk exceeds the empirical risk at most by the noise tensor width.

% Further KL divergence bound when assuming gendistribution in the hypothesis
When the generating distribution is in the hypothesis, we can further show the following KL-divergence bound for the estimated distribution.

\begin{theorem}
    Let us assume that for $\gencanparam\in\canparamhypothesis$ we have $\gendistribution=\stanexpdistof{\gencanparam}$. %and that $\partitionfunctionof{\canparam}$ is constant among $\canparamin$.
    Then for any solution $\datacanparam$ of the empirical problem we have
    \begin{align}
        \kldivof{\stanexpdistof{\gencanparam}}{\stanexpdistof{\datacanparam}} \leq 2\widthwrtof{\canparamhypothesis}{\sstatnoise} \, .
    \end{align}
\end{theorem}
\begin{proof}
    For the solution $\datacanparam$ of the empirical risk minimization on $\canparamhypothesis$ we have since $\gencanparam\in\canparamhypothesis$ that
    \begin{align*}
        \centropyof{\empdistribution}{\stanexpdistof{\datacanparam}}
        \leq \centropyof{\empdistribution}{\stanexpdistof{\gencanparam}} \, .
    \end{align*}
    It follows that
    \begin{align*}
        \kldivof{\stanexpdistof{\gencanparam}}{\stanexpdistof{\datacanparam}}
        & \leq \kldivof{\stanexpdistof{\gencanparam}}{\stanexpdistof{\datacanparam}}
        + \centropyof{\empdistribution}{\stanexpdistof{\gencanparam}}
        - \centropyof{\empdistribution}{\stanexpdistof{\datacanparam}} \\
        & = \left(\centropyof{\stanexpdistof{\gencanparam}}{\stanexpdistof{\datacanparam}} - \centropyof{\empdistribution}{\stanexpdistof{\datacanparam}}\right) \\
        & \quad - \left(\centropyof{\stanexpdistof{\gencanparam}}{\stanexpdistof{\gencanparam}} - \centropyof{\empdistribution}{\stanexpdistof{\gencanparam}}\right) \, ,
    \end{align*}
    where we expanded the KL-divergence as a difference of cross entropies.
    We apply \lemref{lem:centropyWidthCharacterization} to estimate the terms in brackets and get
    \begin{align*}
       & \left(\centropyof{\stanexpdistof{\gencanparam}}{\stanexpdistof{\datacanparam}} - \centropyof{\empdistribution}{\stanexpdistof{\datacanparam}}\right)
        - \left(\centropyof{\stanexpdistof{\gencanparam}}{\stanexpdistof{\gencanparam}} - \centropyof{\empdistribution}{\stanexpdistof{\gencanparam}}\right) \\
       & \quad\quad \leq 2 \widthwrtof{\canparamhypothesis}{\sstatnoise} \, .
    \end{align*}
    Combined with the above inequality we arrive at
    \begin{align*}
        \kldivof{\stanexpdistof{\gencanparam}}{\stanexpdistof{\datacanparam}} \leq 2\widthwrtof{\canparamhypothesis}{\sstatnoise} \, .
        \qedhere
    \end{align*}
\end{proof}

% Unconstrained parameter estimation
One technical issue arises from the fact, that when we allow for $\canparamhypothesis=\parspace$, then $\widthwrtof{\canparamhypothesis}{\sstatnoise}$ vanishes or is infinity.
To apply the result on the unconstrained parameter estimation, we therefore need to argue on bounded sets for the canonical parameter.
When restricting to the sphere $\subsphere\subset\parspace$ we have
\begin{align*}
    \normof{\datamean-\genmean} = \widthwrtof{\subsphere}{\mlnnoise} \, ,
\end{align*}
We apply this insight to state the following guarantee for unconstrained parameter estimation.
\begin{theorem}
    \label{the:detGuaranteeUnconstrained}
    Let $\canparam$
    \begin{align*}
        \absof{\centropyof{\empdistribution}{\stanexpdistof{\datacanparam}}-\centropyof{\gendistribution}{\stanexpdistof{\datacanparam}}}
        \leq \widthwrtof{\subsphere}{\mlnnoise} \cdot \normof{\datacanparam} \, .
    \end{align*}
\end{theorem}
\begin{proof}
    As in the proof of \lemref{lem:centropyWidthCharacterization} we use that
    \begin{align*}
        \centropyof{\empdistribution}{\stanexpdistof{\datacanparam}}-\centropyof{\gendistribution}{\stanexpdistof{\datacanparam}}
        = \contraction{\datamean-\genmean,\datacanparam}  \, .
    \end{align*}
    By Cauchy-Schwartz we further have
    \begin{align*}
        \absof{\contraction{\datamean-\genmean,\datacanparam}} \leq \normof{\datamean-\genmean}\cdot\normof{\datacanparam} \, .
    \end{align*}
    Using that $\normof{\datamean-\genmean}=\widthwrtof{\subsphere}{\mlnnoise}$ we arrive at the claim.
\end{proof}

\subsect{Structure Learning}

In the gradient heuristic of structure learning, one selects the statistic to the maximal coordinate of the energy tensor of the proposal distribution.
This tensor coincides with the mean parameter of a \MarkovLogicNetwork{} and has thus a fluctuation by the noise tensor.
We now use these insights to show a guarantee, that the formula chosen by grafting with respect to the empirical proposal distribution coincides with the formula chosen with respect to the expected proposal distribution.
To this end, we need to define the max gap, which is the difference between the maximal coordinate of a tensor to the second maximal coordinate.

\begin{definition}
    The max gap of a tensor $\hypercoreat{\shortcatvariables}$ is the quantity
    \begin{align*}
        \maxgapof{\hypercore} =
        \left(\max_{\shortcatindices} \hypercoreat{\indexedshortcatvariables}\right) -
        \left(\max_{\shortcatindices\notin\argmax_{\shortcatindices}\hypercoreat{\indexedshortcatvariables}}
        \hypercoreat{\indexedshortcatvariables}\right) \, .
    \end{align*}
\end{definition}

When comparing the gap with the noise width, we get the following guarantee.

\begin{theorem}
    \label{the:detGuaranteeProposalDist}
    Whenever
    \begin{align*}
        \maxgapof{\genmean}
        > 2 \cdot \widthwrtof{\{\onehotmapof{\shortcatindices}:\shortcatindices\in\facstates\}}{\sstatnoise} \, ,
    \end{align*}
    then any mode $\shortcatindices$ of the empirical proposal distribution is a mode of the expected proposal distribution.
\end{theorem}
\begin{proof}
    Let us assume that for a mode $\selindex^{\datamap}\in\argmax_{\selindexin}\datameanat{\indexedselvariable}$ of the empirical mean parameter we have
    \begin{align*}
        \selindex^{\datamap}\notin\argmax_{\selindexin}\genmeanat{\indexedselvariable} \, .
    \end{align*}
    For a mode $\selindex^{*}\in\argmax_{\selindexin}\genmeanat{\indexedselvariable}$ of the expected mean parameter we then have
    \begin{align*}
        \genmeanat{\selvariable=\selindex^{\datamap}} \leq \genmeanat{\selvariable=\selindex^{*}} - \maxgapof{\genmean}
    \end{align*}
    and
    \begin{align*}
        \datameanat{\selvariable=\selindex^{\datamap}} \geq \datameanat{\selvariable=\selindex^{*}} \, .
    \end{align*}
    Comparing both intequalities we get
    \begin{align*}
        \left(\datameanat{\selvariable=\selindex^{\datamap}} - \genmeanat{\selvariable=\selindex^{\datamap}}\right)
        + \left( - \datameanat{\selvariable=\selindex^{*}} + \genmeanat{\selvariable=\selindex^{*}} \right)
        \geq \maxgapof{\genmean} \, .
    \end{align*}
    Estimating the terms in the bracket by the width of the noise tensor with respect to basis vectors, we get
    \begin{align*}
        2 \cdot  \widthwrtof{\{\onehotmapof{\shortcatindices}:\shortcatindices\in\facstates\}}{\sstatnoise}
        \geq \maxgapof{\genmean} \, ,
    \end{align*}
    which is a contradiction to the assumption.
    Thus, any mode of the empirical mean parameter is also a model of the expected mean parameter.
\end{proof}


\subsect{Mode recovery}

Let us now consider a more general problem than in the section above, namely the estimation of the modes of a distribution.
Let $\estcanparam$ be the estimator of the canonical parameter $\gencanparam$, then the mode set of both coincide, if and only if they are elements in the same max cone, i.e.
\begin{align*}
    \genmaxconeof{\estcanparam} = \genmaxconeof{\gencanparam} \, .
\end{align*}

To ensure, that this is the case, we generalize the gap at $\gencanparam$ as the minimal distance to other cones and bound uniform concentration events implying that the distance between $\estcanparam$ and $\gencanparam$ is smaller than the gap.

\begin{definition}
    Let $\canmetricof{\cdot}{\cdot}$ be a metric on $\parspace$, then the generalized gap of $\canparam\in\parspace$ is defined as
    \begin{align*}
        \maxgapofat{\canmetric}{\canparam} = \inf_{\seccanparam\notin\genmaxconeof{\canparam}} \canmetricof{\seccanparam}{\canparam} \, .
    \end{align*}
\end{definition}

Atomic norms induce metrics, which are widths (see \cite{chandrasekaran_convex_2012} and Chapter~5 in \cite{goesmann_uniform_2021}).

\begin{definition}
    Given a set $\Gamma\subset\parspace$ such that an open neighborhood of the origin is contained in $\convhullof{\Gamma}$.
    Then
    \begin{align*}
        \|\canparam\|_{\Gamma} = \widthwrtof{\Gamma}{\canparam}
    \end{align*}
    is the dual atomic norm and
    \begin{align*}
        \canmetricwrtof{\Gamma}{\canparam}{\seccanparam} = \widthwrtof{\Gamma}{\canparam-\seccanparam}
    \end{align*}
    is the dual atomic distance to $\Gamma$.
\end{definition}

Examples of atomic norms are:
\begin{itemize}
    \item Euclidean distance $\ell_2$, when $\Gamma=\subsphere$
    \item Supremum distance $\ell_{\infty}$, when $\Gamma=\{\lambda\cdot \onehotmapofat{\selindex}{\selvariable} \wcols \lambda\in\{-1,+1\}\ncond \selindexin\}$
\end{itemize}


\begin{theorem}
    Let $\Gamma\subset\parspace$ induce an atomic norm.
    If
    \begin{align*}
        \widthwrtof{\Gamma}{\estcanparam-\gencanparam} < \maxgapofat{\canmetricwrt{\Gamma}}{\gencanparam}
    \end{align*}
    then the modes of $\stanexpdistof{\estcanparam}$ and $\stanexpdistof{\gencanparam}$ coincide.
\end{theorem}
\begin{proof}
    If $\estcanparam\notin\genmaxconeof{\gencanparam}$, then
    \begin{align*}
        \maxgapofat{\canmetricwrt{\Gamma}}{\gencanparam} \geq \canmetricwrtof{\Gamma}{\estcanparam}{\gencanparam} = \widthwrtof{\Gamma}{\estcanparam-\gencanparam} \, ,
    \end{align*}
    which contradicts the assumption.
    Therefore if the assumption holds, then $\estcanparam\in\genmaxconeof{\gencanparam}$ and the modes of $\stanexpdistof{\estcanparam}$ and $\stanexpdistof{\gencanparam}$ coincide.
\end{proof}

The guarantee on structure learning is the special case, where $\Gamma=\{\lambda\cdot \onehotmapofat{\selindex}{\selvariable} \wcols \lambda\in\{-1,+1\}\ncond \selindexin\}$ and
\begin{align*}
     \maxgapofat{\canmetricwrt{\Gamma}}{\canparam}
     = \frac{1}{2} \max_{\selindex\notin\argmax_{\selindex}\canparamat{\indexedselvariable}} \left| \canparamat{\indexedselvariable} - \max_{\selindex}\canparamat{\indexedselvariable} \right| \, .
\end{align*}

%This is a slight abuse of the special gap definition above.



\sect{Fluctuations in Logic Networks}

\red{In case of logical formulas being statistics, the coordiantes of the mean parameter are satisfaction rates to the formulas.}

For Logic Networks we have statistics consistent of boolean statistics $\enumformula$, which are logical formulas.
In this case the marginal distributions of the coordinates of $\sstatnoise$ are scaled and centered binomials, which we show now.

\begin{theorem}
    \label{the:noiseTensorBinomial}
    For any $\hlnstat$ the marginal distribution of the coordinate $\mlnnoiseat{\indexedselvariable}$ is the scaled and centered binomial distribution
    \begin{align*}
        \frac{1}{\datanum}\left(\bidistof{\datanum,\meanparamat{\indexedselvariable}}- \meanparamat{\indexedselvariable}\right)
    \end{align*}
    with parameters $\datanum$ and $\meanparamat{\indexedselvariable}$.
\end{theorem}
\begin{proof}
    We notice that when forwarding a random sample $\datapoint$ of $\gendistribution$ is the random tensor
    \[ \onehotmapofat{\datapoint}{\shortcatvariables} \, \]
    and since $\imageof{\sstatcoordinate}\subset \{0,1\}$ the contraction
    \[ \contraction{\sstatcoordinate, \onehotmapofat{\datapoint}{\shortcatvariables}} \]
    is a random variable taking values in $\{0,1\}$.
    The variable therefore follows a Bernoulli distribution with mean parameter
    \[ \meanparamat{\indexedselvariable}
    = \expectationof{\contraction{\sstatcoordinate, \onehotmapofat{\datapoint}{\shortcatvariables}}}
    = \contraction{\sstatcoordinate, \gendistribution}  \, \qedhere\]
\end{proof}

The mean parameter of the M-projection of the empirical distribution on the family of \MarkovLogicNetworks{} with statistic $\fselectionmap$ is the random tensor
\begin{align*}
    \datameanat{\selvariable}
    = \contractionof{\sencmlnstat,\empdistribution}{\selvariable} \, .
\end{align*}

The expectation of this random tensor is
\begin{align*}
    \expectationof{\datamean}
    =  \contractionof{\sencmlnstat,\expectationof{\empdistribution}}{\selvariable}
    =  \contractionof{\sencmlnstat,\gendistribution}{\selvariable}
    =  \genmean \, ,
\end{align*}
where we used that the expectation and contraction operation can be commuted due to the multilinearity of contractions.

\subsect{Energy tensor in proposal distributions}

The fluctuation tensor appears as a fluctuation of the energy of the proposal distribution.
The expectation of the energy of the proposal distribution is
\begin{align*}
    \expectationof{\energytensorof{\proposalstat,\empdistribution-\currentdistribution}}
    &= \expectationof{\contractionof{\sencproposalstat,\empdistribution-\currentdistribution}{\selvariable}}
    = \contractionof{\sencproposalstat,\expectationof{\empdistribution-\currentdistribution}}{\selvariable}
    = \contractionof{\sencproposalstat,\gendistribution-\currentdistribution}{\selvariable}\\
    &= \expectationof{\energytensorof{\proposalstat,\gendistribution-\currentdistribution}} \, .
\end{align*}

% Fluctuation
The fluctuation of this random tensor is
\begin{align*}
    \expectationof{\energytensorof{\proposalstat,\empdistribution-\currentdistribution}}  - \expectationof{\energytensorof{\proposalstat,\gendistribution-\currentdistribution}}
    = \expectationof{\energytensorof{\proposalstat,\empdistribution-\gendistribution}}
\end{align*}
and coincides with $\mlnnoise$.

\subsect{Minterm Exponential Family} % Interesting, since here is the connection with probability tensors: Forwarding of each random datapoint by the one hot encoding to get a multinomial random tensor.


In case of the minterm exponential family, we have $\sstat=\identityat{\shortcatvariables,\selvariable}$ and the noise tensor is
\begin{align*}
    \mintermnoise = \empdistribution - \gendistribution \, .
\end{align*}

% Multinomial as a more detailed characterization
This noise tensor follows a multinomial distribution as we show next.
To this end, we notice that a multinomial distribution can be defined as the average of one-hot encodings of independently and identically distributed datapoints.
When drawing $\data$ independently from $\gendistribution$ we denote
\begin{align*}
    \sum_{\datindexin}\onehotmapofat{\datapointof{\datindex}}{\shortcatvariables}
    \distassymbol \multidistof{\datanum,\gendistribution} \, .
\end{align*}

\begin{theorem}
    \label{the:multinomialEmpdistFluctuation}
    The noise tensor $\mintermnoise$ is a by $\frac{1}{\datanum}$ rescaled centered multinomial random tensor with parameters $\gendistribution$ and $\datanum$, that is
    \begin{align*}
        \mintermnoise \distassymbol \frac{1}{\datanum} \left( \multidistof{\datanum,\gendistribution} - \gendistribution \right) \, .
    \end{align*}
\end{theorem}
\begin{proof}
    By the above construction we have
    \begin{align*}
        \empdistribution - \gendistribution
        = \dataaverage \left( \onehotmapofat{\datapoint}{\shortcatvariables} - \expectationof{\onehotmapofat{\datapoint}{\shortcatvariables}} \right)
    \end{align*}
    We further have
    \begin{align*}
        \expectationof{\onehotmapofat{\datapoint}{\shortcatvariables}} = \gendistributionat{\shortcatvariables} \, .
    \end{align*}
\end{proof}

The noise tensor characterization by multinomial distributions, which holds for minterm statistics, is a more detailed characterization compared to the characterization of its marginals by binomial distribution in \theref{the:noiseTensorBinomial}, which holds for generic statistics $\hlnstat$.

\subsect{Guarantees for Mode of the Proposal Distribution}

Let us now derive probabilistic guarantees, that the mode of the proposal distribution at the empirical and the generating distribution are equal.

\begin{theorem}
    \label{the:probGuaranteeProposalDist}
    Whenever the energy tensor of the expected proposal distribution has a gap of $\maxgap$, then for every $\failprob>0$ any mode of the empirical proposal distribution coincides is also a mode of the expected proposal distribution with probability at least $1-\expof{-\frac{1}{\failprob^2}}$, provided that
    \begin{align*}
        \datanum > C\frac{(1+\lnof{\seldim})}{\maxgap^2}
    \end{align*}
    where $C$ is a universal constant.
\end{theorem}
\begin{proof}
    To proof the theorem we combine the deterministic guarantee \theref{the:detGuaranteeProposalDist} with the width bound of \theref{the:basisTensorWidthBound}, which we show in the next section.
    Given the assumed bound, the sub-gaussian norm of the width is upper bounded by $C_2\cdot \maxgap$, thus for any $\failprob>0$ we have
    \begin{align*}
        \widthwrtof{\selbasislong}{\mlnnoise}  < 2 \maxgap
    \end{align*}
    with probability at least $1-\expof{-\frac{1}{\failprob^2}}$.
    The claim thus follows with \theref{the:detGuaranteeProposalDist}.
\end{proof}


\begin{example}[Gap of a MLNs with single formulas]
    Let there be the MLN of a maxterm $\formula$ with $\atomorder$ variables, and let $\formulaset$ be the maxterm selecting tensor, then
    \[ \maxgapof{
    \energytensorof{(\formulaset, \expdistof{(\{\formula\},\singlecanparam)} - \normalizationof{\ones}{\shortcatvariables} )}
    } = \frac{1}{2^{\atomorder}-1 + \expof{-\singlecanparam}}  \]
    If $\singlecanparam>0$ we have an exponentially small gap.
    Thus, for the above Lemma to apply, the width needs to be exponentially in $\atomorder$ small.


    Let there be the MLN of a minterm $\formula$ with $\atomorder$ variables, then
    \[ \maxgap(
    \energytensorof{(\formulaset, \expdistof{(\{\formula\},\singlecanparam)} - \normalizationof{\ones}{\shortcatvariables} )}
    ) = \frac{1}{1+(2^{\atomorder}-1)\cdot\expof{-\singlecanparam}}  \]
    For large $\singlecanparam$ and $\atomorder$, the gap tends to $1$.
\end{example}

\subsect{Guarantees for Unconstrained Parameter Estimation}

%
We here the sphere bounds and combine with \theref{the:detGuaranteeUnconstrained}.

\begin{theorem}
    For any $\failprob\in(0,1)$ we have the following with probability at least $1-\failprob$.
    Let $\hat{\canparam}$ and $\precision>0$, then
    \[ \absof{\centropyof{\gendistribution}{\mlnexpdistof{\datacanparam}} - \centropyof{\empdistribution}{\mlnexpdistof{\datacanparam}}} \leq \tau \cdot \normof{\datacanparam} \]
    provided that
    \[ \datanum \geq \frac{\contraction{\genmean}-\contraction{(\genmean)^2}}{\failprob \precision^2} \, . \]
\end{theorem}
\begin{proof}
    The claim follows from the deterministic guarantee \theref{the:detGuaranteeUnconstrained} with the probabilistic width bound \theref{the:sphereBoundVariance} to be shown in the next section.
\end{proof}




\sect{Width bounds for the noise tensor}\label{sec:widthBounds}

We here provide width bounds on the noise tensors $\mlnnoise$ to logic networks, which coordinates have marginal distributions by Binomials, as shown in \theref{the:noiseTensorBinomial}.
All bounds hold for arbitrary statistics $\hlnstat$ of propositional formulas and number $\datanum$ of data and the appearing constants are universal, that is independent of particular choices of $\hlnstat$ and $\datanum$.

\subsect{Basis Vectors}

We first introduce the sub-Gaussian Norm and show how we can exploit it to state concentration inequalities.

\begin{definition}[Sub-Gaussian Norm, see Def.~2.5.6 in \cite{vershynin_high-dimensional_2018}]
    The sub-Gaussian norm of a random variable $X$ is defined as
    \begin{align*}
        \sgnormof{X} = \inf \left\{ C > 0 \, : \expectationof{\expof{\frac{X^2}{C^2}}} \leq 2 \right\} \, .
    \end{align*}
\end{definition}

The moment bound used to define the sub-Gaussian norm can then be combined with Markovs inequality to state concentration bounds.
% Relate with the contraction formalism, but not needed?
Before showing the utility of these norm, let us first connect with the contraction formalism of this work.
When $X$ is a random coordinate of $\hypercoreat{\shortcatvariables}$, selected by a probability tensor $\probat{\shortcatvariables}$ we have
\begin{align*}
    \expectationof{\expof{\frac{X^2}{C^2}}}
    = \contraction{\probat{\shortcatvariables},\expof{\frac{1}{C^2}\cdot\hypercoreat{\shortcatvariables}}}
\end{align*}
and thus
\begin{align*}
    \sgnormof{X} = \inf \left\{ C > 0 \, : \contraction{\probat{\shortcatvariables},\expof{\frac{1}{C^2}\cdot\hypercoreat{\shortcatvariables}}} \leq 2 \right\} \, .
\end{align*}

We now show a sub-Gaussian norm bound on the coordinates of the noise tensor.

\begin{lemma}
    \label{lem:mlnMeanSubGaussianCoordinates}
    The marginal distribution of any coordinate of $\mlnnoiseat{\selvariable}$ is sub-Gaussian with
    \begin{align*}
        \sgnormof{\mlnnoiseat{\indexedselvariable}} \leq C_0 \frac{1}{\sqrt{\datanum}} \,
    \end{align*}
    where $C_0>0$ is a universal constant .
\end{lemma}
\begin{proof}
    Any centered Bernoulli variable is bounded and therefore sub-Gaussian with
    \begin{align*}
        \sgnormof{\contraction{\enumformulaat{\shortcatvariables},\onehotmapofat{\datapoint}{\shortcatvariables}}-\contraction{\enumformulaat{\shortcatvariables},\gendistribution}}
        \leq \frac{1}{\sqrt{\lnof{2}}} \, .
    \end{align*}
    Binomial variables are sums of independent Bernoulli variables.
    We apply the sub-Gaussian norm bound for sums from Proposition~2.6.1 in \cite{vershynin_high-dimensional_2018}, which states that for a universal constant $C>0$ we have
    \begin{align*}
        \sgnormof{\contraction{\enumformulaat{\shortcatvariables},\left(\sum_{\datindexin}\onehotmapofat{\datapoint}{\shortcatvariables} - \gendistribution\right)}}
        \leq \frac{C\cdot\sqrt{\datanum}}{\sqrt{\lnof{2}}} \, .
    \end{align*}
    We therefore have
    \begin{align*}
        \sgnormof{\mlnnoiseat{\indexedselvariable}} =
        \frac{1}{m}\sgnormof{\contraction{\enumformulaat{\shortcatvariables},\left(\sum_{\datindexin}\onehotmapofat{\datapoint}{\shortcatvariables} - \gendistribution\right)}}
        \leq \frac{C}{\sqrt{\lnof{2} \cdot \datanum}} \, .
    \end{align*}
    We arrive at the claimed bound with a transform of the universal constant to $C_0=\frac{C}{\sqrt{\lnof{2}}}$.
\end{proof}

Based on this norm bound, we now show a bound on the sub-Gaussian norm of the width with respect to basis vectors.

\begin{theorem}
    \label{the:basisTensorWidthBound}
    For the set of basis vectors
    \begin{align*}
        \selbasisshort = \selbasislong
    \end{align*}
    we have
    \begin{align*}
        \sgnormof{\widthatof{\selbasisshort}{\mlnnoise}} \leq C_1 \sqrt{\frac{1+\lnof{\seldim}}{\datanum}} \, ,
    \end{align*}
    where $C_1>0$ is a universal constant.
\end{theorem}
\begin{proof}
    We first notice, that
    \begin{align*}
        \widthatof{\selbasisshort}{\noisetensor} = \max_{\selindexin} \absof{\mlnnoiseat{\indexedselvariable}}
    \end{align*}
    By a generic bound on the supremum of sub-Gaussian variables (see Exercise~2.5.10 in \cite{vershynin_high-dimensional_2018}) we have for a universal constant $C>0$
    \begin{align*}
        \sgnormof{\max_{\selindexin}\absof{\mlnnoiseat{\indexedselvariable}}}
        \leq C \left(\max_{\selindexin}\sgnormof{\mlnnoiseat{\indexedselvariable}}\right) \sqrt{1 + \lnof{\seldim}} \, .
    \end{align*}
    We now apply \lemref{lem:mlnMeanSubGaussianCoordinates} and get with $C_1=C\cdot C_0$ that
        \begin{align*}
        \sgnormof{\widthatof{\selbasisshort}{\mlnnoise}} \leq C_1 \sqrt{\frac{1+\lnof{\seldim}}{\datanum}}  \, .
    \end{align*}
\end{proof}

% Sharpness comment
The bound in \theref{the:basisTensorWidthBound} is furthermore sharp, see the construction of an identically scaling lower bound in Exercise~2.5.11 in \cite{vershynin_high-dimensional_2018}.
Note that the binomials used here tend to normal distributed variables used in the construction therein.

\subsect{Sphere}

For any tensor $\noiseat{\selvariable}$ and the sphere $\subsphere\subset\parspace$ we have
\begin{align*}
    \widthwrtof{\subsphere}{\noiseat{\selvariable}}
    = \normof{\noiseat{\selvariable}} \, .
\end{align*}
To show probabilistic width bounds with respect to the sphere, we therefore apply in the following Chebyshevs inequality on the norm of random tensors.

\begin{theorem}
    \label{the:sphereBoundVariance}
    Let $\meanparamat{\selvariable}$ be a deterministic vector with coordinates in $[0,1]$ and $\noiseat{\selvariable}$ a random vector, which coordiantes are for $\selindexin$ marginally distributed as
    \begin{align*}
        \noiseat{\indexedselvariable} \distassymbol \bidistof{\datanum,\meanparamat{\indexedselvariable}} \, .
    \end{align*}
    Then we have for any $\failprob>0$, $\precision>0$ and $\datanum\in\nn$ with probability at least $1-\failprob$
    \begin{align*}
        \normof{\frac{\noisetensor-\expectationof{\noisetensor}}{\datanum}} \leq \precision
    \end{align*}
    provided that
    \begin{align*}
        \datanum \geq \frac{\contraction{\meanparamat{\selvariable},(\onesat{\selvariable}-\meanparamat{\selvariable})}}{\failprob\cdot\precision^2} \, .
    \end{align*}
\end{theorem}
\begin{proof}
    Since the squared norm of the noise is the sum of squared centered and averaged Binomials, we have
    \begin{align*}
        \expectationof{\normof{\noiseat{\selvariable}-\expectationof{\noiseat{\selvariable}}}^2}
        = \datanum \cdot \left(\sum_{\selindexin} \meanparamat{\indexedselvariable}(1-\meanparamat{\indexedselvariable})\right)
    \end{align*}
    Here we used that the variance of a variable distributed by $\bidistof{\datanum,\meanparamat{\indexedselvariable}}$ is $\datanum\cdot\meanparamat{\indexedselvariable}(1-\meanparamat{\indexedselvariable})$.

    If follows, that
    \[ \expectationof{\left(\normof{\frac{\noisetensor-\expectationof{\noisetensor}}{\datanum}}\right)^2}
    = \frac{ \contraction{\meanparamat{\selvariable},(\onesat{\selvariable}-\meanparamat{\selvariable})}}{\datanum} \, . \]

    Then we apply a Chebyshev Bound to get for any $\precision>0$
    \begin{align}
        \probat{\normof{\frac{\noisetensor-\expectationof{\noisetensor}}{\datanum}} > \precision}
        = \probat{\left(\normof{\frac{\noisetensor-\expectationof{\noisetensor}}{\datanum}}\right)^2 > \precision^2}
        \leq \frac{ \contraction{\meanparamat{\selvariable},(\onesat{\selvariable}-\meanparamat{\selvariable})}}{\datanum \cdot \precision^2}
    \end{align}
    For a $\failprob>0$ we choose any $\datanum$ with
    \[ \datanum \geq  \frac{ \contraction{\meanparamat{\selvariable},(\onesat{\selvariable}-\meanparamat{\selvariable})}}{\precision^2 \failprob} \, \]
    and get
    \begin{align}
        \probat{\normof{\frac{\noisetensor-\expectationof{\noisetensor}}{\datanum}} > \precision} \leq \failprob \, .
    \end{align}
    Thus, we have
    \begin{align}
        \probat{\normof{\frac{\noisetensor-\expectationof{\noisetensor}}{\datanum}} \leq \precision}
        = 1 - \probat{\normof{\frac{\noisetensor-\expectationof{\noisetensor}}{\datanum}} > \precision}  \geq 1-\failprob \, .
    \end{align}
\end{proof}


% Multinomial
For the minterm family where $\hlnstat = \naivestat$ the noise tensor is a rescaled and centered multinomial.
In that case, the bound of \theref{the:sphereBoundVariance} can be simplified by
\begin{align*}
    \contraction{\meanparamat{\selvariable},(\onesat{\selvariable}-\meanparamat{\selvariable})} = 1- \contraction{\meanparamat{\selvariable}^2} \,.
\end{align*}


\sect{Discussion}

We in this chapter only provided probabilistic width bounds for logic networks, that are exponential families with boolean statistics.
Similar recovery bounds for parameter estimation and structure learning for more general exponential families would require width bounds in these generic cases.
A general approach towards width bounds are chaining techniques on stochastic processes, see \cite{talagrand_upper_2014}.
While we showed bounds based on the sub-Gaussian norm, more general sub-exponential bounds could be used, see \cite{wainwright_high-dimensional_2019}.

We further assumed that our random tensors to be projected are empirical distributions.
More general random tensor networks and corresponding width bounds have been developed in \cite{goesmann_uniform_2021}.
% Refer to FOL Models, extraction query stuff?

    \chapter{\chatextfolModels}\label{cha:folModels}

We in this chapter extend the tensor network based treatment of \propositionalLogic{} towards \firstOrderLogic{}.
In contrast to \propositionalLogic{}, worlds in \firstOrderLogic{} contain objects, which have relations between each others.
We show in this chapter, how these relations are captured by the \substitutionStructure{} of each world and derive tensor representations of each world.
The \substitutionStructure{} is then combined with the \semanticStructure{}, which enumerates possible worlds in a theory.
We then generalize \HybridLogicNetworks{} to the situation of \firstOrderLogic{}s and show in particular, that the likelihood of \HybridLogicNetworks{} on a single \firstOrderLogic{} world is in some cases equivalent to the likelihood of a dataset in propositoinal logics.

\sect{Syntax and Semantics of \firstOrderLogic{}}

%Since \firstOrderLogic{} follows structured representations of a system, a \firstOrderLogic{} world consists in objects and relations between them.
The framework of \firstOrderLogic{} generalizes \propositionalLogic{} analogously to the generalization of factored system representations towards structured system representations \cite{russell_artificial_2021}.
This more expressive framework is designed to reason about relations and functions between objects.
To capture this framework by tensors we here first define the syntax and then investigate tensor representations of the semantics.

\subsect{Syntax}

A \firstOrderLogic{} theory consists in a finite set of constant symbols $\worlddomain$, a set of function symbols $\{\folfunctionof{\secatomenumerator}\wcols\secatomenumeratorin\}$, a set of predicate symbols $\{\folpredicateof{\atomenumerator}\wcols\atomenumeratorin\}$ and an arity map assigning the arity $\indorderof{\secatomenumerator}$ to the function $\folfunctionof{\secatomenumerator}$ and the arity $\indorderof{\atomenumerator}$ to the predicate $\folpredicateof{\atomenumerator}$.

\subsect{Tensor representation of semantics}

We here follow the model-theoretic semantics of \firstOrderLogic{}, where sets of possible worlds are considered.
% Database
We only treat in this work database semantics, where we assume that the domain $\worlddomain$ of each world has a one-to-one mapping to the constants of the theory and is therefore constant among the worlds.
Database semantics is central to combine the \semanticStructure{} with the \substitutionStructure{}.
Under this assumption, the dimension of the \substitutionStructure{} does not depend on the specific world, and we can combine both structures in a single tensor space to be defined in the next sections.

% Index interpretation of world domain
To each world there is a world domain $\worlddomain$ of objects, which we assume to be finite.
We exploit the set-encoding formalism discussed in more detail in \charef{cha:basisCalculus} and use bijective index interpretation maps
\begin{align*}
    \indexinterpretation \defcols [\inddim] \rightarrow \worlddomain \, .
\end{align*}
A so-called term variable $\indvariable$ takes states $\indindexin$, which represent objects
\begin{align*}
    \indexinterpretationat{\indindex} \in \worlddomain \, .
\end{align*}

% Relations
The relations between objects are described by $\indorder$-ary predicates $\folpredicate$.
Given a specific world $\worldindices$ the truth of relations is represented by boolean tensors
\begin{align*}
    \groundingof{\folpredicate} \defcols \symindstates\rightarrow\ozset \, .
\end{align*}
Given a tuple $\indindexlist\in\symindstates$ the boolean
\begin{align*}
    \groundingofat{\folpredicate}{\indexedindvariableof{0},\ldots,\indexedindvariableof{\indorder-1}} \in\ozset
\end{align*}
is called a grounding and encodes, whether the relation $\folpredicate$ is satisfied in the world $\worldindices$ for the objects $\indexinterpretationat{\indindexof{0}},\ldots,\indexinterpretationat{\indindexof{\indorder-1}}$.

% Functions
Functions in \firstOrderLogic{} are object-valued maps
\begin{align*}
    \groundingof{\folfunction}\defcols \worlddomain^{\inddim} \rightarrow \worlddomain \, .
\end{align*}
While relations are represented by their coordinate encodings, functions are represented by directed basis encodings
\begin{align*}
    \bencodingofat{\groundingof{\folfunction}}{\indvariableof{\folfunction},\shortindvariables}
\end{align*}
where to each object tuple $\indexinterpretationat{\indindexof{0}},\ldots,\indexinterpretationat{\indindexof{\indorder-1}}$ the vector $\bencodingofat{\groundingof{\folfunction}}{\indvariableof{\folfunction},\indexedshortindvariables}$ is the one-hot encoding of the object, that is
\begin{align*}
    \bencodingofat{\groundingof{\folfunction}}{\indvariableof{\folfunction},\indexedshortindvariables}
    = \onehotmapofat{\groundingofat{\folfunction}{\indexinterpretationat{\indindexof{0}},\ldots,\indexinterpretationat{\indindexof{\indorder-1}}}}{\indvariableof{\folfunction}} \, .
\end{align*}


%\subsect{Database semantics}

% Database Semantics
%\begin{align*}
%    \bigotimes_{\atomenumeratorin,\shortindindices\in[\inddim]^{\indorder}} \rr^2 \, .
%\end{align*}

\subsect{Two levels of tensor representation}

In comparison with \propositionalLogic{}, \firstOrderLogic{} bears two natural tensor structures.
\begin{itemize}
    \item \textbf{\SemanticStructure{}:} As in \propositionalLogic{}, we enumerate possible worlds by a collection of variables $\catvariable$.
    This is a representation of the model-theoretic semantics, where subsets of worlds are represented by boolean tensors.
    \item \textbf{\SubstitutionStructure{}:}
    Different to \propositionalLogic{}, a formula can have variables and the evaluation of a formula on a world is represented by its grounding tensor.
    Given a world which contains objects in the domain $\worlddomain$, we can substitute variables by objects in the domain.
    In this way, each predicate with $\inddim$ variables is represented in that world as a boolean tensor of order $\inddim$.
\end{itemize}
% Russel reference
While the \semanticStructure{} appears already in factored representations of systems, the \substitutionStructure{} arises in more generality when treating structured representations of systems \cite{russell_artificial_2021}.



\sect{\SubstitutionStructure{}}

We in this section investigate the structure of terms and formulas in a single \firstOrderLogic{} world, which we for now index by $\worldindices$.
As we have derived in \charef{cha:logicalRepresentation} for \propositionalLogic{}, we develop efficient tensor network representations of the representing tensors based on the corresponding syntax.

\subsect{Grounding tensors}

Given a \firstOrderLogic{} world $\worldindices$, arbitrary formulas are interpreted in terms of the satisfactions of their groundings.
We define their semantic first, and then relate their syntactical decomposition to tensor networks, similar to our approach to \propositionalLogic{} in \charef{cha:logicalRepresentation}.

\begin{definition}[Grounding of a first-order formula given a world]
    Given a specific world $\worldindices$, with a domain $\worlddomain$ enumerated by $[\inddim]$, the grounding of a formula $\folexformula$ with variables $\indvariableof{\folexformula}$  is the tensor
    \begin{align*}
        \groundingofat{\folexformula}{\indvariableof{\folexformula}} :
        \bigtimes_{\indenumerator\in[\indvariableof{\folexformula}]} [\inddim] \rightarrow \ozset \, .
    \end{align*}
    Each coordinate represents thereby the boolean, whether the substitution of the variables in the formula is satisfied given a world $\worldindices$, that is
    \begin{align*}
        \groundingofat{\folexformula}{\indexedindvariableof{\folexformula}} = 1
    \end{align*}
    if and only if the substitution of $\folexformula$ with the variables $\indvariableof{\folexformula}$ replaced by the objects $\indexinterpretationat{\indindexof{\indenumerator}}$ is satisfied on the world $\worldindices$.
\end{definition}


%% Basis encoding
%When interpreting this map as a basis encoding, formulas are tensors in the tensor space
%\begin{align*}
% 	\left(\bigotimes_{\atomenumeratorin, \indindexlist\in[\inddim]} \rr^{2} \right) \otimes
%	\left(  \bigotimes_{\atomenumeratorin, \indindexof{0},\ldots,\indindexof{\indorder_{\folexformula}}\in[\inddim]} \rr^{2} \right) \, .
%\end{align*}

\subsect{Terms}

Terms are object valued maps on $\worlddomain^{n}$.
The basis encoding of each term corresponds is a boolean and directed tensor.
% Constants: Functions with $n=0$
Constants are the functions with $n=0$, and are represented by basis vectors in $\rr^{\cardof{\worlddomain}}$.
Given database semantics, this basis vector does not vary among different worlds.
% General terms
Most general terms are combinations of functions and constants and their basis encodings are acyclic contractions of basis encodings to functions.


%\subsect{Substitution by slicing}
% Slicing interpretation
%Slicing the grounding tensor of a formula a first-order formula amounts to substitution of the respective variable by the constant at the enumeration index.

%\subsect{Syntactical Decomposition of quantifier-free formulas}

\subsect{Formula synthesis by connectives}\label{sec:folConnectiveRepresentation}

In order to have a sound semantic, the grounding of \firstOrderLogic{} formulas is determined by the syntax of the formula, i.e. a decomposition of the formula into connectives and quantifiers acting on atomic formulas.

% Formulas as maps from worlds to groundings
Quantifier-free formulas are connectives acting on atomic formulas.
We can describe them as in the case of \propositionalLogic{} in the $\bencodingof{}$-formalism.
While the atomic formulas where delta tensors copying states, they are more involved here.

\begin{theorem}
    For any connective $\exconnective$ and formulas $\folexformula_1$ and $\folexformula_2$ we have
    \begin{align}
        &\groundingofat{(\folexformula_1\exconnective\folexformula_2)}{\indvariableof{\folexformula_1}\cup\indvariableof{\folexformula_2}} \\
        &\quad=
        \contractionof{
            \bencodingofat{\groundingof{\folexformula_1}}{\headvariableof{\folexformula_1},\indvariableof{\folexformula_1}},
            \bencodingofat{\groundingof{\folexformula_2}}{\headvariableof{\folexformula_2},\indvariableof{\folexformula_2}},
            \bencodingofat{\exconnective}{\headvariableof{\folexformula_1\exconnective\folexformula_2}, \headvariableof{\folexformula_1}, \headvariableof{\folexformula_2}},
            \tbasisat{\headvariableof{\folexformula_1\exconnective\folexformula_2}}
        }
        {\shortindvariablelist} \, .
    \end{align}
\end{theorem}
\begin{proof}
    This directly follows from \theref{the:compositionByContraction}.
%	By the semantic interpretation of the groundings, which has to be sound.
\end{proof}

% Shared variables
Here, variables can be shared by the connected formulas, therefore the variables in the combined formula are unions of the possible not disjoint variables of the connected formulas.

%% Propositional interpretation
%When we understand the head variables in the basis encoding of atoms as the categorical variables, and get a similar interpretation of the tensor network decomposition as in the propositional case.
%\subsect{Propositionalization}

When interpreting the head variables of relational encoded atomic formulas as the atoms of a propositional theory, we find a propositional formula $\exformula$ associated with any decomposable \firstOrderLogic{} formula.

\begin{definition}
    \label{def:propositionalEquivalent}
    Given a formula $\folexformula$ in \firstOrderLogic{}, we say that a propositional formula $\formulaat{\shortcatvariables}$ is the propositional equivalent to $\folexformula$ given atomic formulas $\extformulaof{\atomenumerator}$ in \firstOrderLogic{}, when for any world $\worldindices$ we have
    \begin{align*}
        \groundingofat{\folexformula}{\indvariableof{\folexformula}}
        = \contractionof{
            \{\bencodingofat{\groundingof{\extformulaof{\atomenumerator}}}{\catvariableof{\atomenumerator},\indvariableof{\extformulaof{\atomenumerator}}} : \atomenumeratorin\}
            \cup \{\formulaat{\shortcatvariables}\}
        }{\indvariableof{\folexformula}} \, .
    \end{align*}
    We here denote the head variables of the basis encoding to $\bencodingof{\groundingof{\extformulaof{\atomenumerator}}}$ by $\catvariableof{\atomenumerator}$ to highlight their interpretation as propositional atoms.
\end{definition}

We depict the relation of a grounding tensor to a propositional formula as:
\begin{center}
    \begin{tikzpicture}[scale=0.35, yscale=1, thick] % , baseline = -3.5pt



%\draw[] (2,-1) -- (2,1) node[midway,left] {\tiny $\atomicformulaof{0}$};
%\node[anchor=center] (text) at (4,0) {$\cdots$};
%\draw[] (6,-1) -- (6,1) node[midway,right] {\tiny $\atomicformulaof{\atomorder-1}$};

\draw (1,-1) rectangle (7,-3);
\node[anchor=center] (text) at (4,-2) {$\groundingof{\folexformula}$};

\draw[] (2,-3) -- (2,-5) node[midway,left] {\tiny $\individualvariableof{0}$};
\node[anchor=center] (text) at (4,-4) {$\cdots$};
\draw[] (6,-3) -- (6,-5) node[midway,right] {\tiny $\individualvariableof{\individualorder-1}$};


\node[anchor=center] (text) at (10,-2) {${=}$};



%\draw (1,-1) rectangle (7,-3);
%\node[anchor=center] (text) at (4,-2) {$\rencodingof{\impformula}$};

\begin{scope}[shift={(12,-2)}]

\draw (1,3) rectangle (12,5);
\node[anchor=center] (text) at (6.5,4) {$\exformula$};

\draw[->] (2.5,1) -- (2.5,3) node[midway,right] {\tiny $\atomicformulaof{0}$};
\draw (1,-1) rectangle (4,1);
\node[anchor=center] (text) at (2.5,0) {$\rencodingof{\groundingof{\extformulaof{0}}}$};
\node[anchor=center] (text) at (2.5,-2) {$\cdots$};

\node[anchor=center] (text) at (6.5,0) {$\cdots$};

\draw[->] (10.5,1) -- (10.5,3) node[midway,right] {\tiny $\atomicformulaof{\atomorder-1}$};
\draw (8.75,-1) rectangle (12.25,1);
\node[anchor=center] (text) at (10.5,0) {$\rencodingof{\groundingof{\extformulaof{\atomorder\shortminus1}}}$};
\node[anchor=center] (text) at (10.5,-2) {$\cdots$};

\draw[-] (11.5,-3) -- (3.5,-3) ;
\draw[-] (9.5,-5) -- (1.5,-5) ;

\draw[fill] (11.5,-3) circle (0.25cm);
\draw[->] (11.5,-3) -- (11.5,-1);

\draw[fill] (9.5,-5) circle (0.25cm);
\draw[->] (9.5,-5) -- (9.5,-1);

\draw[fill] (3.5,-3) circle (0.25cm);
\draw[->] (3.5,-3) -- (3.5,-1);

\draw[fill] (1.5,-5) circle (0.25cm);
\draw[->] (1.5,-5) -- (1.5,-1);


\draw[fill] (7.5,-3) circle (0.25cm);
\draw[<-] (7.5,-3) -- (7.5,-7) node[right] {\tiny $\individualvariableof{\individualorder-1}$} ;

\node[anchor=center] (text) at (6.5,-6) {$\cdots$};

\draw[fill] (5.5,-5) circle (0.25cm);
\draw[<-] (5.5,-5) -- (5.5,-7) node[left] {\tiny $\individualvariableof{0}$} ;


%\draw (13,-2) rectangle (15,-6);
%\node[anchor=center] (text) at (14,-4) {$\rencodingof{\impformula}$};
%\node[anchor=center] (text) at (12,-3.75) {$\vdots$};
%\draw[->] (15,-4) -- (16,-4);
%\draw[] (17,-4) -- (16,-4);
%\draw (17,-3) rectangle (19,-5);
%\node[anchor=center] (text) at (18,-4) {$\tbasis$};

\end{scope}




		


\end{tikzpicture}
\end{center}


\subsect{Quantifiers}

Existential and universal quantifiers appear in generic \firstOrderLogic{} and are besides substitutions further means to reduce the number of variables in a formula.
%They are not representable as linear transform of the respective quantifier-free formula.


% Definition of existential and universal quantifiction needed!
The semantics of existential quantification consists in a formula being true, if at least one state of the quantified variable is true, as we define next.

\begin{definition}
    Given a grounding tensor
    \begin{align*}
        \groundingofat{\folexformula}{\indvariableof{0},\ldots,\indvariableof{\indorder-1}} \,
    \end{align*}
    the existential and universal quantification with respect to the first variable are the tensors
    \begin{align*}
        \groundingofat{\left(\exists_{\indindexof{0}}\folexformula\right)}{\indvariableof{1},\ldots,\indvariableof{\indorder-1}} \quad \text{and} \quad
        \groundingofat{\left(\forall_{\indindexof{0}}\folexformula\right)}{\indvariableof{1},\ldots,\indvariableof{\indorder-1}} \,
    \end{align*}
    with coordinates as follows.
    For an assignment $\indindexof{1},\ldots,\indindex$ to the non-quantified variables we have
    \begin{align*}
        \groundingofat{\left(\exists_{\indindexof{0}}\folexformula\right)}{\indexedindvariableof{1},\ldots,\indexedindvariableof{\indorder-1}} = 1
    \end{align*}
    if and only if there is an assignment $\indindexofin{0}$ such that
    \begin{align*}
        \groundingofat{\folexformula}{\indexedindvariableof{0},\indexedindvariableof{1},\ldots,\indexedindvariableof{\indorder-1}} = 1 \, .
    \end{align*}
    Conversely, we have for the universal quantification that
    \begin{align*}
        \groundingofat{\left(\forall_{\indindexof{0}}\folexformula\right)}{\indexedindvariableof{1},\ldots,\indexedindvariableof{\indorder-1}} = 1
    \end{align*}
    if and only if for any assignment $\indindexofin{0}$ we have
    \begin{align*}
        \groundingofat{\folexformula}{\indexedindvariableof{0},\indexedindvariableof{1},\ldots,\indexedindvariableof{\indorder-1}} = 1 \, .
    \end{align*}
\end{definition}


Let us now show, that existential and universal quantification are coordinatewise transforms (see \defref{def:coordinatewiseTransform}) of contracted grounding tensors.
To this end, let us introduce the greater-$z$ indicator $\greaterthanfunction{z}$, where $z\in\rr$, as the function
\begin{align*}
    \greaterthanfunction{z} : \rr \rightarrow \ozset
    \quad, \quad \greaterthanfunctionof{z}{x} =
    \begin{cases}
        1 & \ifspace x > z\\
        0 & \text{else}
    \end{cases} \, .
\end{align*}

\begin{theorem}
    For any formula $\folexformula$ with variables $\shortindvariablelist$ we have
    \begin{align*}
        \groundingofat{\left(\exists{\indindexof{0}}\folexformula\right)}{\indvariableof{1},\ldots,\indvariableof{\indorder-1}} =
        \coordinatetrafowrtofat{\existquanttrafo}{\contractionof{\groundingof{\folexformula}}{\indvariableof{1},\ldots,\indvariableof{\indorder-1}}}{\indvariableof{1},\ldots,\indvariableof{\indorder-1}}
    \end{align*}
    and
    \begin{align*}
        \groundingofat{\left(\forall{{\indindexof{0}}} \folexformula\right)}{\indvariableof{1},\ldots,\indvariableof{\indorder-1}}=
        \coordinatetrafowrtofat{\universalquanttrafo}{\contractionof{\groundingof{\folexformula}}{\indvariableof{1},\ldots,\indvariableof{\indorder-1}}}{\indvariableof{1},\ldots,\indvariableof{\indorder-1}}
    \end{align*}
\end{theorem}
\begin{proof}
    We proof the claimed equalities to arbitrary slices of the remaining variables, which amount to arbitrary substitutions of the formulas.
    For any indices $\indindexofin{1},\ldots,\indindexofin{\indorder-1}$ we notice, that
    \begin{align*}
        \contractionof{\groundingof{\folexformula}}{\indexedindvariableof{1},\ldots,\indexedindvariableof{\indorder-1}}
        &= \sum_{\indindexofin{0}} \groundingofat{\folexformula}{\indexedindvariableof{0},\ldots,\indexedindvariableof{\indorder-1}} \\
        &= \cardof{\indindexofin{0} \, : \, \groundingofat{\folexformula}{\indexedindvariableof{0},\ldots,\indexedindvariableof{\indorder-1}}=1} \, .
    \end{align*}
    We can thus understand the contracted grounding tensor as storing in its coordinates the count of the coordinate extensions to the zeroth variable, such that the grounding tensor is satisfied.
    This is analogous to our interpretation of contracted propositional formulas as world counts.
    From this it is obvious, that the existential quantification is satisfied, if the count is different from zero, which is captured by the coordinatewise transform with $\existquanttrafo$.
    We therefore arrive at
    \begin{align*}
        \groundingofat{\left(\exists_{\indindexof{0}}\folexformula\right)}{\indexedindvariableof{1},\ldots,\indexedindvariableof{\indorder-1}} =
        \coordinatetrafowrtofat{\existquanttrafo}{\contractionof{\groundingof{\folexformula}}{\indvariableof{1},\ldots,\indvariableof{\indorder-1}}}{\indexedindvariableof{1},\ldots,\indexedindvariableof{\indorder-1}} \, .
    \end{align*}
    The first claim follows, since the assignment to the non-quantified variables was arbitrary.
    The universal quantification is satisfied, when all extensions are satisfied, and the count is $\inddim$.
    Since $\inddim$ is the maximal count, this is captured by the coordinatewise transform with $\universalquanttrafo$ and we get
    \begin{align*}
        \groundingofat{\left(\forall{\indindexof{0}}\folexformula\right)}{\indexedindvariableof{1},\ldots,\indexedindvariableof{\indorder-1}} =
        \coordinatetrafowrtofat{\universalquanttrafo}{\contractionof{\groundingof{\folexformula}}{\indvariableof{1},\ldots,\indvariableof{\indorder-1}}}{\indexedindvariableof{1},\ldots,\indexedindvariableof{\indorder-1}} \, .
    \end{align*}
    With the same argument, the second claim is established.
\end{proof}

% Customized quantifiers
We can extend this discussion towards more generic counting quantifiers, of which the existential and the universal quantifier are extreme cases.
One can define quantifiers by demanding that at least $z\in\nn$ compatible groundings are satisfied, and show that they amount to coordinatewise transforms with $\greaterthanfunction{z}$.
What is more, quantifiers demanding that at most $z\in\nn$ are satisfied would be representable by transforms with an analogously defined function $\ones_{\leq z}$.
Such customized quantifiers appear for example in the $\mathrm{OWL\,2}$ standard of description logics (see \cite{rudolph_foundations_2011} and \secref{sec:kgRepresentation}).

% basis encodings
As will be discussed in \charef{cha:basisCalculus}, any coordinatewise transform can be performed by a contraction of a basis encoding of the tensor with a head vector prepared by the transform function (see \theref{the:tensorFunctionComposition}).
In the case here, a direct implementation would require a dimension of these head variables by $\inddim$, which can be infeasible when having large object sets.

% Prenex
To summarize, let us assume a formula is in its prenex normal form, that is a collection of quantifiers are acting on a qantifier free part.
We can represent its grounding tensor by
\begin{itemize}
    \item Instantiations of the tom groundings with the assigned variables, as contractions of the basis encoding of the world tensor with atom selecting tensors.
    \item Propositional formula acting on the head variables of the predicate instantiations, representing the connectives combining the formula.
    \item Quantifiers as a composition of contractions closing the quantified variable and coordinatewise transforms with the respective greater-than indicators.
\end{itemize}



\subsect{Storage in basis CP decomposition}\label{sec:basisCPgrounding}

In many situations, grounding cores are sparse and representations as single tensor cores comes with a drastic overhead.
We often encounter sparse grounding tensors, where the number of non-zero coordinates (to be investigated by basis CP ranks in \charef{cha:sparseRepresentation}) satisfies
\begin{align*}
    \sparsityof{\groundingof{\folexformula}} << \inddim^{\cardof{\indvariableof{\folexformula}}} \, .
\end{align*}
In this case, since most coordinates vanish, the basis CP decomposition (see \secref{sec:basisCP}) enables a representation of the grounding with significantly lower storage demand, see \theref{the:sparseBasisCP}.
This is particularly useful for representing large relational databases, where each object has only a few relations with others, while the majority of possible relations remains unsatisfied.
We depict such CP decomposition of a formula grounding in \theref{fig:groundingCP}.

% Standard KB Encoding and Assumptions
Most logical syntaxes exploit $\ell_0$-sparsity, explicitly storing only known assertions.
The interpretation of unspecified assertions depends on the underlying assumptions.
Under the Closed World Assumption, for example, all unspecified assertions are assumed to be false.

\begin{figure}[t]
    \begin{center}
        \begin{tikzpicture}[scale=0.35, yscale=-1, thick] % , baseline = -3.5pt





%\drawatomindices{0}{2}


\draw (-1,1) rectangle (5,-1);
\node[anchor=center] (text) at (2,0) {$\groundingof{\exformula}$};

%\draw[->] (2,-1) -- (2,-3) node[midway, right] {\tiny $\atomlegindexof{\exformula}$};
%\draw[] (3,-3) rectangle (1, -5);
%\node[anchor=center] (text) at (2,-4) {\small $\tbasis$};
%\draw[->] (4,-1) -- (4,-3) node[midway, right] {\tiny $\datindex$};
%\draw[dashed] (3,-3) rectangle (5, -5);
%\node[anchor=center] (text) at (4,-4) {\small $\ones$};

\draw[] (0,1) -- (0,3) node[midway,left] {\tiny $\exindividualof{0}$};
\draw[] (1.5,1) -- (1.5,3) node[midway,left] {\tiny $\exindividualof{1}$};
\node[anchor=center] (text) at (2.75,2) {$\cdots$};
\draw[] (4,1) -- (4,3) node[midway,right] {\tiny $\exindividualof{\variableorder\shortminus1}$};

\node[anchor=center] (text) at (7,0) {${=}$};


\begin{scope}[shift={(10,2)}]

\newcommand{\conposseldec}{4.5,-5.5}

\draw[fill] (\conposseldec) circle (0.25cm);
\draw (\conposseldec) -- (4.5,-7.5) node[midway, right] {\tiny $\datindex$};
\draw[] (3.5,-7.5) rectangle (5.5, -9.5);
\node[anchor=center] (text) at (4.5,-8.5) {\small $\ones$};

\draw[<-] (0,1) -- (0,-1) node[midway,left] {\tiny $\exindividualof{0}$};
\draw (-1,-1) rectangle (1, -3);
\node[anchor=center] (text) at (0,-2) {\small $\legcoreof{\exformula,0}$};
\draw[<-] (0,-3) to[bend right=20] (\conposseldec);


\draw[<-] (3,1) -- (3,-1) node[midway,left] {\tiny $\exindividualof{1}$};
\draw (2,-1) rectangle (4, -3);
\node[anchor=center] (text) at (3,-2) {\small $\legcoreof{\exformula,1}$};
\draw[<-] (3,-3) to[bend right=20]  (\conposseldec);

\node[anchor=center] (text) at (6,-2) {$\cdots$};

\draw[<-] (9,1) -- (9,-1) node[midway,left] {\tiny $\exindividualof{\variableorder\shortminus1}$};
\draw (8,-1) rectangle (10, -3);
\node[anchor=center] (text) at (9,-2) {\small $\legcoreof{\exformula,\variableorder\shortminus1}$};
\draw[<-] (9,-3) to[bend left=20]  (\conposseldec);




\end{scope}

		


\end{tikzpicture}
    \end{center}
    \caption{Basis CP Decomposition of the grounding of $\folexformula$, following the scheme of \theref{the:sparseBasisCP}.
    Instead of direct storage of the grounding tensor $\groundingof{\folexformula}$, the non-zero coordinates are enumerated by a variable $\datvariable$ and the corresponding coordinates stored in leg-matrices $\legcoreof{\folexformula,\indenumerator}$.}
    \label{fig:groundingCP}
\end{figure}

\subsect{Summary}

Let us summarize the tensor encodings of the representations of the different concepts, given a single \firstOrderLogic{} world:
\begin{center}
    \begin{tabular}{|p{\threecolumnwidth}|p{\threecolumnwidth}|p{\threecolumnwidth}|}
        \hline
        \textbf{Concept}        & \textbf{Representation}                       & \textbf{Decomposition}                             \\
        \hline
        Constant Symbol         & Basis vector                                  &                                                    \\
        \hline
        Function Symbol         & \BasisEncoding{}: Boolean and directed tensor &                                                    \\
        Term                    & ""                                            & Acyclic tensor network of represented functions    \\
        \hline
        Predicate Symbol        & \CoordinateEncoding{}: Boolean tensor         &                                                    \\
%        Relation & Boolean tensor \\
        Quantifier-free Formula & ""                                            & Contraction of represented terms with relations    \\
        Formula                 & ""                                            & Transform of corresponding quantifier-free formula \\
        \hline
    \end{tabular}
\end{center}


\subsect{Example: Relational Databases}

A database is understood as a specific \firstOrderLogic{} world, and are operations on such a single world.
Queries are described by a formula $\impformula$, which are asked against a specific world $\worldindices$ to retrieve the grounding $\groundingof{\impformula}$.
The variables of such formulas are called projection variables.
The answer $\groundingof{\impformula}$ of a query is most conveniently represented as a list of solution mappings from the projection variables to objects in the world, such that the query formula is satisfied.
Answering a query by solution mappings corresponds with finding the basis CP Decomposition (see \secref{sec:basisCP}) of $\groundingof{\impformula}$.
We can understand these solution mappings as stored in the leg-matrices $\legcoreof{\folexformula,\indenumerator}$ (see \figref{fig:groundingCP}).

Let us give with the outer join an example of a popular operation to define queries, which efficient execution and storage can be improved based on considerations in the tensor network formalism.

\begin{definition}[Outer join]
    Let there be a world $\worldindices$ and formulas $\extformulaof{\selindex}$ depending on variables $\indvariableof{\nodesof{\selindex}}$, which have grounding tensors by
    \begin{align*}
        \groundingofat{\extformulaof{\selindex}}{\indvariableof{\node}} \, : \,  \bigtimes_{\node\in\nodesof{\selindex}}[\inddimof{\node}] \rightarrow \ozset \, .
    \end{align*}
    Then their (outer) $\joinsymbol$ is defined as the grounding of their conjunctions, as
    \begin{align*}
        \groundingofat{\joinsymbol\left(\extformulaof{0},\ldots,\extformulaof{\seldim-1}\right)}{\bigcup_{\selindexin}\indvariableof{\nodesof{\selindex}}}
        = \contractionof{\groundingofat{\extformulaof{\selindex}}{\indvariableof{\nodesof{\selindex}}}\,:\,\selindexin}{\bigcup_{l\in[p]}\indvariableof{\nodesof{\selindex}}} \, .
    \end{align*}
\end{definition}

%Visualization and efficiency
We can understand the $\joinsymbol$ of groundings by a factor graph, where each grounding tensor decorates the hyperedge to the node set $\nodesof{\selindex}$.
The projection variable assignment to each formula combined in a $\joinsymbol$ operation provide a basic tensor network format to store the output of the operation.
There are thus situations, in which the solution map storage corresponding with a CP Decomposition comes with unnecessary overheads compared with other formats.

% Coordinatewise transform
We can also understand the $\joinsymbol$ operation as a coordinatewise transform (see \defref{def:coordinatewiseTransform}) with the product as transform function.
To make this connection solid, one would need to extend each joined formula trivially to the variables appearing in other formulas.

% Evaluation similar constraint propagation
The efficiency of evaluating the contraction to a $\joinsymbol$ operation might be improved by understanding it as an Constraint Satisfaction Problem (see \charef{cha:logicalReasoning}).
When applying efficient Message Passing algorithms such as Knowledge Propagation (see \algoref{alg:knowledgePropagation}), the groundings can be sparsified by local constraint propagation operations before turning to more global and more demanding contraction operations.
Here the groundings $\groundingof{\extformulaof{\selindex}}$ would be used to initialize Knowledge Cores $\kcoreof{\edge}$ and sequentially sparsified during the algorithm.

%\begin{example} % WOULD NEED OVERWORK: DRAW!
%	For example take a query with many basic graph patterns with pairwise different projection variables.
%	The global CP Decomposition would come here with an exponential storage overhead compared with storage as a tensor product of CP Decompositions to each Basic graph pattern.
%\end{example}

%% CONFUSING?
%\begin{remark}[Distinguishing from probabilistic queries]
%	Let us distinguish the discussion here from those of queries in probabilistic reasoning, which have two main differences.
%	First, we ask queries against all possible pairs of variables, instead of asking the probability of satisfaction of a specific formula.
%	Second, since we made the epistemologic assumption of knowing possibilities and not probabilities in logics, a query is answered by a truth value.
%	We then only output in the shape of solution mappings the variable assignments where the query formula is true.
% 	Thus, the queries here can be thought of as a batch of probabilistic queries with Boolean answers.
%	% Alternative -> Later?
%	Probabilistic queries can furthermore be understood in terms of the data extraction process described in this section.
%	We can ask the query in probabilistic form (decomposed into atomic formulas) on the resulting empirical distribution.
%	This results in the ratio of the worlds satisfying the query among those worlds satisfying the extraction query $\impformula$.
%\end{remark}







\sect{\SemanticStructure{}}

We now investigate the \semanticStructure{} of a \firstOrderLogic{} theory, when restricting to database semantics.
This involves the representation of collections of worlds, where each has a \substitutionStructure{} as discussed above.

\subsect{World enumerating variables}

Given database semantics, we have a finite set of worlds, which we enumerate by tuples of variables $\catvariable$.

\begin{definition}[World enumerating variables]
    \label{def:worldEnumeratingVariables}
    Given a \firstOrderLogic{} theory with constant symbols $\worlddomain$, function symbols $\{\folfunctionof{\secatomenumerator}\wcols\secatomenumeratorin\}$ and predicate symbols $\{\folpredicateof{\atomenumerator}\wcols\atomenumeratorin\}$, we enumerate the world under database semantics by a tuple
    \begin{align*}
        \worldvariables
        = \left(\bigtimes_{\seccatenumeratorin}\bigtimes_{\indindexof{\seccatenumerator}\in[\inddim]^{\indorderof{\seccatenumerator}+1}} \catvariableof{\seccatenumerator,\indindexof{\seccatenumerator}} \right)
        \times \left(\bigtimes_{\catenumeratorin}\bigtimes_{\indindexof{\catenumerator}\in\in[\inddim]^{\indorderof{\catenumerator}}} \catvariableof{\catenumerator,\indindexof{\catenumerator}} \right)
    \end{align*}
    of boolean variables.
    In the world indexed by $\worldindices$, the grounding tensor of the function $\folfunctionof{\secatomenumerator}$ is
    \begin{align*}
        \bencodingofat{\groundingof{\folfunctionof{\secatomenumerator}}}{\indvariableof{\secatomenumerator},\indvariableof{[\inddimof{\secatomenumerator}]}}
        = \sum_{\indindexof{\seccatenumerator}\in[\inddim]^{\indorderof{\seccatenumerator}+1}\wcols\indindexof{\seccatenumerator}=1} \onehotmapofat{\indindexof{\seccatenumerator}}{\indvariableof{\seccatenumerator}}
    \end{align*}
    and of the predicate $\folpredicateof{\atomenumerator}$
    \begin{align*}
        \groundingofat{\folpredicateof{\atomenumerator}}{\indvariableof{[\inddimof{\atomenumerator}]}}
        = \sum_{\indindexof{\seccatenumerator}\in[\inddim]^{\indorderof{\atomenumerator}}\wcols\indindexof{\atomenumerator}=1}
        \onehotmapofat{\indindexof{\atomenumerator}}{\indvariableof{\atomenumerator}} \, .
    \end{align*}
\end{definition}

We further have the restriction that each basis encoded function is a directed tensor.
To restrict to those worlds, where this is true, we further introduce the validation base measure $\basemeasureat{\worldvariables}$ with coordinates
\begin{align*}
    \basemeasureat{\indexedworldvariables} =
    \begin{cases}
        1 & \ifspace \uniquantwrtof{\secatomenumeratorin}{\sum_{\indindexof{\secatomenumerator}\in[\inddim]^{\indorderof{\seccatenumerator}+1}}=1} \\
        0 & \text{else}
    \end{cases} \, .
\end{align*}

\subsect{Representation of terms and formulas}

Combining its \substitutionStructure{} and \semanticStructure{} we represent a \firstOrderLogic{} term $\folterm$ with arity $\inddimof{\folterm}$ as a tensor
\begin{align*}
    \bencodingofat{\folterm}{\indvariableof{\folterm},\indvariableof{[\inddimof{\folterm}]},\worldvariables} =
    \sum_{\worldindices\wcols\basemeasureat{\indexedworldvariables}=1}
    \bencodingofat{\groundingof{\folterm}}{\indvariableof{\folterm},\indvariableof{[\inddimof{\folterm}]}}
    \otimes \onehotmapofat{\worldindices}{\worldvariables} \, .
\end{align*}
and a formula as the tensor
\begin{align*}
    \folexformulawith =
    \sum_{\worldindices\wcols\basemeasureat{\indexedworldvariables}=1} \groundingofat{\folexformula}{\indvariableof{\folexformula}}
    \otimes \onehotmapofat{\worldindices}{\worldvariables} \, .
\end{align*}

From \defref{def:worldEnumeratingVariables} the representation of predicate and function symbols are directly derived from the $\worldvariables$, that is
\begin{align*}
    \folpredicateofat{\atomenumerator}{\indvariableof{[\inddimof{\atomenumerator}]},\indexedworldvariables}
    = \sum_{\indindexof{\seccatenumerator}\in[\inddim]^{\indorderof{\seccatenumerator}+1}\wcols\indindexof{\seccatenumerator}=1}
    \onehotmapofat{\indindexof{\seccatenumerator}}{\indvariableof{\seccatenumerator}}
\end{align*}
and
\begin{align*}
    \bencodingofat{\folfunctionof{\secatomenumerator}}{\indvariableof{\secatomenumerator},\indvariableof{[\inddimof{\secatomenumerator}]},\indexedworldvariables}
    = \sum_{\indindexof{\seccatenumerator}\in[\inddim]^{\indorderof{\seccatenumerator}+1}\wcols\indindexof{\seccatenumerator}=1} \onehotmapofat{\indindexof{\seccatenumerator}}{\indvariableof{\seccatenumerator}}
\end{align*}

\subsect{Case of \propositionalLogic{}}

\PropositionalLogic{} as discussed in (see \charef{cha:logicalRepresentation}) is the special case of \firstOrderLogic{} with empty sets of constant and function symbols, and $\indorderof{\atomenumerator}=0$ for the predicate symbols.
With the convention $[]^{0}=[2]$ the worlds are enumerated by the tuple
\begin{align*}
    \worldvariables = \bigtimes_{\atomenumeratorin} \catvariableof{\atomenumerator}
\end{align*}
which we have in previous chapters abbreviated by $\shortcatindices$.


To represent logical formulas as sets of possible worlds, and distributions of worlds, we applied in \parref{par:one} one-hot encodings of possible worlds.
For the case of \propositionalLogic{}, this is
\begin{align*}
    \onehotmapofat{\worldindices}{\shortcatvariables}
    = \bigotimes_{\atomenumeratorin} \onehotmapofat{\catindexof{\atomenumerator}}{\catvariableof{\atomenumerator}} \, .
\end{align*}

\subsect{One-hot encoding of worlds}

Let us now generalize the one-hot encodings of propositional logic worlds to worlds of \firstOrderLogic{}.
To encode the boolean tensors $\worldindices$ describing \firstOrderLogic{}s as basis elements of a tensor space, we take the one-hot encoding
\begin{align*}
    \onehotmap \defcols \worldindexset \rightarrow \worldtensorspace
\end{align*}
defined by
\begin{align*}
    \onehotmapofat{\worldindices}{\worldvariables}
    =
    \left(\bigotimes_{\seccatenumeratorin}\bigotimes_{\indindexof{\seccatenumerator}\in[\inddim]^{\indorderof{\seccatenumerator}+1}}
        \onehotmapofat{\catindexof{\seccatenumerator,\indindexof{\seccatenumerator}}}{\catvariableof{\seccatenumerator,\indindexof{\seccatenumerator}}}\right)
    \otimes\left(\bigotimes_{\atomenumeratorin}\bigotimes_{\indindexof{\atomenumerator}\in[\inddim]^{\indorderof{\atomenumerator}}}
        \onehotmapofat{\catindexof{\catenumerator,\indindexof{\catenumerator}}}{\catvariableof{\catenumerator,\indindexof{\catenumerator}}} \right)
\end{align*}
This is a tensor of order $\catorder\cdot\inddim^{\indorder}$, in a tensor space of dimension $2^{\left(\catorder\cdot\inddim^{\indorder}\right)}$.
Storage of such tensors in naive formats would not be possible.
However, the basis $\cpformat$ format discussed in \charef{cha:sparseRepresentation} still provides storage with demand linear in the order $\catorder\cdot\inddim^{\indorder}$.


\subsect{Probability distributions}

Having established the formalism of one-hot encodings also in the case of \firstOrderLogic{} worlds, we can now proceed with the definition of distributions and formulas, analogously to the development in \parref{par:one}.
Probability distributions over worlds coinciding on their domain are then non-negative and normed tensors $\probat{\worldvariables}$ where each coordinate of a world $\worldindices$ is captured by a boolean random variable $\catvariableof{\atomenumerator,\shortindindices}$, indicating whether the $\atomenumerator$-th predicate holds on the object tuple indexed by $\shortindindices$.

% High-dimensional - watch out for repetitions!
We notice, that by definition these probability distributions are distributions of
\begin{align*}
    \left(\sum_{\catenumeratorin}\inddim^{\indorderof{\catenumerator}}\right) +  \left(\sum_{\seccatenumeratorin}\inddim^{\indorderof{\seccatenumerator}}\right)
\end{align*}
Booleans.
% One-hot encodings minimal
Unfortunately, it is not possible to design encoding spaces of smaller dimension, when our aim is to get any distribution over possible worlds by an element in the encoding space.
This is due to the fact, that one-hot encodings provide a basis in the tensor space, as will be shown in \charef{cha:coordinateCalculus}.
The reason for the large encoding space dimension is therefore rooted in the equal number of possible worlds and not in an overhead in the dimension of the one-hot encoding space.
We will later in this chapter investigate methods to handle such high-dimensional distributions in the formalism of exponential families.





\sect{Representation of Knowledge Graphs}\label{sec:kgRepresentation}

Let us now represent a specific fragment of \firstOrderLogic{}, namely Description Logics which Knowledge Bases are often referred to as Knowledge Graphs.
We here use the $\mathrm{OWL\,2}$ standard, which encodes the syntax of the description logic $\mathcal{SROIQ(D)}$ \cite{rudolph_foundations_2011}.

\subsect{Representation as unary and binary predicates}

% Reduction to binary
Predicates in knowledge graphs are binary (owl:ObjectProperties) and unary (owl:Class).
%Larger formulas are created by logical connections of these atomic formulas using disjunctions, conjunctions etc.
We enumerate the predicates by $[\folpredicateorder]$, the objects in the domain $\worlddomain$ by $[\inddim]$, and extend the unary predicates to binaries by tensor product with $\onehotmapofat{0}{\indvariableof{1}}$.
A Knowledge Graph on the set $\worlddomain$ of constants (owl:NamedIndividuals) is then the tensor
\begin{align*}
    \kgat{\selvariable,\indvariableof{0},\indvariableof{1}} : [\folpredicateorder] \times [\inddim] \times [\inddim] \rightarrow \ozset \, .
\end{align*}


\subsect{Representation as ternary predicate}\label{subsec:knowledgeGraphTernaryRep}

It has been particulary convenient to represent a Knowledge Graph instead as a grounding of a single ternary predicate $\rdf$.
To this end, the predicates $\folpredicateof{\catenumerator}$ and another object $\mathrdftype$ are added to a domain $\worlddomain$, by extending the $\inddim$ and the index interpretation function accordingly.


% RDF triple: Alternative viewpoint to collection of unary and binary predicates!
Following our notation we understand a Knowledge Graph as a grounding of the rdf triple relation $\rdf$ (being a formula of order 3) on a specific world $\kg$ with individuals $\worlddomain$

We then construct a grounding tensor $\kggroundingof{\rdf}$ out of the world $\kgat{\selvariable,\indvariableof{0},\indvariableof{1}}$ by
\begin{align*}
    \kggroundingof{\rdf} : [\inddim] \times [\inddim] \times [\inddim] \rightarrow \ozset
\end{align*}
where
\begin{align*}
    &\kggroundingofat{\rdf}{\indexedindvariableof{s}, \indexedindvariableof{p}, \indexedindvariableof{o}} \\
    &\quad =
    \begin{cases}
        \kgat{\selvariable=\indindexof{s},\indvariableof{0}=\indindexof{o},\indvariableof{1}=0}
        & \ifspace \indindexof{p} = \invindexinterpretationat{\mathrdftype} \\
        \kgat{\selvariable=\indindexof{p},\indvariableof{0}=\indindexof{s},\indvariableof{1}=\indindexof{o}}
        & \ifspace \indindexof{p} = \invindexinterpretationat{\folpredicateof{\catenumerator}} \quad \text{for some} \quad \catenumerator \\
        0  \quad & \text{else}
    \end{cases} \, .
\end{align*}


Slicing the tensor $\kggroundingof{\rdf}$ along the predicate axis retrieves specific information about roles and can be efficiently be performed on these formats.
The role $\mathrdftype$ has a specific meaning, since it contains from a DL perspective classifications (memberships of named concepts).
Further slicing the tensor along object axis therefore results in membership lists for specific classes (concepts).
One can thus regard $\mathrdftype$ as a placeholder for unitary formulas in a space of binary formulas.

% Triple Stores, sparsity
Exploiting the $\ell_0$-sparsity now leads to a so-called triple store, where $\kggroundingof{\rdf}$ is stored by a listing of those triples $\indindexof{\subsymbol},\indindexof{\predsymbol},\indindexof{\objsymbol}$ such that $\kggroundingofat{\rdf}{\indexedindvariableof{s}, \indexedindvariableof{p}, \indexedindvariableof{o}}=1$
A recent implementation of a triple store exploiting these intuitions is $\mathrm{TENTRIS}$, see \cite{pan_tentris_2020}.
In this work, such decompositions are generalized into more generic CP formats, see \charef{cha:sparseRepresentation}.
% Approximation of KG Groundings
Approximations of grounding tensors by decompositions leads to embeddings of the individuals such as $\mathrm{Tucker}$, $\mathrm{ComplEx}$ and $\mathrm{RESCAL}$ (see \cite{nickel_review_2016}).

% Sparse representation
%Sparse representation of the grounding tensor to a knowledge graph is of central importance, as investigated in \cite{pan_tentris_2020}.
%We here do basis CP for sparse representation.


% basis encoding
For our purposes of evaluating logical formulas such as $\sparql$ queries we use the basis encoding of the groundings, which are depicted by
\begin{center}
    \begin{tikzpicture}[scale=0.3, thick] % , baseline = -3.5pt

    \draw[->-] (0,1)--(0,3) node[midway,left] {\tiny $\headvariable$};
    \draw (-3,1) rectangle (3,-1);
    \node[anchor=center] (text) at (0,0) {\small $\rencodingof{\kggroundingof{\rdf}}$};
    \draw[-<-] (-2,-1)--(-2,-3) node[midway,left] {\tiny $\sindvariable$};
    \draw[-<-] (0,-1)--(0,-3) node[midway,left] {\tiny $\pindvariable$};
    \draw[-<-] (2,-1)--(2,-3) node[midway,left] {\tiny $\oindvariable$};

\end{tikzpicture}
\end{center}




\subsect{$\sparql$ Queries}

The $\sparql$ query language is a syntax to express \firstOrderLogic{} formulas $\folexformula$ and intended to be evaluated given a Knowledge Graph.
We here consider tensor network representations of the $\mathrm{WHERE}{\cdot}$ block.
Given a specific knowledge graph $\kggroundingof{\rdf}$, the execution of query is the interpretation $\groundingof{\folexformula}$, typical represented in a sparse basis CP format where each slice represents a solution mapping.

\subsubsect{Triple Patterns}

\red{Central to $\sparql$ queries are triple patterns, which we understand as slicings of the tensor $\kggroundingof{\rdf}$.}
To each so-called triple pattern we build a corresponding creation tensor. %(see \defref{def:atomCreatingTensor}).
The triple pattern is then evaluated by contraction of the atom creating tensor with $\kggroundingof{\rdf}$.

Let us now provide examples of such pattern tensors.
A unary triple patterns contains a single projection variable, typically related with the subject variable $\sindvariable$ of $\kggroundingof{\rdf}$.
The corresponding pattern tensor is then
\begin{align*}
    \atomcreatorofat{\kgtriple{\provariable}{\mathrdftype}{\folpredicateof{\catenumerator}}}{
        \sindvariable, \pindvariable, \oindvariable, \provariable
    }
    = \identityat{\sindvariable,\provariable}
    \otimes \onehotmapofat{\invindexinterpretationat{\mathrdftype}}{\pindvariable}
    \otimes \onehotmapofat{\invindexinterpretationat{\folpredicateof{\atomenumerator}}}{\oindvariable} \, .
\end{align*}

Binary triple patterns come with two projection variables, typically related with the subject and the object variables $\sindvariable$ and $\oindvariable$.
The pattern tensor to the $\catenumerator$-th predicate is then
\begin{align*}
    \atomcreatorofat{\kgtriple{\provariableof{0}}{\folpredicateof{\catenumerator}}{\provariableof{1}}}{
        \sindvariable, \pindvariable, \oindvariable, \provariableof{0}, \provariableof{1}
    }
    = \identityat{\sindvariable,\provariableof{0}}
    \otimes \onehotmapofat{\invindexinterpretationat{\folpredicateof{\atomenumerator}}}{\pindvariable}
    \otimes \identityat{\oindvariable,\provariableof{1}} \, .
\end{align*}

Contraction with these pattern tensor evaluated the specific triple pattern, and outputs in a boolean tensor the indicator, which objects are members of a specific class (for unary patterns) or which pair of objects are related by a specific relation.
Again, the output of such contractions is a subset encodings of the set of solutions (see \defref{def:subsetEncoding}).

%%%%%%%%%%%% END OF FRIDAY 14.3.
%%%%%%%%%%%%

% Examples
Examples of triple patterns, drawn in \figref{fig:triplePatterns} are
\begin{itemize}
    \item Unary triple pattern with one variable, representing a formula with a single projection variable.
    For the example $\exunarytriple$ see Figure~\ref{fig:triplePatterns}a.
    \begin{align*}
        \atomcreatorofat{\kgtriple{\provariable}{\mathrdftype}{\folpredicateof{\catenumerator}}}{
            \sindvariable, \pindvariable, \oindvariable, \provariable
        }
        = \identityat{\sindvariable,\provariable}
        \otimes \onehotmapofat{\invindexinterpretationat{\mathrdftype}}{\pindvariable}
        \otimes \onehotmapofat{\invindexinterpretationat{\exaunaryrelation}}{\oindvariable}
    \end{align*}
    If and only if the output slice is $\tbasis$, then the corresponding object encoded by the input indices is of class $\exaunaryrelation$.
    \item Binary triple pattern with two variables, representing a formula with two projection variables.
    For the example  $\exbinarytriple$ see Figure~\ref{fig:triplePatterns}b.
    If and only if the output slice is $\tbasis$, then the corresponding object tuple encoded by the input indices has a relation $\exabinaryrelation$.
\end{itemize}

% Projection picture
The composition $\psi (\psi^T)$ of the matrification of the tensor $\psi$ is an orthogonal projection.
That means that applying $\psi (\psi^T)$ is the same map as applying once.


\begin{figure}[t]
    \begin{center}
        \begin{tikzpicture}[scale=0.3,thick] % , baseline = -3.5pt

    \begin{scope}
        [shift={(0,0)}]

        \node[anchor=center] (text) at (-12,2) {$a)$};

        \begin{scope}
            [shift={(-7,2)}]

            \draw (0,-3) rectangle (-6,-5);
            \draw[<-] (-3,-1)--(-3,-3) node[midway,right] {\tiny $\headvariable$};
            \node[anchor=center] (text) at (-3,-4) {$\rencodingof{\kggroundingof{\exunarytriple}}$};
            \draw[<-] (-3,-5)--(-3,-7) node[midway,left] {\tiny $\provariableof{0}$};

        \end{scope}

        \node[anchor=center] (text) at (-5.5,-2) {${=}$};

        \draw[->] (0,1)--(0,3) node[midway,left] {\tiny $\headvariable$};
        \draw (-4,1) rectangle (4,-1);
        \node[anchor=center] (text) at (0,0) {\small $\rencodingof{\kggroundingof{\rdf}}$};

        \draw (-2,-3) rectangle (-4,-5);
        \draw[<-] (-3,-1)--(-3,-3) node[midway,left] {\tiny $\sindvariable$};
        \node[anchor=center] (text) at (-3,-4) {$\delta$};
        \draw[<-] (-3,-5)--(-3,-7) node[midway,left] {\tiny $\provariableof{0}$};

        \draw (-1,-3) rectangle (1,-5);
        \draw[<-] (0,-1)--(0,-3) node[midway,left] {\tiny $\pindvariable$};
        \node[anchor=center] (text) at (0,-4) {$\onehotmapof{\invrdftypesymbol}$};

        \draw (2,-3) rectangle (4,-5);
        \draw[<-] (3,-1)--(3,-3) node[midway,left] {\tiny $\oindvariable$};
        \node[anchor=center] (text) at (3,-4) {$\onehotmapof{\exaunaryrelation}$};

    \end{scope}


    \begin{scope}
        [shift={(24,0)}]

        \node[anchor=center] (text) at (-13,2) {$b)$};

        \begin{scope}
            [shift={(-8,2)}]

            \draw (0.5,-3) rectangle (-6.5,-5);
            \draw[<-] (-3,-1)--(-3,-3) node[midway,right] {\tiny $\headvariable$};
            \node[anchor=center] (text) at (-3,-4) {$\rencodingof{\kggroundingof{\exbinarytriple}}$};

            \draw[<-] (-2,-5)--(-2,-7) node[midway,right] {\tiny $\provariableof{0}$};
            \draw[<-] (-4,-5)--(-4,-7) node[midway,left] {\tiny $\provariableof{1}$};

        \end{scope}

        \node[anchor=center] (text) at (-5.5,-2) {${=}$};

        \draw[->] (0,1)--(0,3) node[midway,left] {\tiny $\headvariable$};
        \draw (-4,1) rectangle (4,-1);
        \node[anchor=center] (text) at (0,0) {\small $\rencodingof{\kggroundingof{\rdf}}$};

        \draw (-2,-3) rectangle (-4,-5);
        \draw[<-] (-3,-1)--(-3,-3) node[midway,left] {\tiny $\sindvariable$};
        \node[anchor=center] (text) at (-3,-4) {$\delta$};
        \draw[<-] (-3,-5)--(-3,-7) node[midway,left] {\tiny $\provariableof{1}$};

        \draw (-1,-3) rectangle (1,-5);
        \draw[<-] (0,-1)--(0,-3) node[midway,left] {\tiny $\pindvariable$};
        \node[anchor=center] (text) at (0,-4) {$\onehotmapof{\exabinaryrelation}$};

        \draw (2,-3) rectangle (4,-5);
        \draw[<-] (3,-1)--(3,-3) node[midway,left] {\tiny $\oindvariable$};
        \node[anchor=center] (text) at (3,-4) {$\delta$};
        \draw[<-] (3,-5)--(3,-7) node[midway,right] {\tiny $\provariableof{0}$};

    \end{scope}

\end{tikzpicture}
    \end{center}
    \caption{Triple patterns of $\sparql$ as tensor networks.
    a) Example of unary triple pattern $\exunarytriple$ specifying whether an individual $\indexinterpretationof{\indindexof{1}}$ is a member of class $C$.
    %Here by $0$ we denote the element $\invindexinterpretationat{\mathrdftype}$
    b) Example of a binary triple pattern $\exbinarytriple$ specifying whether individuals $\indexinterpretationof{\indindexof{1}}$ and $\indexinterpretationof{\indindexof{2}}$ have a relation $R$.
    By $\onehotmapof{\invrdftypesymbol},\onehotmapof{\exaunaryrelation},\onehotmapof{\exabinaryrelation}$ we denote the one-hot encodings of the enumeration of the resources $rdf:type, C$ and $R$.
    }
    \label{fig:triplePatterns}
\end{figure}




\subsubsect{Basic Graph Patterns}

Generic $\sparql$ queries are compositions of triple patterns by logical connectives. % Except for some stuff like regex
These triple patterns possibly share projection variables.
Statements in $\sparql$ can be translated into \propositionalLogic{} combining the triple patterns:
\begin{center}
    \begin{tabular}{|c|c|c|}
        \hline
        \textbf{$\sparql$}                                        & \textbf{\PropositionalLogic{}} & \textbf{Tensor Representation}                                                                   \\
        \hline
        $\{f_1, f_2\}$                                            & $f_1\land f_2$                 & $\bencodingofat{\land}{\headvariableof{f_1\land f_2},\headvariableof{f_1},\headvariableof{f_2}}$ \\
        \hline
        $\mathrm{UNION}\{f_1, f_2\} $                             & $f_1\lor f_2$                  & $\bencodingofat{\lor}{\headvariableof{f_1\lor f_2},\headvariableof{f_1},\headvariableof{f_2}}$   \\
        \hline
        $\mathrm{FILTER}\,\,\mathrm{NOT}\,\,\mathrm{EXISTS}\{f\}$ & $\lnot f$                      & $\bencodingofat{\lnot}{\headvariableof{\lnot f},\headvariableof{f}}$                             \\
        \hline
    \end{tabular}
\end{center}

If a $\sparql$ query consists of these keywords, we find a straight forward corresponding network of triple patterns and encoded logical connectives, by applying our findings of \secref{sec:folConnectiveRepresentation}.
To this end, we prepare for each appearing triple pattern the corresponding pattern tensor, and a copy of $\kggroundingof{\rdf}$.
Here we also copy the term variables $\sindvariable,\pindvariable$ and $\oindvariable$, to ensure that each copy of $\kggroundingof{\rdf}$ shares variables with a single pattern tensor.
Projection variables are not copied, since we need to keep track of them shared among triple patterns.
Then we prepare the basis encoding of logical connectives according to the hierarchy specified in the $\sparql$ query.
Finally we add a $\tbasis$-vector to the final head variable representing the complete $\sparql$ query, to restrict the support to coordiantes corresponding with solution mappings.
We then contract the resulting tensor network, leaving all projection variables open.

If a projection variable is not appearing in the $\mathrm{SELECT}$ statement in front of the $\mathrm{WHERE}\{\cdot\}$-block, we simply exclude it from the open variables of the described contraction.
Note that in that case, the coordinates contain solution counts, i.e. how many assignments to the dropped variable have been a $1$ coordinate.
We can drop this additional information simply by performing a coordinatewise transform with the greater zero indicator $\existquanttrafo$.

% Effective calculus alternative
Here we represented a $\sparql$ query $\impformula$ consistent of multiple triple pattern by instantiating a head variables to each triple pattern.
Alternatively, the more direct hybrid calculus developed in \secref{sec:hybridCalculus} can be applied and the additional head variables avoided.
This is especially compelling, when the $\mathrm{WHERE}\{\cdot\}$-block does not contain further keywords, i.e. it is the conjunction of all triple patterns.
In that case, we avoid the instantiation of head variables (i.e. close the head variables separately by $\tbasis$-vectors) and represent the query by a contraction of all triple pattern tensors.

% Expressivity
We further notice, that any propositional formula acting on the head variables of the triple patterns can be expressed by a hierarchical combination of the key words in the above table.
To find the expression, one can transform a given formula into its conjunctive or disjunctive normal form and apply the statements according to the apperaing operations $\land,\lor$ and $\lnot$.


%% Further $\sparql$ features
%Further $\sparql$ features, which cannot be expressed by a tensor network are:
%\begin{itemize}
%    \item $\mathrm{FILTER}\{\cdot\}$ does not depend on triple patterns (e.g. numeric inequalities, regex functions on strings).
%    We can regard it as another basic formula, which does not result from a slicing of the $\rdf$ grounding tensor.
%    Besides that, we can understand it as formulas and include it in compositions.
%    \item $\mathrm{OPTIONAL}\{\cdot\}$ would result in $\ones$ leg vectors, when there is a missing variable assignment resulting.
%\end{itemize}



\sect{Probabilistic Relational Models}

% MLN in FOL and PL
So far we have studied \MarkovLogicNetworks{} in \propositionalLogic{} as probability distributions over worlds.
In FOL they define probability distributions over relations in worlds with a fixed set of objects.
More generally, such models are probabilistic relational models (see for an overview \cite{getoor_introduction_2019}.


We in this section treat random worlds in \firstOrderLogic{} with fixed domains $\worlddomain$.

%
We in this section show, when and how we can interpret likelihoods of \MarkovLogicNetworks{} in \firstOrderLogic{} in terms of samples of a \MarkovLogicNetwork{} in \propositionalLogic{}.

\subsect{\HybridFOLNetworks{}}

% Templates
Following \cite{richardson_markov_2006} \MarkovLogicNetworks{} in \firstOrderLogic{} are templates for distributions, which instantiate random worlds when choosing a set of objects $\worlddomain$.
Given a fixed set of constants, they then define a distribution over the worlds, which objects correspond with the constants. % this is database semantics!
This applies database semantics, where only those worlds are considered, where the unique name and domain closure assumptions given a set of constants are satisfied.

We count the number of true groundings to a formula by
\begin{align*}
    \countquantifier\enumfolformula|_{\worldindices}
    = \contraction{\groundingofat{\enumfolformula}{\indvariableof{\enumfolformula}}} \, .
\end{align*}
Using these counts as statistics, we now define \HybridFOLNetworks{} as a Computation Activation Network.

\begin{definition}[\HybridFOLNetworks{} (HFLN)]
    Let there be a set $\worlddomain$ of objects and a set $\folformulaset$ of \firstOrderLogic{} formulas.
    Further let $\hybridparam$ as in \defref{def:hybridLogicNetwork} be a tuple of a subset $\hardlegset\subset[\seldim]$, a tuple $\headindexof{\hardlegset}\in\bigtimes_{\selindex\in\hardlegset}[2]$ and $\canparamwithin$.
    We then define the \HybridFOLNetwork{} as the distribution
    \begin{align*}
        \probofat{\folhlnparameters}{\worldvariables}
        &= \breakablenormalizationof{\{\bencodingofat{\countquantifier\enumfolformula}{\headvariableof{\selindex},\worldvariables}\wcols\selindexin\} \\
        & \quad \cup \{\actcorewith\wcols\selindexin\}
        \cup \{\kcoreofat{\selindex,\hardparam}{\headvariableof{\selindex}}\wcols\selindexin\}
        }{\worldvariables}
    \end{align*}
    where
    \begin{align*}
        \kcoreofat{\selindex,\hardparam}{\headvariableof{\selindex}}
        = \begin{cases}
              \fbasisat{\headvariableof{\selindex}} & \ifspace \selindex\in\hardlegset \ncond \headindexof{\selindex} = 0 \\
              \onehotmapofat{\headdimof{\selindex}-1}{\headvariableof{\selindex}} & \ifspace \selindex\in\hardlegset \ncond \headindexof{\selindex} = 1 \\
              \onesat{\headvariableof{\selindex}} & \text{else} \, .
        \end{cases}
    \end{align*}
\end{definition}



The mean parameter polytope is the convex hull of the vectors
\begin{align*}
    \sencodingofat{\folformulaset}{\indexedworldvariables,\selvariable}
\end{align*}
to the worlds $\worldindices$ with $\basemeasureat{\indexedworldvariables}=1$.
These vectors store are the counts of satisfied groundings to each formula, that is
\begin{align*}
    \sencodingofat{\folformulaset}{\indexedworldvariables,\selvariable} = \cardof{
        \{\indindexof{\enumfolformula} \wcols \groundingofat{\enumfolformula}{\indexedindvariableof{\enumfolformula}} = 1 \}
    } \, .
\end{align*}

\subsect{Propositionalization}

% Propositionalization
Let us notice, that different to the case of propositional \HybridLogicNetworks{} treated in \charef{cha:networkRepresentation}, the statistic does not consist of boolean features, when formulas contain variables and we have multiple objects.
One could, however, replace each $\enumfolformula$ by the set of the possible groundings, i.e. substitutions of the formulas variables by any tuple of objects in $\worlddomain$.
The resulting distribution would be an \HybridLogicNetwork{} with boolean statistic, which coincides with the \HybridFOLNetwork{} when posing certain weight sharing conditions on $\canparam$.
The downside of this construction is the increase in the number of features from $\seldim$ to $\sum_{\selindexin} \cardof{\worlddomain}^{\cardof{\indvariableof{\enumfolformula}}}$.
This polynomial in the cardinality of the domain set increase poses significant computational challenges, see \cite{richardson_markov_2006}.

We now show that the \HybridFOLNetwork{} can be propositionalized to a \HybridLogicNetwork{} in \propositionalLogic{}.

\begin{lemma}
    We have
    \begin{align*}
        &\contractionof{\bencodingofat{\countquantifier\enumfolformula}{\headvariableof{\selindex},\worldvariables},\actcorewith}{\worldvariables} \\
        &\quad = \contractionof{
            \{\bencodingofat{\groundingofat{\enumfolformula}{\worldindices,\indindexof{\enumfolformula}}}{\headvariableof{\indindexof{\enumfolformula}}} \wcols \indindexofin{\enumfolformula}\}
            \cup
            \{\actcoreofat{\selindex,\indexedcanparam}{\headvariableof{\indindexof{\enumfolformula}}} \wcols \indindexofin{\enumfolformula}\}
        }{\worldvariables}
    \end{align*}
    and
    \begin{align*}
        &\contractionof{\bencodingofat{\countquantifier\enumfolformula}{\headvariableof{\selindex},\worldvariables},\kcoreofat{\selindex,\hardparam}{\headvariableof{\selindex}}}{\worldvariables} \\
        &\quad = \contractionof{
            \{\bencodingofat{\groundingofat{\enumfolformula}{\worldindices,\indindexof{\enumfolformula}}}{\headvariableof{\indindexof{\enumfolformula}}} \wcols \indindexofin{\enumfolformula}\}
            \cup
            \{\kcoreofat{\selindex,\hardparam}{\headvariableof{\indindexof{\enumfolformula}}} \wcols \indindexofin{\enumfolformula}\}
        }{\worldvariables} \, .
    \end{align*}
\end{lemma}
\begin{proof}
    The first claim holds, since for any world $\worldindices$ we have
    \begin{align*}
        &\contractionof{\bencodingofat{\countquantifier\enumfolformula}{\headvariableof{\selindex},\worldvariables},\actcorewith}{\indexedworldvariables} \\
        &\quad = \prod_{\indindexof{\enumfolformula}\wcols\groundingofat{\enumfolformula}{\worldindices,\indindexof{\enumfolformula}}=1} \expof{\indexedcanparam} \\
        &\quad = \contractionof{
            \{\bencodingofat{\groundingofat{\enumfolformula}{\worldindices,\indindexof{\enumfolformula}}}{\headvariableof{\indindexof{\enumfolformula}}} \wcols \indindexofin{\enumfolformula}\}
            \cup
            \{\actcoreofat{\selindex,\indexedcanparam}{\headvariableof{\indindexof{\enumfolformula}}} \wcols \indindexofin{\enumfolformula}\}
        }{\indexedworldvariables} \, .
    \end{align*}
    For $\selindex\notin\hardlegset$ the second claim is trivial, since any contraction of a basis encoding with a trivial head is trivial.
    If $\selindex\in\hardlegset$ and $\headindexof{\selindex}=0$ then
    \begin{align*}
        &\contractionof{\bencodingofat{\countquantifier\enumfolformula}{\headvariableof{\selindex},\worldvariables},\kcoreofat{\selindex,\hardparam}{\headvariableof{\selindex}}}{\indexedworldvariables} \\
        &\quad=
        \begin{cases}
            1 & \ifspace \forall \indindexof{\enumfolformula} \defcols  \groundingofat{\enumfolformula}{\worldindices,\indindexof{\enumfolformula}} = 0\\
            0 & \text{else}
        \end{cases} \\
        &\quad= \contractionof{
            \{\bencodingofat{\groundingofat{\enumfolformula}{\worldindices,\indindexof{\enumfolformula}}}{\headvariableof{\indindexof{\enumfolformula}}} \wcols \indindexofin{\enumfolformula}\}
            \cup
            \{\kcoreofat{\selindex,\hardparam}{\headvariableof{\indindexof{\enumfolformula}}} \wcols \indindexofin{\enumfolformula}\}
        }{\indexedworldvariables} \, .
    \end{align*}
    For $\headindexof{\selindex}=1$ this holds analogously.
\end{proof}

Each substitution of the variables in $\enumfolformula$ by objects in $\worlddomain$, which satisfies the formula in the world $\worldindices$, therefore provides a factor of $\expof{\canparamat{\indexedselvariable}}$ to the probability of $\worldindices$.


%\red{Here we directly define them as exponential families distributing $\worldvariables$ for a given set of objects $\worlddomain$.}
%\red{To avoid a similar discussion as in \charef{cha:networkRepresentation} we directly allow for boolean base measures and call the distributions \HybridFOLNetworks{}.}
%
%\begin{definition}[\HybridFOLNetworks{} (HFLN)]
%    Let there be a set $\folformulaset$ of \firstOrderLogic{} formulas with maximal arity $\indorder$, which is enumerated by a selection variable $\selvariable$ of dimension $\seldim$.
%    Further, let there be a set of objects $\worlddomain$ and a boolean base measure $\basemeasureat{\shortindvariables}$.
%    The family of \HybridFOLNetworks{} $\expfamilyof{\restfolformulaset,\basemeasure}$ defined by the tuple $(\folformulaset,\worlddomain,\basemeasure)$ is the exponential family of joint distributions to the variables $\worldvariables$ with the statistics
%    \begin{align*}
%        \sstat^{\restfolformulaset}_{\selindex}\left[\indexedworldvariables\right]
%        = \contraction{\groundingof{\enumfolformula}}
%    \end{align*}
%    and the base measure $\basemeasure$.
%%    The \MarkovLogicNetwork{} instantiated for a given set of objects $\worlddomain$ and a base measure $\basemeasure$ is the random world, which is a member of the exponential family with sufficient statistics
%%    \begin{align*}
%%        \sstatcoordinateofat{\selindex}{\indexedworldvariables} = \contraction{\groundingof{\enumfolformula}}
%%        %\sstat_{\selindex}(\worldindices)  = \contraction{\groundingof{\folexformula_\selindex}} % Formulas can have different
%%    \end{align*}
%%    and canonical parameters $\weight$.
%\end{definition}
%
%Each element of the family $\expfamilyof{\restfolformulaset,\basemeasure}$ is represented by a canonical parameter $\canparamat{\selvariable}$.


%We will in the next sections explore an alternative way to apply the theory of \charef{cha:networkRepresentation} and \charef{cha:networkReasoning}, namely based on importance formulas.


%% Interpretation
%The statistics
%\begin{align*}
%    \contraction{\groundingof{\folexformula_\selindex}} % Formulas can have different
%\end{align*}
%can be interpreted as the number of substitutions to a formula, such that the formula ist satisfied.
%Each substitution satisfying a formula adds a factor of $\expof{\canparam_\selindex}$ to the probability of the respective world before normalization.


%
%When constructing a world tensor to a theory with predicates of different order, we already argued that we extend the arity of predicates by tensor products with $\onehotmapof{0}$.
%To define random world tensors, we then restrict the corresponding base measure to be supported only on those worlds where the extended predicates hold only at the individual $\exindividualof{0}$ at the extended axis.


% Comparison with PL MLN
%We choose extraction formulas $\extformulaof{\atomenumerator}$ such that any formula in the FOL MLN has a propositional equivalent (see \defref{def:propositionalEquivalent}).
%The statistic map is then a formula selecting tensor as in the propositional logic case contracted with the groundings of $\extformulaof{\atomenumerator}$.






\subsect{Conditioning on an importance formulas}

%\red{Analogous to a guard formula in \cite[Definition 6.11]{koller_probabilistic_2009}!}

%The boolean base measure $\basemeasure$ of a \HybridFOLNetwork{} is the subset encoding of the possible worlds which have a non-vanishing probability with respect to any member of the family.
%We now construct specific base measures based on a fixed grounding tensor of an importance formula.
%This will reduce the number of object tuples influencing the probability distribution in order to arrive at an interpretation of FOL MLNs as likelihoods to datasets of propositional MLNs.

To reduce the number of object tuples influencing the probability distribution, we now condition \HybridFOLNetworks{} on situations where a formula, called the importance formula, has a fixed grounding tensor.

To this end, we mark pairs of term indices relevant to the distributions by an auxiliary index $\datindexin$.
Given a set $\{\indindexof{[\indorder]}^{\datindex} \wcols \datindexin \}$ of indices to the important tuples we build a set encoding (see \defref{def:subsetEncoding})
\begin{align*}
    \fixedimpformula = \sum_{\datindexin} \left(
    \bigotimes_{\indenumeratorin} \onehotmapofat{\indindexof{\indenumerator}^{\datindex}}{\indvariableof{\indenumerator}}
    \right) \, .
\end{align*}

% Interpretation as grounding
We interpret the tensor $\fixedimpformula$ as the grounding of a formula, which we call the importance formula.

% Restricting to worlds with identical grounding
To have a constant importance formula we define a syntactic representation and restrict the support of the \HybridFOLNetwork{} to those world coinciding with groundings of the importance formula coinciding with $\fixedimpformula$ by designing a base measure
\begin{align*}
    \fixedimpbm
    = \begin{cases}
          1 & \ifspace \groundingofat{\impformula}{\indvariableof{\impformula}} = \fixedimpformula \\
          0 & \text{else}
    \end{cases} \, .
\end{align*}

% Conditioning on exquery
The base measure restricts the \HybridFOLNetwork{} to be those worlds, where $\groundingof{\impformula}$ is coincides with the fixed tensor $\fixedimpformula$.
Intuitively, $\groundingof{\impformula}$ represents certain evidence about a \firstOrderLogic{} world, whereas other formulas are uncertain.


%\begin{assumption}
%    \label{ass:importanceBasemeasure}
%    Given a base measure $\fixedimpbm$, we assume that there is an importance formula $\impformulaat{\shortindvariables}$ such that
%    \begin{align*}
%        \fixedimpbm
%        = \begin{cases}
%              1 & \ifspace \groundingofat{\impformula}{\indvariableof{\impformula}} = \fixedimpformula \\
%              0 & \text{else}
%        \end{cases} \, .
%    \end{align*}
%\end{assumption}


\subsect{Decomposition of the log likelihood}


% Extraction query
To reduce the likelihood of a world to we make the assumption that all formulas in a \HybridFOLNetwork{} are of the form
\begin{align}
    \label{eq:folImplicationForm}
%    \folexformula_{\selindex}(\individuals) =
    \enumfolformulaat{\indvariableof{\enumfolformula}}
    = \left( \impformulaat{\shortindvariables} \Rightarrow \headfolformulaofat{\selindex}{\indvariableof{\enumfolformula}} \right)
\end{align}
that is a rule with the importance formula being the premise.
In particular, we assume, that they depend on all term variables $\shortindvariables$.
If this is not the case, we extend the formula trivially on the missing term variables.
When this assumption holds, we can think of the importance formula as a conditions on individuals to satisfy a statistical relation given by $\headfolexformula$.

Towards connecting with \propositionalLogic{}, we further make the assumption, that we can decompose the formula $\headfolformulaof{\selindex}$ in what we will call extraction formulas.

\begin{assumption}
    \label{ass:propositionalHeads}
    We assume that there exist formulas $\{\extformulaofat{\catenumerator}{\worldvariables,\shortindvariables} \wcols \catenumeratorin\}$, which we refer to as atom extraction formulas, and an importance formula $\impformulaat{\shortindvariables}$ such that the following holds.
    To each \firstOrderLogic{} formula $\enumfolformula$ there is another \firstOrderLogic{} formula $\headfolformulaofat{\selindex}{\indvariableof{\enumfolformula}}$ and a propositional formula $\enumformulaat{\shortcatvariables}$ such that
    \begin{align*}
        \enumfolformulaat{\worldvariables,\shortindvariables}
        = \left( \impformulaat{\worldvariables,\shortindvariables} \Rightarrow \headfolformulaofat{\selindex}{\worldvariables,\shortindvariables} \right)
    \end{align*}
    and
    \begin{align*}
        \headfolformulaofat{\selindex}{\shortindvariables} =
        \contractionof{
            \{\enumformulaat{\shortcatvariables}\} \cup \{\bencodingofat{\extformulaof{\catenumerator}}{\catvariableof{\catenumerator},\worldvariables,\shortindvariables}
            \wcols \catenumeratorin\}
        }{\shortindvariables} \, .
    \end{align*}
\end{assumption}

We depict the assumption, that any formula is of the form \eqref{eq:folImplicationForm} in the diagram
\begin{center}
    \begin{tikzpicture}[scale=0.35, yscale=1, thick] % , baseline = -3.5pt




\draw (1,-1) rectangle (7,-3);
\node[anchor=center] (text) at (4,-2) {$\groundingof{\left(\impformula\Rightarrow\folexformula\right)}$};

\draw[] (2,-3) -- (2,-5) node[midway,left] {\tiny $\individualvariableof{0}$};
\node[anchor=center] (text) at (4,-4) {$\cdots$};
\draw[] (6,-3) -- (6,-5) node[midway,right] {\tiny $\individualvariableof{\individualorder-1}$};


\node[anchor=center] (text) at (10,-2) {${=}$};



%\draw (1,-1) rectangle (7,-3);
%\node[anchor=center] (text) at (4,-2) {$\rencodingof{\impformula}$};

\begin{scope}[shift={(12,-2)}]

\draw (1,3) rectangle (12,5);
\node[anchor=center] (text) at (6.5,4) {$\exformula$};

\draw[->] (2.5,1) -- (2.5,3) node[midway,right] {\tiny $\atomicformulaof{0}$};
\draw (1,-1) rectangle (4,1);
\node[anchor=center] (text) at (2.5,0) {$\rencodingof{\groundingof{\extformulaof{0}}}$};
\node[anchor=center] (text) at (2.5,-2) {$\cdots$};

\node[anchor=center] (text) at (6.5,0) {$\cdots$};

\draw[->] (10.5,1) -- (10.5,3) node[midway,right] {\tiny $\atomicformulaof{\atomorder-1}$};
\draw (8.75,-1) rectangle (12.25,1);
\node[anchor=center] (text) at (10.5,0) {$\rencodingof{\groundingof{\extformulaof{\atomorder\shortminus1}}}$};
\node[anchor=center] (text) at (10.5,-2) {$\cdots$};

\draw[<-] (13,-3) -- (3.5,-3) ;
\draw[<-] (13,-5) -- (1.5,-5) ;

\draw[fill] (11.5,-3) circle (0.25cm);
\draw[->] (11.5,-3) -- (11.5,-1);

\draw[fill] (9.5,-5) circle (0.25cm);
\draw[->] (9.5,-5) -- (9.5,-1);

\draw[fill] (3.5,-3) circle (0.25cm);
\draw[->] (3.5,-3) -- (3.5,-1);

\draw[fill] (1.5,-5) circle (0.25cm);
\draw[->] (1.5,-5) -- (1.5,-1);


\draw[fill] (7.5,-3) circle (0.25cm);
\draw[<-] (7.5,-3) -- (7.5,-7) node[right] {\tiny $\individualvariableof{\individualorder-1}$} ;

\node[anchor=center] (text) at (6.5,-6) {$\cdots$};

\draw[fill] (5.5,-5) circle (0.25cm);
\draw[<-] (5.5,-5) -- (5.5,-7) node[left] {\tiny $\individualvariableof{0}$} ;


\draw (13,-2) rectangle (15,-6);
\node[anchor=center] (text) at (14,-4) {$\rencodingof{\impformula}$};
\node[anchor=center] (text) at (12,-3.75) {$\vdots$};
\draw[->] (15,-4) -- (16,-4);
\draw[] (17,-4) -- (16,-4);
\draw (17,-3) rectangle (19,-5);
\node[anchor=center] (text) at (18,-4) {$\tbasis$};

\end{scope}




\node[anchor=center] (text) at (30,-2) {${+}$};





\begin{scope}[shift={(22,1)}]

\draw (13,0) rectangle (15,2);
\node[anchor=center] (text) at (14,1) {$\fbasis$};

\draw[] (14,-1) --(14,0);
\draw[->] (14,-2) --(14,-1);

\draw (12,-2) rectangle (16,-4);
\node[anchor=center] (text) at (14,-3) {$\rencodingof{\impformula}$};

\draw[<-] (12.5,-4) -- (12.5,-6) node[left] {\tiny $\individualvariableof{0}$} ;
\node[anchor=center] (text) at (14,-5) {$\cdots$};
\draw[<-] (15.5,-4) -- (15.5,-6) node[right] {\tiny $\individualvariableof{\individualorder\shortminus1}$} ;


		
\end{scope}

\end{tikzpicture}
\end{center}
where the second summand depends only on the query $\impformula$ and therefore does not appear in the likelihood.


%\begin{example}[Trivial importance formula]
%	When the importance formula is always satisfied, any tuple of objects contributes to the likelihood. 
%	This original approach to \MarkovLogicNetworks{} \cite{richardson_markov_2006} however leads to many datapoints which are also dependent on each other.
%\end{example}


\subsect{Reduction to \propositionalLogic{}}

We now make additional assumptions to decompose the partition function of an \HybridFOLNetwork{} as a product of \HybridLogicNetwork{} partition functions.

\begin{assumption}
    \label{ass:independentTuples}
    Given an importance formula $\impformula$ and a tensor $\fixedimpformula$ we consider the base measure
    \begin{align*}
        \fixedimpbm
        = \begin{cases}
              1 & \ifspace \groundingofat{\impformula}{\indvariableof{\impformula}} = \fixedimpformulawith \\
              0 & \text{else}
        \end{cases}
    \end{align*}
    of worlds with grounding of $\impformula$ by $\fixedimpformula$.
    Let further enumerate by $\datindexin$ the support of $\fixedimpformula$, that is $\sampleind$ are the indices of $\shortindvariables$ with $\fixedimpformulaat{\shortindvariables=\sampleind}=1$.
    We assume that with respect to $\fixedimpformula$ the variables
    \begin{align}
        \left(\groundingofat{\extformulaof{\atomenumerator}}{\shortindvariables=\sampleind}\right)
    \end{align}
    are for $\atomenumeratorin$ and $\datindexin$ independent and uniformly distributed.
\end{assumption}

When the above assumption holds, we now show that the probability of a \firstOrderLogic{} world with respect to a \HybridFOLNetwork{} coincides with the likelihood of a dataset in a propositional \HybridLogicNetwork{}.

\begin{theorem}
    \label{the:FOLworldToPLdataset}
    Let there be a set of formulas $\folformulaset$ such that \assref{ass:propositionalHeads} and \assref{ass:independentTuples} hold with an importance formula $\impformula$ and a tensor $\fixedimpformula$.
    Then for any $\hybridparam$ and any world $\catindexof{\folworldsymbol}$ with $\groundingof{\impformula}=\fixedimpformula$ we have
    \begin{align*}
        \frac{1}{\datanum} \lnof{\condprobwrtof{\folhlnparameters}{\indexedworldvariables}{\groundingof{\impformula}=\fixedimpformula}}
        = \centropyof{\empdistribution}{\probof{\hlnparameters}} -
        \frac{\lnof{\contraction{\fixedimpbm}}}{\datanum}
    \end{align*}
    where $\formulaset$ is the set of propositional equivalents to $\folformulaset$ (see \assref{ass:propositionalHeads}) and $\datamap$ the data map with evaluation at $\datindexin$ by the enumerated non-vanishing coordinates of $\fixedimpformulawith$
    \begin{align*}
        \datapoint
        = \big(\groundingofat{\extformulaof{0}}{\shortindvariables=\sampleind},\ldots,\groundingofat{\extformulaof{\atomorder-1}}{\shortindvariables=\sampleind}\big) \, .
    \end{align*}
\end{theorem}

To show the theorem, we show first in the following lemma the factorization of the partition function of the \HybridFOLNetwork{}.

% NEW
\begin{lemma}
    \label{lem:FOLpartitionfunctionfactorization}
    Under the assumptions of \theref{the:FOLworldToPLdataset}, we have
    \begin{align*}
        &\contraction{
            \bencodingofat{\restfolformulaset}{\headvariables,\worldvariables},\hypercoreofat{\hybridparam}{\headvariables},\fixedimpbm
        } \\
        &\quad=\contraction{\fixedimpbm} \cdot
        \left(  \prod_{\shortindindices\wcols\fixedimpformulaat{\indexedshortindvariables}=0} \hypercoreofat{\selindex,\hybridparam}{\headvariable=1}\right) \cdot \\
        &\quad\quad \left(\frac{1}{2^{\atomorder}}
        \cdot \contraction{\hlnstatccwith,\hypercoreofat{\hybridparam}{\headvariables}}
        \right)^{\datanum} \, .
    \end{align*}
\end{lemma}
\begin{proof}
    Using the assumption on the structure of the formulas we have
    \begin{align*}
        \groundingof{\countquantifier\enumfolformula}
        = \groundingof{\countquantifier\lnot\impformula} + \groundingof{\countquantifier(\impformula\land\headfolformulaof{\selindex})}
    \end{align*}
    and for any $\worldindices$
    \begin{align*}
        &\contractionof{\bencodingofat{\countquantifier\enumfolformula}{\headvariableof{\selindex}},\hypercoreofat{\selindex,\hybridparam}{\headvariable},\fixedimpbm}{\indexedworldvariables} \\
        &\quad = \left(\prod_{\shortindindices\wcols\fixedimpformulaat{\indexedshortindvariables}=0} \hypercoreofat{\selindex,\hybridparam}{\headvariable=1}\right) \cdot
        \contractionof{\bencodingofat{\countquantifier(\impformula\land\headfolformulaof{\selindex})}{\headvariableof{\selindex}},\hypercoreofat{\selindex,\hybridparam}{\headvariableof{\selindex}},\fixedimpbm}{\indexedworldvariables} \\
    \end{align*}
    Here by $\hypercoreofat{\selindex,\hybridparam}{\headvariable}$ we denote the two-dimensional realization of the activation core.
    We use that the grounding tensor of $\impformula$ is constant among the by $\fixedimpbm$ supported worlds and thus
    \begin{align*}
        &\contractionof{\bencodingofat{\countquantifier(\impformula\land\headfolformulaof{\selindex})}{\headvariableof{\selindex}},\hypercoreofat{\selindex,\hybridparam}{\headvariableof{\selindex}},\fixedimpbm}{\worldvariables} \\
        % Using that impformula is constant
        &\quad =
        \contractionof{
            \bigcup_{\datindexin}\left\{\bencodingofat{\groundingofwrt{\headfolformulaof{\selindex}}{\sampleind}}{\headvariableof{\selindex,\datindex}},\hypercoreofat{\selindex,\hybridparam}{\headvariableof{\selindex,\datindex}}\right\}
            \cup\{\fixedimpbm\}
        }{\worldvariables} \, .
    \end{align*}
    For each $\datindexin$ we have by \assref{ass:propositionalHeads}
    \begin{align*}
        % Using that headformula decomposes to propositional
        &\contractionof{
            \bencodingofat{\groundingofwrt{\headfolformulaof{\selindex}}{\sampleind}}{\headvariableof{\selindex,\datindex}},\hypercoreofat{\selindex,\hybridparam}{\headvariableof{\selindex,\datindex}}
        }{\worldvariables} \\
        & \quad =
        \contractionof{
            \{\bencodingofat{\groundingofwrt{\extformulaof{\atomenumerator}}{\sampleind}}{\headvariableof{\atomenumerator,\datindex}} \wcols \atomenumeratorin \}
            \cup \{\bencodingofat{\enumformula}{\formulavar,\headvariableof{\atomenumerator,\datindex}},\hypercoreofat{\selindex,\hybridparam}{\formulavar}\}
        }{\worldvariables} \, .
    \end{align*}
    From \assref{ass:independentTuples} we know
    \begin{align*}
        \contractionof{
            \{\bencodingofat{\groundingofwrt{\extformulaof{\atomenumerator}}{\sampleind}}{\headvariableof{\atomenumerator,\datindex}}
            \wcols \atomenumeratorin \ncond \datindexin \} \cup \{\fixedimpbm\}
        }{\headvariableof{[\atomorder]\times[\datanum]}}
        = \frac{\contraction{\fixedimpbm}}{2^{\atomorder\cdot\datanum}} \cdot \onesat{\headvariableof{[\atomorder]\times[\datanum]}} \, .
    \end{align*}
    Combining the above we get
    \begin{align*}
        &\contraction{
            \bencodingofat{\restfolformulaset}{\headvariables,\worldvariables},\hypercoreofat{\hybridparam}{\headvariables},\fixedimpbm
        } \\
        &\quad=
        \frac{\contraction{\fixedimpbm}}{2^{\atomorder\cdot\datanum}} \cdot \\
        &\quad\quad \left(  \prod_{\shortindindices\wcols\fixedimpformulaat{\indexedshortindvariables}=0} \hypercoreofat{\selindex,\hybridparam}{\headvariable=1}\right)
        \prod_{\datindexin}\contraction{
            \bencodingofat{\hlnstat}{\headvariables,\headvariableof{[\atomorder]\times\{\datindex\}}},\hypercoreofat{\hybridparam}{\headvariables}
        } \\
        &\quad=\contraction{\fixedimpbm} \cdot
        \left(  \prod_{\shortindindices\wcols\fixedimpformulaat{\indexedshortindvariables}=0} \hypercoreofat{\selindex,\hybridparam}{\headvariable=1}\right) \\
        &\quad\quad \left(\frac{1}{2^{\atomorder}}
        \cdot \contraction{\hlnstatccwith,\hypercoreofat{\hybridparam}{\headvariables}}
        \right)^{\datanum} \, . \qedhere
    \end{align*}
\end{proof}

% For
We notice, that for the event $\groundingof{\impformula}=\fixedimpformula$ to be of non-vanishing probability, we need to have $\headindexof{\hardlegset}=\ones_\hardlegset$ or $\fixedimpformulawith=\onesat{\indvariableof{\impformula}}$.
With this lemma, we are now show \theref{the:FOLworldToPLdataset}.

\begin{proof}[Proof of \theref{the:FOLworldToPLdataset}]
    We have
    \begin{align*}
        \lnof{\condprobwrtof{\folhlnparameters}{\indexedworldvariables}{\groundingof{\impformula}=\fixedimpformula}}
        &= \lnof{\contractionof{
            \bencodingofat{\restfolformulaset}{\headvariables,\worldvariables},\hypercoreofat{\hybridparam}{\headvariables},\fixedimpbm
        }{\indexedworldvariables}} \\
        &\quad - \lnof{\contraction{
            \bencodingofat{\restfolformulaset}{\headvariables,\worldvariables},\hypercoreofat{\hybridparam}{\headvariables},\fixedimpbm
        }}
    \end{align*}
    While the second term is decomposed by \lemref{lem:FOLpartitionfunctionfactorization} we now derive a decomposition of the first term.
    By \assref{ass:propositionalHeads} and $\groundingof{\impformula}=\fixedimpformula$ we have
    \begin{align*}
        &\contractionof{
            \bencodingofat{\restfolformulaset}{\headvariables,\worldvariables},\hypercoreofat{\hybridparam}{\headvariables},\fixedimpbm
        }{\indexedworldvariables} \\
        &\quad =
        \left(\prod_{\shortindindices\wcols\fixedimpformulaat{\indexedshortindvariables}=0} \hypercoreofat{\selindex,\hybridparam}{\headvariable=1}\right)
        \cdot \\
        &\quad\quad \prod_{\datindexin}
        \contraction{
            \{\bencodingofat{\groundingofwrt{\extformulaof{\atomenumerator}}{\sampleind}}{\headvariableof{\atomenumerator},\indexedworldvariables} \wcols \atomenumeratorin \}
            \cup \{\hlnstatccwith,\hypercoreof{\hybridparam}{\headvariables}\}
        } \, .
    \end{align*}
    With \lemref{lem:FOLpartitionfunctionfactorization} we then have
    \begin{align*}
        &\frac{1}{\datanum}\lnof{\condprobwrtof{\folhlnparameters}{\indexedworldvariables}{\groundingof{\impformula}=\fixedimpformula}} \\
        &\quad= \frac{1}{\datanum}\sum_{\datindexin} \lnof{\contraction{
            \{\bencodingofat{\groundingofwrt{\extformulaof{\atomenumerator}}{\sampleind}}{\headvariableof{\atomenumerator},\indexedworldvariables} \wcols \atomenumeratorin \}
            \cup \{\hlnstatccwith,\hypercoreof{\hybridparam}{\headvariables}\}
        }} \\
        &\quad\quad - \lnof{\contraction{
            \{\bencodingofat{\groundingofwrt{\extformulaof{\atomenumerator}}{\sampleind}}{\headvariableof{\atomenumerator},\worldvariables} \wcols \atomenumeratorin \}
            \cup \{\hlnstatccwith,\hypercoreof{\hybridparam}{\headvariables}\}
        }} - \frac{\lnof{\contraction{\fixedimpbm}}}{\datanum} \\
        &\quad = \centropyof{\empdistribution}{\probof{\hlnparameters}} - \frac{\lnof{\contraction{\fixedimpbm}}}{\datanum} \, . \qedhere
    \end{align*}
\end{proof}

% Independent data investigation
Let us now investigate, in which cases the \assref{ass:independentTuples} of independent data can be matched.

\begin{example}
    If the $\impformula$ and $\extformulas$ are predicates applied on variables, and the index tuples $\sampleind$ are pairwise different, then \assref{ass:independentTuples} is met.
    This is the case, since the values $\groundingof{\extformulaofat{\atomenumerator}{\shortindvariables=\sampleind}}$ are determined by different variables in $\worldvariables$.
\end{example}



There are situations, where \assref{ass:independentTuples} is violated.
\begin{itemize}
    \item extraction formula being a) conjunctions of predicates: Probability that they are satisfied decreases
    b) disjunctions of predicates: Probability that they are satisfied increases
    \item extraction formula coinciding with importance formula: Always satisfied, in this case still boolean
    \item extraction formulas contradicting each other, more general not independent from each other
\end{itemize}

%Let us notice, that non-boolean base measures could be treated in a same manner, but several developments in this work, such as cross-entropy decompositions in \charef{cha:probReasoning} would receive further terms.



\begin{remark}[Approximation by Independent Samples]
    As argued above, we do not have independent samples in general.
    As a consequence, we cannot apply \lemref{lem:FOLpartitionfunctionfactorization} to decompose the partition function term of the log-probability into factors to each solution map of $\impformula$.
    In this case, it might be still benefitial to use the reduction to the likelihood of a HLN, but needs to understand it as a approximation to the true world probability.

    %
    If the expectations of each sample with respect to the marginalized distributions coincide, the average of empirical distribution also coincides with these (by linearity).
    When the creation of samples has sufficient mixing properties, the empirical distribution converges to this expectation in the asymptotic case of large numbers of samples.

\end{remark}



\subsect{Sample extraction from \firstOrderLogic{} worlds}

We have observed that in certain situations the log-likelihood of a \firstOrderLogic{} world with respect to a \HybridFOLNetwork{} coincides with the likelihood of a data set in a propositional \HybridLogicNetwork{}.
Let us now investigate the extraction process of these data set.
%The decomposition of the likelihood suggests the following approach to generate samples from groundings:
%%We propose the following approach to generate datacores from groundings:
%\begin{itemize}
%    \item Define an importance formula $\impformula$, which we decompose in the basis CP decomposition and interpret each slice as the one-hot encoding of the datapoint.
%    \item Define for $\atomenumeratorin$ extraction formulas $\extformulaof{\atomenumerator}$ generating the atoms $\catvariableof{\atomenumerator}$.
%    %Predicates along with assignment of variables / constants to its positions.
%    \item Contract the groundings of each formula $\extformulaof{\atomenumerator}$ with the grounding of $\impformula$ to build a data core.
%\end{itemize}
%\subsubsect{Representation by Tensor Networks}
We model the extraction process as a relation between a tuple of individuals and the extracted world in the factored system of atoms $\catvariableof{\atomenumerator}$.

\begin{definition}
    \label{def:extractionRelation}
    Given a \firstOrderLogic{} world $\worldindices$, an importance formula $\impformula$ and extraction formulas $\extformulaof{\catenumerator}$ for $\catenumeratorin$, we define the extraction relation
    \begin{align*}
        \extractionrelation \subset \left(\symindstates\right) \otimes \left(\atomstates\right)
    \end{align*}
    by
    \begin{align*}
        \extractionrelation
        = \{ (\shortindindices, \shortcatindices)
        \wcols  \groundingofat{\impformula}{\indexedshortindvariables} = 1 \ncond \uniquantwrtof{\catenumeratorin}{\catindexof{\atomenumerator} = \extformulaofat{\atomenumerator}{\indexedshortindvariables}} \} \, .
    \end{align*}
\end{definition}

The encoding of an extraction relation is the tensor
\begin{align*}
    \bencodingofat{\extractionrelation}{\shortindvariables,\shortcatvariables} \subset \left(\indspace\right) \otimes \left(\atomspace\right) \,
\end{align*}
and drawn in a contraction diagram by
\begin{center}
    \begin{tikzpicture}[scale=0.35, yscale=1, thick] % , baseline = -3.5pt


    \draw[] (2,-1) -- (2,1) node[midway,left] {\tiny $\catvariableof{0}$};
    \node[anchor=center] (text) at (4,0) {$\cdots$};
    \draw[] (6,-1) -- (6,1) node[midway,right] {\tiny $\catvariableof{\atomorder-1}$};

    \draw (1,-1) rectangle (7,-3);
    \node[anchor=center] (text) at (4,-2) {$\bencodingof{\extractionrelation}$};

    \draw[] (2,-3) -- (2,-5) node[midway,left] {\tiny $\individualvariableof{0}$};
    \node[anchor=center] (text) at (4,-4) {$\cdots$};
    \draw[] (6,-3) -- (6,-5) node[midway,right] {\tiny $\individualvariableof{\individualorder-1}$};


    \node[anchor=center] (text) at (10,-2) {${=}$};


    \begin{scope}
        [shift={(12,0)}]

        \draw[->-] (2.5,1) -- (2.5,3) node[midway,right] {\tiny $\catvariableof{0}$};
        \draw (1,-1) rectangle (4,1);
        \node[anchor=center] (text) at (2.5,0) {$\bencodingof{\groundingof{\extformulaof{0}}}$};
        \node[anchor=center] (text) at (2.5,-2) {$\cdots$};

        \node[anchor=center] (text) at (6.5,0) {$\cdots$};

        \draw[->-] (10.5,1) -- (10.5,3) node[midway,right] {\tiny $\catvariableof{\atomorder-1}$};
        \draw (8.75,-1) rectangle (12.25,1);
        \node[anchor=center] (text) at (10.5,0) {$\bencodingof{\groundingof{\extformulaof{\atomorder\shortminus1}}}$};
        \node[anchor=center] (text) at (10.5,-2) {$\cdots$};

        \draw[-<-] (13,-3) -- (3.5,-3) ;
        \draw[-<-] (13,-5) -- (1.5,-5) ;

        \drawvariabledot{11.5}{-3}
        \draw[->-] (11.5,-3) -- (11.5,-1);

        \drawvariabledot{9.5}{-5}
        \draw[->-] (9.5,-5) -- (9.5,-1);

        \drawvariabledot{3.5}{-3}
        \draw[->-] (3.5,-3) -- (3.5,-1);

        \drawvariabledot{1.5}{-5}
        \draw[->-] (1.5,-5) -- (1.5,-1);

        \drawvariabledot{7.5}{-3}
        \draw[-<-] (7.5,-3) -- (7.5,-7) node[right] {\tiny $\individualvariableof{\individualorder-1}$} ;

        \node[anchor=center] (text) at (6.5,-6) {$\cdots$};

        \drawvariabledot{5.5}{-5}
        \draw[-<-] (5.5,-5) -- (5.5,-7) node[left] {\tiny $\individualvariableof{0}$} ;


        \draw (13,-2) rectangle (17,-6);
        \node[anchor=center] (text) at (15,-4) {$\bencodingof{\groundingof{\impformula}}$};
        \node[anchor=center] (text) at (12,-3.75) {$\vdots$};
        \draw[->-] (17,-4) -- (18,-4);
        \drawvariabledot{18}{-4}
        \draw[] (18,-4) -- (19,-4);
        \draw (19,-3) rectangle (21,-5);
        \node[anchor=center] (text) at (20,-4) {$\tbasis$};

    \end{scope}


\end{tikzpicture}
\end{center}
Here the contraction of $\bencodingof{\impformula}$ with the truth vector $\tbasis$ represents the matching condition posed by $\impformula$ when extracting pairs of individuals.

%% Empirical Distribution
The empirical distribution of the extracted data is then the normalized contraction leaving only the legs to the extracted atomic formulas open, that is
\begin{align*}
    \empdistribution
    = \frac{
        \contractionof{\bencodingof{\extractionrelation}}{\shortcatvariables}
    }{
        \contraction{\bencodingof{\extractionrelation}}
    }  \, .
\end{align*}
Here the number of extracted data is the denominator
\begin{align*}
    \datanum
    = \contraction{\bencodingof{\extractionrelation}}
    = \contraction{\bencodingofat{\impformula}{\headvariableof{\impformula},\shortindvariables},\tbasisat{\headvariableof{\impformula}}}\, .
\end{align*}

We depict this by
\begin{center}
    
\begin{tikzpicture}[scale=0.35, yscale=1, thick] % , baseline = -3.5pt


    \draw[->] (2,-1) -- (2,1) node[midway,left] {\tiny $\catvariableof{0}$};
    \node[anchor=center] (text) at (4,0) {$\cdots$};
    \draw[->] (6,-1) -- (6,1) node[midway,right] {\tiny $\catvariableof{\atomorder-1}$};

    \draw (1,-1) rectangle (7,-3);
    \node[anchor=center] (text) at (4,-2) {$\empdistribution$};
    \node[anchor=center] (text) at (-1,-2) {$\datanum \,\, \cdot $};

    \node[anchor=center] (text) at (10,-2) {${=}$};

    \begin{scope}
        [shift={(12,0)}]

        \draw[->] (2.5,1) -- (2.5,3) node[midway,right] {\tiny $\catvariableof{0}$};
        \draw (1,-1) rectangle (4,1);
        \node[anchor=center] (text) at (2.5,0) {$\rencodingof{\groundingof{\extformulaof{0}}}$};
        \node[anchor=center] (text) at (2.5,-2) {$\cdots$};

        \node[anchor=center] (text) at (6.5,0) {$\cdots$};

        \draw[->] (10.5,1) -- (10.5,3) node[midway,right] {\tiny $\catvariableof{\atomorder-1}$};
        \draw (8.75,-1) rectangle (12.25,1);
        \node[anchor=center] (text) at (10.5,0) {$\rencodingof{\groundingof{\extformulaof{\atomorder\shortminus1}}}$};
        \node[anchor=center] (text) at (10.5,-2) {$\cdots$};

        \draw[<-] (13,-3) -- (3.5,-3) ;
        \draw[<-] (13,-5) -- (1.5,-5) ;

        \drawvariabledot{11.5}{-3}
        \draw[->] (11.5,-3) -- (11.5,-1);

        \drawvariabledot{9.5}{-5}
        \draw[->] (9.5,-5) -- (9.5,-1);

        \drawvariabledot{3.5}{-3}
        \draw[->] (3.5,-3) -- (3.5,-1);

        \drawvariabledot{1.5}{-5}
        \draw[->] (1.5,-5) -- (1.5,-1);

        \draw (13,-2) rectangle (17,-6);
        \node[anchor=center] (text) at (15,-4) {$\rencodingof{\groundingof{\impformula}}$};
        \node[anchor=center] (text) at (12,-3.75) {$\vdots$};
        \draw[->] (17,-4) -- (18,-4);
        \drawvariabledot{18}{-4}
        \draw[] (18,-4) -- (19,-4);
        \draw (19,-3) rectangle (21,-5);
        \node[anchor=center] (text) at (20,-4) {$\tbasis$};

    \end{scope}


\end{tikzpicture}
\end{center}

%\subsubsect{Decomposition of extracted data}

To connect with the empirical distribution introduced in \secref{sec:empDistribution} we now show how the empirical distribution extracted from the interpretations of the formulas $\impformula,\extformulas$ on a \firstOrderLogic{} world $\worldindices$ can be represented by tensor networks.

First of all, we decompose the importance formula into a basis $\cpformat$ format (see \charef{cha:sparseRepresentation}), that is a decomposition
\begin{align*}
    \groundingofat{\impformula}{\shortindvariables}
    = \contractionof{
        \{\legcoreofat{\indenumerator}{\indvariableof{\indenumerator},\datvariable} \, : \, \indenumeratorin \}
    }{\shortindvariables}
\end{align*}
such that all $\legcoreofat{\indenumerator}{\indvariableof{\indenumerator},\datvariable}$ are directed and boolean tensors.
Here an auxiliary variables $\datvariable$ taking values in $[\datanum]$ is introduced, which we call the data variable, which enumerates the non-vanishing coordinates of $\groundingof{\impformula}$.
With this decomposition, we can understand the decomposition of $\groundingofat{\impformula}{\shortindvariables}$ as a basis encoding of an term selection map $\secdatamap$ with coordinate maps defined such that
\begin{align*}
    \bencodingofat{\secdatamap_{\indenumerator}}{\indvariableof{\indenumerator},\datvariable}
    = \legcoreofat{\indenumerator}{\indvariableof{\indenumerator},\datvariable} \, .
\end{align*}
We depict this decomposition by:
\begin{center}
    \begin{tikzpicture}[scale=0.35, thick] % , baseline = -3.5pt

    \begin{scope}
        [shift={(0,2)}]
        \draw[] (0,1)--(0,-1) node[midway,left] {\tiny $\indvariableof{0}$};
        \draw[] (1.5,1)--(1.5,-1) node[midway,left] {\tiny $\indvariableof{1}$};
        \node[anchor=center] (text) at (3,0) {$\cdots$};
        \draw[] (4,1)--(4,-1) node[midway,right] {\tiny $\indvariableof{\indorder\shortminus1}$};
    \end{scope}


    \draw (-1,1) rectangle (5,-1);
    \node[anchor=center] (text) at (2,0) {\small ${\groundingof{\impformula}}$};


    \node[anchor=center] (text) at (7,0) {${=}$};


    \begin{scope}
        [shift={(10,2)}]



        \coordinate (conposseldec) at (4.5,-5.5);
        \drawvariabledot{4.5}{-5.5}

        \draw (conposseldec) -- (4.5,-7.5) node[midway, right] {\tiny ${\datvariable}$}; % Unclear, whether this is the best notation!
        \draw (3.5,-7.5) rectangle (5.5, -9.5);
        \node[anchor=center] (text) at (4.5,-8.5) {\small $\ones$};

        \draw[-<-] (0,1) -- (0,-1) node[midway,left] {\tiny $\indvariableof{0}$};
        \draw (-1,-1) rectangle (1, -3);
        \node[anchor=center] (text) at (0,-2) {\small $\secdatacoreof{0}$};
        \draw[-<-] (0,-3) to[bend right=20] (conposseldec);


        \draw[-<-] (3,1) -- (3,-1) node[midway,left] {\tiny $\indvariableof{1}$};
        \draw (2,-1) rectangle (4, -3);
        \node[anchor=center] (text) at (3,-2) {\small $\secdatacoreof{1}$};
        \draw[-<-] (3,-3) to[bend right=20]  (conposseldec);

        \node[anchor=center] (text) at (6,-2) {$\cdots$};

        \draw[-<-] (9,1) -- (9,-1) node[midway,left] {\tiny $\indvariableof{\indorder-1}$};
        \draw (7.75,-1) rectangle (10.25, -3);
        \node[anchor=center] (text) at (9,-2) {\small $\secdatacoreof{\indorder-1}$};
        \draw[-<-] (9,-3) to[bend left=20]  (conposseldec);

    \end{scope}


\end{tikzpicture}
\end{center}

Based on these construction, we now provide a tensor network decomposition of the extracted empirical distribution.

\begin{theorem}
    \label{the:extractionrelationDecomposition}
    Given a \firstOrderLogic{} world $\worldindices$, an importance formula $\impformula$ and extraction formulas $\extformulaof{\catenumerator}$ for $\catenumeratorin$, we have
    \begin{align*}
        \bencodingofat{\extractionrelation}{\shortindvariables,\shortcatvariables} =
        \contractionof{
            \{\bencodingofat{\groundingof{\extformulaof{\atomenumerator}}}{\catvariableof{\catenumerator},\shortindvariables} \, : \, \catenumeratorin\}
            \cup \{\bencodingofat{\secdatamap_{\indenumerator}}{\indvariableof{\indenumerator},\datvariable} \, : \, \indenumeratorin\}
        }{\shortindvariables,\shortcatvariables}
    \end{align*}
    and thus
    \begin{align*}
        \empdistributionat{\shortcatvariables} =
        \frac{1}{\datanum}  \contractionof{
            \{\bencodingofat{\groundingof{\extformulaof{\atomenumerator}}}{\catvariableof{\catenumerator},\shortindvariables} \, : \, \catenumeratorin\}
            \cup \{\bencodingofat{\secdatamap_{\indenumerator}}{\indvariableof{\indenumerator},\datvariable} \, : \, \indenumeratorin\}
        }{\shortcatvariables} \, .
    \end{align*}
\end{theorem}
\begin{proof}
    To show the first claim, let us choose arbitrary state tuples $\shortindindices$ and $\shortcatindices$.
    We then have
    \begin{align*}
        &\contractionof{
            \{\bencodingofat{\groundingof{\extformulaof{\atomenumerator}}}{\catvariableof{\catenumerator},\shortindvariables} \, : \, \catenumeratorin\}
            \cup \{\bencodingofat{\secdatamap_{\indenumerator}}{\indvariableof{\indenumerator},\datvariable} \, : \, \indenumeratorin\}
        }{\indexedshortindvariables,\indexedshortcatvariables} \\
        & \quad  =  \contraction{
            \{\bencodingofat{\groundingof{\extformulaof{\atomenumerator}}}{\indexedcatvariableof{\catenumerator},\indexedshortindvariables} \, : \, \catenumeratorin\}
            \cup \{\bencodingofat{\secdatamap_{\indenumerator}}{\indexedindvariableof{\indenumerator},\datvariable} \, : \, \indenumeratorin\}
        } \, .
    \end{align*}
    This contraction evaluates to $1$, if and only if for all $\catenumeratorin$ we have $\bencodingofat{\groundingof{\extformulaof{\atomenumerator}}}{\catvariableof{\catenumerator},\shortindvariables}=1$ and
    \begin{align*}
        \contraction{\{\bencodingofat{\secdatamap_{\indenumerator}}{\indexedindvariableof{\indenumerator},\datvariable} \, : \, \indenumeratorin\}}  = 1 \, .
    \end{align*}
    The first condition is equal to $\catindexof{\atomenumerator} = \extformulaofat{\atomenumerator}{\indexedshortindvariables}$ for all $\catenumeratorin$ and the second to
    \begin{align*}
        \groundingofat{\impformula}{\indexedshortindvariables} = 1 \, .
    \end{align*}
    Comparing with the definition of the extraction relation (see \defref{def:extractionRelation}), we notice that these conditions are equal to $(\shortindindices,\shortcatindices)\in\extractionrelation$ and therefore to
    \begin{align*}
        \bencodingofat{\extractionrelation}{\indexedshortindvariables,\indexedshortcatvariables} \, .
    \end{align*}
    The first claim follows, since $\bencodingof{\extractionrelation}$ is boolean, as is the contraction of the cores $\bencodingof{\groundingof{\extformulaof{\atomenumerator}}}$ with the cores $\bencodingof{\secdatamap_{\indenumerator}}$, which leaves the outgoing variables $\shortcatvariables$ open.
    The second claim follows from the first using that $\empdistributionat{\shortcatvariables}=\frac{1}{\datanum}\contractionof{\bencodingof{\extractionrelation}}{\shortcatvariables}$.
\end{proof}

To connect with the representation of empirical distributions based on data cores (see \secref{sec:empDistribution}), we now form data cores by contractions with the grounding of extraction formulas with the cores $\bencodingof{\secdatamap_{\indenumerator}}$ (see \figref{fig:datacoreGeneration}),
\begin{align*}
    \datacoreofat{\atomenumerator}{\catvariableof{\catenumerator},\datvariable}
    = \contractionof{
        \{\bencodingofat{\groundingof{\extformulaof{\atomenumerator}}}{\catvariableof{\catenumerator},\shortindvariables}\}
        \cup \{ \legcoreofat{\indenumerator}{\indvariableof{\indenumerator},\datvariable} \, : \, \indenumeratorin\}
    }{\catvariableof{\atomenumerator},\datvariable} \, .
\end{align*}

% Empirical distribution
The empirical distribution is then a tensor network of these tensors, as we show next.

\begin{theorem}
    \label{the:extractionDataCores}
    We have
    \begin{align*}
        \contractionof{\bencodingof{\extractionrelation}}{\shortcatvariables}
        = \contractionof{\{\datacoreofat{\atomenumerator}{\datvariable,\catvariableof{\atomenumerator}} \, : \, \atomenumeratorin\}}{\shortcatvariables}
    \end{align*}
    and thus
    \begin{align*}
        \empdistributionat{\shortcatvariables}
        = \frac{1}{\datanum} \contractionof{\{\datacoreofat{\atomenumerator}{\datvariable,\catvariableof{\atomenumerator}}  \, : \, \atomenumeratorin\}}{\shortcatvariables} \, .
    \end{align*}
\end{theorem}
\begin{proof}
    By \theref{the:extractionrelationDecomposition} we have
    \begin{align*}
        \bencodingofat{\extractionrelation}{\shortindvariables,\shortcatvariables} =
        \contractionof{
            \{\bencodingofat{\groundingof{\extformulaof{\atomenumerator}}}{\catvariableof{\catenumerator},\shortindvariables} \, : \, \catenumeratorin\}
            \cup \{\bencodingofat{\secdatamap_{\indenumerator}}{\indvariableof{\indenumerator},\datvariable} \, : \, \indenumeratorin\}
        }{\shortindvariables,\shortcatvariables} \, .
    \end{align*}
    Since $\bencodingofat{\secdatamap_{\indenumerator}}{\indvariableof{\indenumerator},\datvariable}$ are directed and boolean, they can be copied and separately contracted with each $\groundingof{\extformulaof{\atomenumerator}}$, without changing the contraction.
    We arrive at
    \begin{align*}
        &\bencodingofat{\extractionrelation}{\shortindvariables,\shortcatvariables} \\
        &\quad = \contractionof{
            \big\{\contractionof{
                \{\bencodingofat{\groundingof{\extformulaof{\atomenumerator}}}{\catvariableof{\catenumerator},\shortindvariables}\}
                \cup \{\bencodingofat{\secdatamap_{\indenumerator}}{\indvariableof{\indenumerator},\datvariable} \, : \, \indenumeratorin\}
            }{\catvariableof{\catenumerator},\datvariable} \, : \, \catenumeratorin \big\}
        }{\shortindvariables,\shortcatvariables} \\
        & \quad =  \contractionof{\{\datacoreofat{\atomenumerator}{\datvariable,\catvariableof{\atomenumerator}}  \, : \, \atomenumeratorin\}}{\shortcatvariables} \, ,
    \end{align*}
    which established the claim.
\end{proof}

% Efficient contraction: Do also basis decomposition of the extraction query and use efficient contraction!
%Towards efficient calculation of the data cores, we build a basis CP decomposition of $\groundingof{\impformula}$, where we further demand $\scalarcore=\ones$.
%This is a collection of basis leg cores $\legcoreof{\fixedimpformula,\indenumerator}$ such that
%\begin{align*}
%    \fixedimpformula[\shortindvariablelist]
%    = \contractionof{ \left\{ \legcoreofat{\fixedimpformula,\indenumerator}{\datvariable,\indvariableof{\indenumerator}} \, : \, \indenumeratorin \right\} }{\shortindvariablelist} \, .
%\end{align*}

% Data enumeration -> To representation
%We can further utilize any decomposition of $\impformula$ into a directed and binary CP Format to enumerate the datapoints by the slice index $\datindex$. % Approaches like SPARQL directly give us these by solution mappings.
%Understanding $\impformula$ as a query on the world being the database, such decomposition is given by the set of solution mappings.


\begin{figure}[t]
    \begin{center}
        \begin{tikzpicture}[scale=0.35, yscale=1, thick] % , baseline = -3.5pt


    \draw[->-] (4,-1) -- (4,1) node[midway, right] {\tiny $\catvariableof{\atomenumerator}$};
    \draw (3,-1) rectangle (5,-3);
    \node[anchor=center] (text) at (4,-2) {$\datacoreof{\atomenumerator}$};
    \draw[-<-] (4,-3) -- (4,-5) node[midway, right] {\tiny $\datvariable$};

    \node[anchor=center] (text) at (7,-2) {${=}$};

    \begin{scope}
        [shift={(10,0)}]

        \draw[->-] (3,1) -- (3,3) node[midway, right] {\tiny $\catvariableof{\atomenumerator}$};
        \draw (-1,1) rectangle (7,-1);
        \node[anchor=center] (text) at (3,0) {$\rencodingof{\groundingof{\extformulaof{\atomenumerator}}}$};

        \draw[->-] (0,-3) -- (0,-1) node[midway,left] {\tiny $\indvariableof{0}$};
        \draw[->-] (3,-3) -- (3,-1) node[midway,left] {\tiny $\indvariableof{1}$};
        \draw[->-] (6,-3) -- (6,-1) node[midway,left] {\tiny $\indvariableof{2}$};


    \end{scope}

    \begin{scope}
        [shift={(10,-2)}]

        \coordinate (conposseldec) at (4.5,-5.5);
        \drawvariabledot{4.5}{-5.5}
        \draw[-<-] (conposseldec) -- (4.5,-7.5) node[midway, right] {\tiny $\indexvariable$};

        \draw (-1,-1) rectangle (1, -3);
        \node[anchor=center] (text) at (0,-2) {\small $\rencodingof{\secdatamap_0}$};%{\small $\legcoreof{\fixedimpformula,0}$};
        \draw[-<-] (0,-3) to[bend right=20] (conposseldec);

        \draw (2,-1) rectangle (4, -3);
        \node[anchor=center] (text) at (3,-2) {\small $\rencodingof{\secdatamap_1}$};%{\small $\legcoreof{\fixedimpformula,1}$};
        \draw[-<-] (3,-3) to[bend right=20]  (conposseldec);

        \draw (5,-1) rectangle (7, -3);
        \node[anchor=center] (text) at (6,-2) {\small $\rencodingof{\secdatamap_2}$};%{\small $\legcoreof{\fixedimpformula,2}$};
        \draw[-<-] (6,-3) to[bend right=-20]  (conposseldec);

        \draw[<-] (9,1) -- (9,-1) node[midway,left] {\tiny $\indvariableof{3}$};
        \draw (8,-1) rectangle (10, -3);
        \node[anchor=center] (text) at (9,-2) {\small $\rencodingof{\secdatamap_3}$};%{\small $\legcoreof{\fixedimpformula,3}$};
        \draw[<-] (9,-3) to[bend right=-20]  (conposseldec);


        \node[anchor=center] (text) at (12,-2) {$\cdots$};

        \draw[<-] (15,1) -- (15,-1) node[midway,left] {\tiny $\indvariableof{\indorder-1}$};
        \draw (13.5,-1) rectangle (16.5, -3);
        \node[anchor=center] (text) at (15,-2) {\small $\rencodingof{\secdatamap_{\indorder-1}}$};%{\small $\legcoreof{\fixedimpformula,\variableorder-1}$};
        \draw[<-] (15,-3) to[bend left=20]  (conposseldec);


        \draw (8,1) rectangle (16, 3);
        \node[anchor=center] (text) at (12,2) {\small $\ones$};


    \end{scope}


    \node[anchor=center] (text) at (29,-2) {${=}$};


    \begin{scope}
        [shift={(32,0)}]

        \draw[->-] (3,1) -- (3,3) node[midway, right] {\tiny $\catvariableof{\atomenumerator}$};
        \draw (-1,1) rectangle (7,-1);
        \node[anchor=center] (text) at (3,0) {$\rencodingof{\groundingof{\extformulaof{\atomenumerator}}}$};

        \draw[->-] (0,-3) -- (0,-1) node[midway,left] {\tiny $\indvariableof{0}$};
        \draw[->-] (3,-3) -- (3,-1) node[midway,left] {\tiny $\indvariableof{1}$};
        \draw[->-] (6,-3) -- (6,-1) node[midway,left] {\tiny $\indvariableof{2}$};


    \end{scope}

    \begin{scope}
        [shift={(32,-2)}]


        \coordinate (conposseldec) at (3,-5.5);
        \drawvariabledot{3}{-5.5}
        \draw[<-] (conposseldec) -- (3,-7.5) node[midway, right] {\tiny $\datvariable$};

        \draw (-1,-1) rectangle (1, -3);
        \node[anchor=center] (text) at (0,-2){\small $\rencodingof{\secdatamap_0}$};%{\small $\legcoreof{\fixedimpformula,0}$};
        \draw[<-] (0,-3) to[bend right=20] (conposseldec);

        \draw (2,-1) rectangle (4, -3);
        \node[anchor=center] (text) at (3,-2) {\small $\rencodingof{\secdatamap_1}$};%{\small $\legcoreof{\fixedimpformula,1}$};
        \draw[<-] (3,-3) to[bend right=0]  (conposseldec);

        \draw (5,-1) rectangle (7, -3);
        \node[anchor=center] (text) at (6,-2) {\small $\rencodingof{\secdatamap_2}$};%{\small $\legcoreof{\fixedimpformula,2}$};
        \draw[<-] (6,-3) to[bend right=-20]  (conposseldec);


    \end{scope}


\end{tikzpicture}
    \end{center}
    \caption{Generation of a data core for the variable $\catvariableof{\catenumerator}$ given an extraction formula $\extformulaof{\catenumerator}$ and an importance formula, which grounding is decomposed into a basis CP format with leg vectors $\bencodingofat{\secdatamap_{\indenumerator}}{\indvariableof{\indenumerator},\datvariable}$.
    Term variables, which are appearing in the importance formula, but not in the extraction formula $\extformulaof{\catenumerator}$ can be treated trivally by contraction with the trivial tensor (here $\indvariableof{4},\ldots,\indvariableof{\indorder-1})$.
    }
    \label{fig:datacoreGeneration}
\end{figure}


% Comment: Exploitation of common structure
When many atom extraction formulas differ only by a constant, we can replace the constant by an auxiliary term variable.
The atoms are then the atomizations of this variable (see \secref{sec:categoricalTN}), treated as a categorical variable, with respect to the constant in the extraction query.
The advantages are that we can avoid the $\bencodingof{}$-formalism and directly model the categorical distributions.

This also enables a batchwise computation of multiple $\sparql$ queries, which differ only in one constant.


%\subsect{Design of the Formulas}
%
%Most intuitive when labeling individuals by classes.
%Extraction formulas $\extformulas$ can then be defined by subclasses of the member of a class and relations between objects of different classes. % Koller calls atomic formulas the template attributes
%We then choose $\formulaset$ as more involved formulas decomposed into connectives acting on these atoms.
%The importance formula $\impformula$ is then designed based on class memberships to ensure, that the arguments of the formulas are always of specific classes. % Koller specifies to each argument of the attributes a class
%
%% Approach
%We propose to
%\begin{itemize}
%    \item Execute an extraction query to get pairs of individuals (the pairDf).
%    \item Propositionalize the FOL Formulas independently on each tuple taking the individuals as a set of constant and filtering on the possible properties of each individuals.
%    (Can understand as adding knowledge that most of the relations do not hold)
%    \item Understand each such generated knowledge base as datapoint and average over them to get the empirical distribution to be fit.
%    \item Fit a MLN describing the statistical relations of unseen results of the extraction query, based on likelihood maximation.
%\end{itemize}




\sect{Generation of \firstOrderLogic{} worlds}

\red{
    So far we have discussed, how Probabilistic Relational Models for \firstOrderLogic{} Knowledge Bases such as Knowledge Graphs can be built by extracting data.
    Conversely, any binary tensor can be interpreted as a Knowledge Graph.
    To be more precise, we follow the intuition that the ones coordinates mark possible worlds compatible with the knowledge about a factored system.
    Each possible world can then be encoded in a subgraph of the Knowledge Graph representing the world.
%
    This amounts to an "inversion" of the data generation process described in the subsection above.
}

In the previous section we have described a way to extract an effective empirical distribution for the likelihood of a \firstOrderLogic{} world given a \HybridFOLNetwork{}.
We now want to investigate methods to reproduce an empirical distribution based on a constructed \firstOrderLogic{} world.

\begin{definition}[Reproduction of Empirical Distributions]
    Given an empirical distribution $\empdistribution\in\atomspace$, we say that a triple $(\worldindices,\impformula,\shortextformulas)$ of a \firstOrderLogic{} world $\worldindices$ an importance formula $\impformula$ and extraction formulas $\shortextformulas=\{\extformulaof{\atomenumerator}\,:\,\atomenumeratorin\}$ reproduces $\empdistribution$, when
    \begin{align*}
        \empdistribution
        = \normalizationof{\{\groundingofat{\impformula}{\shortindvariables}\}\cup
        \{\bencodingofat{\kggroundingof{\extformulaof{\atomenumerator}}}{\catvariableof{\catenumerator},\shortindvariables} \wcols \atomenumeratorin\}
        }{\shortcatvariables} \, .
    \end{align*}
\end{definition}

% If \datamap is not known
Note that for distribution $\probtensor$ to be reproducible, it needs to have rational coordinates. %, since each coordinate can be interpreted as the frequency of the respective world in the data $\datamap$.
If any only if all coordinates are rational, we find a $\datanum\in\nn$ such that $\imageof{\datanum\cdot\probtensor}\subset\nn$.
We can then interpret $\datanum$ as the number of samples, and construct a sample selector map by understanding each coordinate of $\datanum\cdot\probtensor$ as the number of appearances of the respective world in the samples.

We show different schemes and give examples on Knowledge Graphs, where we provide examples for importance and extraction formulas by $\sparql$ queries.


%\subsect{Example: Generation of Knowledge Graphs} % To generation of \firstOrderLogic{} worlds?
%
% Having a directed and binary CP decomposition of $\exformula$, each possible world is encoded by a slice.


% Formalization
%\begin{definition}[Reproduction of Empirical Distributions]
%    Given an empirical distribution $\empdistribution\in\bigotimes_{\atomenumeratorin}\rr^2$, we say that a tuple $(\kg,\impformula,\{\extformulas\})$ of a Knowledge Graph $\kg$ and queries $\impformula,\extformulaof{\atomenumerator}$ reproduces $\empdistribution$, when
%    \[\empdistribution = \normalizationof{\{\kggroundingof{\impformula}\}\cup\{\bencodingof{\kggroundingof{\extformulaof{\atomenumerator}}\, : \, \atomenumeratorin}\}}{\shortcatvariables} \, .  \]
%\end{definition}

%

%In a frequentist interpretation we instantiate each world according to the rate $\probtensor(\atomindices)$.
%This interpretation requires a rounding of the real probabilities by rational numbers.


\subsect{Samples by single objects}

%\subsect{Samples by single objects}

In the first reproduction scheme we construct datapoints by dedicated objects, which represent a sample, that is we choose a domain $\worlddomain=[\datdim]$.

\begin{theorem}
    \label{the:reproducingSingleObjects}
    Let there be an empirical distribution $\empdistribution$ to a sample selector map $\datamap$ (see \defref{def:dataMap}), we construct a world $\worldindices[\selvariable,\indvariable]$ with $\atomorder$ unary predicates by
    \begin{align*}
        \worldindices[{\selvariable,\indvariable}]
        = \sum_{\atomenumeratorin} \sum_{\datindexin \wcols \datamap_{\atomenumerator}(\datindex)=1} \onehotmapofat{\atomenumerator}{\selvariable} \otimes \onehotmapofat{\datindex}{\indvariable} \, .
    \end{align*}
    We further choose a trivial importance query, that is
    \begin{align*}
        \groundingofat{\impformula}{\indvariable} = \onesat{\indvariable} \, ,
    \end{align*}
    and extraction queries coinciding with the unary predicates, that is for $\atomenumeratorin$
    \begin{align*}
        \extformulaof{\atomenumerator} = \folpredicateof{\atomenumerator} \, .
    \end{align*}
    Then, the triple $(\worldindices,\impformula,\shortextformulas)$ reproduces $\empdistribution$.
%    reproduces with the trivial importance query and extraction queries coinciding with the predicates the dataset $\datamap$.
\end{theorem}
\begin{proof}
    By \theref{the:extractionDataCores} it is enough to show, that the data cores constructed from the data extraction process coincide with those of $\empdistribution$.
    We enumerate to this end the non-vanishing coordinates of $\groundingof{\impformula}$ by the data variable $\datvariable$ taking values $\datindexin$, as
    \begin{align*}
        \groundingofat{\impformula}{\indvariable=\datindex} = 1 \,
    \end{align*}
    and choose
    \begin{align*}
        \secdatamap = \identity \, .
    \end{align*}
    For arbitrary $\atomenumeratorin$ and $\datindexin$ we now have
    \begin{align*}
        \datacoreofat{\atomenumerator}{\catvariableof{\catenumerator},\indexeddatvariable}
        &= \contractionof{
            \bencodingofat{\groundingof{\extformulaof{\atomenumerator}}}{\catvariableof{\catenumerator},\indvariable},
            \legcoreofat{0}{\indvariable,\indexeddatvariable}
        }{\catvariableof{\atomenumerator},\datvariable} \\
        &= \contractionof{
            \bencodingofat{\groundingof{\extformulaof{\atomenumerator}}}{\catvariableof{\catenumerator},\indvariable},
            \onehotmapofat{\secdatamap(\datindex)}{\indvariable}
        }{\catvariableof{\atomenumerator},\indexeddatvariable} \\
        &= \onehotmapofat{\datamap_\atomenumerator(\datindex)}{\catvariableof{\catenumerator}} \, .
    \end{align*}
    This coincides with the slice of the data core of the CP representation of empirical distributions used in \theref{the:empCPRep}.
    Since the slice and the core was arbitrary, the tensor network representations in \theref{the:empCPRep} and \theref{the:extractionDataCores} are equal and thus the triple $(\worldindices,\impformula,\shortextformulas)$ reproduces $\empdistribution$.
\end{proof}


We now give by the next theorem an example of a Knowledge Graph with $\sparql$ queries reproducing and arbitrary empirical distribution.

\begin{theorem}
    \label{the:reproducingKGSingelObjects}
    Let $\empdistribution$ be an empirical distribution to the sample selector $\datamap$.
    We construct a Knowledge Graph of the resources $\worlddomain = \{s_\datindex \, : \, \datindexin\} \cup \{C\} \cup \{C_\atomenumerator \, : \, \atomenumeratorin\}$, where $s_{\datindex}$ represent samples and $C_\atomenumerator$ unary predicates, by
    \begin{align*}
        \kggroundingof{\rdf}
        =
        \sum_{\datindexin}
        \onehotmapof{\indexinterpretationof{s_\datindex}}{\sindvariable}
        \otimes \onehotmapof{\indexinterpretationof{\mathrdftype}}{\pindvariable}
        \otimes \onehotmapof{\indexinterpretationof{C}}{\oindvariable}
        +
        \sum_{\datindexin} \sum_{\atomenumeratorin \, : \, \datamap_{\atomenumerator}(\datindex)=1}
        \onehotmapof{\indexinterpretationof{s_\datindex}}{\sindvariable}
        \otimes \onehotmapof{\indexinterpretationof{\mathrdftype}}{\pindvariable}
        \otimes \onehotmapof{\indexinterpretationof{C_\atomenumerator}}{\oindvariable} \, .
    \end{align*}
    We further define an importance formula by the $\sparql$ query
    \begin{centeredscript}
        \impformula = SELECT \{ ?x \} WHERE \{ ?x \quad \rdftype\quad C \, .\}
    \end{centeredscript}
    and for each $\atomenumeratorin$ an extraction formula by the query
    \begin{centeredscript}
        $\extformulaof{\atomenumerator}$ = SELECT \{ ?x \} WHERE \{ ?x \quad \rdftype \quad $C_\atomenumerator$ \, .\} \, .
    \end{centeredscript}
    Then the triple $(\kg,\impformula,\shortextformulas)$ reproduces $\empdistribution$.
\end{theorem}
\begin{proof}
    We show the theorem analogously to \theref{the:reproducingSingleObjects}, with the slide difference in the importance formula.
    We have for the grounding of $\impformula$ on $\kg$ that
    \begin{align*}
        \kggroundingofat{\impformula}{\indvariable} = \sum_{\datindexin}  \onehotmapof{\indexinterpretationof{s_\datindex}}{\indvariable}
    \end{align*}
    and enumerate the non-vanishing coordinates by $\datvariable$.

    For each extraction formula we have
    \begin{align*}
        \kggroundingofat{\extformulaof{\atomenumerator}}{\indvariable} = \sum_{\datindexin \, : \, \datamap_{\atomenumerator}(\datindex)=1} \onehotmapof{\indexinterpretationof{s_\datindex}}{\indvariable} \,.
    \end{align*}
    It follows that the data cores used in \theref{the:extractionDataCores} are
    \begin{align*}
        \bencodingofat{\datamap_\atomenumerator}{\catvariableof{\atomenumerator},\datindex}
        = \onehotmapofat{0}{\catvariableof{\atomenumerator}} \otimes \left(\sum_{\datindexin \, : \, \datamap_{\atomenumerator}(\datindex)=0} \onehotmapofat{\datindex}{\datvariable}\right)
        +\tbasisat{\catvariableof{\atomenumerator}} \otimes \left(\sum_{\datindexin \, : \, \datamap_{\atomenumerator}(\datindex)=1} \onehotmapofat{\datindex}{\datvariable}\right)
    \end{align*}
    and they thus coincide with those in the decomposition in \theref{the:empCPRep}.
    The claim follows therefore with the same argumentation as in the proof of \theref{the:reproducingSingleObjects}.
\end{proof}

%
Let us provide some more insights on the construction of the reproducing Knowledge Graph in \theref{the:reproducingKGSingelObjects}.
By the insertions to the one-hot encodings $\onehotmapof{\indexinterpretationof{s_\datindex}}{\sindvariable} \otimes \onehotmapof{\indexinterpretationof{\mathrdftype}}{\pindvariable} \otimes \onehotmapof{\indexinterpretationof{C}}{\oindvariable}$ we mark each sample representing resource by a class and ensure its appearance as a $\mathrm{owl:NamedIndividual}$ in the graph.
The insertions $\onehotmapof{\indexinterpretationof{s_\datindex}}{\sindvariable}\otimes \onehotmapof{\indexinterpretationof{\mathrdftype}}{\pindvariable} \otimes \onehotmapof{\indexinterpretationof{C_\atomenumerator}}{\oindvariable}$ on the other side encode the sample selecting map, by inserting exactly the assertions corresponding with the respective sample.
% 
In this simple Knowledge Graph, Description Logic is expressive enough to represent any formula $\folexformula$ composed of the formulas $\extformulas$.

%
%\begin{theorem}
%    Let there any empirical distribution $\empdistribution\in\bigotimes_{\atomenumeratorin}\rr^2$ and $\datanum\in\nn$ such that $\imageof{\datanum\cdot\empdistribution}\subset\nn$.
%    Then the tuple $(\kg,\impformula,\{\extformulas\})$ defined by a Knowledge Graph
%    \begin{align}
%        \kg =
%        & \bigcup_{\atomindicesin}  \{(
%        s_{j, \atomindices} \quad \mathrm{rdf:type} \quad C ) : j \in [\datanum\cdot\empdistribution(\atomindices)] \}  \\
%        &\bigcup_{\atomindicesin}  \{(
%        s_{j, \atomindices} \quad \mathrm{rdf:type} \quad C_\atomenumerator
%        ) : j \in [\datanum\cdot\empdistribution(\atomindices)], \atomenumeratorin , \atomlegindexof{\atomenumerator}=1\}
%    \end{align}
%    further an importance formula by the query
%    \begin{centeredscript}
%        \impformula = SELECT \{ ?x \} WHERE \{ ?x \quad \rdftype\quad C \, .\}
%    \end{centeredscript}
%    and extraction formulas for each $\atomenumeratorin$ by the query
%    \begin{centeredscript}
%        $\extformulaof{\atomenumerator}$ = SELECT \{ ?x \} WHERE \{ ?x \quad \rdftype \quad $C_\atomenumerator$ \, .\}
%    \end{centeredscript}
%    reproduces $\empdistribution$.
%\end{theorem}
%\begin{proof}
%    With respect to any enumeration of the resources of $\kg$ we have
%    \begin{align}
%        \kggroundingof{\impformula}
%        = \sum_{\atomindicesin} \sum_{j \in [\datanum\cdot\empdistribution(\atomindices)]} \onehotmapof{s_{j, \atomindices} }
%    \end{align}
%    and
%    \begin{align}
%        \kggroundingof{\extformulaof{\atomenumerator}}
%        = \sum_{\atomindicesin \, : \, \atomlegindexof{\atomenumerator} = 1} \sum_{j \in [\datanum\cdot\empdistribution(\atomindices)]} \onehotmapof{s_{j, \atomindices} } \, .
%    \end{align}
%    Summing over the resource variables of these tensors in a contraction we get
%    \begin{align}
%        \contractionof{\{\kggroundingof{\impformula}\}\cup\{\bencodingof{\kggroundingof{\extformulaof{\atomenumerator}}\, : \, \atomenumeratorin}\}}{\shortcatvariables}
%        & = \sum_{\atomenumeratorin}  \datanum\cdot\empdistribution(\atomindices) \cdot \onehotmapof{\atomindices} = \datanum \cdot \empdistribution
%    \end{align}
%    and therefore
%    \begin{align}
%        \normalizationof{\{\kggroundingof{\impformula}\}\cup\{\bencodingof{\kggroundingof{\extformulaof{\atomenumerator}}}\, : \, \atomenumeratorin\}}{\shortcatvariables} = \empdistribution \, .
%    \end{align}
%\end{proof}







\subsect{Samples by pairs of objects}

%\paragraph{TBox:} The categorical variables of the factored system are the classes.
%We define atomic formulas by the state indicators of each categorical variable as in \secref{sec:categoricalTN}.
%Each such atomic formula corresponds with a sub-class of the classes.
%By definition, each collection of state indicators define thus pairwise disjoint subclasses.
%
%\paragraph{ABox:} The samples are represented by single individuals in the Knowledge Graph.
%Their sub-class memberships corresponding with the categorical variables of the system are instantiated whenever the atom is true in the sample.
%%\subsubsect{Samples by pairs of resources}
%
%\begin{remark}[Refinement of the Samples]
%    We can split each sample node into a pair of individuals.
%    For this we need to specify, which each class membership will be encoded in a unary or binary attribute of the splitted individuals.
%    This specification is possible based on the extraction query and the atomic formulas.
%\end{remark}
%
%%
%Taking any importance query $\impformula$, which has no permutation symmetries, we can instantiate each projection variable for each sample and prepare the links according to the triple patterns.
%When the atom queries $\extformulas$ have different triple patterns compared with $\impformula$, we instantiate those in cases where $\atomlegindexof{\atomenumerator}=1$.


%
We now instantiate multiple objects for each datapoint, one for each variable of the importance formula, i.e. $\worlddomain=[\datdim]\times[\indorder]$
Label individuals $s_{\datindex,\indenumerator}$ by data index and variable index.

\begin{lemma}
    Let there a data map $\datamap$, queries $\impformula,\shortextformulas$ and a \firstOrderLogic{} world containing objects $s_{\datindex,\indenumerator}$ for $\datindexin$ and $\indenumeratorin$
    If
    \begin{align*}
        \kggroundingof{\impformula}
        = \sum_{\datindexin} \bigotimes_{\indenumeratorin} \onehotmapofat{\indexinterpretationof{s_{\datindex,\indenumerator}}}{\indvariableof{\indenumerator}}
    \end{align*}
    and for any $\atomenumeratorin$
    \begin{align*}
        \kggroundingof{\extformulaof{\atomenumerator}}
        = \sum_{\datindex : \datamap_{\atomenumerator}(\datindex)=1} \bigotimes_{\indvariableof{\indenumerator} \in \indvariableof{\extformulaof{\atomenumerator}}}
        \onehotmapofat{\indexinterpretationof{s_{\datindex,\indenumerator}}}{\indvariableof{\indenumerator}} \, .
    \end{align*}
%    \[ \kggroundingof{\extformulaof{\atomenumerator}}
%    = \sum_{\datindex : \datamap^{\atomenumerator}(\datindex)=1} \bigotimes_{\indenumerator \in \extformulaof{\atomenumerator}} \onehotmapof{\datindex,\indenumerator} \, . \]
    Then the tuple $(\kg,\impformula,\{\extformulas\})$ reproduces $\empdistribution$.
\end{lemma}
\begin{proof}
    We notice, that the grounding of the importance formula is in a basis CP format, since by assumption
    \begin{align*}
        \kggroundingof{\impformula}
        = \sum_{\datindexin} \bigotimes_{\indenumeratorin} \onehotmapofat{\indexinterpretationof{s_{\datindex,\indenumerator}}}{\indvariableof{\indenumerator}} \, .
    \end{align*}
    We choose $\datvariable$ to enumerate the non-vanishing entries and get a term selecting map
    \begin{align*}
        \secdatamap_{\indenumerator}(\datindex) = \indexinterpretationof{s_{\datindex,\indenumerator}} \, .
    \end{align*}
    From this we have
    \begin{align*}
        \contractionof{
            \{\bencodingofat{\kggroundingof{\extformulaof{\atomenumerator}}}{\catvariableof{\atomenumerator},\indvariableof{\extformulaof{\atomenumerator}}}\} \cup
            \{\bencodingofat{\secdatamap_{\indenumerator}}{\indvariableof{\indenumerator},\datvariable} \, : \, \indenumeratorin\}
        }{\catvariableof{\catenumerator},\datvariable}
        = \bencodingofat{\datamap_{\atomenumerator}}{\catvariableof{\catenumerator},\datvariable}
    \end{align*}
    and the claim follows with the same argumentation as in the proof of \theref{the:reproducingSingleObjects}.
\end{proof}


%Let us construct a Knowledge Graph
%\begin{align*}
%        \kggroundingof{\rdf}
%        =
%        \sum_{\datindexin}\sum_{\indenumeratorin}
%        \onehotmapof{\indexinterpretationof{s_{\datindex,\indenumerator}}}{\sindvariable}
%        \otimes \onehotmapof{\indexinterpretationof{\mathrdftype}}{\pindvariable}
%        \otimes \onehotmapof{\indexinterpretationof{C}}{\oindvariable}
%        +
%        \sum_{\datindexin} \sum_{\atomenumeratorin \, : \, \datamap_{\atomenumerator}(\datindex)=1} \sum_{\indvariableof{\indenumerator}\in\indvariableof{}}
%        \onehotmapof{\indexinterpretationof{s_{\datindex,\indenumerator}}}{\sindvariable}
%        \otimes \onehotmapof{\indexinterpretationof{\mathrdftype}}{\pindvariable}
%        \otimes \onehotmapof{\indexinterpretationof{C_\atomenumerator}}{\oindvariable} \, .
%\end{align*}
%    We further define an importance formula by the $\sparql$ query
%\begin{centeredscript}
%        \impformula = SELECT \{ ?x_0 \cdots ?x_{\indorder-1} \} WHERE \{ ?x_0 \quad \rdftype\quad C \, .\}
%\end{centeredscript}
%    and for each $\atomenumeratorin$ an extraction formula by the query
%\begin{centeredscript}
%        $\extformulaof{\atomenumerator}$ = SELECT \{ ?x \} WHERE \{ ?x \quad \rdftype \quad $C_\atomenumerator$ \, .\} \, .
%\end{centeredscript}
%    Then the triple $(\kg,\impformula,\shortextformulas)$ reproduces $\empdistribution$.



\sect{Discussion}


% Probabilistic Relational Models
Statistical Models are called Probabilistic Relational Models. % (RUSSELL - Chapter Probabilistic Programming).
Extensions are models that also handle structural uncertainty, i.e. distributions of worlds with varying $\worlddomain$.

% Comparison with network science
In the emerging area of network science \cite{barabasi_network_2016, giovanni_russo_vito_latora_complex_2017}, statistical models for random graphs are investigated.
Statistical Models of \firstOrderLogic{} go beyond the typical single edge type perspective of network science.


%
\begin{remark}[Alternative Representation of empirical distributions]
    So far, we have motivated the representation of empirical distributions based on basis CP decompositions based on data maps.
    In this section, based on the extraction queries, we have observed that empirical distributions might have more efficient representation formats.
    In many applications such as the computation of log-likelihoods we can use any representation of the empirical distribution by tensor networks.
    It is thus not necessary to compute the data cores as above, unless one requires a list of the extracted samples.
\end{remark}


    \part{\partthreetext}\label{par:three}
    \chapter{Introduction to \parref{par:three}}

\begin{highlight}
	In view of all that..., the many obstacles we appear to have surmounted, what casts the pall over our victory celebration?
	It is the curse of dimensionality, a malediction that has plagued the scientists from earliest days. - \text{Richard Bellman \cite{bellman_adaptive_1961}}
\end{highlight}

In \parref{par:three} we provide a more general discussion of tensor calculus, which has been applied in \parref{par:one} and \parref{par:two}.
Since the contraction operation is central in these calculus schemes, we refer to the set of techniques as contraction calculus.

\sect{Encoding Schemes for Functions}

\textbf{One-hot encodings} are the central encoding schemes of states in factored system representations.
As will be presented in more detail in \charef{cha:coordinateCalculus}, they build a basis of the tensor space.
We use one-hot encodings in the construction of further encoding schemes of functions on the set of states in factored system representations.

\textbf{\coordinateEncodings{}} use the real coordinates multiplying each one-hot encoding to store information.
They have implicitly been used in the representation of factored systems in \parref{par:one}, where probability distributions and logical formulas have been treated as tensors.
More precise, these tensors are the coordinate encodings of probability distributions and logical formulas, which are maps of the state set into the interval $[0,1]$ and into the set $\ozset$.

\textbf{\basisEncodings{}} are sums of a collection of one-hot encodings.
They differ from coordinate encodings that they do not allow weights by general real numbers in these sums, and restrict to booleans.
Basis encodings therefore map the set of subsets to an enumerated set bijectively onto the set of boolean tensors.
We introduce them in \charef{cha:basisCalculus} in most generality as an encoding scheme for subsets of enumerated set, which we then extend towards relations and functions.
The main advantage of basis encodings is the efficient representation of function compositions by tensor networks of basis encodings.
This scheme is the main representation paradigm to derive efficient tensor network decompositions in artificial intelligence.

\textbf{Selection encodings} are specific coordinate encodings for tensor-valued functions on the state set of a factored representation.
Here selection variables are introduced to enumerate the coordinates of the target tensor space and treated as additional variables of a factored system representation.
We have exploited them in \parref{par:two} for efficient representation of function sets, where the functions rely on a common structure.

\sect{Schemes for Tensor Calculus}

We applied in \parref{par:one} and \parref{par:two} two schemes of tensor calculus, which we now investigate in \parref{par:three} in more detail.
\begin{itemize}
    \item \textbf{\CoordinateCalculus{}:} We study in \charef{cha:coordinateCalculus} schemes to exploit \coordinateEncodings{} in calculus.
    Retrieval of single coordinates is done by contractions of the tensors with one-hot encodings, with no open variables (see \theref{the:coordinateCalculus}).
    \item \textbf{\BasisCalculus{}:} In \charef{cha:basisCalculus} we investigate the properties of \basisEncodings{} to perform calculus.
    The evaluation of a function at a state is performed by contractions with one-hot encodings, with the computed variable left open (see \theref{the:basisCalculus}).
    As the main advantage of \basisEncodings{} over \coordinateEncodings{}, we can represent composed function by contractions of \basisEncodings{} to each component (see \theref{the:compositionByContraction}).
\end{itemize}

\sect{Classification of Tensors}

We frequently worked in \parref{par:one} and \parref{par:two} with tensors, which have non-negative coordinates and occasionally are boolean (see \defref{def:booleanTensor}) or directed (see \defref{def:directedTensor}).
While boolean tensors have appeared as semantical representation of formulas, directed tensors have appeared mostly as conditional distributions.
The set of tensors, which are both boolean and directed receive further interest, since they are exactly the basis encodings of functions.
We sketch this coarse classification scheme in \figref{fig:dbTensorSketch}.
\begin{figure}[h]
    \begin{center}
        \begin{tikzpicture}[yscale=0.6]
	\draw[dashed] (-10.5,12) rectangle (5.5,2);
	\node[anchor=center] (text) at (-2.5,11) {Tensors with non-negative coordinates};
	
	\draw[red] (-10,10) rectangle (2.5,5); 
	\node[anchor=center,red] (text) at (-5,9) {Directed Tensors: Conditional probability distributions};
	\draw[blue] (-7.5,7.5) rectangle (5,2.5); 
	\node[anchor=center,blue] (text) at (0,3.5) {Boolean Tensors: Encoding of subsets (see \defref{def:subsetEncoding}) and relations (see \defref{def:daryRelation})};

	\node[anchor=center] (text) at (-2.5,6.5) {Directed and Boolean Tensors: Encoding of functions (see \theref{the:rencodingDirected})};
\end{tikzpicture}
    \end{center}
    \caption{Sketch of the tensors with non-negative coordinates.
    We investigate in this chapter tensors, which are directed and boolean.}\label{fig:dbTensorSketch}
\end{figure}

\sect{Efficient Representation and Reasoning}

\textbf{Efficient storage schemes} can be derived for tensors with specific properties, whereas generic tensor storage schemes suffer from the curse of dimensions.
In \charef{cha:sparseRepresentation} we investigate the efficient representation of tensors based on $\cpformat$ decompositions.
When restricting the allowed leg tensors in different ways, we show that relational databases can be utilized as sparse storage of tensors.

\textbf{Efficient executions of contractions} will be derived based on message-passing schemes in \charef{cha:messagePassing}.
They are in most generality derived from expectation propagation schemes in variational inference.

\textbf{Approximation schemes} to tensors are investigated in \charef{cha:approximation}.
Approximating formats are oriented on the efficient representation of the approximating tensor.

%\sect{Outline}



    \chapter{\chatextcoordinateCalculus} \label{cha:coordinateCalculus}

In the previous chapters, information to states has been stored in coordinates of a tensor.
To distinguish from other schemes of calculus such as the basis calculus (see \charef{cha:basisCalculus}), we call this scheme of storing and retrieving information the coordinate calculus.
%We in this chapter investigate in more depth, which operations can be performed based on such tensors and proof the applied properties.

\sect{One-hot encodings as basis}

Let us first show, that the one-hot encodings, which we have used to motivate tensor representations, build an orthonormal basis of the respective tensor spaces.

\begin{lemma}%[Basis of tensor spaces]
    \label{lem:tensorBasisDecomposition}
    The image of the one-hot encoding map is an orthonormal basis of the tensor space $\facspace$, that is for any $\shortcatindices,\tildeshortcatindices\in\facstates$ we have
    \begin{align*}
        \contraction{\onehotmapofat{\shortcatindices}{\shortcatvariables},\onehotmapofat{\tildeshortcatindices}{\shortcatvariables}}
        = \deltaof{\shortcatindices,\tildeshortcatindices}
        \coloneqq
        \begin{cases}
            1 & \ifspace \shortcatindices=\tildeshortcatindices \\
            0 & \text{else}
        \end{cases} \, .
    \end{align*}
    Any element $\hypercore\in\facspace$ has a decomposition
    \begin{align*}
        \hypercoreat{\shortcatvariables}
        = \sum_{\shortcatindicesin} \hypercoreat{\indexedshortcatvariables} \cdot \onehotmapofat{\shortcatindices}{\shortcatvariables} \, .
    \end{align*}
    We notice that the coordinates are the weights to the basis elements in the one-hot decomposition.
\end{lemma}
\begin{proof}
    The first claim follows from an elementary decomposition of one-hot encodings and the orthogonality of basis vectors as
    \begin{align*}
        \contraction{\onehotmapofat{\shortcatindices}{\shortcatvariables},\onehotmapofat{\tildeshortcatindices}{\shortcatvariables}}
        = \prod_{\catenumeratorin} \contraction{\onehotmapofat{\catindexof{\atomenumerator}}{\catvariableof{\atomenumerator}},\onehotmapofat{\tildecatindexof{\atomenumerator}}{\catvariableof{\atomenumerator}}}
        = \prod_{\catenumeratorin} \delta_{\catindexof{\atomenumerator},\tildecatindexof{\atomenumerator}}
        = \deltaof{\shortcatindices,\tildeshortcatindices} \, .
    \end{align*}
    To show the second claim, it is enough to notice that for any $\tildeshortcatindices\in\facstates$ we have
    \begin{align*}
        \sum_{\shortcatindicesin} \hypercoreat{\indexedshortcatvariables} \cdot \onehotmapofat{\shortcatindices}{\shortcatvariables=\tildeshortcatindices}
        &= \sum_{\shortcatindicesin} \hypercoreat{\indexedshortcatvariables} \cdot \deltaof{\shortcatindices,\tildeshortcatindices} \\
        &=   \hypercoreat{\shortcatvariables=\tildeshortcatindices} \, . \qedhere
    \end{align*}
\end{proof}

Any tensor can be understood as a coordinate encoding of a real-valued function, as we define next.

\begin{definition}\label{def:coordinateEncoding}
    Given any real-valued function
    \begin{align*}
        \exfunction \defcols \facstates \rightarrow \rr
    \end{align*}
    we define the coordinate encoding by
    \begin{align*}
        \cencodingofat{\exfunction}{\shortcatvariables}
        = \sum_{\shortcatindicesin} \exfunctionat{\shortcatindices} \cdot \onehotmapofat{\shortcatindices}{\shortcatvariables} \, .
    \end{align*}
\end{definition}

In \parref{par:one} and \parref{par:two} we did not distinguish between a real-valued function $\exfunction$ and its coordinate encoding $\hypercoreof{\exfunction}$, in order to abbreviate notation.
Based on coordinate encodings, we now show, that function evaluation can be performed by contractions.

\begin{theorem}[Function evaluation in Coordinate Calculus]
    \label{the:coordinateCalculus}
    Given any real-valued function
    \begin{align*}
        \exfunction \defcols \facstates \rightarrow \rr
    \end{align*}
    and any input state $\shortcatindicesin$, we have
    \begin{align*}
        \exfunctionat{\shortcatindices}
        = \contraction{\cencodingofat{\exfunction}{\shortcatvariables},\onehotmapofat{\shortcatindices}{\shortcatvariables}} \, .
    \end{align*}
\end{theorem}
\begin{proof}
    We use the decomposition in \lemref{lem:tensorBasisDecomposition} and have by linearity of contractions for any index tuple $\shortcatindices\in\facstates$
    \begin{align*}
        \contraction{\cencodingofat{\exfunction}{\shortcatvariables},\onehotmapofat{\shortcatindices}{\shortcatvariables}}
        & = \sum_{\tildeshortcatindices\in\facstates}
        \cencodingofat{\exfunction}{\shortcatvariables=\tildeshortcatindices}
        \cdot \contraction{\onehotmapofat{\tildeshortcatindices}{\shortcatvariables},\onehotmapofat{\shortcatindices}{\shortcatvariables}} \\
        & = \sum_{\tildeshortcatindices\in\facstates}
        \exfunctionat{\tildeshortcatindices}
        \cdot \delta_{\tildeshortcatindices,\shortcatindices} \\
        & = \exfunctionat{\shortcatindices}
    \end{align*}
    where we used that one-hot encodings are orthonormal.
\end{proof}

% Coordinate Calculus
Coordinate calculus is the representation of real-valued functions as tensors, from which its evaluations can be retrieved by the scheme of \theref{the:coordinateCalculus}.
This is in contrast to the basis calculus scheme to be discussed (see \theref{the:basisCalculus}), where the contraction-based evaluations of functions outputs one-hot encodings.

% Retrieval of Coordinates from tensor networks
Tensors of large orders often admit a decomposition by tensor networks.
We in the next theorem show, how such a decomposition can be exploited for efficient contractions and in particular coordinate retrieval.

\begin{theorem}
    \label{the:slicedContractionToCores}
    Given a tensor network $\tnetof{\graph}$ on a hypergraph $\graph=(\nodes,\edges)$, disjoint subsets $\nodesa,\nodesb\subset\nodes$ and $\catindexofin{\nodesb}$, we have
    \begin{align*}
        \contractionof{\tnetof{\graph}}{\catvariableof{\nodesa},\indexedcatvariableof{\nodesb}}
        = \contractionof{
            \{\contractionof{\hypercoreof{\edge}}{\catvariableof{\edge/\nodesb},\indexedcatvariableof{\edge\cap\nodesb}} \, : \, \edge\in\edges \}
        }{\catvariableof{\nodesa}} \, .
    \end{align*}
\end{theorem}
\begin{proof}
    By definition of contractions we have for any $\catindexof{\nodesa}$
    \begin{align*}
        \contractionof{\tnetof{\graph}}{\indexedcatvariableof{\nodesa},\indexedcatvariableof{\nodesb}}
        &= \sum_{\catindexof{\nodes/(\nodesa\cup\nodesb)}\in\nodestatesof{\nodes/(\nodesa\cup\nodesb)}} \prod_{\edge\in\edges} \hypercoreofat{\edge}{\indexedcatvariableof{\edge/\nodesb},\indexedcatvariableof{\edge\cap\nodesb}} \\
        &= \contraction{
            \{\contraction{\hypercoreof{\edge}}{\catvariableof{\edge/(\nodesa\cup\nodesb)},\indexedcatvariableof{\edge\cap\nodesa},\indexedcatvariableof{\edge\cap\nodesb}} \, : \, \edge\in\edges \}
        } \\
        &= \contractionof{
            \{\contraction{\hypercoreof{\edge}}{\catvariableof{\edge/\nodesb},\indexedcatvariableof{\edge\cap\nodesb}} \, : \, \edge\in\edges \}
        }{\indexedcatvariableof{\nodesa}}
    \end{align*}
    and the claim follows.
\end{proof}

% Special case of retrieving single coordinates
If we retrieve a single coordinate of a tensor, we have the situation $\nodesa=\varnothing$, $\nodesb=\nodes$.
In that case, \theref{the:slicedContractionToCores} shows, that the coordinate is the product of the coordinates of the cores. % Thus no contraction required!

\sect{Coordinatewise Transforms}\label{sec:coordinatewiseTransforms}

Let us now discuss a scheme to perform transformations of tensors.
We call them coordinatewise, when the target tensor has the same variables as the input tensors, and each coordinate of the target tensor depends only on the respective coordinates of the input tensors. %Examples for non-coordinatewise transforms are e.g. backward and forward maps

\begin{definition}
    \label{def:coordinatewiseTransform}
    Let $\chainingfunction: \parspace \rightarrow \rr$ be a function.
    Then the coordinatewise transform of tensors $\hypercoreofat{\selindex}{\shortcatvariables}$, where $\selindexin$, under $\exfunction$ is the tensor
    \begin{align*}
        \coordinatetrafowrtofat{\chainingfunction}{\hypercoreof{0},\ldots,\hypercoreof{\seldim-1}}{\shortcatvariables}
    \end{align*}
    with coordinates
    \begin{align*}
        \coordinatetrafowrtofat{\chainingfunction}{\hypercoreof{0},\ldots,\hypercoreof{\seldim-1}}{\indexedshortcatvariables}
        = \chainingfunctionof{\hypercoreofat{0}{\indexedshortcatvariables},\ldots,\hypercoreofat{\seldim-1}{\indexedshortcatvariables}} \, .
    \end{align*}
\end{definition}

% \seldim=1
Coordinatewise transforms in case of $\seldim=1$ have been indicated by ellipses in the diagrammatic depiction of contractions.
We will provide a generic tensor network representation in \charef{cha:basisCalculus}, see \theref{the:tensorFunctionComposition}.


In the following lemma, we state that coordinatewise transforms can be restricted to slices of tensors, when
Although this is an obvious fact, this property can tremendously reduce the computational demand of contractions with coordinatewise transforms of tensors.

\begin{lemma}\label{lem:coordinatewisetrafoSliceReduction}
    For any function $\chainingfunction: \rr \rightarrow \rr$, any tensor $\hypercoreat{\shortcatvariables}$ and index $\catindexof{\variableset}$, where $\variableset\subset[\catorder]$, we have
    \begin{align*}
        \coordinatetrafowrtofat{\chainingfunction}{\hypercoreat{\shortcatvariables}}{\catvariableof{[\catorder]/\variableset},\indexedcatvariableof{\variableset}}
        = \coordinatetrafowrtofat{\chainingfunction}{\catvariableof{[\catorder]/\variableset},\indexedcatvariableof{\variableset}}{\catvariableof{[\catorder]/\variableset}} \, .
    \end{align*}
\end{lemma}
\begin{proof}
    For any state $\catindexof{[\catorder]/\variableset}$ we have that
    \begin{align*}
        \coordinatetrafowrtofat{\chainingfunction}{\hypercoreat{\shortcatvariables}}{\indexedcatvariableof{[\catorder]/\variableset},\indexedcatvariableof{\variableset}}
        &= \chainingfunctionof{\hypercoreat{\indexedshortcatvariables}} \\
        &= \coordinatetrafowrtofat{\chainingfunction}{\catvariableof{[\catorder]/\variableset},\indexedcatvariableof{\variableset}}{\indexedcatvariableof{[\catorder]/\variableset}} \, . \qedhere
    \end{align*}
\end{proof}



% Examples
\begin{example}[Hadamard products as coordinatewise transforms]
    Hadamard products of tensors (see \exaref{exa:hadamard}) are a special way of coordinate calculus, where the transform is the product and thus
    \begin{align*}
        \coordinatetrafowrtofat{\cdot\,}{\hypercoreof{0},\ldots,\hypercoreof{\seldim-1}}{\shortcatvariables}
        = \contractionof{\{\hypercoreofat{\selindex}{\shortcatvariables} \, : \, \selindexin \}}{\shortcatvariables} \, .
    \end{align*}
    These hadamard products are applied in the effective computation of conjunctions, as we will discuss in more detail in \secref{sec:hybridCalculus}.
\end{example}

\begin{example}[Exponentiation of energies]
    In \defref{def:expFamily} we introduced exponential families, based on the exponentiation of energies.
    For a statistic $\sstat$, a base measure $\basemeasure$ and a canonical parameter $\canparam$ we defined
    \begin{align*}
        \stanexpdistof{\canparam} = \frac{
            \contractionof{\expof{\contractionof{\sencsstat,\canparam}{\shortcatvariables},\basemeasureat{\shortcatvariables}}}{\shortcatvariables}
        }{
            \contraction{\expof{\contractionof{\sencsstat,\canparam}{\shortcatvariables},\basemeasureat{\shortcatvariables}}}
        } \, .
    \end{align*}
    Both the nominator and the denominator involve a coordinatewise transform of the energy tensor $\expenergy$ by the exponentiation.
    \theref{the:expFamilyTensorRep} provided a transform-free contraction expression by basis encodings, which is the central tool of basis calculus (see \charef{cha:basisCalculus}).

    Let us note, that \lemref{lem:coordinatewisetrafoSliceReduction} enables the energy-based answering of conditional queries, as has been shown in \theref{the:energyContractionQueries}.
\end{example}




\sect{Directed Tensors}

Directionality as defined in \defref{def:directedTensor} is a constraint on the structure of a tensor, namely that the contraction leaving only incoming variables open trivializes the tensor.
We have motivated such constraints by conditional distributions, see \defref{def:condIndependence}, and referred to Markov Networks (see \defref{def:markovNetwork}) satisfying these by Bayesian Networks (see \defref{def:bayesianNetwork}).
To support our findings therein, we now discuss in more detail the connection between directed hypergraphs and directed tensors.

\begin{definition}[Directed Hypergraph]
    A directed hyperedge is a hyperedge, which node set is split into disjoint sets of incoming and outgoing nodes.
    We say a hypercore $\hypercoreof{\edge}$ decorating a directed hyperedge respects the direction, when it is a conditional probability tensor with respect to the direction of the hyperedge.
    The hypergraph is acyclic, when there is no nonempty cycle of node tuples $(\node_1,\node_2)$, such that $\node_1$ is an incoming node and $\node_2$ an outgoing node of the same hyperedge.
\end{definition}

% Multiple Directions possible
There can be multiple ways to direct a tensor, with an extreme example being Diracs Delta Tensors to be introduced in the next example.
More general examples are basis encodings of invertible functions.

\begin{example}[Dirac Delta Tensors]\label{exa:diracDeltaTensor}
    Given a set of variables $\shortcatvariables=\catvariables$ with identical dimension $\catdim$, Diracs Delta Tensor is the element
    \begin{align*}
        \dirdeltawith \in \bigotimes_{\catenumeratorin} \rr^{\catdim}
    \end{align*}
    with coordinates
    \begin{align}
        \dirdeltaofat{[\catorder],\catdim}{\indexedshortcatvariables} =
        \begin{cases}
            1 \quad & \ifspace \catindexof{0} = \ldots = \catindexof{\catorder-1} \\
            0 & \text{else}
        \end{cases} \, .
    \end{align}
    The contractions with respect to subsets $\secnodes\subset[\catorder]$ are
    \begin{align}
        \contractionof{\dirdeltaof{[\catorder],\catdim}}{\catvariableof{\secnodes}} =
        \begin{cases}
            \catdim & \ifspace \secnodes = \varnothing \\
            \onesat{\catvariableof{\secnodes}} & \ifspace \cardof{\secnodes} = 1\\
            \dirdeltaofat{\secnodes,\catdim}{\catvariableof{\secnodes}} & \text{else}
        \end{cases} \, .
    \end{align}
    Thus are directed for any orientation of the respective edge with exactly one incoming variable.
\end{example}

We can use Diracs Delta Tensors to represent any contraction of a tensor network on a hypergraph by a tensor network on a graph, as we show next.

\begin{lemma}
    \label{lem:deltification}
    Let $\graph=(\nodes,\edges)$ be a hypergraph and $\extnet$ a tensor network on $\graph$.
    We build a graph $\secgraph=(\secnodes,\secedges\cup\Delta^{\graph})$ and a tensor network $\tnetof{\secgraph}$ by % ! See Bethe Cluster Graph definition !
    \begin{itemize}
        \item Recolored Edges $\secedges = \{\tilde{\edge} \, : \, \edge\in \edges\}$ where $\tilde{\edge} = \{\node^{\edge} \, : \, \node\in\edge\}$, which decoration tensor $\hypercoreof{\tilde{\edge}}$ has same coordinates as $\hypercoreof{\edge}$
        \item Nodes $\secnodes = \bigcup_{\edge\in\edges}\tilde{\edge}$ %$\secnodes = \bigcup_{\edge\in\edges}\{\node^{\edge} \, : \, \node\in\edge \}$
        \item Delta Edges $\Delta^{\graph} =  \big\{ \{\node\} \cup \{\node^{\edge} \, : \, \edge\ni\node \} \, : \, \node\in\nodes \big\} $ each of which decorated by a delta tensor $\delta^{\{\node^{\edge} \, : \, \edge\ni\node \}}$
    \end{itemize}
    Then we have
    \begin{align*}
        \contractionof{\extnet}{\catvariableof{\nodes}} =  \contractionof{\tnetof{\secgraph}}{\catvariableof{\nodes}}  \, .
    \end{align*}
\end{lemma}
\begin{proof}
    For any $\catindexof{\nodes}$ we have
    \begin{align*}
        \contractionof{\tnetof{\secgraph}}{\indexedcatvariableof{\nodes}}
        & = \contraction{\{\hypercoreof{\tilde{\edge}}[\catvariableof{\{\node^{\edge} : \node\in\edge\}}] : \edge \in \edges \}\cup
        \{\delta^{\{\node\} \cup \{\node^{\edge} \, : \, \edge\ni\node \}}[\catvariableof{\{\node^{\edge} : \edge\ni\node \}}, \indexedcatvariableof{\node} ]  : \node\in\nodes \}
        } \\
        & =  \contraction{\{\hypercoreof{\tilde{\edge}}[\catvariableof{\{\node^{\edge} : \node\in\edge\}} = \catindexof{\{\node : \node\in\edge\}} ] : \edge \in \edges \}
        } \\
        & = \contractionof{\extnet}{\indexedcatvariableof{\nodes}} \, ,
    \end{align*}
    which establishes the claim.
\end{proof}

\subsect{Normalization}

Normed tensors (see \defref{def:normalization}) are directed and directed tensors invariant under normalization wrt their incoming and outgoing variable, as we show next.

\begin{theorem}
    \label{the:normalizationDirected}
    For any tensor network $\extnet$ on variables $\nodes$ that can be normed with respect to $\innodes$ and $\outnodes$, the normalization is directed with $\innodes$ incoming and $\outnodes$ outgoing.
\end{theorem}
\begin{proof}
    We have for any incoming state ${\atomlegindexof{\innodes}\in\bigtimes_{\node\in\innodes}\catdimof{\node}}$ that
    \begin{align*}
        \contraction{\normalizationofwrt{\extnet}{\innodes}{\outnodes}, \onehotmapof{\atomlegindexof{\innodes}}}
        & =  \frac{
            \contraction{\extnet\cup\{\onehotmapof{\atomlegindexof{\innodes}}\}}
        }{
            \contraction{\extnet\cup\{\onehotmapof{\atomlegindexof{\innodes}}\}}
        } \, .
    \end{align*}
    By \defref{def:directedTensor}, $\normalizationofwrt{\extnet}{\outnodes}{\innodes}$ is thus directed.
\end{proof}

The normalization operation coincides in cases of non-negative tensors with the conditioning of a Markov Network representing a probability distribution.

\subsect{Normalization Equations}

Normalization equations capture certain properties of normalizations of tensors.
We first show that any normalizable tensor is the contraction of its normalization and an accompanying contraction, which generalizes the Bayes \theref{the:bayes} towards more generic normalizable tensors.

\begin{theorem}[normalization as a Contraction Equation]
    \label{the:normalizationContractionEQ}
    For any on $\innodes$ normalizable tensor $\hypercoreat{\catvariableof{\nodes}}$, where $\innodes\dot{\cup}\outnodes=\nodes$, we have
    \begin{align*}
        \contractionof{\hypercore}{\catvariableof{\nodes}}
        = \contractionof{\normalizationofwrt{\hypercore}{\catvariableof{\outnodes}}{\catvariableof{\innodes}},\contractionof{\hypercore}{\catvariableof{\innodes}}}{\catvariableof{\nodes}} \, .
    \end{align*}
\end{theorem}
\begin{proof}
    Let us choose indices $\catindexof{\innodes}$ and $\catindexof{\outnodes}$.
    We have that
    \begin{align*}
        %\contractionof{
        \normalizationofwrt{\hypercore}{\indexedcatvariableof{\innodes}}{\indexedcatvariableof{\outnodes}}
        %}{\indexedcatvariableof{\innodes},\indexedcatvariableof{\outnodes}}
        = \frac{
            \contractionof{\hypercore}{\indexedcatvariableof{\innodes},\indexedcatvariableof{\outnodes}}
        }{
            \contractionof{\hypercore}{\indexedcatvariableof{\innodes}}
        }
    \end{align*}
    and therefor
    \begin{align*}
        \contractionof{\hypercore}{\indexedcatvariableof{\innodes},\indexedcatvariableof{\outnodes}} =
        \normalizationofwrt{\hypercore}{\indexedcatvariableof{\innodes}}{\indexedcatvariableof{\outnodes}}
        \cdot
        \contractionof{\hypercore}{\indexedcatvariableof{\innodes}}
    \end{align*}
    Since the equation holds for arbitrary indices, the claim is established.
\end{proof}

Based on this property, we now show a generic decomposition scheme of tensors, which generalizes the chain rule of \theref{the:chainRule}.

\begin{theorem}[Generic Chain Rule]
    \label{the:genericChainRule}
    For any Tensor $\hypercoreat{\catvariableof{\nodes}}$ and any total order $\prec$ on the nodes $\nodes$ we have % ! CAN DIRECTLY USE [d] when having the order !
    \begin{align*}
        \hypercoreat{\catvariableof{\nodes}} =
        \contractionof{
            \{\normalizationofwrt{\hypercore}{\catvariableof{\node}}{\catvariableof{\prenodes}}  \, : \nodein \}
        }{\catvariableof{\nodes}} \, ,
    \end{align*}
    provided that the normalizations exist.
\end{theorem}
\begin{proof}
    We apply \theref{the:normalizationContractionEQ} on the tensor
    \begin{align*}
        \normalizationofwrt{\hypercore}{
            \catvariableof{\node},\catvariableof{\afternodes}
        }{
            \indexedcatvariableof{\prenodes}
        } \, ,
    \end{align*}
    where $\nodein$ and $\catindexof{\nodes}$ are chosen arbitrarly.
    For any $\nodein$ we get
    \begin{align*}
        \normalizationofwrt{\hypercore}{
            \catvariableof{\node},\catvariableof{\afternodes}
        }{
            \catvariableof{\prenodes}
        }
        = \contractionof{
            \normalizationofwrt{\hypercore}{
                \catvariableof{\afternodes}
            }{
                \catvariableof{\node},\catvariableof{\prenodes}
            },
            \normalizationofwrt{\hypercore}{
                \catvariableof{\node}
            }{
                \catvariableof{\prenodes}
            }
        }{
            \catvariableof{\nodes}
        } \, .
    \end{align*}
    Applying this equation iteratively and making use of the commutation of contractions we get for any $\nodein$
    \begin{align*}
        \normalizationofwrt{\hypercore}{
            \catvariableof{\node},\catvariableof{\afternodes}
        }{
            \catvariableof{\prenodes}
        }
        = \contractionof{
            \normalizationofwrt{\hypercore}{
                \catvariableof{\secnode}
            }{
                \catvariableof{\{\thirdnode : \thirdnode \prec \secnode, \thirdnode\neq\secnode\}}
            }
            \, : \node \prec \secnode
        }{
            \catvariableof{\nodes}
        } \, .
    \end{align*}
    With the maximal node $\node$, that is the $\node$, such that no $\secnode\in\nodes$ with $\node\prec\secnode$ and $\node\neq\secnode$ exists, this is the claim.
\end{proof}


\subsect{Contraction of Directed Tensors}

Let us now investigate, which contractions inherit the directionality of the tensors.

%Next we state that specific contraction of conditional probability tensors are still conditional probability tensors.

%\red{Can be extended to single outgoing legs, by using delta tensors at hyperedges.}

%\begin{theorem}
%	Given a directed acyclic hypergraph, which hyperdedges are decorated by tensor cores respecting the direction.
%	Then the contractions, where all closed nodes appear exactly once as an incoming node and exactly once as an outgoing node, and where all open nodes appear in single hyperedges, are conditional probability tensors
%\end{theorem}
%\begin{proof}
%	It is enough to show this property on the contraction of two hypercores.
%	Since the hypergraph is acyclic, the coinciding nodes are all outgoing on the one and incoming to the other hyperedge.
%	Let the hyperedge with the incoming nodes $\edge_1$ and the one with the outgoing nodes $\edge_2$
%	We need to show that when further contracting the contraction with trivial tensors on the outgoing and basis tensors on the incoming legs we get $1$.
%	For any $\catindex$ and $\seccatindex$ this holds since

%	Here we used in the first equality, that $\hypercoreof{\edge_1}$ is a conditional probability tensor and in the second the same property for $\hypercoreof{\edge_2}$.
%\end{proof}

% Hadamard does not preserve probabilities
%We need to ass as assumption in \theref{the:conditionalContractionPreservation}, that each node is to at most one hyperedge and to at most one hyperedge outgoing.
%This is due to the failure of Hadamard products of probability tensors to be probability tensors themself.


\begin{theorem}
    \label{the:conditionalContractionPreservation}
    Let $\graph=(\nodes,\edges)$ be a directed acyclic hypergraph, such that each node $\node\in\edge$ appears at most in one hyperedge as an outgoing variable and denote $\innodes$ as those nodes, which do not appear as outgoing variables.
    For any tensor network $\extnet$ respecting the direction of $\graph$ we have that
    \[ \contractionof{\extnet}{\catvariableof{\innodes}} = \onesat{\catvariableof{\innodes}} \, , \]
    that is $\contractionof{\extnet}{\catvariableof{\nodes}}$ is a directed tensor with $\innodes$ incoming and $\nodes/\innodes$ outgoing.
\end{theorem}
\begin{proof}
    We show the theorem only for the case of hypergraphs, where variables are appearing at most in two hyperedges.
    If a hypergraph fails to satisfy this assumption, we apply \lemref{lem:deltification} and add delta tensors copying the variables, which are appearing in multiple tensors, and arrive at a tensor network with nodes appearing in at most two hyperedges.

% Approach: Contracting with ones
    We show the theorem over induction on the number $n$ of cores.

    \paragraph{$n=1$:} The claim holds trivially, when $\extnet$ consists of a single core.

    \paragraph{$n-1\rightarrow n$:} Let us assume, that the claim holds for graphs with $n-1$ hyperedges and let $\tnetof{\graph}$ be a tensor network with $n$ hyperedges.
    Since the hypergraph is acyclic, we find an edge $\edge\in\edges$ such that all outgoing nodes of $\edge$ are not appearing as an incoming node in any edge.
    We then apply \theref{the:splittingContractions} and get
    \begin{align*}
        \contractionof{\extnet}{\catvariableof{\innodes}}
        &= \contractionof{
            \tnetof{(\nodes,\edges/\{\edge\})} \cup \{\hypercoreofat{\edge}{\catvariableof{\incomingnodes},\catvariableof{\outgoingnodes}}\}
        }{\catvariableof{\innodes}} \\
        & = \contractionof{
            \tnetof{(\nodes,\edges/\{\edge\})} \cup \{\contractionof{\hypercoreof{\edge}}{\catvariableof{\incomingnodes}} \}
        }{\catvariableof{\innodes}} \\
        & = \contractionof{
            \tnetof{(\nodes,\edges/\{\edge\})} \cup \{\onesat{\catvariableof{\incomingnodes}} \}
        }{\catvariableof{\innodes}} \\
        & \contractionof{
            \tnetof{(\nodes,\edges/\{\edge\})} \}
        }{\catvariableof{\innodes}} \, .
    \end{align*}
    We then notice that the hypergraph $(\nodes,\edges/\{\edge\})$ has $n-1$ hyperedges and each node appears at most once as an incoming and at most once as an outgoing node.
    Thus, we apply the assumption of the induction and get
    \begin{align*}
        \contractionof{\extnet}{\catvariableof{\innodes}} = \contractionof{
            \tnetof{(\nodes,\edges/\{\edge\})} \}
        }{\catvariableof{\innodes}} = \onesat{\catvariableof{\innodes}} \, . & \qedhere
    \end{align*}
\end{proof}


\sect{Proof of Hammersley-Clifford Theorem}\label{sec:proofHCTheorem}

Let us now proof the Hammersley-Clifford theorem formulated in \charef{cha:probRepresentation} as \theref{the:condIndMN}.
Different to the original statement (see \cite{clifford_markov_1971}), we here proof the analogous statement for hypergraphs, where we have to demand the property of clique-capturing defined in \defref{def:ccHypergraph}.
We start with showing the following Lemmata to be exploited in the proof.

\begin{lemma}
    \label{the:contractionFactorization}
    Let $\hypercoreat{\catvariableof{\nodes}}$ be a positive tensor and $\seccatindexof{\nodes}$ an arbitrary index.
    Then we have
    \begin{align*}
        \hypercoreat{\catvariableof{\nodes}}
        = \contractionof{
            \big(\contractionof{\hypercore}{\catvariableof{\nodes/\thirdnodes}, \catvariableof{\thirdnodes} = \seccatindexof{\thirdnodes}}\big)^{(-1)^{\cardof{\secnodes}-\cardof{\thirdnodes}}} \, : \, \thirdnodes \subset \secnodes \subset \nodes
        }{\catvariableof{\nodes}} \, ,
    \end{align*}
    where the exponentiation is performed coordinatewise and positivity of $\hypercore$ ensures the well-definedness.
\end{lemma}
\begin{proof}
    It suffices to show, that for an arbitrary index $\catindexof{\nodes}$ be an arbitrary index we have
    \begin{align*}
        \hypercoreat{\indexedcatvariableof{\nodes}}
        = \prod_{\secnodes\subset\nodes} \prod_{\thirdnodes\subset\secnodes}
        \big(\contractionof{\hypercore}{\indexedcatvariableof{\nodes/\thirdnodes}, \catvariableof{\thirdnodes} = \seccatindexof{\thirdnodes}}\big)^{(-1)^{\cardof{\secnodes}-\cardof{\thirdnodes}}} \, .
    \end{align*}
    We do this by applying a logarithm on the right hand side and grouping the terms by $\thirdnodes$ as
    \begin{align*}
        %\lnof{\hypercoreat{\indexedcatvariableof{\nodes}}}
        & \lnof{\prod_{\secnodes\subset\nodes} \prod_{\thirdnodes\subset\secnodes}
            \contractionof{\hypercore}{\indexedcatvariableof{\nodes/\thirdnodes}, \catvariableof{\thirdnodes} = \seccatindexof{\thirdnodes}}\big)^{(-1)^{\cardof{\secnodes}-\cardof{\thirdnodes}}}} \\
        & = \sum_{\thirdnodes\subset\nodes} \lnof{\contractionof{\hypercore}{\indexedcatvariableof{\nodes/\thirdnodes}, \catvariableof{\thirdnodes} = \seccatindexof{\thirdnodes}}}
        \left( \sum_{\secnodes\subset\nodes \, : \, \thirdnodes\subset \secnodes} (-1)^{\cardof{\secnodes}-\cardof{\thirdnodes}} \right) \\
        & =  \sum_{\thirdnodes\subset\nodes} \lnof{\contractionof{\hypercore}{\indexedcatvariableof{\nodes/\thirdnodes}, \catvariableof{\thirdnodes} = \seccatindexof{\thirdnodes}}}
        \left( \sum_{i \in [\cardof{\nodes}-\cardof{\thirdnodes}]}  (-1)^{i}  \binom{\cardof{\nodes}-\cardof{\thirdnodes}}{i}  \right)
    \end{align*}
    Now, by the generic binomial theorem we have that for $n\in\nn, n \neq 0$
    \[ 0 = (1-1)^n = \sum_{i \in [n]}  (-1)^{i}  \binom{n}{i}   \, . \]
    Therefore, the summands for $\thirdnodes\neq\nodes$ vanish and we have
    \begin{align*}
        & \lnof{ \prod_{\secnodes\subset\nodes} \prod_{\thirdnodes\subset\secnodes}
            \big(\contractionof{\hypercore}{\indexedcatvariableof{\nodes/\thirdnodes}, \catvariableof{\thirdnodes} = \seccatindexof{\thirdnodes}}\big)^{(-1)^{\cardof{\secnodes}-\cardof{\thirdnodes}}} } \\
        & = \lnof{\hypercoreat{\indexedcatvariableof{\nodes}}}
        \left( \sum_{i \in [0]}  (-1)^{i}  \binom{0}{i}  \right) \\
        & = \lnof{\hypercoreat{\indexedcatvariableof{\nodes}}} \, .
    \end{align*}
    Applying the exponential function on both sides establishes the claim.
\end{proof}

\begin{lemma}
    \label{lem:independentContractionFactorization}
    Let $\hypercore$ be a positive tensor, $\secnodes\subset\nodes$ and arbitrary subset and $\catindexof{\secnodes}$ an arbitrary index.
    When there are $\nodesa,\nodesb \in\secnodes$, such that
    \begin{align*}
        \normalizationofwrt{\hypercore}{\catvariableof{\nodesa,\nodesb}}{\catvariableof{\nodes/\{\nodesa,\nodesb\}}}
        = \contractionof{
            \normalizationofwrt{\hypercore}{\catvariableof{\nodesa}}{\catvariableof{\nodes/\{\nodesa,\nodesb\}}},
            \normalizationofwrt{\hypercore}{\catvariableof{\nodesb}}{\catvariableof{\nodes/\{\nodesa,\nodesb\}}}
        }{\catvariableof{\secnodes}}
    \end{align*}
    then
    \begin{align*}
        \prod_{\thirdnodes\subset\secnodes}
        \left(\contractionof{\hypercore}{\indexedcatvariableof{\nodes/\thirdnodes}, \catvariableof{\thirdnodes} = \seccatindexof{\thirdnodes}}\right)^{(-1)^{\cardof{\secnodes}-\cardof{\thirdnodes}}} = 1 \, .
    \end{align*}
\end{lemma}
\begin{proof}
    We abbreviate
    \[ Z_{\thirdnodes} = \contractionof{\hypercore}{\indexedcatvariableof{\nodes/\thirdnodes}, \catvariableof{\thirdnodes} = \seccatindexof{\thirdnodes}} \, .
    \]
    By reorganizing the sum over $\thirdnodes\subset\secnodes$ into  $\thirdnodes\subset\secnodes/\nodesa\cup\nodesb$ we have
    \begin{align}
        \label{eq:indContFacProof}
        \prod_{\thirdnodes\subset\secnodes}
        \left(
        Z_{\thirdnodes}
        \right)^{(-1)^{\cardof{\secnodes}-\cardof{\thirdnodes}}} =
        \prod_{\thirdnodes\subset\secnodes/\{\nodesa,\nodesb\}}
        \left(
        \frac{
            Z_{\thirdnodes} \cdot Z_{\thirdnodes\cup\{\nodesa,\nodesb\}}
        }{
            Z_{\thirdnodes\cup\{\nodesa\}} \cdot Z_{\thirdnodes\cup\{\nodesb\}}
        }
        \right)^{(-1)^{\cardof{\secnodes}-\cardof{\thirdnodes}}} \, .
    \end{align}
    From the independence assumption it follows that for any index $\catindex$
    \begin{align*}
        & \normalizationofwrt{\hypercore}{
            \indexedcatvariableof{\nodesa}
        }{\indexedcatvariableof{\nodes/\thirdnodes\cup\{\nodesa,\nodesb\}},\catvariableof{\thirdnodes}=\seccatindexof{\thirdnodes},  \indexedcatvariableof{\nodesb} }
        \\
        & \quad =
        \normalizationofwrt{\hypercore}{
            \indexedcatvariableof{\nodesa}
        }{\indexedcatvariableof{\nodes/\thirdnodes\cup\{\nodesa,\nodesb\}}, \catvariableof{\thirdnodes}=\seccatindexof{\thirdnodes}} \\
        & \quad  =
        \normalizationofwrt{\hypercore}{
            \indexedcatvariableof{\nodesa}
        }{\indexedcatvariableof{\nodes/\thirdnodes\cup\{\nodesa,\nodesb\}},\catvariableof{\thirdnodes}=\seccatindexof{\thirdnodes},  \catvariableof{\nodesb} = \seccatindexof{\nodesb}}
    \end{align*}
    Applying this in each squares bracket term of \eqref{eq:indContFacProof} we get
    \begin{align*}
        \frac{
            Z_{\thirdnodes}
        }{
            Z_{\thirdnodes\cup\{\nodesa\}}
        }
        & =
        \frac{
            \normalizationofwrt{\hypercore}{
                \indexedcatvariableof{\nodesa}
            }{\indexedcatvariableof{\nodes/\thirdnodes\cup\{\nodesa,\nodesb\}}, \catvariableof{\thirdnodes}=\seccatindexof{\thirdnodes}, \indexedcatvariableof{\nodesb} }
        }{
            \normalizationofwrt{\hypercore}{
                \catvariableof{\nodesa} =\seccatindexof{\nodesa}
            }{\indexedcatvariableof{\nodes/\thirdnodes\cup\{\nodesa,\nodesb\}} , \catvariableof{\thirdnodes}=\seccatindexof{\thirdnodes}, \indexedcatvariableof{\nodesb}}
        } \\
        & =
        \frac{
            \normalizationofwrt{\hypercore}{
                \indexedcatvariableof{\nodesa}
            }{\indexedcatvariableof{\nodes/\thirdnodes\cup\{\nodesa,\nodesb\}}, \catvariableof{\thirdnodes}=\seccatindexof{\thirdnodes}, \catvariableof{\nodesb} = \seccatindexof{\nodesb}}
        }{
            \normalizationofwrt{\hypercore}{
                \catvariableof{\nodesa} =\seccatindexof{\nodesa}
            }{\indexedcatvariableof{\nodes/\thirdnodes\cup\{\nodesa,\nodesb\}}, \catvariableof{\thirdnodes}=\seccatindexof{\thirdnodes},\catvariableof{\nodesb} = \seccatindexof{\nodesb}}
        } \\
        & =
        \frac{
            Z_{\thirdnodes\cup\{\nodesb\}}
        }{
            Z_{\thirdnodes\cup\{\nodesa,\nodesb\}}
        } \, .
    \end{align*}
    Thus, each factor in \eqref{eq:indContFacProof} is trivial, which establishes the claim.
\end{proof}

We are finally ready to proof the Hammersley-Clifford \theref{the:condIndMN} based on the Lemmata above.

%\begin{theorem}[\theref{the:condIndMN}]
%	Let $\probat{\catvariableof{\nodes}}$ be a probability distribution and $\graph$ a clique-capturing hypergraph, such that for $\nodesa$, $\nodesb$, $\nodesc$ we have that $\catvariableof{\nodesa}$ is independent of $\catvariableof{\nodesb}$ conditioned on $\catvariableof{\nodesc}$, when $\nodesc$ separates $\nodesa$ and $\nodesb$ in the hypergraph.
%	Then there is a Markov Network on $\graph$, which distribution is equal to $\probat{\catvariableof{\nodes}}$.
%\end{theorem}
\begin{proof}[Proof of \theref{the:condIndMN}]
    By \lemref{the:contractionFactorization} we have for any index $\catindexof{\nodes}$
    \begin{align*}
        \probat{\indexedcatvariableof{\nodes}} =
        \prod_{\secnodes\subset\nodes} \prod_{\thirdnodes\subset\secnodes}
        \left(
        \probat{\indexedcatvariableof{\thirdnodes},\catvariableof{\nodes/\thirdnodes}=\seccatindexof{\nodes/\thirdnodes}}
        %	\contractionof{\extnet\cup\{\onehotmapof{\catindexof{\nodes/\thirdnodes}}\}}{\catvariableof{\thirdnodes}}
        \right)^{(-1)^{\cardof{\secnodes}-\cardof{\thirdnodes}}}
    \end{align*}
    For any subset $\secnodes\subset\nodes$, which is not contained in a hyperedge, we find $\nodesa,\nodesb \in\secnodes$ such that $\catvariableof{\nodesa}$ is independendent on $\catvariableof{\nodesb}$ conditioned on $\catvariableof{\secnodes/\{\nodesa,\nodesb\}}$.
    If no such nodes $\nodesa,\nodesb \in\secnodes$ exists, $\secnodes$ would be contained in a hyperedge, since the hypergraph is assumed to be clique-capturing.
    By \lemref{lem:independentContractionFactorization} we then have
    \begin{align*}
        \prod_{\thirdnodes\subset\secnodes}
        \left(
        \probat{\indexedcatvariableof{\thirdnodes},\catvariableof{\nodes/\thirdnodes}=\seccatindexof{\nodes/\thirdnodes}}
        \right)^{(-1)^{\cardof{\secnodes}-\cardof{\thirdnodes}}} = 1 \, .
    \end{align*}
    We label by a function
    \begin{align*}
        \alpha: \{\secnodes : \exists\edge\in\edges: \secnodes \subset \edge \} \rightarrow \edges
    \end{align*}
    the remaining node subsets by a hyperedge containing the subset.
    We build the tensor
    \begin{align*}
        \hypercoreofat{\edge}{\catvariableof{\edge}} = \prod_{\secnodes \, : \, \alpha(\secnodes) = \edge} \prod_{\thirdnodes\subset\secnodes}
        \left(
        \probat{\indexedcatvariableof{\thirdnodes},\catvariableof{\nodes/\thirdnodes}=\seccatindexof{\nodes/\thirdnodes}}
        \right)^{(-1)^{\cardof{\secnodes}-\cardof{\thirdnodes}}} \, .
    \end{align*}
    and get, that
    \begin{align*}
        \probat{\catvariableof{\nodes}} & = \contractionof{\extnetasset}{\catvariableof{\nodes}} \\
        & = \normalizationof{\extnetasset}{\catvariableof{\node}} \, .
    \end{align*}
    We have thus constructed a Markov Network with trivial partition function, which contraction coincides with the probability distribution.
\end{proof}

\sect{Differentiation of Contraction}

The structured mean field approaches discussed in \charef{cha:probReasoning} used differentiations of the parametrized tensor networks.
Let us now develop in more detail, how the contraction of tensor networks with variable cores is differentiated.
We capture in additional variables $\seccatvariable$ selecting the coordinates of a tensor, which are varied in a differentiation.

\begin{lemma}
    \label{lem:difMNprob}
    For any tensor network $\extnet$ with positive $\hypercoreof{\edge}$ we have
    \begin{align*}
        \difwrt{\hypercoreofat{\edge}{\seccatvariableof{\edge}}} \extnetdist
        & = \contractionof{
            \identityat{\seccatvariableof{\edge},\edgevariables},
            \frac{\contractionof{\extnet}{\edgevariables}}{\hypercoreofat{\edge}{\edgevariables}},
            \normalizationofwrt{\extnet}{\catvariableof{\nodes/\edge}}{\edgevariables} }{\seccatvariableof{\edge},\nodevariables} \\
        & \quad -  \extnetdist \otimes \contractionof{\frac{\contractionof{\extnet}{\seccatvariableof{\edge}}}{\hypercoreofat{\edge}{\seccatvariableof{\edge}}}
        }{\seccatvariableof{\edge}} \, .
    \end{align*}
\end{lemma}
\begin{proof}
    By multilinearity of tensor network contractions we have
    \begin{align*}
        \difwrt{\hypercoreofat{\edge}{\seccatvariableof{\edge}}} \contractionof{\extnet}{\nodevariables}
        & = \contractionof{\{\identityat{\seccatvariableof{\edge},\edgevariables}\}\cup\{\hypercoreofat{\secedge}{\catvariableof{\secedge}} \, : \, \secedge\neq\edge \}}{\seccatvariableof{\edge},\nodevariables}
    \end{align*}
    and thus
    \begin{align*}
        \difwrt{\hypercoreofat{\edge}{\seccatvariableof{\edge}}} \contraction{\extnet}
        & = \contractionof{\{\identityat{\seccatvariableof{\edge},\edgevariables}\}\cup\{\hypercoreofat{\secedge}{\catvariableof{\secedge}} \, : \, \secedge\neq\edge \}}{\seccatvariableof{\edge}} \, .
    \end{align*}

    Using both we get
    \begin{align}
        \difwrt{\hypercoreofat{\edge}{\seccatvariableof{\edge}}} \extnetdist
        & = \difwrt{\hypercoreofat{\edge}{\seccatvariableof{\edge}}}  \frac{\contractionof{\extnet}{\nodevariables}}{\contraction{\extnet}} \nonumber \\
        & = \frac{ \difwrt{\hypercoreofat{\edge}{\seccatvariableof{\edge}}} \contractionof{\extnet}{\nodevariables}}{\contraction{\extnet}}
        - \frac{ \contractionof{\extnet}{\nodevariables} \difwrt{\hypercoreofat{\edge}{\seccatvariableof{\edge}}} \contraction{\extnet} }{(\contraction{\extnet})^2} \nonumber \\
        & = \frac{ \contractionof{\{\identityat{\seccatvariableof{\edge},\edgevariables}\}\cup\{\hypercoreofat{\secedge}{\catvariableof{\secedge}} \, : \, \secedge\neq\edge \}}{\seccatvariableof{\edge},\nodevariables}}{\contraction{\extnet}} \nonumber \\
        & \quad\quad - \extnetdist \cdot  \frac{\contractionof{\{\identityat{\seccatvariableof{\edge},\edgevariables}\}\cup\{\hypercoreofat{\secedge}{\catvariableof{\secedge}} \, : \, \secedge\neq\edge \}}{\seccatvariableof{\edge}}}{\contraction{\extnet}} \label{eq:differentiatingMNpreresult}
        % = \contractionof{\{\identityat{\seccatvariableof{\edge},\edgevariables}\}\cup\{\hypercoreofat{\secedge}{\catvariableof{\secedge}} \, : \, \secedge\neq\edge \}}{\seccatvariableof{\edge},\nodevariables}
    \end{align}

    For the first term we get with a normalization equation (see \theref{the:normalizationContractionEQ}) that
    \begin{align*}
        \frac{ \contractionof{\{\identityat{\seccatvariableof{\edge},\edgevariables}\}\cup\{\hypercoreofat{\secedge}{\catvariableof{\secedge}} \, : \, \secedge\neq\edge \}}{\seccatvariableof{\edge},\nodevariables}}{\contraction{\extnet}}
        &= \frac{\contractionof{\{\identityat{\seccatvariableof{\edge},\edgevariables}\}\cup\{\hypercoreofat{\secedge}{\catvariableof{\secedge}} \, : \, \secedge\in\edges \}}{\seccatvariableof{\edge},\nodevariables}}{\hypercoreofat{\edge}{\edgevariables}  \cdot \contraction{\extnet}} \\
        &= \frac{
            \contractionof{\identityat{\seccatvariableof{\edge},\edgevariables},\extnetdist}{\seccatvariableof{\edge},\nodevariables}
        }{\hypercoreofat{\edge}{\edgevariables}}  \\
        &= \frac{\contractionof{\identityat{\seccatvariableof{\edge},\edgevariables},
            \normalizationof{\extnet}{\edgevariables},
            \normalizationofwrt{\extnet}{\catvariableof{\nodes/\edge}}{\edgevariables}
        }{\seccatvariableof{\edge},\nodevariables}
        }{\hypercoreofat{\edge}{\edgevariables}}  \, .
    \end{align*}

    Analogously, we have
    \begin{align*}
        \frac{ \contractionof{\{\identityat{\seccatvariableof{\edge},\edgevariables}\}\cup\{\hypercoreofat{\secedge}{\catvariableof{\secedge}} \, : \, \secedge\neq\edge \}}{\seccatvariableof{\edge}}}{\contraction{\extnet}}
        &= \frac{\contractionof{\identityat{\seccatvariableof{\edge},\edgevariables},
            \normalizationof{\extnet}{\edgevariables}%,
        %\normalizationofwrt{\extnet}{\catvariableof{\nodes/\edge}}{\edgevariables}
        }{\seccatvariableof{\edge}}
        }{\hypercoreofat{\edge}{\edgevariables}}  \, .
    \end{align*}

    With \eqref{eq:differentiatingMNpreresult}, we arrive at the claim
    \begin{align*}
        \difwrt{\hypercoreofat{\edge}{\seccatvariableof{\edge}}} \extnetdist
        & = \contractionof{
            \identityat{\seccatvariableof{\edge},\edgevariables},
            \frac{\contractionof{\extnet}{\edgevariables}}{\hypercoreofat{\edge}{\edgevariables}},
            \normalizationofwrt{\extnet}{\catvariableof{\nodes/\edge}}{\edgevariables} }{\seccatvariableof{\edge},\nodevariables} \\
        & \quad -  \extnetdist \otimes \contractionof{\frac{\contractionof{\extnet}{\seccatvariableof{\edge}}}{\hypercoreofat{\edge}{\seccatvariableof{\edge}}}
        }{\seccatvariableof{\edge}} \, . \qedhere
    \end{align*}
\end{proof}


% Could put it into contraction equations?
\begin{lemma}
    \label{lem:difMNExpectation}
    %See Proposition 11.9 in Koller Book.
    For any function $\exfunction(\hypercoreof{\edge})[\nodevariables]$ we have
    \begin{align*}
        \difwrt{\hypercoreofat{\edge}{\seccatvariableof{\edge}}} &
        \contraction{\extnetdist,\exfunction(\hypercoreof{\edge})[\nodevariables]} \\
        = &
        \frac{\normalizationof{\extnet}{\indexedcatvariableof{\edge}}}{\hypercoreofat{\edge}{\indexedcatvariableof{\edge}}}
        \Big( \contraction{\normalizationofwrt{\extnet}{\catvariableof{\nodes/\edge}}{\indexedcatvariableof{\edge}}, \exfunction(\hypercoreof{\edge})[\nodevariables,\seccatvariableof{\edge}]} \\
        & \quad \quad \quad \quad \quad - \contraction{\extnetdist, \exfunction(\hypercoreof{\edge})[\nodevariables]}
        \Big) \\
        & + \contraction{ \extnetdist
        \difofwrt{\exfunction(\hypercoreof{\edge})[\nodevariables]}{\hypercoreofat{\edge}{\seccatvariableof{\edge}}}
        }
    \end{align*}
\end{lemma}
\begin{proof}
    By product rule of differentiation we have
    \begin{align*}
        \difwrt{\hypercoreofat{\edge}{\indexedcatvariableof{\edge}}} \contraction{\extnetdist,\exfunction(\hypercoreof{\edge})[\nodevariables]}
        & =  \contraction{\difwrt{\hypercoreofat{\edge}{\indexedcatvariableof{\edge}}}\extnetdist,\exfunction(\hypercoreof{\edge})[\nodevariables]} \\
        & \quad +  \contraction{\extnetdist,\difwrt{\hypercoreofat{\edge}{\indexedcatvariableof{\edge}}}\exfunction(\hypercoreof{\edge})[\nodevariables]}  \, .
    \end{align*}
    The claim now follows with the application of \lemref{lem:difMNprob} on the first term.
\end{proof}

%\sect{Discussion}
%Representations of linear maps is the typical application of tensors, reason for referring to tensor networks as multilinear algebra.

    \chapter{\chatextbasisCalculus}\label{cha:basisCalculus}

Basis Calculus stores information in the selection of basis elements, while coordinate calculus uses the coordinates to each index for storage.
While coordinate calculus is more expressive, basis calculus can be exploited in sparse representations of composed functions.


\sect{Basis Encoding of Subsets}

Based on the concept of one-hot encodings of states we in this chapter develop the construction of encodings to sets, relations and functions.
We start with the definition of subset encodings, which represent set memberships in their boolean coordinates.

\begin{definition}[Basis encoding of subsets]
    \label{def:subsetEncoding}
    We say that an arbitrary finite set $\arbset$ is enumerated by an enumeration variable $\indvariableof{\arbset}$ taking values in $[\inddimof{\arbset}]$, when $\inddimof{\arbset}=\absof{\arbset}$ and there is a bijective index interpretation function
    \begin{align*}
        \indexinterpretation \defcols [\inddimof{\arbset}] \rightarrow \arbset \, .
    \end{align*}
    Given an set $\arbset$ enumerated by the variable $\indvariableof{\arbset}$, any subset $\arbsubset\subset\arbset$ is encoded by the tensor $\onehotmapto{\arbsubset}[\indvariable]$ defined for $\indindex\in[\absof{\arbset}]$ as
    \begin{align*}
        \onehotmapofat{\arbsubset}{\indexedindvariable}
        = \begin{cases}
              1 & \ifspace \indexinterpretationat{\indindex} \in \arbsubset \\
              0 & \text{else}
        \end{cases} \, .
    \end{align*}
\end{definition}

% Decomposition
In a one-hot basis decomposition we have
\begin{align*}
    \onehotmapofat{\arbsubset}{\indvariable}
    \coloneqq \sum_{\indindex\in[\inddimof{\arbset}]\,:\,\indexinterpretationat{\indindex}\in\arbsubset}\onehotmapofat{\indindex}{\indvariable} \, .
\end{align*}


The inclusion of subsets is represented by the partial ordering of tensors.
Let us first define this property for arbitrary tensors.

% Here for general tensors, not just propositional formulas!
\begin{definition}[Partial ordering of tensors]
    \label{def:partialOrder}
    We say that two tensors $\exformulaat{\shortcatvariables}$ and $\secexformulaat{\shortcatvariables}$ attached with the same variables are partially ordered, denoted by
    \begin{align*}
    {\exformula}
        \prec{\secexformula} \, ,
    \end{align*}
    if for all $\shortcatindices\in\facstates$
    \begin{align*}
        \exformulaat{\indexedshortcatvariables} \leq \secexformulaat{\shortcatindices}  \, .
    \end{align*}
\end{definition}

For boolean tensors, the partially ordering is equal to a subset relation of the coordinates with value $1$, as we show next.

\begin{theorem}
    \label{the:subsetRelationSubsetEncoding}
    Let $\arbset$ be an arbitrary set enumerated by the variable $\indvariable$ and index interpretation function $\indexinterpretation$.
    For two subsets $\arbsetof{0},\arbsetof{1}$ of $\arbset$ we have
    \begin{align*}
        \arbsetof{0} \subset \arbsetof{1}
    \end{align*}
    if and only if
    \begin{align*}
        \onehotmapofat{\arbsetof{0}}{\indvariable} \prec \onehotmapofat{\arbsetof{1}}{\indvariable} \, .
    \end{align*}
\end{theorem}
\begin{proof}
    We have $\arbsetof{0} \subset \arbsetof{1}$ if and only if
    \begin{align*}
        \forall{\indindex\in[\cardof{\arbset}]} \big(\indexinterpretationat{\indindex}\in\arbsetof{0}\big) \Rightarrow \big(\indexinterpretationat{\indindex}\in\arbsetof{1}\big) \, ,
    \end{align*}
    which is equal to
    \begin{align*}
        \forall{\indindex\in[\cardof{\arbset}]} \big(\onehotmapofat{\arbsetof{0}}{\indexedindvariable}=1\big) \Rightarrow \big(\onehotmapofat{\arbsetof{0}}{\indexedindvariable}=1\big)  \, .
    \end{align*}
    Since subset encodings are boolean tensors, this is equivalent to
    \begin{align*}
        \onehotmapofat{\arbsetof{0}}{\indvariable} \prec \onehotmapofat{\arbsetof{1}}{\indvariable} \, . & \qedhere
    \end{align*}
\end{proof}

\subsect{Binary Relations}

% Explanation
%Encoding of subsets as vectors: Each coordinate associated with a possible element, $\{0,1\}$ encoding whether in subset.
%The encodings is thus a boolean tensor.
%Any subset encoding is a boolean tensor.

% Relation
Since relations are subsets of cartesian products between two sets, their encoding is a straightforward generalization of \defref{def:subsetEncoding}.

\begin{definition}[Basis encoding of binary relations]
    A relation between two finite sets $\inset$ and $\outset$ is a subset of their cartesian product
    \begin{align*}
        \exrelation \subset \inset \times \outset \, .
    \end{align*}
    Given an enumeration of $\inset$ and $\outset$ by the categorical variables $\indvariableof{\insymbol}$ and $\indvariableof{\outsymbol}$ and interpretation maps $\indexinterpretationof{\insymbol}$, $\indexinterpretationof{\outsymbol}$, we define the basis encoding of this subset as the tensor $\onehotmapto{\exrelation}[\indvariableof{\insymbol},\indvariableof{\outsymbol}]$ with the coordinates
    \begin{align*}
        \onehotmapofat{\exrelation}{\indexedindvariableof{\insymbol},\indexedindvariableof{\outsymbol}}
        = \begin{cases}
              1 & \ifspace (\indexinterpretationofat{\insymbol}{\indindexof{\insymbol}},\indexinterpretationofat{\outsymbol}{\indindexof{\outsymbol}}) \in \exrelation \\
              0 & \text{else}
        \end{cases} \, .
    \end{align*}
\end{definition}

% Decomposition
The basis encoding has a decomposition into one-hot encodings as
\begin{align*}
    \onehotmapofat{\exrelation}{\indvariableof{\insymbol},\indvariableof{\outsymbol}}
    = \sum_{\indindexof{\insymbol},\indindexof{\outsymbol} \wcols (\indexinterpretationofat{\insymbol}{\indindexof{\insymbol}},\indexinterpretationofat{\outsymbol}{\indindexof{\outsymbol}}) \in \exrelation}
    \onehotmapofat{\indindexof{\insymbol}}{\indvariableof{\insymbol}}  \otimes \onehotmapofat{\indindexof{\outsymbol}}{\indvariableof{\outsymbol}}  \, .
\end{align*}

basis encodings have a matrix structure by the cartesian product, which can be further folded to tensors, when the sets itself are cartesian products.
The basis encoding is a bijection between the relations of two sets and the boolean tensors with their enumeration variables.

%They provide representations of generic relations by boolean tensors, in the sense that each relation between two sets is represented
%\begin{theorem}
%	The basis encoding is a bijection between the set of relations and the set of boolean tensors.
%\end{theorem}
%\begin{proof}
%	% =>
%	By definition, a basis encoding is the encoding of a subset and thus a boolean tensor.
%	% <=
%	Any matrification of a boolean tensor marks by its $1$ coordinates the elements of a relation.
%\end{proof}
%
%% Significance
%We can thus understand any matrification of a boolean tensor as the encoding of a relation and vice versa.



\subsect{Higher-Order Relations}

We can extend this contraction to relations of higher order, and arrive at encoding schemes usable for relational databases.

\begin{definition}[Basis encoding of $\atomorder$-ary relations]
    \label{def:daryRelation}
    Given sets $\arbsetof{\atomenumerator}$ for $\atomenumeratorin$, a $\atomorder$-ary relation is a subset of a their cartesian product, that is
    \begin{align*}
        \exrelation \subset\bigtimes_{\atomenumeratorin} \arbsetof{\atomenumerator} \, .
    \end{align*}
    Given an enumeration of each set $\arbsetof{\atomenumerator}$ by a variable $\indvariableof{\atomenumerator}$ and an interpretation map $\indexinterpretationof{\atomenumerator}$, we define the basis encoding of the relation as the tensor $\onehotmapto{\exrelation}[\indvariableof{[\atomorder]}]$ with coordinates
    \begin{align*}
        \onehotmapofat{\exrelation}{\indexedindvariableof{[\catorder]}}
        = \begin{cases}
              1 & \ifspace (\indexinterpretationofat{0}{\indindexof{0}},\ldots,\indexinterpretationofat{\atomorder-1}{\indindexof{\atomorder-1}}) \in \exrelation \\
              0 & \text{else}
        \end{cases} \, .
    \end{align*}
\end{definition}

%\begin{example}[Propositional Formulas]
Let there be for $\atomenumeratorin$ sets $\arbsetof{\atomenumerator}$ of truth assignments to the $\atomenumerator$-th atom, which are all enumerated by $[2]$.
A propositional formula then corresponds with a $\atomorder$-ary relation and we directly defined them in \defref{def:formulas} by their basis encoding.
%\end{example}

% Minterm interpretation
%If we demand $\catdimof{\atomenumerator}=2$, we can interpret the one-hot encoding of  propositional formulas.

% Knowledge Bases
\begin{theorem}
    The encoding of any $\catorder$-ary relation
    \begin{align*}
        \exrelation = \{ \shortcatindices^{\decindex} \wcols \decindexin \} \subset \bigtimes_{\catenumeratorin} \truthset \,
    \end{align*}
    where the objects in $\truthset$ are enumerated by $\catvariableof{\atomenumerator}$ with the standard index interpretation function (see \secref{sec:booleanEncoding})
    \begin{align*}
        \indexinterpretationat{\truesymbol} = 1 \andspace \indexinterpretationat{\falsesymbol} = 0 \, ,
    \end{align*}
    coincides with the propositional formula
    \begin{align*}
        \formulaat{\shortcatvariables} = \bigvee_{\decindexin} \termof{\catindex_{[\atomorder]}^{\decindex}} \, .
    \end{align*}
\end{theorem}
\begin{proof}
    By definition, the encoding $\onehotmapof{\exrelation}$ is decomposed as
    \begin{align*}
        \onehotmapofat{\exrelation}{\shortcatvariables}
        = \sum_{\decindexin} \onehotmapofat{\shortcatindices^{\decindex}}{\shortcatvariables} \, .
    \end{align*}
    By \theref{the:tensorToMaxMinTerms} this is equal to
    \begin{align*}
        \hypercoreat{\shortcatvariables} = \left( \bigvee_{\hyperonecoordinates}
        \termof{\catzeropositions}{\catonepositions}
        \right)[\shortcatvariables]
        = \bigvee_{\decindexin} \termof{\catindex_{[\atomorder]}^{\decindex}} \, . & \qedhere
    \end{align*}
\end{proof}


\begin{example}[Relational Databases]
    Relational Databases can be encoded as tensors using the relation encoding scheme.
    Each column is thereby understood as an enumeration variable, which values form the sets $\arbsetof{\catenumerator}$.
\end{example}

% Sparse Representations
Let us notice, that the dimensionality of the tensor space used for representing a relation is
\begin{align*}
    \prod_{\catenumeratorin} \cardof{\arbsetof{\catenumerator}}
\end{align*}
and therefore growing exponentially with the number of variables.
Relations are however often sparse, in the sense that
\begin{align*}
    \cardof{\exrelation} << \prod_{\catenumeratorin} \cardof{\arbsetof{\catenumerator}} \, .
\end{align*}
It is therefore often beneficially to choose sparse encoding schemes, for example by restricted CP formats (see \charef{cha:sparseRepresentation}) to represent $\onehotmapof{\exrelation}$.


\sect{Basis Encoding of Functions}

Let us now restrict to relations, which have an expression by functions.
We in this section then show, how contractions of their encodings can be exploited in function evaluation.

\subsect{Definition}

%We now generalize the representation scheme towards maps between arbitrary unstructured sets.

\begin{definition}[Basis encoding of maps]
    \label{def:functionRelationEncoding}
    Any map
    \begin{align*}
        \exfunction\defcols \inset \rightarrow \outset
    \end{align*}
    can be represented by a relation
    \begin{align*}
        \exrelationof{\exfunction} \coloneqq \left\{ \big(x,\exfunction(x)\big) \wcols x \in\inset \right\} \subset \inset \times \outset \, .
    \end{align*}
    Given a enumeration of the sets by $\indvariableof{\insymbol}$ and $\indvariableof{\outsymbol}$ we define the basis encoding of $\exfunction$ as the tensor
    \begin{align*}
        \bencodingofat{\exfunction}{\indvariableof{\outsymbol},\indvariableof{\insymbol}}
        = \onehotmapofat{\exrelationof{\exfunction}}{\indvariableof{\insymbol},\indvariableof{\outsymbol}}  \, .
    \end{align*}
\end{definition}

\begin{remark}[Reduction to images]
    % Image enumeration
    When $\exfunction$ maps into a set of infinite cardinality, we restrict $\outset$ to the image of $\exfunction$ and enumerate the image by a variable $\indvariableof{\exfunction}$.
    This scheme is applied, when $\exfunction$ is itself a tensor, i.e. $\outset=\rr$.
    While the variable $\indvariableof{\exfunction}$ can in general be of the same cardinality as the domain set $\inset$, it will be valued in $[2]$ when considering boolean tensors.
\end{remark}

% Characterization of the directed and boolean tensors
We notice that any basis representation of a function is also a directed tensor with incoming variables to the domain and outgoing variables to the image.
It furthermore holds, that the set of directed and boolean tensors is characterized by the basis encoding of functions.
This is shown in the next theorem, by the claim that any boolean tensor which is directed is the basis representation of a function.

\begin{theorem}
    \label{the:bencodingDirected}
    Let $\inset,\outset$ be sets and $\exrelation\subset\inset\times\outset$ a relation.
    If and only if there exists a map $\exfunction:\inset\rightarrow\outset$ such that $\exrelation=\exrelationof{\exfunction}$, the basis encoding $\bencodingof{\exfunction}$ is a directed tensor with $\indvariableof{\insymbol}$ incoming and $\indvariableof{\outsymbol}$ outgoing.
\end{theorem}
\begin{proof}
    \proofrightsymbol{}:
    When $\exfunction$ is a function, we have for any $\indindexofin{\insymbol}$
    \begin{align*}
        \sum_{\indindexofin{\outsymbol}} \bencodingofat{\exfunction}{\indexedindvariableof{\outsymbol},\indexedindvariableof{\insymbol}}
        =  \bencodingofat{\exfunction}{\indvariableof{\outsymbol}=\invindexinterpretationofat{\outsymbol}{\exfunctionat{\indexinterpretationofat{\insymbol}{\indindexof{\insymbol}}}},\indexedindvariableof{\insymbol}}
        = 1 \, .
    \end{align*}
    Thus, $\bencodingofat{\exfunction}{\indvariableof{\outsymbol},\indvariableof{\insymbol}}$ is a directed tensor with variables $\indvariableof{\insymbol}$ incoming and $\indvariableof{\outsymbol}$ outgoing.

    \proofleftsymbol{}:
    Conversely let there be a relation $\exrelation$, such that $\bencodingof{\exrelation}$ is directed.
    To this end, we observe that for any $\indindexofin{\insymbol}$ the tensor
    \begin{align*}
        \onehotmapofat{\exrelation}{\indexedindvariableof{\insymbol},\indvariableof{\outsymbol}}
    \end{align*}
    is a boolean tensor with coordinate sum one and therefore a basis vector.
    It follows that the function $\exfunction : \inset \rightarrow \outset $ defined for $x\in\inset$ as
    \begin{align*}
        \exfunctionat{x}
        = \indexinterpretationofat{\outsymbol}{\invonehotmapof{\onehotmapofat{\exrelation}{\indvariableof{\insymbol}=\invindexinterpretationofat{\insymbol}{x},\indvariableof{\outsymbol}}}}
    \end{align*}
    is well-defined.
    We then have by construction
    \begin{align*}
        \bencodingofat{\exfunction}{\indvariableof{\outsymbol},\indvariableof{\insymbol}}
        & = \sum_{\indindexofin{\insymbol}}
        \onehotmapofat{\exfunction(\indindexof{\insymbol})}{\indvariableof{\outsymbol}} \otimes
        \onehotmapofat{\indindexof{\insymbol}}{\indvariableof{\insymbol}} \\
        & =  \sum_{\indindexofin{\insymbol}} \onehotmapofat{\exrelation}{\indexedindvariableof{\insymbol},\indvariableof{\outsymbol}} \otimes
        \onehotmapofat{\indindexof{\insymbol}}{\indvariableof{\insymbol}} \\
        & = \onehotmapofat{\exrelation}{\indvariableof{\outsymbol},\indvariableof{\insymbol}}
    \end{align*}
    and therefore by \defref{def:functionRelationEncoding} $\exrelation=\exrelationof{\exfunction}$.
\end{proof}

% Grid sets
We are specially interested in sets of states of a factored system, which amounts to the case in \defref{def:functionRepresentation}.
Those state sets have a decomposition into a cartesian product of $\atomorder$ sets
\[ \arbset = \facstates \, . \]
The most obvious enumeration of the set $\arbset$ is therefore by the collection of state variables $\{\catvariableof{\atomenumerator}\wcols \atomenumeratorin \}$.
Functions between states of factored systems with $\atomorder_{\insymbol}$ and $\atomorder_{\outsymbol}$ state variables can be represented by $\atomorder_{\insymbol}+\atomorder_{\outsymbol}$-ary relations and \defref{def:functionRelationEncoding} has an obvious generalization to this case with multiple enumeration variables.

%% NOT NEEDED -> Done in propositional logics
%% Conditional
%Since the basis encoding of any map between factored systems is directed, it can be interpreted by a conditional probability tensor, as we state next.
%
%%% Maps
%\begin{corollary}%\label{the:condProbFunctionRepresentation}
%	The basis encoding $\bencodingof{\exfunction}$ (see \defref{def:functionRepresentation}) of a function $\exfunction$ between factored systems is a conditional probability tensor, where the legs to the image system are the conditions and the legs to the target system the distribution legs.
%\end{corollary}
%
%%% Deterministic by construction
%These are deterministic conditional probability tensors, in the sense that any slice with respect to the input variables is a basis tensor.
%Through contractions with distribution tensors (e.g. distributions in domain systems) they get stochastic.
%This is for example the case in the empirical distribution, which can be understood as the forwarding of the uniform distribution on the sample enumeration.



\subsect{Function Evaluation}

We now justify the nomenclature of basis calculus, by showing that contraction with basis elements produce the one-hot encoded function evaluation.

\begin{theorem}[Function evaluation in Basis Calculus]
    \label{the:basisCalculus}
    Retrieving the value of the function $\exfunction$ at a specific state is then the contraction of the tensor representation with the one-hot encoded state.
    For any $\arbelement\in\inset$ we have
    \begin{align*}
        \onehotmapofat{\invindexinterpretationofat{\outsymbol}{\exfunctionat{\arbelement}}}{\indvariableof{\outsymbol}}
        = \contractionof{
            \bencodingofat{\exformula}{\indvariableof{\outsymbol},\indvariableof{\insymbol}},
            \onehotmapofat{\invindexinterpretationofat{\insymbol}{\arbelement}}{\indvariableof{\insymbol}}
        }{\indvariableof{\outsymbol}} \, .
    \end{align*}
    Thus, we can retrieve the function evaluation by the inverse one-hot mapping as
    \begin{align*}
        \exfunctionat{\arbelement} = \invonehotmapof{\contractionof{
            \bencodingofat{\exformula}{\indvariableof{\outsymbol},\indvariableof{\insymbol}},
            \onehotmapofat{\invindexinterpretationofat{\insymbol}{\arbelement}}{\indvariableof{\insymbol}}
        }{\indvariableof{\outsymbol}}} \, .
    \end{align*}
\end{theorem}
\begin{proof}
    From the representation
    \begin{align*}
        \bencodingofat{\exfunction}{\indvariableof{\outsymbol},\indvariableof{\insymbol}}
        & =  \sum_{\indindexofin{\insymbol}}
        \onehotmapofat{(\invindexinterpretationof{\outsymbol} \circ \exfunction \circ \indexinterpretationof{\insymbol}) \indindexof{\insymbol}
        }{\indvariableof{\insymbol}}
        \otimes
        \onehotmapofat{\indindexof{\insymbol}}{\indvariableof{\insymbol}}
    \end{align*}
    and the orthonormality of the one-hot encodings of the input enumeration we get
    \begin{align*}
        \contractionof{
            \bencodingofat{\exformula}{\indvariableof{\outsymbol},\indvariableof{\insymbol}},
            \onehotmapofat{\invindexinterpretationofat{\insymbol}{\arbelement}}{\indvariableof{\insymbol}}
        }{\indvariableof{\outsymbol}}
        = \onehotmapofat{\invindexinterpretationofat{\outsymbol}{\exfunctionat{\arbelement}}}{\indvariableof{\outsymbol}} \, . & \qedhere
    \end{align*}
\end{proof}

% Comparsion with coordinate calculus
In comparison with the Coordinate Calculus scheme (see \theref{the:coordinateCalculus}), the Basis Calculus produces basis vectors of a functions evaluation instead of scalars.
While this seems to produce unnecessary redundancy in representing a function, we will see in the following section, that this scheme is efficient in representing compositions of functions.

\sect{Decomposition of Basis Encodings}

We now show the utility of basis encodings for functions, by developing tensor network representation to composed functions.
We in this section use the notation of factored system representation, as developed in \parref{par:one} and enumerate states of factored systems by variables $\catvariable$ with states in $[\catdim]$, instead of combinations of variables $\indvariable$ with index interpretation functions $\indexinterpretation$ enumerating arbitrary sets.

\subsect{Composition of Function}

We have already used (see \defref{def:formulaDecomposition}), that combination of propositional formulas by connectives can be represented by contractions.
We now show in a more general perspective, that in basis calculus, any composition of functions in its basis encoding the contraction of the encoded functions.

\begin{theorem}[Composition of Functions]
    \label{the:compositionByContraction}
    Let there be two maps between factored systems
    \begin{align*}
        \exfunction \defcols \nodestatesof{\nodesone} \rightarrow \nodestatesof{\nodestwo}
    \end{align*}
    and
    \begin{align*}
        \secexfunction \defcols \nodestatesof{\nodestwo} \rightarrow \nodestatesof{\nodesthree}
    \end{align*}
    with the image system of $\exfunction$ is the domain system of $\secexfunction$.
    Then the basis encoding of the composition
    \begin{align*}
        \compositionof{\secexfunction}{\exfunction}\defcols \nodestatesof{\nodesone} \rightarrow \nodestatesof{\nodesthree}
    \end{align*}
    is the contraction
    \begin{align*}
        \bencodingofat{\compositionof{\secexfunction}{\exfunction}}{\catvariableof{\nodesthree},\catvariableof{\nodesone}}
        = \contractionof{
            \bencodingofat{\secexfunction}{\catvariableof{\nodesthree},\catvariableof{\nodestwo}},
            \bencodingofat{\exfunction}{\catvariableof{\nodestwo},\catvariableof{\nodesone}},
        }{\catvariableof{\nodesthree},\catvariableof{\nodesone}} \, .
    \end{align*}
\end{theorem}
\begin{proof}
    By definition we have the basis encoding of the composition as
    \begin{align*}
        \bencodingofat{\compositionof{\secexfunction}{\exfunction}}{\catvariableof{\nodesthree},\catvariableof{\nodesone}}
        = \sum_{\catindexof{\nodesone}\in\nodestatesof{\nodesone}}
        \onehotmapofat{\compositionofat{\secexfunction}{\exfunction}{\catindexof{\nodesone}}}{\catvariableof{\nodesthree}} \otimes
        \onehotmapofat{\catindexof{\nodesone}}{\catvariableof{\nodesone}}  \, .
    \end{align*}
    By using a similar representation for $\bencodingof{\secexfunction}$ and $\bencodingof{\exfunction}$ we now show, that this coincides with the contraction of these basis encodings with closed variables $\catvariableof{\nodestwo}$.
    By the linearity of the contraction operation we get
    \begin{align*}
        \contractionof{\bencodingof{\exfunction},\bencodingof{\secexfunction}}{\catvariableof{\nodesthree},\catvariableof{\nodesone}}
        & = \sum_{\catindexof{\nodesone}\in\bigtimes_{\node\in\nodesone}[\catdimof{\node}]}
        \sum_{\catindexof{\nodestwo} \in \bigtimes_{\node\in\nodestwo}[\catdimof{\node}]}
        \breakablecontractionof{
            \left( \onehotmapofat{\secexfunctionat{\catindexof{\nodestwo}}}{\catvariableof{\nodesthree}} \otimes
            \onehotmapofat{\catindexof{\nodestwo}}{\catvariableof{\nodestwo}} \right), \\
            & \hspace{4.5cm} \left( \onehotmapofat{\exfunctionat{\catindexof{\nodesone}}}{\catvariableof{\nodestwo}} \otimes
            \onehotmapofat{\catindexof{\nodesone}}{\catvariableof{\nodesone}} \right)
        }{\catvariableof{\nodesthree},\catvariableof{\nodesone}} \\
        & = \sum_{\catindexof{\nodesone}\in\bigtimes_{\node\in\nodesone}[\catdimof{\node}]}
        \delta_{\catindexof{\nodestwo},\catindexof{\nodesone}} \, \cdot \,
        \onehotmapofat{\secexfunctionat{\catindexof{\nodestwo}}}{\catvariableof{\nodesthree}} \otimes
        \onehotmapofat{\catindexof{\nodesone}}{\catvariableof{\nodesone}} \\
        & = \sum_{\catindexof{\nodesone}\in\nodestatesof{\nodesone}}
        \onehotmapofat{\compositionofat{\secexfunction}{\exfunction}{\catindexof{\nodesone}}}{\catvariableof{\nodesthree}} \otimes
        \onehotmapofat{\catindexof{\nodesone}}{\catvariableof{\nodesone}} \\
        & = \bencodingofat{\compositionof{\secexfunction}{\exfunction}}{\catvariableof{\nodesthree},\catvariableof{\nodesone}} \, ,
    \end{align*}
    where we exploited the orthonormality of the one-hot encodings to the states of $\catvariableof{\nodestwo}$, which contraction thus results in the delta symbol $\delta$ applied on the respective states.
\end{proof}

% Iterative usage
We can use \theref{the:compositionByContraction} iteratively to further decompose the function $\secexfunction$.
In this way, the basis encoding of a function consistent of multiple compositions can be represented as the contractions of all the functions.
This has been applied in syntactical compositions (see \defref{def:formulaDecomposition}) to efficiently represent propositional formulas, for which syntactical expressions are given.

\subsect{Compositions with Real Functions}

We here investigate how the composition of a tensor
\begin{align*}
    \hypercore \defcols \facstates \rightarrow \rr
\end{align*}
with arbitrary functions
\begin{align*}
    \chainingfunction \defcols \rr \rightarrow \rr
\end{align*}
can be represented.
This is for example relevant, when representing coordinatewise tensor transforms (see \secref{sec:coordinatewiseTransforms}) based on tensor network contractions.
To this end we understand the tensor $\hypercoreat{\shortcatvariables}$ as a map of the states $\facstates$ onto its by a variable $\indvariableof{\hypercore}$ and index interpretation $\indexinterpretation$ enumerated image $\imageof{\hypercore}$.
We then define the restriction of $\chainingfunction$ onto $\imageof{\hypercore}$ as the tensor $\restrictionofto{\chainingfunction}{\imageof{\hypercore}}\left[\indvariableof{\hypercore}\right]$ with coordinates $\indindexof{\hypercore}$
\begin{align*}
    \restrictionofto{\chainingfunction}{\imageof{\hypercore}}\left[\indexedindvariableof{\hypercore}\right]
    = \compositionofat{\chainingfunction}{\indexinterpretation}{\indindexof{\hypercore}} \, .
\end{align*}
Let us now show, how contractions with these vectors represents compositions with tensors.

\begin{theorem}
    \label{the:tensorFunctionComposition}
    The coordinatewise transform of any tensor $\hypercore$ (see \defref{def:coordinatewiseTransform}) by a real function $\chainingfunction$ is the contraction (see \figref{fig:tensorFunctionComposition})
    \begin{align*}
        \chainingfunction(\hypercore)[\shortcatvariables]
        = \contractionof{\bencodingofat{\hypercore}{\indvariableof{\hypercore},\shortcatvariables},\restrictionofto{\chainingfunction}{\imageof{\hypercore}}\left[\indvariableof{\hypercore}\right] }{\shortcatvariables} \, .
    \end{align*}
\end{theorem}
\begin{proof}
    By the basis calculus \theref{the:basisCalculus} we have for any state $\shortcatindices\in\facstates$, that
    \begin{align*}
        \contractionof{\bencodingofat{\hypercore}{\indvariableof{\hypercore},\shortcatvariables},\restrictionofto{\chainingfunction}{\imageof{\hypercore}}\left[\indvariableof{\hypercore}\right]}{\indexedshortcatvariables}
        &= \contraction{\bencodingofat{\hypercore}{\indvariableof{\hypercore},\indexedshortcatvariables},\restrictionofto{\chainingfunction}{\imageof{\hypercore}}\left[\indvariableof{\hypercore}\right]} \\
        & = \contraction{\onehotmapofat{\indexinterpretationof{\hypercoreat{\indexedshortcatvariables}}}{\indvariableof{\hypercore}},\restrictionofto{\chainingfunction}{\imageof{\hypercore}}\left[\indvariableof{\hypercore}\right]} \\
        & = \chainingfunction(\hypercore)[\indexedshortcatvariables] \, .
    \end{align*}
    Since both tensors coincide on all coordinates, they are equal.
\end{proof}

\begin{figure}[t]
    \begin{center}
        \begin{tikzpicture}[scale=0.35] % , baseline = -3.5pt

\begin{scope}[shift={(-15,0)}]

\drawatomcore{-6}{-8}{$\chainingfunction\circ\hypercore$}

	\begin{scope}[shift={(-6,-12)}]
		\draw[] (0,1)--(0,-1) node[midway,left] {\tiny $\catvariableof{0}$}; 
		\draw[] (1.5,1)--(1.5,-1) node[midway,left] {\tiny $\catvariableof{1}$}; 
		\node[anchor=center] (text) at (3,0) {$\cdots$};
		\draw[] (4,1)--(4,-1) node[midway,right] {\tiny $\catvariableof{\atomorder\shortminus1}$}; 
	\end{scope}

\end{scope}

\begin{scope}[shift={(-12.5,0)}]

\node[anchor=center] (text) at (-0.5,-10) {${=}$};

\drawatomcore{3.5}{-8}{$\rencodingof{\hypercore}$}
\drawatomindices{3.5}{-12}	
\draw (5.5,-9)--(5.5,-6) node[midway,right] {\tiny $\indvariableof{\hypercore}$};
\draw[->] (5.5,-9) -- (5.5,-7.5);

\draw (3.25,-4) rectangle (7.5,-6);
\node[anchor=center] (text) at (5.5,-5) {$\restrictionofto{\chainingfunction}{\imageof{\hypercore}}$
};

\end{scope}

\end{tikzpicture}
    \end{center}
    \caption{Representation of the composition of a tensor $\hypercore$ with a real function $\chainingfunction$.}
    \label{fig:tensorFunctionComposition}
\end{figure}


\begin{corollary}
    \label{cor:rhoToNormal}
    For any tensor $\hypercoreat{\shortcatvariables}$ we have
    \begin{align*}
        \hypercoreat{\shortcatvariables}
        = \contractionof{\bencodingofat{\hypercore}{\indvariableof{\hypercore},\shortcatvariables},\idrestrictedto{\imageof{\hypercore}}\left[\indvariableof{\hypercore}\right]}{\shortcatvariables} \, .
    \end{align*}
\end{corollary}
\begin{proof}
    This follows from \theref{the:tensorFunctionComposition} using $\chainingfunction=\idsymbol$ and by noticing that
    \begin{align*}
        \hypercoreat{\shortcatvariables} = \idsymbol(\hypercore)[\shortcatvariables] \, . & \qedhere
    \end{align*}
\end{proof}

\begin{remark}[Transform of basis into coordinate encodings]
    \corref{cor:rhoToNormal} states in particular the transformation of the basis encoding of a function into its coordinate encoding.
    Given a real function $\exfunction:\facstates\rightarrow\rr$ we have
    \begin{align*}
        \cencodingofat{\exfunction}{\shortcatvariables}
        = \contractionof{\bencodingofat{\exfunction}{\headvariableof{\exfunction},\shortcatvariables},\indexinterpretationofat{\exfunction}{\headvariableof{\exfunction}}}{\shortcatvariables} \, ,
    \end{align*}
    where $\headvariableof{\exfunction}$ is a variable, which enumerates the image of $\exfunction$ with the interpretation $\indexinterpretationofat{\exfunction}{\headvariableof{\exfunction}}$.
\end{remark}


\begin{corollary}
    \label{cor:onesHead}
    For any tensor $\hypercore$, which is directed with $\shortcatvariables$ incoming, we have
    \[ \onesat{\shortcatvariables} = \contractionof{\bencodingof{\hypercore}}{\shortcatvariables} \, . \]
\end{corollary}
\begin{proof}
    This follows from \theref{the:tensorFunctionComposition} using $\chainingfunction=\ones$ and by noticing that
    \begin{align*}
        \onesat{\shortcatvariables} = \ones(\hypercore)[\shortcatvariables] \, . & \qedhere
    \end{align*}
\end{proof}


%% COULD STATE SLICING THEOREM AS A COMPOSITION OF CHAININGS! But unclear, wheter needed
%\begin{theorem}
%	\begin{align*}
%		\coordinatetrafowrtofat{\chainingfunction}{\contractionof{\exvector[\indvariableof{\exfunction}],\bencodingofat{\hypercore}{\indvariableof{\exfunction},\shortcatvariables}}{\shortcatvariables}}{\shortcatvariables}
%		= \coordinatetrafowrtofat{\left(\coordinatetrafowrtofat{\chainingfunction}{\exvector}{\indvariableof{\exfunction}}\right)}{\hypercore}{\shortcatvariables}
%	\end{align*}
%\end{theorem}
%\begin{proof}
%	Simply by compositions of transforms.
%\end{proof}
%
%
%% Replacement of Slicing Theorem
%\begin{corollary}\label{cor:directedTrafo}
%	Let $\basisslices$ be a directed and boolean tensor with incoming variables being $\shortcatvariables$, and $\gentensor$ a tensor, which variables are the outgoing variables of $\basisslices$.
%	Let further $\chainingfunction:\rr\rightarrow\rr$ be any real function.
%	Then
%		\[ \chainingfunction \circ \contractionof{\basisslices,\gentensor}{\shortcatvariables}
%		= \contractionof{\basisslices,\chainingfunction\circ\gentensor}{\shortcatvariables} \, . \]
%\end{corollary}
%\begin{proof}
%	Since $\basisslices$ is a directed and boolean tensor, we find a map
%		\[ \exfunction\defcols \facstates \rightarrow \secfacstates \]
%	such that $\basisslices=\bencodingof{\exfunction}$ and a map $V$ such that $\gentensor=\restrictionofto{V}{\imageof{\exfunction}}$.
%	Then \theref{the:tensorFunctionComposition} implies that
%		\[ \contractionof{\basisslices,\gentensor}{\shortcatvariables} = V \circ \exfunction \, . \]
%	It follows that
%	\begin{align*}
%		\chainingfunction \circ \contractionof{\basisslices,\gentensor}{\shortcatvariables} = \chainingfunction \circ V \circ \exfunction
%	\end{align*}
%	and by another application of \theref{the:tensorFunctionComposition} that
%	\begin{align*}
%		\chainingfunction \circ V \circ \exfunction
%		& = \contractionof{\bencodingof{\exfunction}, \restrictionofto{\chainingfunction \circ V}{\imageof{\exfunction}}}{\shortcatvariables} \\
%		& = \contractionof{\basisslices,\chainingfunction\circ\gentensor}{\shortcatvariables} \, .
%	\end{align*}
%	The claim follows as a combination of both equations.
%\end{proof}





\subsect{Decomposition in Case of Structured Images}

When a set is structured as the cartesian product of other sets, that is
\begin{align*}
    \outset = \bigtimes_{\catenumeratorin} \arbsetof{\catenumerator} \, ,
\end{align*}
we can enumerate it by a collection $\{\indvariableof{\catenumerator} \wcols \catenumeratorin\}$ of enumeration variables, each with respective index interpretation maps.
When the image of a function admits such a cartesian representation, we now show that the basis encoding can be represented by a contraction of basis encodings to each image coordinate.

\begin{theorem}
    \label{the:functionImageDecompositionContraction}
    Let $\exfunction$ be a function between factored systems
    \begin{align*}
        \exfunction\defcols [\catdim] \rightarrow  \facstates
    \end{align*}
    and denote by
    \begin{align*}
        \exfunctionof{\catenumerator}\defcols [\catdim] \rightarrow [\catdimof{\catenumerator}]
    \end{align*}
    the image coordinate restrictions of $\exfunction$, that is we have $\exfunction=(\exfunctionof{0},\ldots,\exfunctionof{\catorder-1})$.
    Let us assign the variable $\catvariable$ to the factored system in the domain system of $\exfunction$ and the variables $\catvariableof{\atomenumerator}$ for $\atomenumeratorin$ to the image system of $\exfunction$.
    We can then decompose the basis encoding of $\exfunction$ into the basis encodings of its image coordinate restrictions, that is
    \begin{align*}
        \bencodingofat{\exfunction}{\shortcatvariables,\catvariable}
        = \contractionof{
            \{\bencodingofat{\exfunctionof{\atomenumerator}}{\catvariableof{\atomenumerator},\catvariable} \wcols \atomenumeratorin \}
        }{\shortcatvariables,\catvariable} \, .
    \end{align*}
\end{theorem}
\begin{proof}
    For any $\catindexin$ we have
    \begin{align*}
        \bencodingofat{\exfunction}{\shortcatvariables,\indexedcatvariable}
        &= \onehotmapofat{\exfunctionat{\catindex}}{\shortcatvariables} \\
        &= \bigotimes_{\atomenumeratorin} \bencodingofat{\exfunctionof{\atomenumerator}}{\catvariableof{\atomenumerator},\indexedcatvariable} \\
        &= \contractionof{
            \{\bencodingofat{\exfunctionof{\atomenumerator}}{\catvariableof{\atomenumerator},\indexedcatvariable} \wcols \atomenumeratorin\}
        }{\shortcatvariables} \\
        &= \contractionof{
            \{\bencodingofat{\exfunctionof{\atomenumerator}}{\catvariableof{\atomenumerator},\catvariable} \wcols \atomenumeratorin\}
        }{\shortcatvariables,\indexedcatvariable}
    \end{align*}
    and therefore equality of both tensors.
\end{proof}

% Continue discussion in Sparse TC
In \charef{cha:sparseRepresentation} we will apply \theref{the:functionImageDecompositionContraction} in \theref{the:functionDecompositionBasisCP} to show sparse basis $\cpformat$ decompositions to $\bencodingof{\exfunction}$.
These decompositions are then applied for efficient representation of the empirical distribution, which involve the basis encoding of data maps (see \exaref{exa:empDistCP}), and for exponential families, which statistics have images, which are included in cartesian products of the images to each coordinate (see \exaref{exa:expFamCP}).



\sect{Selection Encodings}

Selection encodings as introduced in \defref{def:selectionEncoding} are best understood in terms of linear mapping interpretations of tensors.
We will first provide by basis representations a generic relation between the coordinatewise tensor definitions in this work and linear maps.

We then show the utility of this perspective in the representation of composed linear functions.
The results are applicable in the exponential family theory, in the tensor representation of energies and means.

\subsect{Basis Representations of Linear Maps}

% Matrices
Basis representations are standard linear algebra tools, where matrices are understood as linear maps between vector spaces.
The state sets $\facstates$ can be interpreted as an enumeration of basis elements $\onehotmapof{\catindex}$ of the tensor space $\facspace$.
Along this interpretation, tensors have an interpretation as maps between tensor spaces.
Any tensor and any partition of its variables into two sets can be interpreted as the basis elements of a linear map between the tensor spaces of the respective variables.
Tensor valued functions on state sets $\facstates$ are an intermediate representation.

\begin{definition}
    Let there be two tensor spaces $V_1$ and $V_2$ with basis by sets $\arbsetof{1}\subset V_1$ and $\arbsetof{2}\subset V_2$ of cardinality $\inddimof{1}$ and $\inddimof{2}$, which are enumerated by variables $\indvariableof{1},\indvariableof{2}$ and index interpretation functions $\indexinterpretationof{1},\indexinterpretationof{2}$.
    The basis representation of any linear map $\linmap\in\linmapspace(V_1,V_2)$ is then the tensor
    \begin{align*}
        \brepresentationofat{\exfunction}{\indvariableof{1},\indvariableof{2}} \in \rr^{\inddimof{1}} \otimes \rr^{\inddimof{2}}
    \end{align*}
    defined for $\indindexofin{1}$ and $\indindexofin{2}$ by
    \begin{align*}
        \brepresentationofat{\linmap}{\indexedindvariableof{1},\indexedindvariableof{2}}
        = \contraction{\linmapof{\indexinterpretationofat{1}{\indindexof{1}}},\indexinterpretationofat{2}{\indindexof{2}}} \, .
    \end{align*}
\end{definition}

Basis representations for compositions of linear functions can be computed via contractions of the respective basis representations, as we show next.

\begin{theorem}
    \label{the:linearCompositionBasisEncoding}
    If $\linmapof{1}$ is a linear function between $V_1$ and $V_2$  and $\linmapof{2}$ between $V_2$ and $V_3$, and let $\indvariableof{1},\,\indvariableof{2}$ and $\indvariableof{3}$ be enumerations of orthonormal bases in the spaces with index interpretation functions $\indexinterpretationof{1},\,\indexinterpretationof{2}$ and $\indexinterpretationof{3}$.
    We have
    \begin{align*}
        \brepresentationofat{\linmapof{2}\circ\linmapof{1}}{\indvariableof{1},\indvariableof{3}}
        = \contractionof{
            \brepresentationofat{\linmapof{2}}{\indvariableof{2},\indvariableof{3}}, \brepresentationofat{\linmapof{1}}{\indvariableof{1},\indvariableof{2}}
        }{\indvariableof{1},\indvariableof{3}}  \, .
    \end{align*}
\end{theorem}
\begin{proof}
    For arbitrary $\indindexofin{1}$ and $\indindexofin{3}$ we have to show that
    \begin{align*}
        \brepresentationofat{\linmapof{2}\circ\linmapof{1}}{\indexedindvariableof{1},\indexedindvariableof{3}}
        = \contraction{
            \brepresentationofat{\linmapof{2}}{\indvariableof{2},\indexedindvariableof{3}},\brepresentationofat{\linmapof{1}}{\indexedindvariableof{1},\indvariableof{2}}
        } \, .
    \end{align*}
    By definition we have
    \begin{align*}
        \brepresentationofat{\linmapof{2}\circ\linmapof{1}}{\indexedindvariableof{1},\indexedindvariableof{3}}
        = \contraction{\linmapof{2}\circ\linmapof{1}(\indexinterpretationofat{1}{\indindexof{1}}),\indexinterpretationofat{3}{\indindexof{3}}} \, .
    \end{align*}
    Decomposing the linear maps using their basis representation we get
    \begin{align*}
        \contraction{\linmapof{2}\circ\linmapof{1}(\indexinterpretationofat{1}{\indindexof{1}}),\indexinterpretationofat{3}{\indindexof{3}}}
        &= \contraction{\linmapofat{2}{\sum_{\indindexofin{2}} \brepresentationofat{\linmapof{1}}{\indexedindvariableof{1},\indexedindvariableof{2}} \cdot \indexinterpretationofat{2}{\indindexof{2}}},\indexinterpretationofat{3}{\indindexof{3}}} \\
        &= \sum_{\indindexofin{2}} \contraction{\linmapofat{2}{\brepresentationofat{\linmapof{1}}{\indexedindvariableof{1},\indexedindvariableof{2}} \cdot \indexinterpretationofat{2}{\indindexof{2}}},\indexinterpretationofat{3}{\indindexof{3}}} \\
        &= \sum_{\indindexofin{2}} \contraction{\brepresentationofat{\linmapof{1}}{\indexedindvariableof{1},\indexedindvariableof{2}} \cdot \brepresentationofat{\linmapof{2}}{\indexedindvariableof{2},\indexedindvariableof{1}}} \\
        &= \contraction{
            \brepresentationofat{\linmapof{2}}{\indvariableof{2},\indexedindvariableof{3}},\brepresentationofat{\linmapof{1}}{\indexedindvariableof{1},\indvariableof{2}}
        } \, .
    \end{align*}
    Therefore, both tensors are equivalent.
\end{proof}

% Comparison with basis representations
For basis representations we thus have a similar composition theorem as for basis encodings of arbitrary functions (see \theref{the:compositionByContraction}).
What is more, one can understand each basis encodings as a basis representation of a linear function.
Along this line, the composition theorem \theref{the:linearCompositionBasisEncoding} as the principle of linear algebra, which underlies \theref{the:compositionByContraction}.
% Matrix Multiplication
A typical interpretation of \theref{the:linearCompositionBasisEncoding} is matrix multiplication, where matrices understood since matrices are basis representations of linear maps.

\begin{example}[Basis encodings as basis representations]
    Let us justify that we referred to the contractions of basis encodings by basis calculus, by describing that basis encodings are a special case of basis representations.
    To that end, we understand the sets $\inset=[\inddimof{\insymbol}]$ and $\outset=[\inddimof{\outsymbol}]$ as labels of a basis in $\rr^{\inddimof{\insymbol}}$ and $\rr^{\inddimof{\outsymbol}}$.
    Then, given a relation $\exrelation \subset \inset \times \outset$, we define a linear map $\linmap : \rr^{\inddimof{\insymbol}} \rightarrow \rr^{\inddimof{\outsymbol}}$ through the action on the $i\in[\inddimof{\insymbol}]$-th basis element as
    \begin{align*}
        \linmap(\onehotmapof{i}) = \sum_{j\in[\inddimof{\outsymbol}]\wcols (i,j) \in \exrelation} \onehotmapof{j} \, .
    \end{align*}
    Comparing the coefficients of the basis representation of $\linmap$ and the basis encoding $\exrelation$ we get
    \begin{align*}
        \brepresentationofat{\linmap}{\indexedindvariableof{\outsymbol},\indexedindvariableof{\insymbol}}
        = \onehotmapofat{\exrelation}{\indexedindvariableof{\outsymbol},\indexedindvariableof{\insymbol}} \, .
    \end{align*}
\end{example}



\subsect{Selection Encodings as Basis Representations}

Selection encodings (see \defref{def:selectionEncoding}) are related to basis representations of linear maps as we show in the next theorem.

\begin{theorem}
    \label{the:selectionToBasisEncoding}
    Let there be tensor spaces $\facstates$ and $\selspace$ with basis by the one-hot encodings, enumerated by the categorical variables $\shortcatvariables$ and $\shortselvariables$ with index interpretation functions by the one-hot map $\onehotmap$.
    Given a function
    \begin{align*}
        \exfunction\defcols \facstates \rightarrow \selspace
    \end{align*}
    we define a linear map $\linmapof{\exfunction}\in\linmapspace(\facspace,\selspace)$ by the action on the basis elements to $\shortcatindices\in\facstates$ as the tensors %and $\catindexof{2}\in\selstates$ by
    \begin{align*}
        \linmapof{\exfunction}(\onehotmapof{\shortcatindices}) \coloneqq \exfunctionat{\shortcatindices} \,
    \end{align*}
    carrying the variables $\shortselvariables$.
    We then have
    \begin{align*}
        \sencodingofat{\exfunction}{\shortcatvariables,\shortselvariables}
        = \brepresentationofat{\linmapof{\exfunction}}{\shortcatvariables,\shortselvariables} \, .
    \end{align*}
\end{theorem}
\begin{proof}
    We show equality on each slice with respect to the variables $\shortcatvariables$ and therefore choose arbitrary $\shortcatindices$. % and $\shortselindices$.
    It holds by definition of selection encodings and the map $\linmapof{\exfunction}$ that
    \begin{align*}
        \sencodingofat{\exfunction}{\indexedshortcatvariables,\selvariables}
        = \exfunctionat{\shortcatindices}[\shortselvariables]
        = \linmapof{\exfunction}(\onehotmapof{\shortcatindices})[\shortselvariables] \, .
    \end{align*}
    We further have
    \begin{align*}
        \linmapof{\exfunction}(\onehotmapof{\shortcatindices})[\shortselvariables]
        &= \sum_{\shortselindices} \contraction{\linmapof{\exfunction}(\onehotmapof{\shortcatindices})[\indexedshortselvariables],\onehotmapofat{\shortselindices}{\shortselvariables}} \cdot \onehotmapofat{\shortselindices}{\shortselvariables} \\
        &= \sum_{\shortselindices} \brepresentationofat{\linmapof{\exfunction}}{\indexedshortcatvariables,\indexedshortselvariables} \cdot \onehotmapofat{\shortselindices}{\shortselvariables} \\
        &= \brepresentationofat{\linmapof{\exfunction}}{\indexedshortcatvariables,\shortselvariables} \, .
    \end{align*}
    For arbitrary $\shortcatindices$ the slices of $\sencodingof{\exfunction}$ and $\brepresentationof{\linmapof{\exfunction}}$ thus coincide, which proves the equivalence of both tensors.
\end{proof}

% Comparison with basis encodings - definition
While basis encoding works for maps from $\facstates$ to arbitrary sets (which are enumerated), selection encodings as introduced in \defref{def:selectionEncoding} require and exploit that their image is embedded in a tensor space.

% Slicing
Given a selection encoding of a function, the function is retrieved by slicing with respect to the
\begin{align*}
    \exfunction(\catindex) = \sencodingofat{\exfunction}{\indexedcatvariable,\selvariable} \, .
\end{align*}
More generally, we show in the next Lemma how to construct to any tensor and any partition of its variables functions by slicing operations, such that the tensor is the selection encoding of the function.

\begin{theorem}
    \label{the:inverseSelectionEncoding} % To be used for MLN - proposal distribution
    Let $\hypercoreat{\nodevariables}$ be a tensor in $\bigotimes_{\nodein}\rr^{\catdimof{\node}}$ and let $\nodesa$, $\nodesb$ be a disjoint partition of $\nodes$, that is $\nodesa\dot{\cup}\nodesb=\nodes$.
    Then the function
    \begin{align*}
        \exfunction\defcols \bigtimes_{\node\in\nodesa}[\catdimof{\node}] \rightarrow \bigotimes_{\node\in\nodesb} \rr^{\catdimof{\node}}
    \end{align*}
    defined for $\catindexof{\nodesa}\in\nodestatesof{\nodesa}$ as
    \begin{align*}
        \exfunctionat{\catindexof{\nodesa}} \coloneqq \hypercoreat{\indexedcatvariableof{\nodesa},\catvariableof{\nodesb}}
    \end{align*}
    obeys
    \begin{align*}
        \sencodingofat{\exfunction}{\catvariableof{\nodesa},\catvariableof{\nodesb}} = \hypercoreat{\nodevariables} \, ,
    \end{align*}
    where we understand the variables $\catvariableof{\nodesb}$ as selection variables.
\end{theorem}
\begin{proof}
    We have for any $\catindexof{\nodesb}$ that
    \begin{align*}
        \sencodingofat{\exfunction}{\catvariableof{\nodesa},\indexedcatvariableof{\nodesb}}
        &= \sum_{\catindexof{\nodesa}\in\nodestatesof{\nodesa}} \onehotmapofat{\catindexof{\nodesa}}{\catvariableof{\nodesa}}
        \otimes \exfunction(\catindexof{\nodesa})[\indexedcatvariableof{\nodesb}] \\
        &= \sum_{\catindexof{\nodesa}\in\nodestatesof{\nodesa}} \onehotmapofat{\catindexof{\nodesa}}{\catvariableof{\nodesa}}
        \otimes \hypercoreat{\indexedcatvariableof{\nodesa},\indexedcatvariableof{\nodesb}} \\
        &= \hypercoreat{\catvariableof{\nodesa},\indexedcatvariableof{\nodesb}}
    \end{align*}
    and the equivalence follows.
%    From \theref{the:linearCompositionBasisEncoding} using the basis representation equivalence of \theref{the:selectionToBasisEncoding}.
\end{proof}


\begin{example}[\MarkovLogicNetworks{} and Proposal Distributions]
    % Via inverse selection encodings
    While the statistic of MLN (namely $\fselectionmap$) and the proposal distribution (namely $\tranfselectionmap$) have a common selection encoding, both result from the inverse selection encoding described in \theref{the:inverseSelectionEncoding}.
    We can construct $\tranfselectionmap$ by first building the selection encoding to $\fselectionmap$ and then applying the construction of \theref{the:inverseSelectionEncoding} with $\nodesa=\selvariable$ and $\nodesb=\shortcatvariables$.
\end{example}


% Composition
We use selection encodings to represent weighted sums of functions, based on the next theorem.

\begin{theorem}[Weighted formula sums by selection encodings]
    \label{the:linCompSelEncoding}
    Let $\sstat$ be a tensor valued function from $\facstates$ to $\parspace$ with image coordinates $\sstatcoordinate$ and let $\canparamat{\selvariable}$ be a tensor.
    Then
    \[ \left(\sum_{\selindexin}\canparamat{\indexedselvariable} \cdot \sstatcoordinate \right) [\shortcatvariables] \defcols \facstates \rightarrow \rr \]
    is represented as
    \[ \left(\sum_{\selindexin}\canparamat{\indexedselvariable}\cdot \sstatcoordinate \right) [\shortcatvariables]
    = \contractionof{\sencodingofat{\sstat}{\shortcatvariables,\selvariable} , \canparamat{\selvariable}}{\shortcatvariables} \, . \]
\end{theorem}
\begin{proof}
    The representation holds, since for any $\shortcatindicesin$ we have
    \begin{align*}
        \contractionof{\sencodingofat{\sstat}{\shortcatvariables,\selvariable},\canparamat{\selvariable}}{\indexedshortcatvariables}
        = \sum_{\selindexin}\exfunctionat{\indexedselvariable}\cdot\sstatcoordinateofat{\selindex}{\indexedshortcatvariables} \, . & \qedhere
    \end{align*}
\end{proof}

% Linear
This theorem shows, that while relation encodings can represent any composition with another function by a contractions, selection encodings can be used to represent linear transforms.
To see this, we interpret $\sstat$ and $\exfunction$ in \theref{the:linCompSelEncoding} as basis decompositions of linear maps.


\sect{Indicator Features to Functions}\label{sec:indicatorFeatures}

We here provide a subspace perspective for the sparse representation of decomposable functions as tensor networks in the $\bencodingof{\cdot}$ basis encoding scheme of basis calculus.

\begin{definition}
    Given any function $\exfunction:\inset\rightarrow\outset$ and index interpretation functions $\indexinterpretationof{\insymbol},\indexinterpretationof{\outsymbol}$ enumerating $\inset$ and $\outset$ we define the corresponding indicator subspace of $\rr^{\cardof{\inset}}$ as
    \begin{align*}
        \subspaceof{\exfunction}
        = \left\{ \contractionof{
            \bencodingofat{\exfunction}{\indvariableof{\outsymbol},\indvariableof{\insymbol}},\acttensorat{\indvariableof{\outsymbol}}
        }{\indvariableof{\outsymbol},\indvariableof{\insymbol}} \wcols \acttensorat{\indvariableof{\outsymbol}} \in \rr^{\inddimof{\outsymbol}}
        \right\} \, .
    \end{align*}
\end{definition}

% Trivial tensor included
For any function $\exfunction$ we have $\onesat{\shortcatvariables}\in\subspaceof{\exfunction}$, when choosing $\acttensorat{\indvariableof{\outsymbol}}=\onesat{\indvariableof{\outsymbol}}$.

% Indicator features to functions
We now characterize the indicator subspace as a span of boolean indicator features, indicating whether the function value coincides with an element $\exfunctionimageelement\in\imageof{\exfunction}$.
Each indicator feature is a tensor $\indicatorofat{\exfunction=\exfunctionimageelement}{\indvariableof{\insymbol}}$ with coordinates
\begin{align*}
    \indicatorofat{\exfunction=\exfunctionimageelement}{\indexedindvariableof{\insymbol}}
    = \begin{cases}
          1 & \ifspace \exfunction(\indexinterpretationof{\insymbol}(\indindexof{\insymbol}))=\exfunctionimageelement \\
          0 & \text{else}
    \end{cases} \, .
\end{align*}

\begin{lemma}
    A value subspace is spanned by the boolean indicator features of the underlying function, that is
    \begin{align*}
        \subspaceof{\exfunction}
        = \spanof{\indicatorofat{\exfunction=\exfunctionimageelement}{\indvariableof{\insymbol}} \wcols \exfunctionimageelement\in\imageof{\exfunction}} \, .
    \end{align*}
\end{lemma}
\begin{proof}
    By linearity of contractions we have
    \begin{align*}
        \subspaceof{\exfunction} = \spanof{
            \contractionof{\bencodingofat{\exfunction}{\indvariableof{\outsymbol},\indvariableof{\insymbol}}}, \onehotmapofat{\indindexof{\outsymbol}}{\indvariableof{\outsymbol}}
            \wcols \indindexof{\outsymbol} \in [\inddimof{\outsymbol}]
        } \, .
    \end{align*}
    It further holds that
    \begin{align*}
        \indicatorofat{\exfunction=\indexinterpretationof{\indindexof{\outsymbol}}}{\indvariableof{\insymbol}}
        = \contractionof{\bencodingofat{\exfunction}{\indvariableof{\outsymbol},\indvariableof{\insymbol}}}, \onehotmapofat{\indindexof{\outsymbol}}{\indvariableof{\outsymbol}}
    \end{align*} \, .
    The claim follows as a combination of both equations.
\end{proof}





\subsect{Connections with Computable Families}

% Application on exponential families
We now apply indicator spaces to provide further intuition into computable families (see \defref{def:realizableStatDistributions}).


\begin{lemma}
    For any statistic $\sstat : \facstates\rightarrow\parspace$ we have
    \begin{align*}
        \realizabledistsof{\sstat,\maxgraph} = \subspaceof{\sstat} \cup \subsphere
    \end{align*}
\end{lemma}
\begin{proof}
    For any non-negative $\probwith$ we have $\probwith\in\realizabledistsof{\sstat,\maxgraph}$ if and only if $\contraction{\probwith}=1$ and there is an activation core $\acttensorat{\headvariables}$ with
    \begin{align*}
        \probwith = \contractionof{\bencodingofat{\sstat}{\headvariables}}{\shortcatvariables} \, .
    \end{align*}
    This is equivalent to $\probwith\in\subspaceof{\sstat} \cup \subsphere$.
\end{proof}

In the other extreme of tensor network formats for the activation tensor, we have the elementary graph $\elgraph$.
To provide a characterization of $\realizabledistsof{\sstat,\elgraph}$, we understand any statistic as a cartesian product of its features $\sstatcoordinateof{\selindex}$.
For cartesian products we can show that the distributions realizable with elementary activation tensors coincide with the subspace contraction of the spaces $\subspaceof{\sstatcoordinateof{\selindex}}$ to be introduced next.

\begin{definition}
    Given two subspaces $\subspaceof{1},\subspaceof{2}$ of tensors with variables $\catvariableof{\nodesone}$, $\catvariableof{\nodestwo}$, their contraction is
    \begin{align*}
        \contractionof{\subspaceof{1},\subspaceof{2}}{\catvariableof{\secnodes}}
        = \left\{ \contractionof{\hypercoreofat{1}{\catvariableof{\nodesone}},\hypercoreofat{2}{\catvariableof{\nodestwo}}}{\catvariableof{\secnodes}}
        \wcols \hypercoreofat{1}{\catvariableof{\nodesone}}\in\subspaceof{1}, \, \hypercoreofat{2}{\catvariableof{\nodestwo}}\in\subspaceof{2} \right\} \, .
    \end{align*}
\end{definition}

We notice that the contraction of two subspaces is in general not a subspace.
This fact will in the following become clearer, where we characterize contractions of subspaces with elementarily computable distributions, which are known to not be linear subspaces.
%Elementary tensors are here known to not form a subspace.

%in the characterization of distributions for which a statistic $\sstat$ is sufficient.
The cartesian product of functions on the same input set $\inset$ and output sets $\arbsetof{1,\outsymbol}$, $\arbsetof{2,\outsymbol}$ is the function
\begin{align*}
(\exfunction,\secexfunction)
   \defcols \inset \rightarrow \arbsetof{1,\outsymbol} \times \arbsetof{2,\outsymbol}
\end{align*}
with
\begin{align*}
(\exfunction,\secexfunction)(z)
    = (\exfunctionat{z},\secexfunctionat{z}) \, .
\end{align*}
Its indicator subspace is the contraction of indicator subspaces
\begin{align*}
    \contractionof{\subspaceof{\exfunction},\subspaceof{\secexfunction}}{\indvariableof{\insymbol}}
    & = \contractionof{
        \spanof{\{\indicatorofat{\exfunction=\exfunctionimageelement}{\indvariableof{\insymbol}} \wcols \exfunctionimageelement\in\imageof{\exfunction}\}},
        \spanof{\{\indicatorofat{\secexfunction=\exfunctionimageelement}{\indvariableof{\insymbol}} \wcols \exfunctionimageelement\in\imageof{\secexfunction}\}}
    }{\indvariableof{\insymbol}} \\
    & = \realizabledistsof{\{\exfunction, \secexfunction\},\elgraph} \, .
\end{align*}
We generalize this in the next lemma to cartesian products of multiple features using that any statistic $\sstat$ is the cartesian product of its features $\sstatcoordinateof{\selindex}$.

\begin{lemma}
    For any statistic $\sstat:\facstates\rightarrow\parspace$ we have
    \begin{align*}
        \realizabledistsof{\sstat,\elgraph} = \contractionof{\subspaceof{\sstatcoordinateof{\selindex}} \wcols \selindexin}{\shortcatvariables}  \cup \subsphere \, .
    \end{align*}
    where $\subsphere$ is the sphere of normalized tensors in $\facspace$.
\end{lemma}
\begin{proof}
    For any non-negative tensor $\probwith$ we have $\probwith\in\realizabledistsof{\sstat,\elgraph}$ if and only if
    \begin{align*}
        \contraction{\probwith}=1
    \end{align*}
    and there exist activation cores $\acttensorofat{\selindex}{\headvariableof{\selindex}}$ such that
    \begin{align*}
        \probwith
        = \contractionof{\bigcup_{\selindexin}\{\bencodingofat{\sstatcoordinateof{\selindex}}{\headvariableof{\selindex},\shortcatvariables},\acttensorofat{\selindex}{\headvariableof{\selindex}}\}}{\shortcatvariables} \, .
    \end{align*}
    Since for any $\selindexin$ we have
    \begin{align*}
        \left\{\contractionof{\bencodingofat{\sstatcoordinateof{\selindex}}{\headvariableof{\selindex},\shortcatvariables},\acttensorofat{\selindex}{\headvariableof{\selindex}}}{\shortcatvariables} \wcols
        \acttensorofat{\selindex}{\headvariableof{\selindex}} \in \rr^{\headdimof{\selindex}}
        \right\} = \subspaceof{\sstatcoordinateof{\selindex}}
    \end{align*}
    $\probwith\in\realizabledistsof{\sstat,\elgraph}$ is equal to
    \begin{align*}
        \probwith \in \contractionof{\subspaceof{\sstatcoordinateof{\selindex}}\wcols\selindexin}{\shortcatvariables} \, . & \qedhere
    \end{align*}
\end{proof}


Since always $\onesat{\shortcatvariables}\in\subspaceof{\exfunction}$, we have for any subspace $\subspaceof{1}$
\begin{align*}
    \contractionof{\subspaceof{1}}{\catvariableof{\secnodes}}
    \subset \contractionof{\subspaceof{1},\subspaceof{\exfunction}}{\catvariableof{\secnodes}} \, .
\end{align*}
We can therefore understand the addition of a feature to a set of feature as a monotonously increasing set of elementarily computable distributions, that is
\begin{align*}
    \realizabledistsof{\sstat,\elgraph} \subset \realizabledistsof{\sstat\cup\{\exfunction\},\elgraph} \, .
\end{align*}


We now relate the by a function $\exfunction$ computable distributions with those, which are computable by its indicator features with elementary activation cores.

\begin{lemma}
    \label{lem:computableByFunctionEqualElementaryIndicators}
    For any function $\exfunction:\facstates\rightarrow\outset$ we have
    \begin{align*}
        \realizabledistsof{\exfunction,\maxgraph}
        = \realizabledistsof{\{\indicatorof{\exfunction=\exfunctionimageelement}\wcols\exfunctionimageelement\in\imageof{\exfunction}\},\elgraph}
    \end{align*}
\end{lemma}
\begin{proof}
    "$\subset$":
    For any $\probwith\in\realizabledistsof{\exfunction,\maxgraph}$ we find an activation core $\acttensorat{\headvariable}$ such that
    \begin{align*}
        \probwith = \contractionof{\bencodingofat{\exfunction}{\headvariable,\shortcatvariables},\acttensorat{\headvariable}}{\shortcatvariables} \, .
    \end{align*}
    Given this activation tensor $\acttensorat{\headvariable}$ we construct an elementary activation tensor
    \begin{align*}
        \bigotimes_{\selindexin}\acttensorofat{\selindex}{\headvariableof{\selindex}}
    \end{align*}
    which reproduces $\probwith$ by contraction with the basis encoding of the indicator features to $\exfunction$. %$\contractionof{}$
    To this end, we define for $\selindexin$ two-dimensional leg vectors
    \begin{align*}
        \acttensorofat{\selindex}{\headvariableof{\selindex}}
        = \begin{bmatrix}
              \acttensorat{\headvariable=\selindex} \\
              1
        \end{bmatrix}[\headvariableof{\selindex}] \, .
    \end{align*}
    For these head variables we have for any index tuple $\shortcatindices$
    \begin{align*}
        \contractionof{\bigcup_{\selindexin}\{\bencodingofat{\indicatorof{\exfunction=\indexinterpretationof{\selindex}}}{\headvariableof{\selindex}},\acttensorat{\headvariable=\selindex}\}}{\indexedshortcatvariables}
        & = \acttensorofat{\exfunctionat{\shortcatindices}}{\headvariableof{\selindex}=1} \cdot \prod_{\selindexin \wcols \exfunctionat{\shortcatindices} \neq \selindex} \acttensorofat{\exfunctionat{\shortcatindices}}{\headvariableof{\selindex}=0} \\
        & = \acttensorofat{\exfunctionat{\shortcatindices}}{\headvariableof{\selindex}=1} \\
        & = \contractionof{\bencodingofat{\exfunction}{\headvariable,\shortcatvariables},\acttensorat{\headvariable}}{\indexedshortcatvariables} \\
        & = \probat{\indexedshortcatvariables} \, .
    \end{align*}

    "$\supset$":
    Conversely, given $\probwith\in\realizabledistsof{\{\indicatorof{\exfunction=\exfunctionimageelement}\wcols\exfunctionimageelement\in\imageof{\exfunction}\},\elgraph}$ we find an elementary activation tensor
    \begin{align*}
        \bigotimes_{\selindexin}\acttensorofat{\selindex}{\headvariableof{\selindex}}
    \end{align*}
    such that
    \begin{align*}
        \probwith = \contractionof{\bigcup_{\selindexin}\{\bencodingofat{\indicatorof{\exfunction=\indexinterpretationof{\selindex}}}{\headvariableof{\selindex}},\acttensorat{\headvariable=\selindex}\}}{\shortcatvariables}
    \end{align*}
    If $\acttensorofat{\selindex}{\headvariableof{\selindex}=0}=0$ for an $\selindex$, then we have $\probwith=\normalizationof{\indicatorofat{\exfunction=\indexinterpretationof{\selindex}}{\shortcatvariables}}{\shortcatvariables}$, which is an element of $\realizabledistsof{\exfunction,\maxgraph}$ since then
    \begin{align*}
        \probwith = \normalizationof{\bencodingof{\exfunction}{\headvariable,\shortcatvariables},\onehotmapofat{\selindex}{\headvariable}}{\shortcatvariables} \, .
    \end{align*}
    In all other cases we multiply scalars to the leg tensors such that $\acttensorofat{\selindex}{\headvariableof{\selindex}=0}=1$.
    Notice, that the product of all such scalars needs to be $1$ since $\contraction{\probwith}=1$.
    We then construct an activation core $\acttensorat{\headvariable}$
    \begin{align*}
        \acttensorat{\headvariable=\selindex} = \acttensorofat{\selindex}{\headvariableof{\selindex}=1}
    \end{align*}
    and have with a similar argument as in the converse proof direction
    \begin{align*}
        \probwith = \contractionof{\bencodingof{\exfunction}{\headvariable,\shortcatvariables},\acttensorat{\headvariable}}{\shortcatvariables} \, . & \qedhere
    \end{align*}
\end{proof}


\subsect{Composition of Functions} % TO DO: Connect with HT Formats, when doing compositions with multiple argument connectives!

Let $\exfunction: \arbsetof{1}\rightarrow\arbsetof{2}$ and $\chainingfunction:\arbsetof{2}\rightarrow\arbsetof{3}$ be arbitrary functions, then the indicator of their composition obeys
\begin{align*}
    \subspaceof{\chainingfunction\circ\exfunction} \subset \subspaceof{\exfunction} \, .
\end{align*}


%Composition of functions are then understood as contractions of their corresponding subspaces.
%The subspace of a composition satisfied an inclusion relation, which is the opposite of those for cartesian products, that is
%\begin{align}
%    \subspaceof{\exconnective(\exfunction,\secexfunction)}
%    \subset \subspaceof{(\exfunction,\secexfunction)} \, .
%\end{align}

\begin{example}[Propositional Formulas]
    Each formula defines by its basis encoding the subspace of $\atomspace$
    \begin{align*}
        \subspaceof{\exformula} = \spanof{\lnot\formulaat{\shortcatvariables},\formulaat{\shortcatvariables}} \, .
    \end{align*}
    For composition of formulas $\exformula,\secexformula$ with a connective $\exconnective$ acting on their images we have % is then a choice of a subspace %\contractionof{\subspaceof{(\exformula,\secexformula)}}
    \begin{align*}
        \subspaceof{\exconnective(\exformula,\secexformula)}
        \subset\subspaceof{(\exformula,\secexformula)} \, .
    \end{align*}
    A connective $\exconnective$ can thus be understood as a selection of a two-dimensional subspace in the four-dimensional subspace of the cartesian product of the connected formulas.
    The contraction of atomic subspaces further span the space
    \begin{align*}
        \atomspace = \spanof{\contractionof{
            \subspaceof{\atomicformulaof{\atomenumerator}} \wcols \atomenumeratorin
        }{\shortcatvariables}} \, .
    \end{align*}
\end{example}


\subsect{Effective Representation of Partition Statistics}\label{sec:partitionStatistics}

\begin{definition}
    \label{def:partitionStatistic}
    We call a statistic $\sstat : \facstates \rightarrow \bigtimes_{\selindexin}[2]$ a partition statistic if
    \begin{align*}
        \contractionof{\sencsstatwith}{\shortcatvariables} = \onesat{\shortcatvariables} \, .
    \end{align*}
\end{definition}

We now show, that partition statistics are exactly those statistics, which are collections of indicator features to a function.

\begin{lemma}
    \label{lem:partitionStatisticFunctionIndicator}
    A statistic $\sstat$ is a partition statistic, if and only if there exists a function $\exfunction:\facstates\rightarrow\outset$ with an image interpretation $\indexinterpretation : [\seldim] \rightarrow \outset$ such that for all $\selindexin$
    \begin{align*}
        \sencsstatat{\shortcatvariables,\indexedselvariable} = \indicatorofat{\exfunction=\indexinterpretationat{\selindex}}{\shortcatvariables} \, .
    \end{align*}
\end{lemma}
\begin{proof}
    \proofleftsymbol{}:
    Given any function $\exfunction:\facstates\rightarrow \outset$ we have
    \begin{align*}
        \sum_{\selindexin} \indicatorofat{\exfunction=\indexinterpretationat{\selindex}}{\shortcatvariables} =1
    \end{align*}
    and the selection tensor of indicator features is thus a partition statistic.

    \proofrightsymbol{}:
    Conversely, given a partition statistic $\sstat$, we can construct a function $\exfunctionof{\sstat}$ such that the partition statistic coincides with the collection of indicator features to the function.
    To this end we notice that for any index tuple $\shortcatindices$ there is a unique $\selindexin$ such that
    \begin{align*}
        \sencsstatat{\indexedshortcatvariables,\indexedselvariable} = 0 \, .
    \end{align*}
    This follows from $\sencsstatwith$ being boolean by assumption and $\contractionof{\sencsstatwith}{\indexedshortcatvariables}=1$.
    We define a function $\exfunctionof{\sstat}:\facstates\rightarrow[\seldim]$ with image interpretation by the identity on $[\seldim]$ coordinatewise by
    \begin{align*}
        \exfunctionofat{\sstat}{\indexedshortcatvariables} = \selindex
    \end{align*}
    and have for each $\selindexin$
    \begin{align*}
        \sencsstatat{\shortcatvariables,\indexedselvariable} = \indicatorofat{\exfunctionof{\sstat}=\selindex}{\shortcatvariables} & \qedhere
    \end{align*}
\end{proof}

\begin{example}[Edge statistics of Markov Networks]
    Each edge statistics $\sstatcoordinateof{\edge}$ of a Markov Network (see \theref{the:markovNetworkExponentialFamilies}) defines a partition statistic
    \begin{align*}
        \sstatcoordinateof{\edge}(\catindexof{\nodes}) = \catindexof{\edge} \, .
    \end{align*}
    This holds since for any $\catindexof{\nodes}$ we have
    \begin{align*}
        \contractionof{\sencodingof{\sstatcoordinateof{\edge}}{\nodevariables,\selvariableof{\edge}}}{\indexednodevariables}
        &= \sum_{\selindexof{\edge}\in[\catdimof{\edge}]} \sencodingof{\sstatcoordinateof{\edge}}{\indexednodevariables,\indexedselvariableof{\edge}} \\
        &= \sencodingof{\sstatcoordinateof{\edge}}{\indexednodevariables,\selvariableof{\edge}=\catvariableof{\edge}} \\
        &= 1 \, .
    \end{align*}
    The corresponding indicators stated in \lemref{lem:partitionStatisticFunctionIndicator} are enumerated by the image elements $\selindexof{\edge}\in\catdimof{\edge}$ and given by
    \begin{align*}
        \sstatcoordinateofat{\edge,\selindexof{\edge}}{\catindexof{\nodes}}
        = \begin{cases}
              1 & \ifspace \catindexof{\edge} = \selindexof{\edge} \\
              0 & \text{else}
        \end{cases} \, .
    \end{align*}
\end{example}

We have already observed in \charef{cha:probRepresentation}, that Markov Networks have a tensor-network representation involving the selection encodings of their edge statistics.
This efficiency gain compared with featurewise releational encodings is in the following generalized to arbitrary partition statistics.

\begin{theorem}
    \label{the:selectionRepresentationPartitionStatistics}
    For any partition statistic and any elementary activation tensor $\bigotimes_{\selindexin}\acttensorofat{\selindex}{\headvariableof{\selindex}}$ we have
    \begin{align*}
        \contractionof{\bigcup_{\selindexin}\{\bencodingofat{\sstatcoordinateof{\selindex}}{\headvariableof{\selindex},\shortcatvariables},\acttensorofat{\selindex}{\headvariableof{\selindex}}\}}{\shortcatvariables}
        = \contractionof{\sencsstatwith,\acttensorat{\headvariable}}{\shortcatvariables}
    \end{align*}
    where
    \begin{align*}
        \acttensorat{\headvariable}
        =
        \begin{cases}
            \left(\prod_{\selindexin} \acttensorofat{\selindex}{\headvariableof{\selindex}=0} \right) \sum_{\selindexin}\acttensorofat{\selindex}{\headvariableof{\selindex}=1} \cdot \onehotmapofat{\selindex}{\headvariable}
            & \ifspace \uniquantwrtof{\selindexin}{\acttensorofat{\selindex}{\headvariableof{\selindex}=0}\neq 0} \\ %\forall_{\selindexin} :  \\
            \acttensorofat{\secselindex}{\headvariableof{\secselindex}=1} \cdot \prod_{\selindex\neq\secselindex}\acttensorofat{\secselindex}{\headvariableof{\secselindex}=0}
            & \ifspace \existquantwrtof{\secselindex\in[\seldim]}{\acttensorofat{\secselindex}{\headvariableof{\secselindex}=0}=0} %\ncond \forall_{\selindex\neq\secselindex} \acttensorofat{\selindex}{\headvariableof{\selindex}=0}=0
        \end{cases} \, .
    \end{align*}
    We notice that by definition $\acttensorat{\headvariable}=\zerosat{\headvariable}$ if there is more than one $\selindexin$ with $\acttensorofat{\selindex}{\headvariableof{\selindex}=0}=0$.
\end{theorem}
\begin{proof}
    \lemref{lem:partitionStatisticFunctionIndicator} implies, that there is a function $\exfunction$ such that the features of the partition statistics are the indicator features to $\exfunction$.
    Now, as in the proof of \lemref{lem:computableByFunctionEqualElementaryIndicators} we transform the activation cores to arrive at the statement.
\end{proof}
%Let us now characterize the set of by $\exfunction$ and $\maxgraph$ (respectively $\elgraph$) computable distributions

\theref{the:selectionRepresentationPartitionStatistics} is especially useful, when instantiating the distribution of a \HybridLogicNetwork{} with a subset $\secformulaset\subset\formulaset$ of features satisfying
\begin{align*}
    \sum_{\exformula\in\secformulaset} \formulaat{\shortcatvariables} \prec \onesat{\shortcatvariables} \, .
\end{align*}
If $\sum_{\exformula\in\secformulaset} \formulaat{\shortcatvariables} \neq \onesat{\shortcatvariables}$ we can add a dummy feature $\onesat{\shortcatvariables}-\sum_{\exformula\in\formulaset} \formulaat{\shortcatvariables}$ to $\secformulaset$ with trivial activation core to get a partition statistics.
Now, given a partition statistic $\secformulaset$ we can instantiate the to $\secformulaset$ corresponding tensors in the tensor network representation of \theref{the:expFamilyTensorRep} by the selection encoding $\sencodingofat{\secformulaset}{\shortcatvariables,\selvariable}$ as
\begin{align*}
    \contractionof{\sencodingofat{\secformulaset}{\shortcatvariables,\selvariable},\expof{\canparamat{\selvariable}}}{\shortcatvariables}
    = \contractionof{\bigcup_{\selindexin}\{\bencodingofat{\enumformula}{\headvariableof{\selindex},\shortcatvariables},\expof{\canparamat{\indexedselvariable}\cdot\indexinterpretationofat{\selindex}{\headvariableof{\selindex}}}\}}{\shortcatvariables} \, .
\end{align*}


\sect{Hybrid Basis and Coordinate Calculus}\label{sec:hybridCalculus}

% Motivation: Effective Coordinate Calculus
In some situations, we can perform basis calculus more effectively by avoiding image enumeration variables, and instead apply coordinatewise transforms on tensors (see \defref{def:coordinatewiseTransform}).
As we show here, these include conjunctions, which correspond with coordinatewise multiplication, and negation, which correspond with coordinatewise subtraction from the trivial tensor.
Such schemes are applied for example in \cite{tsilionis_tensor-based_2024} in batchwise logical inference.

\begin{figure}
    \begin{center}
        \begin{tikzpicture}[scale=0.3] 

\node at (-12,2) [above]  {a)};

	\begin{scope}[shift={(-7,0)}]
	  	\draw[]  (-3,2) rectangle (1,4);
		\node at (-1,1.9) [above] {${\exformula\land\secexformula}$};
		\draw (-2,2) -- (-2,-2) node[midway,left] {$\catvariableof{1}$};
		\draw (0,2) -- (0,-2) node[midway,right] {$\catvariableof{2}$};	
	\end{scope}
	\node at (-4.5,1.9) [above] {$=$};	
  	\draw[] (-3,2) rectangle (1,4);
	\node at (-1,1.9) [above] {${\exformula}$};
  	\draw[] (3,2) rectangle (5,4);
	\node at (4,1.9) [above] {${\secexformula}$};
	
	\node at (0,0) [left,] {$\delta$};
	\draw[]  (0,2) -- (0,0);% node[midway,left] {$\placeholderof{1}$};
	\draw[fill,] (0,0) circle (0.15cm);
	\draw[] (0,0) -- (0,-2) node[midway,right] {$\catvariableof{2}$};
	\draw[] (4,2) to[bend left=20] (0,0);
	\draw[] (-2,2) -- (-2,-2) node[midway,left] {$\catvariableof{1}$};

\begin{scope}[shift={(27,0)}]
	\node at (-18,2) [above]  {b)};
  	\begin{scope}[shift={(-14,0)}]
	  	\draw[]  (-3,2) rectangle (1,4);
		\node at (-1,1.9) [above] {${\lnot\exformula}$};
		\draw (-2,2) -- (-2,-2) node[midway,left] {$\catvariableof{1}$};
		\draw (0,2) -- (0,-2) node[midway,right] {$\catvariableof{2}$};	
	\end{scope}
	
    	\begin{scope}[shift={(-2,-2)}]
  		\draw[] (-8,2) rectangle (-4,4);
		\node at (-6,2) [above,] {$\ones$};
		\draw[] (-7,2) -- (-7,0) node[midway,left] {$\catvariableof{1}$};
		\draw[] (-5,2) -- (-5,0) node[midway,right] {$\catvariableof{2}$};
	\end{scope}
	
	\draw[] (-2,2) -- (-2,-2) node[midway,left] {$\catvariableof{1}$};
	\draw[](0,2) -- (0,-2) node[midway,right] {$\catvariableof{2}$};
	\node[] at (-4.5,0) [above] {$-$};



	\node at (-11.5,1.9) [above] {$=$};	
	
  	\draw[]  (-3,2) rectangle (1,4);
	\node at (-1,1.9) [above] {${\exformula}$};	
	
\end{scope}

\end{tikzpicture}
    \end{center}
    \caption{Decomposition schemes by hybrid calculus, using coordinatewise transforms of tensors (see \defref{def:coordinatewiseTransform}).
    a) Conjunction performed by coordinatewise multiplications.
    b) Negations performed by coordinatewise subtraction from one.}\label{fig:ConNegDecomposition}
\end{figure}

\begin{theorem}
    \label{the:effectiveConjunction}
    For any formulas $\exformula,\secexformula$ we have
    \begin{align*}
        \contractionof{
            \bencodingofat{\land}{\headvariableof{\exformula\land\secexformula},\formulavar,\secexformulavar},\tbasisat{\headvariableof{\exformula\land\secexformula}}
        }{\formulavar,\secexformulavar}
        = \tbasisat{\formulavar} \otimes \tbasisat{\secexformulavar} \, .
    \end{align*}
    In particular, it holds that (see \figref{fig:ConNegDecomposition}a)
    \begin{align*}
    (\exformula\land\secexformula)[\shortcatvariables]
        = \contractionof{\exformula,\secexformula}{\shortcatvariables} \, .
    \end{align*}
\end{theorem}
\begin{proof}
    We decompose
    \begin{align*}
        \bencodingofat{\land}{\headvariableof{\exformula\land\secexformula},\formulavar,\secexformulavar}
        = \tbasisat{\headvariableof{\exformula\land\secexformula}} \otimes \tbasisat{\formulavar} \otimes \tbasisat{\secexformulavar}
        + \fbasisat{\headvariableof{\exformula\land\secexformula}} \left( \onesat{\formulavar,\secexformulavar} -  \tbasisat{\formulavar} \otimes \tbasisat{\secexformulavar} \right)
    \end{align*}
    and get the first claim as
    \begin{align*}
        \contractionof{
            \bencodingofat{\land}{\headvariableof{\exformula\land\secexformula},\formulavar,\secexformulavar},\tbasisat{\headvariableof{\exformula\land\secexformula}}
        }{\formulavar,\secexformulavar}
        & = \contractionof{
            \tbasisat{\headvariableof{\exformula\land\secexformula}} \otimes \tbasisat{\formulavar} \otimes \tbasisat{\secexformulavar},\tbasisat{\headvariableof{\exformula\land\secexformula}}
        }{\formulavar,\secexformulavar} \\
        & = \tbasisat{\formulavar} \otimes \tbasisat{\secexformulavar} \, .
    \end{align*}
    To show the second claim we use
    \begin{align*}
    (\exformula\land\secexformula)[\shortcatvariables]
        &= \contractionof{
            \bencodingofat{\exformula}{\formulavar,\shortcatvariables},
            \bencodingofat{\secexformula}{\secexformulavar,\shortcatvariables},
            \bencodingofat{\land}{\headvariableof{\exformula\land\secexformula},\formulavar,\secexformulavar},
            \tbasisat{\headvariableof{\exformula\land\secexformula}}
        }{\shortcatvariables} \\
        &  = \contractionof{
            \bencodingofat{\exformula}{\formulavar,\shortcatvariables},
            \bencodingofat{\secexformula}{\secexformulavar,\shortcatvariables},
            (\tbasisat{\formulavar}\otimes \tbasisat{\secexformulavar})
        %\bencodingofat{\land}{\formulavar,\secexformulavar,\headvariableof{\exformula\land\secexformula}}
        }{\shortcatvariables} \\
        &= \contractionof{\exformula,\secexformula}{\shortcatvariables} \, . \qedhere
    \end{align*}
\end{proof}

A similar decomposition holds for negations, as we show next.

\begin{theorem}
    For any formula $\exformula$ we have
    \begin{align*}
        \contractionof{
            \bencodingofat{\lnot}{\headvariableof{\lnot\exformula},\formulavar},\tbasisat{\headvariableof{\lnot\exformula}}
        }{\formulavar}
        = \fbasisat{\formulavar} =  \onesat{\formulavar} - \tbasisat{\formulavar} \, .
    \end{align*}
    and
    \begin{align*}
        \contractionof{
            \bencodingofat{\lnot}{\formulavar,\headvariableof{\lnot\exformula}},\fbasisat{\headvariableof{\lnot\exformula}}
        }{\formulavar}
        = \tbasisat{\formulavar} \, .
    \end{align*}
    In particular, it holds that (see \figref{fig:ConNegDecomposition}b)
    \begin{align*}
    (\lnot\exformula)[\shortcatvariables]
        = \onesat{\shortcatvariables} - \formulaat{\shortcatvariables}  \, .
    \end{align*}
\end{theorem}
\begin{proof}
    We have
    \begin{align*}
        \bencodingofat{\lnot}{\headvariableof{\lnot\exformula},\formulavar}
        = \tbasisat{\headvariableof{\lnot\exformula}} \otimes \fbasisat{\formulavar}
        +\fbasisat{\headvariableof{\lnot\exformula}} \otimes \tbasisat{\formulavar}
    \end{align*}
    and therefore get the second claim by contraction with $\fbasisat{\headvariableof{\lnot\exformula}}$
    \begin{align*}
        \contractionof{
            \bencodingofat{\lnot}{\formulavar,\headvariableof{\lnot\exformula}},\fbasisat{\headvariableof{\lnot\exformula}}
        }{\formulavar}
        &=  \contractionof{
            \tbasisat{\headvariableof{\lnot\exformula}} \otimes \fbasisat{\formulavar},\fbasisat{\headvariableof{\lnot\exformula}}
        }{\formulavar} \\
        &\quad +  \contractionof{
            \fbasisat{\headvariableof{\lnot\exformula}} \otimes \tbasisat{\formulavar},\fbasisat{\headvariableof{\lnot\exformula}}
        }{\formulavar}
        \\
        &= \tbasisat{\formulavar} \, .
    \end{align*}
    The first equation of the first claim follows similarly by contraction with $\tbasisat{\headvariableof{\lnot\exformula}}$ as
    \begin{align*}
        \contractionof{
            \bencodingofat{\lnot}{\headvariableof{\lnot\exformula},\formulavar},\tbasisat{\headvariableof{\lnot\exformula}}
        }{\formulavar}
        = \fbasisat{\formulavar}
    \end{align*}
    and the second equation using that $\onesat{\catvariable}=\fbasisat{\catvariable}+\tbasisat{\catvariable}$ and hence
    \begin{align*}
        \fbasisat{\formulavar} =  \onesat{\formulavar} - \tbasisat{\formulavar} \, .
    \end{align*}
    To show the third claim, we contract the computation tensor $\bencodingofat{\exformula}{\formulavar,\shortcatvariables}$ to the formula $\exformula$ on both sides of the first claim and get
    \begin{align*}
        &\contractionof{
            \bencodingofat{\exformula}{\formulavar,\shortcatvariables},\bencodingofat{\lnot}{\headvariableof{\lnot\exformula},\formulavar},\tbasisat{\headvariableof{\lnot\exformula}}
        }{\formulavar} \\
        &\quad=
        \contractionof{\bencodingofat{\exformula}{\formulavar,\shortcatvariables},\onesat{\formulavar}}{\formulavar}
        - \contractionof{\bencodingofat{\exformula}{\formulavar,\shortcatvariables},\tbasisat{\formulavar}}{\formulavar} \, .
    \end{align*}
    We simplify this equation using the trivial head contraction identity of directed tensors of \corref{cor:onesHead} and \lemref{lem:formulaEncodingDecomposition} stating that
%    For any formula $\formulaat{\shortcatvariables}$ we further have by \lemref{lem:formulaEncodingDecomposition}
    \begin{align*}
        \formulaat{\shortcatvariables} = \contractionof{\bencodingofat{\exformula}{\formulavar,\shortcatvariables},\tbasisat{\formulavar}}{\shortcatvariables}
        \andspace
        \lnot\formulaat{\shortcatvariables} = \contractionof{\bencodingofat{\exformula}{\formulavar,\shortcatvariables},\fbasisat{\formulavar}}{\shortcatvariables}
    \end{align*}
    and arrive at the third claim
    \begin{align*}
    (\lnot\exformula)[\shortcatvariables]
        = \onesat{\shortcatvariables} - \formulaat{\shortcatvariables}  \, . & \qedhere
    \end{align*}
\end{proof}

% Usage
These theorems provide a mean to represent logical formulas by sums of one-hot encodings.
Since any propositional formula can be represented by compositions of negations and conjunctions, they are universal.
We further notice, that the resulting decomposition is a basis+ CP format, as further discussed in \charef{cha:sparseRepresentation}.
In \figref{fig:DecompositionExample} we provide an example of this decomposition.


\begin{figure}
    \begin{center}
        \begin{tikzpicture}[scale=0.275] % , baseline = -3.5pt



\begin{scope}[shift={(-19,-1.7)}]
		%\draw[] (-1,2.2) ellipse (4 and 2.5);
	  	\draw  (-3,2) rectangle (1,4);
		\node at (-1,1.9) [above] {${\exformula}$};
		\draw (-2,2) -- (-2,0) node[midway,left] {$\catvariableof{0}$};
		\draw (0,2) -- (0,0) node[midway,right] {$\catvariableof{1}$};	
		\node at (3.5,2.2)[right]  {$=$};
\end{scope}

\draw[] (-4,0.5) ellipse (9 and 3);
   	\begin{scope}[shift={(-2,-2)}]
  		\draw[\skeletoncolor] (-8,2) rectangle (-4,4);
		\node at (-6,2) [above,\skeletoncolor] {$\ones$};
		\draw[\skeletoncolor] (-7,2) -- (-7,0) node[midway,left] {$\catvariableof{0}$};
		\draw[\skeletoncolor] (-5,2) -- (-5,0) node[midway,right] {$\catvariableof{1}$};
	\end{scope}
	
	\draw[\skeletoncolor] (-2,2) -- (-2,-2) node[midway,left] {$\catvariableof{0}$};
	\draw[\skeletoncolor](0,2) -- (0,-2) node[midway,right] {$\catvariableof{1}$};
	\node[\skeletoncolor] at (-4.5,0) [above] {$-$};
\draw[thick,dashed] (-4,-2.5) -- (5,-3.5);
%\draw[] (6,-2.5) -- (1,-2.5);

%% Into negation core
\draw[thick,dashed] (-1,4.7) -- (-1,2);%-4,3.5);

%% Into conjunction core
\draw[thick,dashed] (10,4.7) -- (10,-4);%(6,-2.5);


\begin{scope}[shift={(10,-6)}]
	\draw[] (-4,0.5) ellipse (9 and 3);
	\begin{scope}[shift={(-4,0)}]
		\renewcommand{\skeletoncolor}{red}
	\node at (0,0) [left,\skeletoncolor] {$\delta$};
	\draw[\skeletoncolor]  (0,2) -- (0,0);% node[midway,left] {$\placeholderof{1}$};
	\draw[fill,\skeletoncolor] (0,0) circle (\dotsize);
	\draw[\skeletoncolor] (0,0) -- (0,-2) node[midway,right] {$\catvariableof{1}$};
	\draw[\skeletoncolor] (5,2) to[bend left=20] (0,0);


	\draw[fill,\skeletoncolor] (-2,0.5) circle (\dotsize);
	\draw[\skeletoncolor] (3,2) to[bend left=20] (-2,0.5);
	\draw[\skeletoncolor] (-2,2) -- (-2,-2) node[midway,left] {$\catvariableof{0}$};
	\end{scope}
\end{scope}



\begin{scope}[shift={(0,5)}]
		%\draw[] (-1,2.2) ellipse (4 and 2.5);
	  	\draw  (-3,2) rectangle (1,4);
		\node at (-1,1.9) [above] {${\secexformula^{(1)}}$};
		\draw (-2,2) -- (-2,0) node[midway,left] {$\catvariableof{0}$};
		\draw (0,2) -- (0,0) node[midway,right] {$\catvariableof{1}$};	
\end{scope}

\begin{scope}[shift={(11,5)}]
		%\draw[] (-1,2.2) ellipse (4 and 2.5);
	  	\draw  (-3,2) rectangle (1,4);
		\node at (-1,1.9) [above] {${\secexformula^{(2)}}$};
		\draw (-2,2) -- (-2,0) node[midway,left] {$\catvariableof{0}$};
		\draw (0,2) -- (0,0) node[midway,right] {$\catvariableof{1}$};	
		
\end{scope}




\node at (14,0.5)[right]  {$=$};


\begin{scope}[shift={(29,0)}]

\begin{scope}[shift={(-7.5,-1.7)}]
		%\draw[] (-1,2.2) ellipse (4 and 2.5);
	  	\draw[]  (-3,2) rectangle (1,4);
		\node at (-1,1.9) [above] {${\secexformula^{(2)}}$};
		\draw[] (-2,2) -- (-2,0) node[midway,left] {$\catvariableof{0}$};
		\draw[] (0,2) -- (0,0) node[midway,right] {$\catvariableof{1}$};	
		\node at (2.25,2.2)[right]  {$-$};		
\end{scope}


\begin{scope}[shift={(1,-3.7)}]
%	\renewcommand{\skeletoncolor}{\conjunctioncolor}
	\node at (0,0) [left] {$\delta$};
	\draw[]  (0,2) -- (0,0);% node[midway,left] {$\placeholderof{1}$};
	\draw[fill] (0,0) circle (0.15cm);
	\draw[] (0,0) -- (0,-2) node[midway,right] {$\catvariableof{1}$};
	\draw[] (5,2) to[bend left=20] (0,0);


	\draw[fill] (-2,0.5) circle (0.15cm);
	\draw[] (3,2) to[bend left=20] (-2,0.5);
	\draw[] (-2,2) -- (-2,-2) node[midway,left] {$\catvariableof{0}$};
\end{scope}


\begin{scope}[shift={(1,-1.7)}]
		%\draw[] (-1,2.2) ellipse (4 and 2.5);
	  	\draw  (-3,2) rectangle (1,4);
		\node at (-1,1.9) [above] {${\secexformula^{(1)}}$};
		\draw[] (-2,2) -- (-2,0); % node[midway,left] {$\catvariableof{0}$};
		\draw[] (0,2) -- (0,0); % node[midway,right] {$\catvariableof{1}$};	
	
\end{scope}
	
\begin{scope}[shift={(6,-1.7)}]
		%\draw[] (-1,2.2) ellipse (4 and 2.5);
	  	\draw[]  (-3,2) rectangle (1,4);
		\node at (-1,1.9) [above] {${\secexformula^{(2)}}$};
		\draw[]  (-2,2) -- (-2,0); % node[midway,left] {$\catvariableof{0}$};
		\draw[]  (0,2) -- (0,0); % node[midway,right] {$\catvariableof{1}$};	
	
\end{scope}

\end{scope}



	
	
\end{tikzpicture}
    \end{center}
    \caption{
        Example of a decomposition by hybrid calculus of a formula
        $\formulaat{\catvariableof{1},\catvariableof{2}} = \textcolor{blue}{\lnot} \secexformula^{(1)}[\catvariableof{1},\catvariableof{2}] \textcolor{red}{\land}  \secexformula^{(2)}[\catvariableof{1},\catvariableof{2}]$ into a sum of contractions.
    }\label{fig:DecompositionExample}
\end{figure}


\sect{Applications in Machine Learning}

% Neural paradigm
Basis calculus provides a tool suited to represent decompositions of function by efficient tensor networks.
The decomposition of function into smaller components, called neurons, is often referred to as the neural paradigm of machine learning.
Our model of the neural paradigm are tensor network decompositions, seen as decomposition of functions into smaller functions, which take each other as input.
Summations along input axis are avoided, when having directed and boolean tensor networks with basis calculus interpretation.
Inference is then performed by contractions with basis tensors representing the input, as shown in \theref{the:basisCalculus}.
These contractions can further be executed neuron-wise.

% + Symbolic -> Neuro-symbolic
What is more, basis calculus provides an efficient scheme to represent symbols such as logical connectives by their basis encodings.
Basis calculus is therefore a tool for neuro-symbolic AI \cite{garcez_neural-symbolic_2019, sarker_neuro-symbolic_2022, marra_statistical_2024}.

% ADD?
% \secref{sec:hybridCalculus} -> Connection to conjunction-based knownledge bases
% \secref{sec:indicatorFeatures} -> Connection with factor-wise graphical model contractions


%The neural paradigm of Machine Learning describes the relevance of sparse function to be effective models in the sense of learning and approximation.
% Neural Paradigm by Tensor Network Decompositions
% Basis Calculus
%We have already observed in \theref{the:basisCalculus}, that the value of discrete maps can be calculated by contractions of the directed boolean relation encodings.
%This has been framed as Basis Calculus.
%What is more, tensor network decompositions into directed boolean tensors correspond with representation of functions as compositions of smaller functions.
%We can understand each composition as marking a neuron in an architecture and thus have established a neural perspective on boolean directed tensor networks.


    \chapter{\chatextsparseCalculus}\label{cha:sparseRepresentation}

We in this chapter develop sparse tensor representation formats based on constrained $\cpformat$ formats.
Our motivation for these formats result from the connection to encoding mechanisms, which we have applied in \parref{par:one} and \parref{par:two}, and to sparse optimization formats.
%We further provide constructive bounds on the corresponding tensor ranks.


\sect{$\cpformat$ Decomposition}

% Motivation by Singular Value Decomposition
The $\cpformat$ decomposition is one way to generalize the ranks of matrices to tensors.
It is oriented on the Singular Value Decomposition of matrices, providing a representation of the matrix as a weighed sum of the tensor product of singular vectors.
Given a matrix $\matrixat{\catvariableof{0},\catvariableof{1}}$, we enumerate its singular values by $\decvariable$ taking values in $[\decdim]$ and store them in a vector $\scalarcoreat{\decvariable}$.
With the corresponding singular vectors by $\legcoreofat{0}{\catvariableof{0},\decvariable}$ and $\legcoreofat{1}{\catvariableof{1},\decvariable}$, the singular value decomposition of $\exmatrix$ is
\begin{align*}
    \matrixat{\catvariableof{0},\catvariableof{1}}
    &= \sum_{\decindexin} \scalarcoreat{\indexeddecvariable} \cdot \legcoreofat{0}{\catvariableof{0},\indexeddecvariable} \otimes \legcoreofat{1}{\catvariableof{1},\indexeddecvariable} \, .
\end{align*}
Here the smallest $\decdim$ such that this decomposition exists, is the matrix rank $\cprankof{\exmatrix}$.
In contraction notation we abbreviate this to
\begin{align*}
    \matrixat{\catvariableof{0},\catvariableof{1}}
    &= \contractionof{\scalarcoreat{\decvariable},\legcoreofat{0}{\catvariableof{0},\decvariable},\legcoreofat{1}{\catvariableof{1},\decvariable}}{\catvariableof{0},\catvariableof{1}} \, .
\end{align*}
Given a tensor of higher order, a generalization of this decomposition is a tensor product over multiple vectors, as we define next.
%% REPETITION
%\begin{align*}
%	\hypercorewith
%	&= \sum_{\decindexin} \scalarcoreat{\indexeddecvariable} \cdot \bigotimes_{\catenumeratorin}\legcoreofat{\catenumerator}{\catvariableof{\catenumerator},\indexeddecvariable} \, .
%	& = \contractionof{
%		\{\scalarcoreat{\decvariable}\} \cup \{ \legcoreofat{\atomenumerator}{\decvariable,\catvariableof{\atomenumerator}} \, : \, \atomenumeratorin \}
%		}{\shortcatvariables} \, .
%\end{align*}


\begin{definition}
    \label{def:cpFormats}
    A $\cpformat$ decomposition of size $\decdim$ of a tensor $ \hypercorewithin$ is a collections of a scalar core $\scalarcoreat{\decvariable}$ and leg cores $\legcoreofat{\atomenumerator}{\decvariable,\catvariableof{\atomenumerator}}$ for $\atomenumeratorin$, where $\decvariable$ is an enumeration variable taking values in $[\decdim]$, such that
    \begin{align*}
        \hypercorewith
        = \contractionof{
            \{\scalarcoreat{\decvariable}\} \cup \{ \legcorewith \, : \, \atomenumeratorin \}
        }{\shortcatvariables} \, .
    \end{align*}
%	where for each $\decindexin$ and $\atomenumeratorin$ we have $\scalarcoreat{\decindex} \in \rr$ and $\legcoreof{\atomenumerator,\decindex}\in\rr^{\catindexof{\atomenumerator}}$.
    We say that the $\cpformat$ Decomposition is
    \begin{itemize}
        \item directed, when for each $\atomenumerator$ the core $\legcoreof{\atomenumerator}$ is directed with $\decvariable$ incoming and $\catvariableof{\atomenumerator}$ outgoing.
        \item boolean, when for each $\atomenumerator$ the core $\legcoreof{\atomenumerator}$ is boolean.
        \item basis, where we demand both properties, that is for each $\atomenumeratorin$ and $\decindexin$
        \begin{align*}
            \legcoreofat{\atomenumerator}{\catvariableof{\atomenumerator},\indexeddecvariable}\in \{\onehotmapofat{[\catindexof{\atomenumerator}]}{\catvariableof{\atomenumerator}} \catindexof{\atomenumerator}\in[\catdimof{\atomenumerator}] \}\, .
        \end{align*}
        \item basis+, when for each $\atomenumeratorin$ and $\decindexin$  %$\legcoreof{\atomenumerator,\decindex}\in\onehotmapof{[\catindexof{\atomenumerator}]}$ or $\legcoreof{\atomenumerator,\decindex}=\ones$.
        \begin{align*}
            \legcoreofat{\atomenumerator}{\catvariableof{\atomenumerator},\indexeddecvariable}\in \{\onehotmapofat{[\catindexof{\atomenumerator}]}{\catvariableof{\atomenumerator}} \catindexof{\atomenumerator}\in[\catdimof{\atomenumerator}] \} \cup \{\onesat{\catvariableof{\atomenumerator}}\}\, .
        \end{align*}
    \end{itemize}
    We denote by $\cprankof{\hypercore}$, respectively $\bincprankof{\hypercore}$, $\bascprankof{\hypercore}$ and $\baspluscprankof{\hypercore}$ the minimal cardinality such that $\hypercore$ has a $\cpformat$ Decomposition, respectively with directed cores, boolean cores, basis cores and basis+ cores.
\end{definition}

%To see that all ranks are finite, we can easily
All ranks have a naive bound by the space dimension, which is obvious from the coordinate decomposition (see \charef{cha:coordinateCalculus})
\begin{align*}
    \hypercorewith
    = \sum_{\shortcatindices\in\facstates} \hypercoreat{\indexedshortcatvariables} \cdot \bigotimes_{\catenumeratorin} \onehotmapofat{\catenumerator}{\catvariableof{\catenumerator}} \, .
\end{align*}
If we construct $\decindex$ as an enumeration of the coordinates in $\facstates$, that is $\decdim=\prod_{\catenumeratorin}\catdimof{\catenumerator}$, this is a $\cpformat$ decomposition, which is basis and therefore also directed, boolean and basis+.

%
$\cpformat$ decomposition as a tensor network format come with some drawbacks.
The set of tensors with a fixed rank are not closed \cite{beylkin_algorithms_2005} and approximation problems are often ill posed \cite{de_silva_tensor_2008}.
Since as a consequence their numerical treatment comes with many problems \cite{espig_variational_2012}, alternative formats have gained popularity.
Common formats are the $\mathrm{TUCKER}$-format originally introduced in \cite{hitchcock_expression_1927}, and often refered to as higher-order singular value decomposition, and the more recently developed $\ttformat$ and $\htformat$ decomposition formats (see \charef{cha:introduction}).
Given a $\htformat$ the best approximation of a tensor always exists (Theorem 11.58 in \cite{hackbusch_tensor_2012}).

%% Sum of elementary tensors
%We have by definition
%	\[ \hypercorewith
%	= \sum_{\decindexin} \scalarcoreat{\inddecvar} \left( \bigotimes_{\atomenumeratorin} \legcoreofat{\atomenumerator}{\catvariableof{\atomenumerator},\indexeddecvariable} \right) \, . \]
%The right side can be seen as an alternative definition of $\cpformat$ decompositions by summations of elementary tensors.


\begin{figure}[h]
    \begin{center}
        \begin{tikzpicture}[scale=0.35, thick] % , baseline = -3.5pt

    \begin{scope}
        [shift={(0,2)}]
        \draw[] (0,1)--(0,-1) node[midway,left] {\colorlabelsize $\catvariableof{0}$};
        \draw[] (1.5,1)--(1.5,-1) node[midway,left] {\colorlabelsize $\catvariableof{1}$};
        \node[anchor=center] (text) at (3,0) {$\cdots$};
        \draw[] (4,1)--(4,-1) node[midway,right] {\colorlabelsize $\catvariableof{\atomorder\shortminus1}$};
    \end{scope}


    \draw (-1,1) rectangle (5,-1);
    \node[anchor=center] (text) at (2,0) {\corelabelsize $\hypercore$};


    \node[anchor=center] (text) at (7,0) {${=}$};


    \begin{scope}
        [shift={(10,2)}]

        \coordinate (conposseldec) at (4.5,-5.5);
        \drawvariabledot{4.5}{-5.5}

%\draw[fill] (\conposseldec) circle (\dotsize);
        \draw (conposseldec) -- (4.5,-7.5) node[midway, right] {\colorlabelsize ${\decvariable}$}; % Unclear, whether this is the best notation!
        \draw[] (3.5,-7.5) rectangle (5.5, -9.5);
        \node[anchor=center] (text) at (4.5,-8.5) {\corelabelsize $\scalarcore$};

        \draw[] (0,1) -- (0,-1) node[midway,left] {\colorlabelsize $\catvariableof{0}$};
        \draw (-1,-1) rectangle (1, -3);
        \node[anchor=center] (text) at (0,-2) {\corelabelsize $\legcoreof{0}$};
        \draw[] (0,-3) to[bend right=20] (conposseldec);


        \draw[] (3,1) -- (3,-1) node[midway,left] {\colorlabelsize $\catvariableof{1}$};
        \draw (2,-1) rectangle (4, -3);
        \node[anchor=center] (text) at (3,-2) {\corelabelsize $\legcoreof{1}$};
        \draw[] (3,-3) to[bend right=20]  (conposseldec);

        \node[anchor=center] (text) at (6,-2) {$\cdots$};

        \draw[] (9,1) -- (9,-1) node[midway,left] {\colorlabelsize $\catvariableof{\atomorder-1}$};
        \draw (7.75,-1) rectangle (10.25, -3);
        \node[anchor=center] (text) at (9,-2) {\corelabelsize $\legcoreof{\atomorder-1}$};
        \draw[] (9,-3) to[bend left=20]  (conposseldec);

    \end{scope}


\end{tikzpicture}
    \end{center}
    \caption{Tensor Network diagram of a generic $\cpformat$ decomposition (see \defref{def:cpFormats})}
\end{figure}

%We introduce different notions of sparsities based on $\cpformat$ decomposition with different properties of their leg cores.

\subsect{Directed Leg Cores}

The contraint of directionality of the leg cores does not influence decomposablitiy of a tensor, as we show next.

\begin{lemma}
    \label{lem:cprankEqualsDir}
    For any tensor $\hypercorewith$ we have
    \begin{align*}
        \cprankof{\hypercore} = \dircprankof{\hypercore} \, .
    \end{align*}
\end{lemma}
\begin{proof}
    Let there be a $\cpformat$ decomposition of $\hypercore$ by
    \begin{align*}
        \hypercorewith
        = \contractionof{
            \{\scalarcoreat{\decvariable}\} \cup \{ \legcoreofat{\atomenumerator}{\catvariableof{\atomenumerator},\decvariable} \, : \, \atomenumeratorin \}
        }{\shortcatvariables} \, .
    \end{align*}
    We then transform the scalar core to another core $\tilde{\scalarcore}[\decvariable]$ with coordinates to $\decindexin$ by
    \begin{align*}
        \tilde{\scalarcore}[\indexeddecvariable]
        = \scalarcoreat{\indexeddecvariable} \cdot \prod_{\catenumeratorin} \contraction{\legcoreofat{\atomenumerator}{\catvariableof{\atomenumerator},\indexeddecvariable}} \, .
    \end{align*}
    It follows for any $\decindexin$, that
    \begin{align*}
        \scalarcoreat{\indexeddecvariable} \cdot \bigotimes_{\catenumeratorin}\legcoreofat{\catenumerator}{\catvariableof{\catenumerator},\indexeddecvariable}
        = \tilde{\scalarcore}[\indexeddecvariable] \cdot \bigotimes_{\catenumeratorin}\normalizationof{\legcoreofat{\catenumerator}{\catvariableof{\catenumerator},\indexeddecvariable}}{\catvariableof{\catenumerator}}
    \end{align*}
    and thus
    \begin{align*}
        \hypercorewith
        = \contractionof{
            \{\tilde{\scalarcore}[\decvariable] \cup \{ \normalizationof{\legcoreofat{\atomenumerator}{\catvariableof{\atomenumerator},\decvariable}}{\catvariableof{\atomenumerator},\decvariable} \, : \, \atomenumeratorin \}
        }{\shortcatvariables} \, .
    \end{align*}
    We have thus constructed a directed $\cpformat$ decomposition of same size $\decdim$ to an arbitrary $\cpformat$ decomposition and conclude that $\cprankof{\hypercore} = \dircprankof{\hypercore}$.
\end{proof}

%This is the canonical $\cpformat$ decomposition, where the vectors $\legcoreof{\atomenumerator,\decindex}$ are interpreted as generalized singular vectors.
%Any $\cpformat$ decomposition can be transformed into a directed $\cpformat$ decomposition without enlarging the index set $\indexset$, simply by diving the vectors by their norms and multiplying it to $\scalarcoreat{\inddecvar}$.

%% Directionality
%We then have a partially directed Tensor Network representing the decomposed tensor.
%The only undirected core is $\scalarcore$, since we do not demand it to be normed.
%In many applications applications, however, also the $\scalarcore$ is directed with a single outgoing leg (see for example the empirical distributions as discussed in \secref{sec:empDistribution}).
%In that case, also the decomposed tensor is directed with outgoing legs.



\subsect{Basis $\cpformat$ decompositions and the $\ell_0$-norm}\label{sec:basisCP}

% From FOL Chapter: Bayesian Network interpretation of Basis CP
%	The basis CP can further be understood as a Bayesian network, where we understand $\datindex$ as condition and each decomposition core as a conditional probability distribution.
%	We notice that in this interpretation the direction of the dependency is inversed compared with previous representation of grounding tensors in Figure~\ref{fig:groundingCP}. 


The slices of directed and boolean tensors with respect to incoming variables are basis tensors.
We have thus called $\cpformat$ decomposition with the restiction of directed an boolean leg vectors basis $\cpformat$ decomposition.
Based on this intuition, we can interpret basis $\cpformat$ decomposition by mappings to non-vanishing coordinates of the decomposed tensor.
To start, let us define the number of nonzero coordinates of tensors by the $\ell_0$-norm.

\begin{definition}
    The $\ell_0$-norm counts the nonzero coordinates of a tensor by
    \begin{align*}
        \sparsityof{\hypercore} = \#\big\{ \catindices \, : \, \hypercore_{\catindices }\neq 0 \big\} \, .
    \end{align*}
\end{definition}

The $\ell_0$-norm is not a norm, but at each tensor the limit of $\ell_p$-norms (which are norms for $p\geq1$) for $p \rightarrow 0$.

%%%%%%%%%%%%%%%%%%

% Interpretation
The $\ell_0$ norm is the number of non-vanishing coordinates of a tensor.
We understand the leg cores as the basis encoding of functions mapping to the slices of these coordinates given an enumeration.
This is consistent with the previous analysis of \charef{cha:basisCalculus}, where we characterized boolean and directed cores by the encoding of associated functions.
Based on this idea, we can proof, that any tensor has a directed and boolean $\cpformat$ decomposition with rand $\sparsityof{\hypercore}$.


\begin{theorem}
    \label{the:sparseBasisCP}
    For any tensor $\hypercorewith$ we have
    \begin{align*}
        \bascprankof{\hypercore} = \sparsityof{\hypercore} \, .
    \end{align*}
\end{theorem}
\begin{proof}
    Let us first show, that $\bascprankof{\hypercore} \leq \sparsityof{\hypercore}$.
    We find a map
    \begin{align*}
        \datamap : [\sparsityof{\hypercore}] \rightarrow  \facstates
    \end{align*}
    which image is the set of non-vanishing coordinates of $\hypercorewith$.
    Denoting its image coordinate maps by $\datamapof{\catenumerator}$ we have
    \begin{align*}
        \hypercorewith
        = \sum_{\datindexin} \scalarcoreat{\datamapat(\datindex)} \left(\bigotimes_{\atomenumeratorin} \onehotmapofat{\datamapof{\atomenumerator}(\datindex)}{\catvariableof{\catenumerator}} \right) \, .
    \end{align*}
    This is a basis $\cpformat$ decomposition of size $\sparsityof{\hypercore}$ and we thus have $\bascprankof{\hypercore}\leq\sparsityof{\hypercore}$.

    Conversely, let us show $\bascprankof{\hypercore} \geq \sparsityof{\hypercore}$.
    Any basis $\cpformat$ decomposition of $\hypercore$ with size $r$ would has at most $r$ coordinates different from zero and thus $\sparsityof{\hypercore}\leq r$.
    Thus, there cannot be a $\cpformat$ decomposition with a dimension $r\leq\sparsityof{\hypercore}$.
\end{proof}

%
The next theorem relates the basis $\cpformat$ decomposition with encodings of $\atomorder$-ary relations (see \defref{def:daryRelation}).

\begin{theorem}
    Any boolean tensor $ \hypercorewithin$ is the encoding of a $\atomorder$-ary relation $\exrelation\subset\facstates$ with cardinality
    \begin{align*}
        \cardof{\exrelation} = \bascprankof{\hypercore} \, .
    \end{align*}
\end{theorem}
\begin{proof}
    We fine a basis $\cpformat$ decomposition of $\hypercorewith$ with $\decdim=\bascprankof{\hypercore}$.
    Since $\hypercorewith$ is boolean, and since each $\decindex$ labels a disjoint non-vanishing coordinate (see proof of \theref{the:sparseBasisCP}), the decomposition has a trivial scalar core $\scalarcoreat{\decvariable}=\onesat{\decvariable}$.
    It follows, that
    \begin{align*}
        \hypercorewith = \sum_{\decindexin} \left(\bigotimes_{\catenumeratorin}\legcoreofat{\catenumerator}{\catvariableof{\catenumerator},\indexeddecvariable}\right)
    \end{align*}
    Since the $\cpformat$ decomposition is basis, the slice $\legcoreofat{\catenumerator}{\catvariableof{\catenumerator},\indexeddecvariable}$ is for any $\catenumeratorin$ and $\decindexin$ a basis vector.
    We then define
    \begin{align*}
        \catindexof{\catenumerator}^{\decindex} = \invonehotmapof{\legcoreofat{\catenumerator}{\catvariableof{\catenumerator},\indexeddecvariable}}
    \end{align*}
    and notice, that for the relation
    \begin{align*}
        \exrelation = \{ \shortcatindices^{\decindex}  \, : \, \decindexin\} \subset \facstates \,
    \end{align*}
    we have
    \begin{align*}
        \onehotmapofat{\exrelation}{\shortcatvariables} = \sum_{\decindexin} \left(\bigotimes_{\catenumeratorin}\legcoreofat{\catenumerator}{\catvariableof{\catenumerator},\indexeddecvariable}\right) \, .
    \end{align*}
    This coincides with the above $\cpformat$ decomposition of $\hypercorewith$ and the claim is established.
\end{proof}



\begin{remark}[Matrix Storage of basis $\cpformat$ decompositions]
    \label{rem:matStorageBas}
    % Storage
    The storage demand of any $\cpformat$ decomposition is at most linear in the size and the sum of its leg dimension.
    When we have a basis $\cpformat$ decomposition, this demand can be further improved.
    The basis vectors can be stored by its preimage of the one hot encoding $\onehotmapof{\cdot}$, that is the number of the basis vector in $[\catdim]$.
    This reduces the storage demand of each basis vector to the logarithms of the space dimension without the need of storing the full vector.
% Matrix Representation
    More precisely, we can define a leg selecting variable $\selvariable$ taking values in $[\catorder+1]$ and store a basis $\cpformat$ decomposition of size $\decdim$ by the matrix
    \begin{align*}
        \matrixat{\decvariable,\selvariable} \in \rr^{\datanum \times (\atomorder+1)}
    \end{align*}
    defined for $\decindexin$ and $\atomenumeratorin$ by
    \begin{align*}
        \matrixat{\inddecvar,\selvariable=\atomenumerator} =
        \begin{cases}
            \scalarcoreat{\inddecvar} & \text{if} \quad \atomenumerator = \catorder \\
            \invonehotmapof{\legcoreofat{\atomenumerator}{\catvariableof{\atomenumerator},\indexeddecvariable}} & \text{else} \\% \quad \atomenumerator < \catorder \\
        \end{cases} \, .
    \end{align*}
    This is a common trick to store relational databases.
\end{remark}


\subsect{Basis+ $\cpformat$ decompositions and polynomials}

The basis+ $\cpformat$ decompositions are closely related to monomial decompositions of a tensor, which we will define next.

\begin{definition}
    \label{def:polynomialSparsity}
    A monomial decomposition of a tensor $ \hypercorewithin$ is a set $\sliceset$ of tuples $\slicetupleof{}$ where $\slicescalar\in\rr, \variableset\subset[\atomorder]$ and $\catindexof{\variableset}\in\bigtimes_{\atomenumerator\in\variableset} [\catdimof{\atomenumerator}]$ such that
    \begin{align}
        \label{eq:decIntoMonomials}
        \hypercorewith
        = \sum_{\slicetupleof{}\in\sliceset} \slicescalar \cdot \contractionof{\onehotmapofat{\catindexof{\variableset}}{\catvariableof{\variableset}}}{\shortcatvariables} \, .
    \end{align}
    For any tensor $ \hypercorewithin$ we define its polynomial sparsity of order $\sliceorder$ as
    \begin{align*}
        \slicerankwrtof{\sliceorder}{\hypercore} =
        \min \left\{ \cardof{\sliceset} \, : \,
        \hypercorewith = \sum_{\slicetupleof{}\in\sliceset} \slicescalar \cdot \contractionof{\onehotmapofat{\catindexof{\variableset}}{\catvariableof{\variableset}}}{\shortcatvariables} \, , \, \forall_{\slicetupleof{}\in\sliceset} \cardof{\variableset} \leq \sliceorder \, .
        \right\}
    \end{align*}
\end{definition}


% Explanation of monomials
We refer to the terms in a decomposition \eqref{eq:decIntoMonomials} in \defref{def:polynomialSparsity} as monomials of boolean features, which are enumerated by pairs $(\atomenumerator,\catindexof{\atomenumerator})$ and indicate whether the variable $\catvariableof{\atomenumerator}$ is in state $\catindexof{\atomenumerator}\in[\catdimof{\atomenumerator}]$.
Each such boolean features is represented by the indicator
\begin{align*}
    \indicatorofat{\indexedcatvariableof{\atomenumerator}}{\catvariableof{\atomenumerator}} =
    \onehotmapofat{\catindexof{\atomenumerator}}{\catvariableof{\atomenumerator}} \, .
\end{align*}
The monomial of multiple such boolean features indicates, whether all variables labelled by a set $\variableset$ are in the state $\catvariableof{\variableset}$.
We have
\begin{align*}
    \indicatorofat{\forall{\atomenumerator\in\variableset}: \, \indexedcatvariableof{\atomenumerator}}{\catvariableof{\variableset}}
    = \onehotmapofat{\catindexof{\variableset}}{\catvariableof{\variableset}} = \bigotimes_{\atomenumerator\in\variableset} \onehotmapofat{\catindexof{\atomenumerator}}{\catvariableof{\atomenumerator}}  \, .
\end{align*}
The states of the variables labeled by $\atomenumerator\in[\atomorder]/\variableset$ are not specified in the monomial and the indicators are trivially extended to
\begin{align*}
    \contractionof{\onehotmapofat{\catindexof{\variableset}}{\catvariableof{\variableset}}}{\shortcatvariables}
    = \onehotmapofat{\catindexof{\variableset}}{\catvariableof{\variableset}} \otimes \onesat{\catvariableof{[\atomorder]/\variableset}} \, .
\end{align*}
Since we are working with boolean features, there is no need to consider higher-order powers of individual features, since for any $n\in\nn, \, n\geq1$ and any boolean value $z\in\ozset$ we have $z^n=z$.

% Infinity
For some monomial orders $\sliceorder<\catorder$ there are tensors $\hypercorewith$, which do not have a monomial decomposition of order $\sliceorder$.
In that case the minimum is over an empty set and we define $\slicerankwrtof{\sliceorder}{\hypercore}=\infty$.
We characterize in the next theorem the set of tensors with monomial decompositions of order $\sliceorder$.

\begin{theorem}
    \label{the:polynomialSubspaces}
    For any $\atomorder, \sliceorder$, the set of tensors of $\catorder$ variables with leg dimension $\catdim$, which have a monomial decomposition of order $\sliceorder$, is a linear subspace $\subspaceof{\atomorder,\sliceorder}$ with dimension
    \begin{align*}
        \subspacedimof{\subspaceof{\atomorder,\sliceorder}} \leq  \sum_{s \in [\sliceorder]} \catdim^s \binom{\catorder}{s} \, .
    \end{align*}
\end{theorem}
\begin{proof}
    The set of tensors admitting a monomial decomposition of order $\sliceorder$ is closed under addition and scalar multiplication.
    Specifically, the sum of two such tensors retains a monomial decomposition, formed by concatenating their respective decompositions.
    Scalar multiplication can be performed by a rescaling of each scalar $\slicescalar$ and therefore preserves the decomposition structure.
    Hence, these tensors form a linear subspace.

%    Any sum of tensors with a monomial decomposition of order $\sliceorder$ admits again a monomial decomposition, which is the concatenation of both.
%    The same holds for a scalar multiplication, and thus, the sets of such tensors form a linear subspace.

    To bound the dimension of this subspace, we consider tensors of the form $\contractionof{\onehotmapof{\catindexof{\variableset}}}{\shortcatvariables}$.
    The number of such tensors is given by
    \begin{align*}
        \sum_{s \in [\sliceorder]} \catdim^s \binom{\catorder}{s}  \, .
    \end{align*}
    Since any tensor with a monomial decomposition is a weighted sum of those, this provides an upper bound on the dimension.

    We notice, that the set of slices is in general not linear independent, and therefore forms a frame instead of a linear basis \cite{casazza_introduction_2013}.
    The number of elements in the frame is therefore in general a loose upper bound on the dimension.
\end{proof}


% Infinite rank
\theref{the:polynomialSubspaces} states, that the tensors admitting a monomial decomposition of a small order build a low-dimensional subspace in the $\catdim^\catorder$ dimensional space of tensors, since for $\sliceorder << \catorder $ we have
\begin{align*}
    \subspacedimof{\subspaceof{\atomorder,\sliceorder}} << \catdim^{\catorder} \, .
\end{align*}
If $\sliceorder\geq\catorder$, we always find a monomial decomposition by an enumeration of nonzero coordinates.
In the next theorem, we show that in that case the $\slicerankwrtof{\sliceorder}{\hypercore}$ furthermore coincides with the basis+ $\cpformat$ rank $\baspluscprankof{\hypercore}$.

\begin{theorem}
    For any tensor $\hypercorewithin$ we have
    \begin{align*}
        \slicesparsityof{\hypercore} = \baspluscprankof{\hypercore} \, .
    \end{align*}
    In case of two-dimensional legs, that is $\catdimof{\atomenumerator}=2$ for all $\atomenumeratorin$, we also have
    \begin{align*}
        \bincprankof{\hypercore} = \slicesparsityof{\hypercore}  \, .
    \end{align*}
\end{theorem}
\begin{proof}
    To proof the first claim, we construct a basis+ $\cpformat$ decomposition given a monomial decomposition and vice versa.
    To show $\slicesparsityof{\hypercore} \geq \baspluscprankof{\hypercore}$, let there be an arbitrary tensor $\hypercorewith$ with a monomial decomposition by $\sliceset$ with $\cardof{\sliceset}=m$ and let us enumerate the elements in $\sliceset$ by $\slicetupleof{\decindex}$ for $\decindexin$.
    We define for each $\atomenumeratorin$ the tensors
    \begin{align*}
        \legcoreofat{\atomenumerator}{\decvariable,\catvariableof{\atomenumerator}}
        = \left( \sum_{\decindexin \, : \, \atomenumerator\in\variableset} \onehotmapofat{\decindex}{\decvariable} \otimes \onehotmapofat{\catindexof{\atomenumerator}^{\decindex}}{\catvariableof{\atomenumerator}} \right)
        + \left(\sum_{\decindexin \, : \, \atomenumerator\notin\variableset} \onehotmapofat{\decindex}{\decvariable} \otimes \onesat{\catvariableof{\atomenumerator}} \right)
    \end{align*}
    and
    \begin{align*}
        \scalarcoreat{\decvariable} = \sum_{\decindexin} \slicescalar^{\decindex} \cdot \onehotmapofat{\decindex}{\decvariable}
    \end{align*}
    and notice that
    \begin{align*}
        \hypercorewith
        & = \sum_{\decindexin} \slicescalar^{\decindex} \cdot \contractionof{\onehotmapof{\catindexof{\variableset}^{\decindex}}}{\shortcatvariables} \\
        & = \sum_{\decindexin} \left(  \scalarcoreat{\inddecvar} \cdot \bigotimes_{\atomenumeratorin} \legcoreofat{\atomenumerator}{\inddecvar, \catvariableof{\atomenumerator}} \right) \\
        & = \contractionof{
            \{\scalarcoreat{\decvariable}\} \cup \{\legcoreofat{\atomenumerator}{\decvariable,\catvariableof{\atomenumerator}} \, : \, \atomenumeratorin \}
        }{\shortcatvariables} \, .
    \end{align*}
    By construction this is a basis+ $\cpformat$ decomposition with rank $\decdim$.
    Since any monomial decomposition can be transformed into a basis+ $\cpformat$ decomposition with same rank we have
    \begin{align*}
        \slicesparsityof{\hypercore} \geq \baspluscprankof{\hypercore} \, .
    \end{align*}

    To show $\slicesparsityof{\hypercore} \leq \baspluscprankof{\hypercore}$, let there now be a basis+ $\cpformat$ decomposition of an arbitrary $\hypercorewith$.
    We define for each $\decindexin$
    \begin{align*}
        \variableset^{\decindex} = \{\atomenumeratorin : \legcoreofat{\atomenumerator}{\inddecvar, \catvariableof{\atomenumerator}} \neq \onesat{\catvariableof{\atomenumerator}} \}
        \quad \text{and} \quad
        \catindexof{\variableset}^{\decindex} = \{\invonehotmapof{\legcoreofat{\atomenumerator}{\inddecvar, \catvariableof{\atomenumerator}} } \, : \atomenumerator\in\variableset\}
    \end{align*}
    where by $\invonehotmapof{\cdot}$ we denote the inverse of the one-hot encoding.

    We notice that this is a monomial decomposition of $\hypercorewith$ to the tuple set
    \begin{align*}
        \sliceset = \{(\scalarcoreat{\inddecvar}, \variableset^{\decindex}, \catindexof{\variableset^{\decindex}}^{\decindex} ) \, : \, \decindexin \} \, .
    \end{align*}
    It follows from this that
    \begin{align*}
        \slicesparsityof{\hypercore} \leq \baspluscprankof{\hypercore} \,
    \end{align*}
    and the first claim is shown.

    The second claim follows from the observation, that the set of non-vanishing boolean vectors coincides with the set of one-hot encodings extended by the trivial vector.
    Thus, a $\cpformat$ decomposition with non-vanishing slices is boolean if and only if it is basis+.
    This establishes, that both ranks are equal, since a $\cpformat$ decomposition of minimal rank cannot contain non-vanishing slices.
\end{proof}

\begin{remark}[Sparse representation of propositional formulas]
    When all leg dimensions of a boolean tensor $\hypercore$ are $2$, we can further interpret $\hypercore$ as a logical formula.
    We can use the boolean $\cpformat$ decomposition of any tensor $\sechypercore$ with $\nonzeroof{\sechypercore}=\hypercore$ as a CNF of $\hypercore$.
    Finding the sparsest CNF thus amounts to finding the $\sechypercore$ with minimal $\slicesparsityof{\sechypercore}$ such that $\nonzeroof{\sechypercore}=\hypercore$.
\end{remark}

\begin{remark}[Matrix Storage of basis+ $\cpformat$ decompositions]
    \label{rem:matStorageBasPlus}
    We can adapt the storage format of \remref{rem:matStorageBas} from basis to basis+ $\cpformat$ decompositions.
    To this end, let there be a basis+ $\cpformat$ decomposition of a tensor with scalar core $\scalarcoreat{\decvariable}$ and leg cores $\{\legcoreofat{\atomenumerator}{\catvariableof{\atomenumerator},\decvariable} \, : \, \atomenumeratorin\}$.
    We use a value $z\in\rr/\imageof{\scalarcore}$ distinguished from the coordinates of the scalar core and define a matrix
    \begin{align*}
        \matrixofat{z}{\decvariable,\selvariable} \,
    \end{align*}
    where $\selvariable$ takes values in $[\catorder]$, coordinatewise as
    \begin{align*}
        \matrixofat{z}{\inddecvar,\selvariable=\atomenumerator} =
        \begin{cases}
            \scalarcoreat{\inddecvar} & \text{if} \quad \atomenumerator = \catorder \\
            z & \text{if} \quad \legcoreofat{\atomenumerator}{\catvariableof{\atomenumerator},\indexeddecvariable} = \onesat{\catvariableof{\atomenumerator}} \\
            \invonehotmapof{\legcoreofat{\atomenumerator}{\catvariableof{\atomenumerator},\indexeddecvariable}} & \text{else}
        \end{cases} \, .
    \end{align*}
\end{remark}



\sect{Constructive Bounds on CP Ranks}

After having defined different $\cpformat$ decompositions, let us investigate bounds on their ranks, which proofs rely on explicit core constructions.


\subsect{Cascade of ranks}

%% ? Case of boolean legs
%Especially useful, when the leg dimensions are two, where the slice decomposition shows decomposition of the tensor into monomials.

We start by showing a cascade of bounds of $\cpformat$ ranks, when demanding different leg restrictions as in \defref{def:cpFormats}.

\begin{theorem}
    \label{the:rankCascade}
    For any tensor $\hypercorewithin$ we have
    \begin{align*}
        \cprankof{\hypercore} = \dircprankof{\hypercore} \leq \bincprankof{\hypercore} \leq \baspluscprankof{\hypercore} \leq \bascprankof{\hypercore} \, .
    \end{align*}
\end{theorem}
\begin{proof}
    The equality $\cprankof{\hypercore} = \dircprankof{\hypercore}$ has been established in \lemref{lem:cprankEqualsDir}.
    The further inequalities follow by consecutive subset relations of the set of allowed leg slices in the respective $\cpformat$ decompositions.
    These imply, that any basis $\cpformat$ decomposition of a tensor $\hypercorewith$ is also a basis+ $\cpformat$ decomposition, further that any basis+ $\cpformat$ decomposition is also a boolean $\cpformat$ decomposition and that any boolean $\cpformat$ decomposition is trivially an unrestricted $\cpformat$ decomposition.
    Thus, the ranks are minima of enlarging sets and the claimed rank cascade is established.
\end{proof}

%% Tightness of the Bounds
Let us notice, that the stated bounds are not tight in general.
To give an example, let us consider the tensor $\hypercorewith=\oneswith$, for which we have
\begin{align*}
    \bascprankof{\hypercore} = \sparsityof{\hypercore} = \prod_{\catenumeratorin}\catdimof{\catenumerator} \, .
\end{align*}
Since in the other restricted $\cpformat$ formats we can choose trivial slices to the leg cores, we have
\begin{align*}
    \baspluscprankof{\hypercore} = 1 = \bincprankof{\hypercore} = \dircprankof{\hypercore} = \cprankof{\hypercore} \, .
\end{align*}
The trivial tensor serves thus as an example, where the demand of the the storage format in \remref{rem:matStorageBas} has an exponential overhead compared to the storage format in \remref{rem:matStorageBasPlus}.


\subsect{Operations on $\cpformat$ decompositions}

When using $\cpformat$ decompositons of tensors in practicle applications, such as those investigated in \parref{par:one} and \parref{par:two}, we have to perform numerical manipulations in the form of summations, contractions and normalizations of the represented tensors.
Let us here investigate, how these operations influence the decomposition.

\subsubsect{Summation}

We start with the sum of tensors in a $\cpformat$ decomposition, which can be captured by a concatenation of the slices.

\begin{theorem}
    \label{the:CPrankSumBound}
    For any collections of tensors $\{\hypercoreofat{\selindex}{\catvariableof{\nodes}} : \selindexin\}$ with identical variables and scalars $\lambda^{\selindex} \in \rr$ for $\selindexin$ we have
    \begin{align*}
        \cprankof{\sum_{\selindexin} \lambda^{\selindex} \cdot \hypercoreof{\selindex}} \leq \sum_{\selindexin} \cprankof{\hypercoreof{\selindex}}  \, .
    \end{align*}
    The bound still holds, when we replace on both sides $\cprankof{\cdot}$ by $\bincprankof{\cdot}$, by $\bascprankof{\cdot}$ or by $\baspluscprankof{\cdot}$.
\end{theorem}
\begin{proof}
    Products with scalars do not change the rank, since they just rescale the core $\scalarcore$.
    The sum of $\cpformat$ decomposition is just the combination of all slices, thus the rank is at most additive.
\end{proof}

% Loose upper bounds
Let us notice, that the upper bound is loose in many applications.
For example, if two slice tuples of two decomposed tensors agree on $\catindexof{\variableset},\variableset$, then their sum can be performed by a sum of the corresponding scalar.

%%%%%%%%%%
% 20.3. noon
%%%%%%%%%%

\subsubsect{Contraction}

We continue to show rank bounds for arbitrary contractions by the product of the ranks of contracted tensors.

\begin{theorem}
    \label{the:CPrankContractionBound}
    For any tensor network $\extnetat{\catvariableof{\nodes}}$ on a graph $\graph=(\nodes,\edges)$, we have for any subset $\secnodes\subset\nodes$
    \begin{align*}
        \cprankof{\contractionof{\extnet}{\catvariableof{\secnodes}}} \leq
        \prod_{\edge\in\edges \, : \, \secnodes\cap\edge \neq \varnothing} \cprankof{\hypercoreof{\edge}} \, .
    \end{align*}
    The bound still holds, when we replace on both sides $\cprankof{\cdot}$ by $\bincprankof{\cdot}$, by $\bascprankof{\cdot}$ or by $\baspluscprankof{\cdot}$.
\end{theorem}

Remarkably, in \theref{the:CPrankContractionBound} the upper bound on the CP rank is build only by the ranks of the tensor cores, which have remaining open edges.
We prepare for its proof by first showing the following lemmata.

\begin{lemma}
    \label{lem:sparsityGeneralContraction}
    For any tensors $\hypercoreofat{1}{\catvariableof{\nodesone}}$ and $\hypercoreofat{2}{\catvariableof{\nodestwo}}$ and any set of variables $\secnodes\subset\nodesone\cup\nodestwo$ we have
    \begin{align*}
        \cprankof{\contractionof{\hypercoreof{1},\hypercoreof{2}}{\catvariableof{\secnodes}}} \leq \cprankof{\hypercoreof{1}} \cdot \cprankof{\hypercoreof{2}} \, .
    \end{align*}
    The bound still holds, when we replace on both sides $\cprankof{\cdot}$ by $\bincprankof{\cdot}$, by $\bascprankof{\cdot}$ or by $\baspluscprankof{\cdot}$.
\end{lemma}
\begin{proof}
    Let there be $\cpformat$ decompositions of $\hypercoreofat{1}{\catvariableof{\nodesone}}$ and $\hypercoreofat{2}{\catvariableof{\nodestwo}}$ by
    \begin{align*}
        \hypercoreofat{1}{\catvariableof{\nodesone}}
        = \contractionof{\{\scalarcoreofat{1}{\decvariableof{1}}\}\cup\{\legcoreofat{1,\atomenumerator}{\catvariableof{\atomenumerator},\decvariableof{1}} \, : \, \atomenumerator\in\nodesone\}}{\catvariableof{\nodesone}}
    \end{align*}
    and
    \begin{align*}
        \hypercoreofat{2}{\catvariableof{\nodesone}}
        = \contractionof{\{\scalarcoreofat{2}{\decvariableof{2}}\}\cup\{\legcoreofat{2,\secatomenumerator}{\catvariableof{\secatomenumerator},\decvariableof{2}} \, : \, \secatomenumerator\in\nodestwo\}}{\catvariableof{\nodestwo}} \, .
    \end{align*}
    By linearity of contractions we have
    \begin{align*}
        \contractionof{\hypercoreof{1},\hypercoreof{2}}{\catvariableof{\secnodes}}
        &= \sum_{\decindexof{1}\in\decdimof{1}} \sum_{\decindexof{2}\in\decdimof{2}}
        \scalarcoreofat{1}{\indexeddecvariableof{1}} \cdot \scalarcoreofat{2}{\indexeddecvariableof{2}} \\
        & \quad\quad \cdot \contractionof{\{\legcoreofat{1,\atomenumerator}{\catvariableof{\atomenumerator},\indexeddecvariableof{1}} \, : \, \atomenumerator\in\nodesone\} \cup \{\legcoreofat{2,\secatomenumerator}{\catvariableof{\secatomenumerator},\indexeddecvariableof{2}} \, : \, \secatomenumerator\in\nodestwo\}}{\catvariableof{\secnodes}} \\
        &= \sum_{\decindexof{1}\in\decdimof{1}} \sum_{\decindexof{2}\in\decdimof{2}}
        \scalarcoreofat{1}{\indexeddecvariableof{1}} \cdot \scalarcoreofat{2}{\indexeddecvariableof{2}} \cdot \\
        & \quad\quad \contraction{\{\legcoreofat{1,\atomenumerator}{\catvariableof{\atomenumerator},\indexeddecvariableof{1}} \, : \, \atomenumerator\in\nodesone/\secnodes\} \cup \{\legcoreofat{2,\secatomenumerator}{\catvariableof{\secatomenumerator},\indexeddecvariableof{2}} \, : \, \secatomenumerator\in\nodestwo/\secnodes\}} \cdot \\
        & \quad\quad \bigotimes_{\atomenumerator\in\secnodes} \legcoreofat{\atomenumerator}{\catvariableof{\atomenumerator},\indexeddecvariableof{1},\indexeddecvariableof{2}} \, ,
    \end{align*}
    where we denote
    \begin{align*}
        \legcoreofat{\atomenumerator}{\catvariableof{\atomenumerator},\indexeddecvariableof{1},\indexeddecvariableof{2}}
        = \begin{cases}
              \legcoreofat{1,\atomenumerator}{\catvariableof{\atomenumerator},\indexeddecvariableof{1}}  & \text{if} \quad \atomenumerator \notin \nodestwo \\
              \legcoreofat{2,\atomenumerator}{\catvariableof{\atomenumerator},\indexeddecvariableof{2}}  & \text{if} \quad \atomenumerator \notin \nodesone \\
              \contractionof{\legcoreofat{1,\atomenumerator}{\catvariableof{\atomenumerator},\indexeddecvariableof{1}}, \legcoreofat{2,\atomenumerator}{\catvariableof{\atomenumerator},\indexeddecvariableof{2}}}{\catvariableof{\atomenumerator}}  & \text{else}
        \end{cases} \, .
    \end{align*}
    Note, that since $\atomenumerator\in\secnodes\subset\nodesone\cup\nodestwo$, these slices are well-defined.
    We build a new decomposition variable $\decvariable$ enumerating the summands to indices $[\decdimof{1}]\times[\decdimof{2}]$ and have thus found a $\cpformat$ decomposition of $\contractionof{\hypercoreof{1},\hypercoreof{2}}{\catvariableof{\secnodes}}$ of size $\decdim=\decdimof{1}\cdot\decdimof{2}$.
    This shows the claim in the case of $\cprankof{\cdot}$.

    When the $\cpformat$ decompositions of $\hypercoreof{1}$ and $\hypercoreof{2}$ are boolean, basis or basis+, then the property is preserved in the constructed $\cpformat$ decomposition, since the constructed slices $\legcoreofat{\atomenumerator}{\catvariableof{\atomenumerator},\indexeddecvariableof{1},\indexeddecvariableof{2}}$ are either copies of the leg cores or their contractions and the respective property is preserved in both cases.
    Thus, the constructive rank bounds hold also for $\bincprankof{\cdot}$, $\bascprankof{\cdot}$ and $\baspluscprankof{\cdot}$.
\end{proof}

When one core of the contracted tensor network does not contain variables which are left open, we can drastically sharpen the bound provided by \lemref{lem:sparsityGeneralContraction} as we show next.

\begin{lemma}
    \label{lem:sparsityDisjointContraction}
    For any two tensors $\hypercoreofat{1}{\catvariableof{\nodesone}}$, $\hypercoreofat{2}{\catvariableof{\nodestwo}}$ and any set $\secnodes$ with $\secnodes\cap\nodestwo=\varnothing$ we have
    \begin{align*}
        \cprankof{\contractionof{\{\hypercoreof{1},\hypercoreof{2}\}}{\catvariableof{\secnodes}}} \leq \cprankof{\hypercoreof{1}} \, .
    \end{align*}
    The bound still holds, when we replace on both sides $\cprankof{\cdot}$ by $\bincprankof{\cdot}$, by $\bascprankof{\cdot}$ or by $\baspluscprankof{\cdot}$.
\end{lemma}
\begin{proof}
    As in the proof of \lemref{lem:sparsityGeneralContraction} we assume a $\cpformat$ decomposition of $\hypercoreofat{1}{\catvariableof{\nodesone}}$ and $\hypercoreofat{2}{\catvariableof{\nodestwo}}$ and use the linearity of contractions to get
    \begin{align*}
        \contractionof{\hypercoreof{1},\hypercoreof{2}}{\catvariableof{\secnodes}}
        %&= \sum_{\decindexof{1}\in\decdimof{1}} \sum_{\decindexof{2}\in\decdimof{2}}
        %\scalarcoreofat{1}{\indexeddecvariableof{1}} \cdot \scalarcoreofat{2}{\indexeddecvariableof{2}} \\
        %& \quad\quad \cdot \contractionof{\{\legcoreofat{1,\atomenumerator}{\catvariableof{\atomenumerator},\indexeddecvariableof{1}} \, : \, \atomenumerator\in\nodesone\} \cup \{\legcoreofat{2,\secatomenumerator}{\catvariableof{\secatomenumerator},\indexeddecvariableof{2}} \, : \, \secatomenumerator\in\nodestwo\}}{\catvariableof{\secnodes}} \\
        &= \sum_{\decindexof{1}\in\decdimof{1}} \sum_{\decindexof{2}\in\decdimof{2}}
        \scalarcoreofat{1}{\indexeddecvariableof{1}} \cdot \scalarcoreofat{2}{\indexeddecvariableof{2}} \cdot \\
        & \quad\quad \contraction{\{\legcoreofat{1,\atomenumerator}{\catvariableof{\atomenumerator},\indexeddecvariableof{1}} \, : \, \atomenumerator\in\nodesone/\secnodes\} \cup \{\legcoreofat{2,\secatomenumerator}{\catvariableof{\secatomenumerator},\indexeddecvariableof{2}} \, : \, \secatomenumerator\in\nodestwo/\secnodes\}} \cdot \\
        & \quad\quad \bigotimes_{\atomenumerator\in\secnodes} \legcoreofat{1,\atomenumerator}{\catvariableof{\atomenumerator},\indexeddecvariableof{1}} \, ,
    \end{align*}
    where we used that $\secnodes\cup\nodestwo=\varnothing$.
    By rearranging the sum of $\decindexof{2}$, we have a $\cpformat$ decomposition with decomposition variable $\decvariableof{1}$ and slices
    \begin{align*}
        \scalarcoreat{\indexeddecvariableof{1}}
        & = \sum_{\decindexof{2}\in\decdimof{2}}  \scalarcoreofat{1}{\indexeddecvariableof{1}} \cdot \scalarcoreofat{2}{\indexeddecvariableof{2}} \cdot \\
        & \quad \quad \contraction{\{\legcoreofat{1,\atomenumerator}{\catvariableof{\atomenumerator},\indexeddecvariableof{1}} \, : \, \atomenumerator\in\nodesone/\secnodes\} \cup \{\legcoreofat{2,\secatomenumerator}{\catvariableof{\secatomenumerator},\indexeddecvariableof{2}} \, : \, \secatomenumerator\in\nodestwo/\secnodes\}} \, .
    \end{align*}
    This shows the rank bound for $\cprankof{\cdot}$.
    The properties of the $\cpformat$ decomposition are trivially inherited by the constructed decomposition, since the leg cores of the decomposition of $\hypercoreofat{1}{\catvariableof{\nodesone}}$ are chosen.
    Thus, the rank bounds hold also for any other rank in the claim.
%    We show the lemma by constructing a $\cpformat$ decomposition of $\cprankof{\contractionof{\{\hypercoreof{1},\hypercoreof{2}\}}{\catvariableof{\secnodes}}} $ for any $\cpformat$ decomposition of $\hypercoreof{1}$.
%    Let therefore take any $\cpformat$ decomposition of $\hypercoreof{1}$ consistent of the leg cores $\{\legcoreof{\node} \, : \, \node \in \nodesone \}$ and a scalar core $\scalarcore$.
%    Then we define a new $\scalarcore$ by
%    \[ \tilde{\scalarcore} = \contractionof{\{\scalarcore\}\cup \{\legcoreof{\node} \, : \, \node \in \nodesone , \node \notin \secnodes \} \cup \{\hypercoreof{2}\} }{\decvariable} \, . \]
%    Then, the leg cores $\{\legcoreof{\node} \, : \, \node \in \secnodes \}$ build with the scalar core $\tilde{\scalarcore}$ a $\cpformat$ decomposition of $\contractionof{\{\hypercoreof{1},\hypercoreof{2}\}}{\catvariableof{\secnodes}}$.
%% boolean of basis
%    When the $\cpformat$ decomposition of $\hypercoreof{1}$ was boolean, basis or basis+, this property is also satisfied by the constructed $\cpformat$ decomposition.
%    Thus the bound also holds for the ranks $\bincprankof{\cdot}$ or $\bascprankof{\cdot}$.
\end{proof}

\begin{proof}[Proof of \theref{the:CPrankContractionBound}]
    We partition the edges into the set $\edgesof{1}= \{\edge\in\edges \, : \, \edge\cup\secnodes\neq\varnothing\}$ and $\edgesof{2}= \{\edge\in\edges \, : \, \edge\cup\secnodes=\varnothing\}$.
    We then have
    \begin{align}
        \label{eq:cprankContractionPartition}
        \contractionof{\extnet}{\catvariableof{\secnodes}}
        = \contractionof{
            \contractionof{\{\hypercoreofat{\edge}{\catvariableof{\edge}} \, : \, \edge\in\edgesof{1}\}}{\catvariableof{\bigcup_{\edge\in\edgesof{1}}\edge}},
            \contractionof{\{\hypercoreofat{\edge}{\catvariableof{\edge}} \, : \, \edge\in\edgesof{2}\}}{\catvariableof{\bigcup_{\edge\in\edgesof{2}}\edge}}
        }{\catvariableof{\secnodes}}
    \end{align}
    By an iterative application of \lemref{lem:sparsityGeneralContraction} when including the cores to $\edge\in\edgesof{1}$ after each other to the contraction, we get the bound
    \begin{align*}
        \cprankof{\contractionof{\{\hypercoreofat{\edge}{\catvariableof{\edge}} \, : \, \edge\in\edgesof{1}\}}{\catvariableof{\bigcup_{\edge\in\edgesof{1}}\edge}}}
        \leq \prod_{\edge\in\edgesof{1}} \cprankof{\hypercoreof{\edge}} \, .
    \end{align*}
    With the decomposition \eqref{eq:cprankContractionPartition} and \lemref{lem:sparsityDisjointContraction} we then arrive at the claim
    \begin{align*}
        \contractionof{\extnet}{\catvariableof{\secnodes}} \leq \prod_{\edge\in\edgesof{1}} \cprankof{\hypercoreof{\edge}} \, .
    \end{align*}
    Since the applied lemmata hold also for the restricted $\cpformat$ ranks in the claim, the derived bound is also for those valid.
    %Use delta tensor representation to represent contractions by graphs.
    %We then iterate through the cores and contract them to the previously contracted tensor, where we apply \lemref{lem:sparsityGeneralContraction} when the tensor core has variables left open and \lemref{lem:sparsityDisjointContraction} if not.
\end{proof}


\begin{example}[Composition of formulas with connectives]
    For any formula $\exformula$ we have $1-\exformula$ = $\lnot\exformula$.
    The CP rank bound brings an increase by at most factor $2$ when taking the contraction with $\bencodingof{\lnot}$ which has slice sparsity of $2$.
    This is not optimal, since $\lnot\exformula$ has at most an absolute slice sparsity increase of $1$.

    For any formulas $\exformula$ and $\secexformula$ we have $\exformula\cdot\secexformula = \exformula\land\secexformula$.
    Here the CP rank bounds on contractions can also be further tightened.
\end{example}


\begin{example}[Distributions of independent variables]
    Independence means factorization, conditional independence means sum over factorizations.
    Again, the $\ell_0$ norm is bounded by the product of the $\ell_0$ norm of the factors.
\end{example}


%\subsect{normalizations}
%\red{As a theorem: If any of the above $\cpformat$ decomposition is normable, the normalization has the same CP ranks.
%Especially interesting when learning Bayesian Networks, where each core has a CP bound by the number of datapoints.}


\sect{Sparse Encoding of Functions}

%Using the proof idea of \theref{the:sparseBasisCP}, we can state a more general CP bound on the encoding of functions.

We now state that the basis CP rank of basis encodings is equal to the cardinality of the domain.
The basis CP format can therefore not provide a sparse representation when the factored system contains many categorical variables.

%\subsect{}

\begin{theorem}
    \label{the:bencodingBasCP}
    For any function
    \[ \exfunction : \facstates \rightarrow  \secfacstates \]
    between factored systems we have
    \[ \bascprankof{\bencodingof{\exfunction}} =  \facdim \, . \]
\end{theorem}
\begin{proof}
    The bound follows from \theref{the:sparseBasisCP}, using that $\sparsityof{\bencodingof{\exfunction}}=\facdim$.
\end{proof}

Let us further provide a construction scheme to find a basis $\cpformat$ decomposition of $\bencodingof{\exfunction}$ of size $\facdim$.
We notice that
\begin{align*}
    \bencodingofat{\exfunction}{\secshortcatvariables,\shortcatvariables}
    = \sum_{\shortcatindices\in\facstates} \onehotmapofat{\shortcatindices}{\shortcatvariables} \otimes \onehotmapofat{\exfunctionat{\indexedshortcatvariables}}{\secshortcatvariables} \, .
\end{align*}
We build for $\catenumeratorin$ decomposition variables $\decvariableof{[\atomorder]}$ with $\decdimof{\atomenumerator}=\catdimof{\atomenumerator}$ and define leg cores %a function $\indexinterpretation$ and define leg cores by
\begin{align*}
    \legcoreofat{\catenumerator}{\catvariableof{\catenumerator},\decvariableof{[\atomorder]}} = \identityat{\catvariableof{\catenumerator},\decvariableof{\atomenumerator}}
\end{align*}
and for $\seccatenumerator\in[\secatomorder]$ and $\decindexof{[\atomorder]}$
\begin{align*}
    \legcoreofat{\seccatenumerator}{\seccatvariableof{\seccatenumerator},\indexeddecvariableof{[\atomorder]}}
    = \onehotmapof{\exfunction_{\seccatenumerator}[\shortcatvariables=\decindexof{[\atomorder]}]}{\seccatvariableof{\seccatenumerator}} \, .
\end{align*}
We then have with a trivial scalar core
\begin{align*}
    \bencodingofat{\exfunction}{\secshortcatvariables,\shortcatvariables}
    = \contractionof{
        \{\onesat{\decvariableof{[\atomorder]}}\}\cup
        \{\legcoreofat{\catenumerator}{\catvariableof{\atomenumerator},\decvariableof{[\atomorder]}} \, : \, \catenumeratorin\} \cup
        \{\legcoreofat{\seccatenumerator}{\seccatvariableof{\seccatenumerator},\decvariableof{[\atomorder]}} \, : \, \seccatenumerator\in[\secatomorder]\}
    }{\secshortcatvariables,\shortcatvariables} \, .
\end{align*}
This is a basis $\cpformat$ decomposition of size $\facdim$.

% Extension by rank cascade
In combination with \theref{the:rankCascade}, \theref{the:bencodingBasCP} also provides bounds on all other $\cpformat$ ranks defined in \defref{def:cpFormats}.
This is obvious, since basis leg slices are the most restrictive properties compared with boolean, directed or basis+.


We restate \theref{the:functionImageDecompositionContraction} as a basis $\cpformat$ decomposition bound.

\begin{theorem}
    \label{the:functionDecompositionBasisCP}
    Let $\exfunction$ and be a function between factored systems
    \begin{align*}
        \exfunction : [\catdim] \rightarrow  \facstates
    \end{align*}
    and $\exfunctionof{\atomenumerator}$ as in \theref{the:functionImageDecompositionContraction}.
    Then $\bencodingofat{\exfunction}{\shortcatvariables,\catvariable}$ has a basis $\cpformat$ decomposition with decomposition index $\catvariable$, trivial slices $\onesat{\catvariable}$ leg vectors $\bencodingofat{\exfunctionof{\catenumerator}}{\catvariableof{\catenumerator},\catvariable}$, that is
    \begin{align*}
        \bencodingofat{\exfunction}{\catvariable,\shortcatvariables}
        = \contractionof{\{\onesat{\catvariable}\} \cup \{\bencodingofat{\exfunctionof{\catenumerator}}{\catvariableof{\catenumerator},\catvariable}\,:\,\catenumeratorin}{\shortcatvariables}
    \end{align*}
\end{theorem}
\begin{proof}
    The claimed decomposition directly follows from \theref{the:functionImageDecompositionContraction}, since the trivial scalar core $\scalarcoreat{\catvariable}=\onesat{\catvariable}$ does not influence the contraction and can be omitted.
%    We interpret the decompositon as a basis $\cpformat$ decomposition, after adding a trivial scalar core $\scalarcoreat{\catvariable}=\onesat{\catvariable}$ to the contraction.
\end{proof}

Basis $\cpformat$ decompositions can be constructed by understanding the variable $\indvariableof{\insymbol}$ of the basis encoding of a function $\exfunction:\inset \rightarrow \outset$ as the slice selection variable.

\begin{example}[Empirical distributions, see \theref{the:empCPRep}]
    \label{exa:empDistCP}
    Let there be a data map
    \begin{align*}
        \datamap : [\datanum] \rightarrow \facstates \, .
    \end{align*}
    We can use \theref{the:functionDecompositionBasisCP} to find a tensor network representation fo $\bencodingof{\datamap}$ as
    \begin{align*}
        \bencodingofat{\datamap}{\shortcatvariables,\datvariable}
        = \contractionof{
            \{\bencodingofat{\datamapof{\atomenumerator}}{\catvariableof{\atomenumerator},\datvariable} : \atomenumeratorin \}
        }{\shortcatvariables,\datvariable} \, .
    \end{align*}
    This representation is a basis $\cpformat$ decomposition, when adding trivial scalar core.
    This provides also a basis $\cpformat$ decomposition for the empirical distribution, since normalization can be done by setting a slice core to $\frac{1}{\datanum}\onesat{\datvariable}$.
\end{example}

\begin{example}{Exponential families}
    \label{exa:expFamCP}
    The statistic has a $\cpformat$ decomposition with rank by the cardinality of states, that is
    \begin{align*}
        \bencodingofat{\sstat}{\headvariableof{[\seldim]},\shortcatvariables}
        = \contractionof{
            \{ \bencodingofat{\sstatcoordinateof{\selindex}}{\headvariableof{\selindex},\shortcatvariables} \, : \, \selindexin \}
        }{\headvariableof{[\seldim]},\shortcatvariables} \, .
    \end{align*}
\end{example}

% Now a image central bound
While \theref{the:bencodingBasCP} and \theref{the:functionDecompositionBasisCP} provide $\cpformat$ rank bounds based on the domain factored system, we can also show in the next theorem a bound using the structure of the image.

\begin{theorem}
    Let $\exfunction : \inset \rightarrow \outset$ be an arbitrary function and let us consider for each $\exfunctionimageelement\in\imageof{\exfunction}$ the indicator
    \begin{align*}
        \onesofat{\exfunction=\exfunctionimageelement}{\indvariableof{\insymbol}} =
        \begin{cases}
            1 & \text{if} \quad \exfunctionat{\indexedindvariableof{\insymbol}}=y \\
            0 & \text{else} \, .
        \end{cases}
    \end{align*}
    The basis+ rank of the basis encoding of $\exfunction$ then obeys the bound
    \begin{align*}
        \cprankof{\bencodingof{\exfunction}} \leq \sum_{\exfunctionimageelement\in\imageof{\exfunction}} \cprankof{\onesof{\exfunction=\exfunctionimageelement} } \, .
    \end{align*}
    The bound still holds, when we replace on both sides $\cprankof{\cdot}$ by $\bincprankof{\cdot}$, by $\bascprankof{\cdot}$ or by $\baspluscprankof{\cdot}$.
\end{theorem}
\begin{proof}
    We have
    \begin{align*}
        \bencodingofat{\exfunction}{\indvariableof{\outsymbol},\indvariableof{\insymbol}}
        = \sum_{\exfunctionimageelement\in\imageof{\exfunction}} \onehotmapofat{\indexinterpretationat{\exfunctionimageelement}}{\indvariableof{\outsymbol}}
        \otimes \onesofat{\exfunction=\exfunctionimageelement}{\indvariableof{\insymbol}} \, .
    \end{align*}
    For any $\exfunctionimageelement\in\imageof{\exfunction}$ it is obvious that
    \begin{align*}
        \cprankof{\onehotmapofat{\indexinterpretationat{\exfunctionimageelement}}{\indvariableof{\outsymbol}}
        \otimes \onesofat{\exfunction=\exfunctionimageelement}{\indvariableof{\insymbol}}}
        = \cprankof{\onesofat{\exfunction=\exfunctionimageelement}{\indvariableof{\insymbol}}} \, ,
    \end{align*}
    which also holds true for the other bounds in the claim.
    We then use the summation bound of \theref{the:CPrankSumBound} to get
    \begin{align*}
        \cprankof{\bencodingofat{\exfunction}{\indvariableof{\outsymbol},\indvariableof{\insymbol}}}
        &\leq \sum_{\exfunctionimageelement\in\imageof{\exfunction}} \cprankof{\onehotmapofat{\indexinterpretationat{\exfunctionimageelement}}{\indvariableof{\outsymbol}}
        \otimes \onesofat{\exfunction=\exfunctionimageelement}{\indvariableof{\insymbol}}} \\
        &\leq  \sum_{\exfunctionimageelement\in\imageof{\exfunction}} \cprankof{\onesofat{\exfunction=\exfunctionimageelement}{\indvariableof{\insymbol}}} \, .
    \end{align*}
    Again, the bound still hold for the other ranks in the claim.
    % For each $\exfunctionimageelement\in\imageof{\exfunction}$ we represent $\onehotmapofat{\indexinterpretationat{\exfunctionimageelement}}{\headvariableof{\exfunction}}$ in an basis+ CP format with $\baspluscprankof{\onesof{\exfunction=\exfunctionimageelement} } $ summands and arrive at a basis+ $\cpformat$ decomposition of $\bencodingof{\exfunction}$ with $\sum_{\exfunctionimageelement\in\imageof{\exfunction}} \baspluscprankof{\onesof{\exfunction=\exfunctionimageelement} } $ summands.
\end{proof}

The above claim still holds when replacing $\baspluscprankof{\cdot}$ with the ranks $\bascprankof{\cdot}$ or $\bincprankof{\cdot}$.
For the rank $\bascprankof{\cdot}$ it leads to the bound of \theref{the:bencodingBasCP}, since summing the number of non zero coordinators of the indicators is the cardinality of the domain.

\begin{example}[Propositional formulas]
    Let us now illustrate how the above representation scheme can be leveraged for the sparse representation of propositional formulas.
    For an arbitrary propositional formula $\exformula$ we have $\imageof{\exformula}\subset\ozset$ and the indicators
    \begin{align*}
        \onesofat{\exformula=1}{\shortcatvariables} = \formulaat{\shortcatvariables} \quad \text{and} \quad
        \onesofat{\exformula=0}{\shortcatvariables} = \lnot\formulaat{\shortcatvariables} = \onesat{\shortcatvariables} - \formulaat{\shortcatvariables} \, .
    \end{align*}

    For the conjunction $\land[\catvariableof{0},\catvariableof{1}] = \catvariableof{0} \land \catvariableof{1}$ we have
    \begin{align*}
        \bencodingofat{\land}{\catvariableof{0},\catvariableof{1}}
        = \tbasisat{\headvariableof{\land}} \otimes \onehotmapofat{1,1}{\catvariableof{0},\catvariableof{1}}
        + \fbasisat{\headvariableof{\land}} \otimes (\onesat{\catvariableof{0},\catvariableof{1}} - \onehotmapofat{1,1}{\catvariableof{0},\catvariableof{1}})
    \end{align*}
    and thus
    \begin{align*}
        \baspluscprankof{\bencodingof{\land}} \leq 3
    \end{align*}
    while $\bascprankof{\bencodingof{\land}} = 4$.

    We can even generalize this observation to $\catorder$-ary conjunctions $\land\left[\shortcatvariables\right]=\catvariableof{0}\land\ldots\land\catvariableof{\catorder-1}$ (see \remref{rem:naryConnectives})
    \begin{align*}
    {\land}[\shortcatvariables]
        = \bigotimes_{\catenumeratorin} \tbasisat{\catvariableof{\catenumerator}}
        \quad \text{and} \quad
        {\lnot\land}[\shortcatvariables] = \onesat{\shortcatvariables} - \bigotimes_{\catenumeratorin} \tbasisat{\catvariableof{\catenumerator}}
    \end{align*}
    and thus
    \begin{align*}
        \bencodingofat{\land}{\shortcatvariables} =
        \tbasisat{\headvariableof{\land}} \otimes \left(\bigotimes_{\catenumeratorin} \tbasisat{\catvariableof{\catenumerator}}\right)
        + \fbasisat{\headvariableof{\land}} \otimes \left(\bigotimes_{\catenumeratorin} \tbasisat{\catvariableof{\catenumerator}}\right)
    \end{align*}
    Thus, while the basis $\cpformat$ rank is $\bascprankof{\bencodingof{\land}}=2^{\catorder}$, the basis+ rank is bounded by $3$, independently of $\catorder$.
\end{example}






\sect{Optimization of sparse tensors}

Let us now study the problem of searching for the maximal coordinate in a tensor represented by a monomial decomposition.
Given a tensor $\hypercorewith$ we state this as the problem:
\begin{align}
    \tag{$\probtagtypeinst{\mathrm{max}}{\hypercore}$}\label{prob:maxCoordinate}
    \argmax_{\shortcatindices\in\facstates} \hypercoreat{\indexedshortcatvariables}
\end{align}

\probref{prob:maxCoordinate} can be reformulated as optimization over the standard simplex
\begin{align*}
    \meansetof{\mlnmintermsymbol} = \convhullof{\onehotmapofat{\shortcatindices}{\shortcatvariables} \, : \, \shortcatindices\in\facstates}
\end{align*}
as
\begin{align*}
    \argmax_{\meanparamat{\selvariableof{[\catorder]}}\in\meansetof{\mlnmintermsymbol}} \contraction{\meanparam,\hypercore} \, .
\end{align*}

\begin{example}[Mode search in exponential families]
    % \red{This transforms the mean parameter polytope by contracting with some core, here the selection encoding of the statistic!}
    Given a statistic $\sstat$, a canonical parameter $\canparam$ and a boolean base measure $\basemeasure$, the mode search problem for the member $\expdistof{\sstat,\canparam,\basemeasure}$ of the exponential family $\expfamilyof{\sstat,\basemeasure}$ is
    \begin{align*}
        \max_{\shortcatindices\in\atomstates \, : \, \basemeasureat{\indexedshortcatvariables}=1} \contraction{\sencsstatat{\indexedshortcatvariables,\selvariable},\canparamat{\selvariable}}
        = \max_{\meanparam\in\meansetof{\sstat,\basemeasure}} \contraction{\meanparamat{\selvariable},\canparamat{\selvariable}} \, .
    \end{align*}
    Such mode search problems have appeared as generic MAP queries (see \charef{cha:probReasoning}).
    In \charef{cha:networkReasoning} we have discussed them for the specific cases of hybrid logic networks and grafting proposal distributions.
% Appearance of mode search
%    The search for maximal coordinates appears in various reasoning tasks:
%    \begin{itemize}
%        \item MAP query as mode search of MLN: $\hypercore$ is the contraction of evidence with the distribution, leaving the query variables open.
%        \item Grafting as mode search of proposal distribution: $\hypercore$ is the contraction of the gradient of the likelihood with the basis encoding of the hypothesis.
%    \end{itemize}
%Both tasks have been formulated as mode search problems in exponential families.
\end{example}



\subsect{Unconstrained Binary Optimization}

For leg dimensions $\catdimof{\atomenumerator}=2$, \probref{prob:maxCoordinate} is known as the unconstrained binary optimization.
\probref{prob:maxCoordinate} is a Higher-Order Unconstrained Binary Optimization (HUBO), when $\hypercorewith$ has a when $\hypercore$ has a monomial decomposition (see \defref{def:polynomialSparsity}) with $\cardof{\variablesetof{\decindex}}\leq\sliceorder$ for all $\decindexin$, that is when $\slicerankwrtof{\sliceorder}{\hypercore}<\infty$.

\begin{definition}
    Let $\hypercorewith$ be a tensor with a monomial decomposition $\enumeratedslices$, where $\max_{\decindexin}\cardof{\variablesetof{\decindex}}=\sliceorder$.
    Se then call \probref{prob:maxCoordinate} a $\sliceorder$-Order Unconstrained Binary Optimization (HUBO), which we denote as
    \begin{align}
        \tag{$\probtagtypeinst{\mathrm{HUBO}}{\hypercore}$}\label{prob:HUBO}
        \argmax_{\shortcatindices\in\atomstates} \quad
        \sum_{\decindexin} \slicescalar^{\decindex} \contractionof{\onehotmapofat{\catindexof{\variablesetof{\decindex}}^{\decindex}}{\catvariableof{\variablesetof{\decindex}}}}{\indexedshortcatvariables} \, .
    \end{align}
%    The binary optimization of a tensor $\hypercorewith\in\atomstates$ is the problem
%    \begin{align}\tag{$\probtagtypeinst{\mathrm{HUBO}}{\hypercore}$}\label{prob:HUBO}
%        \argmax_{\shortcatindices\in\atomstates} \quad \hypercoreat{\indexedshortcatvariables}
%    \end{align}
%    We call Problem~\ref{prob:HUBO} a Higher-Order Unconstrained Binary Optimization (HUBO) problem of order $\sliceorder$ and sparsity $\slicerankwrtof{\sliceorder}{\hypercore}$, when $\hypercore$ has a monomial decomposition (see \defref{def:polynomialSparsity}) with $\cardof{\variablesetof{\decindex}}\leq\sliceorder$ for all $\decindexin$, that is when $\slicerankwrtof{\sliceorder}{\hypercore}<\infty$.
\end{definition}

\begin{remark}[Leg dimensions larger than 2]
    We demanded leg dimensions $\catdimof{\atomenumerator}=2$ to have boolean valued variables $\catvariableof{\catenumerator}$, which is required to connect with the formalism of binary optimization.
    Categorical variables with larger dimensions can be represented by atomization variables, which are created by contractions with categorical constraint tensors (see \secref{sec:categoricalTN}).
\end{remark}

% Interpretation of sparsity
The sparsity $\slicerankwrtof{\sliceorder}{\hypercore}$ is the minimal number of monomials, for which a weighted sum is equal to $\hypercore$.
Thus we interpret \probref{prob:HUBO} as searching for the maximum in a polynomial consistent of $\slicerankwrtof{\sliceorder}{\hypercore}$ monomial terms.
%\red{Each monomial is also refered to as potential.}

\probref{prob:HUBO} is called Quadratic Unconstrained Binary Optimization problems, if $\sliceorder=2$.
We can transform certain Higher-Order Unconstrained Binary Optimization (HUBO) problems into Quadratic Unconstrained Binary Optimization (QUBO) problems by introducing auxiliary variables.
An example of such an transform is provided by the next lemma.

%% Slack variables
\begin{lemma}
    \label{lem:monomialToQUBO}
    For any $\atomindices\in[2]$ and $\variableset\subset[\atomorder]$ we have
    \begin{align*}
        \left( \prod_{\atomenumerator\in\variableset} \atomlegindexof{\atomenumerator } \right)  \left(  \prod_{\atomenumerator\notin\variableset} (1- \atomlegindexof{\atomenumerator }) \right) =
        \max_{\slackvariable\in[2]} \slackvariable \cdot 2 \cdot \left( \sum_{\atomenumerator\in\variableset}\atomlegindexof{\atomenumerator}  - \cardof{\variableset} - \sum_{\atomenumerator\notin\variableset}\atomlegindexof{\atomenumerator} + \frac{1}{2} \right) \, . % Alternative: no factor 2, but + 1 instead of +1/2 (->pyqubo)
    \end{align*}
\end{lemma}
\begin{proof} %Proof by case distinction
    Only if $\atomlegindexof{\atomenumerator}=1$ for $\atomenumerator\in\variableset$ and $\atomlegindexof{\atomenumerator}=0$ else we have
    \[ \left( \sum_{\atomenumerator\in\variableset}\atomlegindexof{\atomenumerator}  - \cardof{\variableset} - \sum_{\atomenumerator\notin\variableset}\atomlegindexof{\atomenumerator} + \frac{1}{2} \right) \geq 0 \, . \]
    In this case the maximum is taken for $\slackvariable=1$ and we have
    \[ \max_{\slackvariable\in[2]} \slackvariable \cdot 2 \cdot \left( \sum_{\atomenumerator\in\variableset}\atomlegindexof{\atomenumerator}  - \cardof{\variableset} - \sum_{\atomenumerator\notin\variableset}\atomlegindexof{\atomenumerator} + \frac{1}{2} \right)
    = 1 = \left( \prod_{\atomenumerator\in\variableset} \atomlegindexof{\atomenumerator } \right)  \left(  \prod_{\atomenumerator\notin\variableset} (1- \atomlegindexof{\atomenumerator }) \right) \, . \]
    In all other cases, the maximum is taken for $\slackvariable=0$ and thus vanishes, that is
    \[ \max_{\slackvariable\in[2]} \slackvariable \cdot 2 \cdot \left( \sum_{\atomenumerator\in\variableset}\atomlegindexof{\atomenumerator}  - \cardof{\variableset} - \sum_{\atomenumerator\notin\variableset}\atomlegindexof{\atomenumerator} + \frac{1}{2} \right)
    = 0 = \left( \prod_{\atomenumerator\in\variableset} \atomlegindexof{\atomenumerator } \right)  \left(  \prod_{\atomenumerator\notin\variableset} (1- \atomlegindexof{\atomenumerator }) \right) \, . \]
    Thus, the claim holds in all cases.
\end{proof}

\subsect{Integer Linear Programming}

Let us now show how optimization problems can be represented as linear programming problems.
% State vector
To this end, we understand each index tuple $\shortcatindices\in\facstates$ as a vector $\statevectorofat{\shortcatindices}{\selvariable}\in\rr^{\catorder}$ with coordinates
\begin{align*}
    \statevectorofat{\shortcatindices}{\selvariable=\catenumerator} = \catindexof{\catenumerator} \, .
\end{align*}

\begin{definition}
    The integer linear program (ILP) of $\matrixat{\datvariable,\selvariable}\in\rr^{n \times d}$, $\rhssymbol[\datvariable]\in\rr^{n}$ and $c\in\rr^{\catorder}$ is the problem
    \begin{align}
        \tag{$\probtagtypeinst{\mathrm{ILP}}{\objectivesymbol,\exmatrix,\rhssymbol}$}
        \argmax_{\shortcatindices\in\facstates} \contraction{\objectivesymbol[\selvariable], \statevectorofat{\shortcatindices}{\selvariable}}
        \quad \text{subject to } \quad \contraction{\matrixat{\datvariable,\selvariable},\statevectorofat{\shortcatindices}{\selvariable}} \prec \rhssymbol[\datvariable] \, ,
    \end{align}
    where by $\prec$ we denote partial ordering of tensors (see \defref{def:partialOrder}).
\end{definition}


We now show that any binary optimization problem of a tensor can be transformed into a integer linear program, given a monomial decomposition of the tensor $\hypercorewith$ by $\sliceset=\enumeratedslices$.
For this we choose state indices by vectors
\begin{align*}
    \seccatindex_{[\catorder+\decdim]} = \catindexof{0},\ldots,\catindexof{\catorder-1},\slackindexof{0},\ldots\slackindexof{\decdim-1} \in \left(\bigtimes_{\catenumeratorin}[2]\right) \times  \left(\bigtimes_{\decindexin}[2]\right) \, ,
\end{align*}
that is we added for each monomial an index $\slackindexof{\decindex}$, which will represent the evaluations of the respective monomial.


We furthermore define a vector $\objofat{\sliceset}{\selvariable}$, where $\selvariable$ takes values in $[\catorder+\decdim]$, as
\begin{align}
    \label{eq:ilpPotential}
    \objofat{\sliceset}{\indexedselvariable} =
    \begin{cases}
        \slicescalarof{\selindex-\catorder} & \text{if} \quad \selindex>\catorder \\% \decindexin \text{ we have } \selindex = \catorder + \decindex \\
        0 & \text{else}
    \end{cases} \, .
\end{align}

To construct a matrix $\matrixat{\datvariable,\selvariable}$ and a vector $b[\datvariable]$ to the monomial decomposition $\sliceset$, we now introduce a variable $\datvariable$ enumerating linear inequalities, which takes values in $[\datanum]$, where
\[ \datanum =  \sum_{\decindexin} \left(\cardof{\variablesetof{\decindex}} +1\right) \, . \]
We define for each $\decindexin$ an auxiliary number
\[ \datanum_{\decindex} = \sum_{\tilde{\decindex}=0}^{\decindex} \left(\cardof{\variablesetof{\tilde{\decindex}}} +1\right) \]
and further enumerate the set $\variablesetof{\decindex}$ by a function $\indexinterpretation: [\cardof{\variablesetof{\decindex}}] \rightarrow \variablesetof{\decindex}$.

We then construct a matrix $\matrixofat{\sliceset}{\datvariable,\selvariable}$, were for $\selindex\in[\catorder+\decdim]$, $\decindexin$ and $\datindex\in[\cardof{\variablesetof{\decindex}}]$ we have
\begin{align}
    \label{eq:ilpMatrix}
    \matrixofat{\sliceset}{\datvariable=\datanum_{\decindex}+\datindex,\indexedselvariable} =
    \begin{cases}
        1 - 2 \cdot \catindexof{\indexinterpretationat{\datindex}}^{\decindex} & \text{if} \quad \datindex < \cardof{\variablesetof{\decindex}}, \,\, \datindex=\selindex \text{  and  } \selindex = \indexinterpretationat{\datindex} \\ % Upper bounds on z, x position
        1  & \text{if} \quad \datindex < \cardof{\variablesetof{\decindex}} \text{  and  } \selindex = \catorder + \decindex \\ % Upper bound on z, z position
        -\catindexof{\indexinterpretationat{\datindex}}^{\decindex} & \text{if }  \datindex = \cardof{\variablesetof{\decindex}}    \text{ and }  \selindex=\indexinterpretationat{\datindex}  \\ % Last condition on $x$
        -1 & \text{if }  \datindex = \cardof{\variablesetof{\decindex}}  \text{ and }  \selindex = \catorder + \decindex \\ % Last condition on $x$
        0 & \text{else} \\
    \end{cases} \, .
\end{align}
%All further coordinates of $\matrixofat{\sliceset}{\datvariable,\selvariable}$ not reached by this construction are set to $0$.
Similarly, we define $\rhsofat{\sliceset}{\datvariable}$ as the vector which nonvanishing coordinates are for $\decindexin$ at
\begin{align}
    \label{eq:ilpRhs}
    \rhsofat{\sliceset}{\datvariable=\datanum_{\decindex}+\datindex} =
    \begin{cases}
        1 - \catindexof{\indexinterpretationat{\datindex}}^{\decindex} & \text{if }  \datindex < \cardof{\variablesetof{\decindex}} \\ % Upper bounds on z
        %\datindex = \catenumerator \text{ and } \catindexof{\catenumerator}^{\decindex} = 0 \\ % First $\catorder$ conditions, on x
        -1 + \cardof{\{\catenumerator\in\variablesetof{\decindex} \, : \,  \catindexof{\catenumerator}^{\decindex} = 1\}}  & \text{if }  \datindex = \cardof{\variablesetof{\decindex}} \\ % \text{ and }  \selindex = \catorder + \decindex \\ % Last condition on $x$
    \end{cases} \, .
\end{align}


% Intuition
Informally, we pose for each tuple $\slicetupleof{}$ $\cardof{\variableset}+1$ linear equations.
The first $\cardof{\variableset}$ enforce, that the slice representing variable $\slackvariable$ is zero once a leg is $0$.
The last enforces that the slice representing variable is $1$.
We prove this claim more formally in the next theorem.

\begin{theorem}
    Given a monomial decomposition $\sliceset=\enumeratedslices$ of a tensor $\hypercore$, let  $\seccatindex^{ILP,\sliceset}$ be a solution of the integer linear program defined by the matrix and vectors in equations \eqref{eq:ilpPotential}, \eqref{eq:ilpMatrix} and \eqref{eq:ilpRhs}.
    Then we have
    \begin{align*}
        \restrictionofto{\seccatindex^{ILP,\sliceset}}{[\catorder]} \in \argmax_{\shortcatindices\in\atomstates} \quad \hypercoreat{\indexedshortcatvariables} \, .
    \end{align*}
    where by  $\restrictionofto{\seccatindex^{ILP,\sliceset}}{[\catorder]}$ we denote the restriction of the index tuple $\seccatindex^{ILP,\sliceset}$ to the first $\catorder$.
\end{theorem}
\begin{proof}
    We show that the linear constraints by
    \begin{align*}
        \contraction{\matrixofat{\sliceset}{\datvariable,\selvariable},\statevectorofat{\seccatindexof{[\atomorder+\decdim]}}{\selvariable}} \prec \rhsofat{\sliceset}{\datvariable}
    \end{align*}
    are satisfied for a vector $\seccatindexof{[\atomorder+\decdim]}=(\catindexof{[\atomorder]},\slackindexof{[\decdim]})$, if and only if for all $\decindexin$ the product constraints
    \begin{align}
        \label{eq:slackindexInequalityILPProof}
        \slackindexof{\decindex}
        = \left( \prod_{\catenumerator\in\variablesetof{\decindex} \, , \,  \catindexof{\catenumerator}^{\decindex} = 0} (1 - \catindexof{\catenumerator} \right)
        \cdot \left( \prod_{\catenumerator\in\variablesetof{\decindex} \, , \,  \catindexof{\catenumerator}^{\decindex} = 1}  \catindexof{\catenumerator} \right) \,
    \end{align}
    hold.
    We will see, that the linear constraints where $\datvariable$ takes indices in $\datanum_{\decindex}+[\cardof{\variablesetof{\decindex}}]$ are equivalent to the upper bound on $\slackindexof{\decindex}$ and the constraint to $\datvariable=\datanum_{\decindex}+\cardof{\variablesetof{\decindex}}$ is equivalent to an lower bound on $\slackindexof{\decindex}$.
    To show the upper bound, we notice that for any $\datindex\in[\cardof{\variablesetof{\decindex}}]$ the constraint $\datvariable = \datanum_{\decindex}+\datindex$ is
    \begin{align*}
        \slackindexof{\decindex} \leq
        \begin{cases}
            \catindexof{\indexinterpretationat{\datindex}}  & \text{if} \quad \catindexof{\indexinterpretationat{\datindex}}^\decindex = 1 \\
            (1- \catindexof{\indexinterpretationat{\datindex}}) & \text{if} \quad \catindexof{\indexinterpretationat{\datindex}}^\decindex = 0 \\
        \end{cases} \, .
    \end{align*}
    Thus, whenever a factor on the right side of \eqref{eq:slackindexInequalityILPProof} is $0$, we have $\slackindexof{\decindex}=0$ if the respective constraint is satisfied.
    We conclude, that
    \begin{align*}
        \slackindexof{\decindex}
        \leq \left( \prod_{\catenumerator\in\variablesetof{\decindex} \, , \,  \catindexof{\catenumerator}^\decindex = 0} (1 - \catindexof{\catenumerator}) \right)
        \cdot \left( \prod_{\catenumerator\in\variablesetof{\decindex} \, , \,  \catindexof{\catenumerator}^\decindex = 1}  \catindexof{\catenumerator} \right) \, .
    \end{align*}
    To show the lower bound, we have the constraint to $\datvariable=\datanum_{\decindex}+\cardof{\variablesetof{\decindex}}$ by
    \begin{align*}
        \slackindexof{\decindex} \geq 1 -
        \left( \sum_{\catenumerator\in\variablesetof{\decindex} \, , \,  \catindexof{\catenumerator}^\decindex = 0} \catindexof{\catenumerator} \right)
        + \left( \sum_{\catenumerator\in\variablesetof{\decindex} \, , \,  \catindexof{\catenumerator}^\decindex = 1}  (\catindexof{\catenumerator} - 1 )\right) \, .
    \end{align*}
    The right side of this inequality is $1$, if and only if all factors on the right side of \eqref{eq:slackindexInequalityILPProof} are $1$, and less or equal to $0$ else.
    Thus, whenever this constraint is satisfied, we have
    \begin{align*}
        \slackindexof{\decindex}
        \geq \left( \prod_{\catenumerator\in\variablesetof{\decindex} \, , \,  \catindexof{\catenumerator}^\decindex = 0} (1 - \catindexof{\catenumerator}) \right)
        \cdot \left( \prod_{\catenumerator\in\variablesetof{\decindex} \, , \,  \catindexof{\catenumerator}^\decindex = 1}  \catindexof{\catenumerator} \right)  \, .
    \end{align*}

    In summary, the equation \eqref{eq:slackindexInequalityILPProof} holds, if and only if the constraints where $\datvariable$ takes indices in $\datanum_{\decindex}+[\cardof{\variablesetof{\decindex}}+1]$ are satisfied.
%    In summary we arrive at
%    \[ \slackindexof{\decindex}
%    = \left( \prod_{\catenumerator\in\variablesetof{\decindex} \, , \,  \catindexof{\catenumerator}^\decindex = 0} (1 - \catindexof{\catenumerator}^\decindex) \right)
%    \cdot \left( \prod_{\catenumerator\in\variablesetof{\decindex} \, , \,  \catindexof{\catenumerator}^\decindex = 1}  \catindexof{\catenumerator}^\decindex \right)
%    \]
%    if and only if the indices $\seccatindex_{[\catorder+\decdim]}$ are feasible.

    This characterization of the constraints implies, that for any $\shortcatindices\in\facstates$ there is exactly one feasible index $\seccatindex_{[\catorder+\decdim]}$ with $\restrictionofto{(\seccatindex_{[\catorder+\decdim]})}{[\catorder]}=\shortcatindices$, and the objective takes for this index the value
    \begin{align*}
        \contraction{\objectivesymbol[\selvariable], \statevectorofat{\seccatindex_{[\catorder+\decdim]}}{\selvariable}}
        &= \sum_{\decindexin} \slicescalarof{\decindex} \cdot \slackindexof{\decindex} \\
        &= \sum_{\decindexin} \slicescalarof{\decindex} \cdot \left( \prod_{\catenumerator\in\variablesetof{\decindex} \, , \,  \catindexof{\catenumerator}^\decindex = 0} (1 - \catindexof{\catenumerator}^\decindex) \right)
        \cdot \left( \prod_{\catenumerator\in\variablesetof{\decindex} \, , \,  \catindexof{\catenumerator}^\decindex = 1}  \catindexof{\catenumerator}^\decindex \right)  \\
        & = \hypercoreat{\indexedshortcatvariables} \, .
    \end{align*}
    Therefore, any solution of the ILP reduced to the first $\catorder$ indices corresponding with the axis of $\hypercore$, is a solution of the binary optimization problem to $\hypercore$.
\end{proof}


% Sparsity
In order to achieve a sparse linear program it is benefitial to use a monomial decomposition with small order and rank.
Beside this sparsity, the matrix $\matrixofat{\sliceset}{\datvariable,\selvariable}$ is often $\ell_0$-sparse, and has thus an efficient representation in a basis CP format.
More precisely we have by the above construction
\begin{align*}
    \sparsityof{\matrixofat{\sliceset}{\datvariable,\selvariable}} \leq \sum_{\decindexin} 3 \cdot \cardof{\variablesetof{\decindex}} +1 \, .
\end{align*}

    \section{Reasoning by Tensor Approximation}\label{cha:tensorApproximation}

Often reasoning requires the execution of demanding contractions of tensors networks, or combinatorical search of maximum coordinates.
We in this chapter investigate methods, to replace hard to be sampled tensor networks by approximating tensor networks, which then serve as a proxy in inference tasks.


\subsection{Approximation of Energy tensors}

\subsubsection{Direct Approximation}

Direct approximation is the problem
	\[ \argmin_{\canparam\in\Gamma^{\graph}} \|\energytensorat{\shortcatvariables} - \canparamat{\shortcatvariables}\|^2 \, . \]


\subsubsection{Approximation involving Selection Architectures}

Approximation involving a selection architecture $\fselectionmap$ is the problem
	\[ \argmin_{\canparam\in\Gamma^{\graph}} \|\energytensor - \sbcontractionof{\sencodingof{\fselectionmap},\canparam}{\shortcatvariables}\|^2 \, . \]

In a tensor network diagram we depict this as
\begin{center}
    \begin{tikzpicture}[scale=0.3] % , baseline = -3.5pt

	%\drawformulatensorof{\targettensor}


    \draw (-5,1) rectangle (5,-1);
    \node at (0,-1) [above] {$\targettensor$} ;
    \draw (-4,-1)--(-4,-3) node[midway,left] {\tiny $\catvariableof{0}$};
    \draw (4,-1)--(4,-3) node[midway,right] {\tiny $\catvariableof{\catorder\shortminus1}$};
    \node[anchor=center] at (0,-2) {$\cdots$};

		
\begin{scope}[shift={(-19,2)}]

    \draw (0,-5)--(0,-3) node[midway,left] {\tiny $\catvariableof{0}$};
    \draw (4,-5)--(4,-3) node[midway,right] {\tiny $\catvariableof{\catorder\shortminus1}$};
    \node[anchor=center]  at (2,-4) {$\cdots$};

%\drawatomindices{0}{-4}
\draw (-1,1) rectangle (5, -3);
\node[anchor=center] (text) at (2,-1) {$\sencodingof{\formulaset}$};

%\draw[->] (2,-1)--(2,1) node[midway,right] {\tiny ${\atomicformulaof{\parindexof{1}} \land \atomicformulaof{\parindexof{2}}}$}; 

\draw[] (5,0.5) -- (7,0.5) node[midway, above] {\tiny $\selvariableof{\selorder\shortminus1}$};
%\draw[<-] (5,-1)--(7,-1) node[midway,above] {\tiny $\selvariableof{\vselectionsymbol,1}$}; 
\node[anchor=center] (text) at (6,-1) {$\vdots$};
\draw[] (5,-2.5)--(7,-2.5) node[midway,below] {\tiny $\selvariableof{0}$}; 

\draw (7,1) rectangle (9, -3);
\node[anchor=center] (text) at (8,-1) {$\canparam$};


	
%	\draw (-4,5) rectangle (-2,3);
%    	\node at (-3,3.1) [above] {$\varcore{1}$} ;
%			
%	
%	\draw (-5,1) rectangle (-1,-1);
%    	\node at (-3,-1.1) [above] {$\gtensorof{\placeholderof{1}}$} ;
%	\draw (-3,1) -- (-3,3) node [midway,left] {$\atomlegindexof{1}$};
%
%	\node at (0.65,-1.1) [above] {$\cdots$};
%
%	\draw (2,1) rectangle (6,-1);
%    	\node at (4,-1.1) [above] {$\gtensorof{\placeholderof{\atomorder}}$} ;
%	\draw (4,1) -- (4,3) node [midway,left] {$\atomlegindexof{\atomorder}$};
%	\draw (3,5) rectangle (5,3);
%    	\node at (4,3.1) [above] {$\varcore{\atomorder}$} ;
%	\renewcommand{\skeletoncolor}{}
%	\drawskeleton
	
	
	
\end{scope}
	
\node at (-7.5,0) {{$-$}};
	
\draw (-22,3) -- (-22,-3);
\draw (-22.2,3) -- (-22.2,-3);

\draw (6,3) -- (6,-3);
\draw (6.2,3) node[right] {$2$} -- (6.2,-3) ;
	
\node at (-30,0) [right] {{$\argmin_{\canparam\in\Gamma^{\graph}}$}};
	
\end{tikzpicture}
\end{center}


\begin{example}[Approximate based on a slice sparsity selecting architecture]
	Use a term selecting neural network (conjunction neuron on $\atomorder$ unary neurons selecting a variable and $\mathrm{Id},\lnot,\mathrm{True}$ as connective selector.
	Demand the parameter tensor $\canparam$ to be in a basis CP format, then each slice of the parameter tensor corresponds with the slice of the energy.
	The use the approximation for MAP search.
	Same construction possible for probability tensors, but often more involved to instantiate them as tensor network.
\end{example}



\subsection{Transformation of Maximum Search to Risk Minimization}

By the squares risk trick, maximum coordinate searches involving contractions with boolean tensors can be turned into squares risk minimization problems.
This trick can be applied in MAP inference of MLN and the proposal distribution.

\subsubsection{Weighted Squares Loss Trick}

\begin{lemma}
	Let $\hypercore$ be a Boolean tensor, that is $\imageof{\hypercore}\subset\{0,1\}$.
	Then
		\[ \hypercoreat{\shortcatvariables} = \onesat{\shortcatvariables} - \left( \hypercoreat{\shortcatvariables} - \onesat{\shortcatvariables} \right)^2  \]
	where $\ones$ is a tensor with same shape as $\hypercore$ and all coordinates being $1$.
\end{lemma}
\begin{proof}
	Since for each $\shortcatindices\in\facstates$ we have $\hypercore[\shortcatvariables=\shortcatindices]\in\{0,1\}$, it holds that
		\[ \hypercoreat{\shortcatvariables=\shortcatindices} = 1 - (\hypercoreat{\shortcatvariables=\shortcatindices}-1)^2 \]
	and thus in coordinatewise calculus
		\[ \hypercoreat{\shortcatvariables} = \onesat{\shortcatvariables} - \left( \hypercoreat{\shortcatvariables} - \onesat{\shortcatvariables} \right)^2 \, .   \]
\end{proof}

We apply this property to reformulate optimization problems over boolean tensors into weighted least squares problems.

\begin{theorem}[Weighted Squares Loss Trick]\label{the:reweightedLeastSquares}
	Let $\Gamma$ be a set of boolean tensors in $\facspace$ and $\importancetensor\in\facspace$ arbitrary.
	Then we have
	\begin{align}
		\argmax_{\hypercore\in\Gamma} \contraction{\importancetensor,\hypercore} 
		= \argmin_{\hypercore\in\Gamma} \contraction{\importancetensor, (\hypercoreat{\shortcatvariables}-\onesat{\shortcatvariables})^2}
	\end{align} 
\end{theorem}
\begin{proof}
	Using the Lemma above, $\hypercore$ is identical to $\onesat{\shortcatvariables}-(\hypercoreat{\shortcatvariables}-\onesat{\shortcatvariables})^2$ and we get
	\begin{align*}
		 \contraction{\importancetensor,\hypercore} 
		 &=  \contraction{\importancetensor,\onesat{\shortcatvariables}}-\contraction{\importancetensor,(\hypercoreat{\shortcatvariables}-\onesat{\shortcatvariables})^2} 
	\end{align*}
	Since the first term does not depend on $\hypercore$, it can be dropped in the maximization problem.
	The $(-1)$ factor then turns the maximization into a minimization problem.
\end{proof}

% Interpretation and Importance Tensor
\theref{the:reweightedLeastSquares} reformulates maximation of binary tensors with respect to an angle to another tensor into minimization of a squares risk.
This squares risk trick is especially useful when combining it with a relaxation of $\Gamma$ to differentiably parametrizable sets, since then common squares risk solvers can be applied.
We will call $\importancetensor$ in the \theref{the:reweightedLeastSquares} importance tensor, since it manipulates the relevance of each coordinate in the squares loss.

%
As a result, we interpret the objective
	\[ \contraction{\importancetensor, (\hypercoreat{\shortcatvariables}-\onesat{\shortcatvariables})^2} \]
as a weighted squares loss.

\begin{example}[Proposal distribution maxima]
	The Problem~\ref{prob:steepestAscent} of finding the maximal coordinate can thus be turned into
	\begin{align*}
		\argmax_{\shortselindices} \contractionof{(\empdistribution-\currentdistribution),\fselectionmap}{\shortselvariables=\shortselindices}  
		= \argmin_{\shortselindices} \sbcontraction{(\empdistribution-\currentdistribution),
		\left(\contractionof{\fselectionmap,\onehotmapofat{\shortselindices}{\shortselvariables}}{\shortcatvariables}-\onesat{\shortcatvariables}\right)^2} \, . 
	\end{align*}
\end{example}


\subsubsection{Problem of the trivial tensor}

By the above we motivated least squares problems on the set of one-hot encoded states.
One is tempted to extend this set to $\mnexpfamily$ for efficient solutions by alternating algorithms.

However, for any hypergraph $\graph$ we have $\onesat{\shortcatvariables}\in\mnexpfamily$.
In many situations (e.g. disjoint model sets supported at positive data) the objective is more in favor at the trivial tensor than at the one-hot encoding.
As a result, we do not solve the previously posed one-hot encoding problem, when allowing such an hypothesis embedding.


\begin{example}[Fitting a tensor by a formula tensor]\label{exa:formulaFitting}
	Task: Given a tensor $\hypercore$, find a formula $\exformula\in\formulaset$ such that it coincides with $\hypercore$.

	If $\hypercore$ is a binary tensor, we understand it as a formula and want to find an $\exformula$ such that its number of worlds is maximal, that is solve the problem
		\[ \argmax_{\exformula\in\formulaset}\sbcontraction{\exformula\Leftrightarrow\hypercore}  \, . \]

	We can use the squares risk trick and get an equivalent problem
		\[ \argmin_{\exformula\in\formulaset} \| \sbcontractionof{\exformula\Leftrightarrow\hypercore}{\shortcatvariables}  - \onesat{\shortcatvariables} \|^2 \, . \]
\end{example}

%\begin{remark}{Least Squares Loss by Tensor Fitting}
%	\red{Alternative approach to least squares problems: Tensor Fitting}
%	And, if the target is another formula $y$, such that $\exformula$ conincides with $\tilde{f} \iff y $ we have
%		\[ \left(\polynomialof{\exformula}(\datamap)-1\right)^2 = \left(  \polynomialof{\tilde{f}}(\datamap) - y(\datamap) \right)^2  \]
%	This is exactly the least squares loss, which would appear in a supervised interpretation of the learning.
%\end{remark}




\subsection{Alternating Solution of Least Squares Problems}

When the parameter tensor $\canparam$ is only restricted to have a decomposition as a tensor network on $\graph$, we can iteratively update each core.
The resulting algorithm is called Alternating Least Squares (ALS) (see Algorithm \ref{alg:ALS}).

\begin{algorithm}[hbt!]
\caption{Alternating Least Squares (ALS)}\label{alg:ALS}
\begin{algorithmic}
\For{$\edgein$}
	\State Set $\hypercoreofat{\edge}{\catvariableof{\edge}}$ to a random element in $\rr^{\atomlegdimof{\atomenumerator}}$ 
\EndFor
%\For{$\atomenumeratorin$}
%	\State Set $\varcore{\atomenumerator}$ to a random element in $\rr^{\atomlegdimof{\atomenumerator}}$ 
%\EndFor
\While{Stopping criterion is not met}
\For{$\edgein$}
	\State Set $\hypercoreofat{\edge}{\catvariableof{\edge}}$ to a to a solution of the local problem, that is
	\[ 
	\hypercoreofat{\edge}{\catvariableof{\edge}}
	 \algdefsymbol 
	 \argmin_{\hypercoreofat{\edge}{\catvariableof{\edge}}} 
	 \contraction{\importancetensor, (\contractionof{\fselectionmap,\canparam}{\shortcatvariables} - \targettensor[\shortcatvariables])^2}
	 \]
\EndFor
\EndWhile
\end{algorithmic}
\end{algorithm}


\subsubsection{Choice of Representation Format}

\red{The choice of the hypergraph $\graph$ used for approximation bears a tradeoff between expressivity and complexity in sampling.
Hidden variables, that is variables only present in $\graph$, but not in the sensing matrix, increase the expressivity, especially when assigning large dimensions to them.
When there are no hidden variables, the maximum of $\canparam$ can be found by maximum calibration through a message passing algorithm, since no hidden variable has to marginalized.}


In case of skeleton expressions with many placeholders further decomposition for algorithmic efficiency are required.
\begin{itemize}
	\item Elementary Format ($\elformat$-Format): 
	\item $\cpformat$-Format: Closest to sum of formula tensors (when all vectors are basis, then have a sum).
	\item $\ttformat$-Format: Showed better heuristic performance in optimization
\end{itemize}

For any tensor network decomposition into cores $\canparamof{\parenumerator}$ have the derivative $\frac{\partial}{\partial \canparamof{\parenumerator}} \canparam$ as the tensor network with out the core $\canparamof{\parenumerator}$.

%\begin{remark}[Parametercores being basis tensors]
%	When the parameter core is a basis tensor, the contraction with the parametercore coincides with the respective formula tensor.
%	Thus, we will search for basis tensors optimizing in contractions objectives to specific reasoning tasks, and add them iteratively to the network at hand.
%\end{remark}


%\subsection{Projection onto Basis Tensors}
%\red{This is sampling!}
%We project onto basis tensors to achieve single formulas.


\subsection{Regularization and Compressed Sensing}


When regularizing the least squares problem by enforcing the sparsity of $\canparam$, we arrive at the compressed sensing problem
\begin{align}
	\argmin_{\canparamat{\selvariable}} \sparsityof{\canparam} 
	\quad \text{subject to } \quad 
	\left\| \contractionof{\sencsstat,\canparam}{\shortcatvariables} - \energytensorat{\shortcatvariables} \right\|_2 \leq \eta
\end{align}
Here, the sensing matrix is the selection tensor.


\begin{example}[Formula fitting to an example]
	Choosing the best formula fitting data (see Example~\ref{exa:formulaFitting}) is the problem
	\begin{align}
	\argmin_{\canparamat{\selvariable}\, : \,  \sparsityof{\canparam}=1} \left\| \contractionof{\importancetensor,\sencsstat,\canparam}{\shortcatvariables} - \targettensor \right\|_2 
	\end{align}
	where $\importancetensor$ has nonzero entries at marked coordinates and $\targettensor$ stores in Boolean coordinates whether the marked coordinates are positive or negative examples.
	\red{When the number of positive and negative examples are identical, we can linearly transform the objective to that of a grafting instance, where the current model is the empirical distribution of negative examples and the data consists of the positive examples.}
\end{example}

% Usage as sparse tensor
The sparse tensor solving the problem then has a small number of nonzero coordinates and the selection tensor can be restricted to those.
As a consequence, inference can be performed more efficiently.

% Algorithmic solution
The algorithmic solution of these problems can be done by greedy algorithms, thresholding based algorithms or optimization based algorithms \cite{foucart_mathematical_2013}.

% Guarantees
Guarantees for the success of the algorithms depend on the properties of the sensing matrices.
Here the sensing matrices are deterministic, since constructed as selection tensors, and concentration based approaches towards probabilistic bounds on these properties (see \cite{goesmann_uniform_2021}) are not applicable.





\begin{example}[Propositional Formulas]
	Let there be a set $\formulaset$ of formulas, then we have
	\begin{align*}
		\sbcontractionof{\sencodingofat{\formulaset}{\shortcatvariables,\selvariableof{\insymbol}},\sencodingofat{\formulaset}{\shortcatvariables,\selvariableof{\outsymbol}}}{\indexedselvariableof{\insymbol},\indexedselvariableof{\outsymbol}}
		= \sbcontraction{\formulaof{\selindexof{\insymbol}}, \formulaof{\selindexof{\outsymbol}}} \, . 
	\end{align*}
	If the formulas have disjoint model sets then 
	\begin{align*}
		\sbcontractionof{\sencodingofat{\formulaset}{\shortcatvariables,\selvariableof{\insymbol}},\sencodingofat{\formulaset}{\shortcatvariables,\selvariableof{\insymbol}}}{\indexedselvariableof{\insymbol},\indexedselvariableof{\outsymbol}}
		= \begin{cases}
			\sbcontraction{\formulaof{\insymbol}} & \text{if } \selindexof{\insymbol}=\selindexof{\outsymbol} \\
			0 & \text{else} 
		\end{cases} \, . 
	\end{align*}
\end{example}


\begin{example}[Slice selection networks]
	
	For the slice selection network
	\begin{align*}
		\sbcontractionof{\sliceselectionmapat{\shortcatvariables,\selvariableof{\insymbol}},\sliceselectionmapat{\shortcatvariables,\selvariableof{\outsymbol}}}{\indexedselvariableof{\insymbol},\indexedselvariableof{\outsymbol}}
		= \begin{cases}
			0 & \text{if for a }\seccatenumerator\in\variablesetof{\selindexof{\insymbol}}\cap\variablesetof{\selindexof{\outsymbol}}\text{ we have }\catindexof{\seccatenumerator}^{\selindexof{\insymbol}}\neq\catindexof{\seccatenumerator}^{\selindexof{\outsymbol}} \\
			\prod_{\seccatenumerator\notin\variablesetof{\selindexof{\insymbol}}\cup\variablesetof{\selindexof{\outsymbol}}} \catdimof{\seccatenumerator}& \text{else} 
		\end{cases} \, . 
	\end{align*}

	Given a fixed $\selindexof{\insymbol}$, the maximum value in the respective slice is thus taken at $\selindexof{\insymbol}=\selindexof{\outsymbol}$
\end{example}









    \chapter{\chatextmessagePassing}\label{cha:messagePassing}

In this chapter we introduce local contraction passed along tensor clusters to calculate global contractions exactly or approximatively.
These message passing schemes provide tradeoffs between efficiency increases and exactness of the global contraction.

We use the $\cpformat$ decompositions to investigate the asymptotic behavior of the message passing algorithms.

\red{The application of message passing schemata to calculate contractions are motivated by commutations of contractions.
We first show this property and then provide message passing schemata.
}

\sect{Commutation of Contractions}

We show in the next theorem, that a contractions can be performed by contracting a subnetwork first and then further contracting the result with the rest.

\begin{theorem}\label{the:splittingContractions}
	Let $\tnetof{\graph}$ be a tensor network on a hypergraph $\graph=(\nodes,\edges)$.
	Let us now split the $\graph$ into two graphs $\graph_1=(\nodesone,\edges_1)$ and $\graph_2=(\nodestwo,\edges_2)$, such that $\edges_1\dot{\cup}\edges_2=\edges$, $\nodesone\cup\nodestwo=\nodes$ and all nodes in $\nodestwo$ are contained in an hyperedge of $\edges_2$.
	We then have for any $\secnodes\subset\nodes$
	\begin{align*}
		\contractionof{\tnetof{\graph}}{\catvariableof{\secnodes}}
		= \contractionof{
			\tnetofat{\graph_1}{\catvariableof{\nodesone}}
			\cup \{\contractionof{\tnetof{\graph_2}}{\catvariableof{\nodestwo\cap(\nodesone\cup\secnodes)}}\}
		}{\catvariableof{\secnodes}}   \, .
\end{align*}
\end{theorem}
\begin{proof}
	For any index $\catindexof{\secnodes}$ we show that
	\begin{align*}
		\contractionof{\tnetof{\graph}}{\indexedcatvariableof{\secnodes}}
		=\contractionof{\tnetof{\graph_1} \cup \{
			\contractionof{\tnetof{\graph_2}}{\catvariableof{\nodestwo\cap(\nodesone\cup\secnodes)}}
		\}}{\indexedcatvariableof{\secnodes}}   \, .
	\end{align*}
	By definition we have
	\begin{align*}
		\contractionof{\tnetof{\graph}}{\indexedcatvariableof{\secnodes}}
		& = \sum_{\catindexof{\nodes/\secnodes}} \prod_{\edge\in\edges} \hypercoreofat{\edge}{\indexedcatvariableof{\edge}} \\
		& = \sum_{\catindexof{\nodes/\secnodes}}
		 	\left( \prod_{\edge\in\edges_1} \hypercoreofat{\edge}{\indexedcatvariableof{\edge}} \right)
		 	\cdot \left( \prod_{\edge\in\edges_2} \hypercoreofat{\edge}{\indexedcatvariableof{\edge}}  \right) \\
		& =  \sum_{\catindexof{\nodesone/\secnodes}} \sum_{\catindexof{\nodestwo/(\secnodes\cup\nodesone)}}
			\left( \prod_{\edge\in\edges_1} \hypercoreofat{\edge}{\indexedcatvariableof{\edge}} \right)
		 	\cdot \left( \prod_{\edge\in\edges_2} \hypercoreofat{\edge}{\indexedcatvariableof{\edge}}  \right) \\
		& =  \sum_{\catindexof{\nodesone/\secnodes}}
			\left( \prod_{\edge\in\edges_1} \hypercoreofat{\edge}{\indexedcatvariableof{\edge}} \right)
		 	\cdot \left( \sum_{\catindexof{\nodestwo/(\secnodes\cup\nodesone)}}  \prod_{\edge\in\edges_2} \hypercoreofat{\edge}{\indexedcatvariableof{\edge}}  \right) \, .
	\end{align*}
	When contracting the variables $\catvariableof{\nodestwo/(\secnodes\cup\nodesone)}$ on $\tnetof{\graph_2}$, the variables $\catvariableof{\nodestwo\cap(\secnodes\cup\nodesone)}$ are left open.
	We therefore have for any $\catindexof{\nodestwo\cap(\secnodes\cup\nodesone)}$
	\begin{align*}
		\sbcontractionof{\tnetof{\graph_2}}{\indexedcatvariableof{\nodestwo\cap(\secnodes\cup\nodesone)}} =
		 \left( \sum_{\catindexof{\nodestwo/(\secnodes\cup\nodesone)}}  \prod_{\edge\in\edges_2} \hypercoreofat{\edge}{\indexedcatvariableof{\edge}}  \right) \, .
	\end{align*}
	It follows with the above, that
	\begin{align*}
		\contractionof{\tnetof{\graph}}{\indexedcatvariableof{\secnodes}}
		& =  \sum_{\catindexof{\nodesone/\secnodes}}  \left( \prod_{\edge\in\edges_1} \hypercoreofat{\edge}{\indexedcatvariableof{\edge}} \right) \cdot \sbcontractionof{\tnetof{\graph_2}}{\indexedcatvariableof{\nodestwo\cap(\secnodes\cup\nodesone)}} \\
		& = \contractionof{\tnetof{\graph_1} \cup \{
			\contractionof{\tnetof{\graph_2}}{\catvariableof{\nodestwo\cap(\nodesone\cup\secnodes)}}
		\}}{\indexedcatvariableof{\secnodes}}   \, .
	\end{align*}
\end{proof}


% Message interpretation
We can interpret the inner contraction $\contractionof{\tnetof{\graph_2}}{\catvariableof{\nodestwo\cap(\nodesone\cup\secnodes)}}$ as a message, sent from $\graph_2$ to $\graph_1$.
Based on this intuition, we will define message passing schemes in the next section.


\sect{Exact Contractions}

%\red{This is the junction tree algorithm!}

We apply Theorem~\ref{the:splittingContractions} to split a contraction into subcontractions, which are consecutively performed.

% Message Passing
Contractions can be performed partially, and the result passed to the rest of the network as a message.

\subsect{Construction of Cluster Graphs}

Let us first introduce with the cluster graph a mechanism to coarse grain the hypergraph capturing a tensor network.

% Cluster Graphs
\begin{definition}[Cluster Graph]
	Given a tensor network $\extnet$ a cluster partition is a partition of the tensor network into $n$ clusters, by a function
		\[ \alpha : \edges \rightarrow [n] \, . \]
	The clusters are with tensors decorated edge sets $\enc = \{\edge \, : \, \alpha(\edge) = i\}$ with variables $\nodes_i = \bigcup_{\edge \in \enc} \edge$.

	We say, that the cluster graph satisfies the running intersection property, when for any clusters $C_i$ and $C_j$ and any $\node\in\nodes_i\cup\nodes_j$ there is a path between $C_i$ and $C_j$ with $\node\in\nodes_k$ for any cluster $C_k$ along the path.
	%The clusters form a graph where edges between $\enc$ and $C_j$ exist, when the node sets $\nodes_i$ and $\nodes_j$ are not disjoint.
	%In this case, we define separation sets $S_{i,j}=\enc\cup C_j$
\end{definition}

Given a cluster graph to a tensor network, we can execute any global contraction by a contraction of local contraction to each cluster.

\begin{theorem}\label{the:contractionClusterSplit}
	Given a tensor network $\extnet$ and a cluster graph.
	We then define for each cluster the node set
	\begin{align*}
		\tilde{\nodes}_i = \bigcup_{j\neq i} \nodes_j
	\end{align*}
	and have
		\[ \contractionof{\extnet}{\catvariableof{\secnodes}} = 
		\contractionof{
			\{ \contractionof{\tnetof{\enc}}{\catvariableof{\nodes_i \cap (\tilde{\nodes}_i\cup\secnodes)}}  : i \in [n]\}
		}{\catvariableof{\secnodes}}  \, . \]
\end{theorem}
\begin{proof}
	By Theorem~\ref{the:splittingContractions} applied for each cluster seen as a subgraph.
\end{proof}



\subsect{Message Passing to calculate contractions}

% Cluster Graphs
Having a hypergraph $\graph$, we iteratively apply Theorem~\ref{the:splittingContractions} and call the $\graph_2$ a cluster.
When iterating until $\graph$ is empty, we get a cluster graph, where all tensors are assigned to a cluster.

% Cluster Trees -> Clique Trees in Koller Book
When the cluster are a polytree, that is a union of disjoint trees, we define messages between neighbored clusters $\enc$ and $\secenc$ with $\secenc\prec\enc$ by the contractions
\begin{align*}
	\mesfromtoat{\secclusterenumerator}{\clusterenumerator}{\catvariableof{\nodesof{\clusterenumerator}\cap\nodesof{\secclusterenumerator}}}
	= \contractionof{\{
	\mesfromtoat{\thirdclusterenumerator}{\secclusterenumerator}{\catvariableof{\nodesof{\thirdclusterenumerator}\cap\nodesof{\secclusterenumerator}}}
	\, : \, \thirdenc\prec\secenc\} \cup \tnetof{\secenc}}{\catvariableof{\nodesof{\clusterenumerator}\cap\nodesof{\secclusterenumerator}}} \, .
\end{align*}



We note, that the messages are well defined by these recursive equations, exactly when the cluster graph is a polytree.
%Since messages are recursively defined, we need the tree structure to ensure well-definedness.

% Tree advantage
When the cluster graph is a tree, we can choose a root cluster and order the clusters by the topological order $\prec$.


\begin{lemma}\label{lem:clusterContractionMessage}
	When the cluster graph is a tree satisfying the running intersection property, we have for neighbored clusters $\enc$ and $\secenc$ with $\secenc\prec\enc$
	\begin{align*}
		\mesfromtoat{\clusterenumerator}{\tilde{\clusterenumerator}}{\catvariableof{\nodesof{\clusterenumerator}\cap\nodesof{\tilde{\clusterenumerator}}}}
		= \contractionof{\{\tnetof{\secenc}\,:\,\secenc\prec\enc\}}{\catvariableof{\nodesof{\clusterenumerator}\cap\nodesof{\tilde{\clusterenumerator}}}}   \, .
	\end{align*}
\end{lemma}
\begin{proof}
	By induction over the cardinality $n$ of the preceding clusters.
	\paragraph{$n=1$}: For a single preceding cluster the statement holds trivial, since the preceding cluster is the cluster itself.

	\paragraph{$n+1\rightarrow n$}: Let us now assume, that the statement holds for up to $n$ preceding clusters, and let there be $n+1$ preceding clusters.
	We build another cluster graph for the cores different from $\enc$, by assigning each cluster $\thirdenc$ to the neighbor $\secenc$ where $\secclusterenumerator\in\neighborsof{\clusterenumerator}$, for which
	\begin{align*}
		\thirdenc\prec\secenc \, .
	\end{align*}
	We use \theref{the:contractionClusterSplit} on this constructed cluster graph and get
	\begin{align*}
		\contractionof{\{\tnetofat{\clusterof{\thirdclusterenumerator}}{\nodevariablesof{\thirdclusterenumerator}} \, : \, \thirdclusterenumerator\neq\clusterenumerator\}}{\nodevariablesof{\clusterenumerator}}
		= \contractionof{
			\Big\{ \contractionof{
				\{\tnetofat{\clusterof{\thirdclusterenumerator}}{\nodevariablesof{\thirdclusterenumerator}} \, : \, \thirdclusterenumerator\prec\secclusterenumerator\}
			}{\catvariableof{\secnodesof{\secclusterenumerator}}}
			\, : \, \secclusterenumerator\in\neighborsof{\clusterenumerator} \Big\}
			}{\nodevariablesof{\clusterenumerator}}
	\end{align*}
	Here by $\secnodesof{\secclusterenumerator}$ we denote the intersection of
	\begin{align*}
		\secnodesof{\secclusterenumerator} = \left(\bigcup_{\thirdclusterenumerator\prec\secclusterenumerator} \nodesof{\thirdclusterenumerator}\right) \cap \left(\bigcup_{\thirdclusterenumerator\nprec\secclusterenumerator} \nodesof{\thirdclusterenumerator}\right)
	\end{align*}
	By the running intersection property, we have $\nodesof{\secclusterenumerator}\cap\nodesof{\clusterenumerator}=\secnodesof{\secclusterenumerator}$.

	We further have for any $\secclusterenumerator\in\neighborsof{\clusterenumerator}$ that
	\begin{align*}
		\cardof{\{\thirdclusterenumerator\prec\secclusterenumerator\}} \leq n \, .
	\end{align*}
	We can therefore apply the assumption of the induction and get
	\begin{align*}
 		\contractionof{\{\tnetofat{\clusterof{\thirdclusterenumerator}}{\nodevariablesof{\thirdclusterenumerator}} \, : \, \thirdclusterenumerator\prec\secclusterenumerator\}
			}{\catvariableof{\secnodesof{\secclusterenumerator}}}
		= \mesfromtoat{\secclusterenumerator}{\clusterenumerator}{\catvariableof{\nodesof{\secclusterenumerator}\cap\nodesof{\clusterenumerator}}}
	\end{align*}

	With the above, we arrive at 
	\begin{align*}
		\mesfromtoat{\clusterenumerator}{\tilde{\clusterenumerator}}{\catvariableof{\nodesof{\clusterenumerator}\cap\nodesof{\tilde{\clusterenumerator}}}}
		= \contractionof{\{\tnetof{\secenc}\,:\,\secenc\prec\enc\}}{\catvariableof{\nodesof{\clusterenumerator}\cap\nodesof{\tilde{\clusterenumerator}}}}   \, .
	\end{align*}
%	\begin{align*}
%		\contractionof{\extnet}{\catvariableof{\nodesof{\clusterenumerator}}} =
%		\contractionof{\bigcup_{\secclusterenumerator\in\neighborsof{\clusterenumerator}}
%		\{\contractionof{\tnetof{\clusterof{\thirdclusterenumerator}}}{\catvariableof{\nodesof{\thirdclusterenumerator}\cap\nodesof{\secclusterenumerator}}} \, : \, \thirdenc\prec\secenc \}
%		\cup \{\tnetof{\enc}\}}{\catvariableof{\nodes_i}} \, .
%	\end{align*}
%	Now, we use the assumption of the induction to get
%	\begin{align*}
%		\contractionof{
%			\{\contractionof{\tnetof{\clusterof{\thirdclusterenumerator}}}{\catvariableof{\nodesof{\thirdclusterenumerator} \cap (\tilde{\nodes}_i\cup\secnodes)}}   \, : \, \cluster_l \prec \cluster_j \}
%		}{\catvariableof{\nodesof{\thirdclusterenumerator}\cap\nodesof{\secclusterenumerator}}}
%		= \mesfromtoat{\thirdclusterenumerator}{\clusterenumerator}{\catvariableof{\nodesof{\thirdclusterenumerator}\cap\nodesof{\secclusterenumerator}}}
%	\end{align*}
%	Note that we used the running intersection property ensuring that whenever a variable of the cluster appears in a previous cluster, the variable is passed in the message.
\end{proof}


\begin{theorem}
	When the cluster graph is a tree satisfying the running intersection property, then we have for each cluster $\enc$ with neighbors $\neighborsof{\clusterenumerator}$
%	Then for each clique we have the conditional probability of its variables being the contraction of the messages with the cliques cores, that is
	\begin{align}
		\contractionof{\extnet}{\catvariableof{\nodes_i}} =
		\contractionof{\{ \mesfromtoat{\secclusterenumerator}{\clusterenumerator}{\catvariableof{\nodesof{\clusterenumerator}\cap\nodesof{\secclusterenumerator}}}  \, : \, j \in N(i) \} \cup \{\tnetof{\enc}\}}{\catvariableof{\nodes_i}} \, .
	\end{align}
\end{theorem}
\begin{proof}
	We use the topological order $\prec$ of the clusters by the tree, when choosing a root by cluster $\enc$.

	The claim then follows from \theref{the:contractionClusterSplit} and \lemref{lem:clusterContractionMessage}.
%
%	Thus, we find to each cluster $\cluster_l$ exactly one neighboring cluster $j\in N(i)$ with $\cluster_l \prec \cluster_j$.
%	We use \theref{the:contractionClusterSplit} and reorder the contractions as
%	\begin{align*}
%		\contractionof{\extnet}{\catvariableof{\nodes_i}} =
%		\contractionof{\bigcup_{j\in N(i)}
%		\{\contractionof{\tnetof{\cluster_l}}{\catvariableof{\nodes_l \cap (\tilde{\nodes}_i\cup\secnodes)}}   \, : \, \cluster_l \prec \cluster_j \}
%		\cup \{\tnetof{\enc}\}}{\catvariableof{\nodes_i}} \, .
%	\end{align*}
%	By \lemref{lem:clusterContractionMessage}, the contractions to the clusters preceeding $\cluster_l$ are exactly the messages sent to the root cluster.
%	Therefore we have
%	\begin{align}
%		\contractionof{\extnet}{\catvariableof{\nodes_i}} =
%		\contractionof{\{ \upmes{\secclusterenumerator}{\clusterenumerator}  \, : \, j \in N(i) \} \cup \{\tnetof{\enc}\}}{\catvariableof{\nodes_i}} \, .
%	\end{align}

%	We then show by	induction, that any message
%	\begin{align*}
%		\upmes{\secclusterenumerator}{\clusterenumerator}
%		= \contractionof{
%			\{ \tnetof{\cluster} \, : \, \cluster \prec \clusterof{\secclusterenumerator} \}
%		}{\catvariableof{\nodes_\secclusterenumerator \cup \nodes_{\clusterenumerator}}}
%	\end{align*}
%	To this end, use \theref{the:contractionClusterSplit}.
%	Having established the induction, we use this result for the messages and get the claim.
%	By Theorem~\ref{the:splittingContractions} we split into contractions of the clusters up and down of the respective neighbors and apply the above lemma.
\end{proof}


% Downward messages
While we have defined message passing along the topological order of a graph, we can also define messages against the topological order, that is
\begin{align*}
	\downmes{i}{j}  = \contractionof{\{\downmes{\tilde{j}}{i} \, : \,  \enc \prec  \thirdenc\} \cup \tnetof{\enc}}{\catvariableof{\nodes_i\cap \nodes_j}} \,
\end{align*}
To this end, we can get a similar statement for nodes, which are not the rotes of the cluster tree.
The constractions at each cluster can then be computed batchwise, based on message passed along a topological order and against.

%
These message passing schemes can be derived from Lagrangian parameters given a local consistency polytope \cite{wainwright_graphical_2008}.



\subsect{Variable Elimination Cluster Graphs}


\begin{remark}[Construction of Cluster Graphs by Variable Elimination]
	% Build a cluster graph
	Following an elimination order of the colors, mark those tensors containing the colors, which have not been marked before, as the cluster.
	% Extension to clique tree
	A clique tree can be constructed by these cluster, when iterating through the clusters and either connect them to previous disconnected clusters or leave the current cluster disconnected.
	Add the disconnected clusters with the current cluster in case there are overlaps of their open colors.
	If the disconnected cluster added has more open colors, 
\end{remark}


\subsect{Bethe Cluster Graphs}


\begin{figure}[h]
\begin{center}
	\begin{tikzpicture}[scale=0.35, thick] % , baseline = -3.5pt




\begin{scope}[shift={(23,0)}]

\node[anchor=center] (text) at (-6,8) {$b)$};

\draw (2,8) rectangle (4,6);
\node[anchor=center] (text) at (3,7) {\small $\rencodingof{\lor}$};

\draw (2,5) rectangle (4,3);
\node[anchor=center] (text) at (3,4) {\small $\rencodingof{\land}$};

\draw (2,2) rectangle (4,0);
\node[anchor=center] (text) at (3,1) {\small $\datacoreof{c}$};

\draw (2,-1) rectangle (4,-3);
\node[anchor=center] (text) at (3,-2) {\small $\datacoreof{b}$};

\draw (2,-4) rectangle (4,-6);
\node[anchor=center] (text) at (3,-5) {\small $\datacoreof{a}$};

\draw (2,-7) rectangle (4,-9);
\node[anchor=center] (text) at (3,-8) {\small $\lambda$};


\draw[fill] (-3,-8.5) circle (0.25cm);
\node[anchor=center] (text) at (-4,-8.5) {\tiny $\indexset$};

\draw[] (-3,-8.5) to[bend left=-10]  (2,-8);
\draw[] (-3,-8.5) to[bend left=0]  (2,-5.5);
\draw[] (-3,-8.5) to[bend left=10]  (2,-2.5);
\draw[] (-3,-8.5) to[bend left=20]  (2,0.5);


\draw[fill] (-3,-5.5) circle (0.25cm);
\node[anchor=center] (text) at (-4,-5.5) {\tiny $a$};

\draw[] (-3,-5.5) to[bend left=-10]  (2,-4.5);
\draw[] (-3,-5.5) to[bend left=10]  (2,3.5);

\draw[fill] (-3,-2.5) circle (0.25cm);
\node[anchor=center] (text) at (-4,-2.5) {\tiny $b$};

\draw[] (-3,-2.5) to[bend left=-10]  (2,-1.5);
\draw[] (-3,-2.5) to[bend left=10]  (2,4);

\draw[fill] (-3,0.5) circle (0.25cm);
\node[anchor=center] (text) at (-4,0.5) {\tiny $c$};

\draw[] (-3,0.5) to[bend left=-10]  (2,1.5);
\draw[] (-3,0.5) to[bend left=10]  (2,6.5);

\draw[fill] (-3,3.5) circle (0.25cm);
\node[anchor=center] (text) at (-4.5,3.5) {\tiny $a\land b$};

\draw[] (-3,3.5) to[bend left=-10]  (2,4.5);
\draw[] (-3,3.5) to[bend left=10]  (2,7);

\draw[fill] (-3,6.5) circle (0.25cm);
\node[anchor=center] (text) at (-5.5,6.5) {\tiny $(a\land b)\lor c$};

\draw[] (-3,6.5) to[bend left=10]  (2,7.5);


\draw[dashed] (-0.5,-12) -- (-0.5,8);

\node[right] (text) at (0.5,-11) {$\tilde{\edges}$};
\node[left] (text) at (-1.5,-11) {$\Delta$};

\end{scope}


\node[anchor=center] (text) at (-2,8) {$a)$};

\newcommand{\conposseldec}{3,-5.5}

\draw[fill] (\conposseldec) circle (0.25cm);
\draw (\conposseldec) -- (3,-7.5) node[midway, right] {\tiny ${\indexset}$}; % Unclear, whether this is the best notation!
\draw[] (2,-7.5) rectangle (4, -9.5);
\node[anchor=center] (text) at (3,-8.5) {\small $\lambda$};

\draw[] (0,1) -- (0,-1) node[midway,left] {\tiny $a$};
\draw (-1,-1) rectangle (1, -3);
\node[anchor=center] (text) at (0,-2) {\small $\datacoreof{a}$};
\draw[] (0,-3) to[bend right=20] (\conposseldec);


\draw[] (3,1) -- (3,-1) node[midway,left] {\tiny $b$};
\draw (2,-1) rectangle (4, -3);
\node[anchor=center] (text) at (3,-2) {\small $\datacoreof{b}$};
\draw[] (3,-3) to[bend right=0]  (\conposseldec);


\draw[] (6,5) -- (6,-1) node[midway,left] {\tiny $c$};
\draw (5,-1) rectangle (7, -3);
\node[anchor=center] (text) at (6,-2) {\small $\datacoreof{c}$};
\draw[] (6,-3) to[bend left=20]  (\conposseldec);


\draw[] (1.5,5) -- (1.5,3) node[midway,left] {\tiny $a \land b $};
\draw (-1,3) rectangle (4, 1);
\node[anchor=center] (text) at (1.5,2) {\small $\rencodingof{\land}$};


\draw[] (3.5,9) -- (3.5,7) node[midway,left] {\tiny $(a \land b) \lor c $};
\draw (0,7) rectangle (7, 5);
\node[anchor=center] (text) at (3.5,6) {\small $\rencodingof{\lor}$};

%\draw[] (6,1) to[bend left=20]  (\conposseldec);


		


\end{tikzpicture}
\end{center}
\caption{Example of a Bethe Cluster Graph.
	a) Example of a Tensor Network $\tnetof{\graph}$, which represents the by $\lambda$ averaged evaluation of the formula $(a\land b)\lor c$ on data $\datamap$.
	b) Corresponding Bethe Cluster Hypergraph, which dual is bipartite by the sets $\Delta$ and $\tilde{\edges}$.
	}
\label{fig:betheDataExample} 
\end{figure}

By adding delta tensors to each node $\node\in\nodes$ and defining its leg variables by $\node^{\edge}$ for $\edge\in\edges$.
We mark each such delta tensor by a cluster in $\Delta^{\graph}$, as defined in the following (see also Figure~\ref{fig:betheDataExample}).

\begin{definition}
	Given a tensor network $\tnetof{\graph}$ on a decorated hypergraph $\graph$, we define the Bethe Cluster Hypergraph $\secgraph$ as
	$(\secnodes, \secedges \cup \Delta^{\graph})$ where we have
	\begin{itemize}
		\item Recolored Edges $\secedges = \{\tilde{\edge} \, : \, \edge\in \edges\}$ where $\tilde{\edge} = \{\node^{\edge} \, : \, \node\in\edge\}$, which decoration tensor has same coordinates as $\hypercoreof{\edge}$
		\item Nodes $\secnodes = \bigcup_{\edge\in\edges}\tilde{\edge}$ %$\secnodes = \bigcup_{\edge\in\edges}\{\node^{\edge} \, : \, \node\in\edge \}$ 
		\item Delta Edges $\Delta^{\graph} =  \big\{ \{\node^{\edge} \, : \, \edge\ni\node \} \, : \, \node\in\nodes \big\} $, each of which decorated by a delta tensor $\delta^{\{\node^{\edge} \, : \, \edge\ni\node \}}$
	\end{itemize}
\end{definition}

By \lemref{lem:deltification} this construction does not change contractions.

% Dual graph
The dual is bipartite, since any variable appears exactly in one cluster in $\secedges$ and in one cluster of $\Delta^{\graph}$.
This further makes the dual of the Bethe Cluster Hypergraph a proper graph (i.e. edges consistent of node pairs). 





\subsect{Computational Complexity}

\red{Tree-width here: By building a cluster tree to any partition, simply by including variables for the running intersection property.
The maximum number of variables at a cluster is the tree-width and provides a complexity bound for the local contractions.}

Naive execution of $\contractionof{\tnetof{\graph}}{\secnodes}$: $\prod_{\node\in\nodes} \catdimof{\node}$ many products are built and summed up.
When splitting contractions into local subcontractions, the product can be turned into sums with tremendous decrease in complexity.





\sect{Boolean Message Passing}\label{sec:supportContractionEquations}

Instead of the exact calculation of a contraction, let us now investigate schemes to sparsify the tensors before a contraction.
To this end, we first show underlying properties of contractions enabling these schemes.


\subsect{Monotonocity of tensor contraction}

To state the next theorem we use the nonzero function $\nonzerofunction: \rr \rightarrow [2]$ by $\nonzeroof{x}=1$ if $x\neq0$ and $\nonzeroof{x}=0$ else.
Applied coordinatewise on tensors it marks the nonzero coordinates by $1$.

We show that adding boolean tensor cores to an contraction orders the results by the partial ordering introduced in \defref{def:partialOrder}.

\begin{theorem}[Monotonicity of Tensor Contractions]\label{the:monotonicityBinaryContractions}
	Let $\extnet, \secextnet$ be tensor network of non-negative tensors and $\catvariableof{\secnodes}$ an arbitrary set of random variables. %, and $\tilde{\theta}$ another binary tensor.
	Then we have
	\begin{align*}
		\nonzeroof{\contractionof{\extnet\cup\secextnet}{\catvariableof{\secnodes}}} \prec
		\nonzeroof{\contractionof{\extnet}{\catvariableof{\secnodes}}} \, .
	\end{align*}
\end{theorem}
\begin{proof}
	It suffices to show that for any $\catindexof{\secnodes}$ with
		\[ \nonzeroof{\contractionof{\extnet\cup\secextnet}{\indexedcatvariableof{\secnodes}}}=1 \]
	we also have
		\[ \nonzeroof{\contractionof{\extnet}{\indexedcatvariableof{\secnodes}}}=1 \, . \]
	For any $\catindexof{\secnodes}$ satisfying the first equation we find an extension $\catindexof{\nodes}$ to all variables of the tensor networks such that
		\[ \contractionof{\extnet\cup\secextnet}{\indexedcatvariableof{\nodes}} > 0 \]
	and it follows that
		\[ \contractionof{\extnet}{\indexedcatvariableof{\nodes}} > 0 \quad\text{and}\quad  \contractionof{\secextnet}{\indexedcatvariableof{\nodes}} > 0  \, . \]
	But this already implies, that
		\[ \nonzeroof{\contractionof{\extnet}{\indexedcatvariableof{\secnodes}}}=1 \, . \]
\end{proof}

\subsect{Invariance of adding subcontractions}

%We now show \theref{the:booleanContractionInvariance} of \charef{cha:logicalReasoning}.
Let us now state an equivalence of the contraction, when we add the result of the same contraction.
This property was used in the proof of \theref{the:soundnessKnowledgePropagation}.

\begin{theorem}[Invariance under adding subcontractions]\label{the:invarianceAddingSubcontractions}
	Let $\extnet$ be a tensor network of non-negative tensors with variables $\catvariableof{\nodes}$ and let $\secextnet$ be a subset.
	Then we have for any subset $\catvariableof{\secnodes}$ of $\catvariableof{\nodes}$
		\[ \contractionof{\extnet \cup\{
			\nonzeroof{
			\contractionof{\secextnet}{\catvariableof{\secnodes}}
			}
		\}}{\catvariableof{\nodes}}
		= \contractionof{\extnet}{\catvariableof{\nodes}}
		\, . \]
\end{theorem}
\begin{proof}
	For any $\catindexof{\nodes}$ with
		\[ \contractionof{\extnet}{\indexedcatvariableof{\nodes}} = 0 \]
	we also have
		\[ \contractionof{\extnet \cup\{
			\nonzeroof{
			\contractionof{\secextnet}{\catvariableof{\secnodes}}
			}
		\}}{\indexedcatvariableof{\nodes}} = 0 \, . \]
	For any $\catindexof{\nodes}$ with
		\[ \contractionof{\extnet}{\indexedcatvariableof{\nodes}} \neq 0 \]
	we have for the reduction $\catindexof{\secnodes}$ of the index $\catindexof{\nodes}$ that
		\[  \contractionof{\secextnet}{\indexedcatvariableof{\secnodes}} \neq 0 \]
	and thus
	\begin{align*}
		\contractionof{\extnet \cup\{
			\nonzeroof{
			\contractionof{\secextnet}{\catvariableof{\secnodes}}
			}
		\}}{\indexedcatvariableof{\nodes}}
		= \contractionof{\extnet}{\indexedcatvariableof{\nodes}} \cdot \nonzeroof{
			\contractionof{\secextnet}{\catvariableof{\secnodes}}
			}[\indexedcatvariableof{\secnodes}]
		= \contractionof{\extnet}{\indexedcatvariableof{\nodes}} \, .
	\end{align*}
%	When the subcore transformed by $\nonzeroof{\cdot}$ contains a zero slice, then this
%	 zero slice is also appearing in the rest contraction.
%	Multiplying a zero slice with zero does not affect the contraction, neither does multiplication with one on any slice.
\end{proof}


\begin{remark}
	Similar statements hold, when dropping the non-negativity assumption on the, but demanding that all variables are left open.
\end{remark}

\subsect{Basis Calculus as message passing scheme}

Message Passing of directed and boolean message by relational encoding of functions can be interpreted as function evaluation.
Each subfunction evaluation is passed in its one-hot encoding.

This is because any relational encoding of a function, the decomposition
\begin{align*}
	\rencodingof{\exfunction} = \sum_{y \in\imageof{\exfunction}} ( \sum_{i: \exfunction(i)=y}\onehotmapof{i} )  \otimes \onehotmapof{y}
\end{align*}
is a SVD of the matrification of $\rencodingof{\exfunction}$ with respect to incoming and outgoing legs.


Passing a message $\onehotmapof{i}$ in direction thus gives the message $\onehotmapof{\exfunction(i)}$.

Note, that this is exact, whenever the graph is directed and acyclic.
We do not need acyclicity of the underlying undirected graph.


%After having established a one-to-one connection between the directed and binary tensors with the encoding of functions, we now interpret contractions as evaluations of the respective functions.
%Applying this insight iteratively on composed functions we show the following theorem.

\begin{remark}[Basis Calculus as Message Passing]
	Given a tensor network of directed and binary tensor cores, each representing a function $\exfunctionof{\edge}$ depending on variables $\incomingnodes$.
	When there are not directed cycles, we define the compositions of $\exfunctionof{\edge}$ to be the function $\exfunction$ from the nodes $\nodesone$ not appearing as incoming nodes to the nodes $\nodestwo$ not appearing as outgoing nodes in an edge.
	Choosing arbitrary $\catindexof{\node}\in[\catdimof{\node}]$ for $\node\in\nodesone$ we have
	\begin{align*}
		\contractionof{\{\rencodingofat{\exfunctionof{\edge}}{\catvariableof{\outgoingnodes},\catvariableof{\incomingnodes}} \, : \edge=(\outgoingnodes,\incomingnodes)\in\edges\}}{\nodestwo}
	= \onehotmapof{\exfunction(\catindexof{\node} \, : \, \node\in\nodesone)}\, .
	\end{align*}
\end{remark}



\subsect{Application}

% Application
This properties can be applied as a sparsification of tensors before the execution of a contraction.
The Knowledge Propagation \algoref{alg:knowledgePropagation} produces in the knowledge cores conditions on non-vanishing coordinates.
Thus, the knowledge cores can be locally contracted with the tensors, as a sparsification before performing a global contraction.






\sect{Discussion}

Computing contractions by message passing is known to the graphical model community as belief propagation.
There, the objective is the calculation of marginal probabilities of Markov Networks, which involve contractions of the corresponding factor tensors.

\begin{remark}[Approximate Message Passing Schemes]
	\red{When the cluster graphs are not trees, we cannot find a topological order of the clusters any more.
	Messages can still be defined implicitly by received neighbored messages, but the equivalence with global contractions cannot be established in general.}

	Such algorithms are known in the graphical model community as loopy belief propagation.
\end{remark}




% Application: Dynamic programming
When queries share same parts, can perform their contraction using dynamic programming.
For conditional probability queries, which variables are the clusters of a cluster tree, this results in belief propagation.













    \appendix
    \chapter{Implementation in the \tnreason package}\label{cha:implementation}

We here document the implementation of the discussed concepts in the \python package \tnreason.
 
 % Name
\tnreason is an abbreviation of \textbf{t}ensor \textbf{n}etwork \textbf{reason}ing, by which we emphasize the capabilities of this package to represent and answer reasoning tasks by tensor network contractions. 

% Installation
The package can be installed either by cloning \href{https://github.com/EnexaProject/enexa-tensor-reasoning}{https://github.com/EnexaProject/enexa-tensor-reasoning} or by
\begin{lstlisting}
	!pip install tnreason
\end{lstlisting}

\sect{Architecture}

\tnreason is structured in four subpackages and three layers
\begin{itemize}
	\item Layer 1: Storage and numerical manipulations, by subpackage \spengine, "Tensor Networks" -> building "tn" of \tnreason
	\item Layer 2: Specification of workload, subpackage \spencoding specific for storage, subpackage \spalgorithms specific for manipulations
	\item Layer 3: Applications in reasoning, by subpackage \spknowledge, "Reasoning" -> building "reason" of \tnreason
\end{itemize}

We sketch this structure by
\begin{center}
%! Author = alexgoessmann
%! Date = 09.03.25

\begin{tikzpicture}[scale=0.35]
    \draw[dashed] (-30,15) -- (12,15) -- (12,-3) -- (-30,-3) -- (-30,15);

    \draw (-10,10) rectangle (10,14);
    \node [anchor=center] at (0,12) {\spapplication};

    \node [anchor=center] at (-20,12) {\layerthreespec};
    \draw[dashed] (-30,9) -- (12,9);
    \node [anchor=center] at (-20,6) {\layertwospec};

    \draw[->-] (6,10) -- (6,8);
    \draw (2,4) rectangle (10,8);
    \node [anchor=center] at (6,6) {\spreasoning};
    \draw[->-] (6,4) -- (6,2);

    \draw[->-] (-6,10) -- (-6,8);
    \draw (-10,4) rectangle (-2,8);
    \node [anchor=center] at (-6,6) {\sprepresentation};
    \draw[->-] (-6,4) -- (-6,2);

    \draw[dashed] (-30,3) -- (12,3);
    \node [anchor=center] at (-20,0) {\layeronespec};

    \draw (-10,-2) rectangle (10,2);
    \node [anchor=center] at (0,0) {\spengine};
\end{tikzpicture}
\end{center}



\sect{Subpackage \spengine}

The \spengine subpackage is for the storage and numerical manipulation of tensors and tensor networks.

We think of it as the lowest layer, specializing in storage of Tensor Networks and performing the contractions.

\subsect{Contraction Calculus} % -> To engine section?

We have described two main encoding schemes of functions, by a direct interpretation of functions as tensors or a more relational encoding.
Both come with a different calculus scheme, which we have framed coordiante calculus and basis calculus.



\subsect{Cores and Contractions}

\textbf{Cores}

Each Tensor core has attributes
\begin{itemize}
	\item values (array-like): storing the value of the coordinates
	\item colors (list of str): specifying the name of the variables represented by its axes
	\item name (str): to distinguish from other cores
\end{itemize} 
The implemented core types differ in the values argument.
Cores are instantiated by
\begin{lstlisting}
	engine.getCore(coreType)(coreValues, coreColors, coreName)
\end{lstlisting}

\textbf{Polynomial Cores}
Polynomial Cores are implementations of the monomial decomposition or basis+ (see \defref{def:polynomialSparsity}).
Here the each tuple $(\lambda,\variableset,\catvariableof{\variableset})$ is stored as a tuple of the scalar $\lambda$ and a dictionary with $\variableset$ as keys and $\catvariableof{\variableset}$ as values.

\red{The spare cores (Polynomial and Pandas Core) exploit the matrix representation of \remref{rem:matSotrageBasPlus}.}

% Contraction Method List
The supported cores are
\begin{center}
\begin{tabular}{|c|c|c|}
  	\hline
 	\textbf{coreType} & \textbf{Package} & \text{Explanation}  \\
  	\hline
 	\stringof{NumpyTensorCore} 	&  $\mathrm{numpy}$  & Numpy array storing the values\\
  	\hline
 	\stringof{PolynomialCore} 	&  $\mathrm{numpy}$  & Storeing the values in a binary CP Decomposition\\
  	\hline
\end{tabular}
\end{center}


\textbf{Binary CP Decomposition}

Based on the monomial decomposition $\slicesparsityof{\cdot}$ as specified in \defref{def:polynomialSparsity}.
To store the values of a tensor we store the slices of tensors by the indices $\catindexof{\variableset}$. 

% Trick -> To BinaryCP
Contractions can be performed by partially contracting the cores of the decomposition.
In this way, one can avoids coordinatewise storages of high-order tensors, which can be intractable.

\textbf{Tensor Networks}

Tensor networks $\tnetof{\graph}$ are defined by hypergraphs with hyperedges decorated by tensor cores. 
We store them by dictionaries with values being tensor cores and keys coinciding with the name of each tensor core.


\textbf{Contractions}

Reflected in the notation
	\[ \contractionof{\tnetof{\graph}}{\nodevariables} \]
a contraction is defined by
\begin{itemize}
	\item Tensor Network $\tnetof{\graph}$, i.e. a dictionary of tensor cores
	\item Open Variables $\nodes$
\end{itemize}
Contraction calls are done by
\begin{lstlisting}
	engine.contract(contractionMethod, coreDict, openColors, dimensionDict, evidenceColorDict)
\end{lstlisting}
Where
\begin{itemize}
	\item contractionMethod: str, chooses one of the contraction providers
	\item coreDict: Dictionary of TensorCores (of the above formats), representing the Tensor Network $\tnetof{\graph}$ 
	\item openColors: List of str, each str identifying a color, that is a variable to be left open in the contraction
	\item dimensionDict: Dict valued by int and keys by str, storing dimensions to each variable. This is of optional usage, when a color in openColors does not appear in the coreDict.
	\item evidenceColorDict: Dict valued by int and keys by str, indicating sliced variables
\end{itemize}

% Contraction Method List
The supported contraction methods are
\begin{center}
\begin{tabular}{|c|c|c|}
  	\hline
 	\textbf{contractionMethod} (str) & \textbf{Package} & \text{Explanation}  \\
  	\hline
 	\stringof{NumpyEinsum} 	&  $\mathrm{numpy}$  & Einstein summation of $\mathrm{numpy}$ arrays\\
  	\hline
 	\stringof{TensorFlowEinsum} 	&  $\mathrm{tensorflow}$  & Einstein summation of $\mathrm{tensorflow}$ tensors\\
  	\hline
	\stringof{TorchEinsum} 	&  $\mathrm{torch}$  & Einstein summation of $\mathrm{torch}$ tensors\\
  	\hline
	\stringof{TentrisEinsum} 	&  $\mathrm{tentris}$  & Einstein summation of $\mathrm{tentris}$ hypertries\\
  	\hline
	\stringof{PgmpyVariableEliminator} 	&  $\mathrm{pgmpy}$  & Variable Elimination of DiscreteFactors in $\mathrm{pgmpy}$\\
  	\hline
	\stringof{PolynomialContractor} 	&  $\mathrm{numpy}$  & Contraction of CP Decompositions stored in $\mathrm{numpy}$ arrays\\
  	\hline	
\end{tabular}
\end{center}


%\textbf{Einstein Summation}
Contractions represented as Einstein summation, as implemented in:
\begin{itemize}
	\item numpy
	\item tensorflow
	\item pytorch
	\item tentris
\end{itemize}

%\textbf{Variable Elimination}
Contractions can be executed by variable elimination as implemented in:
\begin{itemize}
	\item pgmpy
\end{itemize}

\textbf{Manipulation of Binary CP Decomposition}
Contraction of tensors in Binary CP Decomposition as in \secref{sec:BinaryCPManipulation}.

\textbf{Coordinate Calculus}

Main function
\begin{lstlisting}
engine.coordinate_transform(coresList, transformFunction)
\end{lstlisting}

\textbf{Basis Calculus}

Main function 
\begin{centeredcode}
	engine.relational\_encoding()
\end{centeredcode}
basis calculus then based on contractions





\sect{Subpackage \spencoding}

In the \spencoding subpackage we encode maps 
Here the relational encodings $\rencodingof{\exfunction}$ of various maps $\exfunction$ are created.
%
The maps are either specified by the script language (logical formulas or neuro-symbolic architectures), categorical constraints or data.
Given a specification of a formula $\exformula$ in script language $\synencodingof{\cdot}$, the task amounts to building a semantic representation based on the syntactic specification.

We arrange the \spencoding subpackage into the second layer of the \tnreason architecture, since it specifies tensor cores which formats are specified in \spengine.


\subsect{Script Language}\label{subsec:scriptLanguage} % To encoding?

To specify propositional sentences, neuro-symbolic architectures and Markov Logic Networks, we developed a script language.

\textbf{Propositional Sentences by Nested Lists}

%\textbf{Production Rules}
Are those of Propositional Logics, but instead of brackets we nest the symbols into lists.

% Connectives
\textbf{Connectives} are represented by strings, where the following are supported (see \defref{def:connectives}):
\begin{center}
\begin{tikzpicture}
\node [anchor=center] at (0,0) {
	\begin{tabular}{|c|c|}
  	\hline
 	\textbf{Unary connective $\exconnective$} & \textbf{$\synencodingof{\exconnective}$} \\
  	\hline
 	$\lnot$ 	&  \stringof{not} \\
  	\hline
 	$()$		&  \stringof{id} \\
  	\hline
	\end{tabular}};
\node [anchor=center] at (7,0) {
	\begin{tabular}{|c|c|}
  	\hline
 	\textbf{Binary connective $\exconnective$} & \textbf{$\synencodingof{\exconnective}$} \\
  	\hline
 	$\land$ 		&  \stringof{and} \\
  	\hline
 	$\lor$ 		&  \stringof{or} \\
  	\hline
 	$\Rightarrow$ 	&  \stringof{imp} \\
  	\hline
	 $\oplus$ 		&  \stringof{xor} \\
  	\hline
	 $\Leftrightarrow$ &  \stringof{eq} \\
  	\hline
	\end{tabular}};
\end{tikzpicture}
\end{center}

% WOLFRAM Numbers
Besides these specific connectives we exploit a generic representation scheme of propositional formulas by the so-called Wolfram code orginially designed for the classification of cellular automaton rules \cite{wolfram_statistical_1983} and popularized in the book \cite{wolfram_new_2002}.
Along this, the coordinate encodings of connectives $\exconnective$ with differing arity are flattened and interpreted as a binary number, which is transformed into a decimal number and represented as a string $\synencodingof{\exconnective}$.
We then choose a prefix to encode the arity by
\begin{itemize}
	\item \stringof{u} for unary
	\item \stringof{b} for binary
	\item \stringof{t} for ternary
	\item \stringof{q} for quarternary
\end{itemize}
connectives.
Together, the connective is represented by the string concatenation
	\[  \synencodingof{\exconnective} = \synencodingof{\catorder} + \synencodingof{\exconnective} \, . \]


% Atoms
\textbf{Atomic Formulas} are represented by arbitrary strings, which are not used for the representation of connectives. 
We further avoid the symbols \{\stringof{(}, \stringof{)}, \stringof{\_}\} in the names of atoms, to not confuse them with colors of categorical variables.

% Composed Formulas
\textbf{Composed Formulas} $\exformula_1\exconnective,\exformula_2$ are represented by 
\begin{centeredcode}
	$\synencodingof{\exformula_1\exconnective,\exformula_2}$ = [$\synencodingof{\exconnective}$, $\synencodingof{\exformula_1}$, $\synencodingof{\exformula_2}$]
\end{centeredcode}
where we apply the conventions
\begin{itemize}
	\item Connectives are at the 0th position in each list
	\item Further entries are either atoms as strings or encoded formulas itself
\end{itemize}

% Backus-Naur
The applied grammar in Backus-Naur form is \\
\begin{tabular}{|l|l|}
  	\hline
 	Unary Connective & \stringof{not} | \stringof{id}\\
  	\hline
 	Binary Connective & \stringof{and} | \stringof{or} | \stringof{imp} | \stringof{xor}  | \stringof{eq} \\ 
  	\hline
 	Atomic Formula & Set of strings not in Connectives\\
  	\hline
	Complex Formula & Atomic Formula | [Unary Connective, Complex Formula] | \\
	&  [Binary Connective, Complex Formula, Complex Formula] \\
	\hline
\end{tabular}


\begin{example}[Encoding of the Wet Street example]
For example we have
\begin{itemize}
	\item 
	Atomic variable $\var{Rained}$ by
		\begin{centeredcode}
			$\synencodingof{\var{Rained}}$ 
			= \stringof{Rained}
		\end{centeredcode}
	\item 
	Negative literal $\lnot\var{Rained}$ by
		\begin{centeredcode}
			$\synencodingof{\lnot\var{Rained}}$ 
			= [\stringof{not},\stringof{Rained}]
		\end{centeredcode}
	\item 
	Horn clause $\left(\var{Rained}\Rightarrow\var{Wet}\right)$ by 
		\begin{centeredcode}
			$\synencodingof{\var{Rained}\Rightarrow\var{Wet}}$ 
			= [\stringof{imp},\stringof{Rained},\stringof{Wet}]
		\end{centeredcode}
	\item 
	Knowledge Base
	$(\lnot\var{Rained})\land(\var{Rained}\Rightarrow\var{Wet})$ by
		\begin{centeredcode}
			$\synencodingof{\lnot\var{Rained})\land(\var{Rained}\Rightarrow\var{Wet}}$ 
			=  [\stringof{and}, [\stringof{not}, \stringof{Rained}], [\stringof{imp}, \stringof{Rained}, \stringof{Wet}]]
		\end{centeredcode}
\end{itemize}
\end{example}




\textbf{Knowledge Bases}

% Should distinguish these in knowledge?
We distinguish here formulas, with propositional logic interpretation and formulas which have a soft logic interpretation.
%\textbf{Facts.} 
The formulas with hard interpretation are called facts in a knowledge base $\kb$ and encoded by dictionaries
\begin{centeredcode}
	\{key($\exformula$) : $\synencodingof{\exformula}$ for $\exformula\in\kb$ \}
\end{centeredcode}

\textbf{Markov Logic Networks}

%\textbf{Weighted formulas.} 
The formulas with soft interpretation are called weighted formulas and encoded by $\expof{\weightof{\exformula}\cdot\exformula}$.
We thus require a specification of the weights, which we do by adding $\weightof{\exformula}$ as a $\mathrm{float}$ or an $\mathrm{int}$ to the list $\synencodingof{\exformula}$.
We then store Markov Logic Networks by dictionaries
\begin{centeredcode}
	\{key($\exformula$) : $\synencodingof{\exformula}$ + [$\weightof{\exformula}$] for $\exformula\in\formulaset$\}
\end{centeredcode}

\textbf{Neuro-Symbolic Architecture by Nested Lists}

% Generalizing the script language to specify architectures
To specify neuro-symbolic architectures in terms of formula selecting maps, as has been the subject of \charef{cha:formulaSelection} we further exploit the nested list structure of encoding propositional logics.
We replace, in each hierarchy of the nested structure each entry by a list of possible choices.
In this way, we reinterpret the list index as the choice indices $\selindex$ introduced for connective and formula selections (see \defref{def:connectiveSelector} and \ref{def:formulaSelector}).

% Neurons
A connective selector (see \defref{def:connectiveSelector}) is encoded by the list
	\begin{centeredcode}
			$\synencodingof{\exconnective}$ 
			= [$\synencodingof{\exconnective_{0}}$, $\ldots$, $\synencodingof{\exconnective_{\seldim\shortminus1}}$]
	\end{centeredcode}
and a formula selector (see \defref{def:formulaSelector}) by
	\begin{centeredcode}
			$\synencodingof{\fselectionmap}$ 
			= [$\synencodingof{\exconnective_{0}}$, $\ldots$, $\synencodingof{\exconnective_{\seldim\shortminus1}}$]
	\end{centeredcode}
A logical neuron of order $\selorder$ (see \defref{def:fsNeuron}), defined by a connective selector $\exconnective$, and a formula selector $\fselectionmap_\atomenumerator$ on each argument $\atomenumerator\in[\selorder]$, is encoded by
		\begin{centeredcode}
			$\synencodingof{\lneuron}$ 
			= [$\synencodingof{\exconnective}$, $\synencodingof{\fselectionmap_0}$, $\ldots$,  $\synencodingof{\fselectionmap_{\selorder-1}}$]
		\end{centeredcode}
Only the unary $\selorder=1$ and the $\selorder=2$ cases are supported.


% Confusing?
The resulting nested lists indices have an alternating interpretation at each level compared with the elements of each list.
That is, when $\synencodingof{\lneuron}$ is the encoding of a neuron, then any element $x\in\synencodingof{\lneuron}$ represents a list of choices.
When $x$ is not the first element, then each choice is either the encoding $\synencodingof{\catvariable}$ of an atomic formula, or another neuron. 

% Find another symbol?
A neural architecture $\larchitecture$ is then represented in the dictionary
\begin{centeredcode}
	$\synencodingof{\larchitecture}$ = \{key($\lneuron$) : $\synencodingof{\lneuron}$ for $\exformula\in\larchitecture$\}
\end{centeredcode}
%To store this structure, we choose dictionaries of neuron spe
%\begin{centeredcode}
%	\{key($\lneuron$) : $\synencodingof{\lneuron}$ for $\exformula\in\formulaset$\}
%\end{centeredcode}
where key($\lneuron$) is a string, which can be used in the formula selections of other neurons.

% Important for well-definedness
It is important that the directed graph of neurons induced by the choice possibilities is acyclic, to ensure well-definedness of the architecture.


% Backus-Naur
In order to represent neuro-symbolic architectures, the grammar of $\synencodingof{\cdot}$ in Backus-Naur Form is extended by the production rules \\
\begin{tabular}{|l|l|}
  	\hline
 	Unary Connectives & [Unary Connective] | [Unary Connective] + Unary Connectives \\
  	\hline
 	Binary Connectives & [Unary Connective] | [Binary Connective] + Binary Connectives \\
	%\hline
	%Neuron Name & Any set of strings not used for atoms or connectives \\
  	\hline
 	Dependency Choice & Atomic Formula | Neuron \\ 
  	\hline
	Dependency Choices & [Dependency Choice] | [Dependency Choice] + Dependency Choices \\ 
	\hline
	Neuron & [Unary Connectives, Dependency Choices] | \\
	&  [Binary Connectives, Dependency Choices, Dependency Choices] \\
	\hline
\end{tabular}


\begin{example}[Neuro-Symbolic Architecture for the Wet Street]
	Following the wet street example, we can define a neuron by
	\begin{centeredcode}
		$\synencodingof{\lneuron}$ = [[\stringof{imp},\stringof{eq}],[\stringof{Wet},\stringof{Sprinkler}],[\stringof{Street}]] 
	\end{centeredcode}
	from which the formulas 
	\begin{centeredcode}
		[\stringof{imp}, \stringof{Wet}, \stringof{Street}] \\
		\hspace{0.25cm} [\stringof{eq}, \stringof{Wet}, \stringof{Street}] \\
		\hspace{1cm}[\stringof{imp}, \stringof{Sprinkler}, \stringof{Street}] \\
		\hspace{1cm}[\stringof{eq}, \stringof{Sprinkler}, \stringof{Street}]		
	\end{centeredcode}
	can be chosen.
	Combining this neuron with further neurons, e.g. by the architecture
	\begin{centeredcode}
		$\synencodingof{\larchitecture}$ = \{ \stringof{neur1}: [[\stringof{imp},\stringof{eq}],[\stringof{neur2}],[\stringof{Street}]] , \\
		\hspace{1.8cm}\stringof{neur2}: [[\stringof{lnot},\stringof{id}],[\stringof{Wet},\stringof{Sprinkler}],[\stringof{Street}]] \}
	\end{centeredcode}
	the expressitivity increases.
	In this case, the further neuron provides the flexibility of the first atoms to be replaced by its negation.	
\end{example}





\subsect{Core Nomenclature}

In encoding.suffixes we defined suffixes for the names of cores and colors, which highlight their origin and purpose.

% See in encoding subpackage
Cores are named with suffixes based on their functionality
\begin{itemize}
	\item \comCoreSuf: Computation core: logical connectives (relational encoding of the connective map)
	\item \actCoreSuf: Activation core: two-dimensional vectors representing of the activation core to a formula
\end{itemize}

Exploiting efficient representation tricks we further have:
\begin{itemize}
	\item \atoCoreSuf: Atomization core, for sparse representation of categorical constraints
	\item \vselCoreSuf: Variable selection core: For sparse representation of variable selectors
\end{itemize}

\subsect{Color Nomenclature}

Suffixes in color string denote the type of the variable:
\begin{itemize}
	\item \disVarSuf: Distributed variables $\catvariableof{\cdot}$
	\item \comVarSuf: Computed variables $\headvariableof{\cdot}$
	\item \selVarSuf: Selection variables $\selvariableof{\cdot}$
	\item \terVarSuf: Term variables $\indvariableof{\cdot}$
%	\item %\item Selection variables $\selvariableof{\cdot}$: \stringof{\_sVar} % differing suffixes actVar, selVar! -> unify them
%	\item Term variables $\indvariableof{\cdot}$: tVar % so far without suffix, just atom names, toDo: tVar
\end{itemize}

\subsect{Refinement by infixes}

Both the cores and the colors are further refined by infixes before the suffices to denote specific instantiations.

\begin{itemize}
	\item \selCoreIn: Involving a selection variable
	\item \eviCoreIn: Storing evidence about a variable
	\item \heaIn: Head of a function, typically the variable computed at a activation selector
	\item \funIn: Function selection variables
	\item \posIn+\stringof{i}: Variable selection for argument at position $i$
	\item \datIn: Involving data (data cores and colors)
\end{itemize}

Further infixes are strings denoting atom names and neuron neames.


\subsect{Relational encoding of formulas}

Propositional formulas $\exformula$ are represented in three schemes:
\begin{itemize}
	\item Script language $\synencodingof{\exformula}$ by nested lists (see \secref{subsec:scriptLanguage}).
		Most practical to choose a formula from a neuro-symbolic architecture.
	\item Strings specifying the categorical variables $\catvariableof{\exformula}$.
	\item Representation of formulas by tensor networks being contracted to $\rencodingof{\exformula}$
\end{itemize}

Conversions of the formats:
\begin{itemize}
	\item $\synencodingof{\exformula}$ to color by
		\begin{centeredcode} 
			encoding.get\_formula\_color($\synencodingof{\exformula}$)
		\end{centeredcode}
		Here the nested lists are turned in a string by concatenating all elements of a list with \stringof{\_} and adding \stringof{[} and \stringof{]} at the beginning and end of each list.
	\item  $\synencodingof{\exformula}$ to tensor network 
		\begin{centeredcode}
			encoding.create\_raw\_cores($\synencodingof{\exformula}$)
		\end{centeredcode}
		This creates the connective cores for the semantic representation of $\rencodingof{\exformula}$.
We encode them by
\end{itemize}

When encoding formulas with hard interpretation, we furthermore add a head core of type \stringof{truthEvaluation} since we have
 	\[ {\exformula} = \sbcontractionof{\rencodingof{\exformula},\tbasis}{\catvariableof{\exformula}} \, . \]



\subsect{Representation of MLNs}

\textbf{Computation Cores} are binary cores relating the variables in a predefined way, which is not changing during reasoning.
\begin{itemize}
	\item Logical interpretation: Cores $\rencodingof{\exconnective}$ \red{Structure Cores are those of the Bayesian Propositional Network}
	\item Categorical constraints: Cores $\categoricalcore$
	%\item Data: Cores $\datacore$
\end{itemize}

\textbf{Activation Cores} encode the weights of the formulas in a Markov Logic Network.
%For proper MLN only have unary cores, which we call headCores.
%Head cores with suffix "headCore" in name.

They are modified during reasoning: Selection of activation cores in structure learning, assigning a weight in parameter estimation.



\subsect{Formula Selecting Maps}

Encoding of Neurons according to \defref{def:fsNeuron}:
\begin{itemize}
	\item Activation selection core with suffix \stringof{actCore} in name.
		 Selection by variable with suffix \stringof{actVar}
	\item Selection of neurons as arguments with suffix \stringof{selCore} in name.
		Each argument of each neuron comes with a control variable with suffix \stringof{selVar}.
\end{itemize}

Encoding of Formula Selecting Neural Networks (\defref{def:fsNeuron}) by creating all formula selecting neurons.

Skeleton expression (\defref{def:skeleton}) are stored with placeholderkeys and the candidatelists by dictionaries with the placeholderkeys and values being the possible symbols.



\sect{Subpackage \spalgorithms}

The \spalgorithms subpackage implements basic tensor network algorithms with calls of specific execution in \spengine.
As the \spencoding subpackage it is arranged in the second layer of the \tnreason architecture, since it specifies the manipulation of tensor networks in the \spengine subpackage.


\subsect{Alternating Least Squares}

\begin{itemize}
	\item Tensor Network of Structure Cores
	\item Tensor Network of Parameter Cores
	\item List of importance cores
\end{itemize}

\begin{centeredcode}
	algorithms.ALS
\end{centeredcode}

\subsect{Gibbs Sampling}

\begin{itemize}
	\item Tensor Network of Structure Cores
	\item Parameter cores: Variable tensor network cores representing basis vectors.
	\item List of importance cores
\end{itemize}

\begin{centeredcode}
	algorithms.Gibbs
\end{centeredcode}


\subsect{Knowledge Propagation}

\begin{centeredcode}
	algorithms.ConstraintPropagator
\end{centeredcode}


\subsect{Energy-based Algorithms}

\begin{centeredcode}
	algorithms.NaiveMeanField
\end{centeredcode}

\begin{centeredcode}
	algorithms.GenericMeanField
\end{centeredcode}

\begin{centeredcode}
	algorithms.EnergyBasedGibbs
\end{centeredcode}


\sect{Subpackage \spknowledge}

With the \spknowledge subpackage we provide an interface for reasoning workload.
It builds a third layer, since it used \spencoding to represent knowledge by tensor networks and \spalgorithms in the execution of reasoning tasks.

\subsect{Distributions}

We encode Markov Networks by specifying a set of tensor cores.
Each distribution needs to have a routine
\begin{centeredcode}
	.create\_cores()
\end{centeredcode}
creating the factor cores and 
\begin{centeredcode}
	.get\_partition\_function()
\end{centeredcode}
calculating the partition function.
Although the partition function can be calculated by the contraction of all cores, we separate the method since there are situations where a faster calculation can be performed.


\textbf{Empirical Distributions} are distributions of sample data.
We represent the values as a CP Format of data cores as specified in \secref{sec:empDistribution}
\begin{centeredcode}
	knowledge.EmpiricalDistribution
\end{centeredcode}
Here the partition function is the number of samples used in the creation of the empirical distribution.


\textbf{HybridKnowledgeBases} are probability distributions, which are specified by propositional formulas in the script language.
\begin{centeredcode}
	knowledge.HybridKnowledgeBase
\end{centeredcode}
They are initialized with arguments
\begin{itemize}
	\item facts: Dictionary of propositional formulas stored as $\synencodingof{\exformula}$ representing hard logical constraints
	\item weightedFormulas: Dictionary of propositional formulas stored as $\synencodingof{\exformula}$+$[\weightof{\exformula}]$ representing soft logical constraints
	\item evidence: Dictionary of atomic formulas, where key are the formulas in string representation and values the certainty in $[0,1]$ (float or int) of the atom being true
	\item categoricalConstraints: Dictionary of categorical constrained, which values are lists of atomic formulas stored as strings $\synencodingof{\atomicformula}$
\end{itemize}


\subsect{Inference}

By
\begin{centeredcode}
	knowledge.InferenceProvider
\end{centeredcode}
taking a distribution from the above as argument.

% Probabilistic Queries
Probabilistic queries as specified \defref{def:queries})  by
\begin{centeredcode}
	.query(variableList, evidenceDict)
\end{centeredcode}

MAP queries by
\begin{centeredcode}
	.exact\_map\_query()
\end{centeredcode}
or by
\begin{centeredcode}
	.annealed\_sample()
\end{centeredcode}
using Simulated Annealing (see Remark~\ref{rem:simulatedAnnealing}) to find an approximate maximum.
The second method circumvents the creation of the coordinatewise representation of the distribution and circumvents therefore, at the expense of potentially approximative solutions, a bottleneck in case of many query variables.

% Entailment Queries
Entailment from the distribution (\defref{def:entailment}) is be decided by
\begin{centeredcode}
	.ask(queryFormula, evidenceDict)
\end{centeredcode}
where queryFormula is the formula $\exformula$ to be tested for entailment in the representation $\synencodingof{\exformula}$.

% Sampling
Samples can be drawn by
\begin{centeredcode}
	.draw\_samples(sampleNum, variableList, annealingPattern)
\end{centeredcode}
based on Gibbs sampling, where
\begin{itemize}
	\item sampleNum (int) gives the number of samples to be drawn
	\item variableList (list of str) defines the variables to be represented by the samples (default: all atoms in the distribution)
	\item annealingPattern specifies an annealing pattern 
\end{itemize}


\subsect{Parameter Estimation}

\textbf{EntropyMaximizer} implements Algorithm~\ref{alg:AWO}, which is motivated by the maximum entropy principle (see \secref{sec:maxEntDuality}) to optimize Markov Logic Networks.
The class  
\begin{centeredcode}
	knowledge.EntropyMaximizer
\end{centeredcode}
is initialized with the arguments
\begin{itemize}
	\item expressionsDict: Dictionary of formulas in the format $\synencodingof{\exformula}$ 
	\item satisfactionDict: Dictionary of the satisfaction rates (mean parameters) to be matched by the optimal distribution
\end{itemize}
The optimization is then performed by
\begin{centeredcode}
	.alternating\_optimization(sweepNum, updateKeys)
\end{centeredcode}
method, where the iteration in Algorithm~\ref{alg:AWO} through the updateKeys is performed sweepNum times.

\subsect{Structure Learning}

\textbf{Formula Booster} chooses a formula given a formula selecting map.
\begin{centeredcode}
	knowledge.FormulaBooster
\end{centeredcode}
is initialized with the arguments
\begin{itemize}
	\item knowledgeBase: Distribution representing a current model to be improved
	\item specDict: A neuro-symbolic architecture encoded in a dictionary of neurons
\end{itemize}




    \chapter{Glossary}

% Match the encoding.suf terminology
\sect{Tensors}

Small greek letters are reserved for the notation of tensors:
\begin{center}
    \begin{tabular}{|p{\threecolumnwidth}|p{\threecolumnwidth}|p{\threecolumnwidth}|}
        \hline
        \rule{0pt}{\rowheight} \textbf{Notation}   & \textbf{Name}                                                & \textbf{Reference}                                                  \\
        \hline
        % alpha
        \rule{0pt}{\rowheight} $\actcorewith$      & Activation Core                                              & \theref{the:expFamilyTensorRep}                                     \\
        \hline
        % beta
        \rule{0pt}{\rowheight} $\bencodingwith$    & Basis Encoding of a function $\exfunction$                   & \defref{def:functionRelationEncoding} \\
        % gamma
        % delta
        \rule{0pt}{\rowheight}  $\dirdeltawith$      & Diracs delta                                                 & \exaref{exa:diracDeltaTensor}                                       \\
        % epsilon
        \rule{0pt}{\rowheight} $\onehotmapwith$    & One-hot Encoding                                             & \defref{def:oneHotEncoding}                                         \\
        % zeta
        % eta
        \rule{0pt}{\rowheight} $\sstatnoisewith$   & Noise tensor                                                 & \defref{def:noiseTensor}                                            \\
        % theta
        \rule{0pt}{\rowheight} $\canparamwith$     & Canonical Parameter                                          & \defref{def:expFamily}                                              \\
        % iota
        % kappa
        \rule{0pt}{\rowheight} $\kcorewith$        & Knowledge core                                               & \defref{def:knowledgeCoreSoundComplete}                             \\
        % lambda
        \rule{0pt}{\rowheight} $\scalarcorewith$   & Scalar core in $\cpformat$ decompositions                    & \defref{def:cpFormats}                                              \\
        % mu
        \rule{0pt}{\rowheight} $\meanparamwith$    & Mean Parameter                                               & \defref{def:meanPolytope}                                           \\
        % nu
        \rule{0pt}{\rowheight} $\basemeasurewith$  & Boolean base measure                                         & \secref{sec:baseMeasure}                                            \\
        % xi
        % omicron
        % pi
        % rho
        \rule{0pt}{\rowheight} $\legcorewith$      & Leg core in $\cpformat$ decompositions                       & \defref{def:cpFormats}                                              \\
        % sigma ! also: neuron
        \rule{0pt}{\rowheight} $\sencodingwith$    & Selection Encoding of a vector-valued function $\exfunction$ & \defref{def:selectionEncoding}                                      \\
        % tau
        \rule{0pt}{\rowheight} $\extnetwith$       & Tensor network of tensors $\hypercorewith$                   & \defref{def:tensorNetwork}                                          \\
        % ypsilon
        % phi -> \sstat
        \rule{0pt}{\rowheight} $\energytensorwith$ & Energy tensor                                                & \defref{def:expFamily}                                              \\
        % chi
        \rule{0pt}{\rowheight} $\cencodingwith$    & Coordinate Encoding of a function $\exfunction$              & \defref{def:coordinateEncoding}, often abbreviated by $\exfunction$ \\
        % psi
        % omega -> width
        \hline
    \end{tabular}
\end{center}

Sets of tensors are represented by large greek letters:
\begin{center}
    \begin{tabular}{|p{\threecolumnwidth}|p{\threecolumnwidth}|p{\threecolumnwidth}|}
        \hline
        \rule{0pt}{\rowheight} \textbf{Notation}      & \textbf{Name}                                             & \textbf{Reference}                       \\
        \hline
        % Gamma
        \rule{0pt}{\rowheight} $\expfamilywith$       & Exponential family                                        & \defref{def:expFamily}                   \\
        % Lambda
        \rule{0pt}{\rowheight} $\realizabledistswith$ & Sets of by $\sstat$ and $\graph$ computable distributions  & \defref{def:realizableStatDistributions} \\
        % Mu
        \rule{0pt}{\rowheight} $\genmeanset$          & Polytope of mean parameters                               & \defref{def:meanPolytope}                \\
        %$\formulasetof{\larchitecture}$ & Expressivity of formula selecting neural networks & \defref{def:fsNeuralNetwork}
        \hline
    \end{tabular}
\end{center}

In the implementation in \tnreason,we distinguish between computation and activation cores.
The coarse roles are the computation of a function using basis calculus and the activation of the prepared variable to shape a probability distribution.

\begin{center}
    \begin{tabular}{|p{\threecolumnwidth}|p{\threecolumnwidth}|p{\threecolumnwidth}|}
        \hline
        \rule{0pt}{\rowheight} \textbf{Name}    & \textbf{Notation}                      & \textbf{String Suffix} \\
        \hline
        \rule{0pt}{\rowheight} Computation Core & $\bencodingof{\cdot}$                  & \comCoreSuf            \\
        \rule{0pt}{\rowheight} Activation Core  & $\actcoreof{\cdot}$, $\kcoreof{\cdot}$ & \actCoreSuf            \\
        \hline
    \end{tabular}
\end{center}


\sect{Variables}

Variables are denoted by large latin letters, their indices by the corresponding small letters.
We distinguish between the coarse types of variables:
\begin{center}
    \begin{tabular}{|p{\threecolumnwidth}|p{\threecolumnwidth}|p{\threecolumnwidth}|}
        \hline
        \rule{0pt}{\rowheight} \textbf{Name}        & \textbf{Notation}        & \textbf{String Suffix} \\
        \hline
        \rule{0pt}{\rowheight} Distributed Variable & $\catvariableof{\cdot}$  & \disVarSuf             \\
        \rule{0pt}{\rowheight} Computed Variable    & $\headvariableof{\cdot}$ & \comVarSuf             \\
        \rule{0pt}{\rowheight} Selection Variable   & $\selvariableof{\cdot}$  & \selVarSuf             \\
        \rule{0pt}{\rowheight} Term Variable        & $\indvariableof{\cdot}$  & \terVarSuf             \\
        \hline
    \end{tabular}
\end{center}


\sect{Maps}

We have in this work encountered different maps, which have been encoded as tensors.
Note, that in order to ease the notation, when not specified otherwise the coordinate encoding $\cencodingof{\cdot}$ has been used.

\begin{center}
    \begin{tabular}{|p{\fivecolumnwidth}|p{\threecolumnwidth}|p{\fivecolumnwidth}|p{\fivecolumnwidth}|p{\fivecolumnwidth}|}
        \hline
        \rule{0pt}{\rowheight} \textbf{Notation} & \textbf{Name}                                       & \textbf{Domain}                                             & \textbf{Range}                                              & \textbf{Reference}                   \\
        \hline
        % f
        \rule{0pt}{\rowheight} $\exformula$      & propositional formula                               & $\atomstates$                                               & $\ozset$                                                    & \defref{def:formulas}                \\
        % h
        \rule{0pt}{\rowheight} $\cselectionmap$  & $\atomorder$-ary connective selecting map & $\atomstates$                                               & $\bigtimes_{\selindex\in[\seldimof{\cselectionsymbol}]}[2]$ & \defref{def:connectiveSelector} \\
        \rule{0pt}{\rowheight} $\vselectionmap$  & Variable selecting map                              & $\bigtimes_{\selindex\in[\seldimof{\vselectionsymbol}]}[2]$ & $\bigtimes_{\selindex\in[\seldimof{\vselectionsymbol}]}[2]$ & \defref{def:variableSelector} \\ % ! They are the identity reduced to
        \rule{0pt}{\rowheight} $\sselectionmap$  & State selection map                                 & $[\catdim]$                                                 & $\bigtimes_{\catenumeratorin}[2]$ & \defref{def:stateSelector} \\
        % k
        \rule{0pt}{\rowheight} $\kb$             & Knowledge Base (conjunction of formulas)            & $\atomstates$                                               & $\ozset$                                                    &                                      \\
        % P
        \rule{0pt}{\rowheight} $\probwith$       & Probability distribution                            & $\facstates$                                                & $[0,1]$                                                     & \defref{def:probabilityDistribution} \\
        % q
        \rule{0pt}{\rowheight} $\exfunction$     & Function between states of factored representations                   & $\facstates$                                                & $\secfacstates$                                             &                                      \\
        % H
        \hline
    \end{tabular}
\end{center}

\sect{Contraction equations}

We here provide a summary for the application of contractions and normalization in probabilistic and logical reasoning. %, which will be introduced in \parref{par:one}.

\begin{center}
    \begin{tabular}{|p{\threecolumnwidth}|p{7cm}|p{2cm}|}
        \hline
        \rule{0pt}{\rowheight} \textbf{Concept}        & \textbf{Contraction Equation}        & \textbf{Reference} \\
        \hline
        \rule{0pt}{\rowheight} Marginal probability & $\probat{\exrandom} = \contractionof{\probtensor}{\exrandom}$  & \defref{def:marginalProbability}    \\
        \rule{0pt}{\rowheight} Conditional probability & $\condprobof{\exrandom}{\secexrandom} = \normalizationofwrt{\probtensor}{\exrandom}{\secexrandom}$  & \defref{def:conditionalProbability}        \\
        \rule{0pt}{\rowheight} Markov Network Distribution & $\probtensor^{\extnet} = \normalizationof{\extnet}{\nodes}$ & \defref{def:markovNetwork} \\
        \rule{0pt}{\rowheight} Partition Function  & $\partitionfunctionof{\extnet} = \contraction{\extnet}$ & \defref{def:markovNetwork} \\
        \rule{0pt}{\rowheight} Independence of $\exrandom$ and $\secexrandom$ &
        $\contractionof{\probtensor}{\exrandom,\secexrandom}
		    =  \contractionof{\probtensor}{\exrandom}
			\otimes  \contractionof{\probtensor}{\secexrandom}$
        & \defref{def:independence}, \theref{the:independenceProductCriterion} \\
        \rule{0pt}{\rowheight} Independence of $\exrandom$ and $\secexrandom$ conditioned on $\thirdexrandom$ &
        $\normalizationofwrt{\probtensor}{\exrandom,\secexrandom}{\thirdexrandom}
		= \normalizationofwrt{\probtensor}{\exrandom}{\thirdexrandom}
		\otimes \normalizationofwrt{\probtensor}{\secexrandom}{\thirdexrandom}$
        & \defref{def:condIndependence}, \theref{the:condIndependenceProductCriterion} \\
        \hline
    \end{tabular}
\end{center}


%In \charef{cha:probRepresentation} we introduce:
%\begin{itemize}
%	\item Marginal probabilities (\defref{def:marginalProbability}, \theref{the:marginalContraction})
%		\[ \probat{\exrandom} = \contractionof{\probtensor}{\exrandom} \]
%	\item Conditional probabilities (\defref{def:conditionalProbability}, \theref{the:conditionalContraction})
%		\[ \condprobof{\exrandom}{\secexrandom} = \normalizationofwrt{\probtensor}{\exrandom}{\secexrandom} \]
%	\item The probability distribution of a Markov Network is (\defref{def:markovNetwork})
%		\begin{align*}
%			\probtensor^{\extnet} = \normalizationof{\extnet}{\nodes}
%		\end{align*}
%		The partition function of a Markov Networks
%		\begin{align*}
%			\partitionfunctionof{\extnet} = \contraction{\extnet}
%		\end{align*}
%		Bayesian Networks (\defref{def:bayesianNetwork}), when hypergraph directed and acyclic, such that the decorating tensors are accordingly directed.
%\end{itemize}
%
%Further the following properties are defined by contraction equations:
%\begin{itemize}
%	\item $\exrandom$ and $\secexrandom$ are independent when (\defref{def:independence}, \theref{the:independenceProductCriterion})
%		\[  \contractionof{\probtensor}{\exrandom,\secexrandom}
%		=  \contractionof{\probtensor}{\exrandom}
%			\otimes  \contractionof{\probtensor}{\secexrandom} \]
%	\item $\exrandom$ and $\secexrandom$ are called independent conditioned on $\thirdexrandom$ when (\defref{def:condIndependence}, \theref{the:condIndependenceProductCriterion})
%		\[ \normalizationofwrt{\probtensor}{\exrandom,\secexrandom}{\thirdexrandom}
%		= \normalizationofwrt{\probtensor}{\exrandom}{\thirdexrandom}
%		\otimes \normalizationofwrt{\probtensor}{\secexrandom}{\thirdexrandom} \]
%\end{itemize}
%
%% Populate!
%In \charef{cha:logicalRepresentation} we introduce:
%\begin{itemize}
%	\item Propositional formulas by boolean tensors (\defref{def:formulas})
%		\[ \formulaat{\shortcatvariables} : \atomstates \rightarrow [2] \subset \rr \, . \]
%	\item Syntactical representation of formulas corresponding with tensor networks of boolean tensors (\theref{the:formulaDecomposition})
%\end{itemize}

    \bibliographystyle{plainnat}
    \bibliography{../references}

\end{document}