\section{Binary Optimization of Sparse Tensors}

Let us now study the problem of searching for the maximal coordinate in a tensor, when the tensor is represented in a basis+ CP Format. 


%\red{Search for the maximal coordinate in a tensor network is a binary optimization problem, given leg dimensions of 2.
%We explain this perspective in this chapter and connect it to the HUBO/QUBO formalism.}


\subsection{Mode search in exponential families}

Mode search 
\begin{align*}
	\max_{\shortcatindices\in\atomstates} \sbcontraction{\sencsstatat{\indexedshortcatvariables,\selvariable},\canparam} 
	= \max_{\meanparam\in\meanset} \sbcontraction{\meanparamat{\selvariable},\canparamat{\selvariable}}
\end{align*}


% Appearance of mode search
The search for maximal coordinates appears in various reasoning tasks:
\begin{itemize}
	\item MAP query as mode search of MLN: $\hypercore$ is the contraction of evidence with the distribution, leaving the query variables open.
	\item Grafting as mode search of proposal distribution: $\hypercore$ is the contraction of the gradient of the likelihood with the relational encoding of the hypothesis.
\end{itemize}
Both tasks have been formulated as mode search problems in exponential families.



\subsection{Higher-Order Unconstrained Binary Optimization (HUBO)}

\red{
Here binary refers to the leg dimensions $\catdimof{\atomenumerator}$ being 2, not to binary coordinates as often refered to in this work.
}


\begin{definition}
	The binary optimization of a tensor $\hypercoreat{\shortcatvariables}\in\atomstates$ is the problem
	\begin{align}\tag{$\mathrm{P}_{\hypercore}$}\label{prob:HUBO}
		\argmax_{\shortcatindices\in\atomstates} \hypercoreat{\indexedshortcatvariables} 
	\end{align}
	
	We call Problem~\ref{prob:HUBO} a Higher Order Unconstrained Binary Optimization (HUBO) problem of order $\sliceorder$ and sparsity $\slicerankwrtof{\sliceorder}{\hypercore}$, when $\hypercore$ has a monomial decomposition (see Definition~\ref{def:polynomialSparsity}) with $\cardof{\variablesetof{\decindex}}\leq\sliceorder$ for all $\decindexin$, that is when $\slicerankwrtof{\sliceorder}{\hypercore}<\infty$.
	
	
\end{definition}


\begin{remark}[Leg dimensions larger than 2]
% Leg dimension needs to be 2
	We demanded leg dimensions $\catdimof{\atomenumerator}=2$ to have binary valued variables $\catvariableof{\catenumerator}$, which is required to connect with the formalism of binary optimization.
	Categorical variables with larger dimensions can be represented by atomization variables, which are created by contractions with categorical constraint tensors (see Section~\ref{sec:categoricalTN}).
\end{remark}


% Interpretation of sparsity
The sparsity $\slicerankwrtof{\sliceorder}{\hypercore}$ is the minimal number of monomials, for which a weighted sum is equal to $\hypercore$.
Thus we interpret Problem~\ref{prob:HUBO} as searching for the maximum in a polynomial consistent of $\slicerankwrtof{\sliceorder}{\hypercore}$ monomial terms.
\red{Each monomial is also refered to as potential.}



%\begin{remark}[Sparsity]% To sparse Tensor Calculus?
%
%The number $\slicesparsityof{\hypercore}$ of a tensor $\hypercore$ defining a HUBO is of central importance to have an effective solution.
%%Here the number of nonzero coordinates coincides with the number of monomials required to represent the polynomial as a sum.
%
%\red{We investigated the Sparsity as the slice sparsity not the vector sparsity in Chapter~\ref{cha:sparseTC}.}
%%We here investigate, whether the same reasoning assumptions used for sparse representation by tensor networks also lead to $\ell_0$-sparse tensors. 
%
%\end{remark}




\subsection{Quadratic Unconstrained Binary Optimization (QUBO)}

\red{Quadratic Unconstrained Binary Optimization problems are HUBOs of order $\sliceorder=2$.}

We refine the monomial decomposition of tensors (see Definition~\ref{def:polynomialSparsity}) by demanding that monomials consist of at most two variables.

\begin{definition}
	We call a monomial decomposition $\sliceset$ of a tensor $\hypercore\in\atomspace$ a quadratic decomposition, if $\cardof{\variableset}\leq 2$ for all $(\lambda,\variableset,\catindexof{\variableset}) \in \sliceset$.
	We denote the smallest cardinality $\cardof{\sliceset}$ among quadratic decompositions of $\hypercore$ by $\quacprankof{\hypercore}$.

	If a tensor $\hypercore\in\bigotimes_{\atomenumeratorin}\rr^2$ has a quadratic decomposition, we call Problem~\ref{prob:HUBO} a Quadratic Unconstrained Binary Optimization (QUBO) problem of sparsity $\quacprankof{\hypercore}$.
\end{definition}

% CP Decompositions
Analogously to monomial decompositions, quadratic decompositions have an equivalence in a CP decomposition of $\hypercore$.
Beyond being binary tensors, the leg cores are further restricted that for each slice $\decindexin$ at most two of them are basis vectors and the rest trivial vectors $\ones$.

% Existence
We notice, that there are tensors, for which no quadratic decomposition exists.
This is already obvious from the fact, that the tensors with a quadratic decomposition build a $\binom{\atomorder}{2}$ dimensional submanifold in the $2^\atomorder$ dimensional tensor space.
This is in contrast with monomial decompositions, where one can always construct a decomposition.



%% OLD THEOREM: FALSE!
%However, for any non-negative tensor $\hypercore$ the Problem~\ref{prob:HUBO} is equivalent to a QUBO problem of possibly larger order as we state next.
%To turn HUBO problems into QUBO we need the slack variable trick, as described in the next lemma.

%\begin{theorem}\label{the:HUBOtoQUBO}
%	Let there be a tensor $\hypercore\in\in\bigotimes_{\atomenumeratorin}\rr^2$, which has a monomial decomposition with dimension $r$ and non-negative scalar core $\scalarcore$.
%	Then, the HUBO defined by $\hypercore$ is equivalent to a QUBO of order at most $\atomorder+r$ and sparsity at most $\atomorder \cdot r $.
%%	The maximal coordinate problem to any tensor $\hypercore\in\bigotimes_{\atomenumeratorin}\rr^2$ is equivalent to a QUBO with at most $\atomorder+\slicesparsityof{\hypercore}$ variables.
%%	\red{Need positive coordinates!}
%\end{theorem}

%To show the theorem we state the following lemma.


We can transform certain HUBO problems in QUBO problems with the usage of auxiliary variables, as we show in the next lemma.

%% Slack variables
\begin{lemma}\label{lem:monomialToQUBO}
	For any $\atomindices\in[2]$ and $\variableset\subset[\atomorder]$ we have 
		\[ \left( \prod_{\atomenumerator\in\variableset} \atomlegindexof{\atomenumerator } \right)  \left(  \prod_{\atomenumerator\notin\variableset} (1- \atomlegindexof{\atomenumerator }) \right)
		=
		\max_{\slackvariable\in[2]} \slackvariable \cdot 2 \cdot \left( \sum_{\atomenumerator\in\variableset}\atomlegindexof{\atomenumerator}  - \cardof{\variableset} - \sum_{\atomenumerator\notin\variableset}\atomlegindexof{\atomenumerator} + \frac{1}{2} \right) \, . % Alternative: no factor 2, but + 1 instead of +1/2 (->pyqubo)
 		\]
\end{lemma}
\begin{proof} %Proof by case distinction
	Only if $\atomlegindexof{\atomenumerator}=1$ for $\atomenumerator\in\variableset$ and $\atomlegindexof{\atomenumerator}=0$ else we have
		\[ \left( \sum_{\atomenumerator\in\variableset}\atomlegindexof{\atomenumerator}  - \cardof{\variableset} - \sum_{\atomenumerator\notin\variableset}\atomlegindexof{\atomenumerator} + \frac{1}{2} \right) \geq 0 \, . \]
	In this case the maximum is taken for $\slackvariable=1$ and we have
		\[ \max_{\slackvariable\in[2]} \slackvariable \cdot 2 \cdot \left( \sum_{\atomenumerator\in\variableset}\atomlegindexof{\atomenumerator}  - \cardof{\variableset} - \sum_{\atomenumerator\notin\variableset}\atomlegindexof{\atomenumerator} + \frac{1}{2} \right) 
		= 1 = \left( \prod_{\atomenumerator\in\variableset} \atomlegindexof{\atomenumerator } \right)  \left(  \prod_{\atomenumerator\notin\variableset} (1- \atomlegindexof{\atomenumerator }) \right) \, . \]
	In all other cases, the maximum is taken for $\slackvariable=0$ and thus vanishes, that is 
		\[ \max_{\slackvariable\in[2]} \slackvariable \cdot 2 \cdot \left( \sum_{\atomenumerator\in\variableset}\atomlegindexof{\atomenumerator}  - \cardof{\variableset} - \sum_{\atomenumerator\notin\variableset}\atomlegindexof{\atomenumerator} + \frac{1}{2} \right) 
		= 0 = \left( \prod_{\atomenumerator\in\variableset} \atomlegindexof{\atomenumerator } \right)  \left(  \prod_{\atomenumerator\notin\variableset} (1- \atomlegindexof{\atomenumerator }) \right) \, . \]
	Thus, the claim holds in all cases.
\end{proof}	


%\begin{proof}[Proof of Theorem~\ref{the:HUBOtoQUBO}]
%	For each summand in the monomial decomposition apply Lemma~\ref{lem:monomialToQUBO}.
%\end{proof}



\subsection{Integer Linear Programming}

Let us now show how optimization problems can be represented as linear programming problems.



\begin{definition}
	A Binary Integer Linear Program (ILP) is a problem of the form
	\begin{align*}
		\max_{x \in\{0,1\}^n} c^T x \quad \text{subject to } \quad A^{upper} x \leq b^{upper} , A^{lower} x \geq b^{lower} 
	\end{align*}
	where $A^{upper}\in\rr^{n^{upper}\times n}$, $b^{upper}\in\rr^{n^{upper}}$, $A^{lower}\in\rr^{n^{lower}\times n}$, $b^{lower}\in\rr^{n^{lower}}$.
\end{definition}



\begin{theorem}
	Given a monomial decomposition $\sliceset=\enumeratedslices$ of a tensor $\hypercore$ we define an Binary ILP as the maximation of 
	\begin{align*}
		\sum_{\decindexin} \slicescalar^{\decindex} \slackvariable^{\decindex} 
	\end{align*}
	with the constraints for any $\decindex$
	\begin{itemize}
		\item 
		\begin{align*}
			\slackvariable^{\decindex}  \leq \catvariableof{\atomenumerator} \quad \text{for} \quad \atomenumerator\in\variableset^j , \catindexof{\atomenumerator} = 1
		\end{align*}
		\item 
		\begin{align*}
			\slackvariable^{\decindex}  \leq (1-\catvariableof{\atomenumerator}) \quad \text{for} \quad \atomenumerator\in\variableset^j , \catindexof{\atomenumerator} = 0
		\end{align*}
		\item 
		\begin{align*}
			\slackvariable^{\decindex} \geq 1 + \sum_{\atomenumerator\in\variableset^{\decindex} : \catindexof{\atomenumerator} = 1} (\catvariableof{\atomenumerator} -1)
		- \sum_{\atomenumerator\in\variableset^{\decindex} : \catindexof{\atomenumerator} = 0} \catvariableof{\atomenumerator} 
		\end{align*}
	\end{itemize}
	The solution $\catindex^{ILP,\sliceset}$ of this ILP and the solution $\catindex^{HUBO,\sliceset}$ of the HUBO coincide on the variables of hypercore, i.e.
		\[ \catindex^{ILP,\sliceset}|_{[d]} =  \catindex^{HUBO,\sliceset} \, . \]
\end{theorem}
\begin{proof}
	We have to show that the constraints are satisfied if and only if $\slackvariable^{\decindex}=\onehotmapofat{\catvariableof{\variableset^{\decindex}}^{\decindex}}{\indexedcatvariableof{\variableset^{\decindex}}}$.
\end{proof}



