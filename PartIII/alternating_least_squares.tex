\section{Reasoning by Tensor Approximation}\label{cha:tensorApproximation}

Often reasoning requires the execution of demanding contractions of tensors networks, or combinatorical search of maximum coordinates.
We in this chapter investigate methods, to replace hard to be sampled tensor networks by approximating tensor networks, which then serve as a proxy in inference tasks.


\subsection{Approximation of Energy tensors}




\subsubsection{Direct Approximation}

Direct approximation would be
	\[ \argmin_{\canparam\in\Gamma^{\graph}} \|\energytensorat{\shortcatvariables} - \canparamat{\shortcatvariables}\|^2 \]


\subsubsection{Approximation involving Selection Architectures}

Direct approximation would be
	\[ \argmin_{\canparam\in\Gamma^{\graph}} \|\energytensor - \contractionof{\sencodingof{\sstat},\canparam}{\shortcatvariables}\|^2 \]

\red{
For example: Approximate based on a slice sparsity selecting architecture.
Use a term selecting neural network (conjunction neuron on $\atomorder$ unary neurons selecting a variable and $\mathrm{Id},\lnot,\mathrm{True}$ as connective selector.
Demand the parameter tensor $\canparam$ to be in a basis CP format, then each slice of the parameter tensor corresponds with the slice of the energy.
The use the approximation for MAP search.
Same construction possible for probability tensors, but often more involved to 
}




\subsection{Transformation of Maximum Search to Risk Minimization}

\red{By the squares risk trick, maximum coordinate searches involving contractions with boolean tensors can be turned into squares risk minimization problems.}
One can use this for MAP inference of MLN and the proposal distribution.


\subsubsection{Squares Risk Trick}

\begin{lemma}
	Let $\gentensor$ be a Boolean tensor, that is $\imageof{\gentensor}\subset\{0,1\}$.
	Then
		\[ \gentensor = \ones - \left( \gentensor - \ones \right)^2  \]
	where $\ones$ is a tensor with same shape as $\gentensor$ and all coordinates being $1$.
\end{lemma}
\begin{proof}
	Since for each $\shortcatindices\in\facstates$ we have $\gentensor[\shortcatvariables=\shortcatindices]\in\{0,1\}$, it holds that
		\[ \gentensor[\shortcatvariables=\shortcatindices] = 1 - (\gentensor[\shortcatvariables=\shortcatindices]-1)^2 \]
	and thus
		\[ \gentensor[\shortcatvariables] = \onesat{\shortcatvariables} - \left( \gentensor - \ones \right)^2  \]
	Applying this to all coordinates shows the claim.
\end{proof}

% OLD?
%Thus, in combination with the slicing theorem \ref{the:CoordinateTransform} we have 
%	\[ \argmax_{\basisslices} \gentensor\basisslices = \argmin_{\basisslices} \left ( \gentensor\basisslices - \ones \right)^2 \, . \]
%\red{ALS reformulation thus only works under the constraint of basis slices, i.e. single active formulas and datapoint contraction.}



\begin{theorem}\label{the:reweightedLeastSquares}
	Let $\Theta$ be a set of binary tensors in $\facspace$ and $\importancetensor\in\facspace$ arbitrary.
	Then we have
	\begin{align}
		\argmax_{\hypercore\in\Theta} \contraction{\importancetensor,\hypercore} 
		= \argmin_{\hypercore\in\Theta} \contraction{\importancetensor, (\hypercoreat{\shortcatvariables}-\onesat{\shortcatvariables})^2}
		%\sum_{\catindices} \left(\theta_{\catindices} -1 \right)^2 \hypercore_{\catindices} \, . 
	\end{align} 
\end{theorem}
\begin{proof}
	Using the Lemma above, $\hypercore$ is identical to $\onesat{\shortcatvariables}-(\hypercoreat{\shortcatvariables}-\onesat{\shortcatvariables})^2$ and we get
	\begin{align*}
		 \contraction{\importancetensor,\hypercore} 
		 &=  \contraction{\importancetensor,\onesat{\shortcatvariables}}-\contraction{\importancetensor,(\hypercoreat{\shortcatvariables}-\onesat{\shortcatvariables})^2} 
	\end{align*}
	Since the first term does not depend on $\hypercore$, it can be dropped in the maximization problem.
	The $(-1)$ factor then turns the maximization into a minimization problem.
\end{proof}

% Interpretation and Importance Tensor
Corollary~\ref{cor:reweightedLeastSquares} reformulates maximation of binary tensors with respect to an angle to another tensor into minimization of a squares risk.
This squares risk trick is especially useful when combining it with a relaxation of $\Theta$ to differentiably parametrizable sets, since then common squares risk solvers can be applied.
We will call $\hypercore$ in the Corollary~\ref{cor:reweightedLeastSquares} importance tensor, since it manipulates the relevance of each coordinate in the squares loss.

\begin{example}[Proposal distribution maxima]
	The Problem~\ref{prob:steepestAscent} of finding the maximal coordinate can thus be turned into
	\begin{align*}
		\argmax_{\shortselindices} \contractionof{(\empdistribution-\currentdistribution),\fselectionmap}{\shortselvariables=\shortselindices}  
		= \argmin_{\shortselindices} \sbcontraction{(\empdistribution-\currentdistribution),
		\left(\contractionof{\fselectionmap,\onehotmapofat{\shortselindices}{\shortselvariables}}{\shortcatvariables}-\onesat{\shortcatvariables}\right)^2} \, . 
	\end{align*}
\end{example}



%\subsubsection{Squares Loss Reformulation of Contracted Binary Tensors}
%
%% Motivation
%We here make use of the fact, that $\rencodingof{\formulaset}$ is a binary tensor, which is also true for all contractions with elementary selection tensors.
%When, like it is the case here, the set $\Gamma$ of hypothesis directions consists of binary tensors, we can apply the squares risk trick (Corollary~\ref{cor:reweightedLeastSquares}) to reformulate Problem~\ref{prob:steepestAscent} to a least squares problem.
%
%\begin{theorem}
%	When $\Gamma$ consists of binary tensors, %that is
%%		\[ \Gamma \subset \facspace \]
%	then %Problem~\ref{prob:steepestAscent} 
%	\begin{align}
%		\argmax_{\theta\in\Gamma} \braket{\theta, \frac{\partial \lossof{\mlntensor}}{\partial \mlntensor}} =
%		\argmin_{\theta\in\Gamma} \sum_{\catindices} \left(\theta_{\catindices} -1 \right)^2  \left(\frac{\partial \lossof{\mlntensor}}{\partial \mlntensor_{\catindices}}\right) \, . 
%	\end{align}
%\end{theorem}
%\begin{proof}
%	Direct application of Corollary~\ref{the:reweightedLeastSquares}.
%\end{proof}


%Since any formula tensor $\ftensorof{\exformula}$ contains only binary coordinates, we can reformulate the loss as:
%\begin{align}
%	\lossof{\mlnparameters} = &
%		\log\partitionfunctionof{\formulasum\weightof{\exformula}\ftensorof{\exformula}} 
%		+ \formulasum\weightof{\exformula}
%		+ \formulasum\weightof{\exformula}\variablesum  \left( 1 - \braket{\ftensorof{\exformula}, \tensordataof{\variableindex}} \right) \\
%		= & 
%		\log\partitionfunctionof{\formulasum\weightof{\exformula}\ftensorof{\exformula}} 
%		+ \formulasum\weightof{\exformula}
%		+ \formulasum\weightof{\exformula}\variablesum  \left( 1 - \braket{\ftensorof{\exformula}, \tensordataof{\variableindex}} \right)^2
%\end{align}


%\begin{remark}{Interpretation of Datapoints}
%	The datapoints are exactly the nonzero entries of the importance core $\importancetensor$.%, i.e. corresponding with individuals in e.g. an \rdf Knowledge Graph.
%\end{remark}







\subsection{Parametercores}

\red{Parameter cores as tradeoff between expressivity and complexity in sampling.
When not hidden variables, their maximum can be found by }


Parametercores need also an efficient representation.

In case of skeleton expressions with many placeholders further decomposition for algorithmic efficiency are required.
\begin{itemize}
	\item Elementary Format ($\elformat$-Format): 
	\item $\cpformat$-Format: Closest to sum of formula tensors (when all vectors are basis, then have a sum).
	\item $\ttformat$-Format: Showed better heuristic performance in optimization
\end{itemize}

For any tensor network decomposition into cores $\parametercoreof{\parenumerator}$ have the derivative $\frac{\partial}{\partial \parametercoreof{\parenumerator}} \parametertensor$ as the tensor network with out the core $\parametercoreof{\parenumerator}$.

\begin{remark}[Parametercores being basis tensors]
	When the parameter core is a basis tensor, the contraction with the parametercore coincides with the respective formula tensor.
	Thus, we will search for basis tensors optimizing in contractions objectives to specific reasoning tasks, and add them iteratively to the network at hand.
\end{remark}

%% Bayesian Neural Network Perspective
When the parametercores are a tensor network representing a probability tensor by there contraction, we can interpret them as a probability distribution over formulas.






\subsection{Alternating Solution of Least Squares Problems}


\begin{remark}{Least Squares Loss by Tensor Fitting}
	\red{Alternative approach to least squares problems: Tensor Fitting}
	And, if the target is another formula $y$, such that $\exformula$ conincides with $\tilde{f} \iff y $ we have
		\[ \left(\polynomialof{\exformula}(\datamap)-1\right)^2 = \left(  \polynomialof{\tilde{f}}(\datamap) - y(\datamap) \right)^2  \]
	This is exactly the least squares loss, which would appear in a supervised interpretation of the learning.


\end{remark}


\begin{figure}[h]
\begin{center}
    \begin{tikzpicture}[scale=0.3] % , baseline = -3.5pt

	\drawformulatensorof{\targettensor}
		
\begin{scope}[shift={(-15,4)}]
	\draw (-5,1) rectangle (-1,-1);
    	\node at (-3,-1.1) [above] {$\gtensorof{\placeholderof{1}}$} ;
	\draw (-3,1) -- (-3,3) node [midway,left] {$\atomlegindexof{1}$};
	\draw (-4,5) rectangle (-2,3);
    	\node at (-3,3.1) [above] {$\varcore{1}$} ;
			
	\node at (0.65,-1.1) [above] {$\cdots$};

	\draw (2,1) rectangle (6,-1);
    	\node at (4,-1.1) [above] {$\gtensorof{\placeholderof{\atomorder}}$} ;
	\draw (4,1) -- (4,3) node [midway,left] {$\atomlegindexof{\atomorder}$};
	\draw (3,5) rectangle (5,3);
    	\node at (4,3.1) [above] {$\varcore{\atomorder}$} ;
	\renewcommand{\skeletoncolor}{}
	\drawskeleton
\end{scope}
	
	\node at (-7.5,0) {{$-$}};
	
\draw (-22,9) -- (-22,-3);
\draw (-22.2,9) -- (-22.2,-3);

\draw (6,9) -- (6,-3);
\draw (6.2,9) node[right] {$2$} -- (6.2,-3) ;
	
\node at (-32,3) [right] {{$\argmin_{\big\{ \varcore{\atomenumerator} : \atomenumeratorin\big\}}$}};
	
\end{tikzpicture}
\end{center}
\caption{\red{Tensor Fitting!} Tensor Network Representation of the optimization.}
\end{figure}


The resulting algorithm is called Alternating Least Squares (ALS) (see Algorithm \ref{alg:ALS}).

\begin{algorithm}[hbt!]
\caption{Alternating Least Squares (ALS)}\label{alg:ALS}
\begin{algorithmic}
\For{$\atomenumeratorin$}
	\State Set $\varcore{\atomenumerator}$ to a random element in $\rr^{\atomlegdimof{\atomenumerator}}$ 
\EndFor
\While{Stopping criterion is not met}
\For{$\atomenumeratorin$}
	\State Set $\varcore{\atomenumerator}$ to a to a solution of $ \argmin_{\varcore{\atomenumerator}}  \left\|  \importancetensor \chadamard ( \groundingof{\parametrization(\varcore{1},\ldots,\varcore{\atomorder})} - \targettensor ) \right\|^2$
\EndFor
\EndWhile
\end{algorithmic}
\end{algorithm}



%\subsection{Projection onto Basis Tensors}
%\red{This is sampling!}
%We project onto basis tensors to achieve single formulas.
