% Text Macros
\newcommand{\python}{$\mathrm{python}$ }
\newcommand{\tnreason}{$\mathrm{tnreason}$ }

\newcommand{\spengine}{$\mathrm{engine}$ }
\newcommand{\sprepresentation}{$\mathrm{representation}$ }
\newcommand{\spreasoning}{$\mathrm{reasoning}$ }
\newcommand{\spapplication}{$\mathrm{application}$ }

\newcommand{\layeronespec}{\textbf{Layer 1}: Storage and manipulations}
\newcommand{\layertwospec}{\textbf{Layer 2}: Specification of workload}
\newcommand{\layerthreespec}{\textbf{Layer 3}: Applications in reasoning}

\newcommand{\curvertnreason}{2.0.0}


% Report Chapters
\newcommand{\partonetext}{Classical Approaches}
\newcommand{\chatextprobRepresentation}{Probability Distributions}
\newcommand{\chatextlogicalRepresentation}{Propositional Logic}
\newcommand{\chatextprobReasoning}{Probabilistic Inference}
\newcommand{\chatextlogicalReasoning}{Logical Inference}

\newcommand{\parttwotext}{Neuro-Symbolic Approaches}
\newcommand{\chatextformulaSelection}{Formula Selecting Networks}
\newcommand{\chatextnetworkRepresentation}{Logical Network Representation}
\newcommand{\chatextnetworkReasoning}{Logical Network Inference}
\newcommand{\chatextconcentration}{Probabilistic Guarantees}
\newcommand{\chatextfolModels}{First Order Logic}

\newcommand{\partthreetext}{Contraction Calculus}
\newcommand{\chatextcoordinateCalculus}{Coordinate Calculus}
\newcommand{\chatextbasisCalculus}{Basis Calculus}
\newcommand{\chatextsparseCalculus}{Sparse Calculus}
\newcommand{\chatextapproximation}{Tensor Approximation}
\newcommand{\chatextmessagePassing}{Message Passing}

\newcommand{\focusonespec}{Focus~I: Representation}
\newcommand{\focustwospec}{Foucs~II: Reasoning}

% Sections in notation chapter and subsection in implementation.notation chapter
\newcommand{\bncategoricals}{Categorical Variables and Representations}
\newcommand{\bntensors}{Tensors}
\newcommand{\bncontractions}{Contractions}
\newcommand{\bnencoding}{Function encoding schemes}


\newcommand{\defref}[1]{Def.~\ref{#1}}
\newcommand{\theref}[1]{The.~\ref{#1}}
\newcommand{\lemref}[1]{Lem.~\ref{#1}}
\newcommand{\corref}[1]{Cor.~\ref{#1}}
\newcommand{\algoref}[1]{Algorithm~\ref{#1}}
\newcommand{\probref}[1]{Problem~\ref{#1}}
\newcommand{\exaref}[1]{Example~\ref{#1}}
\newcommand{\parref}[1]{Part~\ref{#1}}
\newcommand{\charef}[1]{Chapter~\ref{#1}}
\newcommand{\secref}[1]{Sect.~\ref{#1}}
\newcommand{\figref}[1]{Figure~\ref{#1}}
\newcommand{\assref}[1]{Assumption~\ref{#1}}
\newcommand{\remref}[1]{Remark~\ref{#1}}

% Part Intro Texts (unused in scrbook_tnreason)
\newcommand{\partoneintrotext}{
    The computational automation of reasoning is rooted both in the probabilistic and the logical reasoning tradition.
    Both draw on the same ontological commitment that systems have a factored representation, that is their states are described by assignments to a set of variables.
    Based on this commitment both approaches bear a natural tensor representation of their states and a formalism of the respective reasoning algorithms based on multilinear methods.
}

\newcommand{\parttwointrotext}{
    We now employ tensor networks to define architectures and algorithms for neuro-symbolic reasoning based on the logical and probabilistic foundations.
    Markov Logic Networks will be taken as generative models to be learned from data, using formula selecting tensor networks and likelihood optimization algorithms.
}

\newcommand{\partthreeintrotext}{
    Based on the logical interpretation we often handle tensor calculus with specific tensors.
    Often, they are boolean (that is their coordinates are in $\{0,1\}$ corresponding with a Boolean), and sparse (that is having a decomposition with less storage demand).
    We investigate it in this part in more depth the properties of such tensors, which where exploited in the previous parts.
}