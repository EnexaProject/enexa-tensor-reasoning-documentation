\begin{example}[Coin toss experiment]
    Let there be $\catorder$ boolean variables $\shortcatvariables$.
    We have a coin toss family, if the sum is a sufficient statistic.


    For the case $\catorder=2$:
    Consider two coin tosses \(X_1,X_2\in\{0,1\}\) (1=heads). With $p \in [0,1]$ being the probability of heads. Define the statistic
    \[
        S(X_1,X_2)=X_1+X_2\in\{0,1,2\}.
    \]
    Intuitively, \(S\) forgets order and keeps only the \emph{number of heads}. The conditional law of the sequence given \(S\) is uniform over all sequences with that many heads:
    \[
        \mathbb{P}\!\big((X_1,X_2)=(x_1,x_2)\mid S=k\big)=\frac{1}{\binom{2}{k}}
        \quad\text{whenever }x_1+x_2=k.
    \]
    Thus, knowing \(S\) renders the detailed order uninformative about the distribution—this matches the idea of a \emph{sufficient statistic}.

\end{example}