\begin{example}[Continuation of \exaref{exa:hlnAccountingRep}]
    Let us recall the statistic of \exaref{exa:hlnAccountingRep} and consider a dataset of $\datanum=20$ states summarized in the frequency table:
    \begin{center}
        \begin{tabular}{|c|ccc|}
            \hline
            \textbf{Frequency in Dataset} & $\catindexof{A1}$ & $\catindexof{A2}$ & $\catindexof{F}$ \\
            \hline
            0                             & 0                    & 0                    & 0                   \\
            0                             & 0                    & 0                    & 1                   \\
            7                             & 0                    & 1                    & 0                   \\
            2                             & 0                    & 1                    & 1                   \\
            1                             & 1                    & 0                    & 0                   \\
            10                            & 1                    & 0                    & 1                   \\
            0                             & 1                    & 1                    & 0                   \\
            0                             & 1                    & 1                    & 1                   \\
            \hline
        \end{tabular}
    \end{center}
    We then have for the satisfaction rates of $\formulaof{0}=\catvariableof{A1}\oplus\catvariableof{A2}$ and $\formulaof{1}=\catvariableof{F}\Rightarrow\catvariableof{A1}$
    \begin{align*}
        \datameanat{\selvariable=0} = \frac{20}{20} = 1 \andspace
        \datameanat{\selvariable=1} = \frac{7+1+10}{20} = 0.9 \, .
    \end{align*}
    Then \algoref{alg:AMM_HLN} yields with a reasonable convergence criterion choice (such as finite iterations or convergence of $\canparamat{\selvariable}$)
    \begin{align*}
        \hardlegset = \{0\} \quad, \quad \headindexof{\hardlegset} = 1 \andspace
        \canparamat{\selvariable} =
        \begin{bmatrix}
            0 \\
            \lnof{(\frac{0.25}{0.75})\cdot(\frac{0.9}{0.1})}
        \end{bmatrix}
        =
        \begin{bmatrix}
            0 \\
            \lnof{3}
        \end{bmatrix}
        \approx
        \begin{bmatrix}
            0 \\
            1.098612
        \end{bmatrix} \, .
    \end{align*}
\end{example}
