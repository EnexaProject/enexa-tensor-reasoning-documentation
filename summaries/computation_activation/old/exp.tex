\documentclass[standalone]{article}
\usepackage{tikz}
\usepackage{tikz-3dplot}
\usetikzlibrary{calc}

\begin{document}
\tdplotsetmaincoords{70}{110} % Set the viewpoint: (theta, phi)

\begin{tikzpicture}[tdplot_main_coords, scale=2,
    mainline/.style={thick},
    invisibleline/.style={dashed, gray}
]

    % Define the 4 vertices
    \coordinate (A) at (0, 0, 0);
    \coordinate (B) at (2, 0, 0);
    \coordinate (C) at (1, 2, 0);
    \coordinate (D) at (1, 1, 3);

    % Draw the faces with some opacity (optional, requires more complex sorting for correct rendering)
    % \fill[cyan!20, opacity=0.6] (A) -- (B) -- (C) -- cycle;
    % \fill[cyan!20, opacity=0.6] (A) -- (B) -- (D) -- cycle;
    % \fill[cyan!20, opacity=0.6] (A) -- (C) -- (D) -- cycle;
    % \fill[cyan!20, opacity=0.6] (B) -- (C) -- (D) -- cycle;

    % Draw visible edges (adjust based on your chosen viewpoint)
    \draw[mainline] (B) -- (C) -- (D) -- cycle; % Front face BCD
    \draw[mainline] (A) -- (B);
    \draw[mainline] (A) -- (C);
    \draw[mainline] (A) -- (D);

    % Draw the hidden edge (adjust based on your chosen viewpoint)
    % In this view, edge BC is visible, but the edge connecting the back vertex to D is also potentially visible.
    % The edge BC is visible, the edge connecting A to C and B is also visible.
    % A is behind for this viewpoint, so the edges connected to A are back edges
    \draw[invisibleline] (A) -- (B); % Redrawing to ensure dashed style for this view.
    \draw[invisibleline] (A) -- (C);
    \draw[invisibleline] (A) -- (D);


    % Draw vertices as nodes and label them
    \foreach \point/\label/\pos in {A/A/below left, B/B/below right, C/C/above left, D/D/above} {
        \draw[fill=black] (\point) circle (1pt) node[\pos] {$\label$};
    }

    % Add coordinate system axes for context
    \draw[->, gray] (0,0,0) -- (3,0,0) node[right] {$x$};
    \draw[->, gray] (0,0,0) -- (0,3,0) node[above] {$y$};
    \draw[->, gray] (0,0,0) -- (0,0,3) node[above] {$z$};

\end{tikzpicture}
\end{document}
