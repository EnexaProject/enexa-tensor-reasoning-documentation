\begin{example}[Maximum entropy distribution with non-elementary activation cores]

    Consider two atomic variables $\catvariableof{0}$ and $\catvariableof{1}$ and a statistic $\formulaset$ consisting in the formulas
    \begin{align*}
        \formulaof{0} = \left( \catvariableof{0} \land \catvariableof{1} \right) \quad, \quad \formulaof{1} = \left( \catvariableof{0} \Rightarrow \catvariableof{1} \right)
    \end{align*}
    with the coordinatewise expressions
    \begin{align*}
        \formulaof{0} =
        \begin{bmatrix}
            0 & 0 \\
            0 & 1
        \end{bmatrix}
        \quad, \quad
        \formulaof{1} =
        \begin{bmatrix}
            1 & 1 \\
            0 & 1
        \end{bmatrix} \, .
    \end{align*}
    % Interpretation in accounting
    We can think of $\catvariableof{0}$ as a feature on an invoice, and $\catvariableof{1}$ as a feature on the accounting proposal.

    From this we have
    \begin{align*}
        \formulaof{0} \models \formulaof{1}
    \end{align*}
    and therefore $\lnot\formulaof{0}\land\formulaof{1}$ is unsatisfiable.
    The other combinations $\lnot\formulaof{0}\land\lnot\formulaof{1}, \, \formulaof{0}\land\lnot\formulaof{1}$ and $\formulaof{0}\land\formulaof{1}$ are all satisfiable.
    The mean polytope is the convex hull
    \begin{align*}
        \meansetof{(\formulaof{0},\formulaof{1})} =
        \convhullof{\begin{bmatrix}
                        0 \\ 0
        \end{bmatrix},
            \begin{bmatrix}
                0 \\ 1
            \end{bmatrix},
            \begin{bmatrix}
                1 \\ 1
            \end{bmatrix}} \, .
    \end{align*}

    This polytope has a non cube-like face (sketched blue in \figref{fig:nonelHlnstatMaxent}).
    Any mean parameter $\meanparam$ on the interior of that face has a maximum entropy distribution by
    \begin{align*}
        \frac{1}{1+\expof{\canparamat{\selvariable=0}+\canparamat{\selvariable=1}}}
        \begin{bmatrix}
            0 & 0 \\
            1 & \expof{\canparamat{\selvariable=0}+\canparamat{\selvariable=1}} \\
        \end{bmatrix} \, .
    \end{align*}


    \begin{figure}
        \begin{center}
            \begin{tikzpicture}[scale=0.35]
                \node[anchor=east] at (0,0) {$\begin{bmatrix}
                                                  0 \\ 0
                \end{bmatrix}$};
                \node[anchor=west] at (5,0) {$\begin{bmatrix}
                                                  1 \\ 0
                \end{bmatrix}$};
                \node[anchor=west] at (5,5) {$\begin{bmatrix}
                                                  1 \\ 1
                \end{bmatrix}$};
                \node[anchor=east] at (0,5) {$\begin{bmatrix}
                                                  0 \\ 1
                \end{bmatrix}$};

                \drawvectormark{0}{0}
                \drawvectormark{0}{5}
                \drawvectormark{5}{0}
                \drawvectormark{5}{5}

                \draw[thick] (5,5) -- (5,0) -- (0,0);
                \draw[dashed] (0,0) -- (0,5) -- (5,5);
                \draw[\concolor, thick] (0,0) -- (5,5);

                \drawvectormark{3}{3}
                \node[anchor=east] at (3,3) {$\meanparam$};

                \begin{scope}
                    [shift={(20,2)}]
                    \draw[\concolor] (4,3) to[bend right=20] (2,5);
                    \draw[\concolor] (0,3) to[bend left=20] (2,5);
                    \draw[fill,\concolor] (2,5) circle (\dotsize);

                    \draw[\concolor] (-1,1) rectangle (1,3);
                    \node[anchor=center,\concolor] (text) at (0,2) {\corelabelsize $\hardactsymbolof{0}$};

                    \draw[\concolor] (3,1) rectangle (5,3);
                    \node[anchor=center,\concolor] (text) at (4,2) {\corelabelsize $\hardactsymbolof{1}$};

                    \draw[->-] (0,-1)--(0,0);
                    \node[right] (text) at (0,0) {\colorlabelsize $\headvariableof{0}$};
                    \draw[\concolor] (0,0)--(0,1);
                    \drawvariabledot{0}{0}

                    \draw[\probcolor] (0,0) -- (-2,0);
                    \draw[\probcolor] (-2,1) rectangle (-4,-1);
                    \node[anchor=center,\probcolor] (text) at (-3,0) {\corelabelsize $\softactsymbolof{0,\canparam}$};

                    \draw[->-] (4,-1)--(4,0);
                    \node[left] (text) at (4,0) {\colorlabelsize $\headvariableof{1}$};
                    \draw[\concolor] (4,0)--(4,1);
                    \drawvariabledot{4}{0}

                    \draw[\probcolor] (4,0) -- (6,0);
                    \draw[\probcolor] (6,1) rectangle (8,-1);
                    \node[anchor=center,\probcolor] (text) at (7,0) {\corelabelsize $\softactsymbolof{1,\canparam}$};

                    \draw (-1,-1) rectangle (5,-3);
                    \node[anchor=center] (text) at (2,-2) {\corelabelsize $\bencodingof{\hlnstat}$};
                    \draw[-<-] (0,-3)--(0,-5) node[midway,left] {\colorlabelsize $\catvariableof{0}$};

                    \draw[-<-] (4,-3)--(4,-5) node[midway,right] {\colorlabelsize $\catvariableof{1}$};

                \end{scope}

            \end{tikzpicture}
        \end{center}
        \caption{
            Mean polytope of the statistic $\formulaset=(\catvariableof{0} \land \catvariableof{1}, \catvariableof{0} \Rightarrow \catvariableof{1})$ (thick), as a subset of the cube $[0,1]^2$ (dashed).
            The blue line is the face of the polytope, which is not cube like, that is not an intersection of the polytope with the faces of the polytope.
            We further mark a mean parameter $\meanparam = [0.8 \,  0.8]^T$ which is on the interior of that face.}\label{fig:nonelHlnstatMaxent}
    \end{figure}


\end{example}