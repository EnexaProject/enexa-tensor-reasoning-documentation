\begin{example}[Maximum entropy distribution with non-elementary activation cores]

    Consider two atomic variables $\catvariableof{0}$ and $\catvariableof{1}$ and a statistic $\formulaset$ consisting in the formulas
    \begin{align*}
        \formulaof{0} = \left( \catvariableof{0} \land \catvariableof{1} \right) \quad, \quad \formulaof{1} = \left( \catvariableof{0} \Rightarrow \catvariableof{1} \right)
    \end{align*}
    with the coordinatewise expressions
    \begin{align*}
        \formulaof{0} =
        \begin{bmatrix}
            0 & 0 \\
            0 & 1
        \end{bmatrix}
        \quad, \quad
        \formulaof{1} =
        \begin{bmatrix}
            1 & 1 \\
            0 & 1
        \end{bmatrix} \, .
    \end{align*}
    % Interpretation in accounting
    We can think of $\catvariableof{0}$ as a feature on an invoice, and $\catvariableof{1}$ as a feature on the accounting proposal.

    From this we have
    \begin{align*}
        &\bencodingofat{(\formulaof{0},\formulaof{1})}{\headvariableof{0}=0,\headvariableof{1}=0,\catvariableof{0},\catvariableof{1}} =
        \begin{bmatrix}
            0 & 0 \\
            1 & 0
        \end{bmatrix} \quad, \quad
        \bencodingofat{(\formulaof{0},\formulaof{1})}{\headvariableof{0}=0,\headvariableof{1}=1,\catvariableof{0},\catvariableof{1}} =
        \begin{bmatrix}
            1 & 1 \\
            0 & 0
        \end{bmatrix} \quad, \quad \\
        &\bencodingofat{(\formulaof{0},\formulaof{1})}{\headvariableof{0}=1,\headvariableof{1}=0,\catvariableof{0},\catvariableof{1}} =
        \begin{bmatrix}
            0 & 0 \\
            0 & 0
        \end{bmatrix} \quad \text{and} \quad
        \bencodingofat{(\formulaof{0},\formulaof{1})}{\headvariableof{0}=1,\headvariableof{1}=1,\catvariableof{0},\catvariableof{1}} =
        \begin{bmatrix}
            0 & 0 \\
            0 & 1
        \end{bmatrix} \, .
    \end{align*}

    Since the only vanishing slice of $\bencodingof{\formulaset}$ with respect to the head variables is that to $\headindexof{0,1} = (1,0)$, the vertices of the mean polytope are the vectors to the other head indices.
    The mean polytope is the convex hull of these vertices
    %\begin{align*}
    %    \formulaof{0} \models \formulaof{1}
    %\end{align*}
    %and therefore $\lnot\formulaof{0}\land\formulaof{1}$ is unsatisfiable.
    %The other combinations $\lnot\formulaof{0}\land\lnot\formulaof{1}, \, \formulaof{0}\land\lnot\formulaof{1}$ and $\formulaof{0}\land\formulaof{1}$ are all satisfiable.
    %The mean polytope is thus the convex hull
    \begin{align*}
        \meansetof{(\formulaof{0},\formulaof{1})} =
        \convhullof{\begin{bmatrix}
                        0 \\ 0
        \end{bmatrix},
            \begin{bmatrix}
                0 \\ 1
            \end{bmatrix},
            \begin{bmatrix}
                1 \\ 1
            \end{bmatrix}} \, .
    \end{align*}

    This polytope has a non cube-like face (sketched blue in \figref{fig:nonelHlnstatMaxent}), which is the convex hull of the vertices $[0 \, 0]^T, \, [1 \, 1]^T$.
    This face is parametrized by the ($\cpformat$-rank 2) hard activation core
    \begin{align*}
        \kcoreofat{(0,0),(1,1)}{\headvariableof{0},\headvariableof{1}} =
        \onehotmapofat{(0,0)}{\headvariableof{0},\headvariableof{1}} + \onehotmapofat{(1,1)}{\headvariableof{0},\headvariableof{1}} =
        \begin{bmatrix}
            1 & 0 \\
            0 & 1
        \end{bmatrix}
    \end{align*}
    and has the face measure
    \begin{align*}
        \contractionof{\kcoreofat{(0,0),(1,1)}{\headvariableof{0},\headvariableof{1}}
            ,\bencodingofat{\formulaset}{\headvariableof{0},\headvariableof{1},\catvariableof{0},\catvariableof{1}}}{\catvariableof{0},\catvariableof{1}}
        =   \begin{bmatrix}
                0 & 0 \\
                1 & 1
        \end{bmatrix} \, .
    \end{align*}
    Any mean parameter $\meanparam$ on the interior of that face can be parametrized by a scalar $\lambda\in(0,1)$
    \begin{align*}
        \meanparamofat{\lambda}{\selvariable} = \begin{bmatrix}
                                                    \lambda & \lambda
        \end{bmatrix}^T \, .
    \end{align*}
    With the canonical parameters $\canparamat{\selvariable}\in\rr^2$ of the maximum entropy distributions on this face by
    \begin{align*}
        \probat{\catvariableof{0},\catvariableof{1}} =
        \frac{1}{1+\expof{\canparamat{\selvariable=0}+\canparamat{\selvariable=1}}}
        \begin{bmatrix}
            0 & 0                                                               \\
            1 & \expof{\canparamat{\selvariable=0}+\canparamat{\selvariable=1}} \\
        \end{bmatrix} \,
    \end{align*}
    we get the correspondence by the sigmoid
    \begin{align*}
        \lambda = \frac{1}{1+\expof{-(\canparamat{\selvariable=0}+\canparamat{\selvariable=1})}} \, .
    \end{align*}

    Note, that the hard activation core $\kcoreofat{(0,0),(1,1)}{\headvariableof{0},\headvariableof{1}}$ to the blue face is the only non-elementary activation core.
    While the vertices have always elementary cores, the further non-vertex faces have elementary activation cores
    \begin{align*}
        &\kcoreofat{(0,0),(1,0),(1,1)}{\headvariableof{0},\headvariableof{1}}
        = \begin{bmatrix}
              1 & 1 \\
              1 & 1
        \end{bmatrix}
        = \onesat{\headvariableof{0}} \otimes \onesat{\headvariableof{1}}
        \quad, \quad
        \kcoreofat{(0,0),(1,0)}{\headvariableof{0},\headvariableof{1}}
        = \begin{bmatrix}
              1 & 0 \\
              1 & 0
        \end{bmatrix}
        = \onesat{\headvariableof{0}} \otimes \onehotmapofat{0}{\headvariableof{1}}
        \quad, \quad \\
        &\kcoreofat{(1,0),(1,1)}{\headvariableof{0},\headvariableof{1}}
        = \begin{bmatrix}
              1 & 1 \\
              0 & 0
        \end{bmatrix}
        = \onehotmapofat{0}{\headvariableof{0}} \otimes \onesat{\headvariableof{1}}   \, .
    \end{align*}
    The maximum entropy distributions to mean parameters on the interior of all other faces than the blue face are represented by \ComputationActivationNetwork{}s with only elementary activation cores.

    \begin{figure}
        \begin{center}
            \begin{tikzpicture}[scale=0.35]

    \node[anchor=center] at (-4,6) {$a)$};

    \node[anchor=east] at (0,0) {$\begin{bmatrix}
                                      0 \\ 0
    \end{bmatrix}$};
    \node[anchor=west] at (5,0) {$\begin{bmatrix}
                                      1 \\ 0
    \end{bmatrix}$};
    \node[anchor=west] at (5,5) {$\begin{bmatrix}
                                      1 \\ 1
    \end{bmatrix}$};
    \node[anchor=east] at (0,5) {$\begin{bmatrix}
                                      0 \\ 1
    \end{bmatrix}$};

    \drawvectormark{0}{0}
    \drawvectormark{0}{5}
    \drawvectormark{5}{0}
    \drawvectormark{5}{5}

    \draw[thick] (5,5) -- (5,0) -- (0,0);
    \draw[dashed] (0,0) -- (0,5) -- (5,5);
    \draw[\concolor, thick] (0,0) -- (5,5);

    \drawvectormark{3}{3}
    \node[anchor=east] at (3,3) {$\meanparam^{\lambda}$};

    \begin{scope}
        [shift={(20,2)}]

        \node[anchor=center] at (-4,4) {$b)$};

        %\draw[\concolor] (4,3) to[bend right=20] (2,5);
        %\draw[\concolor] (0,3) to[bend left=20] (2,5);
        %\draw[fill,\concolor] (2,5) circle (\dotsize);

        %\draw[\concolor] (-1,1) rectangle (1,3);
        %\node[anchor=center,\concolor] (text) at (0,2) {\corelabelsize $\hardactsymbolof{0}$};

        %\draw[\concolor] (3,1) rectangle (5,3);
        %\node[anchor=center,\concolor] (text) at (4,2) {\corelabelsize $\hardactsymbolof{1}$};

        \draw[\concolor] (-1,1) rectangle (5,4);
        \node[anchor=center,\concolor] (A) at (2,2.5) {\corelabelsize $\begin{bmatrix}
                                        1 & 0 \\
                                        0 & 1
        \end{bmatrix}$};


        \draw[->-] (0,-1)--(0,0);
        \node[right] (text) at (0,0) {\colorlabelsize $\headvariableof{0}$};
        \draw[\concolor] (0,0)--(0,1);
        \drawvariabledot{0}{0}

        \draw[\probcolor] (0,0) -- (-1.5,0);
        \draw[\probcolor] (-1.5,1.5) rectangle (-8.5,-1.5);
        \node[anchor=center,\probcolor] (text) at (-5,0) {\corelabelsize $\begin{bmatrix}
                                                                             1 \\
                                                                             \expof{\canparamat{\selvariable=0}}
        \end{bmatrix}$};
%       \node[anchor=center,\probcolor] (text) at (-3,0) {\corelabelsize $\softactsymbolof{0,\canparam}$};

        \draw[->-] (4,-1)--(4,0);
        \node[left] (text) at (4,0) {\colorlabelsize $\headvariableof{1}$};
        \draw[\concolor] (4,0)--(4,1);
        \drawvariabledot{4}{0}

        \draw[\probcolor] (4,0) -- (5.5,0);
        \draw[\probcolor] (5.5,1.5) rectangle (12.5,-1.5);
        \node[anchor=center,\probcolor] (text) at (9,0) {\corelabelsize $\begin{bmatrix}
                                                                             1 \\
                                                                             \expof{\canparamat{\selvariable=1}}
        \end{bmatrix}$};

%        \node[anchor=center,\probcolor] (text) at (7,0) {\corelabelsize $\softactsymbolof{1,\canparam}$};

        \draw (-1,-1) rectangle (5,-3);
        \node[anchor=center] (text) at (2,-2) {\corelabelsize $\bencodingof{(\formulaof{0},\formulaof{1})}$};
        \draw[-<-] (0,-3)--(0,-5) node[midway,left] {\colorlabelsize $\catvariableof{0}$};

        \draw[-<-] (4,-3)--(4,-5) node[midway,right] {\colorlabelsize $\catvariableof{1}$};

    \end{scope}

\end{tikzpicture}
        \end{center}
        \caption{
            a) Mean polytope of the statistic $\formulaset=(\catvariableof{0} \land \catvariableof{1}, \catvariableof{0} \Rightarrow \catvariableof{1})$ (thick), as a subset of the cube $[0,1]^2$ (dashed).
            The blue line is the face of the polytope, which is not cube like, that is not an intersection of the polytope with the faces of the polytope.
            We further define for $\lambda\in(0,1)$ a mean parameter $\meanparamofat{\lambda}{\selvariable} = [\lambda \,  \lambda]^T$ which is on the interior of the blue face.
            b) Corresponding \ComputationActivationNetwork{} being the maximum entropy distribution reproducing $\meanparamofat{\lambda}{\selvariable}$, when $\lambda$ is the sigmoid of $\canparamat{\selvariable=0}+\canparamat{\selvariable=1}$.
        }\label{fig:nonelHlnstatMaxent}
    \end{figure}


\end{example}