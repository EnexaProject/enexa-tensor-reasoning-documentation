\section{The mean polytope}

The mean polytope is the set of possible mean parameters given the by a base measure representable distributions.
To ease the notation, we use the selection encoding $\sencsstatwith$ of a statistic
\begin{align*}
    \sencsstatat{\shortcatvariables,\indexedselvariable} = \sstatcoordinateofat{\selindex}{\shortcatvariables} \, .
\end{align*}

We define it as
\begin{align*}
    \genmeanset
    = \left\{\contractionof{\probtensor,\sencsstat,\basemeasure}{\selvariable}\wcols\probwith\in\bmrealprobof{\basemeasure} \right\} \, ,
\end{align*}
where we denote by $\bmrealprobof{\basemeasure}$ the set of all probability distributions representable with respect to $\basemeasure$.

\begin{center}
    \begin{tikzpicture}[scale=0.35]
    % Define points
    
    \node[below] at (-1,7) {$\sbinteriorof{\genmeanset}$};
    
    \coordinate (A) at (0,0);
    
    \node[below] at (A) {$\meanparamof{1}$};
    \draw[fill] (A) circle (0.15cm);
    
    \coordinate (B) at (12,2.5);
    \path (A) -- (B) coordinate[pos=0.7] (P1);

    \node[below] at (P1) {$\meanparamof{2}$};
    \draw[fill] (P1) circle (0.15cm);

    \coordinate (P2) at (2,10);
    \node[below] at (P2) {$\meanparamof{3}$};
    \draw[fill] (P2) circle (0.15cm);
    
    \coordinate (C) at (7.5,12);
    \path (B) -- (C) coordinate[pos=0.5] (P4);

    \node[right] at (P4) {$\genfacesetof{\canparam}$};
    \coordinate (D) at (-3,12);
    \coordinate (E) at (-10,5);

    \node[below] at ($0.5*(A)+0.5*(E)-(0,1.3)$) {$\genmeanset/\sbinteriorof{\genmeanset}$};

    \draw[thick] (A) -- (B) -- (C) -- (D) -- (E) -- cycle;

    \node[left] at (-9,11) {$\rr^\seldim$};

%    \coordinate (Or) at (-9,11);
%    \node[below] at (Or) {$\zerosat{\selvariable}$};
%    \drawvariabledot{-9}{11}

%    \draw (Or) -- (Or) + (−9.5,−4.5);

%    % Face normal
%    \draw[thick] (Or) -- ($(Or) + -0.6*(-9.5,-4.5)$) node[midway,below] {$\canparamwith$};
%    \draw[->,dashed] (Or) -- ($(Or) + -1.2*(-9.5,-4.5)$);
%    \draw[dashed] (Or) -- ($(Or) + -1.45*(-9.5,-4.5)$);
%    \draw[dashed] (B) -- ($(B)!1.57!(C)$);
%
%
%    \coordinate (int) at ($(Or) + -1.45*(-9.5,-4.5)$);
%    % angle
%    \draw[thick] ($(Or) + -1.22*(-9.5,-4.5)$) arc[start angle=-154.7, end angle=-64.4, radius=2.5cm];
%%    \coordinate (int) at ($(Or) + -1.45*(-9.5,-4.5)$);
%
%    \draw[fill] ($(Or) + -1.22*(-9.5,-4.5) + (1.75,0)$) circle (0.08cm);

    %\draw pic["", draw=black, angle radius=10, angle eccentricity=1.5, right angle symbol={draw}]{right angle=C--Or--B};
\end{tikzpicture}


\end{center}

\subsection{Convex hull}

\begin{lemma}
    For any statistic $\sstat$ and base measure $\basemeasure$, the set of mean parameters is the convex hull of the set
    \begin{align*}
        \mathcal{N}_{\sstat,\basemeasure} = \{\sencsstatat{\indexedshortcatvariables,\selvariable}\wcols\shortcatindices\in\facstates\ncond\basemeasureat{\indexedshortcatvariables}\neq0 \}
    \end{align*}
    that is $\genmeanset = \convhullof{\mathcal{N}_{\sstat,\basemeasure}}$.
\end{lemma}
\begin{proof}
    This follows from
    \begin{align*}
        \bmrealprobof{\basemeasure}
        = \convhullof{
            \frac{1}{\basemeasureat{\indexedshortcatvariables}} \cdot \onehotmapofat{\shortcatindices}{\shortcatvariables}
            \wcols\shortcatindices\in\facstates\ncond\basemeasureat{\indexedshortcatvariables}\neq0} \, .
    \end{align*}
    and for any vertex of this simplex we have
    \begin{align*}
        \contractionof{\frac{1}{\basemeasureat{\indexedshortcatvariables}} \cdot \onehotmapofat{\shortcatindices}{\shortcatvariables},\sencsstatwith,\basemeasurewith}{\selvariable}
        = \sencsstatat{\indexedshortcatvariables,\selvariable} \, .
    \end{align*}
\end{proof}

Thus the polytope of mean parameters depends on the base measure only through its support.

\subsection{Faces}

Let us now continue with the investigation of the faces of the mean parameter polytope. % which we define analogously to Def.~2.1 in \cite{ziegler_lectures_2013}. % ! Defined in Ziegler directly with normals

\begin{definition}
    We say that the convex hull of a subset $\mathcal{N}^{\facecondset}_{\sstat,\basemeasure}\subset\mathcal{N}_{\sstat,\basemeasure}$ is a face, if and only if there is a face normal vector $\canparamofat{\facecondset}{\selvariable}\in\parspace$ such that
    \begin{align*}
        \mathcal{N}^{\facecondset}_{\sstat,\basemeasure}
        =
        \argmax_{\meanparamwith\in\mathcal{N}_{\sstat,\basemeasure}} \contraction{\meanparamwith,\canparamofat{\facecondset}{\selvariable}} \, .
    \end{align*}
    we denote $\genfacesetof{\facecondset}=\convhullof{\mathcal{N}^{\facecondset}_{\sstat,\basemeasure}}$.
\end{definition}

%\subsection{Characterization of the Boundary by Faces}

%,
%\begin{definition}
%    \label{def:meanPolytopeFaces}
%    Given a mean parameter polytope $\genmeanset$ in the half space representation of \theref{the:meanPolytopeHalfspaces}, and any subset $\mathcal{I}\subset[n]$ we say that the set
%    \begin{align*}
%        \genfacesetof{\facecondset}
%        = \left\{\meanparamwith\in\genmeanset \wcols \forall_{i\in\mathcal{I}} \, \contraction{\meanparamwith,\normalvecofat{i}{\selvariable}}=\normalboundof{i} \right\}
%    \end{align*}
%    is the face to the constraints $\mathcal{I}$.
%\end{definition}
%
%While all inequalities in a half-space representation are satisfied for any element of the polytope, we defined faces by the additional sharp satisfaction of a subset of the half-space inequalities.
%In this way, the faces build the boundary of $\genmeanset$.
%This can be easily verified, since for any vector $\meanparamwith\in\genmeanset$, for which no halfspace inequalities hold sharply, also a neighborhood satisfies the halfspace inequalities.
%If any halfspace inequality holds sharply, in the other case, the vector is a member of the corresponding face.
%
%% Trivial face containing the whole polytope in case of non-minimal statistics
%If $\sstat$ is not minimal with respect to $\basemeasure$, we find a non-vanishing vector $\vectorat{\selvariable}$ and a scalar $\lambda\in\rr$ such that
%\begin{align*}
%    \contractionof{\sencsstatat{\shortcatvariables,\selvariable},\vectorat{\selvariable},\basemeasurewith}{\shortcatvariables} = \lambda\cdot\basemeasurewith \, .
%\end{align*}
%This implies, that any probability distribution $\probwith$ representable with $\basemeasure$ satisfies
%\begin{align*}
%    \contraction{\probwith,\sencsstatat{\shortcatvariables,\selvariable},\vectorat{\selvariable},\basemeasurewith} = \lambda\cdot \contraction{\probwith,\basemeasurewith} = \lambda \, .
%\end{align*}
%Any $\meanparamwith\in\genmeanset$ then satisfies
%\begin{align*}
%    \contraction{\meanparamwith,\vectorat{\selvariable}} = \lambda \, .
%\end{align*}
%Thus, the polytope $\genmeanset$ is contained in an affine linear subspace and has vanishing interior.
%We can further understand this equation as two half-space inequalities
%\begin{align*}
%    \contraction{\meanparamwith,\vectorat{\selvariable}} \leq \lambda \andspace \contraction{\meanparamwith,\vectorat{\selvariable}} \geq \lambda \, ,
%\end{align*}
%which can be integrated into any half-space representation.
%We conclude, that in the case of non-minimal statistics, the whole polytope $\genmeanset$ is a face itself, since it satisfies these half-space inequalities sharply.


%\subsect{Base measures on faces}

%\begin{lemma}
%    \label{lem:faceConvHullPreimage}
%    For each face $\genfacesetof{\facecondset}$ we have
%    \begin{align*}
%        \genfacesetof{\facecondset}
%        = \convhullof{\sencsstatat{\indexedshortcatvariables,\selvariable}\wcols\shortcatindices\in(\sstatencoding)^{-1}(\genfacesetof{\facecondset})\ncond\basemeasureat{\indexedshortcatvariables}=1} \, .
%    \end{align*}
%\end{lemma}
%\begin{proof}
%    This holds, since each face is the convex hull of the contained vertices (see Proposition~2.2 and 2.3 in \cite{ziegler_lectures_2013}).
%    Since the vertices are contained in the image of the statistic encoding $\sstatencoding$, the vertices contained in $\genfacesetof{\facecondset}$ are contained in the set
%    \begin{align*}
%        \sencsstatat{\indexedshortcatvariables,\selvariable}\wcols\shortcatindices\in(\sstatencoding)^{-1}(\genfacesetof{\facecondset}) \, . & \qedhere
%    \end{align*}
%\end{proof}

%\lemref{lem:faceConvHullPreimage} implies in particular, that faces are mean parameter polytopes with respect to refined base measures.
%For reference in later chapters, we define these refined base measures next as face measures.

We notice, that each face itself is a convex polytope.
What is more, we can characterize these as mean parameter polytopes with respect to refined base measures to be defined next.

\begin{definition}
    \label{def:faceMeasure}
    The face measure to the face $\genfacesetof{\facecondset}$ of $\genmeanset$ is the boolean tensor $\basemeasureofat{\sstat,\facecondset}{\shortcatvariables}$ with coordinates to $\shortcatindicesin$ by
    \begin{align*}
        \basemeasureofat{\sstat,\facecondset}{\indexedshortcatvariables}
        = \begin{cases}
              \basemeasureat{\indexedshortcatvariables} & \ifspace \sstatat{\shortcatindices}\in\mathcal{N}^{\facecondset}_{\sstat,\basemeasure} \\
              0 & \text{else}
        \end{cases} \, .
%        = \indicatorofat{\sstatencodingof{\shortcatindices}\in\genfacesetof{\facecondset}}{\shortcatvariables} \, .
    \end{align*}
\end{definition}

We now specify the mean parameter polytope to any face using the face measure as a refinement of the base measure.

\begin{lemma}
    \label{lem:faceAsRefinedPolytope}
%     we have with the refined base measure
%    \begin{align*}
%        \secbasemeasureat{\shortcatvariables} = \contractionof{\basemeasureat{\shortcatvariables},\basemeasureofat{\sstat,\facecondset}{\shortcatvariables}}{\shortcatvariables}
%    \end{align*}
    For any face $\genfacesetof{\facecondset}$ of $\genmeanset$ we have
    \begin{align*}
        \genfacesetof{\facecondset} = \meansetof{\sstat,\basemeasureof{\sstat,\facecondset}} \, .
    \end{align*}
\end{lemma}
\begin{proof}
    For any $\shortcatindicesin$ we have $\basemeasureofat{\sstat,\facecondset}{\indexedshortcatvariables}\neq0$ if and only if $\sstatat{\shortcatindices}\in\mathcal{N}^{\facecondset}_{\sstat,\basemeasure}$ and $\basemeasureat{\indexedshortcatvariables}\neq0$.
    Thus we have
    \begin{align*}
        \genfacesetof{\facecondset}
        &= \convhullof{\mathcal{N}^{\facecondset}_{\sstat,\basemeasure}} \\
        &= \convhullof{\sencsstatat{\indexedshortcatvariables,\selvariable} \wcols \sstatat{\shortcatindices}\in\mathcal{N}^{\facecondset}_{\sstat,\basemeasure}\ncond \basemeasureat{\indexedshortcatvariables}\neq 0} \\
        &= \convhullof{\sencsstatat{\indexedshortcatvariables,\selvariable} \wcols \basemeasureofat{\sstat,\facecondset}{\indexedshortcatvariables}\neq 0} \\
        &= \meansetof{\sstat,\basemeasureof{\sstat,\facecondset}} \, . \qedhere
    \end{align*}
%    \begin{align*}
%        \basemeasureat{\shortcatindices} \neq 0 \land \sstat(\shortcatindices) \in \mathcal{N}^{\facecondset}_{\sstat,\basemeasure}
%        \Leftrightarrow
%        \sstat(\shortcatindices) \in \mathcal{N}^{\facecondset}_{\sstat,\basemeasure}
%    \end{align*}
%
%    We notice that for any $\shortcatindices\in\facstates$, $\shortcatindices\in(\sstatencoding)^{-1}(\genfacesetof{\facecondset})$ is equal to $\basemeasureofat{\sstat,\facecondset}{\indexedshortcatvariables}=1$ and thus
%    \begin{align*}
%        \left\{\shortcatindices\wcols\shortcatindices\in(\sstatencoding)^{-1}(\genfacesetof{\facecondset})\ncond\basemeasureat{\indexedshortcatvariables}=1\right\}
%        = \left\{\shortcatindices\wcols\secbasemeasureat{\indexedshortcatvariables}=1\right\} \, .
%    \end{align*}
%    In combination with \lemref{lem:faceConvHullPreimage} we then get
%    \begin{align*}
%        \genfacesetof{\facecondset}
%        &= \convhullof{\sencsstatat{\indexedshortcatvariables,\selvariable}\wcols\shortcatindices\in(\sstatencoding)^{-1}(\genfacesetof{\facecondset})\ncond\basemeasureat{\indexedshortcatvariables}=1} \\
%        &= \convhullof{\sencsstatat{\indexedshortcatvariables,\selvariable}\wcols\shortcatindices\wcols\secbasemeasureat{\indexedshortcatvariables}=1} \\
%        &= \meansetof{\sstat,\secbasemeasure}
%        \, . \qedhere
%    \end{align*}
\end{proof}

Representability of a distribution with respect to face measures is an equivalent condition for the mean parameter of a distribution to be on a face, as we show next.

\begin{lemma}
    \label{lem:faceMeasureRepCondition} % was \label{the:facePolytopeCharacterization} in the report
    If and only if for a distribution $\probwith\in\bmrealprobof{\basemeasure}$ and a face $\facecondset$ we have
    \begin{align*}
        \contractionof{\sencsstatwith,\probwith,\basemeasurewith}{\selvariable}\in\genfacesetof{\facecondset}\, ,
    \end{align*}
    then $\probwith$ is representable with respect to the base measure
    \begin{align*}
        \genfacemeasurewith \, .
    \end{align*}
\end{lemma}
\begin{proof}
    We have
    \begin{align*}
        \meanparamat{\selvariable} = \sum_{\shortcatindices} \probat{\indexedshortcatvariables}\cdot\basemeasureat{\indexedshortcatvariables} \cdot \genstatshortcatencoding \, .
    \end{align*}
    Let now $\canparamofat{\facecondset}{\selvariable}$ be a face normal to the face $\genfaceset$.
    We then have
    \begin{align*}
        \contraction{\meanparamwith,\canparamofat{\facecondset}{\selvariable}}
        = \sum_{\shortcatindicesin}
        \probat{\indexedshortcatvariables}\cdot\basemeasureat{\indexedshortcatvariables} \cdot \contraction{\genstatshortcatencoding,\canparamofat{\facecondset}{\selvariable}} \, .
    \end{align*}

    Now if and only if $\probat{\indexedshortcatvariables}\cdot\basemeasureat{\indexedshortcatvariables}$ is supported only for $\shortcatindices$ with $\sstatat{\shortcatindices}\in\mathcal{N}^{\facecondset}_{\sstat,\basemeasure}$ we have that
    \begin{align*}
        \contraction{\meanparamwith,\canparamofat{\facecondset}{\selvariable}}
        = \max_{\meanparamwith\in\genmeanset} \contraction{\meanparamwith,\canparamofat{\facecondset}{\selvariable}}
    \end{align*}
    which is equal to $\meanparamwith\in\genfaceset$.
    Thus, if and only if $\meanparamwith\in\genfaceset$ then $\probat{\shortcatvariables}$ is only supported at $\shortcatindices$ in the support of $\genfacemeasurewith$.

%    Now, the $\shortcatindices$ with $\genfacemeasureat{\indexedshortcatvariables}=1$ are exactly those, for which the conditions $\facecondset$ hold straight.
%    If and only if for a $\shortcatindices$ with $\genfacemeasureat{\indexedshortcatvariables}=0$ we have $\probat{\indexedshortcatvariables}>0$, one of the conditions $\facecondset$ would not hold straight.
%    Thus, if and only if $\probwith$ is representable with respect to $\genfacemeasureat{\shortcatvariables}$, we have $\meanparamat{\selvariable}\in\genfacesetof{\facecondset}$.
\end{proof}

%% Stronger for exponential distributions: But need to use smallest face for that! -> Maybe later?
%For members of exponential families, we can make a stronger statement than \lemref{lem:faceMeasureRepCondition}.
%If for any $\probwith\in\expfamilyof{\sstat,\basemeasure}$ and a face $\facecondset$ we have $\contractionof{\probwith,\sencsstatwith}{\selvariable}\in\sbinteriorof{\genfacesetof{\facecondset}}$ then $\probwith$ is positive with respect to the base measure
%\begin{align*}
%    \contractionof{\basemeasurewith,\genfacemeasurewith}{\shortcatvariables} \, .
%\end{align*}

Let us now investigate tensor network representations of face measures, based on the basis encoding $\bencodingof{\sstat}$ of a statistic.
% Vertices
%Vertices of $\genmeanset$ are faces with single elements, that is $\{\meanparamwith\}$.
%By \lemref{lem:faceConvHullPreimage} there must be $\meanparam$ must lie in the image of $\sstatencoding$, since otherwise $\genmeanset$ would be empty.
%The vertex measure is then
%\begin{align*}
%    \basemeasureofat{\sstat,\facecondset}{\shortcatvariables}
%    = \contractionof{\bencodingofat{\sstat}{\headvariables,\shortcatvariables},\onehotmapofat{\meanparam}{\headvariables}}{\shortcatvariables}
%\end{align*}
%Here we use that each $\meanparam\in\genfacesetof{\facecondset}\cap\imageof{\sstatencoding}$ has integer-valued coordinates and denote
%\begin{align*}
%    \onehotmapofat{\meanparam}{\headvariables} = \bigotimes_{\selindexin} \onehotmapofat{\meanparamat{\indexedselvariable}}{\headvariableof{\selindex}} \, .
%\end{align*}

\begin{theorem}[Face measure representation]
    \label{the:faceMeasureCharacterization}
    For any face $\genfacesetof{\facecondset}$ of $\meanset$ we have
    \begin{align*}
        \genfacemeasureat{\shortcatvariables}
        =\contractionof{\sstatccwith,\kcoreofat{\facecondset}{\headvariables},\basemeasurewith}{\shortcatvariables}
    \end{align*}
    where
    \begin{align*}
        \kcoreofat{\facecondset}{\headvariables}
        = \sum_{\meanparam\in\mathcal{N}^{\facecondset}_{\sstat,\basemeasure}} \onehotmapofat{\meanparam}{\headvariables} \, .
    \end{align*}
\end{theorem}
\begin{proof}
    For any $\meanparam\in\mathcal{N}^{\facecondset}_{\sstat,\basemeasure}$ the tensor
    \begin{align*}
        \hypercoreofat{\meanparam}{\shortcatvariables}
        = \contractionof{\sstatccwith,\onehotmapofat{\meanparam}{\headvariables}}{\shortcatvariables}
    \end{align*}
    is the indicator of the preimage of $\meanparam$ under $\sstatencoding$.
    Since preimages the elements in $\mathcal{N}^{\facecondset}_{\sstat,\basemeasure}$ are disjoint, the support of $\hypercoreofat{\meanparam}{\shortcatvariables}$ is disjoint and their sum
    \begin{align*}
        \sum_{\meanparam\in\mathcal{N}^{\facecondset}_{\sstat,\basemeasure}} \hypercoreofat{\meanparam}{\shortcatvariables}
    \end{align*}
    is the indicator of the preimage of $\genfacesetof{\facecondset}$ under $\sstatencoding$.
    The face measure obeys thus
    \begin{align*}
        \basemeasureofat{\sstat,\facecondset}{\shortcatvariables}
        &= \contractionof{\left(
                              \sum_{\meanparam\in\mathcal{N}^{\facecondset}_{\sstat,\basemeasure}} \hypercoreofat{\meanparam}{\shortcatvariables}
        \right), \basemeasurewith}{\shortcatvariables} \\
        &= \sum_{\meanparam\in\mathcal{N}^{\facecondset}_{\sstat,\basemeasure}}
        \contractionof{\sstatccwith,\onehotmapofat{\meanparam}{\headvariables},\basemeasurewith}{\shortcatvariables} \\
        & = \contractionof{\sstatccwith,\kcoreofat{\facecondset}{\headvariables},\basemeasurewith}{\shortcatvariables} \qedhere
    \end{align*}
\end{proof}

%% Going beyond the report

%%  OLD: restriction to CP
%\begin{definition}
%    The $\cpformat$ rank of a face is
%    \begin{align*}
%        \cprankof{\genfaceset}
%        = \min_{
%            \arbset \wcols \arbset\cap\mathcal{N}_{\sstat,\basemeasure} = \mathcal{N}^{\facecondset}_{\sstat,\basemeasure}
%            %\{\shortheadindices \wcols \onehotmapof{\shortheadindices}\in\genfaceset\}\subset \arbset \subset \onehotmap(\bigtimes_{\selindexin}[\headdimof{\selindex}]) \ncond
%            %\arbset \cup \{\shortheadindices \wcols \onehotmapof{\shortheadindices}\in\genmeanset/\genfaceset\} = \varnothing
%        }
%        \cprankof{\sum_{v \in \arbset} \onehotmapofat{v}{\headvariables}} \, .
%    \end{align*}
%    By replacing $\cprankof{\cdot}$ with $\bascprankof{\cdot}$ and $\baspluscprankof{\cdot}$ we further define the basis and basis+ CP rank of a face.
%\end{definition}
%
%
%
%The face measures are contraction of the vertex subset encodings with the computation.
%They are \ComputationActivationNetworks{}, when choosing the $\cpformat$ graph with rank at least $\baspluscprankof{\genfaceset}$.
%
%\begin{lemma}
%    For each face of the mean polytope we have
%    \begin{align*}
%        \genfacemeasureat{\shortcatvariables|\varnothing} \in \realizabledistsof{\sstat,\cpformat^{\cprankof{\genfaceset}}} \, ,
%    \end{align*}
%    where $\cpformat^{\cprankof{\genfaceset}}$ is the CP graph with a hidden variable of dimension $\cprankof{\genfaceset}$.
%\end{lemma}
%\begin{proof}
%    We find by definition a set $\arbset$ of basis vectors containing the vertices of the face $\genfacemeasure$ but no further vertices, which has a bas+ $\cpformat$ rank of $\baspluscprankof{\genfaceset}$.
%    We have therefore, that $\normalizationof{\sstatccwith,\onehotmapofat{\arbset}{\headvariables}}{\shortcatvariables}$ is in $\realizabledistsof{\sstat,\cpformat^{\baspluscprankof{\genfaceset}}}$.
%    Further it holds that
%    \begin{align*}
%        \contractionof{\sstatccwith,\onehotmapofat{\arbset}{\headvariables}}{\shortcatvariables}
%        &= \sum_{\shortheadindices\in\arbset \wcols \onehotmapof{\shortheadindices}\in\genfaceset} \contractionof{\sstatccwith,\onehotmapofat{\shortheadindices}{\headvariables}}{\shortcatvariables} \\
%        & \quad \quad + \sum_{\shortheadindices\in\arbset \wcols \onehotmapof{\shortheadindices}\notin\genfaceset} \contractionof{\sstatccwith,\onehotmapofat{v}{\headvariables}}{\shortcatvariables} \\
%        &= \sum_{\shortheadindices\in\arbset \wcols \onehotmapof{\shortheadindices}\in\genfaceset} \contractionof{\sstatccwith,\onehotmapofat{\shortheadindices}{\headvariables}}{\shortcatvariables} \\
%        &= \genfacemeasureat{\shortcatvariables} \, .
%    \end{align*}
%    Here we used, that for $\shortheadindices\in\arbset$ with $\onehotmapof{\shortheadindices}\notin\genfaceset$ is not in the image of $\sstat$ and therefore the contraction of its one-hot encoding with the computation cores vanishes.
%    Thus, $\normalizationof{\sstatccwith,\onehotmapofat{\arbset}{\headvariables}}{\shortcatvariables}$ coincides with the normalized face measure, which is therefore in $\realizabledistsof{\sstat,\cpformat^{\baspluscprankof{\genfaceset}}}$.
%\end{proof}

We now investigate the representation of face measures by \ComputationActivationNetworks{}.

\begin{definition}
    \label{def:faceRepresentability}
    Let $\graph$ be a hypergraph which nodes include $[\seldim]$.
    We say that a face $\genfaceset$ is representable by $\graph$ if and only if there is a set $\arbset$ of basis vectors with
    \begin{align*}
        \arbset \wcols \arbset\cap\mathcal{N}_{\sstat,\basemeasure} = \mathcal{N}^{\facecondset}_{\sstat,\basemeasure}
    \end{align*}
    and there is a tensor network $\extnet$ with respect to the hypergraph $\graph$ such that
    \begin{align*}
        \contractionof{\extnet}{\headvariables} = \sum_{v\in\arbset} \onehotmapofat{v}{\headvariables} \, .
    \end{align*}
\end{definition}

\begin{lemma}
    If any only if a face is representable by a hypergraph $\graph$ we have
    \begin{align*}
        \genfacemeasureat{\shortcatvariables|\varnothing} \in \realizabledistsof{\sstat,\graph,\basemeasure} \, .
    \end{align*}
\end{lemma}
\begin{proof}
    For any $\arbset$ with $\arbset \wcols \arbset\cap\mathcal{N}_{\sstat,\basemeasure} = \mathcal{N}^{\facecondset}_{\sstat,\basemeasure}$ and tensor network $\extnet$ respecting the assumptions of \defref{def:faceRepresentability} we have
    \begin{align*}
        \contractionof{\contractionof{\extnet}{\headvariables},\bencsstatwith}{\shortcatvariables}
        &= \contractionof{\sum_{v\in\arbset}\onehotmapofat{v}{\headvariables},\bencsstatwith}{\shortcatvariables} \\
        &= \contractionof{\sum_{v\in\mathcal{N}^{\facecondset}_{\sstat,\basemeasure}}\onehotmapofat{v}{\headvariables},\bencsstatwith}{\shortcatvariables} \, .
    \end{align*}
    Here we used in the second equation that only the vertices in $\mathcal{N}_{\sstat,\basemeasure} $ are in the image of $\sstat$.
    It follows, that
    \begin{align*}
        \genfacemeasureat{\shortcatvariables|\varnothing} = \normalizationof{\contractionof{\extnet}{\headvariables},\bencsstatwith,\basemeasurewith}{\shortcatvariables}
    \end{align*}
    and thus $\genfacemeasureat{\shortcatvariables|\varnothing} \in \realizabledistsof{\sstat,\graph,\basemeasure}$.
\end{proof}

Let us now investigate, which normalized face measures can be computed using $\sstat$ and a hypergraph $\graph$.

\begin{example}[Vertices]
    \label{exa:vertexMeasures}
    Vertices $\genfaceset$ are proper faces of affine dimension $0$, that is they consist in single vectors.
    Since all vertices are in the image $\sstatencodingat{\stateset}$, there exists an index tuple $\shortcatindices\in\stateset$ such that $\basemeasureat{\indexedshortcatvariables}=1$ and
    \begin{align*}
        \genfaceset = \{\sencsstatat{\indexedshortcatvariables,\selvariable}\} \, .
    \end{align*}
    Then $\kcoreofat{\facecondset}{\headvariables}$ is the one-hot encoding of the by an interpretation map $\indexinterpretation$ assigned index to $\sencsstatat{\indexedshortcatvariables,\selvariable}$, that is
    \begin{align*}
        \kcoreofat{\facecondset}{\headvariables} = \onehotmapofat{\invindexinterpretationat{\sencsstatat{\indexedshortcatvariables,\selvariable}}}{\headvariables} \, .
    \end{align*}
    In particular, the activation core is elementary and the face measure to any vertex is in $\realizabledistsof{\sstat,\elgraph,\basemeasure}$.
\end{example}

Extending \exaref{exa:vertexMeasures}, we can provide a coarse estimation of the hypergraph $\graph$ required to decompose $\kcoreof{\facecondset}$ for generic faces $\genfaceset$.

\begin{lemma}\label{lem:CPfaceRepresentationBound}
    Any face $\genfaceset$ is representable by a $\cpformat$ graph with hidden rank $r=\cardof{\mathcal{N}^{\facecondset}_{\sstat,\basemeasure}}$.
\end{lemma}
\begin{proof}
    We enumerate the vertices by a variable $\decvariable$ with dimension $r$, i.e. $\mathcal{N}^{\facecondset}_{\sstat,\basemeasure} = \{v^{\decindex}[\selvariable] \wcols \decindexin\}$.
    Then we define core tensors $\hypercoreofat{\selindex}{\headvariableof{\selindex},\decvariable}$
    \begin{align*}
        \hypercoreofat{\selindex}{\headvariableof{\selindex},\indexeddecvariable} = \onehotmapofat{v^{\decindex}[\indexedselvariable]}{\headvariableof{\selindex}} \, .
    \end{align*}
    Then we have 
    \begin{align*}
        \sum_{v\in\mathcal{N}^{\facecondset}_{\sstat,\basemeasure} } \onehotmapofat{v}{\headvariables} 
        = \contractionof{\{\hypercoreofat{\selindex}{\headvariableof{\selindex},\decvariable}\wcols\selindexin\}}{\headvariables}
    \end{align*}
    and thus have found a tensor network representing the face $\genfaceset$.
\end{proof}

Let us notice that in many cases the bound provided by \lemref{lem:CPfaceRepresentationBound} is loose.
By \exaref{exa:maximalFaceMeasure} we provide with the maximal face an example, where the $\cpformat$ rank is given by the tensor space dimension, whereas the generic $\cpformat$ rank is one and the normalized face measure is thus still in $\realizabledistsof{\sstat,\elgraph}$.

\begin{example}[Maximal face]
    \label{exa:maximalFaceMeasure}
    The maximal face $\genfacesetof{\varnothing}=\genmeanset$ coincides with the mean polytope itself and is given by the choice $\canparamofat{\varnothing}{\selvariable}=\zerosat{\selvariable}$.
    In this case the corresponding activation tensor to the face measure is trivial, that is
    \begin{align*}
        \kcoreofat{\varnothing}{\headvariables} = \onesat{\headvariables} \, .
    \end{align*}
    $\kcoreof{\varnothing}$ is elementary and the normalized face measure $\basemeasureof{\sstat,\varnothing}$ to the maximal face is in $\realizabledistsof{\sstat,\elgraph}$.
\end{example}


%\subsection{Partition into relative Interiors of Faces}

Let us now introduce relative interiors, which enables us to find disjoint partitions of the mean polytope.

\begin{definition}[Relative Interior]
    \label{def:relativeInterior}
    Let $\arbset\subset\parspace$ be an arbitrary set and $\mathcal{L}$ be the affine hull of $\arbset$.
    Then the relative interior, denoted $\sbinteriorof{\arbset}$ is the interior of $\arbset$ in the subspace $\mathcal{L}$.
\end{definition}

\begin{lemma}
    \label{lem:relativeInteriorPolytopePartition}
    Any polytope is a disjoint union of the relative interiors of its faces, that is
    \begin{align*}
        \genmeanset = \bigcup_{\facecondset\subset[n]}^{\cdot} \sbinteriorof{\genfaceset} \, .
    \end{align*}
\end{lemma}
\begin{proof}
    For any $\meanparam\in\genmeanset$ we find a face such that $\meanparam\in\genfaceset$.
    If $\meanparam\notin\sbinteriorof{\genfaceset}$, then there is a face $\genfacesetof{\tilde{\facecondset}}\subset\genfaceset$ of smaller affine dimension such that $\meanparam\in\genfacesetof{\facecondset}$.
    When continuing this process we reach a face such that $\meanparam\notin\sbinteriorof{\genfaceset}$, since the faces with affine dimension $0$ are vertices and they coincide with their relative interior because they contain a single vector.
\end{proof}

\begin{definition}
    To each $\meanparam\in\genmeanset$ we denote the unique face $\genfaceset$ with $\meanparam\in\sbinteriorof{\genfaceset}$ by $\facesetofspec{\facecondset(\meanparam)}{\sstat,\basemeasure}$.
\end{definition}

%In this way, we find to each $\meanparam\in\genmeanset$ a unique exponential family with statistics $\sstat$ and base measure by a face measure, such that $\meanparam$ is reproduced by an element of that exponential family.
%We will show in \charef{cha:probReasoning}, that these reproducing distributions maximize the entropy among any other reproducing distribution.

