\section{The mean polytope}

The mean polytope is the set of mean parameters to any distribution.
We define it
\begin{align*}
    \genmeanset
    = \left\{\contractionof{\probtensor,\sencsstat,\basemeasure}{\selvariable}\wcols\probwith\in\bmrealprobof{\basemeasure} \right\} \, ,
\end{align*}
where we denote by $\bmrealprobof{\basemeasure}$ the set of all probability distributions representable with respect to $\basemeasure$.

\begin{center}
    \begin{tikzpicture}[scale=0.35]
    % Define points
    
    \node[below] at (-1,7) {$\sbinteriorof{\genmeanset}$};
    
    \coordinate (A) at (0,0);
    
    \node[below] at (A) {$\meanparamof{1}$};
    \draw[fill] (A) circle (0.15cm);
    
    \coordinate (B) at (12,2.5);
    \path (A) -- (B) coordinate[pos=0.7] (P1);

    \node[below] at (P1) {$\meanparamof{2}$};
    \draw[fill] (P1) circle (0.15cm);

    \coordinate (P2) at (2,10);
    \node[below] at (P2) {$\meanparamof{3}$};
    \draw[fill] (P2) circle (0.15cm);
    
    \coordinate (C) at (7.5,12);
    \path (B) -- (C) coordinate[pos=0.5] (P4);

    \node[right] at (P4) {$\genfacesetof{\canparam}$};
    \coordinate (D) at (-3,12);
    \coordinate (E) at (-10,5);

    \node[below] at ($0.5*(A)+0.5*(E)-(0,1.3)$) {$\genmeanset/\sbinteriorof{\genmeanset}$};

    \draw[thick] (A) -- (B) -- (C) -- (D) -- (E) -- cycle;

    \node[left] at (-9,11) {$\rr^\seldim$};

%    \coordinate (Or) at (-9,11);
%    \node[below] at (Or) {$\zerosat{\selvariable}$};
%    \drawvariabledot{-9}{11}

%    \draw (Or) -- (Or) + (−9.5,−4.5);

%    % Face normal
%    \draw[thick] (Or) -- ($(Or) + -0.6*(-9.5,-4.5)$) node[midway,below] {$\canparamwith$};
%    \draw[->,dashed] (Or) -- ($(Or) + -1.2*(-9.5,-4.5)$);
%    \draw[dashed] (Or) -- ($(Or) + -1.45*(-9.5,-4.5)$);
%    \draw[dashed] (B) -- ($(B)!1.57!(C)$);
%
%
%    \coordinate (int) at ($(Or) + -1.45*(-9.5,-4.5)$);
%    % angle
%    \draw[thick] ($(Or) + -1.22*(-9.5,-4.5)$) arc[start angle=-154.7, end angle=-64.4, radius=2.5cm];
%%    \coordinate (int) at ($(Or) + -1.45*(-9.5,-4.5)$);
%
%    \draw[fill] ($(Or) + -1.22*(-9.5,-4.5) + (1.75,0)$) circle (0.08cm);

    %\draw pic["", draw=black, angle radius=10, angle eccentricity=1.5, right angle symbol={draw}]{right angle=C--Or--B};
\end{tikzpicture}


\end{center}

\subsection{Convex hull}

The mean polytope is the convex hull
\begin{align*}
    \genmeanset
    = \convhullof{\sencsstatat{\indexedshortcatvariables,\selvariable}\wcols\shortcatindices\in\facstates\ncond\basemeasureat{\indexedshortcatvariables}=1} \, .
\end{align*}
It is thus a convex polytope, inherited by the convex polytope of distributions (the standard simplex).
We can characterize the maximum entropy distribution based on the position of the mean parameter in the mean polytope.
To be more precise, any polytope decomposes into effective interiors of its faces and we characterize the maximum entropy distribution depending on the face to the mean parameter.

To be included:
\begin{itemize}
    \item Faces of the mean polytope are itself mean polytopes with respect to refined base measures.
    \item Existence of vectors in the image of the statistic encoding corresponds with satisfiability of the corresponding formula.
\end{itemize}

\subsection{Faces}




%\subsection{Characterization of the Boundary by Faces}

Let us now continue with the investigation of the faces of the mean parameter polytope.%, which we define analogously to Def.~2.1 in \cite{ziegler_lectures_2013}. % ! Defined in Ziegler directly with normals

\begin{definition}
    \label{def:meanPolytopeFaces}
    Given a mean parameter polytope $\genmeanset$ in the half space representation of \theref{the:meanPolytopeHalfspaces}, and any subset $\mathcal{I}\subset[n]$ we say that the set
    \begin{align*}
        \genfacesetof{\facecondset}
        = \left\{\meanparamwith\in\genmeanset \wcols \forall_{i\in\mathcal{I}} \, \contraction{\meanparamwith,\normalvecofat{i}{\selvariable}}=\normalboundof{i} \right\}
    \end{align*}
    is the face to the constraints $\mathcal{I}$.
\end{definition}

While all inequalities in a half-space representation are satisfied for any element of the polytope, we defined faces by the additional sharp satisfaction of a subset of the half-space inequalities.
In this way, the faces build the boundary of $\genmeanset$.
This can be easily verified, since for any vector $\meanparamwith\in\genmeanset$, for which no halfspace inequalities hold sharply, also a neighborhood satisfies the halfspace inequalities.
If any halfspace inequality holds sharply, in the other case, the vector is a member of the corresponding face.

% Trivial face containing the whole polytope in case of non-minimal statistics
If $\sstat$ is not minimal with respect to $\basemeasure$, we find a non-vanishing vector $\vectorat{\selvariable}$ and a scalar $\lambda\in\rr$ such that
\begin{align*}
    \contractionof{\sencsstatat{\shortcatvariables,\selvariable},\vectorat{\selvariable},\basemeasurewith}{\shortcatvariables} = \lambda\cdot\basemeasurewith \, .
\end{align*}
This implies, that any probability distribution $\probwith$ representable with $\basemeasure$ satisfies
\begin{align*}
    \contraction{\probwith,\sencsstatat{\shortcatvariables,\selvariable},\vectorat{\selvariable},\basemeasurewith} = \lambda\cdot \contraction{\probwith,\basemeasurewith} = \lambda \, .
\end{align*}
Any $\meanparamwith\in\genmeanset$ then satisfies
\begin{align*}
    \contraction{\meanparamwith,\vectorat{\selvariable}} = \lambda \, .
\end{align*}
Thus, the polytope $\genmeanset$ is contained in an affine linear subspace and has vanishing interior.
We can further understand this equation as two half-space inequalities
\begin{align*}
    \contraction{\meanparamwith,\vectorat{\selvariable}} \leq \lambda \andspace \contraction{\meanparamwith,\vectorat{\selvariable}} \geq \lambda \, ,
\end{align*}
which can be integrated into any half-space representation.
We conclude, that in the case of non-minimal statistics, the whole polytope $\genmeanset$ is a face itself, since it satisfies these half-space inequalities sharply.


%\subsect{Base measures on faces}

\begin{lemma}
    \label{lem:faceConvHullPreimage}
    For each face $\genfacesetof{\facecondset}$ we have
    \begin{align*}
        \genfacesetof{\facecondset}
        = \convhullof{\sencsstatat{\indexedshortcatvariables,\selvariable}\wcols\shortcatindices\in(\sstatencoding)^{-1}(\genfacesetof{\facecondset})\ncond\basemeasureat{\indexedshortcatvariables}=1} \, .
    \end{align*}
\end{lemma}
\begin{proof}
    This holds, since each face is the convex hull of the contained vertices (see Proposition~2.2 and 2.3 in \cite{ziegler_lectures_2013}).
    Since the vertices are contained in the image of the statistic encoding $\sstatencoding$, the vertices contained in $\genfacesetof{\facecondset}$ are contained in the set
    \begin{align*}
        \sencsstatat{\indexedshortcatvariables,\selvariable}\wcols\shortcatindices\in(\sstatencoding)^{-1}(\genfacesetof{\facecondset}) \, . & \qedhere
    \end{align*}
\end{proof}

\lemref{lem:faceConvHullPreimage} implies in particular, that faces are mean parameter polytopes with respect to refined base measures.
For reference in later chapters, we define these refined base measures next as face measures.

\begin{definition}
    \label{def:faceMeasure}
    The face measure to the face $\genfacesetof{\facecondset}$ of $\genmeanset$ is the boolean tensor $\basemeasureofat{\sstat,\facecondset}{\shortcatvariables}$ with coordinates to $\shortcatindicesin$ by
    \begin{align*}
        \basemeasureofat{\sstat,\facecondset}{\indexedshortcatvariables}
        = \begin{cases}
              1 & \ifspace \sstatencodingof{\shortcatindices}\in\genfacesetof{\facecondset} \\
              0 & \text{else}
        \end{cases} \, .
%        = \indicatorofat{\sstatencodingof{\shortcatindices}\in\genfacesetof{\facecondset}}{\shortcatvariables} \, .
    \end{align*}
\end{definition}

We now specify the mean parameter polytope to any face using the face measure as a refinement of the base measure.

%
\begin{theorem}
    \label{the:faceAsRefinedPolytope}
    For any face $\genfacesetof{\facecondset}$ of $\genmeanset$, we have with the refined base measure
    \begin{align*}
        \secbasemeasureat{\shortcatvariables} = \contractionof{\basemeasureat{\shortcatvariables},\basemeasureofat{\sstat,\facecondset}{\shortcatvariables}}{\shortcatvariables}
    \end{align*}
    that
    \begin{align*}
        \genfacesetof{\facecondset} = \meansetof{\sstat,\secbasemeasure} \, .
    \end{align*}
\end{theorem}
\begin{proof}
    We notice that for any $\shortcatindices\in\facstates$, $\shortcatindices\in(\sstatencoding)^{-1}(\genfacesetof{\facecondset})$ is equal to $\basemeasureofat{\sstat,\facecondset}{\indexedshortcatvariables}=1$ and thus
    \begin{align*}
        \left\{\shortcatindices\wcols\shortcatindices\in(\sstatencoding)^{-1}(\genfacesetof{\facecondset})\ncond\basemeasureat{\indexedshortcatvariables}=1\right\}
        = \left\{\shortcatindices\wcols\secbasemeasureat{\indexedshortcatvariables}=1\right\} \, .
    \end{align*}
    In combination with \lemref{lem:faceConvHullPreimage} we then get
    \begin{align*}
        \genfacesetof{\facecondset}
        &= \convhullof{\sencsstatat{\indexedshortcatvariables,\selvariable}\wcols\shortcatindices\in(\sstatencoding)^{-1}(\genfacesetof{\facecondset})\ncond\basemeasureat{\indexedshortcatvariables}=1} \\
        &= \convhullof{\sencsstatat{\indexedshortcatvariables,\selvariable}\wcols\shortcatindices\wcols\secbasemeasureat{\indexedshortcatvariables}=1} \\
        &= \meansetof{\sstat,\secbasemeasure}
        \, . \qedhere
    \end{align*}
\end{proof}

Positivity of a distribution with respect to face measures is an equivalent condition for the mean parameter of a distribution to be on a face, as we show next.

\begin{theorem}
    \label{the:facePolytopeCharacterization}
    If and only if for a distribution $\probwith$ and a face $\facecondset$ we have
    \begin{align*}
        \contractionof{\probwith,\sencsstatwith}{\selvariable}\in\genfacesetof{\facecondset}\, ,
    \end{align*}
    then $\probwith$ is representable with respect to the base measure
    \begin{align*}
        \contractionof{\basemeasurewith,\genfacemeasurewith}{\shortcatvariables} \, .
    \end{align*}
\end{theorem}
\begin{proof}
    We have
    \begin{align*}
        \meanparamat{\selvariable} = \sum_{\shortcatindices} \probat{\indexedshortcatvariables}\cdot\genstatshortcatencoding \, .
    \end{align*}
    Now, the $\shortcatindices$ with $\genfacemeasureat{\indexedshortcatvariables}=1$ are exactly those, for which the conditions $\facecondset$ hold straight.
    If and only if for a $\shortcatindices$ with $\genfacemeasureat{\indexedshortcatvariables}=0$ we have $\probat{\indexedshortcatvariables}>0$, one of the conditions $\facecondset$ would not hold straight.
    Thus, if and only if $\probwith$ is representable with respect to $\genfacemeasureat{\shortcatvariables}$, we have $\meanparamat{\selvariable}\in\genfacesetof{\facecondset}$.
\end{proof}

% Stronger for exponential distributions: But need to use smallest face for that! -> Maybe later?
For members of exponential families, we can make a stronger statement than \theref{the:facePolytopeCharacterization}.
If for any $\probwith\in\expfamilyof{\sstat,\basemeasure}$ and a face $\facecondset$ we have $\contractionof{\probwith,\sencsstatwith}{\selvariable}\in\sbinteriorof{\genfacesetof{\facecondset}}$ then $\probwith$ is positive with respect to the base measure
\begin{align*}
    \contractionof{\basemeasurewith,\genfacemeasurewith}{\shortcatvariables} \, .
\end{align*}

Let us now investigate tensor network representations of face measures, based on the basis encoding $\bencodingof{\sstat}$ of a statistic.
% Vertices
Vertices of $\genmeanset$ are faces with single elements, that is $\{\meanparamwith\}$.
By \lemref{lem:faceConvHullPreimage} there must be $\meanparam$ must lie in the image of $\sstatencoding$, since otherwise $\genmeanset$ would be empty.
The vertex measure is then
\begin{align*}
    \basemeasureofat{\sstat,\facecondset}{\shortcatvariables}
    = \contractionof{\bencodingofat{\sstat}{\headvariables,\shortcatvariables},\onehotmapofat{\meanparam}{\headvariables}}{\shortcatvariables}
\end{align*}
Here we use that each $\meanparam\in\genfacesetof{\facecondset}\cap\imageof{\sstatencoding}$ has integer-valued coordinates and denote
\begin{align*}
    \onehotmapofat{\meanparam}{\headvariables} = \bigotimes_{\selindexin} \onehotmapofat{\meanparamat{\indexedselvariable}}{\headvariableof{\selindex}} \, .
\end{align*}

\begin{theorem}[Face measure representation]
    \label{the:faceMeasureCharacterization}
    For any face $\genfacesetof{\facecondset}$ of $\meanset$ we have
    \begin{align*}
        \genfacemeasureat{\shortcatvariables}
        =\contractionof{\sstatccwith,\kcoreofat{\facecondset}{\headvariables}}{\shortcatvariables}
    \end{align*}
    where
    \begin{align*}
        \kcoreofat{\facecondset}{\headvariables}
        = \sum_{\meanparam\in\genfacesetof{\facecondset}\cap\imageof{\sstatencoding}} \onehotmapofat{\meanparam}{\headvariables} \, .
    \end{align*}
\end{theorem}
\begin{proof}
    For any $\meanparam\in\genfacesetof{\facecondset}\cap\imageof{\sstatencoding}$ the tensor
    \begin{align*}
        \hypercoreofat{\meanparam}{\shortcatvariables}
        = \contractionof{\sstatccwith,\onehotmapofat{\meanparam}{\headvariables}}{\shortcatvariables}
    \end{align*}
    is the indicator of the preimage of $\meanparam$ under $\sstatencoding$.
    Since preimages the elements in $\genfacesetof{\facecondset}\cap\imageof{\sstatencoding}$ are disjoint, the support of $\hypercoreofat{\meanparam}{\shortcatvariables}$ is disjoint and their sum
    \begin{align*}
        \sum_{\meanparam\in\genfacesetof{\facecondset}\cap\imageof{\sstatencoding}} \hypercoreofat{\meanparam}{\shortcatvariables}
    \end{align*}
    is the indicator of the preimage of $\genfacesetof{\facecondset}$ under $\sstatencoding$, which is the face measure $\basemeasureofat{\sstat,\facecondset}{\shortcatvariables}$.
    Exploiting linearity of contraction we have
    \begin{align*}
        \genfacemeasureat{\shortcatvariables}
        &= \sum_{\meanparam\in\genfacesetof{\facecondset}\cap\imageof{\sstatencoding}} \hypercoreofat{\meanparam}{\shortcatvariables} \\
        &= \contractionof{\bencodingofat{\sstat}{\headvariables,\shortcatvariables},\sum_{\meanparam\in\genfacesetof{\facecondset}\cap\imageof{\sstatencoding}}\onehotmapofat{\meanparam}{\headvariables}}{\shortcatvariables} \\
        &= \contractionof{\bencodingofat{\sstat}{\headvariables,\shortcatvariables},\kcoreofat{\facecondset}{\headvariables}}{\shortcatvariables} \, . \qedhere
    \end{align*}
\end{proof}

%% Going beyond the report

\red{Motivated from the face measure representation, we define a CP rank for faces and show that normalized face measures are computable with respect to a corresponding CP format.}

\begin{definition}
    The bas+ $\cpformat$ rank of a face is
    \begin{align*}
        \baspluscprankof{\genfaceset}
        = \min_{\{\shortheadindices \wcols \onehotmapof{\shortheadindices}\in\genfaceset\}\subset \arbset \subset \onehotmap(\bigtimes_{\selindexin}[\headdimof{\selindex}]) \ncond
        \arbset \cup \{\shortheadindices \wcols \onehotmapof{\shortheadindices}\in\genmeanset/\genfaceset\} = \varnothing
        }
        \baspluscprankof{\sum_{v \in \arbset} \onehotmapofat{v}{\headvariables}} \, .
    \end{align*}
\end{definition}

The face measures are contraction of the vertex subset encodings with the computation.
They are \ComputationActivationNetworks{}, when choosing the $\cpformat$ graph with rank at least $\baspluscprankof{\genfaceset}$.

\begin{lemma}
    For each face of the mean polytope we have
    \begin{align*}
        \genfacemeasureat{\shortcatvariables|\varnothing} \in \realizabledistsof{\sstat,\cpformat^{\baspluscprankof{\genfaceset}}} \, ,
    \end{align*}
    where $\cpformat^{\baspluscprankof{\genfaceset}}$ is the CP graph with a hidden variable of dimension $\baspluscprankof{\genfaceset}$.
\end{lemma}
\begin{proof}
    We find by definition a set $\arbset$ of basis vectors containing the vertices of the face $\genfacemeasure$ but no further vertices, which has a bas+ $\cpformat$ rank of $\baspluscprankof{\genfaceset}$.
    We have therefore, that $\normalizationof{\sstatccwith,\onehotmapofat{\arbset}{\headvariables}}{\shortcatvariables}$ is in $\realizabledistsof{\sstat,\cpformat^{\baspluscprankof{\genfaceset}}}$.
    Further it holds that
    \begin{align*}
        \contractionof{\sstatccwith,\onehotmapofat{\arbset}{\headvariables}}{\shortcatvariables}
        &= \sum_{\shortheadindices\in\arbset \wcols \onehotmapof{\shortheadindices}\in\genfaceset} \contractionof{\sstatccwith,\onehotmapofat{\shortheadindices}{\headvariables}}{\shortcatvariables} \\
        & \quad \quad + \sum_{\shortheadindices\in\arbset \wcols \onehotmapof{\shortheadindices}\notin\genfaceset} \contractionof{\sstatccwith,\onehotmapofat{v}{\headvariables}}{\shortcatvariables} \\
        &= \sum_{\shortheadindices\in\arbset \wcols \onehotmapof{\shortheadindices}\in\genfaceset} \contractionof{\sstatccwith,\onehotmapofat{\shortheadindices}{\headvariables}}{\shortcatvariables} \\
        &= \genfacemeasureat{\shortcatvariables} \, .
    \end{align*}
    Here we used, that for $\shortheadindices\in\arbset$ with $\onehotmapof{\shortheadindices}\notin\genfaceset$ is not in the image of $\sstat$ and therefore the contraction of its one-hot encoding with the computation cores vanishes.
    Thus, $\normalizationof{\sstatccwith,\onehotmapofat{\arbset}{\headvariables}}{\shortcatvariables}$ coincides with the normalized face measure, which is therefore in $\realizabledistsof{\sstat,\cpformat^{\baspluscprankof{\genfaceset}}}$.
\end{proof}


%%%



%
Let us now investigate, which normalized face measures can be computed using $\sstat$ and a hypergraph $\graph$.

\begin{example}[Vertices]
    \label{exa:vertexMeasures}
    Vertices $\genfaceset$ are proper faces of affine dimension $0$, that is they consist in single vectors.
    Since all vertices are in the image $\sstatencodingat{\stateset}$, there exists an index tuple $\shortcatindices\in\stateset$ such that $\basemeasureat{\indexedshortcatvariables}=1$ and
    \begin{align*}
        \genfaceset = \{\sencsstatat{\indexedshortcatvariables,\selvariable}\} \, .
    \end{align*}
    Then $\kcoreofat{\facecondset}{\headvariables}$ is the one-hot encoding of the by an interpretation map $\indexinterpretation$ assigned index to $\sencsstatat{\indexedshortcatvariables,\selvariable}$, that is
    \begin{align*}
        \kcoreofat{\facecondset}{\headvariables} = \onehotmapofat{\invindexinterpretationat{\sencsstatat{\indexedshortcatvariables,\selvariable}}}{\headvariables} \, .
    \end{align*}
    In particular, the activation core is elementary and the face measure to any vertex is in $\realizabledistsof{\sstat,\elgraph,\basemeasure}$.
\end{example}


% basis CP bound
Extending \exaref{exa:vertexMeasures}, we can provide a coarse estimation of the hypergraph $\graph$ required to decompose $\kcoreof{\facecondset}$ for generic faces $\genfaceset$.
We notice that $\kcoreofat{\facecondset}{\headvariables}$ in \theref{the:faceMeasureCharacterization} is a sparse tensor with basis $\cpformat$ rank $\cardof{\genfacesetof{\facecondset}\cap\imageof{\sstatencoding}}$ (see \charef{cha:sparseRepresentation}).
\begin{align*}
    \sparsityof{\kcoreofat{\facecondset}{\headvariables}}
    = \cardof{
        \sstatencodingat{\genfacemeasure}
    }
\end{align*}
where $\cardof{\sstatencodingat{\genfacemeasure}}$ is the number of different statistic encoding vectors to the support of the face measure $\genfacemeasure$.
Using the formalism of sparse $\cpformat$ decompositions \charef{cha:sparseRepresentation}, this characterize the basis $\cpformat$ rank of $\kcoreofat{\facecondset}{\headvariables}$.
However, the basis $\cpformat$ rank is only an upper bound to generic $\cpformat$ rank, which can be loose.
By \exaref{exa:maximalFaceMeasure} we provide with the maximal face an example, where the basis $\cpformat$ rank is given by the tensor space dimension, whereas the generic $\cpformat$ rank is one and the normalized face measure is thus still in $\realizabledistsof{\sstat,\elgraph}$.

\begin{example}[Maximal face]
    \label{exa:maximalFaceMeasure}
    The maximal face $\genfacesetof{\varnothing}=\genmeanset$ coincides with the mean parameter itself.
    In this case the corresponding activation tensor to the face measure is trivial, that is
    \begin{align*}
        \kcoreofat{\varnothing}{\headvariables} = \onesat{\headvariables} \, .
    \end{align*}
    $\kcoreof{\varnothing}$ is elementary and the normalized face measure $\basemeasureof{\sstat,\varnothing}$ to the maximal face is in $\realizabledistsof{\sstat,\elgraph}$.
\end{example}


\subsection{Partition into Effective Interiors of Faces}

Let us now introduce effective interiors, which enables us to find disjoint partitions of the mean polytope.

\begin{definition}[Effective Interior]
    \label{def:effectiveInterior}
    Let $\arbset\subset\parspace$ be an arbitrary set and $\mathcal{L}$ the minimal affine subspace of $\parspace$ containing $\arbset$.
    Then the effective interior, denoted $\sbinteriorof{\arbset}$ is the the interior of $\arbset$ in the space $\mathcal{L}$.
\end{definition}

\begin{lemma}
    \label{lem:effectiveInteriorPolytopePartition}
    Any polytope is a disjoint union of the effective interiors of its faces, that is
    \begin{align*}
        \genmeanset = \bigcup_{\facecondset\subset[n]}^{\cdot} \sbinteriorof{\genfaceset} \, .
    \end{align*}
\end{lemma}
\begin{proof}
    For any $\meanparam\in\genmeanset$ we find a face such that $\meanparam\in\genfaceset$.
    If $\meanparam\notin\sbinteriorof{\genfaceset}$, then there is a face $\genfacesetof{\tilde{\facecondset}}\subset\genfaceset$ of smaller affine dimension such that $\meanparam\in\genfacesetof{\facecondset}$.
    When continuing this process we reach a face such that $\meanparam\notin\sbinteriorof{\genfaceset}$, since the faces with affine dimension $0$ are vertices and they coincide with their effective interior because they contain a single vector.
\end{proof}

In this way, we find to each $\meanparam\in\genmeanset$ a unique exponential family with statistics $\sstat$ and base measure by a face measure, such that $\meanparam$ is reproduced by an element of that exponential family.
We will show in \charef{cha:probReasoning}, that these reproducing distributions maximize the entropy among any other reproducing distribution.

