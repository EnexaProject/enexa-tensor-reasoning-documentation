\section{Main results: Tensor network representation of maximum entropy distributions}

Given the mean polytope discussion we now characterize the tensor network representation of maximum entropy distributions.

\subsection{Main result}

\begin{theorem}[Generic characterization of Maximum Entropy Solutions]
    Let $\sstat$ be a statistic and $\basemeasure$ a base measure.
    For any $\meanparamwith$ the maximum entropy problem has a feasible distribution, if and only if $\meanparamwith\in\genmeanset$.
    In case $\meanparamwith\in\genmeanset$ there is a unique face $\genfaceset$ such that $\meanparam$ is in the effective interior of $\genfaceset$.
    Then the solution of the maximum entropy problem is the member
    \begin{align*}
        \expdistof{(\sstat,\backwardmapwrtof{\sstat,\genfacemeasure}{\meanparam},\genfacemeasure)}
    \end{align*}
    of the exponential family $\expfamilyof{\sstat,\genfacemeasure}$, where $\genfacemeasure$ is the .
    If $\basemeasure$ is an elementary \ComputationActivationNetwork{}, then $\expdistof{(\sstat,\backwardmapwrtof{\sstat,\genfacemeasure}{\meanparam},\genfacemeasure)}$ is a \ComputationActivationNetwork{} with respect to the CP graph of rank $\cprankof{\genfaceset}$.
\end{theorem}
\begin{proof}
    \textbf{Feasibility Claim:}
    If and only if $\meanparamwith\in\genmeanset$ then there is by definition a by $\basemeasure$ representable $\probwith$ reproducing $\meanparamwith$.
    Thus if and only if $\meanparamwith\in\genmeanset$ there is a feasible distribution for the maximum entropy problem. \\

    \textbf{Characterization Claim:}
    We use the following argumentation to show the second claim:
    \begin{itemize}
        \item By \lemref{lem:effectiveInteriorPolytopePartition} for any $\meanparam$ we find a unique face $\genfaceset$.
        \item By \lemref{lem:faceMeasureRepCondition} all feasible distributions are representable by the with the face measure refined base measure.
        The maximum entropy solution is thus the same as for the $(\sstat,\meanparam,\contractionof{\basemeasure,\genfacemeasure}{\shortcatvariables})$ instance, which we characterize in the following.
        \item By \lemref{lem:faceAsRefinedPolytope} the face $\genfaceset$ coincides with the polytope $\meansetof{\sstat,\contractionof{\basemeasure,\genfacemeasure}{\shortcatvariables}}$ and in particular $\meanparam$ is in the effective interior of that polytope.
        \item We can now apply \theref{the:maxEntropyInterior} and get a characterization of the maximum entropy solution as a member of the exponential family.
    \end{itemize}

    \textbf{Representation Claim:}
    By \theref{the:faceMeasureCharacterization} we can represent the face measure as a $\cpformat$ \ComputationActivationNetwork{}.
    Since both the $\basemeasure$ (by assumption) and the soft activation (always) is elementary, they do not change the $\cpformat$ rank when contracting to the face activating tensor of minimal rank.
\end{proof}


\subsection{Maximum entropy on the interior}

A classical result states, that the maximum entropy distribution is in the exponential family $\expfamilyof{\sstat,\basemeasure}$.

\begin{theorem} % TODO: Use the effective interior partition of the mean polytope
    \label{the:maxEntropyInterior}
    %If the only face $\genfacesetof{\facecondset}$ of $\genmeanset$ with $\meanparam\in\genfacesetof{\facecondset}$ is $\genmeanset$ itself
    If and only if $\genmean$ is in the effective interior of $\genmeanset$, then the unique solution of the maximum entropy problem is the distribution
    \begin{align*}
        \expdistofat{\sstat,\genmean,\basemeasure}{\shortcatvariables}\in\expfamilyof{\sstat,\basemeasure}
    \end{align*}
    with $\contractionof{\expdistofat{\sstat,\genmean,\basemeasure}{\shortcatvariables},\sencsstatwith}{\selvariable}=\genmeanat{\selvariable}$.
\end{theorem}
\begin{proof}
    By The~3.3 in [Wainwright and Jordan, 2008], since by assumption
    \begin{align*}
        \meanparamwith \in \sbinteriorof{\genmeanset}  \, ,
    \end{align*}
    there is a canonical parameter $\canparam$ with
    \begin{align*}
        \contractionof{\expdistofat{\sstat,\canparam,\basemeasure}{\shortcatvariables},\sencsstatwith}{\selvariable}=\meanparamat{\selvariable} \, .
    \end{align*}

    For any other feasible distribution $\secprobat{\shortcatvariables}$ we also have $\contractionof{\secprobat{\shortcatvariables},\sencsstatwith}{\selvariable}=\meanparamat{\selvariable}$ and thus
    \begin{align*}
        \centropyof{\secprobtensor}{\expdistof{(\sstat,\canparam,\basemeasure)}}
        &= -\contraction{\secprobtensor,\lnof{\expdistofat{(\sstat,\canparam,\basemeasure)}{\shortcatvariables}}} \\
        &= -\contraction{\secprobtensor,\contractionof{\sencsstatwith,\canparamwith}{\shortcatvariables}} + \cumfunctionof{\canparam} \\
        &= - \contraction{\canparam,\meanparam} + \cumfunctionof{\canparam} \\
        &= \sentropyof{\expdistof{(\sstat,\canparam,\basemeasure)}} \, .
    \end{align*}
    With the Gibbs inequality we have if $\secprobtensor\neq\expdistof{(\sstat,\canparam,\basemeasure)}$
    \begin{align*}
        \sentropyof{\expdistof{(\sstat,\estcanparam,\basemeasure)}} - \sentropyof{\secprobtensor}
        = \centropyof{\secprobtensor}{\expdistof{(\sstat,\estcanparam,\basemeasure)}} - \sentropyof{\secprobtensor} > 0 \, .
    \end{align*}
    Therefore, if $\secprobtensor$ does not coincide with$\expdistof{(\sstat,\estcanparam,\basemeasure)}$, it is not a maximum entropy distribution.
\end{proof}

Exponential families are in $\elrealizabledistsof{\sstat}$, if and only if $\normalizationof{\basemeasure}{\shortcatvariables}\in\elrealizabledistsof{\sstat}$.
If $\normalizationof{\basemeasure}{\shortcatvariables}\in\elrealizabledistsof{\sstat}$ and $\meanparamwith \in \sbinteriorof{\genmeanset} $ we therefore have a sparse representation of the maximum entropy distribution with elementary activation tensors.

\subsection{Mean parameter on faces}

We always find a unique face of the polytope with the mean parameter being in the interior (see \lemref{lem:effectiveInteriorPolytopePartition}).
Any distribution reproducing the mean parameter is realizable with respect to the face measure of that face (see \lemref{}).
We conclude that the maximum entropy distribution of $\genmeanwith$ with respect to $\sstat,\basemeasure$ is also the maximum entropy distribution

\begin{theorem}
    \label{the:tnRepresentationMaxEntropy}
    Given $\sstat$ and $\meanparamwith\in\genmeanset$, let $\facecondset$ be the smallest face of $\genmeanset$ such that
    \begin{align*}
        \meanparamat{\selvariable} \in \genfaceset \, .
    \end{align*}
    Then the corresponding maximum entropy distribution is in $\realizabledistsof{\sstat,\graph,\basemeasure}$ if and only if the face measure (see \defref{def:faceMeasure})
    \begin{align*}
        \kcoreofat{\facecondset}{\headvariables}
        = \sum_{\meanparam\in\genfacesetof{\facecondset}\cap\imageof{\sstatencoding}} \onehotmapofat{\meanparam}{\headvariables}
    \end{align*}
    is in $\tnsetof{\graph}$.
\end{theorem}
\begin{proof}
    By \theref{the:maxEntropyFace} the maximum entropy distribution is an element of the exponential family with by the face measure refined base measure $\secbasemeasure$.
    Let $\canparamwith$ be a canonical parameter such that
    \begin{align*}
        \contractionof{\sencsstatwith,\probofat{\sstat,\canparam,\secbasemeasure}{\shortcatvariables}}{\selvariable} = \meanparamwith \, ,
    \end{align*}
    that is $\probofat{\sstat,\canparam,\secbasemeasure}{\shortcatvariables}$ is the maximum entropy distribution.
    We apply \theref{the:faceMeasureCharacterization} to represent the face measure by
    \begin{align*}
        \basemeasureofat{\sstat,\facecondset}{\shortcatvariables} =
        \contractionof{\bencodingofat{\sstat}{\headvariables,\shortcatvariables},\kcoreofat{\facecondset}{\headvariables}}{\shortcatvariables}
    \end{align*}
    Then for the tensor
    \begin{align*}
        \hypercoreat{\headvariables}
        = \contractionof{
            \{\softactlegwith\wcols\selindexin\}
            \cup\{\kcoreofat{\facecondset}{\headvariables}\}}{\headvariables}
    \end{align*}
    we have
    \begin{align*}
        \probofat{\sstat,\canparam,\secbasemeasure}{\shortcatvariables}
        = \normalizationof{
            \bencodingofat{\sstat}{\headvariables,\shortcatvariables}, \hypercoreat{\headvariables}, \basemeasurewith
        }{\shortcatvariables} \, .
    \end{align*}
    Thus, the maximum entropy distribution is in $\realizabledistsof{\sstat,\graph,\basemeasure}$, if $\hypercore$ admits a tensor network decomposition with respect to $\graph$.
    Since the hard activation cores are elementary, this is the case when $\kcoreof{\facecondset}$ admits a tensor network decomposition with respect to $\graph$.
\end{proof}

\begin{figure}[t]
    \begin{center}
        \begin{tikzpicture}[scale=0.4,thick,xscale=1] % , baseline = -3.5pt

    \begin{scope}[shift={(-15,0)}]
        \draw (-1,-1) rectangle (5,-3);
        \node[anchor=center] (text) at (2,-2) {\small $\probtensor$};
        \draw[->-] (0,-3)--(0,-5) node[midway,left] {\tiny $\catvariableof{0}$};
        \draw[->-] (1.5,-3)--(1.5,-5) node[midway,left] {\tiny $\catvariableof{1}$};
        \node[anchor=center] (text) at (3,-4) {$\cdots$};
        \draw[->-] (4,-3)--(4,-5) node[midway,right] {\tiny $\catvariableof{\atomorder\shortminus1}$};

        \node[anchor=center] (text) at (8,-2) {$= \,\frac{1}{\partitionfunction} \cdot $};
    \end{scope}

    %% Condition cores: Boolean cores selecting faces
    \draw[\concolor] (4,3) to[bend right=20] (2,5);
    \draw[\concolor] (0,3) to[bend left=20] (2,5);
    \draw[fill,\concolor] (2,5) circle (\dotsize);

    \draw[\concolor] (-1,1) rectangle (1,3);
    \node[anchor=center,\concolor] (text) at (0,2) {$\conactcoreof{{0}}$};

    \draw[\concolor] (3,1) rectangle (5,3);
    \node[anchor=center,\concolor] (text) at (4,2) {$\conactcoreof{{\seccatorder\shortminus1}}$};

    \draw[->-] (0,-1)--(0,0);
    \node[right] (text) at (0,0) {\tiny $\headvariableof{0}$};
    \draw[\concolor] (0,0)--(0,1);
    \drawvariabledot{0}{0}
    \node[anchor=center] (text) at (2,0) {$\cdots$};

    \draw[\probcolor] (0,0) -- (-2,0);
    \draw[\probcolor] (-2,1) rectangle (-4,-1);
    \node[anchor=center,\probcolor] (text) at (-3,0) {$\actcoreof{0}$};

    \draw[->-] (4,-1)--(4,0);
    \node[left] (text) at (4,0) {\tiny $\headvariableof{\seccatorder\shortminus1}$};
    \draw[\concolor] (4,0)--(4,1);
    \drawvariabledot{4}{0}

    \draw[\probcolor] (4,0) -- (6,0);
    \draw[\probcolor] (6,1) rectangle (8,-1);
    \node[anchor=center,\probcolor] (text) at (7,0) {$\actcoreof{\seccatorder\shortminus1}$};

    \draw (-1,-1) rectangle (5,-3);
    \node[anchor=center] (text) at (2,-2) {\small $\rencodingof{\sstat}$};
    \draw[-<-] (0,-3)--(0,-5) node[midway,left] {\tiny $\catvariableof{0}$};
    \draw[-<-] (1.5,-3)--(1.5,-5) node[midway,left] {\tiny $\catvariableof{1}$};
    \node[anchor=center] (text) at (3,-4) {$\cdots$};
    \draw[-<-] (4,-3)--(4,-5) node[midway,right] {\tiny $\catvariableof{\atomorder\shortminus1}$};

    \draw (-1,-1) rectangle (5,-3);
    \node[anchor=center] (text) at (2,-2) {\small $\rencodingof{\sstat}$};
    \draw[-<-] (0,-3)--(0,-5) node[midway,left] {\tiny $\catvariableof{0}$};
    \draw[-<-] (1.5,-3)--(1.5,-5) node[midway,left] {\tiny $\catvariableof{1}$};
    \node[anchor=center] (text) at (3,-4) {$\cdots$};
    \draw[-<-] (4,-3)--(4,-5) node[midway,right] {\tiny $\catvariableof{\atomorder\shortminus1}$};

%    \node[anchor=center] (text) at (6,-4.5) {$.$};

%\drawatomcore{3.5}{-8}{$\probtensor$}
%\drawatomindices{3.5}{-12}	
%\draw (5.5,-9)--(5.5,-7) node[midway,right] {\tiny $\catvariableof{\exformula}$};

\end{tikzpicture}
    \end{center}
    \caption{
        Tensor network decomposition of maximum entropy distributions to the constraint $\meanparamat{\selvariable}=\contractionof{\probtensor,\sencsstat}{\selvariable}$.
        Blue: Constraint activation cores $\hardactsymbolof{\selindex}$ in a $\cpformat$ decomposition, representing the face measure to the minimal face, such that $\meanparam\in\genfacesetof{\facecondset}$.
        Red: Probabilistic activation cores $\softactlegwith$ in an elementary decomposition, where each leg core is a scaled exponentials evaluated on the enumerated image $\imageof{\sstatcoordinateof{\selindex}}$.
    }\label{fig:maxEntropyActcore}
\end{figure}