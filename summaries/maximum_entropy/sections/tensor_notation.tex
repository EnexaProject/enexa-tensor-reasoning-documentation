\section{Tensor Notation}


\subsection{\ComputationActivationNetworks{}}

Given a statistic $\sstat:\facstates\rightarrow\selstates$ we build its basis encoding tensor
\begin{align*}
    \sstatccwith = \sum_{\shortcatindicesin} \onehotmapofat{\sstat(\shortcatindices)}{\headvariables} \otimes \onehotmapofat{\shortcatindices}{\shortcatvariables} \, .
\end{align*}
A computation network is any representation of $\sstatccwith$ as a tensor network.
These can be constructed in the case statistics being a composition of connective functions.

An activation tensor is $\hypercoreat{\headvariables}$ and the Computation Activation Network of $\sstat$ and $\hypercore$ the tensor
\begin{align*}
    \probwith = \normalizationof{\sstatccwith,\hypercoreat{\headvariables}}{\shortcatvariables} \, .
\end{align*}

We are interested in decomposition formats of $\hypercoreat{\headvariables}$, where we use sets of tensor networks $\tnsetof{\graph}$ on a hypergraph $\graph$.
The family of by $\sstat$ and a $\graph$ computable distributions are
\begin{align*}
    \realizabledistsof{\sstat,\graph}
    = \left\{ \normalizationof{\bencodingofat{\sstat}{\sstatcatof{[\seldim]},\shortcatvariables},\hypercoreat{\headvariableof{\nodes}}
    }{\shortcatvariables}
          \wcols \hypercoreat{\headvariableof{\nodes}} \in \tnsetof{\graph} \right\} \, .
\end{align*}

\subsection{CP decompositions}

We here introduce the CP decomposition of tensors and the restriction to bas+.
This will be used to represent face measures as computation activation networks.