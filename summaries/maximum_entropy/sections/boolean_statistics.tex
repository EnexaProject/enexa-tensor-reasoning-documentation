\section{Characterization for boolean statistics}

\red{We here study the face CP ranks in case of boolean statistics.
We further show that any elementary \ComputationActivationNetwork{} to boolean statistics is a maximum entropy distribution.}

For boolean statistics $\hlnstat:\facstates\rightarrow\bigtimes_{\selindexin}[2]$ the mean polytope is a subset of the cube $\fullparcube$.
In this case, any boolean vector in $\meansetof{\hlnstat,\basemeasure}$ is a vertex.
It follows, that any distribution reproducing a mean parameter $\meanparamwith$ on the relative interior of $\meansetof{\hlnstat,\basemeasure}$ is positive with respect to $\basemeasure$.

We apply the exponential distribution characterization of the maximum entropy distribution and get that the maximum entropy distribution is in $\elrealizabledistsof{\sstat}$, if and only if the face measure is in $\elrealizabledistsof{\sstat}$.
This is exactly the case, when the face is an intersection of the mean polytope with a face of the cupe $\fullparcube$.

\subsection{Set of maximum entropy distributions}

\begin{example}[Hypercube]
    \label{exa:hypercube}
    In cases where $\shortcatvariables$ are boolean and we have $\selindexin$ features
    \begin{align*}
        \formulaofat{\selindex}{\indexedshortcatvariables} = \catindexof{\selindex}
    \end{align*}
    the mean polytope is the hypercube
    \begin{align*}
        \meansetof{\{\formulaof{\selindex}\wcols\selindexin\},\trivbm} = \fullparcube \, .
    \end{align*}
    Its faces can be enumerated by choosing a subset $\variableset\subset[\seldim]$ and indices $\headindexof{\variableset}\in\bigtimes_{\selindex\in\variableset}[2]$ and represented by the cartesian products
    \begin{align*}
        \facesetofspec{(\variableset,\headindexof{\variableset})}{}%{\{\formulaof{\selindex}\wcols\selindexin\},\trivbm}
        = \bigtimes_{\selindexin} \mathcal{I}_{l}
    \end{align*}
    where
    \begin{align*}
        \mathcal{I}_{l}
        = \begin{cases}
        [0,1]
              & \ifspace \selindex\notin\variableset \\
              \{\headindexof{\selindex}\}& \ifspace \selindex\in\variableset
        \end{cases} \, .
    \end{align*}
    Each of these faces can be represented with respect to the elementary graph $\elgraph$, namely by the tensor product of leg vectors
    \begin{align*}
        \actcoreofat{\selindex}{\headvariableof{\selindex}}
        = \begin{cases}
              \onesat{\headvariableof{\selindex}} & \ifspace \selindex\notin\variableset \\
              \onehotmapofat{\headindexof{\selindex}}{\headvariableof{\selindex}}& \ifspace \selindex\in\variableset
        \end{cases} \, .
    \end{align*}
    \red{This will later be interpreted by propositional logics as the example of atomic formulas.}
\end{example}

\begin{definition}
    We say a polytope $\meansetof{\hlnstat,\basemeasure}$ is cube-like, if for any face $\hlnfaceset$ we find a face of the hypercube parametrized by $(\variableset,\headindexof{\variableset})$ (see \exaref{exa:hypercube}), such that
    \begin{align*}
        \hlnfaceset = \meansetof{\hlnstat,\basemeasure} \cap \facesetofspec{(\variableset,\headindexof{\variableset})}{}\, .%{\{\formulaof{\selindex}\wcols\selindexin\},\trivbm} \, .
    \end{align*}
\end{definition}

\begin{theorem}
    Any distribution in $\realizabledistsof{\hlnstat,\elgraph,\basemeasure}$ is a maximum entropy distribution with respect to $(\hlnstat,\meanparam,\basemeasure)$ where $\meanparam$ is its mean parameter.
    Any maximum entropy distribution is realized by $\realizabledistsof{\hlnstat,\elgraph,\basemeasure}$ if and only if the mean parameter is in the relative interior of a cube-like face.
\end{theorem}
\begin{proof}
    First claim by decomposing any elementary tensor into exponential and hard activation core.
    Second claim by characterization of elementary faces by cube-likeness.
\end{proof}

%The mean parameters, which can be realized by a distribution in $\elrealizabledistsof{\hlnstat}$ are those, which are on the relative interior of the intersection of the mean polytope with a face of the cube.

We can now use the same notation as applied for hypercubes to classify the faces of a cube-like polytope.

\subsection{Interpretation by propositional formulas}

We can understand each feature as a propositonal formula and the variables $\shortcatvariables$ as atoms (possibly after a binarization).

Each vertex of the cube, which is not a vertex of the polytope corresponds with the unsatisfiability of a formula
\begin{align*}
    \bigwedge_{\selindexin} \lnot^{1-\meanparamat{\indexedselvariable}} \formulaofat{\selindex}{\shortcatvariables}
\end{align*}
which is equal with any of the entailment statements for $\variableset\subset[\seldim]$ % Can extend to any partition!
\begin{align*}
    \left(\bigwedge_{\selindex\in\variableset} \lnot^{1-\meanparamat{\indexedselvariable}} \formulaofat{\selindex}{\shortcatvariables} \right)
    \models \left(\bigwedge_{\selindex\in\variableset}\lnot^{\meanparamat{\indexedselvariable}} \formulaofat{\selindex}{\shortcatvariables} \right) \, .
\end{align*}

Along this interpretation we can easily construct examples of statistics, which polytopes are not cube-like.

\begin{example}[Maximum entropy distribution with non-elementary activation cores]

    Consider two atomic variables $\catvariableof{0}$ and $\catvariableof{1}$ and a statistic $\formulaset$ consisting in the formulas
    \begin{align*}
        \formulaof{0} = \left( \catvariableof{0} \land \catvariableof{1} \right) \quad, \quad \formulaof{1} = \left( \catvariableof{0} \Rightarrow \catvariableof{1} \right)
    \end{align*}
    with the coordinatewise expressions
    \begin{align*}
        \formulaof{0} =
        \begin{bmatrix}
            0 & 0 \\
            0 & 1
        \end{bmatrix}
        \quad, \quad
        \formulaof{1} =
        \begin{bmatrix}
            1 & 1 \\
            0 & 1
        \end{bmatrix} \, .
    \end{align*}
    % Interpretation in accounting
    We can think of $\catvariableof{0}$ as a feature on an invoice, and $\catvariableof{1}$ as a feature on the accounting proposal.

    From this we have
    \begin{align*}
        \formulaof{0} \models \formulaof{1}
    \end{align*}
    and therefore $\lnot\formulaof{0}\land\formulaof{1}$ is unsatisfiable.
    The other combinations $\lnot\formulaof{0}\land\lnot\formulaof{1}, \, \formulaof{0}\land\lnot\formulaof{1}$ and $\formulaof{0}\land\formulaof{1}$ are all satisfiable.
    The mean polytope is the convex hull
    \begin{align*}
        \meansetof{(\formulaof{0},\formulaof{1})} =
        \convhullof{\begin{bmatrix}
                        0 \\ 0
        \end{bmatrix},
            \begin{bmatrix}
                0 \\ 1
            \end{bmatrix},
            \begin{bmatrix}
                1 \\ 1
            \end{bmatrix}} \, .
    \end{align*}

    This polytope has a non cube-like face (sketched blue in \figref{fig:nonelHlnstatMaxent}).
    Any mean parameter $\meanparam$ on the interior of that face has a maximum entropy distribution by
    \begin{align*}
        \frac{1}{1+\expof{\canparamat{\selvariable=0}+\canparamat{\selvariable=1}}}
        \begin{bmatrix}
            0 & 0 \\
            1 & \expof{\canparamat{\selvariable=0}+\canparamat{\selvariable=1}} \\
        \end{bmatrix} \, .
    \end{align*}


    \begin{figure}
        \begin{center}
            \begin{tikzpicture}[scale=0.35]
                \node[anchor=east] at (0,0) {$\begin{bmatrix}
                                                  0 \\ 0
                \end{bmatrix}$};
                \node[anchor=west] at (5,0) {$\begin{bmatrix}
                                                  1 \\ 0
                \end{bmatrix}$};
                \node[anchor=west] at (5,5) {$\begin{bmatrix}
                                                  1 \\ 1
                \end{bmatrix}$};
                \node[anchor=east] at (0,5) {$\begin{bmatrix}
                                                  0 \\ 1
                \end{bmatrix}$};

                \drawvectormark{0}{0}
                \drawvectormark{0}{5}
                \drawvectormark{5}{0}
                \drawvectormark{5}{5}

                \draw[thick] (5,5) -- (5,0) -- (0,0);
                \draw[dashed] (0,0) -- (0,5) -- (5,5);
                \draw[\concolor, thick] (0,0) -- (5,5);

                \drawvectormark{3}{3}
                \node[anchor=east] at (3,3) {$\meanparam$};

                \begin{scope}
                    [shift={(20,2)}]
                    \draw[\concolor] (4,3) to[bend right=20] (2,5);
                    \draw[\concolor] (0,3) to[bend left=20] (2,5);
                    \draw[fill,\concolor] (2,5) circle (\dotsize);

                    \draw[\concolor] (-1,1) rectangle (1,3);
                    \node[anchor=center,\concolor] (text) at (0,2) {\corelabelsize $\hardactsymbolof{0}$};

                    \draw[\concolor] (3,1) rectangle (5,3);
                    \node[anchor=center,\concolor] (text) at (4,2) {\corelabelsize $\hardactsymbolof{1}$};

                    \draw[->-] (0,-1)--(0,0);
                    \node[right] (text) at (0,0) {\colorlabelsize $\headvariableof{0}$};
                    \draw[\concolor] (0,0)--(0,1);
                    \drawvariabledot{0}{0}

                    \draw[\probcolor] (0,0) -- (-2,0);
                    \draw[\probcolor] (-2,1) rectangle (-4,-1);
                    \node[anchor=center,\probcolor] (text) at (-3,0) {\corelabelsize $\softactsymbolof{0,\canparam}$};

                    \draw[->-] (4,-1)--(4,0);
                    \node[left] (text) at (4,0) {\colorlabelsize $\headvariableof{1}$};
                    \draw[\concolor] (4,0)--(4,1);
                    \drawvariabledot{4}{0}

                    \draw[\probcolor] (4,0) -- (6,0);
                    \draw[\probcolor] (6,1) rectangle (8,-1);
                    \node[anchor=center,\probcolor] (text) at (7,0) {\corelabelsize $\softactsymbolof{1,\canparam}$};

                    \draw (-1,-1) rectangle (5,-3);
                    \node[anchor=center] (text) at (2,-2) {\corelabelsize $\bencodingof{\hlnstat}$};
                    \draw[-<-] (0,-3)--(0,-5) node[midway,left] {\colorlabelsize $\catvariableof{0}$};

                    \draw[-<-] (4,-3)--(4,-5) node[midway,right] {\colorlabelsize $\catvariableof{1}$};

                \end{scope}

            \end{tikzpicture}
        \end{center}
        \caption{
            Mean polytope of the statistic $\formulaset=(\catvariableof{0} \land \catvariableof{1}, \catvariableof{0} \Rightarrow \catvariableof{1})$ (thick), as a subset of the cube $[0,1]^2$ (dashed).
            The blue line is the face of the polytope, which is not cube like, that is not an intersection of the polytope with the faces of the polytope.
            We further mark a mean parameter $\meanparam = [0.8 \,  0.8]^T$ which is on the interior of that face.}\label{fig:nonelHlnstatMaxent}
    \end{figure}


\end{example}

\input{examples/atomic_formulas.tex}

\input{examples/minterm_formulas.tex}