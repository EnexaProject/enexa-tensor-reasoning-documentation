\chapter{Contraction equations}

We here provide a summary for the application of contractions and normalization in the probabilistic and logical reasoning, which will be introduced in \parref{par:one}.
In \charef{cha:probRepresentation} we introduce:
\begin{itemize}
	\item Marginal probabilities (\defref{def:marginalProbability}, \theref{the:marginalContraction})
		\[ \probat{\exrandom} = \contractionof{\probtensor}{\exrandom} \]
	\item Conditional probabilities (\defref{def:conditionalProbability}, \theref{the:conditionalContraction})
		\[ \condprobof{\exrandom}{\secexrandom} = \normalizationofwrt{\probtensor}{\exrandom}{\secexrandom} \]
	\item The probability distribution of a Markov Network is (\defref{def:markovNetwork})
		\begin{align*}
			\probtensor^{\extnet} = \normalizationof{\extnet}{\nodes}
		\end{align*}
		The partition function of a Markov Networks
		\begin{align*}
			\partitionfunctionof{\extnet} = \contraction{\extnet}
		\end{align*}
		Bayesian Networks (\defref{def:bayesianNetwork}), when hypergraph directed and acyclic, such that the decorating tensors are accordingly directed.
\end{itemize}

Further the following properties are defined by contraction equations:
\begin{itemize}
	\item $\exrandom$ and $\secexrandom$ are independent when (\defref{def:independence}, \theref{the:independenceProductCriterion})
		\[  \contractionof{\probtensor}{\exrandom,\secexrandom}
		=  \contractionof{\probtensor}{\exrandom}
			\otimes  \contractionof{\probtensor}{\secexrandom} \]
	\item $\exrandom$ and $\secexrandom$ are called independent conditioned on $\thirdexrandom$ when (\defref{def:condIndependence}, \theref{the:condIndependenceProductCriterion})
		\[ \normalizationofwrt{\probtensor}{\exrandom,\secexrandom}{\thirdexrandom}
		= \normalizationofwrt{\probtensor}{\exrandom}{\thirdexrandom}
		\otimes \normalizationofwrt{\probtensor}{\secexrandom}{\thirdexrandom} \]
\end{itemize}

% Populate!
In \charef{cha:logicalRepresentation} we introduce:
\begin{itemize}
	\item Propositional formulas by boolean tensors (\defref{def:formulas})
		\[ \formulaat{\shortcatvariables} : \atomstates \rightarrow [2] \subset \rr \, . \]
	\item Syntactical representation of formulas corresponding with tensor networks of boolean tensors (\theref{the:formulaDecomposition})
\end{itemize}