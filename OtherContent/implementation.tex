\chapter{Implementation in the \tnreason package}\label{cha:implementation}

We here document the implementation of the discussed concepts in the \python package \tnreason, in the version \curvertnreason

% Name
\tnreason is an abbreviation of \textbf{t}ensor \textbf{n}etwork \textbf{reason}ing, by which we emphasize the capabilities of this package to represent and answer reasoning tasks by tensor network contractions.

% Installation
The package can be installed either by cloning the repository
\begin{center}
    \href{https://github.com/EnexaProject/enexa-tensor-reasoning}{https://github.com/EnexaProject/enexa-tensor-reasoning}
\end{center}
or by
\begin{lstlisting}
	!pip install tnreason==2.0.0
\end{lstlisting}

\sect{Architecture}

\tnreason is structured in four subpackages and three layers
\begin{itemize}
    \item Layer 1: Storage and numerical manipulations, by subpackage \spengine, "Tensor Networks" -> building "tn" of \tnreason
    \item Layer 2: Specification of workload, subpackage \sprepresentation specific for storage, subpackage \spreasoning specific for manipulations
    \item Layer 3: Applications in reasoning, by subpackage \spapplication, "Reasoning" -> building "reason" of \tnreason
\end{itemize}

We sketch this structure by
\begin{center}
    %! Author = alexgoessmann
%! Date = 09.03.25

\begin{tikzpicture}[scale=0.35]
    \draw[dashed] (-30,15) -- (12,15) -- (12,-3) -- (-30,-3) -- (-30,15);

    \draw (-10,10) rectangle (10,14);
    \node [anchor=center] at (0,12) {\spapplication};

    \node [anchor=center] at (-20,12) {\layerthreespec};
    \draw[dashed] (-30,9) -- (12,9);
    \node [anchor=center] at (-20,6) {\layertwospec};

    \draw[->-] (6,10) -- (6,8);
    \draw (2,4) rectangle (10,8);
    \node [anchor=center] at (6,6) {\spreasoning};
    \draw[->-] (6,4) -- (6,2);

    \draw[->-] (-6,10) -- (-6,8);
    \draw (-10,4) rectangle (-2,8);
    \node [anchor=center] at (-6,6) {\sprepresentation};
    \draw[->-] (-6,4) -- (-6,2);

    \draw[dashed] (-30,3) -- (12,3);
    \node [anchor=center] at (-20,0) {\layeronespec};

    \draw (-10,-2) rectangle (10,2);
    \node [anchor=center] at (0,0) {\spengine};
\end{tikzpicture}
\end{center}


\sect{Implementation of basic notation}

First of all, we explain how the basis notation explained in \charef{cha:notation} is reflected in the implementation.

\subsect{\bncategoricals}
Categorical Variables are identified by strings, which then appear as colors of the corresponding tensor axes.
Their dimension is stored in shapeDicts, but most practically these shapes are stored in the tensors in which variables appear.
Suffixes in the color string (defined in \inlinecode{representation.suffixes}) denote the type of the variable:
\begin{itemize}
    \item Distributed variables with color suffix \disVarSuf: $\catvariableof{\cdot}$
    \item Computed variables with color suffix \comVarSuf: $\headvariableof{\cdot}$
    \item Selection variables with color suffix \selVarSuf: $\selvariableof{\cdot}$
    \item Term variables with color suffix \terVarSuf: $\indvariableof{\cdot}$
\end{itemize}

\subsect{\bntensors}

\paragraph{Tensors} are objects of classes inheriting \inlinecode{engine.TensorCore} with main attributes
\begin{itemize}
    \item \inlinecode{values}: Storing the coordinates of the tensors (individual realization for different cores)
    \item \inlinecode{colors}: List of the variables $[\headvariableof{\formula},\catvariableof{0},\catvariableof{1}]$
    \item \inlinecode{name}: Reflecting the notation such as $\bencodingof{\formula}$
    \item \inlinecode{shape}: Storing the dimension of each appearing variable, as a list of integers with the same length as colors.
\end{itemize}

Suffixes in the name string (defined in \inlinecode{representation.suffixes}) highlight the origin and purpose of the tensor.
Cores are named with suffixes based on their functionality
\begin{itemize}
    \item Computation core with name suffix \comCoreSuf: They represent the computation of a function in basis calculus, and are directed cores.
    Their colors are \inlinecode{[headColors] + [inputColors]}, where \inlinecode{[inputColors]} are either distributed variables or, if having a composition of formulas.
    When the function is a selection augmentation of other functions, selection colors are listed in the end of \inlinecode{[inputColors]}.
    \item Activation core with name suffix \actCoreSuf: two-dimensional vectors representing of the activation core to a formula
\end{itemize}

Both the cores and the colors are further refined by infixes before the suffices to denote specific instantiations.

\begin{itemize}
    \item \selCoreIn: Involving a selection variable
    \item \eviCoreIn: Storing evidence about a variable
    \item \heaIn: Head of a function, typically the variable computed at a activation selector
    \item \funIn: Function selection variables
    \item \posIn+\stringof{i}: Variable selection for argument at position $i$
    \item \datIn: Involving data (data cores and colors)
\end{itemize}

Further infixes are strings denoting atom names and neuron names.

Exploiting efficient representation tricks we further have the tensor name suffices:
\begin{itemize}
    \item \atoCoreSuf: Atomization core, for sparse representation of categorical constraints
    \item \vselCoreSuf: Variable selection core: For sparse representation of variable selectors
\end{itemize}

\paragraph{Initialization}
Tensors are instantiated by
\begin{lstlisting}
	engine.getCore(coreType)(values, colors, name, shape)
\end{lstlisting}
where \inlinecode{coreType} is a string further specifying a specific implementation of tensors (see for more detail \secref{sec:implementationEngine}).
The default tensor implementation \defaultCoreType is chosen, when \inlinecode{coreType} is not specified.


One-hot encodings are specific tensors created in \sprepresentation.

\subsect{\bncontractions}

\paragraph{Tensor networks} $\tnetof{\graph}=\{\hypercoreofat{\edge}{\catvariableof{\edge}}\wcols\edge\in\edges\}$ are stored as dictionaries of tensors, where the keys coincide with the names of the corresponding tensors.
The edges of the hypergraph $\graph$ used in the definition of tensor networks in \defref{def:tensorNetwork} are here labeled by the names of the tensors and the affected variables by the list of colors, being an attribute of each tensor.

\paragraph{Contractions} are implemented in the subpackage \spengine, orienting on \defref{def:contraction}.
Reflected in the notation
\begin{align*}
    \contractionof{\tnetof{\graph}}{\secnodevariables}
\end{align*}
a contraction is defined by
\begin{itemize}
    \item Tensor Network $\tnetof{\graph}$, specified by a dictionary of tensor names as keys and valued by tensor cores.
    \item Open Variables $\secnodes$, specified by a list of colors to the variables.
\end{itemize}
Contraction calls are implemented as
\begin{lstlisting}
	engine.contract(contractionMethod, coreDict, openColors, dimensionDict, evidenceColorDict)
\end{lstlisting}
where the arguments are
\begin{itemize}
    \item \inlinecode{contractionMethod}: str, chooses one of the contraction providers. The default contraction method \defaultContractionMethod is chosen, when
    \item \inlinecode{coreDict}: Dictionary of TensorCores (of the above formats), representing the Tensor Network $\tnetof{\graph}$
    \item \inlinecode{openColors}: List of str, each str identifying a color, that is a variable to be left open in the contraction
    \item \inlinecode{dimensionDict}: Dict valued by int and keys by str, storing dimensions to each variable. This is of optional usage, when a color in openColors does not appear in the coreDict.
    \item \inlinecode{evidenceColorDict}: Dict valued by int and keys by str, indicating sliced variables
\end{itemize}

Coordinates of tensors can be retrieved by
\begin{align*}
    \contractionof{\hypercoreat{\nodevariables}}{\secnodevariables=\catindexof{\secnodes}} \, .
\end{align*}
We implement this by leaving \inlinecode{openColors} empty and passing $\catindexof{\secnodes}$ as the \inlinecode{evidenceColorDict}, as a dictionary with keys by the \inlinecode{str} colors to the variables and values by the corresponding \inlinecode{int} indices.

Graphical illustrations can be generated by
\begin{lstlisting}
	engine.draw_factor_graph(coreDict)
\end{lstlisting}
where \inlinecode{coreDict} is a tensor network to be visualized.


\subsect{\bnencoding}
Encoding schemes are implemented in the subpackage \sprepresentation.



\sect{Subpackage \spengine}\label{sec:implementationEngine}

The \spengine subpackage is for the storage and numerical manipulation of tensors and tensor networks.
We organize the subpackage as the lowest layer of \tnreason, specializing in storage of Tensor Networks and performing the contractions.

\subsect{Basis+ $\cpformat$ Decompositions storing \inlinecode{values}}

\paragraph{Specification of basis+ elementary tensors}
We orient on basis+ sparse tensor decomposition in the initialization of tensor cores, as discussed in detail in \charef{cha:sparseRepresentation}.
The basis+ elementary tensors have basis+ $\cpformat$ rank of $1$ and admit a decomposition as (see \defref{def:polynomialSparsity})
\begin{align*}
    \hypercoreat{\nodevariables}
    = \slicescalar \cdot \contractionof{\onehotmapofat{\catindexof{\variableset}}{\catvariableof{\variableset}}}{\nodevariables} \, .
\end{align*}
Elementary basis+ tensors are in \tnreason stored by a tuple
\begin{lstlisting}
	(value, posDict)
\end{lstlisting}
where \inlinecode{posDict} specifies the values to the variables, which do not have a trivial leg vector, and \inlinecode{value} a scalar scaling the basis vector.
Comparing with the notation of \charef{cha:sparseRepresentation}, the keys of \inlinecode{posDict} correspond with $\variableset$, the values of \inlinecode{posDict} with $\catindexof{\variableset}$ and \inlinecode{value} corresponds with $\slicescalar$.

\paragraph{Elementary Iterators}
The initialization, coordinate retrieval and conversion operations of all tensor cores are oriented on basis+ $\cpformat$ Decompositions \eqref{eq:decIntoMonomials} of tensors.
A tensor
\begin{align*}
    \hypercoreat{\nodevariables}
    = \sum_{\slicetupleof{}\in\sliceset} \slicescalar \cdot \contractionof{\onehotmapofat{\catindexof{\variableset}}{\catvariableof{\variableset}}}{\nodevariables} \, .
\end{align*}
corresponds with an iterator over tuples \inlinecode{(value,posDict)}, each specifying a basis+ elementary tensor in the sum.
%Each tensor core can be iterated, where the iterations are over tuples \inlinecode{(value,posDict)} specifying a basis+ tensor.

\paragraph{Core Arithmetics}
When subscribing an instance \inlinecode{exampleCore} of \inlinecode{engine.TensorCore} by
\begin{lstlisting}
	exampleCore[posDict] = value
\end{lstlisting}
a basis+ elementary tensor specified by \inlinecode{(value,posDict)} is added to its values, that is
\begin{align*}
    \hypercoreat{\shortcatvariables} \algdefsymbol \hypercoreat{\shortcatvariables} + \contractionof{\onehotmapofat{\catindexof{\variableset}}{\catvariableof{\variableset}}}{\shortcatvariables} \, .
\end{align*}
The linear structure of tensors spaces are more further reflected in sums of \inlinecode{engine.TensorCore} instances, which are implemented with the same \inlinecode{coreType}, as
\begin{lstlisting}
	summed = exampleCore1 + exampleCore2
\end{lstlisting}
and scalar multiplication, where a scalar \inlinecode{value} of type \inlinecode{int} or \inlinecode{float}
\begin{lstlisting}
	multiplied = value * exampleCore
\end{lstlisting}
Both operations are performed as manipulations of the tensors \inlinecode{values}.
Contraction of two \inlinecode{engine.TensorCore} instances are performed by
\begin{lstlisting}
	contracted = exampleCore1.contract_with(exampleCore2)
\end{lstlisting}
and are used in corewise contraction, where \inlinecode{contractionMethod="CorewiseContractor"}.

\paragraph{Initialization}
Any instance of \inlinecode{engine.TensorCore} is initialized as a vanishing tensor $\onesat{\nodevariables}$, when \inlinecode{values} is not specified.
The values are then assigned by iteration over a \inlinecode{sliceIterator} over \inlinecode{(value,posDict)} tuples specifying elementary basis+ tensors, where the $\cpformat$ rank is the length of the iterator
%A basis+ $\cpformat$ tensor is specified by an iterator \inlinecode{sliceIterator} over elementary basis+ tensors, where the $\cpformat$ rank is the length of the iterator
This initialization is by applied in the method
\begin{lstlisting}
	engine.create_from_slice_iterator(shape, colors, sliceIterator, coreType, name)
\end{lstlisting}
where \inlinecode{shape, colors, coreType, name} are used in the call of an empty core by \inlinecode{engine.get_core} and \inlinecode{sliceIterator} used to iterative add the basis+ elementary tensors to create the tensor.

\paragraph{Storage of basis+ $\cpformat$ decompositions}
The implemented tensor classes derived from \inlinecode{engine.TensorCore} differ in their implementation of \inlinecode{values}.
Motivated from basis+ $\cpformat$ decompositions, most classes rely on a data base storing the \inlinecode{(value,posDict)} tuples.
An overview over the derived classes is provided in \figref{tab:tensorClasses}.
Here \inlinecode{PandasCore}, \inlinecode{TentrisCore} and \inlinecode{PolynomialCore} support sparse basis+ $\cpformat$ decompositions, by utilizing \inlinecode{pandas.DataFrame}, \inlinecode{tentris.hypertrie} and \inlinecode{list} as storage data base.
These are implementations of the matrix representation of \remref{rem:matStorageBasPlus}.
The \inlinecode{NumpyCore} class on the other hand is based relies on arrays as \inlinecode{numpy.array} as storage solution, which corresponds with the demand that each \inlinecode{posDict} contains all colors of the tensor.
Effectively, this amounts to restricting to basis $\cpformat$ decomposition, which demanded ranks are always larger than basis+ $\cpformat$ decompositions (see \theref{the:rankCascade}).

\begin{figure}
    \begin{center}
        \begin{tabular}{|c|c|c|c|}
            \hline
            \textbf{coreType}           & \textbf{Used Package}                       & \text{Storage of \inlinecode{values}}             & \textbf{Sparse basis+ $\cpformat$ support} \\
            \hline
            \inlinecode{NumpyCore}      & \inlinecode{numpy}                          & \inlinecode{numpy.array}                          & No                                         \\
            \hline
            \inlinecode{PandasCore}     & \inlinecode{pandas}                         & \inlinecode{pandas.DataFrame}                     & Yes                                        \\
            \hline
            \inlinecode{TentrisCore}    & \inlinecode{tentris}  \cite{pan_tentris_2020} & \inlinecode{tentris.hypertrie} & Yes\\
            \hline
            \inlinecode{PolynomialCore} & $--$                                        & \inlinecode{list} of \inlinecode{(value,posDict)} & Yes\\
            \hline
        \end{tabular}
    \end{center}
    \caption{Derived classes from \inlinecode{engine.TensorCore}, differing in the implemented storage of \inlinecode{values}.}\label{tab:tensorClasses}
\end{figure}




\subsect{Contractions}

%\textbf{Binary CP Decomposition}
%
%Based on the monomial decomposition $\slicesparsityof{\cdot}$ as specified in \defref{def:polynomialSparsity}.
%To store the values of a tensor we store the slices of tensors by the indices $\catindexof{\variableset}$.
%
%% Trick -> To BinaryCP
%Contractions can be performed by partially contracting the cores of the decomposition.
%In this way, one can avoids coordinatewise storages of high-order tensors, which can be intractable.


% Contraction Method List
The supported contraction methods are listed in \figref{tab:contractionMethods}.

\begin{figure}
    \begin{center}
        \begin{tabular}{|p{\threecolumnwidth}|p{\threecolumnwidth}|p{\threecolumnwidth}|}
            \hline
            \textbf{contractionMethod} (str)   & \textbf{Package}                            & \textbf{Applied procedure}                                \\
            \hline
            \stringof{NumpyEinsum}             & \inlinecode{numpy}                          & Einstein summation \inlinecode{numpy.einsum}              \\
%        \hline
%        \stringof{TensorFlowEinsum}        & $\mathrm{tensorflow}$ & Einstein summation of $\mathrm{tensorflow}$ tensors         \\
%        \hline
%        \stringof{TorchEinsum}             & $\mathrm{torch}$      & Einstein summation of $\mathrm{torch}$ tensors              \\
            %\hline
            \stringof{TentrisEinsum}           & \inlinecode{tentris}  \cite{pan_tentris_2020} & Einstein summation \inlinecode{tentris.einsum}            \\
            %\hline
            \stringof{PgmpyVariableEliminator} & \inlinecode{pgmpy}                          & Variable Elimination of \inlinecode{pgmpy.DiscreteFactor} \\
            %\hline
            \stringof{CorewiseContractor}      & $--$                                        & Contraction using \inlinecode{core.contract_with()}       \\
            \hline
        \end{tabular}
    \end{center}
    \caption{Implemented contraction methods in \tnreason.}
    \label{tab:contractionMethods}
\end{figure}

\paragraph{Einstein Summation}
is a syntax of specifying the contractions of arrays.
Different possibilities are available to optimize over possible contraction paths, for example in \inlinecode{numpy} by the \inlinecode{numpy.einsum_path}.
%Contractions represented as Einstein summation, as implemented in:
%\begin{itemize}
%	\item numpy
%	\item tensorflow
%	\item pytorch
%	\item tentris
%\end{itemize}

\paragraph{Variable Elimination}
contracts along a junction tree, build by variable elimination.
We here use an implementation in the \inlinecode{pgmpy} package.

\paragraph{Corewise Contraction}
uses the \inlinecode{contract_with} method of tensor cores to contract in a given order.





\sect{Subpackage \sprepresentation}\label{sec:implementationRepresentation}

The \sprepresentation subpackage consists in a collection of core creation methods.
%Here the basis encodings $\bencodingof{\exfunction}$ of various maps $\exfunction$ are created.
We arrange the \sprepresentation subpackage into the second layer of the \tnreason architecture, since it specifies tensor cores which formats are specified in \spengine.

\paragraph{Coordinate Calculus}
Coordinatewise transformations (see \defref{def:coordinatewiseTransform}) are supported by
\begin{lstlisting}
	engine.coordinatewise_transform(coresList, transformFunction)
\end{lstlisting}
where \inlinecode{coresList} is a list of $\seldim$ tensors with identical variables and \inlinecode{transformFunction} a function $\chainingfunction: \rr^{\seldim} \rightarrow \rr$.

\paragraph{Basis Calculus}
Basis encodings (see \defref{def:functionRelationEncoding}) of functions $\exfunction:\facstates \rightarrow \secfacstates$ are created by
\begin{lstlisting}
	engine.create_relational_encoding_from_lambda(inshape, outshape, incolors, outcolors, indicesToIndices)
\end{lstlisting}
where \inlinecode{indicesToIndices} is a lambda-function representing $\exfunction$ and \inlinecode{inshape, outshape, incolors, outcolors} specify the input and output variables.
Let us notice, that this procedure produces sums of $\facdim$ basis tensors corresponding with a basis $\cpformat$ decompositions.
More involved initialization procedures based on basis+ elementary tensors calling \inlinecode{engine.create_from_iterator} might result in sparser representations.

\paragraph{Propositional Connectives} are represented by strings.
\figref{tab:connectives} lists the supported logical connectives, which are implemented in \inlinecode{representation.basisplus_calculus}.
If the \inlinecode{str} to the connective starts with \stringof{n}, then the negated connective is encoded.

\begin{figure}
    \begin{center}
        \begin{tabular}{|p{\fivecolumnwidth}|p{\fivecolumnwidth}|p{\threecolumnwidth}|p{\fivecolumnwidth}|p{\fivecolumnwidth}|}
            \hline
            \textbf{$\exconnective$}          & \textbf{$\synencodingof{\exconnective}$} & \textbf{Notes} & $\baspluscprankof{\exconnective}$ & $\baspluscprankof{\bencodingof{\exconnective}}$ \\
            \hline
            $\land$                           & \stringof{and}                           &                                                                   & 1                                 & 3                                               \\
            %\hline
            $\lor$                            & \stringof{or}                            &                                                                   & 2                                 & 3                                               \\
            %\hline
            $\Rightarrow$                     & \stringof{imp}                           & last variable as head, others premises                            & 2                                 & 3                                               \\
            %\hline
            $\oplus$                          & \stringof{xor}                           & implemented as the negation of \stringof{eq}, i.e. \stringof{neq} & 3 & 5 \\
            %\hline
            $\Leftrightarrow$                 & \stringof{eq}                            &                                                                   & 2                                 & 5                                               \\
            %\hline
            $\catvariableof{\atomenumerator}$ & \stringof{pas} + "$\atomenumerator$"     & $\atomenumerator$th atom & 1 & 2 \\
            %\hline
            $\lnot$                           & \stringof{not}                           & negation of the first argument, i.e. \stringof{npas0}             & 1                                 & 2                                               \\
            \hline
        \end{tabular}
    \end{center}
    \caption{Supported connectives in the script language.
    The arity of all connectives is not restricted.
    We notice, that the basis+ rank $\baspluscprankof{\bencodingof{\exconnective}}$ is independent of the arity and in most cases less than the naive bound of $2^{\catorder}$.
    }\label{tab:connectives}
\end{figure}


% WOLFRAM Numbers

\paragraph{Wolfram codes} provide a classification scheme of propositional formuals by natural numbers, which is supported in the script language.
%They provide a generic representation scheme of propositional formulas by the so-called
The Wolfram code has been designed for the classification of cellular automaton rules \cite{wolfram_statistical_1983} and popularized in the book \cite{wolfram_new_2002}.
Along this, the coordinate encodings of $\catorder$-ary connectives $\exconnective$ are flattened and interpreted as a binary number, which is transformed into a decimal number and represented as a string $\synencodingof{\exconnective}$.
To be more precise, to each $\catorder$-ary connective its Wolfram code is calculated by
\begin{align*}
    N(\exconnective) = \sum_{\catindex\in[2^\catorder]} 2^{2^{\catorder}-\catindex-1} \cdot \exconnective(\catindex) \, .
\end{align*}
In the script language, the connective is represented by the string concatenation of the arity and the Wolfram code as
\begin{align*}
    \synencodingof{\exconnective} = "\catorder" + "_" + "N(\exconnective)" \, .
\end{align*}

\subsect{Computation Activation Networks}

\paragraph{Features} are generation procedures of activation cores given canonical parameters.
They are initialized by
\begin{lstlisting}
    ComputedFeature(featureColors, affectedComputationCores, shape, name)
\end{lstlisting}
where
\begin{itemize}
    \item \inlinecode{featureColors} is a list of \inlinecode{str} variable colors, which are assigned to a created activation core
    \item \inlinecode{affectedComputationCores} is a list of \inlinecode{str} names of computation cores, which are required to compute the feature colors
    \item \inlinecode{shape} is a list of \inlinecode{int} dimensions to the variables
\end{itemize}
Mean parameters to features are computed by
\begin{lstlisting}
    exampleFeature.compute_meanParam(environmentMean)
\end{lstlisting}
where \inlinecode{environmentMean} is the contraction of a tensor network with open colors by the \inlinecode{featureColors}.
They are further capable of computing local changes to canonical parameter in order to match mean parameters by
\begin{lstlisting}
    exampleFeature.local_update(environmentMean, meanParam)
\end{lstlisting}
To customize their purposes, individual feature classes are derived from \inlinecode{representation.ComputedFeature}, as listed in \figref{tab:featureTypes}.

\begin{figure}
    \begin{center}
        \begin{tabular}{|p{\fourcolumnwidth}|p{\fourcolumnwidth}|p{\fourcolumnwidth}|p{\fourcolumnwidth}|}
            \hline
            \textbf{featureType} & \textbf{Purpose} & \textbf{Canonical Parameter} & \textbf{Activation Core} \\
            \hline
            \stringof{SingleSoftFeature} & $\sstatcoordinateof{\selindex}$ of a statistic
            & Scalar (\inlinecode{int} or \inlinecode{float})  & $\expof{\canparam \cdot \indexinterpretationofat{\imageof{\sstatcoordinateof{\selindex}}}{\headvariableof{\selindex}}}$           \\
            \hline
            \stringof{SoftPartitionFeature} & Partition statistics (see \defref{def:partitionStatistic}) $\sstat$ %(Boolean statistic $\sstat$ such that $\contractionof{\sencsstat}{\nodevariables}=\onesat{\nodevariables}$
            & $\canparamat{\selvariable}$  & $\expof{\canparamat{\selvariable}}$   \\
            \hline
            \stringof{HardPartitionFeature} & Partition statistics (see \defref{def:partitionStatistic})
            & Boolean tensor $\canparamat{\selvariable}$  & $\canparamat{\selvariable}$ \\
            \hline
        \end{tabular}
    \end{center}
    \caption{Features derived from \inlinecode{representation.ComputedFeature}, which are implemented in \inlinecode{representation.features}.}
    \label{tab:featureTypes}
\end{figure}

\paragraph{Computation Activation Networks} are the most general models representable in \tnreason.
They generalize distributions such as those computable by a statistic (see \defref{def:realizableStatDistributions}) as well as constraint satisfaction problems (see \defref{def:csp}).
Computation Activation Networks are initialized by
\begin{lstlisting}
	representation.ComputationActivationNetwork(featureDict, computationCoreDict, baseMeasureCoreDict, canParamDict)
\end{lstlisting}
where
\begin{itemize}
    \item \inlinecode{featureDict} is a dictionary of features, where the keys correspond with the names of the features
    \item \inlinecode{computationCoreDict} is a tensor network of computation cores, which needs to contain all names of computation cores required to compute all feature colors
    \item \inlinecode{baseMeasureCoreDict} is a tensor network representing of $\basemeasureat{\nodevariables}$
    \item \inlinecode{canParamDict} is a dictionary of canonical parameters (see \figref{tab:featureTypes})
\end{itemize}
Computation Activation Networks can be instantiated as cores by \inlinecode{exampleCANetwork.create_cores()} or as an energy dictionary by \inlinecode{exampleCANetwork.create_cores()}.
Energy dictionaries are stored as dictionaries with values by tuples \inlinecode{value, tensorNetwork}, representing a by \inlinecode{value} weighted sum of the \inlinecode{tensorNetwork}.

\begin{example}[Representation of a member of an exponential family]\label{exa:expFamilyCARep}
    To represent $\expdistof{\sstat,\canparam,\basemeasure}$ we initialize a
    \begin{itemize}
        \item \inlinecode{featureDict} as a dictionary of \inlinecode{representation.SingleSoftFeature} to each feature $\sstatcoordinateof{\selindex}$
        \item \inlinecode{computationCoreDict} $\extnet$ of computation cores such that
        \begin{align*}
            \extnetat{\headvariables,\nodevariables} = \bencodingofat{\sstat}{\headvariables,\nodevariables} \, .
        \end{align*}
        Decompositions and redundancies between coordinates $\sstatcoordinateof{\selindex}$ can be exploited to find an efficient tensor network.
        \item \inlinecode{baseMeasureCoreDict} $\secextnet$ to represent the base measure $\basemeasure$
        \item \inlinecode{canParamDict} of canonical parameters, valued by the coordinates $\canparamat{\indexedselvariable}$
    \end{itemize}
\end{example}



\sect{Subpackage \spreasoning}\label{sec:implementationReasoning}


The \spreasoning subpackage implements contraction-based reasoning algorithm on representation.ComputationActivationNetworks.
%basic tensor network algorithms with calls of specific execution in \spengine.
As the \sprepresentation subpackage it is arranged in the second layer of the \tnreason architecture, since it specifies the manipulation of tensor networks in the \spengine subpackage.

\subsect{Sampling}

Sampling is performed by MCMC methods calling local sampling methods, which are derived classes from \inlinecode{reasoning.SampleCoreBase}.

The energy-based algorithms execute reasoning tasks solely on energy dictionaries, which are created by \inlinecode{representation.ComputationActivationNetwork.get_energy_dict()}.
\begin{centeredcode}
    reasoning.EnergyBasedGibbs
\end{centeredcode}


\subsect{Variational Inference}
An overview over the variational inference methods in presented in \figref{tab:inferenceMethods}.

\paragraph{Forward mappings} are implemented by \inlinecode{reasoning.ForwardContractor} as contraction of all cores, and in \inlinecode{reasoning.ExpectationPropagator} as a message-passing approach.

\paragraph{Backward mappings} (see \secref{sec:backwardMap}) are implemented by \inlinecode{reasoning.BackwardAlternator} as alternating algorithms iteratively updating the canonical parameters to single features (see \algoref{alg:AMM}).

\paragraph{Mean field methods} are approximation methods of energy tensors by tractable exponential families (see \secref{sec:meanField}).
Given an energy dictionary (e.g. created by \inlinecode{exampleCANetwork.get_energy_dict()}) the naive mean field method is implemented as
\begin{centeredcode}
    reasoning.NaiveMeanField(energyDict)
\end{centeredcode}
and the more general markov network based mean field method by
\begin{centeredcode}
    reasoning.GenericMeanField(energyDict, edgeColorDict)
\end{centeredcode}
where \inlinecode{edgeColorDict} specifies the graph of the approximating markov network.


\begin{figure}
    \begin{center}
        \begin{tabular}{|p{\threecolumnwidth}|p{\threecolumnwidth}|p{\threecolumnwidth}|}
            \hline
            \textbf{inferenceMethod} (str)   & \textbf{Applied procedure}                               & \textbf{Dependency}                                               \\
            \hline
            \stringof{ForwardContractor}     & Contraction of all cores keeping the feature colors open  & $--$\\
            %\hline
            \stringof{BackwardAlternator}    & Iterative local updates to match the mean parameters & Forward inferer to iteratively update the mean parameters \\
            %\hline
            \stringof{ExpectationPropagator} & Iterative updates of messages between feature clusters & Forward and backward inferer used for the computation of messages \\
            \hline
        \end{tabular}
    \end{center}
    \caption{Implemented inference methods in \tnreason.}
    \label{tab:inferenceMethods}
\end{figure}



\subsect{Optimization}

Optimization is a reasoning task of finding a maximal coordinate given a tensor network.
The supported methods are implemented in \inlinecode{reasoning.optimization_handling} and listed in \figref{tab:optimizationMethods}.

\begin{figure}
    \begin{center}
        \begin{tabular}{|p{\threecolumnwidth}|p{\threecolumnwidth}|p{\threecolumnwidth}|}
            \hline
            \textbf{optimizationMethod} (str) & \textbf{Package}      & \textbf{Applied procedure}                                                 \\
            \hline
            \stringof{numpyArgMax}            & \inlinecode{numpy}    & Transformation into a numpy core and solution by \inlinecode{numpy.argmax} \\
            %\hline
            \stringof{gurobi}                 & \inlinecode{gurobipy} & Transformation into an ILP and solution by \inlinecode{gurobipy.optimize}  \\
            %\hline
            \stringof{gibbsSample}            & $--$                  & Simulated annealing based on gibbs sampling                                \\
            %\hline
            \stringof{meanFieldSample}        & $--$                  & Mean field approximation combined with \stringof{gibbsSample}              \\
            \hline
        \end{tabular}
    \end{center}
    \caption{Implemented optimization methods in \tnreason.}
    \label{tab:optimizationMethods}
\end{figure}



\sect{Subpackage \spapplication}\label{sec:implementationApplication}

With the \spapplication subpackage we provide an interface for reasoning workload.
It builds a third layer, since it used \sprepresentation to represent knowledge by tensor networks and \spreasoning in the execution of reasoning tasks.
%
A user-friendly high-level syntax of script language (logical formulas or neuro-symbolic architectures) for the specification of tensor networks creation, such as propositional formulas or categorical constraints, is introduced.
Given a specification of a formula $\exformula$ in script language $\synencodingof{\cdot}$, the task amounts to building a semantic representation based on the syntactic specification.


\subsect{Representation of formulas}

Propositional formulas $\exformula$ are represented in three schemes:
\begin{itemize}
    \item Syntactical representation by a script language $\synencodingof{\exformula}$ as nested lists (see \secref{subsec:scriptLanguage}).
    %Most practical to choose a formula from a neuro-symbolic architecture.
    \item Syntactical representation by a \inlinecode{str} specifying a color to the categorical variables $\headvariableof{\exformula}$.
    \item Representation of formulas by tensor networks being contracted to $\bencodingofat{\exformula}{\headvariableof{\exformula},\nodevariables}$
\end{itemize}

Conversions of the formats:
\begin{itemize}
    \item $\synencodingof{\exformula}$ to color by
    \begin{lstlisting}
        application.get_formula_color(S(f))
    \end{lstlisting}
    Here the nested lists are turned in a string by concatenating all elements of a list with \stringof{\_} and adding \stringof{[} and \stringof{]} at the beginning and end of each list.
    \item  $\synencodingof{\exformula}$ to tensor network
    \begin{lstlisting}
        application.create_raw_cores(S(f))
    \end{lstlisting}
    This creates the connective cores for the semantic representation of $\bencodingof{\exformula}$.
    We encode them by iterative calls of \inlinecode{engine.create_from_iterators}.
\end{itemize}

%When encoding formulas with hard interpretation, we furthermore add a head core of type \stringof{truthEvaluation} since we have
%\[ {\exformula} = \contractionof{\bencodingof{\exformula},\tbasis}{\catvariableof{\exformula}} \, . \]

\subsect{Script Language}\label{subsec:scriptLanguage}

To specify propositional sentences, neuro-symbolic architectures and \MarkovLogicNetworks{}, we developed a script language.

% Atoms

\paragraph{Atomic Formulas} are represented by arbitrary strings, which are not used for the representation of connectives.
We further avoid the symbols \{\stringof{(}, \stringof{)}, \stringof{\_}\} in the names of atoms, to not confuse them with colors of categorical variables.

% Composed Formulas

\paragraph{Composed Formulas} are represented by nested lists, where each sublist is either specifying an atomic formula (if string) or another composed formula.
For example, a formula $\exformula_1\exconnective,\exformula_2$ is represented by
\begin{centeredcode}
    $\synencodingof{\exformula_1\exconnective,\exformula_2}$ = [$\synencodingof{\exconnective}$, $\synencodingof{\exformula_1}$, $\synencodingof{\exformula_2}$]
\end{centeredcode}
where we apply the conventions
\begin{itemize}
    \item Connectives are at the 0th position in each list
    \item Further entries are either atoms as strings or encoded formulas itself
\end{itemize}
The nested lists follows a grammar, which is provided in \figref{tab:bnFunctions} in its Backus-Naur form.


% Backus-Naur
\begin{figure}
    \begin{center}
        \begin{tabular}{|p{2cm}|p{8cm}|}
            %\hline
            %Unary Connective  & \stringof{not} | \stringof{id}                                                  \\
            \hline
            $\atomorder$-ary connective & \stringof{and} | \stringof{or} | \stringof{imp} | \stringof{xor}  | \stringof{eq} | \stringof{not} | $\cdots$ \\
            & \stringof{\atomorder} + "$\_$" + \stringof{N}, where $N<2^{2^{\atomorder}-1}$                               \\
            \hline
            Atomic Formula              & Set of strings not in Connectives                                                                           \\
            \hline
            Complex Formula             & Atomic Formula | [$\atomorder$-ary connective, $\atomorder$ Complex Formulas]                               \\
            %& [Binary Connective, Complex Formula, Complex Formula]                           \\
            \hline
        \end{tabular}
    \end{center}
    \caption{Backus-Naur form of the grammar producing the nested list expressions.
    The string connectives are either appearing in the list of \figref{tab:connectives}, or represented by a Wolfram code.
    }\label{tab:bnFunctions}
\end{figure}



\begin{example}[Encoding of the Wet Street example]
    For example we have
    \begin{itemize}
        \item
        Atomic variable $\var{Rained}$ by
        \begin{centeredcode}
            $\synencodingof{\var{Rained}}$
			= \stringof{Rained}
        \end{centeredcode}
        \item
        Negative literal $\lnot\var{Rained}$ by
        \begin{centeredcode}
            $\synencodingof{\lnot\var{Rained}}$
			= [\stringof{not},\stringof{Rained}]
        \end{centeredcode}
        \item
        Horn clause $\left(\var{Rained}\Rightarrow\var{Wet}\right)$ by
        \begin{centeredcode}
            $\synencodingof{\var{Rained}\Rightarrow\var{Wet}}$
			= [\stringof{imp},\stringof{Rained},\stringof{Wet}]
        \end{centeredcode}
        \item
        Knowledge Base
        $(\lnot\var{Rained})\land(\var{Rained}\Rightarrow\var{Wet})$ by
        \begin{centeredcode}
            $\synencodingof{\lnot\var{Rained})\land(\var{Rained}\Rightarrow\var{Wet}}$
			=  [\stringof{and}, [\stringof{not}, \stringof{Rained}], [\stringof{imp}, \stringof{Rained}, \stringof{Wet}]]
        \end{centeredcode}
    \end{itemize}
\end{example}




\textbf{Knowledge Bases}

% Should distinguish these in knowledge?
We distinguish here formulas, with propositional logic interpretation and formulas which have a soft logic interpretation.
%\textbf{Facts.}
The formulas with hard interpretation are called facts in a knowledge base $\kb$ and encoded by dictionaries
\begin{centeredcode}
    \{key($\exformula$) : $\synencodingof{\exformula}$ for $\exformula\in\kb$ \}
\end{centeredcode}

\textbf{\MarkovLogicNetworks{}}

%\textbf{Weighted formulas.}
The formulas with soft interpretation are called weighted formulas and encoded by $\expof{\weightof{\exformula}\cdot\exformula}$.
We thus require a specification of the weights, which we do by adding $\weightof{\exformula}$ as a $\mathrm{float}$ or an $\mathrm{int}$ to the list $\synencodingof{\exformula}$.
We then store \MarkovLogicNetworks{} by dictionaries
\begin{centeredcode}
    \{key($\exformula$) : $\synencodingof{\exformula}$ + [$\weightof{\exformula}$] for $\exformula\in\formulaset$\}
\end{centeredcode}

\textbf{Neuro-Symbolic Architecture by Nested Lists}

% Generalizing the script language to specify architectures
To specify neuro-symbolic architectures in terms of formula selecting maps, as has been the subject of \charef{cha:formulaSelection} we further exploit the nested list structure of encoding propositional logics.
We replace, in each hierarchy of the nested structure each entry by a list of possible choices.
In this way, we reinterpret the list index as the choice indices $\selindex$ introduced for connective and formula selections (see \defref{def:connectiveSelector} and \ref{def:formulaSelector}).
More formally, the production rules are formalized in \figref{tab:bnNeurons} by the extension of the Backus-Naur form in \figref{tab:bnFunctions}.

% Neurons
\paragraph{Connective selectors} (see \defref{def:connectiveSelector}) are encoded by the list
\begin{centeredcode}
    $\synencodingof{\exconnective}$
			= [$\synencodingof{\exconnective_{0}}$, $\ldots$, $\synencodingof{\exconnective_{\seldim\shortminus1}}$]
\end{centeredcode}
and a formula selector (see \defref{def:formulaSelector}) by
\begin{centeredcode}
    $\synencodingof{\fselectionmap}$
			= [$\synencodingof{\exconnective_{0}}$, $\ldots$, $\synencodingof{\exconnective_{\seldim\shortminus1}}$]
\end{centeredcode}

\paragraph{A logical neuron} of order $\selorder$ (see \defref{def:fsNeuron}), defined by a connective selector $\exconnective$, and a formula selector $\fselectionmap_\atomenumerator$ on each argument $\atomenumerator\in[\selorder]$, is encoded by
\begin{centeredcode}
    $\synencodingof{\lneuron}$
			= [$\synencodingof{\exconnective}$, $\synencodingof{\fselectionmap_0}$, $\ldots$,  $\synencodingof{\fselectionmap_{\selorder-1}}$]
\end{centeredcode}
Only the unary $\selorder=1$ and the $\selorder=2$ cases are supported.


% Confusing?
The resulting nested lists indices have an alternating interpretation at each level compared with the elements of each list.
That is, when $\synencodingof{\lneuron}$ is the encoding of a neuron, then any element $x\in\synencodingof{\lneuron}$ represents a list of choices.
When $x$ is not the first element, then each choice is either the encoding $\synencodingof{\catvariable}$ of an atomic formula, or another neuron.

% Find another symbol?

\paragraph{A neural architecture} $\larchitecture$ is then represented in the dictionary
\begin{centeredcode}
    $\synencodingof{\larchitecture}$ = \{key($\lneuron$) : $\synencodingof{\lneuron}$ for $\exformula\in\larchitecture$\}
\end{centeredcode}
%To store this structure, we choose dictionaries of neuron spe
%\begin{centeredcode}
%	\{key($\lneuron$) : $\synencodingof{\lneuron}$ for $\exformula\in\formulaset$\}
%\end{centeredcode}
where key($\lneuron$) is a string, which can be used in the formula selections of other neurons.

% Important for well-definedness
It is important that the directed graph of neurons induced by the choice possibilities is acyclic, to ensure a well-defined architecture.


\begin{figure}
    \begin{center}
        \begin{tabular}{|p{2cm}|p{8cm}|}
            %\hline
            %Unary Connectives  & [Unary Connective] | [Unary Connective] + Unary Connectives    \\
            \hline
            $\atomorder$-ary connectives & [$\atomorder$-ary connective] | [$\atomorder$-ary connective] + $\atomorder$-ary connectives \\
            %\hline
            %Neuron Name & Any set of strings not used for atoms or connectives \\
            \hline
            Dependency Choice            & Atomic Formula | Neuron                                                                      \\
            \hline
            Dependency Choices           & [Dependency Choice] | [Dependency Choice] + Dependency Choices                               \\
            \hline
            Neuron                       & [$\atomorder$-ary connectives, $\atomorder$ Dependency Choices]                              \\
            % & [Binary Connectives, Dependency Choices, Dependency Choices]   \\
            \hline
        \end{tabular}
    \end{center}
    \caption{Extension of the grammar in Backus-Naur form in \figref{tab:bnFunctions} to describe selections of functions.}\label{tab:bnNeurons}
\end{figure}


\begin{example}[Neuro-Symbolic Architecture for the Wet Street]
    Following the wet street example, we can define a neuron by
    \begin{centeredcode}
        $\synencodingof{\lneuron}$ = [[\stringof{imp},\stringof{eq}],[\stringof{Wet},\stringof{Sprinkler}],[\stringof{Street}]]
    \end{centeredcode}
    from which the formulas
    \begin{centeredcode}
        [\stringof{imp}, \stringof{Wet}, \stringof{Street}] \\
		\hspace{0.25cm} [\stringof{eq}, \stringof{Wet}, \stringof{Street}] \\
		\hspace{1cm}[\stringof{imp}, \stringof{Sprinkler}, \stringof{Street}] \\
		\hspace{1cm}[\stringof{eq}, \stringof{Sprinkler}, \stringof{Street}]
    \end{centeredcode}
    can be chosen.
    Combining this neuron with further neurons, e.g. by the architecture
    \begin{centeredcode}
        $\synencodingof{\larchitecture}$ = \{ \stringof{neur1}: [[\stringof{imp},\stringof{eq}],[\stringof{neur2}],[\stringof{Street}]] , \\
		\hspace{1.8cm}\stringof{neur2}: [[\stringof{lnot},\stringof{id}],[\stringof{Wet},\stringof{Sprinkler}],[\stringof{Street}]] \}
    \end{centeredcode}
    the expressitivity increases.
    In this case, the further neuron provides the flexibility of the first atoms to be replaced by its negation.
\end{example}



\subsect{Distributions}

%\paragraph{}
%We encode Markov Networks by specifying a set of tensor cores.
Distributions are procedures to specify \inlinecode{representation.ComputationActivationNetworks} and are derived from the base class \inlinecode{representation.DistributionBase}.
Each distribution needs to have a routine
\begin{lstlisting}
	exampleDistribution.create_caNetwork()
\end{lstlisting}
creating the corresponding network.
%creating the factor cores and
%\begin{lstlisting}%
%	.get_partition_function()
%\end{lstlisting}
%calculating the partition function.
%Although the partition function can be calculated by the contraction of all cores, we separate the method since there are situations where a faster calculation can be performed.

\paragraph{Markov Networks} are distributions, which do not have computation cores.
The positive coordinates of the factors are represented by \inlinecode{representation.SoftPartitionFactors} and the support of the factors \inlinecode{representation.HardPartitionFactors}.
%\begin{lstlisting}
%	application.
%\end{lstlisting}

\paragraph{Empirical Distributions} are special instances of Markov Networks specifying distributions of sample data.
We represent the values as a $\cpformat$ Format of data cores as specified in \secref{sec:empDistribution}.
They are initialized by
\begin{lstlisting}
	application.get_empirical_distribution(sampleDf, atomColumns, interpretation, dimensionsDict)
\end{lstlisting}
where
\begin{itemize}
    \item \inlinecode{sampleDf} is a \inlinecode{pandas.DataFrame} specifying the data
    \item \inlinecode{atomColumns} is a list of column names in \inlinecode{sampleDf} to be extracted as variables of the distribution.
    \item \inlinecode{interpretation} is either \stringof{atomic} or \stringof{categorical}, specifying whether the entries in \inlinecode{sampleDf} are interpreted as uncertainties in the interval $[0,1]$, or as assignments to
\end{itemize}

Here the partition function is the number of samples used in the creation of the empirical distribution.


\textbf{HybridKnowledgeBases} are probability distributions, which are specified by propositional formulas in the script language.
\begin{lstlisting}
	application.HybridKnowledgeBase
\end{lstlisting}
They are initialized with arguments
\begin{itemize}
    \item facts: Dictionary of propositional formulas stored as $\synencodingof{\exformula}$ representing hard logical constraints
    \item weightedFormulas: Dictionary of propositional formulas stored as $\synencodingof{\exformula}$+$[\weightof{\exformula}]$ representing soft logical constraints
    \item evidence: Dictionary of atomic formulas, where key are the formulas in string representation and values the certainty in $[0,1]$ (float or int) of the atom being true
    \item categoricalConstraints: Dictionary of categorical constrained, which values are lists of atomic formulas stored as strings $\synencodingof{\atomicformula}$
\end{itemize}


\subsect{Inference}

To simplify deductive inference on models a class
\begin{lstlisting}
	application.InferenceProvider
\end{lstlisting}
taking a \inlinecode{representation.ComputationActivationNetwork} or \inlinecode{application.Distribution} has been implemented.

% Probabilistic Queries

\paragraph{Probabilistic queries} as specified \defref{def:queries})  by
\begin{lstlisting}
	.query(variableList, evidenceDict)
\end{lstlisting}

\paragraph{Mode queries} by
\begin{lstlisting}
	.exact_map_query()
\end{lstlisting}
%or by
%\begin{lstlisting}
%	.annealed_sample()
%\end{lstlisting}
%using Simulated Annealing (see Remark~\ref{rem:simulatedAnnealing}) to find an approximate maximum.
%The second method circumvents the creation of the coordinatewise representation of the distribution and circumvents therefore, at the expense of potentially approximative solutions, a bottleneck in case of many query variables.

% Entailment Queries

\paragraph{Entailment} from the distribution (\defref{def:probEntailment}) is be decided by
\begin{lstlisting}
	.ask(queryFormula, evidenceDict)
\end{lstlisting}
where queryFormula is the formula $\exformula$ to be tested for entailment in the representation $\synencodingof{\exformula}$.

% Sampling

\paragraph{Samples} can be drawn by
\begin{lstlisting}
	.draw_samples(sampleNum, variableList, annealingPattern)
\end{lstlisting}
based on Gibbs sampling, where
\begin{itemize}
    \item sampleNum (int) gives the number of samples to be drawn
    \item variableList (list of str) defines the variables to be represented by the samples (default: all atoms in the distribution)
    \item annealingPattern specifies an annealing pattern
\end{itemize}


%\subsect{Parameter Estimation}
%
%\textbf{EntropyMaximizer} implements Algorithm~\ref{alg:AWO}, which is motivated by the maximum entropy principle (see \secref{sec:maxEntProblem}) to optimize \MarkovLogicNetworks{}.
%The class
%\begin{lstlisting}
%	application.EntropyMaximizer
%\end{lstlisting}
%is initialized with the arguments
%\begin{itemize}
%    \item expressionsDict: Dictionary of formulas in the format $\synencodingof{\exformula}$
%    \item satisfactionDict: Dictionary of the satisfaction rates (mean parameters) to be matched by the optimal distribution
%\end{itemize}
%The optimization is then performed by
%\begin{lstlisting}
%	.alternating_optimization(sweepNum, updateKeys)
%\end{lstlisting}
%method, where the iteration in  through the updateKeys is performed sweepNum times.

\subsect{Learning}

To learn instances of \inlinecode{application.HybridKnowledegBase} on data the class
\begin{lstlisting}
	application.HybridLearner
\end{lstlisting}
is initialized with the arguments
\begin{itemize}
    \item knowledgeBase: Distribution representing a current model to be improved
    \item specDict: A neuro-symbolic architecture encoded in a dictionary of neurons
\end{itemize}

\paragraph{Formula Selecting Neural Networks} (\defref{def:fsNeuron})
are specified to define a proposal distribution.
They are encoded by creating all formula selecting neurons, each involving a
\begin{itemize}
    \item a connective selection map (\defref{def:connectiveSelector})
    \item variable selection cores (\defref{def:variableSelector}) to each argument using the decomposition of \theref{the:varSelectorDecomposition}.
\end{itemize}
Each selection variable of each neuron comes with a control variable with suffix \stringof{\_sV}.

\paragraph{Structure Learning}
is performed by
\begin{lstlisting}
	application.HybridLearner.propose_candidate()
\end{lstlisting}
where a proposal distribution is instantiated and then sampled given the specified inference method.

\paragraph{Weight Estimation}
is performed by
\begin{lstlisting}
	application.HybridLearner.infer_weights_on_data(empDistribution)
\end{lstlisting}
where \inlinecode{empDistribution} is used to infer the mean parameters to be matched by the canonical parameters of \inlinecode{knowledgeBase.weightedFormulas}.