%%% Font styles

% Code
\newcommand{\code}[1]{\texttt{#1}}
% Definition
\newcommand{\define}[1]{\emph{#1}}

%%% Spacings

\newcommand{\qspaced}[1]{\quad #1 \quad} % quad auf beiden Seiten

%%% General Environments

% draft environment
\newenvironment{draft}{\color{red}\begin{enumerate}[label={$\vartriangleright$}]}{\end{enumerate}}
\newcommand{\drafting}[1]{{\color{red}#1}}

% Highlight environment
\newenvironment{highlight}{\begin{quote}\itshape}{\end{quote}}

%%% Enumerations

\newenvironment{listing}
{\begin{enumerate}[itemindent=0em,leftmargin=1.2em]}
{\end{enumerate}}

\newenvironment{listingroman}
{\begin{enumerate}[label={(\roman*)}]}
{\end{enumerate}}

% Umgebung zu Aufzählung von Definitionen/Eigenschaften
\newenvironment{properties}[2][2em]
{\begin{enumerate}[label={\textsc{(#2\arabic*)}},leftmargin=#1]}
{\end{enumerate}} 

% Aufzählung in Bemerkungen (spezielle Einrückung)
\newenvironment{rmklist}
{\begin{enumerate}[label={(\arabic*)},itemindent=2em,leftmargin=0em]}
{\end{enumerate}}

% Paragraphen-ähnliche Aufzählung
\newenvironment{paralist}
{\begin{enumerate}[label={(\arabic*)},itemindent=1.7em,leftmargin=0em]}
	{\end{enumerate}}

% Liste in Sätzen (spezielle Einrückung)
\newenvironment{thmlist}
{\begin{enumerate}[label={(\arabic*)}]}
{\end{enumerate}}
% Aufzählung von Eigenschaften in Sätzen (spezielle Einrückung)
\newenvironment{thmproperties}
{\begin{enumerate}[label={(\roman*)}]}
{\end{enumerate}}

% Beweisschritte
\newenvironment{proofsteps}
{\begin{enumerate}[label={(\arabic*)},itemindent=2em,leftmargin=0em]}
{\end{enumerate}}

% Liste in Definitionen (spezielle Einrückung)
\newenvironment{deflist}
{\begin{enumerate}[label={(\arabic*)}]}
{\end{enumerate}}

% Experiments
\newenvironment{expstep}
{\begin{itemize}[label={$\blacktriangleright$},leftmargin=1.5em]}
{\end{itemize}}
\newenvironment{expsubstep}
{\begin{itemize}[label={$\vartriangleright$},leftmargin=3em]}
{\end{itemize}}

\newcommand{\expkwd}[1]{\noindent\textbf{#1}}

%%% Math Environments

% QED-Boxes
\renewcommand{\qedsymbol}{$_\blacksquare$}
\newcommand{\qeddiamond}{\hfill$\Diamond$}
\providecommand{\qedhere}{\hfill\qedsymbol}

% Styles for mathematical statements
\newtheoremstyle{claim}
	{\topsep}{\topsep}%
	{\itshape}%         Body font
	{}%         Indent amount (empty = no indent, \parindent = para indent)
	{}% Thm head font
	{}%        Punctuation after thm head
	{.5em}%     Space after thm head (\newline = linebreak)
	{{\bfseries\boldmath\thmname{#1} \thmnumber{#2}} \thmnote{(#3)}}%         Thm head spec

\newtheoremstyle{definition}
	{\topsep}{\topsep}%
	{}%         Body font
	{}%         Indent amount (empty = no indent, \parindent = para indent)
	{}% Thm head font
	{}%        Punctuation after thm head
	{.5em}%     Space after thm head (\newline = linebreak)
	{\textbf{\thmname{#1} \thmnumber{#2}} \thmnote{(#3)}}%         Thm head spec
	
\newtheoremstyle{algorithm}
	{\topsep}{\topsep}%
	{}%         Body font
	{}%         Indent amount (empty = no indent, \parindent = para indent)
	{\bfseries\boldmath}% Thm head font
	{}%        Punctuation after thm head
	{\newline}%     Space after thm head (\newline = linebreak)
	{\thmname{#1} \thmnumber{#2} \thmnote{(#3)}}%         Thm head spec


% Stil für grauunterlegte Boxen
\definecolor{shadecolor}{gray}{0.90}
\mdfdefinestyle{shaded}{backgroundcolor=shadecolor, innertopmargin = 0pt, splittopskip = \topskip, skipbelow=6pt, skipabove=12pt, topline=false,bottomline=false,leftline=false,rightline=false}

% Stil für schwarz umrahmte Boxen
\mdfdefinestyle{emphframe}{linecolor=black, innertopmargin = 3pt, splittopskip = \topskip, skipbelow=6pt, skipabove=6pt}

% Theorem
\declaretheorem[style=claim,numberwithin=chapter,%
	preheadhook={\begin{mdframed}[style=shaded] \setcounter{mpfootnote}{\value{footnote}}},%
	postfoothook=\setcounter{footnote}{\value{mpfootnote}}\end{mdframed}]{theorem}
% Proposition
\declaretheorem[style=claim,sibling=theorem,%
	preheadhook={\begin{mdframed}[style=shaded] \setcounter{mpfootnote}{\value{footnote}}},%
	postfoothook=\setcounter{footnote}{\value{mpfootnote}}\end{mdframed}]{proposition}
% Lemma
\declaretheorem[style=claim,sibling=theorem,%
	preheadhook={\begin{mdframed}[style=shaded] \setcounter{mpfootnote}{\value{footnote}}},%
	postfoothook=\setcounter{footnote}{\value{mpfootnote}}\end{mdframed}]{lemma}
% Corollary
\declaretheorem[style=claim,sibling=theorem,%
	preheadhook={\begin{mdframed}[style=shaded] \setcounter{mpfootnote}{\value{footnote}}},%
	postfoothook=\setcounter{footnote}{\value{mpfootnote}}\end{mdframed}]{corollary}
% Definition
\declaretheorem[style=definition,sibling=theorem,%
	preheadhook={\begin{mdframed}[style=shaded] \setcounter{mpfootnote}{\value{footnote}}},%
	postfoothook=\setcounter{footnote}{\value{mpfootnote}}\end{mdframed}]{definition}
% Assumption
\declaretheorem[style=definition,sibling=theorem,%
	preheadhook={\begin{mdframed}[style=shaded] \setcounter{mpfootnote}{\value{footnote}}},%
	postfoothook=\setcounter{footnote}{\value{mpfootnote}}\end{mdframed}]{assumption}
% Problem
\declaretheorem[style=definition,sibling=theorem,%
	preheadhook={\begin{mdframed}[style=shaded] \setcounter{mpfootnote}{\value{footnote}}},%
	postfoothook=\setcounter{footnote}{\value{mpfootnote}}\end{mdframed}]{problem}
% Remark
\declaretheorem[style=definition,sibling=theorem,qed=$\Diamond$]{remark}
\declaretheorem[style=definition,sibling=theorem,qed=$\Diamond$,name={Remark},
	postheadhook={\begin{rmklist}},
	prefoothook={\qedhere\end{rmklist}}]{remarks}
% Example
\declaretheorem[style=definition,sibling=theorem,qed=$\Diamond$]{example}
\declaretheorem[style=definition,sibling=theorem,qed=$\Diamond$,name={Example},
	postheadhook={\begin{rmklist}},
	prefoothook={\qedhere\end{rmklist}}]{examples}
% Algorithm
%\declaretheorem[style=algorithm,sibling=theorem,%
%	preheadhook={\begin{mdframed}[style=emphframe] \setcounter{mpfootnote}{\value{footnote}}},%
%	postfoothook=\setcounter{footnote}{\value{mpfootnote}}\end{mdframed}]{algorithm}
% Experiment
\declaretheorem[style=algorithm,sibling=theorem,%
	preheadhook={\begin{mdframed}[style=emphframe] \setcounter{mpfootnote}{\value{footnote}}},%
	postfoothook=\setcounter{footnote}{\value{mpfootnote}}\end{mdframed}]{experiment}
% Recipe
\declaretheorem[style=definition,sibling=theorem,%
	preheadhook={\begin{mdframed}[style=emphframe] \setcounter{mpfootnote}{\value{footnote}}},%
	postfoothook=\setcounter{footnote}{\value{mpfootnote}}\end{mdframed}]{recipe}
	
	
	
%% NEW

% Observation
\declaretheorem[style=definition,sibling=theorem,%
	preheadhook={\begin{mdframed}[style=shaded] \setcounter{mpfootnote}{\value{footnote}}},%
	postfoothook=\setcounter{footnote}{\value{mpfootnote}}\end{mdframed}]{observation}
% Research Question
\declaretheorem[style=definition,sibling=theorem,%
	preheadhook={\begin{mdframed}[style=shaded] \setcounter{mpfootnote}{\value{footnote}}},%
	postfoothook=\setcounter{footnote}{\value{mpfootnote}}\end{mdframed}]{Research Question}
% Chaining Scheme
	\declaretheorem[style=definition,sibling=theorem,%
	preheadhook={\begin{mdframed}[style=shaded] \setcounter{mpfootnote}{\value{footnote}}},%
	postfoothook=\setcounter{footnote}{\value{mpfootnote}}\end{mdframed}]{Chaining Scheme}
% Width Bound Scheme
	\declaretheorem[style=definition,sibling=theorem,%
	preheadhook={\begin{mdframed}[style=shaded] \setcounter{mpfootnote}{\value{footnote}}},%
	postfoothook=\setcounter{footnote}{\value{mpfootnote}}\end{mdframed}]{Width Bound Scheme}

%%
