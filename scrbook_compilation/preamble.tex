%%% ADDED USE PACKAGES
\usepackage[usenames,dvipsnames]{color}
%\usepackage{hyperref}


\usepackage{tikz}
%\usepackage{graphicx}
\usepackage{float}
%\usepackage{comment}
%\usepackage{csquotes}

\usepackage{listings}
\usepackage{verbatim}
\usepackage{etoolbox}
\usepackage{braket}
%\usepackage[utf8]{inputenc}
\usepackage[english]{babel}
\usepackage[T1]{fontenc}
\usepackage{bbm}
%\usepackage{bm}
\usepackage{algpseudocode}
\usepackage{algorithm}
\algrenewcommand\algorithmicrequire{\textbf{Input:}}
\algrenewcommand\algorithmicensure{\textbf{Output:}}

% Bibliography
\usepackage[backend=biber, style=alphabetic]{biblatex}

\hyphenation{representation reasoning mechanisms identification logical}

%%% ====================================================
%%% --> KOMA-Settings <--

%%% === Textbody ==============================================================
\KOMAoptions{%
%    DIV=11,% (Size of Text Body, higher values = greater textbody)
   DIV=areaset, % (also calc/areaset/classic/current/default/last) 
   % -> after setting of spacing necessary!   
   BCOR=0mm,% (binding correction)
}%
%%% === Headings ==============================================================
\KOMAoptions{%
%	headings=small,  % Small Font Size, thin spacing above and below
	headings=normal, % Medium Font Size, medium spacing above and below
%	headings=big, % Big Font Size, large spacing above and below
% 	headings=nochapterprefix,  % no prefix at chapters
	headings=chapterprefix,    % inverse of 'nochapterprefix'
%	headings=noappendixprefix, % chapter in appendix as in body text
	headings=appendixprefix,   % inverse of 'noappendixprefix'
%	headings=openany,   % Chapters start at any side
%	headings=openleft,  % Chapters start at left side
   headings=openright, % Chapters start at right side      
   %%% Add/Dont/Auto Dot behind section numbers 
   %%% (see DUDEN as reference)
%   numbers=autoenddot,
%   numbers=enddot,
	numbers=noenddot,
% 	secnumdepth=3, % depth of sections numbering (???)
}%
\setcounter{secnumdepth}{3}
%%% === Page Layout ===========================================================
\KOMAoptions{% (most options are for package typearea)
	twoside=semi, % two side layout (alternating margins, standard in books)
	% twoside=false, % single side layout 
	% twoside=semi,  % two side layout (non alternating margins!)
	twocolumn=false, % (true)
	headinclude=true,%
	footinclude=false,%
	mpinclude=false,%
% 	headlines=2.1,%
% 	headheight=2em,%
% 	headsepline=true,% calls headinclude automatically
% 	footsepline=false,% calls footinclude automatically
	cleardoublepage=empty, %plain, headings
}%
% Set type area manually, headlines/headheight/footlines/foothight are ignored!
\areaset[0mm]{15cm}{25cm}
%%% === Paragraph Separation ==================================================
\KOMAoptions{%
	parskip=relative, % change indentation according to fontsize (recommended)
% 	parskip=absolute, % do not change indentation according to fontsize
% 	parskip=false    % indentation of 1em
% 	parskip=true   % parksip of 1 line - with free space in last line of 1em
% 	parskip=full-  % parksip of 1 line - no adjustment
% 	parskip=full+  % parksip of 1 line - with free space in last line of 1/4
% 	parskip=full*  % parksip of 1 line - with free space in last line of 1/3
% 	parskip=half   % parksip of 1/2 line - with free space in last line of 1em
% 	parskip=half-  % parksip of 1/2 line - no adjustment
% 	parskip=half+  % parksip of 1/2 line - with free space in last line of 1/3
% 	parskip=half*  % parksip of 1/2 line - with free space in last line of 1em
}%
% Manuell
% \setlength{\parskip}{4pt}
% \setlength{\parindent}{0em}
%%% === Table of Contents =====================================================
% \setcounter{tocdepth}{3} % Depth of TOC Display
\KOMAoptions{%
   %%% Setting of 'Style' and 'Content' of TOC
% 	toc=left,
   toc=indented,
% 	toc=bib,
% 	toc=nobib,
% 	toc=bibnumbered,
% 	toc=index,
% 	toc=noindex,
% 	toc=listof,
% 	toc=nolistof
% 	toc=listofnumbered, 
}%  
\setcounter{tocdepth}{3} % Depth of TOC Display
%%% === Lists of figures, tables etc. =========================================
\KOMAoptions{%
   %%% Setting of 'Style' and 'Content' of Lists 
   %%% (figures, tables etc)
	% --- General List Style ---
   listof=left, % tabular styles
%   listof=indented, % hierarchical style
   % --- chapter highlighting ---
%   listof=chapterentry, % ??? Chapter starts are marked in figure/table
   % listof=chaptergapline, % New chapter starts are marked by a gap 
      		  			   	 % of a single line
	%listof=chaptergapsmall, % New chapter starts are marked by a gap 
   	    					   % of a smallsingle line
   % listof=nochaptergap, % No Gap between chapters
   %
   % listof=leveldown, % lists are moved one level down ???
   % --- Appearance of Lists in TOC
   % listof=notoc, % Lists are not part of the TOC
   listof=totoc, % add Lists to TOC without number
%    listof=totocnumbered, % add Lists to TOC with number
}%  
%%% === Bibliography ==========================================================
%% Setting of 'Style' and 'Content' of Bibliography
\KOMAoptions{%
% 	bibliography=oldstyle,%
   bibliography=openstyle,%
%    bibliography=nottotoc, % Bibliography is not part of the TOC
%    bibliography=totocnumbered, % add Bibliography to TOC with number
   bibliography=totoc, % add Bibliography to TOC without number
}%
\bibliography{references}
%%% === Index =================================================================
%% Setting of 'Style' and 'Content' of Index in TOC
\KOMAoptions{%
%    index=nottotoc % index is not part of the TOC
	index=totoc, % add index to TOC without number
}%
%%% === Titlepage =============================================================
\KOMAoptions{%
   titlepage=true %
%    titlepage=false %
}%
%%% === Miscellaneous =========================================================
\KOMAoptions{% 	
% 	footnotes=multiple% nomultiple
% 	open=any,%
% 	open=left,%
% 	open=right,%
% 	chapterprefix=true,%
% 	appendixprefix=true,%
% 	chapteratlists=10pt,% entry
}%

%%% ====================================================
%%% --> Packages: Programming <--

\makeatletter
% Mark packges that should be loaded later
\newcommand{\LoadPackagesNow}{}
\newcommand{\LoadPackageLater}[2][]{%
   \g@addto@macro{\LoadPackagesNow}{%
      \usepackage[#1]{#2}%
   }%
}
\makeatother

%%% XSPACE
% Define commands that don't eat spaces.
\usepackage{xspace}
%%% XARGS
% More than one optional argument
\usepackage{xargs}
%%% IFTHEN
\usepackage{ifthen}
%%% ETOOLBOX
% Programming in LaTeX
\usepackage{etoolbox}
%%% CALC
\usepackage{calc}

%%% ====================================================
%%% --> Packages: Layout <--

% Recalculate type area
\recalctypearea % use last
% \KOMAoptions{DIV=calc} % specific way to calculate type area
% Modification of length values (must be after calculation of type area!)
% \setlength{\topmargin}{5cm} % Height of the headline (offset)


%%% LAYOUT (print layout)
\usepackage{layout}

%%% TITLEPAGE
% Modify titlepage
\usepackage{scrbook_compilation/titlepage}
%%% TITLESEC
% \usepackage{titlesec}
%%% FRAMED
\usepackage{framed}
%%% MDFRAMED
% Draw fancy frames
\usepackage{mdframed}
%%% PBOX
\usepackage{pbox}
%%% ENUMITEM
% Better than 'paralist' and 'enumerate' because it uses a keyvalue interface !
% Do not load together with enumerate.
\usepackage{enumitem}
%%% TABULARX
\usepackage{tabularx}
\newcolumntype{Z}{>{\raggedright\let\newline\\\arraybackslash\hspace{0pt}}m}
%%% ARRAY
% tables and arrays
\usepackage{array}
%%% MULTIROW
\usepackage{multirow}
%%% LONGTABLE
\usepackage{longtable}
%%% HHLINE
\usepackage{hhline}
% Stretch the vertical spacing in tables
% \renewcommand{\arraystretch}{1.1}
%%% AFTERPAGE
\usepackage{afterpage}
%%% PDFLSCAPE
\usepackage{pdflscape}
%%% ROTATING
\usepackage{rotating}
%%% BOOKTABS
% Better spacing in tablulars
\usepackage{booktabs}
%%% SETSPACE
% Abstand zwischen den Zeilen
\usepackage{setspace}
% \onehalfspacing		% 1,5-facher Abstand
% \doublespacing		% 2-facher Abstand
%%% FOOTMISC
% Modify footnotes
\usepackage[
   bottom,      % Footnotes appear always on bottom. This is necessary
                % especially when floats are used
   stable,      % Make footnotes stable in section titles
%    perpage,     % Reset on each page
   %para,       % Place footnotes side by side of in one paragraph.
   %side,       % Place footnotes in the margin
%    ragged,      % Use RaggedRight
   %norule,     % suppress rule above footnotes
   multiple,    % rearrange multiple footnotes intelligent in the text.
   %symbol,     % use symbols instead of numbers
   flushmargin,
%   splitrule,
   hang,
]{footmisc}
\interfootnotelinepenalty=10000 % avoid footnote breaks
%%% FOOTNOTE
\usepackage{footnote}
\makesavenoteenv{mdframed}
% Footnotestyle in theorem-environments
\renewcommand*{\thempfootnote}{\arabic{mpfootnote}}
% \renewcommand*{\multfootsep}{,\nobreakspace}
%Fußnotenstil
% \renewcommand*{\thefootnote}{\arabic{footnote})}

%%% ANYFONTSIZE
% Removes warnings on font style not available
\usepackage{anyfontsize}

%%% --> Font types <--
% Different styles
% \renewcommand\rmdefault{ptx} % pnc, ptm, phv, bch, cmss, put, ptx
%%% MATHPAZO
\usepackage{mathpazo}
%%% MATHPTMX
% \usepackage{mathptmx} % selects Times Roman as basic font
%%% NEWTX
% \usepackage{newtxtext}
% Important: newtxmath is loaded below after amsmath and amsthm
%%% NEWPX
% \usepackage{newpxtext}
% Important: newpxmath is loaded below after amsmath and amsthm
%%% More fonts
% \usepackage{helvet} % selects Helvetica as sans-serif font
\usepackage{cabin} % selects Cabin as sans-serif font
\usepackage{courier} % selects Courier as typewriter font
% \usepackage{type1cm} % activate if the above 3 fonts are not available on your system


%%% SCRPAGE2
% Modify head- and footlines
\usepackage[%
%    headtopline,
%    plainheadtopline,
   headsepline,
%    plainheadsepline,
%    footsepline,
%    plainfootsepline,
%    footbotline,
%    plainfootbotline,
%    ilines,
%    clines,
%    olines,
   automark,
%    autooneside,% ignore optional argument in automark at oneside
%    komastyle,
%    standardstyle,
%    markuppercase,
%    markusedcase,
%    nouppercase,
]{scrlayer-scrpage}

% clear default settings
\clearscrheadings
\clearscrplain
% content
\ohead{\pagemark}
\ihead{\headmark}
\ofoot[\pagemark]{}
% Vollstaendige Liste der moeglichen Positionierungen
% \lehead[scrplain-links-gerade]{scrheadings-links-gerade}
% \cehead[scrplain-mittig-gerade]{scrheadings-mittig-gerade}
% \rehead[scrplain-rechts-gerade]{scrheadings-rechts-gerade}
% \lefoot[scrplain-links-gerade]{scrheadings-links-gerade}
% \cefoot[scrplain-mittig-gerade]{scrheadings-mittig-gerade}
% \refoot[scrplain-rechts-gerade]{scrheadings-rechts-gerade}
% \lohead[scrplain-links-ungerade]{scrheadings-links-ungerade}
% \cohead[scrplain-mittig-ungerade]{scrheadings-mittig-ungerade}
% \rohead[scrplain-rechts-ungerade]{scrheadings-rechts-ungerade}
% \lofoot[scrplain-links-ungerade]{scrheadings-links-ungerade}
% \cofoot[scrplain-mittig-ungerade]{scrheadings-mittig-ungerade}
% \rofoot[scrplain-rechts-ungerade]{scrheadings-rechts-ungerade}
% \ihead[scrplain-innen]{scrheadings-innen}
% \chead[scrplain-zentriert]{scrheadings-zentriert}
% \ohead[scrplain-außen]{scrheadings-außen}
% \ifoot[scrplain-innen]{scrheadings-innen}
% \cfoot[scrplain-zentriert]{scrheadings-zentriert}
% \ofoot[scrplain-außen]{scrheadings-außen}

% what is displayed in headmark
\automark[section]{chapter} %[rechts]{links}
\automark*[chapter]{chapter}
% modify sep lines
% \setheadtopline{.4pt}     % modifiziert die Parameter fuer die Linie ueber dem
% \setheadsepline{.4pt}[\color{black}]

% width of the header
%\setheadwidth[]{
%   paper % width of paper
%   page  % width of page (paper - BCOR)
%   text  % \textwidth
%   textwithmarginpar % width of text plus margin
%   head  % current width of head
%   foot  % current width of foot
%}

%%% ====================================================
%%% --> Packages: Textprocessing <--

%%% RELSIZE
% relative change of font size
\usepackage{relsize}
%%% ULEM
% Zum Unterstreichen
\usepackage[normalem]{ulem}      
%%% SOUL
% Unterstreichen, Sperren
\usepackage{soul}		           
%%% URL
% Setzen von URLs. In Verbindung mit hyperref sind diese auch aktive Links.
\usepackage{url}
%%% LIPSUM
% dummy text
\usepackage{lipsum}
%%% CLEVEREF
%\LoadPackageLater[nameinlink]{cleveref}
\LoadPackageLater{nameref}

%%% ====================================================
%%% --> Packages: Images, Figures and Color <--

%%% PSTRICKS
% \usepackage{pstricks}
% \usepackage{pst-plot, pst-node, pst-coil, pst-eps}
%%% XCOLOR
% Farben
% Incompatible: Do not load when using pstricks !
%\usepackage[table]{xcolor}
% STANDALONE
\usepackage{standalone}
% TIKZ
\usepackage{tikz}
\usetikzlibrary{arrows,shapes,calc,decorations.pathreplacing,intersections,quotes,angles,positioning,fit,petri,through}
%%% GRAPHICX
% Bilder
\usepackage[%
	%final,
	%draft % do not include images (faster)
]{graphicx}
% Float to page fraction (when does a figure get move to a separate page?)
\renewcommand{\floatpagefraction}{.8}
%%% EPSTOPDF
% If an eps image is detected, epstopdf is automatically called to convert it to pdf format.
% Requires: graphicx loaded
\usepackage{epstopdf}
%%% WRAPFIG
% embed figures in text
\usepackage{wrapfig}
%%% SUBFIGURE
% \usepackage{subfigure}
% %%% CAPTION
\usepackage{caption}
%%% SUBCAPTION
% Don't use this together with subfigure
\usepackage{subcaption}
% % Appearance of captions
\captionsetup{
   margin = 10pt,
   font = {small},
   labelfont = {small},
   format = plain, % oder 'hang'
   indention = 0em,  % Einruecken der Beschriftung
   labelsep = colon, %period, space, quad, newline
   justification = justified, % justified, centering
   singlelinecheck = true, % false (true=bei einer Zeile immer zentrieren)
   position = bottom, %top ,
   subrefformat = simple,
   labelformat=simple,
}
\captionsetup[subfigure]{labelformat=simple}
\renewcommand\thesubfigure{(\alph{subfigure})}

%%% ====================================================
%%% --> Packages: Math <--

% \let\openbox\relax % Fix for font packages
%%% AMSMATH
% Amsmath - Mathematik Basispaket
% fuer pst-pdf displaymath Modus vor pst-pdf benoetigt.
\usepackage[
   %centertags, % (default) center tags vertically
   tbtags,    % 'Top-or-bottom tags': For a split equation, place equation numbers level
               % with the last (resp. first) line, if numbers are on the right (resp. left).
   sumlimits,  %(default) Place the subscripts and superscripts of summation
               % symbols above and below
   %nosumlimits, % Always place the subscripts and superscripts of summation-type
               % symbols to the side, even in displayed equations.
   %intlimits,  % Like sumlimits, but for integral symbols.
   nointlimits, % (default) Opposite of intlimits.
   namelimits, % (default) Like sumlimits, but for certain 'operator names' such as
               % det, inf, lim, max, min, that traditionally have subscripts placed underneath
               % when they occur in a displayed equation.
   %nonamelimits, % Opposite of namelimits.
   %leqno,     % Place equation numbers on the left.
   reqno,     % Place equation numbers on the right.
%    fleqn,     % Position equations at a fixed indent from the left margin
   			  % rather than centered in the text column.
]{amsmath} %

%%% AMSFONTS
\usepackage{amsfonts}
%%% MATHRSFS
\usepackage{mathrsfs} %% Schreibschriftbuchstaben für den Mathematiksatz (nur Großbuchstaben)
%%% DSFONT
\usepackage{dsfont}
%%% AMSSYMB
\usepackage{amssymb}
%%% UNITS
\usepackage{units}
%%% AMSTHM
% Erst nach hyperref laden...
\LoadPackageLater{amsthm}
%%% THMTOOLS
% Frontend für amsthm
% Erst nach hyperref laden...
\LoadPackageLater{thmtools}
%%% MATHTOOLS
% Erweitert amsmath und behebt einige Bugs
\usepackage[fixamsmath,disallowspaces]{mathtools}
% Formelnummern nur anzeigen, wenn auch eine Referenz existiert
\mathtoolsset{showonlyrefs}
\mathtoolsset{centercolon=true}
%%% AUTONUM (replaces showonlyrefs)
% \LoadPackageLater{autonum}
% Break formulas when page break
\allowdisplaybreaks[4]

%%% NEWTXMATH
% Load Math Font (important: this needs to be loaded after amsmath and amsthm, otherwise there will be an \openbox clash)
% \LoadPackageLater{newtxmath}
%%% NEWPXMATH (alternative)
% \LoadPackageLater[smallerops,varbb]{newpxmath}
%%% MATHALFA
% \usepackage[scr=rsfso]{mathalfa}%
%%% BOLDMATH
% Load it after loading math fonts!
\LoadPackageLater{bm}
% \usepackage{bold-extra}
% Boldface also for formulas
\makeatletter
\g@addto@macro\bfseries{\boldmath}
\makeatother

%%% ====================================================
%%% --> Packages: Misc <--

%%% MAKEIDX
% \usepackage{makeidx}
% Create Index
% \makeindex

%%% APPENDIX
\usepackage[toc]{appendix}

%%% BIBLATEX
\usepackage{csquotes}
%\usepackage[
%	bibstyle=alphabetic,
%%	bibstyle=draft,
%	citestyle=alphabetic,
%	sorting=ynt,
%	sortcites=true,
%	giveninits=true,
%	maxbibnames=99,
%	backend=biber,
%	maxalphanames=5,
%	backref=true,
%]{biblatex}

%\renewcommand*{\bibfont}{\raggedright}

\renewbibmacro{in:}{}
%%% Customize Biblography Style
% \DeclareFieldFormat{title}{\mkbibquote{#1}}
% \DeclareFieldFormat{volume}{\mkbibbold{#1}}
% \DeclareFieldFormat[article]{volume}{\mkbibbold{#1}\space}
% \DeclareFieldFormat[article]{number}{\mkbibparens{#1}}
% \DeclareFieldFormat{journal}{\mkbibemph{#1}}
\DeclareFieldFormat{pages}{#1}
% \DeclareFieldFormat[misc]{year}{#1}
% Authors in \textsc
% \renewcommand{\mkbibnamefirst}[1]{\textsc{#1}}
% \renewcommand{\mkbibnamelast}[1]{\textsc{#1}}
% \renewcommand{\mkbibnameprefix}[1]{\textsc{#1}}
% \renewcommand{\mkbibnameaffix}[1]{\textsc{#1}}
% Style for misc
% \DeclareBibliographyDriver{misc}{%
%   \printnames{author}.%
%   \newunit\newblock
%   \printfield{title}%
%   \newunit\newblock
%   \printfield{year}%
%   \newline
%   \printfield{howpublished}%
% %   \finentry
% }

%%% TODONOTES
\usepackage[textsize=tiny,english,colorinlistoftodos,
 	disable,
	]{todonotes}
% ToDo-Marker
\newcommand{\todocontent}[2][{}]{\todo[color=orange, #1]{Content: #2}\xspace}
\newcommand{\todofig}[2][{}]{\todo[color=red, #1]{Fig: #2}\xspace}
\newcommand{\todoref}[2][{}]{\todo[color=red, #1]{Ref: #2}\xspace}
\newcommand{\todolang}[2][{}]{\todo[color=green, #1]{Language: #2}\xspace}
\newcommand{\todolatex}[2][{}]{\todo[color=blue, #1]{Format: #2}\xspace}
%%% HYPERREF
% Hyperref
% Colors for hyper-links
\definecolor{pdfurlcolor}{rgb}{0,0,0.6}
\definecolor{pdffilecolor}{rgb}{0.7,0,0}
\definecolor{pdflinkcolor}{rgb}{0,0,0.6}
\definecolor{pdfcitecolor}{rgb}{0,0,0.6}
\usepackage[
	% Farben fuer die Links
	colorlinks=true,         % Links erhalten Farben statt Kaeten
	urlcolor=pdfurlcolor,    % \href{...}{...} external (URL)
	filecolor=pdffilecolor,  % \href{...} local file
	linkcolor=pdflinkcolor,  %\ref{...} and \pageref{...}
	citecolor=pdfcitecolor,  %
	% Links
	raiselinks=true,			 % calculate real height of the link
	breaklinks,              % Links berstehen Zeilenumbruch
	%   backref=page,            % Backlinks im Literaturverzeichnis (section, slide, page, none)
	%   pagebackref=true,        % Backlinks im Literaturverzeichnis mit Seitenangabe
	verbose,
	hyperindex=true,         % backlinkex index
	linktocpage=true,        % Inhaltsverzeichnis verlinkt Seiten
	hyperfootnotes=false,     % Keine Links auf Fussnoten
	% Bookmarks
	bookmarks=true,          % Erzeugung von Bookmarks fuer PDF-Viewer
	bookmarksopenlevel=1,    % Gliederungstiefe der Bookmarks
	bookmarksopen=true,      % Expandierte Untermenues in Bookmarks
	bookmarksnumbered=true,  % Nummerierung der Bookmarks
	bookmarkstype=toc,       % Art der Verzeichnisses
	% Anchors
	plainpages=false,        % Anchors even on plain pages ?
	pageanchor=true,         % Pages are linkable
	%pdfproducer={pdfeTeX 1.10b-2.1} %Produzent
	pdfdisplaydoctitle=true, % Dokumententitel statt Dateiname im Fenstertitel
	pdfstartview=FitH,       % Dokument wird Fit Width geaefnet
	pdfpagemode=UseOutlines, % Bookmarks im Viewer anzeigen
	pdfpagelabels=true,           % set PDF page labels
	pdfpagelayout=OneColumn, % zweiseitige Darstellung: ungerade Seiten
										% rechts im PDF-Viewer
	%   pdfpagelayout=SinglePage, % einseitige Darstellung
]{hyperref}
\hypersetup{
    breaklinks,
    colorlinks,
    linkcolor=gray,
    citecolor=gray,
    urlcolor=gray,
    pdftitle={The Tensor Network Approach to Efficient and Explainable AI},
    pdfauthor={Alex Goessmann}
}
% PDF Informationen (hypersetup in main tex file)

% BOOKMARK
\usepackage{bookmark}
% Bold chapters
\bookmarksetup{
	addtohook={%
		\ifnum\bookmarkget{level}=0 %
			\bookmarksetup{bold}%
		\fi
	}
}

% Load marked packages
\LoadPackagesNow


%%% ====================================================
%%% --> General Macros <--

% If #2 is defined/non-empty, write #3, otherwise #1
\newcommand{\ifargdef}[3][{}]{\ifthenelse{\equal{#2}{}}{#1}{#3}}

% hanging paragraphs
\newlength{\hangind}
\newcommand{\myhangindent}[1]{\settowidth{\hangind}{\widthof{#1}}\hangindent=\the\hangind}
% \crefformat{equation}{(#2#1#3)}

%%% ====================================================
%%% --> KOMA-Fonts and Headings <--
% \newcommand\SectionFontStyle{\rmfamily\boldmath}

\setkomafont{sectioning}{\bfseries\rmfamily}
\setkomafont{part}{\usekomafont{sectioning}}
\setkomafont{chapter}{\huge\usekomafont{sectioning}}
\setkomafont{section}{\large\usekomafont{sectioning}}
\setkomafont{subsection}{\usekomafont{sectioning}}
\setkomafont{subsubsection}{\usekomafont{sectioning}}
\setkomafont{paragraph}{\usekomafont{sectioning}}
\setkomafont{subparagraph}{\usekomafont{sectioning}}

% \setkomafont{descriptionlabel}{\itshape}

% \setkomafont{caption}{\normalfont}
% \setkomafont{captionlabel}{\normalfont}
% 
% \setkomafont{dictum}{}
% \setkomafont{dictumauthor}{}
% \setkomafont{dictumtext}{}
% \setkomafont{disposition}{}
% \setkomafont{footnote}{}
% \setkomafont{footnotelabel}{}
% \setkomafont{footnotereference}{}
% \setkomafont{minisec}{}
% 
% \setkomafont{partnumber}{\bfseries\SectionFontStyle}
% \setkomafont{partentrynumber}{}
% \setkomafont{chapterentrypagenumber}{}
% \setkomafont{sectionentrypagenumber}{}

% \setkomafont{pageheadfoot}{\smaller\scshape\rmfamily}
\setkomafont{pageheadfoot}{\smaller\rmfamily}
% \setkomafont{pagenumber}{\bfseries\usekomafont{sectioning}}
% 
% \setkomafont{partentry}{\usekomafont{sectioning}\large}
% \setkomafont{chapterentry}{\usekomafont{sectioning}}
% \setkomafont{sectionentry}{\usekomafont{sectioning}}

% %%% Custom chapter format
% \makeatletter
% \renewcommand*\chapterformat{%
%   \if@chapterprefix
%     \smaller[2]{\chapapp~\thechapter}\vspace{-.5\baselineskip}
%   \else
%     \thechapter\autodot\enskip
%   \fi
% }
% \makeatother
% 
% % Add a line over chapter heading
% \newcommand*{\ORIGchapterheadstartvskip}{}%
% \let\ORIGchapterheadstartvskip=\chapterheadstartvskip
% \renewcommand*{\chapterheadstartvskip}{%
%   \ORIGchapterheadstartvskip
%   {%
% 	\hfuzz=11pt % suppress badbox
%     \setlength{\parskip}{0pt}%
% 	\noindent\rule[-.5\baselineskip]{\linewidth}{3pt}\par
%   }%
% }


\RedeclareSectionCommand[innerskip=-\baselineskip]{chapter}
\renewcommand*{\chapterformat}{
  \IfUsePrefixLine{
    \begin{tikzpicture}
%       \fill[black!60!gray] (0,0) rectangle (1,1);
%       \node[draw=black!60!gray, text=white, fill=black!60!gray,rectangle] (0.5,0.5) (title) {\large\normalfont\textbf{\chapapp~\thechapter}};
      \node[draw=black!60!gray, text=white, fill=black!60!gray,rectangle,inner sep=12] (0.5,0.5) (title) {\LARGE\normalfont\textbf{\thechapter}};
      \fill[black!60!gray] (title.south west) rectangle ($(title.south west) + (\textwidth,2pt)$);
    \end{tikzpicture}
  }{
   \mbox{\chapappifchapterprefix{\nobreakspace}
   \thechapter\autodot\enskip}
  }
}

% %%%%%%%%%%%%%%%%%%%%%%%%%%%%%%%%%%%%%%%%%%%%%%%%%%%%%%%%%%%%
% %
% %  Adding lines above and below the chapter head
% %
%  
% % 1st get a new command
% \newcommand*{\ORIGchapterheadstartvskip}{}%
% % 2nd save the original definition to the new command
% \let\ORIGchapterheadstartvskip=\chapterheadstartvskip
% % 3rd redefine the command using the saved original command
% \renewcommand*{\chapterheadstartvskip}{%
%   \ORIGchapterheadstartvskip
%   {%
%     \setlength{\parskip}{0pt}%
%     \noindent\rule[.3\baselineskip]{\linewidth}{1pt}\par
%   }%
% }
%  
% % see above
% \newcommand*{\ORIGchapterheadendvskip}{}%
% \let\ORIGchapterheadendvskip=\chapterheadendvskip
% \renewcommand*{\chapterheadendvskip}{%
%   {%
%     \setlength{\parskip}{0pt}%
%     \noindent\rule[.3\baselineskip]{\linewidth}{1pt}\par
%   }%
%   \ORIGchapterheadendvskip
% }
% %
% %  End of chapter head change
% %
% %%%%%%%%%%%%%%%%%%%%%%%%%%%%%%%%%%%%%%%%%%%%%%%%%%%%%%%%%%%%

% % --------------------------------------------------------------------------
% % Originalcode von: <http://www.komascript.de/fncychap-Sonny>
% % Copyright (c) Markus Kohm
% % Version: 2017-05-24
% % Changes:
% % - 2016-09-02 erste Version
% % - 2017-05-24 Anpassung von beforeskip an aktuelles KOMA-Script
% % Weitergabe und Verwendung gestattet, solange dieser Hinsweiskommentar
% % einschließlich Link und Copyrightinformation erhalten bleibt.
%  
% % 1. Emulation von fncychap mit KOMA-Script-Mitteln:
% \newlength{\ChapterRuleWidth}\setlength{\ChapterRuleWidth}{.5pt}
% \newcommand*{\ChRuleWidth}[1]{\setlength{\ChapterRuleWidth}{\dimexpr #1}}%
% \newcommand*{\ChNameVar}{\setkomafont{chapterprefix}}%
% \newcommand*{\ChTitleVar}{\setkomafont{chapter}}%
% \newcommand*{\ChNumVar}{\setkomafont{chapternumber}}%
% \newcommand*{\ChapterNameCase}[1]{#1}
% \newcommand*{\ChNameUpperCase}{\let\ChapterNameCase\MakeUppercase}
% \newcommand*{\ChNameIs}{\renewcommand*\ChapterNameCase[1]{##1}}
% \newcommand*{\ChNameLowerCase}{\let\ChapterNameCase\MakeLowercase}
% \newcommand*{\ChapterTitleCase}[1]{#1}
% \newcommand*{\ChTitleUpperCase}{\let\ChapterTitleCase\MakeUppercase}
% \newcommand*{\ChTitleIs}{\renewcommand*\ChapterTitleCase[1]{##1}}
% \newcommand*{\ChTitleLowerCase}{\let\ChapterTitleCase\MakeLowercase}
%  
% % 2. Einstellungen für den Stil Sonny:
% \KOMAoptions{chapterprefix}% Es ist ein Präfix-Stil
% \ChNameUpperCase
% \newkomafont{chapternumber}{\Huge}
% \let\raggedchapter\raggedleft% Überschriften rechtsbündig
% \RedeclareSectionCommand[%
%   beforeskip=-5\baselineskip,% Abstand über der Präfixzeile bzw. der Linie
%   innerskip=45pt,% Abstand zwischen Präfixzeile und Text
%   afterskip=40pt,% Abstand unter dem Text
%   font=\normalfont\sffamily\Large,% Schrift des Namens
%   prefixfont=\Large,% Schrift der Präfixzeile
% ]{chapter}
% \renewcommand*{\chapterformat}{%
%   \mbox{\ChapterNameCase{\chapappifchapterprefix{\nobreakspace}}%
%     {\usekomafont{chapternumber}{%
%         \rule{0pt}{.8\baselineskip}\thechapter\IfUsePrefixLine{}{\enskip}}}%
%   }%
% }
% \renewcommand*{\chapterlineswithprefixformat}[3]{% Ebene, Nummer, Text
%   \IfArgIsEmpty{#2}{}{%
%     % Die Prefix-Zeile aus Argument 2 wird nur gesetzt, wenn sie vorhanden
%     % ist.
%     #2%
%   }%
%   \rule[5pt]{\linewidth}{\ChapterRuleWidth}\\*
%   \ChapterTitleCase{#3}\nobreak
%   \rule[-5pt]{\linewidth}{\ChapterRuleWidth}
% }
% % --------------------------------------------------------------------------

%%% --> Titlesec Headers <--
% % Chapter
% \titleformat{\chapter}[block] % shape
% 	{\usekomafont{chapter}} % format
% 	{\rule{\textwidth}{4pt} \vspace{1ex} \thechapter} % label
% 	{1em}{}[] % sep
% \titleformat{name=\chapter,numberless}[block]
% 	{\usekomafont{chapter}}
% 	{\rule{\textwidth}{4pt} \vspace{1ex}}
% 	{0em}{}
% % Section
% \titleformat{\section}[block] % shape
% 	{\usekomafont{section}} % format
% 	{\rule{\textwidth}{1pt} \vspace{1ex} \thesection} % label
% 	{1em}{}[] % sep
% 
% % \titlespacing*{\chapter}{0pt}{*0}{0pt}
% \titlespacing{\section}{0pt}{*0}{0pt}