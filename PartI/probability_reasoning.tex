\section{\chatextprobReasoning}\label{cha:probReasoning}

We have investigated means to store the knowledge about a system and now turn to the retrieval of information, a process called inference.

% 
Contraction of the relational encoding of a function with a Markov Network gives the statistics over the values of the functions.
When contracting the function directly, we get the expectation.

% Message passing
%Another approximation comes from an approximation of the contractions itself. 
One can increase the efficiency of inference algorithms by using approximative contractions.
Here, message passing schemes can be applied as to be introduced in \charef{cha:messagePassing}.


\subsection{Queries}

% Motivation of queries: Avoid distribution instantiation
In the previous chapter, we have derived efficient representation schemes of probability distributions based on tensor network decompositions.
We have argued that one should avoid naive instantiation of these distributions based on an storage of each coordinates.
In the task of reasoning, we want to retrieve information encoded in the probability distribution.
To derive an efficient approach one therefore needs to avoid instantiating the distribution in a coordiantewise manner in an intermediate step.
We thus formalize a basic reasoning scheme by contractions of the decomposed distributions with query tensors.

\subsubsection{Querying by functions}

We can formalize queries by retrieving expectations of functions given a distribution specified by probability tensors. 
We exploit basis calculus in defining categorical variables $\catvariableof{\exfunction}$ to tensors $\exfunction$, which are enumerating the set $\imageof{\exfunction}$.
More details on this scheme are provided in \charef{cha:basisCalculus}, see \defref{def:functionRelationEncoding} therein.

\begin{definition}\label{def:queries}
	The marginal query of a probability distribution $\probat{\shortcatvariables}$ by a tensor
		\[ \exfunction : \facstates \rightarrow \rr \]
	is the vector $\probat{\catvariableof{\exfunction}} \in \rr^{\cardof{\imageof{\exfunction}}}$ defined as the contraction
	\begin{align*}
		\probat{\catvariableof{\exfunction}} = \contractionof{\probat{\shortcatvariables},\rencodingofat{\exfunction}{\shortcatvariables,\catvariableof{\exfunction}}}{\catvariableof{\exfunction}} \, .
	\end{align*}
	
	% Used in connection to mean parameters
	The expectation query of $\probtensor$ by $\exfunction$ is 
	\begin{align*}
		\expectationof{\exfunction} = \sbcontraction{\exfunction, \probtensor} \, . 
	\end{align*}
	
	% Used for sampling
	Given another tensor $\secexfunction: \facstates \rightarrow \rr $ the conditional query of the probability distribution $\probat{\shortcatvariables}$ by the tensor $\exfunction$ conditioned on the tensor $\secexfunction$ is the matrix $\condprobof{\catvariableof{\exfunction}}{\catvariableof{\secexfunction}}\in\rr^{\cardof{\imageof{\exfunction}}}\otimes \rr^{\cardof{\imageof{\secexfunction}}}$ defined as the normation
	\begin{align*}
		\condprobof{\catvariableof{\exfunction}}{\catvariableof{\secexfunction}} 
		= \normationofwrt{\{
		\probat{\shortcatvariables},\rencodingofat{\exfunction}{\shortcatvariables,\catvariableof{\exfunction}},\rencodingofat{\secexfunction}{\shortcatvariables,\catvariableof{\secexfunction}}
		\}}{
		\catvariableof{\exfunction}}{\catvariableof{\secexfunction}
		} \, . 
	\end{align*}
\end{definition}

%% Relation of queries and expectation queries
Expectation queries are contractions of marginal queries with identities, that is
	\[ \expectationof{\exfunction} = \sbcontraction{\probat{\catvariableof{\exfunction}} \idrestrictedto{\imageof{\exfunction}}{\catvariableof{\exfunction}} } \, . \]
This will be shown in more detail in \charef{cha:basisCalculus} in Corollary~\ref{cor:rhoToNormal}.

%% Conditional Probabilities and conditional queries
Conditional probabilities are queries, where the tensors $\exfunction$ and $\secexfunction$ are identity mappings in the respective variable state spaces.
Conversely, we can understand the conditional query $\condprobof{\exfunction}{\secexfunction}$ as the conditional probability of $\exfunction$ conditioned on $\secexfunction$, of the underlying Markov Network with cores $\{\probtensor, \rencodingof{\exfunction}, \rencodingof{\secexfunction} \}$ and variables $\catvariableof{\exfunction},\catvariableof{\secexfunction}$ besides the variables distributed by $\probtensor$.

%% Expectations as event queries -> Consistency with $\probat{X=i}$?
We further denote event queries by
	\[  \expectationof{\exfunction=z} = \sbcontraction{\probtensor,\rencodingof{\exfunction},\onehotmapof{z}}\]
where by $\onehotmapof{z}$ be denote the one hot encoding of the state $z$ with respect to some enumeration.
Let us note that they are further contraction of the queries in \defref{def:queries} since by \theref{the:splittingContractions}
\begin{align*}
	 \expectationof{\exfunction=z} 
	& =  \sbcontraction{ \sbcontractionof{\probtensor,\rencodingof{\exfunction}}{\catvariableof{\exfunction}} ,\onehotmapof{z}}\\
	& =  \sbcontraction{ \probat{\exfunction} ,\onehotmapof{z}} \, .
\end{align*}

%% OLD: Defining queries by 
%\begin{definition}
%	The expectation of functions $\exfunction$ given a probability tensor is the contraction
%		\[ \expectationofwrt{\exfunction(\catvariables)}{\catvariables\sim\probtensor} = 
%			\contractionof{\{\probtensor,\rencodingof{\exfunction}\}}{\{\exfunctiontargetvariables \}} \, . 
%		\]
%\end{definition}
%This is the canonical definition of expectations, since summing function values weighted by the probability of the argument.
%When we have an unnormalized probability distribution $\phi$ the expectation is the quotient
%\begin{align*}
%	\expectationofwrt{\exfunction(\catvariables)}{\catvariables\sim\phi}  = \frac{
%		\contractionof{\{\phi,\rencodingof{\exfunction}\}}{\{\exfunctiontargetvariables \}}
%	}{
%		\contractionof{\{\phi\}}{\varnothing} 
%	} \, . 
%\end{align*}

%\subsubsection{Conditional Probability Queries}
%
%Typical queries are the computation of an a posteriori distribution given evidence.
%This is just the contraction.
%
%%% As expectation
%The query consists of the one-hot encoding of the evidence and Ids elsewhere.
%The result is then interpreted as another probability distribution, defined as a Markov network and the possible need to normalize with the partition function.
%
%Given evidence, condition the probability tensor on that evidence.






\subsubsection{MAP Queries}

Find the maximal variable of a tensor is a problem, which can be approached by sampling methods as we discuss here.

\begin{definition}
	Given a tensor $\hypercore$ the MAP query is the problem 
	\begin{align}
		\argmax_{\catindices} \hypercoreat{\indexedcatvariables} \, .
	\end{align}
\end{definition}

%Often, the generation of a full (conditioned) probability tensor can be infeasible, if too many variables are queries.
%Having a tensor network decomposition of the probability tensor avoids this generation.

% One hot perspective
By coordinate calculus, we notice that
\begin{align}
	\hypercoreat{\indexedcatvariables} 
	\sbcontraction{\hypercore, \onehotmapof{\catindices}} \, .
\end{align}
Given the image $\Gamma^{\elformat}$ of one-hot encodings, the MAP query problem is equivalent to 
\begin{align}
	\max_{\catindices} \hypercoreat{\indexedcatvariables} 
	= \max_{\theta\in\Gamma^{\elformat}} \sbcontraction{\hypercore, \theta} \, .
\end{align}
We can thus understand MAP queries as a Tensor Network approximation problem, where the approximating tensor are the one-hot encodings of states.

\begin{remark}[MAP queries on energy and probability tensors]
% Usage on energies and probabilities
	Since the exponential function is monotonic, MAP queries on the energy tensor of an exponential family with uniform base measure are equivalent to MAP queries of their energies.
\end{remark}


\subsubsection{Answering queries by energy contractions}

Let us now interpret a probability tensor at hand as a member of an exponential family (see \secref{sec:exponentialFamilies}), which is always possible when taking the naive exponential family.

\begin{lemma}\label{lem:energyContractionQueries} % This is a statement about "full" queries.
	For any probability distribution $\probtensor$ with $\probtensor= \normationof{\expof{\energytensorat{\shortcatvariables}}}{\shortcatvariables}$, disjoint subsets $\nodesa,\nodesb \subset [\catorder]$ with $\nodesa\cup\nodesb=[\catorder]$  and any $\catindexof{\nodesb}$ we have
		\[ \condprobof{\catvariableof{\nodesa}}{\indexedcatvariableof{\nodesb}} 
			= \normationof{
				\expof{\energytensorat{\catvariableof{\nodesa},\indexedcatvariableof{\nodesb}}}
		}{\catvariableof{\nodesa}} \, .\]
\end{lemma}
\begin{proof}
	Since no summation is commuted.
\end{proof}

Thus, it suffices to build the selection encoding of the statistics, and we can avoid the usage of the relational encoding.

% 
We notice, that \lemref{lem:energyContractionQueries} does not generalize to situations, where $\nodesa\cup\nodesb\neq[\catorder]$, since summation over the indices of the variables $[\catorder]/\nodesa\cup\nodesb$ and contraction do not commute.
%\red{In that case, each summed index produces a factor.}


\begin{lemma}  %\red{TRUE?}
	For any probability distribution $\probtensor$ with $\probtensor= \normationof{\expof{\energytensorat{\shortcatvariables}}}{\shortcatvariables}$, disjoint subsets $\nodesa,\nodesb \subset [\catorder]$ and any $\catindexof{\nodesb}$ we have
		\[ \condprobof{\catvariableof{\nodesa}}{\indexedcatvariableof{\nodesb}} 
			=
			\normationof{
			 \sum_{\catindexofin{[\catorder]/\nodesa\cup\nodesb}} 
				 \expof{\energytensorat{\catvariableof{\nodesa},\indexedcatvariableof{\nodesb},\indexedcatvariableof{[\catorder]/\nodesa\cup\nodesb}}}
		}{\catvariableof{\nodesa}} \, .\]
\end{lemma}
\begin{proof}
	By splitting the contraction into terms to $\nodesa\cup\nodesb$. % and using \lemref{lem:energyContractionQueries}.
\end{proof}




\subsection{Sampling based on queries}


Let us here investigate how to draw samples from distributions $\probtensor$, based on queries on $\probtensor$.

%Need to generate the full conditional probability distribution by contraction and then sample from it.
Since there are $\prod_{\node\in\nodes}\catdimof{\node}$ coordinates stored in $\probtensor$, naive methods are often infeasible.
One can instead exploit a representation of $\probtensor$ by a Markov network or the energy term in an exponential family for efficient algorithms and sample from local proxy distributions resulting from contractions and interpreted as marginal and conditional probabilities.

\subsubsection{Exact Methods}

Forward Sampling (see Algorithm~\ref{alg:ForwardSampling}) uses a chain decomposition (see \theref{the:chainRule}) of a probability distribution to iteratively sample the variables.

\begin{algorithm}[hbt!]
\caption{Forward Sampling}\label{alg:ForwardSampling}
\begin{algorithmic}
\For{$\catenumeratorin$}
	\State Draw $\catindexof{\catenumerator}\in[\catdimof{\catenumerator}]$ from the conditional query
		\[ \condprobof{\catvariableof{\catenumerator}}{\indexedcatvariableof{\seccatenumerator} \, : \, \seccatenumerator < \catenumerator} \]
\EndFor
\end{algorithmic}
\end{algorithm}

%
Forward Sampling is especially efficient, when sampling from a Bayesian Network respecting the topological order of its nodes.
The reason for this lies in trivilizations of all conditional distributions, which heads are not included in the evidence of previously sampled variables.
More technically, we can show that
	\[ \condprobof{\catvariableof{\catenumerator}}{\indexedcatvariableof{\seccatenumerator} \, : \, \seccatenumerator < \catenumerator}  
	= \condprobof{\catvariableof{\catenumerator}}{\indexedcatvariableof{\parentsof{\catenumerator}}} \, , \]
which is only involving a single core of a Bayesian network.
\red{This can be shown using Corollary~\ref{cor:onesHead} to be derived in \charef{cha:basisCalculus}.}


%% Comment on rejection Sampling 
%When sampling from conditional probability distributions, one can sample from the conditioned distribution instead.
%However, the conditioning changes the structure of the distribution, and conditioned Bayesian Networks are not Bayesian Networks on the same graph.
%One ways around is rejection sampling, where one samples from the unconditioned distribution and rejects samples not satisfying the event conditioned on.
%When the event conditioned on is of small probability, methods like rejection sampling will come with large runtimes.

\subsubsection{Approximate Methods}

% Problem of many variables
When there are many variables to be sample, the computation of the conditional probability to all variables can be infeasible.
One way to overcome this is Gibbs Sampling: Iteratively resemble single variables given the rest as evidence.

%\subsubsection{Gibbs Sampling}

% Still old: Sample from Marginal
Sample each variable independent from the marginal distribution.
Then, alternate through the variables and sample each variable from the conditional distribution taking the others as evidence.

\begin{algorithm}[hbt!]
\caption{Gibbs Sampling}\label{alg:Gibbs}
\begin{algorithmic}
\For{$\catenumeratorin$}
	\State Draw State for atom $\catenumerator$ from initialization distributions. % In implementation: Initialize with ones and draw -> Avoids zero probability state
\EndFor
\While{Stopping criterion is not met}
\For{$\catenumeratorin$}
	\State Draw $\catindexof{\catenumerator}\in[\catdimof{\catenumerator}]$ from the conditional query
		\[ \condprobof{\catvariableof{\catenumerator}}{\indexedcatvariableof{\seccatenumerator}\, : \seccatenumerator \neq \catenumerator} \]
\EndFor
\EndWhile
\end{algorithmic}
\end{algorithm}


% Energy

Gibbs can be implemented based on the energy tensor $\energytensor$ of the probability tensor, as follows form the \lemref{lem:energyContractionQueries}.



%	\[ \condprobof{\catvariableof{\catenumerator}}{\{\catvariableof{\seccatenumerator}=\catindexof{\seccatenumerator} \, : \seccatenumerator \neq \catenumerator\}} 
%	= \normationofwrt{\expof{\contractionof{\{\energytensor\}\cup\{\onehotmapof{\catindexof{\seccatenumerator}} \, : \seccatenumerator \neq \catenumerator \}}{\catvariableof{\catenumerator}}}}{\catvariableof{\catenumerator}}{\varnothing}  \, .\]
	


\red{This is in contrast with forward sampling, where we need to sum over many coordinates of the exponentiated energy tensor, which amounts to the representation of the probability distribution as a tensor network using relational encodings.
}
%where the operation with energy tensors and selection encodings is not efficient.}




\subsubsection{Simulated Annealing}

\red{MAP queries are approximated by sampling from annealed distributions: Use $\hypercore$ as the energy tensor, e.g. as parameter tensor to the naive exponential family.}

%\begin{remark}\label{rem:simulatedAnnealing}
% Simulated annealing
	\red{Here by the naive exponential family!}
	Simulated annealing manipulates the probability used to sample $\catindexof{\catenumerator}$ in terms of an inverse temperature parameter $\invtemp$, by
		\[ \probtensor \rightarrow \frac{\expof{\invtemp\cdot\lnof{\probtensor}}}{\contraction{\expof{\invtemp\cdot\lnof{\probtensor}}} } \, . \]
	When the temperature is larger than $1$, the probability of states with low probability increases while the probability of states with large probability decreases and for low temperatures the opposite.
	Simulated annealing, that is the decrease of the temperature to $0$ during Gibbs sampling biases the algorithm towards states with large probability.
%	Tuning this parameter can improve the convergence of Gibbs Sampling.

	% On exponential families
	For any exponential family the transformation 
		\[ \energytensor \rightarrow \invtemp \cdot \energytensor  \]
	can be performed by rescaling the canonical parameters as
		\[ \canparam \rightarrow \invtemp \cdot \canparam \, . \]
%\end{remark}







\subsection{Maximum Likelihood Estimation} % Stuff from Parameter Estimation - Problem that Part I is called inference?

Let us now turn to inductive reasoning tasks, where a probabilistic model is trained on given data.

\subsubsection{Likelihood and Loss}

Given a datapoint $\datamapof{\datindex}$ consisting of the images of the data selecting map $\datamap$ (see \defref{def:dataMap}), the likelihood given a Markov Logic Network is denoted as
	\[ \probat{\shortcatvariables = \datamapof{\datindex}} \, . \]
	
% Independent assumption
When all $\datamapof{\datindex}$ are drawn independently from $\probat{\shortcatvariablelist}$, we can factorize into
	\[ \probat{\data}  = \prod_{\datindexin} \probat{\shortcatvariables=\datamapof{\datindex}} \, . \]

% Logarithm
It is convenient to apply a logarithm on the objective, which does not influence the optimum when optimizing this quantity.
This is especially useful, when investigating the convergence of the objective for $\datanum\rightarrow\infty$ (see \charef{cha:concentration}).

\begin{definition}\label{def:loss}
	We define the loss of a distribution $\probtensor$ as
	\begin{align*}%\label{eq:defLikelihoodLossPL}
		\lossof{\probtensor} 
		= \frac{1}{\datanum} \lnof{\probat{\data}}
	\end{align*}
\end{definition}

We now state the Maximum Likelihood Problem in the form
\begin{align}\tag{$\probtagtypeinst{\loss}{\Gamma,\empdistribution}$}\label{prob:parameterMaxLikelihood}
	\argmin_{\probtensor\in\Gamma} \lossof{\probtensor} \, . % Naive Exponential Family perspective!
\end{align}



\subsubsection{Entropic Interpretation}



\begin{definition}[Shannon entropy]
	The information content or the Shannon entropy of a distribution is defined as
		\[ \sentropyof{\probtensor} 
		:= \expectationofwrt{-\lnof{\probat{\shortcatvariables}}}{\shortcatvariables\sim\probtensor}
		= \sbcontraction{\probtensor,-\lnof{\probtensor}} \, . \]
	%	= - \sum_{\shortcatindices} \probat{\indexedshortcatvariables} \cdot \lnof{\probat{\indexedshortcatvariables}} \, . \]
	We depict this in a tensor network diagram with an ellipsis denoting a coordinatewise transform (see \charef{cha:coordinateCalculus}) with a natural logarithm $\ln$ as:
	\begin{center}
		\begin{tikzpicture}[scale=0.3,thick] % , baseline = -3.5pt

\node[anchor=center] (text) at (-8,-5) {\corelabelsize $\sentropyof{\probtensor}$};

\node[anchor=center] (text) at (-5,-5) {\corelabelsize ${=}$};

\node[anchor=center] (text) at (-3,-2) {\corelabelsize $\mathrm{ln}$};
\draw (2,-2) ellipse (6 and 2.75);

\draw (-1,-1) rectangle (5,-3);
\node[anchor=center] (text) at (2,-2) {\corelabelsize $\probtensor$};
\draw (-1,-7) rectangle (5,-9);
\node[anchor=center] (text) at (2,-8) {\corelabelsize $\probtensor$};
\draw (0,-5)--(0,-3); 
\draw (0,-5)--(0,-7) node[midway,left] {\colorlabelsize $\catvariableof{0}$};
\draw (1.5,-5)--(1.5,-3); 
\draw (1.5,-5)--(1.5,-7) node[midway,left] {\colorlabelsize $\catvariableof{1}$};
\node[anchor=center] (text) at (3,-4) {$\cdots$};
\draw (4,-5)--(4,-3);
\node[anchor=center] (text) at (3,-6) {$\cdots$};
\draw (4,-5)--(4,-7) node[midway,right] {\colorlabelsize $\catvariableof{\catorder\shortminus1}$};

%\drawatomcore{3.5}{-8}{$\probtensor$}
%\drawatomindices{3.5}{-12}	
%\draw (5.5,-9)--(5.5,-7) node[midway,right] {\colorlabelsize $\catvariableof{\exformula}$};

\end{tikzpicture}
	\end{center}
\end{definition}

\begin{definition}[Cross entropy]\label{def:crossEntropy}
	The cross entropy between two distributions is defined as 
		\[ \centropyof{\probtensor}{\secprobtensor} 
		:=  \expectationofwrt{-\lnof{\secprobtensor[\shortcatvariables]}}{\shortcatvariables\sim\probtensor} 
		= \sbcontraction{\probtensor,-\lnof{\secprobtensor}} \, . \]
		%- \sum_{\catindices}  \probat{\indexedcatvariables} \cdot \lnof{\secprobtensor[\indexedshortcatvariables]}  \, . \]
	We depict this in a tensor network diagram with an ellipsis denoting a coordinatewise transform (here the $\ln$) as :
	\begin{center}
		\begin{tikzpicture}[scale=0.3,thick] % , baseline = -3.5pt

\node[anchor=center] (text) at (-8,-5) {\corelabelsize $\centropyof{\probtensor}{\tilde{\probtensor}}$};

\node[anchor=center] (text) at (-5,-5) {\corelabelsize ${=}$};

\node[anchor=center] (text) at (-3,-2) {\corelabelsize $\mathrm{ln}$};
\draw (2,-2) ellipse (6 and 2.75);

\draw (-1,-1) rectangle (5,-3);
\node[anchor=center] (text) at (2,-2) {\corelabelsize $\tilde{\probtensor}$};
\draw (-1,-7) rectangle (5,-9);
\node[anchor=center] (text) at (2,-8) {\corelabelsize $\probtensor$};
\draw (0,-5)--(0,-3); 
\draw (0,-5)--(0,-7) node[midway,left] {\colorlabelsize $\catvariableof{0}$};
\draw (1.5,-5)--(1.5,-3); 
\draw (1.5,-5)--(1.5,-7) node[midway,left] {\colorlabelsize $\catvariableof{1}$};
\node[anchor=center] (text) at (3,-4) {$\cdots$};
\draw (4,-5)--(4,-3);
\node[anchor=center] (text) at (3,-6) {$\cdots$};
\draw (4,-5)--(4,-7) node[midway,right] {\colorlabelsize $\catvariableof{\catorder\shortminus1}$};

%\drawatomcore{3.5}{-8}{$\probtensor$}
%\drawatomindices{3.5}{-12}	
%\draw (5.5,-9)--(5.5,-7) node[midway,right] {\colorlabelsize $\atomlegindexof{\exformula}$};

\end{tikzpicture}
	\end{center}
\end{definition}

%% Vanishing coordinates case
We here use $\lnof{0}=-\infty$ and $0\cdot \lnof{0} = 0$. 
Then we have $\centropyof{\probtensor}{\secprobtensor} = \infty$ if and only if there is a $\shortcatindices$ such that $\probat{\indexedshortcatvariables}>0$ and $\secprobtensor[\indexedshortcatvariables]=0$.


% KL Divergence
The Gibbs inequality states that
		\[ \centropyof{\probtensor}{\secprobtensor} \geq \sentropyof{\probtensor} \, . \]
The difference between both sides is called the Kullback Leibler Divergence and a useful metric in reasoning, since it vanishes for $\probtensor=\secprobtensor$.

\begin{definition}[Kullback Leibler Divergence]\label{def:KLDivergence}
	The KL divergence between two distributions is defined as 
		\[ \kldivof{\probtensor}{\secprobtensor} = \centropyof{\probtensor}{\secprobtensor} - \sentropyof{\probtensor}  \, . \]
\end{definition}

We are now ready to provide an entropic interpretation of the loss introduced in \defref{def:loss}.

\begin{theorem}\label{the:lossCentropy}
	Given a data selecting map $\datamap$ and a distribution $\probtensor$ we have
	\begin{align}
		\lossof{\probtensor} =  \centropyof{\empdistribution}{\probtensor} \, . % \sbcontraction{\empdistribution,\lnof{\probtensor}} \, . 
	\end{align}
\end{theorem}
\begin{proof}
	We have
	\begin{align*}
		\lossof{\probtensor} 
		& = \frac{1}{\datanum} \lnof{\probat{\data}}
		= \frac{1}{\datanum} \sum_{\datain} \lnof{\probat{\shortcatvariables =\datamap(\datindex)}}
		= \frac{1}{\datanum} \sum_{\datain} \contraction{\{\lnof{\probtensor},\onehotmapof{\datamap(\datindex)}\}} \\ 
		& = \sbcontraction{\empdistribution,\lnof{\probtensor}} \, .
	\end{align*}
	Comparing with the negative log likelihood we notice that that loss coincides with the cross-entropy between the empirical distribution $\empdistribution$ and $\probtensor$, i.e.
		\[ \lossof{\probtensor} = \centropyof{\empdistribution}{\probtensor} \, . \]
\end{proof}


% Interpretation of MLE as Cross-Entropy Minimization

We can therefore rewrite Problem~\ref{prob:parameterMaxLikelihood} as minimization of cross-entropies and of Kullback Leibler divergences as
\begin{align*}
	\argmin_{\probtensor\in\Gamma} \lossof{\probtensor} 
	= \argmin_{\probtensor\in\Gamma} \centropyof{\empdistribution}{\probtensor} 
	= \argmin_{\probtensor\in\Gamma} \kldivof{\empdistribution}{\probtensor} \, .
\end{align*}
	


% M-Projection -> A projection since P^2 = P, i.e. P applied on the image is id
Most general, the Maximum Likelihood Problem is the M-Projection of a distribution $\gendistribution$ onto a set $\Gamma$ of probability tensors is
\begin{align}\tag{$\mathrm{P}_{\Gamma, \gendistribution}$}\label{prob:mProjection}
	\argmax_{\probtensor\in\Gamma} \centropyof{\gendistribution}{\probtensor} 
\end{align}
where the Maximum Likelihood Estimation is the special case $\gendistribution=\empdistribution$.


\begin{example}[Cross entropy with respect to exponential families]\label{exa:cEntropyExp}
	If $\secprobtensor$ from an exponential family with boolean base measure, have with the representation from \lemref{lem:energyCumulantRepresentation}
	\begin{align*}
		\centropyof{\probtensor}{\expdist} 
		= \sbcontraction{\probtensor,\lnof{\expdist}} 
		= \sbcontraction{\probtensor,\sencsstat} - \cumfunctionof{\canparam} + \sbcontraction{\probtensor,\lnof{\basemeasure}} \, . 
	\end{align*}
	For the trivial base measure we can further exploit the existence of the energy tensor and have the representation
		\[ \centropyof{\probtensor}{\expdist} = \sbcontraction{\probtensor,(\expenergy-\cumfunctionof{\canparam}\cdot \ones)}
		=   \sbcontraction{\probtensor,\expenergy} -\cumfunctionof{\canparam} \, .   \]
\end{example}




\subsection{Forward Mapping in Exponential Families} 


%\red{Integrate: 
%Selection encodings suffice for variational methods, relational encodings of statistics are required for markov network instantiations of exponential families.}


%% Mean parameters are expectation queries
Mean parameter coordinates are expectation queries to $\sstatcoordinateof{\selindex}$, by 
	\[ \meanparamat{\indexedselvariable} = \expectationof{\sstatcoordinateof{\selindex}} \, . \]
	
%% Forward mappings are contractions, variational formulation as an alternative to avoid inefficiencies
Forward mappings have a closed form representation by
	\[ \forwardmapof{\canparam}
	= \sbcontractionof{\sencodingof{\sstat},\normationof{\basemeasure,\expof{\contraction{\sencodingof{\sstat},\canparam}}}{\shortcatvariables}}{\selvariable} \, . \]
% Infeasibility and turn to variational alternatives with selection encodings.
This contraction can, however, be infeasible, since it requires the instantiation of the probability tensor, which can be done by basis encodings of the statistic.
We in this section provide alternative characterization of the forward map and approximations of it, which can be computed based on the selection encoding instead.
Following \cite{wainwright_graphical_2008}, we can characterize the forward mapping to exponential families as a variational problem and provide an alternative characterization to this contraction.



\subsubsection{Variational Formulation}

Besides the direct computation of the mean parameter tensor we can give a variational characterization of the forward mapping.
This is especially useful, when the contraction is intractable, for example because the tensor $\expdist$ is infeasible to create.

\begin{theorem}
	We have
	\begin{align*}
		\forwardmapof{\canparam}
		  = \argmax_{\meanparam\in\genmeanset}  \sbcontraction{\meanparam,\canparam} + \sentropyof{\meanrepprob} 
	\end{align*}
	where by $\meanrepprob$ we denote a probability distribution with respect to a base measure $\basemeasure$, which reproduces the mean parameter $\meanparam$.
\end{theorem}
\begin{proof}
	Theorem~3.4 in \cite{wainwright_graphical_2008}.
\end{proof}

Let us now characterize the image of the forward map, which turns out to be the interior of the mean polytope, if the statistic is minimal (see \defref{def:minimalStatistics}).

\begin{theorem}\label{the:meanPolytopeInterior}
	For any statistics $\sstat$, which is minimal with respect to a base measure $\basemeasure$, the image $\imageof{\forwardmap}$ of the forward map is the interior of the convex polytope $\genmeanset$.
\end{theorem}
\begin{proof}
	Theorem 3.3 in \cite{wainwright_graphical_2008}.
\end{proof}

For the practicle usage of this theorem, we need a characterization of the interior of $\genmeanset$.

\begin{theorem}\label{the:meanPolytopeInteriorCharacterization}
	For any minimal statistics $\sstat$ and boolean base measure $\basemeasure$ we have for some $\meanparamat{\selvariable}$ that $\meanparamat{\selvariable}\in\genmeanset$ if and only if there is a positive distribution with respect to $\basemeasure$ such that
		\[ \meanparamat{\selvariable} = \sbcontractionof{\probtensor,\sencsstat}{\selvariable} \, . \]
%	If $\meanparamat{\selvariable}$ is in an minimal exponential family with boolean base measure $\basemeasure$, then it is in the interior of $\genmeanset$ if and only if it is representable by a positive distribution with respect to $\basemeasure$.
\end{theorem}
\begin{proof} 
	\proofrightsymbol: 
		By \theref{the:meanPolytopeInterior} we find a canonical parameter $\canparamat{\selvariable}$ such that
		\begin{align*}
			\meanparamat{\selvariable} = \sbcontractionof{\expdistat{\shortcatvariables},\sencsstatwith}{\selvariable} \, .
		\end{align*}
		We notice, that $\expdist$ is positive with respect to $\basemeasure$, as is any member of an exponential family with base measure $\basemeasure$.
		
	\proofleftsymbol: % Orient on proof of Theorem~3.3 
		Since by assumption the statistics is minimal, the convex set $\genmeanset$ is full dimensional (see e.g. Appendix B in \cite{wainwright_graphical_2008}). 
		We thus use a well-known property for full-dimensional convex sets (see \cite{rockafellar_convex_1997,hiriart-urruty_convex_1993}), that $\meanparam\in\interiorof{\genmeanset}$ if for any non-vanishing vector $\vectorat{\selvariable}$ there is a  % citations from Wainwright - Appendix B	
		there is a $\tilde{\meanparam}[\selvariable]$ with
			\[ \contraction{\vectorat{\selvariable},\meanparamat{\selvariable}} <  \contraction{\vectorat{\selvariable},\tilde{\meanparam}[\selvariable]} \, . \]
		It thus suffices to show for an arbitrary non-vanishing vector $\vectorat{\selvariable}$ the existence of a distribution $\tilde{\probtensor}$, such that
		\begin{align*}
			\contraction{\vectorat{\selvariable},\meanparamat{\selvariable}} < \contraction{\vectorat{\selvariable},\sencsstatwith,\secprobat{\shortcatvariables}} \, .
		\end{align*}
		We define for $\epsilon\in\rr$
		\begin{align*}
			\probofat{\epsilon}{\shortcatvariables} 
			= \sbnormationof{\probat{\shortcatvariables},\expof{\epsilon\cdot\contractionof{\sencsstatwith,\vectorat{\selvariable}}{\shortcatvariables}}}{\shortcatvariables}
		\end{align*}
		The derivation of this map at $\epsilon=0$ is 
		\begin{align*}
			\difwrt{\epsilon}\probofat{\epsilon}{\shortcatvariables}|_{\epsilon=0}
			= \contractionof{\probwith,\sencsstatwith,\vectorat{\selvariable}}{\shortcatvariables} - \contraction{\probwith,\sencsstatwith,\vectorat{\selvariable}} \cdot \probwith 
		\end{align*}
		and thus
		\begin{align*}
			\difwrt{\epsilon} \contraction{\probofat{\epsilon}{\shortcatvariables},\sencsstatwith,\vectorat{\selvariable}}|_{\epsilon=0}
			&= \contractionof{\probwith,(\contractionof{\sencsstatwith,\vectorat{\selvariable}})^2}{\shortcatvariables} \\
			 & \quad - \left(\contractionof{\probwith,\sencsstatwith,\vectorat{\selvariable}}{\shortcatvariables}\right)^2 \, . 
		\end{align*}
		We can interpret this quantity as the variance of the random variable $\contractionof{\sencsstatwith,\vectorat{\selvariable}}{\indexedshortcatvariables}$, where $\shortcatindices$ is drawn from $\probwith$.
		The variance is greater than zero, if this random variable is not constant.
		But from the minimality of $\sstat$ with respect to $\basemeasure$ it follows, that this variable is not constant and we therefore have
		\begin{align*}
			0 < \difwrt{\epsilon} \contraction{\probofat{\epsilon}{\shortcatvariables},\sencsstatwith,\vectorat{\selvariable}}|_{\epsilon=0} \, . 
		\end{align*}
		Thus, there is a $\epsilon>0$ with 
		\begin{align*}
			\contraction{\vectorat{\selvariable},\meanparamat{\selvariable}} < \contraction{\vectorat{\selvariable},\sencsstatwith,\probofat{\epsilon}{\shortcatvariables}} \, .
		\end{align*}

\end{proof}


\subsubsection{Boundary of convex polytopes}

For mean parameters $\meanparamat{\selvariable}$ outside the interior of $\genmeanset$ we know by \theref{the:meanPolytopeInteriorCharacterization}, that any distribution with mean parameter $\meanparamat{\selvariable}$ is not positive with respect to $\basemeasure$ and is therefore not in the exponential family.
We investigate this situation further and provide here a construction scheme to adapt the base measure such that there are exponential families containing these boundary distributions.

\begin{theorem}\label{the:faceToArgmax}
	Let there be a minimal $\sstat$ with respect to the base measure $\basemeasure$ and $\meanparamat{\selvariable}\notin\interiorof{\genmeanset}$.
	Then there is a $\canparamat{\selvariable}$ with 
		\[ \meanparamat{\selvariable} \in \argmax_{\meanparam\in\genmeanset} \contraction{\canparamat{\selvariable},\meanparamat{\selvariable}} \,  \]
	and all distributions with mean parameter $\meanparamat{\selvariable}$ are representable with respect to the base measure
		\[ \secbasemeasureat{\shortcatvariables} = \contractionof{\basemeasure, \indicatorofat{\arbset}{\shortcatvariables}}{\shortcatvariables} \, , \]
	where the indicator is on the set
		\[ \arbset = \argmax_{\shortcatindices} \contraction{\canparam,\sstat(\shortcatindices)}  \, . \]
\end{theorem}
\begin{proof}
	When $\meanparam\notin\interiorof{\genmeanset}$ we find a face such that $\meanparam\in\genfacesetof{\facecondset}$.
	The existence of $\canparamat{\selvariable}$ follows from \theref{the:faceNormal}, in which also a construction procedure is provided given a half-space representation (see \theref{the:meanPolytopeHalfspaces}).
	
	Now, we have 
	\begin{align*}
		 \meanparamat{\selvariable} \in \argmax_{\meanparam\in\genmeanset} \contraction{\canparamat{\selvariable},\meanparamat{\selvariable}} 
	\end{align*}
	and thus 
	\begin{align*}
		 \meanparamat{\selvariable} \in \convhullof{ \sencsstat{\indexedshortcatvariables,\selvariable} \, : \, 
		 \shortcatindices \in \argmax_{\shortcatindices \, : \, \basemeasureat{\indexedshortcatvariables}=1} \contraction{\canparamat{\selvariable},\sencsstat{\indexedshortcatvariables,\selvariable}} }
	\end{align*}	
	Thus, any distribution reproducing meanparam is a convex combination of the one-hot encodings of the states in $\argmax_{\shortcatindices} \contraction{\canparamat{\selvariable},\sencsstat{\indexedshortcatvariables,\selvariable}}$, and therefore representable with respect to the base measure $\secbasemeasure$.
\end{proof}

Each face of $\genmeanset$ thus defines a refinement of a base measure, which is sufficient to reproduce the mean parameters on that face.

\begin{definition}\label{def:faceBaseMeasure}
	The base measure to the face of $\meanset$ with normal $\canparam$ is
		\[ \basemeasureof{\sstat,\canparam} = \indicatorofat{\argmax_{\shortcatindices} \contraction{\canparam,\sstat(\shortcatindices)}}{\shortcatvariables} \, . \]
\end{definition}

\theref{the:faceToArgmax} therefore states, that when a mean parameter is on a face of $\genmeanset$, then each distribution reproducing the mean parameter has a representation with respect to the refined base measure
\begin{align*}
	\secbasemeasureat{\shortcatvariables} = \contractionof{\basemeasure,\basemeasureof{\sstat,\canparam}}{\shortcatvariables} \, . 
\end{align*}

% Base Measure Refinement algorithm
We now utilize these findings and provide in \algoref{alg:baseMeasureRefinement} a procedure to refine the base measure until the reduced mean parameter is in the open set of a reduced mean parameter polytope.

\begin{algorithm}[h!]
\caption{Base Measure Refinement}\label{alg:baseMeasureRefinement}
\begin{algorithmic}
\State \textbf{Input}: Base measure $\basemeasure$, statistic $\sstat$ and mean parameter $\meanparam\in\genmeanset$
\State \textbf{Output}: Refined base measure $\secbasemeasure$, remaining statistic $\secsstat$ and remaining mean parameter $\secmeanparam$
\hrule
%\State \noindent\rule{\linewidth}{0.4pt}
\While{$\meanparam\notin\sbinteriorof{\genmeanset}$}
	\While{$\sstat$ not minimal with respect to $\basemeasure$ (see \defref{def:minimalStatistics})}
		\State Find non-vanishing vector $\vectorat{\selvariable}$ and scalar $\lambda\in\rr$ such that 
			\[ \contractionof{\sencsstatat{\shortcatvariables,\selvariable},\vectorat{\selvariable},\basemeasureat{\shortcatvariables}}{\shortcatvariables} = \lambda\cdot\basemeasureat{\shortcatvariables} \, . \]
		\State Choose a coordinate $\selindexin$ with $\vectorat{\indexedselvariable}\neq0$ and drop it from $\sstat$ and $\meanparam$
	\EndWhile
	\State Find a non-trivial face (i.e. a non-empty face, which is a proper subset of $\genmeanset$) with normal $\canparam$, such that
		\[ \meanparam\in\genfacesetof{\canparam} \]
	\State Refine base measure
		\[ \basemeasure \algdefsymbol \contractionof{\basemeasure,\basemeasureof{\sstat,\canparam}}{\shortcatvariables} \]
\EndWhile
\State \textbf{return} $\basemeasure, \, \sstat,\,\meanparam$
\end{algorithmic}
\end{algorithm}

\begin{theorem}\label{the:baseMeasureRefinement}
	For arbitrary inputs $\basemeasure,\sstat$ and $\meanparam\in\genmeanset$, \algoref{alg:baseMeasureRefinement} terminates in finite time and outputs a triple of base measure $\secbasemeasure$, statistic $\secsstat$ and mean parameter $\secmeanparam$ such that the following holds.
	Any probability tensor $\probtensor$ reproducing $\meanparam$ is representable with respect to $\secbasemeasure$ and $\secmeanparam\in\sbinteriorof{\meansetof{\secsstat,\secbasemeasure}}$.
%	Thus, there is a member of the exponential family $\expfamilyof{\secsstat,\secbasemeasure}$ reproducing $\meanparam$.
\end{theorem}
\begin{proof}
	Let us first show, that \algoref{alg:baseMeasureRefinement} always terminates.
	The inner while loop of \algoref{alg:baseMeasureRefinement} always terminates, since $\sstat$ has a finite number of coordinates, and in each iteration one of the coordinates is dropped.
	To show that the outer while loop also terminates, it suffices to show, that the non-vanishing coordinates of the refined base measure are a proper subset of the base measure before refinement.
	But if this would not be the case, we would have 
		\[ \basemeasureat{\shortcatvariables} = \contractionof{\basemeasure,\basemeasureof{\sstat,\canparam}}{\shortcatvariables} \]
	and thus $\genfacesetof{\canparam}=\genmeanset$, which is a contradiction with the assumption of a non-trivial face.
	
	The second claim follows from an iterative application of \theref{the:faceToArgmax} and the fact, that a probability distribution reproduces $\meanparam$ in a non-minimal representation, if and only if it reproduces the corresponding reduced $\meanparam$ with respect to the reduced statistics.
\end{proof}


\begin{example}[Faces with normals parallel to one-hot encodings]
	To get some intuition how to represent face base measures, let us consider face normals $\canparam\in\{\lambda\cdot\onehotmapofat{\selindex}{\selvariable} \, : \, \selindexin, \, \lambda\in\rr/\{0\}\}$.
	We use relational encodings of the coordinates $\sstatcoordinateof{\selindex}$ of the statistic $\sstat$, with head variables $\catvariableof{\sstatcoordinateof{\selindex}}$ with dimension $\catdimof{\sstatcoordinateof{\selindex}}$ enumerating the image $\imageof{\sstatcoordinateof{\selindex}}\subset\rr$ in an ascending order.
	If $\canparamat{\selvariable}=\lambda\cdot\onehotmapofat{\selindex}{\selvariable}$ with $\lambda>0$, then $\argmax_{\shortcatindices} \contraction{\canparam,\sstat(\shortcatindices)}$ consists of states $\shortcatindices$ with minimal statistic $\sstatcoordinateofat{\selindex}{\indexedshortcatvariables}$, that is
		\[  \basemeasureofat{\sstat,\lambda\cdot\onehotmapof{\selindex}}{\shortcatvariables}
		 = \contractionof{\rencodingofat{\sstatcoordinateof{\selindex}}{\shortcatvariables,\catvariableof{\sstatcoordinateof{\selindex}}},
		 \onehotmapofat{\catdimof{\sstatcoordinateof{\selindex}}-1}{\catvariableof{\sstatcoordinateof{\selindex}}}}{\shortcatvariables}  \, . \]		
	If $\canparamat{\selvariable}=\lambda\cdot\onehotmapofat{\selindex}{\selvariable}$ with $\lambda<0$, then at the states with minimal statistic $\sstatcoordinateofat{\selindex}{\indexedshortcatvariables}$, that is
		\[  \basemeasureofat{\sstat,\lambda\cdot\onehotmapof{\selindex}}{\shortcatvariables}
		 = \contractionof{\rencodingofat{\sstatcoordinateof{\selindex}}{\shortcatvariables,\catvariableof{\sstatcoordinateof{\selindex}}},
		 \onehotmapofat{0}{\catvariableof{\sstatcoordinateof{\selindex}}}}{\shortcatvariables}  \, . \]
\end{example}


% Define sets of realizable distributions
\begin{theorem}
	For the maximal graph $\maxgraph=([\seldim],\{[\seldim]\})$, which has a single hyperedge containing all head variables we have
	\begin{align*}
		\genmeanset = \left\{ \contractionof{\probat{\shortcatvariables},\sencodingofat{\sstat}{\shortcatvariables,\selvariable}}{\shortcatvariables} \, , \, \probtensor \in \realizabledistsof{\sstat,\maxgraph} \right\}
	\end{align*}
\end{theorem}
\begin{proof}
	It is enough show, that for any output tuples $\secbasemeasure$, $\secsstat$ of the Base Measure Refinement \algoref{alg:baseMeasureRefinement} we have
		\[ \expfamilyof{\secbasemeasure,\secsstat} \subset  \realizabledistsof{\sstat,\maxgraph} \, . \]
	We notice, that the normation of any face base measure is realizable by $\realizabledistsof{\sstat,\maxgraph}$, since the objective in the maximation problem in \defref{def:faceBaseMeasure} depends only on $\sstat$.
	Providing a more technical argument, we have
	\begin{align*}
		\indicatorofat{\argmax_{\shortcatindices} \contraction{\canparam,\sstat(\shortcatindices)}}{\shortcatvariables}
		= \contractionof{
			\sstatcc,
			\sum_{\sstat(\shortcatindices) \, : \, \shortcatindices \in \argmax_{\shortcatindices} \contraction{\canparam,\sstat(\shortcatindices)} }
			\onehotmapofat{\indexinterpretationat{\sstat(\shortcatindices)}}{\sstatheadvariables}
		}{\shortcatvariables} \, .
	\end{align*}
	Since during the execution of \algoref{alg:baseMeasureRefinement}, $\secsstat$ is a subset of $\sstat$, we can find a corresponding $\canparamof{i}$ extending the face normal by vanishing coordinates to $\sstat$.
	We then have, that 
	\begin{align*}
		\secbasemeasure = \contractionof{
			\{\rencodingofat{\sstat}{\sstatheadvariables,\shortcatvariables}\} \cup
			\left\{\sum_{\sstat(\shortcatindices) \, : \, \shortcatindices \in \argmax_{\shortcatindices} \contraction{\canparamof{i},\sstat(\shortcatindices)} } 
			\onehotmapofat{\indexinterpretationat{\sstat(\shortcatindices)}}{\sstatheadvariables}
			: i \in [n] \right\}
			}{\sstatheadvariables} 
	\end{align*}
	represents the output base measure, where $i\in[n]$ label the faces chosen during in the loop of \algoref{alg:baseMeasureRefinement}.
	Now, any member $\expdistof{\secbasemeasure,\canparam,\secsstat}\in\expfamilyof{\secsstat,\secbasemeasure}$ can be represented by a member of  $\realizabledistsof{\sstat,\maxgraph}$, by contracting these base measure representing cores with the activation cores $\bigotimes_{\selindexin}\sstatacwith$.
\end{proof}

\subsubsection{Mode Search by annealing}

%% ANNEALING
Finding the mode of a distribution is related to the forward mapping of $\invtemp\cdot\canparam$: 
$\meanparam$ to a delta distribution (or in the convex hull of multiple maxima) in the limit.

% Annealing effect on the optimization problem
This is because 
\begin{align*}
	\argmax_{\meanparam\in\genmeanset}  \sbcontraction{\meanparam,\canparam}
\end{align*}
is taken at an extreme point in $\genmeanset$ (since linear objective over closed convex set), which is a delta distribution of a set and
\begin{align*}
	\argmax_{\meanparam\in\genmeanset}  \sbcontraction{\meanparam,\invtemp\cdot\canparam}+ \sentropyof{\meanrepprob} 
	= 
	\argmax_{\meanparam\in\genmeanset}  \sbcontraction{\meanparam,\canparam} + \frac{1}{\invtemp} \cdot \sentropyof{\meanrepprob} 	
\end{align*}
thus the entropy term is neglectible for large $\invtemp$.
A more precise argument is using a limit of the maxima and can be found in Theorem~8.1 in \cite{wainwright_graphical_2008}





\subsubsection{Mean Field Method}

We rewrite 
\begin{align*}
	\max_{\meanparam\in\genmeanset}  \sbcontraction{\meanparam,\canparam} + \sentropyof{\meanrepprob} 
	=
	\max_{\probtensor} \sbcontraction{\energytensor,\probtensor} + \sentropyof{\probtensor}
\end{align*}
where
	\[ \energytensor = \sbcontractionof{\sencsstat,\canparam}{\shortcatvariables} \, . \]

We now restrict the distributions in the maximum.
Typically we use the family of independent distributions, also called naive mean field method.
The naive mean field is the approximation by distributions of independent random variables $\legcoreof{\catenumerator}$, that is
\begin{align*}
	\argmax_{\legcoreof{\catenumerator} \, : \, \catenumeratorin} \contraction{\{\energytensor\} \cup \{\legcoreof{\catenumerator} \, : \, \catenumeratorin\}}
	+ \sum_{\catenumeratorin} \sentropyof{\legcoreof{\catenumerator}} \, . 
\end{align*}


\begin{theorem}[Update equations for the mean field approximation]
	Keeping all legs but one constant, the problem
	\begin{align*}
		\argmax_{\legcoreof{\catenumerator}} \contraction{\{\energytensor\} \cup \{\legcoreof{\catenumerator} \, : \, \catenumeratorin\}}
		+ \sum_{\catenumeratorin} \sentropyof{\legcoreof{\catenumerator}} 
	\end{align*}
	is solved at 
		\[ \legcoreofat{\catenumerator}{\catvariableof{\catenumerator}} 
			= \normationof{ \expof{ \contractionof{ \{\energytensor[\shortcatvariables] \} \cup
				\{\legcoreofat{\seccatenumerator}{\catvariableof{\seccatenumerator}} \, : \, \seccatenumerator\neq\catenumerator\} }{\shortcatvariables} }
			}{\catvariableof{\catenumerator}} \, . \]
\end{theorem}
\begin{proof}
	We have
	\begin{align*}
		 \difofwrt{\sentropyof{\legcoreof{\catenumerator}}}{\legcoreof{\catenumerator}}
		=  - \lnof{\legcoreofat{\catenumerator}{\catvariableof{\catenumerator}}}
		+ \onesat{\catvariableof{\catenumerator}}
	\end{align*}
	and by multilinearity of tensor contractions
	\begin{align*}
		\difofwrt{\contraction{\{\energytensor\}\cup\{\legcoreof{\seccatenumerator} \, : \, \seccatenumeratorin \}}}{\legcoreof{\catenumerator}}
		=  \contractionof{\{\energytensor\}\cup\{\legcoreof{\seccatenumerator} \, : \, \seccatenumeratorin ,\, \seccatenumerator\neq\catenumerator \}}{\catvariableof{\catenumerator}} \, . 
	\end{align*}
	Combining both, the condition
	\begin{align*}
		0 = \difofwrt{
			\left( \contraction{\{\energytensor\}\cup\{\legcoreof{\seccatenumerator} \, : \, \seccatenumeratorin \}} + \sum_{\catenumeratorin} \sentropyof{\legcoreof{\catenumerator}} \right)
		}{\legcoreof{\catenumerator}}
	\end{align*}
	is equal to
	\begin{align*}
		\lnof{\legcoreofat{\catenumerator}{\catvariableof{\catenumerator}}} =
		 \onesat{\catvariableof{\catenumerator}} + \contractionof{\{\energytensor\}\cup\{\legcoreof{\seccatenumerator} \, : \, \seccatenumeratorin ,\, \seccatenumerator\neq\catenumerator \}}{\catvariableof{\catenumerator}} \, .
	\end{align*}
	Together with the condition $\sbcontractionof{\legcoreof{\catenumerator}}=1$ this is satisfied at
		\[ \legcoreofat{\catenumerator}{\catvariableof{\catenumerator}} 
			= \normationof{ \expof{ \contractionof{ \{\energytensor\} \cup
				\{\legcoreof{\seccatenumerator} \, : \, \seccatenumerator\neq\catenumerator\} }{\catvariableof{\catenumerator}} }
			}{\catvariableof{\catenumerator}} \, . \]
\end{proof}



Algorithm~\ref{alg:NMF} is the alternation of legwise updates until a stopping criterion is met.

\begin{algorithm}[h!]
\caption{Naive Mean Field Approximation}\label{alg:NMF}
\begin{algorithmic}
\For{$\catenumeratorin$}
	\State 
		\[ \legcoreofat{\catenumerator}{\catvariableof{\catenumerator}} 
		\algdefsymbol \normationof{\ones}{\catvariableof{\catenumerator}}  \]
\EndFor
\While{Stopping criterion is not met}
	\For{$\catenumeratorin$}
		\State 
			\[ \legcoreofat{\catenumerator}{\catvariableof{\catenumerator}} 
			\algdefsymbol \normationof{ \expof{ \contractionof{ \{\energytensor[\shortcatvariables] \} \cup
				\{\legcoreofat{\seccatenumerator}{\catvariableof{\seccatenumerator}} \, : \, \seccatenumerator\neq\catenumerator\} }{\catvariableof{\catenumerator}} }
			}{\catvariableof{\catenumerator}} \]
\EndFor
\EndWhile
\end{algorithmic}
\end{algorithm}


\subsubsection{Structured Variational Approximation}

%% Structured Variational approximation
More generically, we restrict the maximum over the mean parameters of efficiently contractable distributions and get a lower bound.
In this section we use any Markov Network as the approximating family. 

Let $\graph$ be any hypergraph, we define the problem
\begin{align}\tag{$\mathrm{P}_{\mnexpfamily, \probtensor}$}\label{prob:structuredApproximation}
	\argmax_{\probtensor\in \mnexpfamily} \sbcontraction{\energytensor,\probtensor} + \sentropyof{\probtensor}
\end{align}

We approximate the solution of this problem again by an alternating algorithm, which iteratively updates the cores of the approximating Markov Network. 

\begin{theorem}[Update equations for the structured variational approximation]\label{the:updateEquationStructuredVariational}
	The Markov Network $\extnet$ with hypercores $\extnetasset$ is a stationary point for Problem~\ref{prob:structuredApproximation}, if for all $\edgein$
	\begin{align*}
	\hypercoreofat{\edge}{\edgevariables}
	= \lambda\cdot \expof{
	\frac{
		\contractionof{\{\energytensor\}\cup\{
		\hypercoreof{\secedge} : \secedge\neq\edge
		\}}{\edgevariables} 
	}{
		\contractionof{\{
		\hypercoreof{\secedge} : \secedge\neq\edge
		\}}{\edgevariables} 
	}
	- \sum_{\thirdedge\neq\edge} 
		\frac{
		\contractionof{\{\lnof{\hypercoreof{\thirdedge}}\}\cup\{
		\hypercoreof{\secedge} : \secedge\neq\thirdedge
		\}}{\edgevariables} 
	}{
		\contractionof{\{
		\hypercoreof{\secedge} : \secedge\neq\thirdedge
		\}}{\edgevariables} 
	}
	}
	\end{align*}
	for any $\lambda>0$ (e.g. by the norm).
	Here, the quotient denotes the coordinatewise quotient.
\end{theorem}
\begin{proof}%[Proof of \theref{the:updateEquationStructuredVariational}]
	We proof the theorem by first order condition on the objective $\objof{\extnet} = \sbcontraction{\energytensor,\extnetdist} + \sentropyof{\extnetdist}$.
	
	To proof the theorem, we use \lemref{lem:difMNExpectation}, which shows a characterization of the derivative of functions
	
	%% Energy Contraction Term
	We have %for $\probtensor\in\mnexpfamily$
	\begin{align*}
		\sbcontraction{\energytensor,\normationof{\extnet}{\shortcatvariables}} 
		=  \frac{
			\contraction{\{\energytensor\}\cup\extnet} 
		}{
			\contraction{\extnet} 			
		} \, . 
	\end{align*}
	
	%% Entropy Term Decomposition
	Further we have
	\begin{align*}
		\sentropyof{\normationof{\extnet}{\shortcatvariables}}
		= \left(\sum_{\secedge\in\edges} \contraction{-\lnof{\hypercoreof{\secedge}},\normationof{\extnet}{\shortcatvariables}} \right)
		+ \lnof{\contraction{\extnet}}	
	\end{align*}
	
	We define the tensor
		\[ \sechypercore[\catvariableof{\nodes}] = \energytensorat{\catvariableof{\nodes}} 
		- \sum_{\secedge\neq\edge} \lnof{\hypercoreofat{\secedge}{\catvariableof{\secedge}}} \otimes \onesat{\catvariableof{\nodes/\secedge}} \]
	and notice, that $\sechypercore$ does not depend on $\hypercoreof{\edge}$.	

	The objective has then a representation as
	\begin{align*}
		\objof{\extnet} = \sbcontraction{\sechypercore[\catvariableof{\nodes}], \extnetdist} - \sbcontraction{ \lnof{\hypercoreof{\edge}}, \extnetdist} +  \lnof{\contraction{\extnet}}
	\end{align*}
	
	Let us now differentiate all terms.
	With \lemref{lem:difMNExpectation} we now get
	\begin{align*}
		\difwrt{\hypercoreofat{\edge}{\seccatvariableof{\edge}}} \sbcontraction{\sechypercore[\catvariableof{\nodes}], \extnetdist}
		& = \sbcontractionof{\sechypercoreat{\nodevariables},
	 	\identityat{\seccatvariableof{\edge},\edgevariables}, 
		\frac{\contractionof{\extnet}{\edgevariables}}{\hypercoreofat{\edge}{\edgevariables}}, 
		\normationofwrt{\extnet}{\catvariableof{\nodes/\edge}}{\edgevariables} }{\seccatvariableof{\edge},\nodevariables} \\
		& \quad -  \contraction{\sechypercoreat{\nodevariables},\extnetdist}
		 \otimes \sbcontractionof{\frac{\contractionof{\extnet}{\seccatvariableof{\edge}}}{\hypercoreofat{\edge}{\seccatvariableof{\edge}}}
		}{\seccatvariableof{\edge}} \, .
	\end{align*}
	
	Further we have
	\begin{align*}
		\difwrt{\hypercoreofat{\edge}{\seccatvariableof{\edge}}} \sbcontraction{ \lnof{\hypercoreof{\edge}}, \extnetdist} 
		& = \sbcontractionof{\lnof{\hypercoreofat{\edge}{\edgevariables}},
	 	\identityat{\seccatvariableof{\edge},\edgevariables}, 
		\frac{\contractionof{\extnet}{\edgevariables}}{\hypercoreofat{\edge}{\edgevariables}}, 
		\normationofwrt{\extnet}{\catvariableof{\nodes/\edge}}{\edgevariables} }{\seccatvariableof{\edge},\nodevariables} \\
		& \quad -  \contraction{\lnof{\hypercoreofat{\edge}{\edgevariables}},\extnetdist}
		 \otimes \sbcontractionof{\frac{\contractionof{\extnet}{\seccatvariableof{\edge}}}{\hypercoreofat{\edge}{\seccatvariableof{\edge}}}
		}{\seccatvariableof{\edge}} \\
		& \quad\quad - \sbcontraction{ \frac{1}{\hypercoreofat{\edge}{\edgevariables}}, \extnetdist}
	\end{align*}
	and (see Proof of \ref{lem:difMNprob})
	\begin{align*}
		\difwrt{\hypercoreofat{\edge}{\seccatvariableof{\edge}}} \lnof{\contraction{\extnet}}
		 = \frac{\difwrt{\hypercoreofat{\edge}{\seccatvariableof{\edge}}} \contraction{\extnet}}{\contraction{\extnet}} 		
		 = \frac{\contractionof{\extnet}{\seccatvariableof{\edge}}}{\hypercoreofat{\edge}{\seccatvariableof{\edge}}} \, .
	\end{align*}
	
	Together, the first order condition
	\begin{align*}
		0 = \difwrt{\hypercoreofat{\edge}{\seccatvariableof{\edge}}} \objof{\extnet}
	\end{align*}
	is equal to all $\seccatindexof{\edge}$ satisfying% here drop seccatvariable to catvariable by slicing 
	\begin{align*}
		0 & = \frac{\contractionof{\extnet}{\indexedseccatvariableof{\edge}}}{\hypercoreofat{\edge}{\indexedseccatvariableof{\edge}}}
		 \Big(
		 	\sbcontraction{\sechypercoreat{\catvariableof{\nodes/\edge},\catvariableof{\edge}=\seccatindexof{\edge}}, \normationofwrt{\extnet}{\catvariableof{\nodes/\edge}}{\catvariableof{\edge}=\seccatindexof{\edge}}} \\
			&\quad \quad - \sbcontraction{\sechypercoreat{\nodevariables}, \extnetdist}  \\
			&\quad \quad - \sbcontraction{\lnof{\hypercoreofat{\edge}{\edgevariables=\seccatindexof{\edge}}}, \normationofwrt{\extnet}{\catvariableof{\nodes/\edge}}{\catvariableof{\edge}=\seccatindexof{\edge}}} \\
			&\quad \quad + \sbcontraction{\lnof{\hypercoreofat{\edge}{\edgevariables}}, \extnetdist} 
		 \Big) \, . 
	\end{align*}
	
	We notice, that by normation
		\[ \sbcontraction{\lnof{\hypercoreofat{\edge}{\edgevariables=\seccatindexof{\edge}}}, \normationofwrt{\extnet}{\catvariableof{\nodes/\edge}}{\catvariableof{\edge}=\seccatindexof{\edge}}} =  \lnof{\hypercoreofat{\edge}{\edgevariables=\seccatindexof{\edge}}} \]
	and that the scalar
		\[ \lambda_1 = \sbcontraction{\sechypercoreat{\nodevariables},\normationof{\extnet}{\catvariableof{\nodes}}}	
		- \sbcontraction{\lnof{\hypercoreofat{\edge}{\edgevariables}},\normationof{\extnet}{\catvariableof{\nodes}}}	\]
	is the constant for all $\seccatindexof{\edge}$.
	
	The first order condition is therefore equal to the existence of a $\lambda_1\in\rr$ such that for all $\seccatindexof{\edge}$ 
	\begin{align*}
		\lnof{\hypercoreofat{\edge}{\catvariableof{\edge}=\seccatindexof{\edge}}}
		= 	\sbcontraction{\sechypercoreat{\catvariableof{\nodes/\edge},\catvariableof{\edge}=\seccatindexof{\edge}}, 
		\normationofwrt{\extnet}{\catvariableof{\nodes/\edge}}{\catvariableof{\edge}=\seccatindexof{\edge}}} + \lambda_1 \, . 
	\end{align*}
	The claim follows when applying the exponential on both sides and with the observation, that 
	\begin{align*}
	\sbcontraction{\sechypercoreat{\catvariableof{\nodes/\edge},\catvariableof{\edge}=\seccatindexof{\edge}}, 
		\normationofwrt{\extnet}{\catvariableof{\nodes/\edge}}{\catvariableof{\edge}=\seccatindexof{\edge}}}
		= 
		\frac{\contractionof{\{\sechypercore\}\cup\{\hypercoreof{\secedge} \, : \, \secedge\neq \edge\}}{\catvariableof{\edge}=\seccatindexof{\edge}} }{
		\contractionof{\{\hypercoreof{\secedge} \, : \, \secedge\neq \edge\}}{\catvariableof{\edge}=\seccatindexof{\edge}} 
		}
	\end{align*}
	and reparametrization of $\lambda_1$ to
		\[ \lambda = \expof{\lambda_1} \, . \]
\end{proof}

%% KL Divergence
The mean field method corresponds with minimization of the KL Divergence to the efficiently contractable family, i.e. the I-projection onto the family.

\begin{theorem}
	For any hypergraph $\graph$ and energy tensor $\energytensor$ we have 
	\begin{align*}
		\argmax_{\probtensor\in \mnexpfamily} \sbcontraction{\energytensor, \probtensor}+ \sentropyof{\probtensor}
		= \argmax_{\probtensor\in \mnexpfamily} \kldivof{\expdistof{(\graph,\canparam)}}{\normationof{\expof{\energytensor}}{\shortcatvariables}}
	\end{align*}
	Problem~\ref{prob:structuredApproximation} is thus the I-projection onto the exponential family $\mnexpfamily$.
\end{theorem}
\begin{proof}
%	This follows from the fact, that the objective is the cross-entropy and the position of the maximum is invariant under substracting $\sentropyof{\probtensor}$.
	By rearranging the objective to the KL divergence.
\end{proof}








\subsection{Backward Mapping in Exponential Families}

%% FROM NETWORK LEARNING
The parameters optimizing the likelihood, will be shown to coincide with the backward mapping evaluated on the expectation of the sufficient statistics (see \theref{the:parEstToBackwardMap}).
This is in most generality true for the parameters of the M-projection of any distribution onto the exponential family.
We therefore investigate methods to compute the backward mapping, in most generality by alternating algorithms and in the special case of Markov Logic Networks by closed form representations.




%\begin{theorem}[Moment Matching Criteria]\label{the:MM}
	We have that $\canparam$ is a solution of the backward problem at $\genmean$, if and only if 
		\[ \sbcontractionof{\expdist,\sencsstat}{\selvariable} = \genmeanat{\selvariable} \, . \]
%\end{theorem}

This contraction equation is called moment matching, since the moment of the empirical distribution is matched by the moment of the fitting distribution.

We find one backward mapping as the dual problem to the forward mapping.


\subsubsection{Variational Formulation}

The backward mapping to $\datameanat{\selvariable} = \sbcontractionof{\empdistribution,\sencsstat}{\selvariable}$ is Maximum Likelihood estimation and the solution of the maximum entropy problem.

\begin{theorem}\label{the:varBackward}
	Let there be a sufficient statistic $\sstat$.
	The map $\backwardmap: \rr^{\seldim}\rightarrow \rr^{\seldim}$ defined as
	\begin{align*}
		\backwardmapof{\meanparam}
		= \argmax_{\canparam\in\rr^{\seldim}}  \sbcontraction{\meanparam,\canparam} - \cumfunctionof{\canparam} \, . 
	\end{align*}
	is a backward mapping.
\end{theorem}
\begin{proof}
	%\red{From duality, see Theorem~3.4 in \cite{wainwright_graphical_2008}.}
	We show the claim can be shown by the first order condition on the objective.	
	It holds that
	\begin{align*}
		\difwrt{\canparamat{\selvariable}}  \cumfunctionof{\canparam}  
		 & = \difwrt{\canparamat{\selvariable}}  \lnof{\contraction{\expof{\contractionof{\sencsstat,\canparam}{\shortcatvariables}}}} \\
		 & = \difwrt{\canparamat{\selvariable}} \frac{\contraction{\sencsstat[\selvariable],\expof{\contractionof{\sencsstat,\canparam}{\shortcatvariables}}}}{\contraction{\expof{\contractionof{\sencsstat,\canparam}{\shortcatvariables}}}}   \\
		 & = \forwardmapof{\canparam}[\selvariable]
	\end{align*}
	and thus
	\begin{align*}
		\difwrt{\canparamat{\selvariable}} \left( \sbcontraction{\meanparam,\canparam} - \cumfunctionof{\canparam}  \right) 
		= \meanparamat{\selvariable} -  \forwardmapof{\canparam}[\selvariable] \, . 
	\end{align*}
	
	The first order condition is therefore 
		\[ \meanparamat{\selvariable} =  \forwardmapof{\canparam}[\selvariable] \]
	and any $\canparam$ satisfies this condition exactly when $\canparam=\backwardmapof{\meanparam}$ for a backward map.
\end{proof}


\subsubsection{Interpretation by Maximum Likelihood Estimation}

% Backward mapping
Backward mapping coincides with the Maximum Likelihood Estimation Problem \eqref{prob:parameterMaxLikelihood}, when we take $\Gamma$ to the distributions in an exponential family $\expfamily$ for a sufficient statistic $\sstat$.

% Cross entropy
The loss is the cross entropy between a distribution with $\meanparam$ and the distribution $\expdistof{(\sstat,\canparam,\basemeasure)}$.


\begin{theorem}
	Let there be any exponential family, a mean parameter vector $\genmean\in\imageof{\forwardmap}$ and a backward map $\backwardmap$.
	Then $\estcanparam=\backwardmapof{\genmean}$ is the parameter of the M-projection (Problem~\ref{prob:mProjection}) of any $\gendistribution$ with $\sbcontractionof{\sencsstat,\gendistribution}{\selvariable}=\genmeanat{\selvariable}$ on to $\expfamily$, that is
		\[ \expdistof{(\sstat,\estcanparam,\basemeasure)} \in \argmax_{\probtensor\in\expfamily} \centropyof{\gendistribution}{\probtensor}  \, . \]
	In particular, if $\meanparam=\datamean$ for a data map $\datamap$, the backward map is a maximum likelihood estimator.
\end{theorem}
\begin{proof}
	We exploit the variational characterization of the backward map by \theref{the:varBackward}, and first show that the objective coincides with the cross entropy between the distribution $\gendistribution$ and the respective member of the exponential family.
	For any $\gendistribution$ and $\canparam$ we have with Example~\ref{exa:cEntropyExp}
	\begin{align*}
		\centropyof{\gendistribution}{\expdistof{(\sstat,\canparam,\basemeasure)}} 
		=   \sbcontraction{\gendistribution,\sencsstat,\canparam} -\cumfunctionof{\canparam} \, .  
	\end{align*}
	We use that by assumption $\sbcontractionof{\gendistribution,\sencsstat}{\selvariable}=\genmeanat{\selvariable}$ and thus
	\begin{align*}
		\centropyof{\gendistribution}{\expdistof{(\sstat,\canparam,\basemeasure)}} 
		=   \sbcontraction{\genmean,\canparam} -\cumfunctionof{\canparam} \, .  
	\end{align*}
	This shows, that the backward map coincides with the M-projection onto $\Gamma=\expfamily$.

	Further, if $\meanparam=\datamean$ for a data map $\datamap$, we have that the corresponding empirical distribution $\empdistribution$ satisfies $\sbcontractionof{\sencsstat,\empdistribution}{\selvariable}=\meanparamat{\selvariable}$.
	The backward map of $\meanparam$ is therefore the M-projection of $\empdistribution$, which is with \theref{the:lossCentropy} the maximum likelihood estimator.
\end{proof}


%\begin{lemma}
%	Let $\sstat\in\facspace\otimes\rr^{\seldim}$ be a sufficient statistic and $\gendistribution\in\facspace$ a probability distribution.
%	For any member $\expdist\in\expfamily$ we have
%		\[ \centropyof{\gendistribution}{\expdist} = \sbcontraction{\canparam,\genmean} - \cumfunctionof{\canparam} \]
%	where 
%		\[ \genmean = \sbcontractionof{\gendistribution,\sencsstat}{\selvariableof{\sstat}} \,  \]
%	and 
%		\[ \cumfunctionof{\canparam} = \lnof{\contraction{\expof{\expenergy}}} \, . \]
%	The M-projection of $\gendistribution$ onto $\expfamily$ is  $\expdistof{(\sstat,\estcanparam,\basemeasure)}$ for
%		\[ \estcanparam\in \argmax_{\canparam}  \sbcontraction{\canparam,\genmean} - \cumfunctionof{\canparam} \, .  \]
%\end{lemma}
%\begin{proof}
%	By decomposing 
%	\begin{align*}
%		\expdist 	& = \normationof{\expof{\sbcontractionof{\sencsstat,\canparam}{\shortcatvariables}}}{\shortcatvariables} \\
%				& = \frac{\expof{\expenergy}}{\sbcontraction{\expof{\expenergy}}}
%	\end{align*}
%	we get
%	\begin{align*}
%		\lnof{\expdist} & = \lnof{\expof{\expenergy}} - \onesat{\shortcatvariables} \cdot \sbcontraction{\expof{\expenergy}} \\ 
%		& = \expenergy - \cumfunction(\canparam) \cdot \onesat{\shortcatvariables}  \, .
%	\end{align*}
%	If follows that
%	\begin{align*}
%		\centropyof{\gendistribution}{\expdist} 
%		&=  \sbcontraction{\gendistribution,\lnof{\expdist}} \\
%		&=  \sbcontraction{\gendistribution,\expenergy} - \cumfunction(\canparam) \cdot \sbcontraction{\gendistribution}   \\
%		&= \sbcontraction{\canparam, \genmean} - \cumfunction(\canparam) \, . 
%	\end{align*}
%\end{proof}




%%\subsection{Maximum Likelihood and Maximum Entropy for Exponential Families}
%
%Parameter Estimation is the M-Projection of a distribution onto the exponential family.
%

%% DONE BEFORE!
%\begin{theorem}[\cite{wainwright_graphical_2008}]\label{the:parEstToBackwardMap}
%	Given any probability distribution $\probat{\shortcatvariables}$ and a exponential family defined by the sufficient statistic $\sstat$, the M-Projection onto the family is the distribution $\probtensorof{(\sstat,\estcanparam,\basemeasure)}$ where
%	\begin{align*}
%		\estcanparam = \backwardmapof{\contractionof{\probtensor,\sencsstat}{\selvariable}} \, .
%	\end{align*}
%\end{theorem}
%\begin{proof}
%	$\contractionof{\probtensor,\sencsstat}{\selvariable}$ is in $\imageof{\forwardmap}$ and MLE has a variational characterization with maximum at the dual $\estcanparam$, see \cite{wainwright_graphical_2008}.
%\end{proof}





\subsubsection{Connection with Maximum Entropy}\label{sec:maxEntDuality}


The Maximum entropy problem with respect to matching expected statistics $\genmean\in\genmeanset$ 
\begin{align}\tag{$\probtagtypeinst{\entropysymbol}{\sstat,\basemeasure,\genmean}$}\label{prob:maxEntropy}
	\argmax_{\probtensor\in\Gamma^{\basemeasure}} \sentropyof{\probtensor} \quad \text{subject to} \quad 
	 \sbcontractionof{\probtensor,\sencsstat}{\selvariable} =  \genmeanat{\selvariable}
\end{align}
where the optimization is over all the distributions $\Gamma^{\basemeasure}$, which are representable with respect to the base measure $\basemeasure$.

\begin{theorem}\label{the:maxEntInterior}
	Let $\sstat$ be a statistic and $\basemeasure$ a base measure.
 	For any $\genmean\in\sbinteriorof{\genmeanset}$ the solution of \probref{prob:maxEntropy} is the distribution $\expdistof{(\secsstat,\estcanparam,\secbasemeasure)}$, where $\estcanparam=\backwardmapwrtof{\secsstat,\secbasemeasure}{\secmeanparam}$.
\end{theorem}
\begin{proof}
	Since $\genmean\in\sbinteriorof{\genmeanset}$, \theref{the:meanPolytopeInteriorCharacterization} implies the existence of $\estcanparam$ such that 
		\[ \genmeanat{\selvariable} = \sbcontractionof{\expdistof{(\sstat,\estcanparam,\basemeasure)},\sencsstat}{\selvariable}   \, . \]
	We now follow the argumentation of the proof of Theorem~20.2 in \cite{koller_probabilistic_2009}.
	Let $\secprobtensor$ further be an arbitrary distribution, possibly different from $\expdistof{(\sstat,\estcanparam,\basemeasure)}$, such that
		\[ \genmeanat{\selvariable} = \sbcontractionof{\secprobtensor,\sencsstat}{\selvariable}  \, . \]
	We then have
	\begin{align*}
		\sentropyof{\expdistof{(\sstat,\estcanparam,\basemeasure)}}
		= \centropyof{\secprobtensor}{\expdistof{(\sstat,\estcanparam,\basemeasure)}}
	\end{align*}
	
	With the Gibbs inequality we have if $\secprobtensor\neq\expdistof{(\sstat,\estcanparam,\basemeasure)}$
	\begin{align*}
		\sentropyof{\expdistof{(\sstat,\estcanparam,\basemeasure)}} - \sentropyof{\secprobtensor}
		= \centropyof{\secprobtensor}{\expdistof{(\sstat,\estcanparam,\basemeasure)}} - \sentropyof{\secprobtensor} > 0 \, . 
	\end{align*}	
	
	Therefore, if $\secprobtensor$ does not coincide with$\expdistof{(\sstat,\estcanparam,\basemeasure)}$, it is not a solution of Problem~\ref{prob:maxEntropy}.
	%Classical result based on duality of maximum entropy and maximum likelihood, shown e.g. in Koller Book.
\end{proof}

% Interpretation
Let us highlight the fact, that in \probref{prob:maxEntropy} we did not restrict to distributions in an exponential family and only demanded representability with respect to the base measure.
When choosing the trivial base measure, this does not pose a restriction on the distributions.
\theref{the:maxEntInterior} states, that when the maximum entropy problem has a solution (i.e. $\genmean\in\genmeanset$), then the solution is in the exponential family to the statistic $\sstat$.

% Generalization
When $\genmean\notin\sbinteriorof{\genmeanset}$, the mean paramater is by \theref{the:meanPolytopeInteriorCharacterization} not reproducable by a member of the exponential family $\expfamilyof{\sstat,\basemeasure}$. 
Instead, in combination with the base measure refinement \algoref{alg:baseMeasureRefinement}, we show that the solution is in a refined exponential family.
% dropping the assumption that the mean parameters are in the interior of the mean parameter polytope.

\begin{theorem}\label{the:maxEntMaxLikeDuality} 
	Let $\sstat$ be a statistic and $\basemeasure$ a base measure.
	For any $\genmean\in\genmeanset$, let $\secsstat,\secbasemeasure$ and $\secmeanparam$ be the outputs of \algoref{alg:baseMeasureRefinement} when passing $\sstat,\basemeasure$ and $\genmean$ as input.
	Then, the distribution $\expdistof{(\secsstat,\estcanparam,\secbasemeasure)}$, where $\estcanparam=\backwardmapwrtof{\secsstat,\secbasemeasure}{\secmeanparam}$, solves \probref{prob:maxEntropy}.
\end{theorem}
\begin{proof}
	\theref{the:baseMeasureRefinement} and the above Lemma.
\end{proof}

% Minimality of the refined base measure
\theref{the:maxEntMaxLikeDuality} further implies, that the base measure $\secbasemeasure$ identified by \algoref{alg:baseMeasureRefinement} is minimal for the maximum entropy problem, in the sense that the solving distribution is positive with respect to it and all feasible distributions have to be representable by it.
This highlights the fact, that the maximum entropy distribution does not vanish beyond those states, which are necessary by \theref{the:baseMeasureRefinement}.


%\begin{theorem}\label{the:maxEntMaxLikeDuality} % In Koller Book, Theorem 20.2
%	If $\genmean\in\imageof{\forwardmap}$, we have that any distribution solving Problem~\ref{prob:maxEntropy} has a representation by $\expdistof{(\sstat,\estcanparam,\basemeasure)}$, 
%	where $\estcanparam=\backwardmapof{\genmean}$ for any backward map of the exponential family. 
%	%where $\estcanparam$ is the Maximum Likelihood Estimate with respect to any $\probtensor$ with $\sbcontractionof{\secprobtensor,\sencsstat}{\selvariable} =\genmean$.
%%
%%	Let $\sstat$ be a map and $\gendistribution$ be any distribution of $\atomstates$ and define
%%		\[ \genmeanat{\selvariable} = \sbcontractionof{\gendistribution,\sencsstat}{\selvariable} \, .  \]
%%	Then the solution of \ref{prob:maxEntropy} coincides with the member $\expdistof{(\sstat,\estcanparam,\basemeasure)}$ of the exponential family $\expfamily$ where
%%		\[ \estcanparam = \backwardmapof{\genmean} \]
%%	for a backward map $\backwardmap$ of $\expfamily$.
%\end{theorem}
%\begin{proof}
%	Since $\genmean\in\imageof{\forwardmap}$, there is a parameter $\estcanparam$ such that 
%		\[ \genmeanat{\selvariable} = \sbcontractionof{\expdistof{(\sstat,\estcanparam,\basemeasure)},\sencsstat}{\selvariable}   \, . \]
%	Let $\secprobtensor$ further be an arbitrary distribution such that
%		\[ \genmeanat{\selvariable} = \sbcontractionof{\secprobtensor,\sencsstat}{\selvariable}  \, . \]
%	We then have
%	\begin{align*}
%		\sentropyof{\expdistof{(\sstat,\estcanparam,\basemeasure)}}
%		= \centropyof{\secprobtensor}{\expdistof{(\sstat,\estcanparam,\basemeasure)}}
%	\end{align*}
%	
%	With the Gibbs inequality we have if $\secprobtensor\neq\expdistof{(\sstat,\estcanparam,\basemeasure)}$
%	\begin{align*}
%		\sentropyof{\expdistof{(\sstat,\estcanparam,\basemeasure)}} - \sentropyof{\secprobtensor}
%		= \centropyof{\secprobtensor}{\expdistof{(\sstat,\estcanparam,\basemeasure)}} - \sentropyof{\secprobtensor} > 0 \, . 
%	\end{align*}	
%	
%	Therefore, if $\secprobtensor$ does not coincide with$\expdistof{(\sstat,\estcanparam,\basemeasure)}$, it is not a solution of Problem~\ref{prob:maxEntropy}.
%	%Classical result based on duality of maximum entropy and maximum likelihood, shown e.g. in Koller Book.
%\end{proof}




\subsubsection{Alternating Algorithms to Approximate the Backward Map}\label{sec:alternatingBackwardMap}


\red{While the forward map always has a representation in closed form by contraction of the probability tensor, the backward map in general fails to have a closed form representation.
Computation of the Backward map can instead be performed by alternating algorithms, as we show here.} % Are these fixpoint iterations?


Alternate through the coordinates of the statistics and adjust $\canparamat{\indexedselvariable}$ to a minimum of the likelihood, i.e. where for any $\selindexin$
\begin{align*}
	0 = \frac{\partial}{\partial \canparamat{\indexedselvariable}} \lossof{\expdist} \, . 
\end{align*}

% Moment matching
This condition is equal to the collection of moment matching equations % (see \theref{the:mm})
\begin{align*}
	\sbcontractionof{\expdist,\sencsstat}{\indexedselvariable} = \sbcontraction{\empdistribution,\sencsstat}{\indexedselvariable} \, . 
\end{align*}


\begin{lemma}\label{lem:mmContractionEquation}
	For any sufficient statistic $\sstat$ a parameter vector $\canparam$ and a $\selindexin$ we define
	\begin{align*}
	 	\hypercoreat{\catvariableof{\sstatcoordinateof{\selindex}}} 
		= \contractionof{\{\sstatcc\}\cup\{\headcoreof{\tilde{\selindex}} : \tilde{\selindex} \in [\seldim], \tilde{\selindex}\neq\selindex\}}{\catvariableof{\sstatcoordinateof{\selindex}}} \, . 
	\end{align*}
	Then the moment matching condition for $\sstatcoordinateof{\selindex}$ relative to $\canparam$ and $\meanparam$ is satisfied for any $\canparamat{\indexedselvariable}$ with
	\begin{align*}
		\sbcontraction{\headcoreof{\selindex}, \idrestrictedto{\imageof{\sstatcoordinateof{\selindex}}}, \hypercoreat{\selvariable_\sstat}}
		= \sbcontraction{\headcoreof{\selindex}, \hypercoreat{\selvariable_\sstat}} \cdot \meanparamat{\indexedselvariable} \, . 
	\end{align*}
\end{lemma}
\begin{proof}
	We have
	\begin{align*}
		\expdist = \frac{
			\sbcontractionof{\headcoreof{\selindex}, \hypercore}{\shortcatvariables}
		}{
			\sbcontraction{\headcoreof{\selindex}, \hypercore}
		}
	\end{align*}
	and 
	\begin{align*}
		\sbcontraction{\expdist, \sstatcoordinateof{\selindex}}
		= \frac{
			\sbcontractionof{\headcoreof{\selindex}, \idrestrictedto{\imageof{\sstatcoordinateof{\selindex}}}, \hypercore}{\shortcatvariables}
		}{
			\sbcontraction{\headcoreof{\selindex}, \hypercore}
		} \, . 
	\end{align*}
	Here we used
		\[ \sstatcoordinateof{\selindex} = \sbcontractionof{\headcoreof{\selindex}, \idrestrictedto{\imageof{\sstatcoordinateof{\selindex}}}}{\shortcatvariables} \]
	and redundancies of copies of relational encodings.
	It follows that 
	\begin{align*}
		\sbcontraction{\expdist,\sstatcoordinateof{\selindex}} = \contraction{\empdistribution,\sstatcoordinateof{\selindex}}
	\end{align*}
	is equal to
	\begin{align*}
		\sbcontraction{\headcoreof{\selindex}, \idrestrictedto{\imageof{\sstatcoordinateof{\selindex}}}, \hypercoreat{\catvariableof{\sstatcoordinateof{\selindex}}}}
		= \sbcontraction{\headcoreof{\selindex},\hypercoreat{\catvariableof{\sstatcoordinateof{\selindex}}}} \cdot \meanparamat{\indexedselvariable} \, . 
	\end{align*}	
\end{proof}

% Alternation necessary
The steps have to be alternated until sufficient convergence, since matching the moment to $\selindex$ by modifying $\canparamat{\indexedselvariable}$ will in general change other moments, which will have to be refit.


%Coordinate descent
An alternating optimization is the coordinate descent of the negative likelihood, seen as a function of the coordinates of $\canparam$, see Algorithm~\ref{alg:AMM}.
Since the log likelihood is concave, the algorithm converges to a global minimum.



\begin{algorithm}[h!]
\caption{Alternating Moment Matching}\label{alg:AMM}
\begin{algorithmic}
\State Set $\canparamat{\selvariable}=0$
\State Compute $\datameanat{\selvariable}= \sbcontractionof{\empdistribution,\sencsstat}{\selvariable}$
%\For{$\selindexin$}
%	\State Set $\canparamat{\indexedselvariable}=0$ 
%	\State Compute $\meanparamat{\indexedselvariable}^{\datamap} = \contractionof{\{\empdistribution,\sstatcoordinateof{\selindex}\}}{\varnothing} $ % Or give those as input!
%\EndFor
\While{Stopping criterion is not met}
\For{$\selindexin$}
	\State Compute 
		\begin{align*}
			\hypercoreofat{\selindex}{\catvariableof{\sstatcoordinateof{\selindex}}} 
			\algdefsymbol \contractionof{\{\sstatcc\}\cup\{\headcoreof{\tilde{\selindex}} : \tilde{\selindex} \in [\seldim], \tilde{\selindex}\neq\selindex\}}{\catvariableof{\sstatcoordinateof{\selindex}}} 
		\end{align*}
	\State Set $\canparamat{\indexedselvariable}$ to a solution of 
	\begin{align*}
		\sbcontraction{\headcoreof{\selindex},\idrestrictedto{\imageof{\sstatcoordinateof{\selindex}}},\hypercoreof{\selindex}}
		\algdefsymbol \sbcontraction{\headcoreof{\selindex},\hypercoreof{\selindex}} \cdot \datameanat{\indexedselvariable} \, . 
	\end{align*}
\EndFor
\EndWhile
\end{algorithmic}
\end{algorithm}


% 
In general, if $\imageof{\sstatcoordinateof{\selindex}}$ contains more than two elements, there exists no closed form solutions.
We will investigate the case of binary images, where there are closed form expressions, later in \secref{sec:alternatingParEstMLN}.


%
The computation of $\hypercoreof{\selindex}$ in Algorithm~\ref{alg:AMM} can be intractable and be replaced by an approximative procedure based on message passing schemes.

\subsection{Discussion}

% Forward mapping as gradient of A
Further in \cite{wainwright_graphical_2008}: Convex Duality.
Forward mapping coincides with gradient, i.e. $\meanparam = \nabla \cumfunction(\canparam)$.

% Gradient property of the backward mapping
In \cite{wainwright_graphical_2008}:
The objective is the conjugate dual $\dualcumfunction$ of $\cumfunction$, and backward mapping has an expression by the gradient, i.e. $\canparam = \nabla \dualcumfunction(\meanparam)$.


