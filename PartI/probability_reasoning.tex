\chapter{\chatextprobReasoning}\label{cha:probReasoning}

After having investigated sparse decomposition schemes of probability distributions into tensor networks, we now exploit these schemes to derive efficient reasoning schemes. % Start of Focus Reasoning here
We first introduce by queries a generic scheme to retrieve information by contractions, and introduce the method of maximum likelihood estimation related to entropy optimization.
Then we focus on inference tasks in exponential families, which have been introduced as a generalization of graphical models in the previous chapter.


\sect{Queries}

The efficient retrieval of information stored in probability distributions has to exploit the available decomposition schemes.
To avoid the instantiation of a distribution based on its decomposition, we directly define deductive reasoning schemes by contractions, which can be executed using the available decomposition.

\subsect{Querying by Functions}

We now formalize queries by retrieving expectations of functions given a distribution specified by probability tensors.
We exploit basis calculus in defining categorical variables $\headvariableof{\exfunction}$ to tensors $\exfunction$, which are enumerating the set $\imageof{\exfunction}$.
More details on this scheme are provided in \charef{cha:basisCalculus}, see \defref{def:functionRelationEncoding} therein.

\begin{definition}
    \label{def:queries}
    The marginal query of a probability distribution $\probat{\shortcatvariables}$ by a query statistic
    \begin{align*}
        \exfunction \defcols \facstates \rightarrow \arbsetof{\exfunction}
    \end{align*}
    is the vector $\probat{\headvariableof{\exfunction}}$, where $\headvariableof{\exfunction}$ is an image enumerating variable (see \charef{cha:basisCalculus} for more details), defined as the contraction
    \begin{align*}
        \probat{\headvariableof{\exfunction}}
        = \contractionof{\probat{\shortcatvariables},\bencodingofat{\exfunction}{\headvariableof{\exfunction},\shortcatvariables}}{\headvariableof{\exfunction}} \, .
    \end{align*}
    If $\arbsetof{\exfunction}\subset\rr$, and the statistic $\exfunction$ is therefore a tensor, we further define the expectation query of $\probwith$ by $\exfunction$ as
    \begin{align*}
        \expectationof{\exfunction} = \contraction{\exfunctionat{\shortcatvariables},\probwith} \, .
    \end{align*}
    Given another query statistic $\secexfunction: \facstates \rightarrow \arbsetof{\secexfunction}$, which image is enumerated by a variable $\headvariableof{\secexfunction}$, the conditional query of the probability distribution $\probat{\shortcatvariables}$ by the statistic $\exfunction$ conditioned on $\secexfunction$ is the matrix $\condprobof{\headvariableof{\exfunction}}{\headvariableof{\secexfunction}}\in\rr^{\cardof{\imageof{\exfunction}}}\otimes \rr^{\cardof{\imageof{\secexfunction}}}$ defined as the normalization
    \begin{align*}
        \condprobof{\headvariableof{\exfunction}}{\headvariableof{\secexfunction}}
        = \normalizationofwrt{
            \probat{\shortcatvariables},\bencodingofat{\exfunction}{\headvariableof{\exfunction},\shortcatvariables},\bencodingofat{\secexfunction}{\headvariableof{\secexfunction},\shortcatvariables}
        }{
            \headvariableof{\exfunction}}{\headvariableof{\secexfunction}
        } \, .
    \end{align*}
\end{definition}

We notice, that marginal and conditional queries generalize the schemes of marginalization and conditioning on the distributed variables.
As such, marginal distributions are marginal queries with respect to a identity query statistic acting on the respective variables.
Conditional distributions can be similarly retrieved by two identity statistics, representing the incoming and outgoing variables.
%The query framework of \defref{def:queries} can be used beyond marginal and conditional quereis for the customized information retrieval.

%% Relation of queries and expectation queries
Expectation queries return the expectation of a real-valued feature $\exfunction : \facstates\rightarrow\rr$.
When understanding $\exfunction$ as a random variable given the probability measure $\probat{\shortcatvariables}$, the expectation query returns the expectation of that random variable.
Expectation queries are further contractions of marginal queries with the identity function restricted on the image of $\exfunction$, since
\begin{align*}
    \expectationof{\exfunction}
    = \contraction{\probat{\headvariableof{\exfunction}},\idrestrictedto{\imageof{\exfunction}}{\headvariableof{\exfunction}} } \, .
\end{align*}
This contraction equation follows from the more general \corref{cor:rhoToNormal}, which will be shown in \charef{cha:basisCalculus}.

% Mean parameters
Expectation queries compute mean parameters to a distribution, see \defref{def:meanPolytope}.
Given a statistic $\sstat$ and a distribution $\probwith$, each feature $\sstatcoordinateof{\selindex}$ for $\selindexin$ corresponds with a mean parameter
\begin{align*}
    \expectationof{\exfunction} = \contraction{\probwith,\sstatcoordinateof{\selindex}} = \meanparamat{\indexedselvariable} \, .
\end{align*}

%% Conditional Probabilities and conditional queries
%Conditional probabilities are queries, where the tensors $\exfunction$ and $\secexfunction$ are identity mappings in the respective variable state spaces.
%Conversely, we can understand the conditional query $\condprobof{\headvariableof{\exfunction}}{\headvariableof{\secexfunction}}$ as the conditional probability of a query statistic $\exfunction$ conditioned on $\secexfunction$, of the underlying Markov Network with cores $\{\probtensor, \bencodingof{\exfunction}, \bencodingof{\secexfunction} \}$ and variables $\catvariableof{\exfunction},\catvariableof{\secexfunction}$ besides the variables distributed by $\probtensor$.

%%% Expectations as event queries -> Consistency with $\probat{X=i}$?
%We further denote event queries by
%	\[  \expectationof{\exfunction=\exfunctionimageelement} = \contraction{\probtensor,\bencodingof{\exfunction},\onehotmapof{\exfunctionimageelement}}\]
%where by $\onehotmapof{z}$ be denote the one hot encoding of the state $z$ with respect to some enumeration.
%Let us note that they are further contraction of the queries in \defref{def:queries} since by \theref{the:splittingContractions}
%\begin{align*}
%	 \expectationof{\exfunction=z}
%	& =  \contraction{ \contractionof{\probtensor,\bencodingof{\exfunction}}{\catvariableof{\exfunction}} ,\onehotmapof{z}}\\
%	& =  \contraction{ \probat{\exfunction} ,\onehotmapof{z}} \, .
%\end{align*}

\subsect{Mode Queries}\label{sec:modeQueries}

A different kind of queries are mode queries, which we formalize by the searches of the indices to maximal coordinates in a tensor.

\begin{definition}
    Given a tensor $\hypercorewith$ the mode query is the problem
    \begin{align}
        \argmax_{\shortcatindicesin} \hypercoreat{\indexedshortcatvariables} \, .
    \end{align}
\end{definition}

%MAP relation - Needed?
%Mode queries are often called MAP queries.
%The name MAP is an abbreviation of maximum a-posteriori, since it searches for the most probable state in probability distributions.
%The term a-posteriori originates from a focus on conditional probabilities, which slices with respect to incoming variables represent the a-posteriori distribution given evidence on the incoming variables.
%We will in this work use this term for generic maximum coordinate searches.

% Convex optimization
We can pose mode queries as convex optimization problems.
To this end we recall, that the set of all probability distributions is a convex hull of the one-hot encoded states, that is
\begin{align*}
    \bmrealprobof{\ones}
    = \convhullof{\onehotmapofat{\shortcatindices}{\shortcatvariables}\wcols\shortcatindicesin} \, .
\end{align*}
With this we have
\begin{align*}
    \max_{\shortcatindices} \hypercoreat{\indexedshortcatvariables}
    &= \max_{\shortcatindices} \contraction{\hypercoreat{\shortcatvariables},\onehotmapofat{\shortcatindices}{\shortcatvariables}} \\
    &= \max_{\meanparamat{\shortcatvariables}\in\bmrealprobof{\ones}} \contraction{\hypercoreat{\shortcatvariables},\meanparamat{\shortcatvariables}} \, .
\end{align*}
We note that the maximization over $\bmrealprobof{\ones}$ is a convex optimization problem and the maxima are taken at
\begin{align*}
    \argmax_{\meanparamat{\shortcatvariables}\in\bmrealprobof{\ones}} \contraction{\hypercoreat{\shortcatvariables},\meanparamat{\shortcatvariables}}
    = \convhullof{\onehotmapofat{\shortcatindices}{\shortcatvariables}\wcols\shortcatindices\in\argmax_{\shortcatindices} \hypercoreat{\indexedshortcatvariables}} \, .
\end{align*}
We will further apply this generic trick to approach mode queries when studying inference problems for exponential families in the following sections.

% Restriction to subsets
Let us note, that we can also approach modified mode queries, where we restrict to $\arbset\subset\facstates$, namely
\begin{align*}
    \argmax_{\shortcatindices\in\arbset} \hypercoreat{\indexedshortcatvariables} \, .
\end{align*}
We can choose the base measure by the subset encoding (see \charef{cha:basisCalculus}) of $\arbset$, namely
\begin{align*}
    \basemeasurewith = \sum_{\shortcatindices\in\arbset} \onehotmapofat{\shortcatindices}{\shortcatvariables} \, ,
\end{align*}
and conclude that
\begin{align*}
    \argmax_{\meanparamat{\shortcatvariables}\in\bmrealprobof{\basemeasure}} \contraction{\hypercoreat{\shortcatvariables},\meanparamat{\shortcatvariables}}
    = \convhullof{\onehotmapofat{\shortcatindices}{\shortcatvariables} \wcols \shortcatindices\in\argmax_{\shortcatindices\in\arbset} \hypercoreat{\indexedshortcatvariables}} \, .
\end{align*}



\subsect{Energy Representations}

For exponential families (see \secref{sec:exponentialFamilies}) we have observed, that often energy tensors have feasible tensor network representations, whereas the corresponding probability distributions can get infeasible.
We therefore investigate here schemes to answer queries based on the energy tensor instead of the distribution.
%Let us now interpret a probability tensor at hand as a member of an exponential family (see \secref{sec:exponentialFamilies}), which is always possible when taking the naive exponential family.

\begin{theorem}
    \label{the:energyContractionQueries} % This is a statement about "full" queries.
    Let $\energytensorwith$ be an energy tensor and $\probwith=\normalizationof{\expof{\energytensorwith}}{\shortcatvariables}$ the corresponding distribution.
%	For any probability distribution $\probtensor$ with $\probtensor= \normalizationof{\expof{\energytensorwith}}{\shortcatvariables}$,
    For disjoint subsets $\nodesa,\nodesb \subset [\catorder]$ with $\nodesa\cup\nodesb=[\catorder]$ and any $\catindexof{\nodesb}$ we have
    \begin{align*}
        \condprobof{\catvariableof{\nodesa}}{\indexedcatvariableof{\nodesb}}
        = \normalizationof{
            \expof{\energytensorat{\catvariableof{\nodesa},\indexedcatvariableof{\nodesb}}}
        }{\catvariableof{\nodesa}} \, .
    \end{align*}
\end{theorem}
\begin{proof}
    To show the theorem, we use a generic simplification property of coordinatewise transforms, which we will show as \lemref{lem:coordinatewisetrafoSliceReduction} in \charef{cha:coordinateCalculus} and get
    \begin{align*}
        \contractionof{\expof{\energytensorat{\catvariableof{\nodesa},\catvariableof{\nodesb}}}}{\catvariableof{\nodesa},\indexedcatvariableof{\nodesb}}
        = \contractionof{\expof{\energytensorat{\catvariableof{\nodesa},\indexedcatvariableof{\nodesb}}}}{\catvariableof{\nodesa}}
    \end{align*}
    Based on this we get
    \begin{align*}
        \condprobof{\catvariableof{\nodesa}}{\indexedcatvariableof{\nodesb}}
        &= \normalizationofwrt{\expof{\energytensorat{\catvariableof{\nodesa},\catvariableof{\nodesb}}}}{\catvariableof{\nodesa}}{\indexedcatvariableof{\nodesb}} \\
        &= \frac{\contractionof{\expof{\energytensorat{\catvariableof{\nodesa},\catvariableof{\nodesb}}}}{\catvariableof{\nodesa},\indexedcatvariableof{\nodesb}}}{
            \contractionof{\expof{\energytensorat{\catvariableof{\nodesa},\catvariableof{\nodesb}}}}{\indexedcatvariableof{\nodesb}}
        } \\
        &= \frac{\contractionof{\expof{\energytensorat{\catvariableof{\nodesa},\indexedcatvariableof{\nodesb}}}}{\catvariableof{\nodesa}}}{
            \contraction{\expof{\energytensorat{\catvariableof{\nodesa},\indexedcatvariableof{\nodesb}}}}
        } \\
        &= \normalizationof{
            \expof{\energytensorat{\catvariableof{\nodesa},\indexedcatvariableof{\nodesb}}}
        }{\catvariableof{\nodesa}} \, . \qedhere
    \end{align*}
\end{proof}

% Situation of marginalized variables
Importantly, \theref{the:energyContractionQueries} does not generalize to situations, where $\nodesa\cup\nodesb\neq[\catorder]$, since summation over the indices of the variables $[\catorder]/\nodesa\cup\nodesb$ and contraction do not commute.
In this more generic situation, we would need to sum over exponentiated coordinates, that is
\begin{align*}
    \condprobof{\catvariableof{\nodesa}}{\indexedcatvariableof{\nodesb}}
    = \normalizationof{
        \sum_{\catindexofin{[\catorder]/\nodesa\cup\nodesb}}
        \expof{\energytensorat{\catvariableof{\nodesa},\indexedcatvariableof{\nodesb},\indexedcatvariableof{[\catorder]/\nodesa\cup\nodesb}}}
    }{\catvariableof{\nodesa}} \, .
\end{align*}


% Mode queries
Mode queries on probability distributions in an energy representation can always be reduces to mode queries on the energy tensor.
This is due to the monotonicity of the exponential function, which implies
\begin{align*}
    \argmax_{\shortcatindicesin} \probat{\indexedshortcatvariables}
    &=\argmax_{\shortcatindicesin} \normalizationof{\expof{\energytensorwith}}{\indexedshortcatvariables} \\
    &=\argmax_{\shortcatindicesin} \expof{\energytensorat{\indexedshortcatvariables}} \\
    &=\argmax_{\shortcatindicesin} \energytensorat{\indexedshortcatvariables} \, .
\end{align*}
Since we are only interested in identifying the index of the maximum coordinate, and not its value, we have further dropped the normalization term by partition functions.
When one instead need to get the value of the maximal, the partition function cannot be ignored.

\sect{Sampling}

Let us here investigate how to draw samples from a probability distribution, based on queries on it.
Naive methods, such as drawing a random number in $[0,1]$, adding iteratively the coordinates and stopping when the sum exceeds the random variables, are infeasible when having large tensor orders causing exponential increases of the coordinate number.
We recall, that the number of coordinates of $\probwith$ is $\prod_{\catenumeratorin}\catdimof{\catenumerator}$, which increases exponentially in the number $\catorder$ of the variables.
Efficient methods instead have to exploit tensor network decompositions of the decompositions.

\subsect{Forward Sampling}

The first insight to derive efficient sampling algorithms is to sample a single variable in each step.
Forward sampling (see \algoref{alg:ForwardSampling}) exploits to this end the generic chain decomposition (see \theref{the:chainRule}) of a probability distribution, namely
\begin{align*}
    \probwith = \contractionof{\{\margprobat{\catvariableof{0}}\} \cup
    \left\{\condprobof{\catvariableof{\catenumerator}}{\catvariableof{0},\ldots,\catvariableof{\catenumerator-1}} \wcols \catenumeratorin \ncond \catenumerator\geq 1\right\}
    }{\shortcatvariables} \, ,
\end{align*}
It then samples iteratively a state $\catindexof{\catenumerator}$ for the variable $\catvariableof{\catenumerator}$ conditioned on the previously sampled states, that is from the conditonal distribution
\begin{align*}
    \condprobof{\catvariableof{\catenumerator}}{\indexedcatvariableof{[\catenumerator]}} \, .
\end{align*}
The generic chain decomposition thereby ensures that probability of getting a state $\shortcatindices$ by this procedure coincides with $\probat{\indexedshortcatvariables}$.

\begin{algorithm}[hbt!]
    \caption{Forward Sampling}\label{alg:ForwardSampling}
    \begin{algorithmic}
        \Require Probability distribution $\probtensor$
        \Ensure Exact sample $\catindices$ of $\probtensor$
    \iosepline
        \ForAll{$\catenumeratorin$}
            \State Draw $\catindexof{\catenumerator}\in[\catdimof{\catenumerator}]$ from the conditional distribution
            \begin{align*}
                \condprobof{\catvariableof{\catenumerator}}{\indexedcatvariableof{[\catenumerator]}}
            \end{align*}
        \EndFor
        \State \Return $\catindices$
    \end{algorithmic}
\end{algorithm}

% Bayesian networks
Forward Sampling is especially efficient, when sampling from a Bayesian Network respecting the topological order of its nodes.
More technically, when the parents $\parentsof{\catenumerator}$ of a node $\catenumerator$ are contained in the preceding variables $[\catenumerator]$, we apply the conditional independence assumption (more precisely \theref{the:conditionDropping} in combination with \theref{the:condIndBN}) to get
\begin{align*}
    \condprobof{\catvariableof{\catenumerator}}{\indexedcatvariableof{[\catenumerator]}}
    = \condprobof{\catvariableof{\catenumerator}}{\indexedcatvariableof{\parentsof{\catenumerator}}} \, .
\end{align*}
Since this conditional probability coincides with a local tensor in the Bayesian Network, we can avoid to contract the network for preparing the conditional distribution.
Different to more general Markov Networks, forward sampling from Bayesian Network can therefore be done efficiently by reduction to conditional queries answerable using local tensors.
We note that it is important to sample in the topological order induced by the underlying directed hypergraph, since the computation of generic conditional distributions is also for Bayesian Networks $\mathsf{NP}$-hard (see Chapter~13 in \cite{koller_probabilistic_2009}).
Sampling along the topological variable order requires only tractable to answer conditional queries on the Bayesian Network.

%% Comment on rejection Sampling 
%When sampling from conditional probability distributions, one can sample from the conditioned distribution instead.
%However, the conditioning changes the structure of the distribution, and conditioned Bayesian Networks are not Bayesian Networks on the same graph.
%One ways around is rejection sampling, where one samples from the unconditioned distribution and rejects samples not satisfying the event conditioned on.
%When the event conditioned on is of small probability, methods like rejection sampling will come with large runtimes.

\subsect{Gibbs Sampling}

While we have seen that forward sampling can be performed efficiently on Bayesian Networks, Gibbs sampling can be also performed efficiently for Markov Networks.
Gibbs sampling \algoref{alg:Gibbs} overcomes the intractability problems of sampling steps during forward sampling at the expense of repetitions of the sampling step.
%The tractability of each sampling step is traded off against the number of repetitions to produce a sample.
When performing finite repetitions, Gibbs sampling in general samples from an approximate distribution to the one desired.
It can be shown, that these approximate distribution tend to one desired in the asymptotic limit of infinite repetitions of the sampling step (see Chapter~12 in \cite{koller_probabilistic_2009}).

\begin{algorithm}[hbt!]
    \caption{Gibbs Sampling}\label{alg:Gibbs}
    \begin{algorithmic}
        \Require Probability distribution $\probtensor$
        \Ensure Approximative sample $\catindices$ of $\probtensor$
        \iosepline
        \ForAll{$\catenumeratorin$}
            \State Draw $\catindexof{\catenumerator}$ from an initialization distribution. % In implementation: Initialize with ones and draw -> Avoids zero probability state
        \EndFor
        \While{Stopping criterion is not met}
            \ForAll{$\catenumeratorin$}
                \State Draw $\catindexof{\catenumerator}$ from the conditional distribution
                \begin{align*}
                    \condprobof{\catvariableof{\catenumerator}}{\indexedcatvariableof{[\catorder]/\{\catenumerator\}}}
                \end{align*}
            \EndFor
        \EndWhile
        \State \Return $\catindices$
    \end{algorithmic}
\end{algorithm}

% Efficiency on Markov Networks
The central problem of forward sampling on Markov Networks has been the need for global contractions to answer the required conditional queries, which originates from large numbers of variables to be marginalized out.
When avoiding the marginalization of variables, and conditioning on them instead, global contractions can be avoided.
To be more precise, for any tensor network $\extnet$ on $\graph=([\catorder],\edges)$ and any $\catenumeratorin$ we have
\begin{align*}
    \contractionof{\extnet}{\catvariableof{\catenumerator},\catvariableof{[\catorder]/\{\catenumerator\}}}
    = \contractionof{\{\hypercoreofat{\edge}{\catvariableof{\catenumerator},\indexedcatvariableof{\edge/\{\catenumerator\}}} \wcols \edge\in\edges \ncond \catenumerator\in\edge\}}{\catvariableof{\catenumerator}}
    \cdot \prod_{\edge\in\edges, \, \catenumerator\notin\edge} \hypercoreofat{\edge}{\indexedcatvariableof{\edge}} \, .
\end{align*}
As a consequence, we get for the Markov Network $\probtensor=\probof{\graph}$ to $\extnet$, that
\begin{align*}
    \condprobof{\catvariableof{\catenumerator}}{\indexedcatvariableof{[\catorder]/\{\catenumerator\}}}
    &= \normalizationof{\extnet}{\catvariableof{\catenumerator},\catvariableof{[\catorder]/\{\catenumerator\}}} \\
    &= \normalizationof{\{\hypercoreofat{\edge}{\catvariableof{\catenumerator},\indexedcatvariableof{\edge/\{\catenumerator\}}} \wcols \edge\in\edges, \, \catenumerator\in\edge\}}{\catvariableof{\catenumerator}} \, .
\end{align*}
The conditional queries on a Markov Network asked in Gibbs Sampling can therefore be answered by contractions only of those tensors containing the variable $\catvariableof{\catenumerator}$.
To find further locally answerable conditional queries, we need to condition only on the neighbored variables, referred to as Markov blanket, such that the other variables are conditionally independent.
This follows from the characterization of the conditional independences in Markov Networks, which has been shown in \theref{the:condIndMN}, and can be used to design further tractable sampling schemes for Markov Networks.

% Energy
We can further answer these conditional queries efficiently, when we perform Gibbs sampling on a probability distribution in an energy representation, that is $\probwith=\normalizationof{\expof{\energytensorwith}}{\shortcatvariables}$.
Using \theref{the:energyContractionQueries}, we have
\begin{align*}
    \condprobof{\catvariableof{\catenumerator}}{\indexedcatvariableof{[\catorder]/\{\catenumerator\}}}
    = \normalizationof{\expof{\energytensorat{\catvariableof{\catenumerator},\indexedcatvariableof{[\catorder]/\{\catenumerator\}}}}
    }{\catvariableof{\nodesa}}
\end{align*}
We note, that the main property of the conditional query exploited here, is that all variables but the one sampled appear as a condition and none is marginalized out.
In the scheme of forward sampling, where most of the variables are marginalized out in many queries, we cannot apply this trick and would have to perform sums over exponentiated coordinates to the variables marginalized out.


\subsect{Simulated Annealing}\label{sec:simulatedAnnealing}

Simulated annealing is an adapted sampling scheme targeting mode queries rather than generating representative samples from a distribution.
It employs an annealing process that gradually transforms the probability distribution by increasingly favoring high-likelihood configurations, thereby improving the chances of sampling a solution to a mode query.
To be more precise, let there be a distribution in energy representation, that is $\probwith=\normalizationof{\expof{\energytensorwith}}{\shortcatvariables}$.
We introduce a parameter $\invtemp\in\rr$, which we understand as the inverse temperature, and anneal the distribution through scaling its energy by this parameter.
In the limit of $\invtemp\rightarrow\infty$, for each state $\shortcatindicesin$ the annealed distribution behaves as
\begin{align*}
    \normalizationof{\expof{\invtemp\cdot\energytensorwith}}{\indexedshortcatvariables} \rightarrow
    \normalizationof{\ones_{\argmax_{\shortcatindicesin}\energytensorat{\indexedshortcatvariables}}}{\indexedshortcatvariables} \, .
\end{align*}
In this limit, the annealed distribution tends to the uniform distribution of the maximal coordinates, that is the uniform distribution of the set
\begin{align*}
    \argmax_{\shortcatindicesin}\energytensorat{\indexedshortcatvariables} = \argmax_{\shortcatindicesin}\probat{\indexedshortcatvariables} \, .
\end{align*}
i
To integrate annealing into Gibbs sampling, one chooses a parameter $\invtemp$ for each repetition of a sampling step and sample from the conditioned annealed distribution $\normalizationof{\expof{\invtemp\cdot\energytensorwith}}{\shortcatvariables}$.
Through increasing $\invtemp$ during the algorithm, the samples are drawn towards states with larger coordinates in $\probat{\shortcatvariables}$.
However, when $\invtemp$ is large, the sampling procedure can get stuck in local maxima, whereas small $\invtemp$ are in favor of overcoming such.
The inverse temperature is thus understood as a tradeoff parameter between the exploration of new regions of the state space and increasing the coordinate of the sample by local coordinate optimization.
It is therefore typically chosen low in the beginning of the sampling algorithm and then sequentially increased to find maximal coordinates.
Due to this typical increasement of the inverse temperature strategy, the algorithm is referred to as simulated annealing.


\sect{Maximum Likelihood Estimation}

So far we have been concerned with deductive reasoning task, that is retrieve information from a given distribution or drawing a sample.
We now turn to inductive reasoning tasks, where a probability distribution is estimated given data.
To present the generic framework of maximum likelihood estimation in the tensor network contraction formalism, we introduce the likelihood loss exploiting the structure of empirical distribution, and then provide interpretations in terms of entropies.

\subsect{Empirical Distributions}\label{sec:empDistribution}

To prepare for reasoning on data, we now derive tensor network representation for empirical distributions, which are defined based on observed states $\dataset$ of a factored system.

\begin{definition}
    \label{def:dataMap}
    Given a dataset $\dataset$ of samples of the factored system we define the sample selector map
    \begin{align*}
        \datamap \defcols [\datanum] \rightarrow \facstates
    \end{align*}
    elementwise by
    \begin{align*}
        \datamapat{\datindex} = \left(\catindicesof{\datindex}\right) \, .
    \end{align*}
%% Empirical Distribution
    The empirical distribution to the sample selector map $\datamap$ is the probability distribution
    \begin{align*}
        \empdistributionat{\shortcatvariables}
        \coloneqq \normalizationof{\bencodingofat{\datamap}{\shortcatvariables,\datvariable}}{\shortcatvariables} \, ,
    \end{align*}
    where we introduced as single incoming for the basis encoding of the sample selector map the sample selecting variable $\datvariable$ taking values in $[\datanum]$.
%    where by $\datacore$ we denote the basis encoding (see \defref{def:functionRepresentation}) of the sample selector map, and the distributed variables $\shortcatvariables$ are the head variables of the basis encoding.
\end{definition}

% Sample Selector map representation
The basis encoding of the sample selector map has a decomposition by
\begin{align*}
    \datacoreat{\shortcatvariables,\datvariable}
    = \sum_{\datindexin} \onehotmapofat{\catindicesof{\datindex}}{\shortcatvariables} \otimes \onehotmapofat{\datindex}{\datvariable} \, .
\end{align*}
% Interpretation of the empirical distribution
Each coordinate $\shortcatindices$ of the empirical distribution can thus be calculated by
\begin{align*}
    \empdistributionat{\indexedshortcatvariables}
    & = \frac{1}{\contraction{\datacore}} \left( \sum_{\datindexin} \onehotmapofat{\catindicesof{\datindex}}{\indexedshortcatvariables}  \right) \\
    &= \frac{\cardof{\left\{\datindexin \wcols (\catindicesof{\datindex}) = (\catindices)\right\}}}{\datanum} \, .
\end{align*}
We can therefore interpret each coordinate of the empirical distribution as the relative frequency of the corresponding state in the observed data.
%and is thus interpreted as the frequency of the corresponding world in the data.

%% Basic CP Decomposition
The basis encoding of the sample selector map is a sum of one-hot encodings of the data indices and the corresponding sample states.
Such sums of basis tensors will be further investigated in \secref{sec:basisCP} as basis $\cpformat$ decompositions.
We now exploit this structure to find efficient tensor network decompositions (see \figref{fig:DataDecomposition}) based on matrices encoding its variables.


\begin{figure}[t]
    \begin{center}
        \begin{tikzpicture}[scale=0.35, thick] % , baseline = -3.5pt


\begin{scope}[shift={(-15,2)}]

\node[anchor=center] (text) at (-1,3) {${a)}$};


\node [circle, draw, thick, fill=gray!50, minimum size = \nodeminsize] (T1) at (0,0) {\tiny $\randomxof{0}$};	
\node [circle, draw, thick, fill=gray!50, minimum size = \nodeminsize] (T2) at (3,0) {\tiny $\randomxof{1}$};	
\node[anchor=center] (text) at (6,0) {\small $\cdots$};
\node [circle, draw, thick, fill=gray!50,minimum size = \nodeminsize] (T3) at (9,0) {};
\node[anchor=center] (text) at (9,0) {\tiny $\randomxof{\atomorder-1}$};	

\node [circle, draw, thick, fill=gray!50, minimum size = \nodeminsize] (C) at (4.5,-5) {};
\node[anchor=center] (text) at (4.5,-5){ \tiny $\datavariable$};	

\draw[->] (C) -- (T1) node [midway,left] {$\edgeof{0}$};
\draw[->] (C) -- (T2) node [midway,right] {$\edgeof{1}$};
\draw[->] (C) -- (T3) node [midway,right] {$\edgeof{\catorder-1}$};

\end{scope}

\node[anchor=center] (text) at (-1,5) {${b)}$};




\drawatomindices{0}{2}
\draw (-1,1) rectangle (5,-1);
\node[anchor=center] (text) at (2,0) {$\datacore$};
\draw[<-] (2,-1) -- (2,-3) node[midway, right] {\tiny $\datavariable$};

\node[anchor=center] (text) at (7,0) {${=}$};


\begin{scope}[shift={(10,2)}]

\newcommand{\conposseldec}{4.5,-5.5}

\draw[fill] (\conposseldec) circle (0.25cm);
\draw[<-] (\conposseldec) -- (4.5,-7.5) node[midway, right] {\tiny $\datavariable$};
%\draw[dashed] (3.5,-7.5) rectangle (5.5, -9.5);
%\node[anchor=center] (text) at (4.5,-8.5) {\small $\frac{1}{\datanum} \ones$};

\draw[<-] (0,1) -- (0,-1) node[midway,left] {\tiny $\catvariableof{0}$};
\draw (-1,-1) rectangle (1, -3);
\node[anchor=center] (text) at (0,-2) {\small $\datacoreof{0}$};
\draw[<-] (0,-3) to[bend right=20] (\conposseldec);


\draw[<-] (3,1) -- (3,-1) node[midway,left] {\tiny $\catvariableof{1}$};
\draw (2,-1) rectangle (4, -3);
\node[anchor=center] (text) at (3,-2) {\small $\datacoreof{1}$};
\draw[<-] (3,-3) to[bend right=20]  (\conposseldec);

\node[anchor=center] (text) at (6,-2) {$\cdots$};

\draw[<-] (9,1) -- (9,-1) node[midway,left] {\tiny $\catvariableof{\atomorder-1}$};
\draw (7.75,-1) rectangle (10.25, -3);
\node[anchor=center] (text) at (9,-2) {\small $\datacoreof{\atomorder-1}$};
\draw[<-] (9,-3) to[bend left=20]  (\conposseldec);




\end{scope}

		


\end{tikzpicture}
    \end{center}
    \caption{
        Decomposition of the basis encoding of a sample selector map to a dataset $\dataset$.
        a) Interpretation as a sample selection variable $\datvariable$ selecting states for the variables $\shortcatvariables$ according to the enumerated dataset.
        b) Corresponding decomposition of the basis encoding $\datacore$ into a tensor network in the basis $\cpformat$ Format (see \secref{sec:basisCP}), where $\hypercoreof{\edgeof{\atomenumerator}}=\datacoreof{\atomenumerator}$.
    }
    \label{fig:DataDecomposition}
\end{figure}


\begin{theorem}
    \label{the:empCPRep}
    Given a data map $\datamap: [\datanum] \rightarrow \facstates$ we define for $\catenumeratorin$ its coordinate maps
    \begin{align*}
        \datamap_{\catenumerator} \defcols [\datanum] \rightarrow [\catdimof{\catenumerator}]
    \end{align*}
    by
    \begin{align*}
        \datamap_{\catenumerator}(\datindex) = \catindexof{\catenumerator}^\datindex \, .
    \end{align*}
    We then have
    \begin{align*}
        \bencodingofat{\datamap}{\shortcatvariables,\datvariable}
        = \contractionof{
            \{\bencodingofat{\datamap^{\atomenumerator}}{\catvariableof{\atomenumerator},\datvariable} \wcols \atomenumeratorin \}
        }{\shortcatvariables,\datvariable}
    \end{align*}
    and
    \begin{align*}
        \empdistributionat{\shortcatvariables}
        = \contractionof{\datacoreat{\shortcatvariables,\datvariable}, \frac{1}{\datanum}\onesat{\datvariable}}{\shortcatvariables}
        = \contractionof{\datacoreofat{0}{\catvariableof{0},\datvariable},\ldots,\datacoreofat{\catorder-1}{\catvariableof{\catorder-1},\datvariable},\frac{1}{\datanum}\onesat{\datvariable}}{\shortcatvariables} \, .
    \end{align*}
    In a contraction diagram this decomposition is represented as
    \begin{center}
        \begin{tikzpicture}[scale=0.35, thick] % , baseline = -3.5pt






\drawatomindices{0}{2}
\draw (-1,1) rectangle (5,-1);
\node[anchor=center] (text) at (2,0) {$\empdistribution$};


\node[anchor=center] (text) at (7,0) {${=}$};

\node[anchor=center] (text) at (22,-1) {${\cdot}$};

\begin{scope}[shift={(10,2)}]

\newcommand{\conposseldec}{4.5,-5.5}

\draw[fill] (\conposseldec) circle (0.15cm);
\draw[-<-] (\conposseldec) -- (4.5,-7.5) node[midway, right] {\tiny $\datvariable$};
\draw (3.5,-7.5) rectangle (5.5, -9.5);
\node[anchor=center] (text) at (4.5,-8.5) {\small $\frac{1}{\datanum} \ones$};

\draw[-<-] (0,1) -- (0,-1) node[midway,left] {\tiny $\catvariableof{0}$};
\draw (-1,-1) rectangle (1, -3);
\node[anchor=center] (text) at (0,-2) {\small $\datacoreof{0}$};
\draw[-<-] (0,-3) to[bend right=20] (\conposseldec);


\draw[-<-] (3,1) -- (3,-1) node[midway,left] {\tiny $\catvariableof{1}$};
\draw (2,-1) rectangle (4, -3);
\node[anchor=center] (text) at (3,-2) {\small $\datacoreof{1}$};
\draw[-<-] (3,-3) to[bend right=20]  (\conposseldec);

\node[anchor=center] (text) at (6,-2) {$\cdots$};

\draw[-<-] (9,1) -- (9,-1) node[midway,left] {\tiny $\catvariableof{\atomorder-1}$};
\draw (7.75,-1) rectangle (10.25, -3);
\node[anchor=center] (text) at (9,-2) {\small $\datacoreof{\atomorder-1}$};
\draw[-<-] (9,-3) to[bend left=20]  (\conposseldec);



\end{scope}

		


\end{tikzpicture}
    \end{center}
\end{theorem}
\begin{proof}
    The first claim is a special case of \theref{the:functionDecompositionBasisCP}, to be shown in \charef{cha:basisCalculus}.
    To show the second claim we notice
    \[ \contraction{\datacore} = \sum_{\datindexin} \contraction{\bencodingofat{\datamap}{\shortcatvariables,\datvariable=\datindex}} = \datanum \,  . \]
    With the first claim it now follows that
    \begin{align*}
        \empdistributionat{\shortcatvariables}
        &= \normalizationof{\datacore}{\shortcatvariables}
        = \frac{\contractionof{\datacore}{\shortcatvariables}}{\contraction{\datacore}} \\
        &=  \contractionof{
            \left\{\bencodingofat{\datamap^{\atomenumerator}}{\catvariableof{\atomenumerator},\datvariable}\wcols\atomenumeratorin \right\} \cup \left\{\frac{1}{\datanum}\onesat{\datvariable}\right\}
        }{\shortcatvariables}  \, . \qedhere
    \end{align*}
\end{proof}


The cores $\datacoreof{\atomenumerator}$ are matrices storing the value of the categorical variable $\catvariableof{\atomenumerator}$ in the sample world indexed by $\datindex$.

% Interpretation
From the proof of \theref{the:empCPRep} we notice that the scalar $\frac{1}{\datanum}$ could be assigned with any core in a representation of $\empdistribution$, and the core $\onesat{\datvariable}$ is thus redundant in the contraction representation.
However, creating the core $\frac{1}{\datanum}\onesat{\datvariable}$ provides us with a simple interpretation of the empirical distribution.
We can understand $\frac{1}{\datanum}\onesat{\datvariable}$ as the uniform probability distribution over the samples, which is by the map $\datamap$ forwarded to a distribution over $\facstates$.
The one-hot encoding of each sample is itself a probability distribution, which is understood as conditioned on the respective state of the sample selection variable $\datvariable$.
The conditional distribution $\datacore$ therefore forwards the uniform distribution of the samples to a distribution of the variables $\shortcatvariables$.
In the perspective of a Bayesian Network (see \figref{fig:DataDecomposition}), the variable $\datvariable$ serves as single parent for each categorical variable $\catvariableof{\catenumerator}$.


%%% Inductive vs Deductive perspective
%Each evidence is a probability distribution
%\begin{itemize}
%	\item Inductive Reasoning: When we interpret evidence as a datapoint, they are typically a basis tensor specifying precisely a world.
%	\item Deductive Reasoning: Evidence is a partial observation of the world, typically basis vectors at each variable, but leaving some unspecified ($\ones$).
%	We then interpret the evidence as being a uniform distribution over the worlds not contradicting with the evidence.
%\end{itemize}


\subsect{Likelihood Loss}

%Following the notation in \secref{sec:empDistribution}, datasets $\dataset$ are collections of observed samples
%\begin{align*}
%    \datamapat{\datindex} = (\catindicesof{\datindex}) \in \facstates
%\end{align*}
%of a factored system.
The likelihood of a probability distribution $\probwith$ to produce an observed sample is
\begin{align*}
    \probat{\shortcatvariables = \datamapat{\datindex}} \, .
\end{align*}
We further introduce the likelihood of $\probwith$ with respect to a dataset as
\begin{align*}
    \probat{\dataset} := \prod_{\datindexin} \probat{\shortcatvariables = \datamapat{\datindex}} \, .
\end{align*}
The likelihood draws on the assumption, that each datapoint in the dataset has been drawn independently from the same distribution.
When this generating distribution coincides with $\probwith$, then the probability of generating a dataset by this scheme is the likelihood.
In inductive reasoning, the true distribution $\probwith$ is unknown and needs to be approximatively estimated based on data and a learning hypothesis.
We will therefore compute and compare the likelihood of distributions, which in general do not coincide with the distribution generating the data.
It is thus important to not understand the likelihood as a probability, which is only true for the generating distribution, as pointed out in Chapter~2 in \cite{mackay_information_2003}.

% Logarithm
Let us now transform the likelihood to find an representation involving the empirical distribution (see \defref{def:dataMap}), for which efficient tensor network decompositions have been derived in \theref{the:empCPRep}.
Applying a scaled logarithm we get
\begin{align*}
    \frac{1}{\datanum} \cdot \lnof{\probat{\dataset}}
    &= \frac{1}{\datanum} \cdot \lnof{\prod_{\datindexin} \probat{\shortcatvariables = \datamapat{\datindex}}} \\
    &= \dataaverage \lnof{\probat{\shortcatvariables = \datamapat{\datindex}}} \\
    &= \contraction{\lnof{\probwith},\empdistributionwith} \, .
\end{align*}
Let us notice, that this transform of the likelihood is monotoneous and therefore does not influence the position of the maximum, when optimizing the likelihood.
Motivated by this property, we use the transformed form to define the loss-likelihood loss and maximum likelihood estimation.
%This is especially useful, when investigating the convergence of the objective for $\datanum\rightarrow\infty$ (see \charef{cha:concentration}).

\begin{definition}
    \label{def:loss}
    The log-likelihood loss of a distribution $\probtensor$ given a dataset
    \begin{align*}
        \dataset
    \end{align*}
    is the functional
    \begin{align*}
        \lossof{\probtensor}
        = -\contractionof{\lnof{\probwith},\empdistributionwith} \, .
    \end{align*}
    Having a hypothesis $\probtensorset\subset\bmrealprobof{\ones}$, that is a set of probability distributions, the maximum likelihood estimation is the problem
    \begin{align}
        \label{prob:parameterMaxLikelihood}\tag{$\probtagtypeinst{\mprojectionsymbol}{\probtensorset,\empdistribution}$}
        \argmin_{\probtensorin} \lossof{\probtensor} \, .
    \end{align}
\end{definition}


\subsect{Entropic Interpretation}

The Shannon entropy, which has been introduced in the seminal paper \cite{shannon_mathematical_1948}, is a foundational concept in various research fields beyond statistical learning, such as information theory or statistical physics.
While a detailed discussion is out of the scope of this work, we here only provide computation schemes of the entropy based on contractions of distributions.

\begin{definition}[Shannon entropy]
    The Shannon entropy of a distribution $\probwith$ is the quantity
    \begin{align*}
        \sentropyof{\probtensor}
        =  \sum_{\shortcatindicesin} \probat{\indexedshortcatvariables} \cdot \left(-\lnof{\probat{\indexedshortcatvariables}}\right)
        = \contraction{\probtensor,-\lnof{\probtensor}} \, .
    \end{align*}
    We represent the Shannon entropy by the tensor network diagram
    \begin{center}
        \begin{tikzpicture}[scale=0.3,thick] % , baseline = -3.5pt

\node[anchor=center] (text) at (-8,-5) {\corelabelsize $\sentropyof{\probtensor}$};

\node[anchor=center] (text) at (-5,-5) {\corelabelsize ${=}$};

\node[anchor=center] (text) at (-3,-2) {\corelabelsize $\mathrm{ln}$};
\draw (2,-2) ellipse (6 and 2.75);

\draw (-1,-1) rectangle (5,-3);
\node[anchor=center] (text) at (2,-2) {\corelabelsize $\probtensor$};
\draw (-1,-7) rectangle (5,-9);
\node[anchor=center] (text) at (2,-8) {\corelabelsize $\probtensor$};
\draw (0,-5)--(0,-3); 
\draw (0,-5)--(0,-7) node[midway,left] {\colorlabelsize $\catvariableof{0}$};
\draw (1.5,-5)--(1.5,-3); 
\draw (1.5,-5)--(1.5,-7) node[midway,left] {\colorlabelsize $\catvariableof{1}$};
\node[anchor=center] (text) at (3,-4) {$\cdots$};
\draw (4,-5)--(4,-3);
\node[anchor=center] (text) at (3,-6) {$\cdots$};
\draw (4,-5)--(4,-7) node[midway,right] {\colorlabelsize $\catvariableof{\catorder\shortminus1}$};

%\drawatomcore{3.5}{-8}{$\probtensor$}
%\drawatomindices{3.5}{-12}	
%\draw (5.5,-9)--(5.5,-7) node[midway,right] {\colorlabelsize $\catvariableof{\exformula}$};

\end{tikzpicture}
    \end{center}
    where we denote a coordinatewise transform by the logarithm as an ellipsis (see \secref{sec:coordinatewiseTransforms}).
\end{definition}

We here make the convention $\lnof{0}=-\infty$ and $0\cdot\lnof{0} = 0$, to have the Shannon entropy well-defined for distributions with non-trivial support.

Among the distributions in the same tensor space, the uniform distribution maximizes the Shannon entropy
\begin{align*}
    \sentropyof{\normalizationof{\ones}{\shortcatvariables}} = \sum_{\catenumeratorin}\lnof{\catdimof{\catenumerator}}
\end{align*}
and the one-hot encodings to states $\shortcatindicesin$ minimize the Shannon entropy
\begin{align*}
    \sentropyof{\onehotmapofat{\shortcatindices}{\shortcatvariables}} = 0 \, .
\end{align*}
%We understand the samples of the uniform concentration as having the maximal information con

The Shannon entropy measures the information content of a distribution and is therefore a central tool for regularization in statistical learning (see for an introduction Chapter~2 in \cite{mackay_information_2003}).
We therefore exploit this information content as a regularizer to identify a distribution among those coinciding in the answer to a collection of expectation queries.
To be more precise, let there be for $\selindexin$ query tensors $\exfunction^{\selindex}$ (see \defref{def:queries}) and $\empdistribution$ an empirical distribution.
The problem of maximal entropy with respect to coinciding expectation queries with $\empdistribution$ is then posed as
\begin{align*}
    \argmax_{\probtensor\in\bmrealprobof{\ones}} \sentropyof{\probtensor} \quad \text{subject to} \quad \uniquantwrtof{\selindexin}{\contraction{\probtensor,\exfunction^{\selindex}} = \contraction{\empdistribution,\exfunction^{\selindex}}}
\end{align*}
where $\bmrealprobof{\ones}$ is the set of probability distributions given a factored system.
We study instances of this maximal entropy problem later in \secref{sec:maxEntProblem}, where we show that its solution is a member of the exponential family, which statistic is build by the query tensors.
We will further provide connections between the problems of maximal entropy and maximum likelihood estimation.

While the Shannon entropy is a property of a single distribution, the cross-entropy is a straight forward generalization towards pairs of distributions.
We first introduce this quantity and then interpret the log-likelihood loss based on the cross-entropy.

\begin{definition}[Cross entropy and Kullback Leibler divergence]
    \label{def:crossEntropy}
    The cross-entropy between two distributions $\probwith$ and $\secprobat{\shortcatvariables}$ defined with respect to the same factored system is the quantity
    \begin{align*}
        \centropyof{\probtensor}{\secprobtensor}
        = \sum_{\catindices}  \probat{\indexedshortcatvariables} \cdot \left(-\lnof{\secprobat{\indexedshortcatvariables}}\right)
        = \contraction{\probtensor,-\lnof{\secprobtensor}} \, .
    \end{align*}
    The cross-entropy is captured by the tensor network diagram
    \begin{center}
        \begin{tikzpicture}[scale=0.3,thick] % , baseline = -3.5pt

\node[anchor=center] (text) at (-8,-5) {\corelabelsize $\centropyof{\probtensor}{\tilde{\probtensor}}$};

\node[anchor=center] (text) at (-5,-5) {\corelabelsize ${=}$};

\node[anchor=center] (text) at (-3,-2) {\corelabelsize $\mathrm{ln}$};
\draw (2,-2) ellipse (6 and 2.75);

\draw (-1,-1) rectangle (5,-3);
\node[anchor=center] (text) at (2,-2) {\corelabelsize $\tilde{\probtensor}$};
\draw (-1,-7) rectangle (5,-9);
\node[anchor=center] (text) at (2,-8) {\corelabelsize $\probtensor$};
\draw (0,-5)--(0,-3); 
\draw (0,-5)--(0,-7) node[midway,left] {\colorlabelsize $\catvariableof{0}$};
\draw (1.5,-5)--(1.5,-3); 
\draw (1.5,-5)--(1.5,-7) node[midway,left] {\colorlabelsize $\catvariableof{1}$};
\node[anchor=center] (text) at (3,-4) {$\cdots$};
\draw (4,-5)--(4,-3);
\node[anchor=center] (text) at (3,-6) {$\cdots$};
\draw (4,-5)--(4,-7) node[midway,right] {\colorlabelsize $\catvariableof{\catorder\shortminus1}$};

%\drawatomcore{3.5}{-8}{$\probtensor$}
%\drawatomindices{3.5}{-12}	
%\draw (5.5,-9)--(5.5,-7) node[midway,right] {\colorlabelsize $\atomlegindexof{\exformula}$};

\end{tikzpicture}
    \end{center}
    The Kullback-Leiber divergence or relative entropy between $\probwith$ and $\secprobat{\shortcatvariables}$ is the quantity
    \begin{align*}
        \kldivof{\probtensor}{\secprobtensor} = \centropyof{\probtensor}{\secprobtensor} - \sentropyof{\probtensor}  \, .
    \end{align*}
\end{definition}

%% Vanishing coordinates case
Let us notice, that we have $\centropyof{\probtensor}{\secprobtensor} = \infty$ if and only if there is a state $\shortcatindicesin$ such that $\probat{\indexedshortcatvariables}>0$ and $\secprobtensor[\indexedshortcatvariables]=0$.

% KL Divergence
The Gibbs inequality (for a proof see for example Chapter~2 in \cite{cover_elements_2006}) states that for any distributions $\probwith$ and $\secprobat{\shortcatvariables}$ we have
\begin{align*}
    \centropyof{\probtensor}{\secprobtensor} \geq \sentropyof{\probtensor} \, ,
\end{align*}
where equality holds if any only if $\probtensor=\secprobtensor$
This ensures, that the Kullback-Leiber Divergence between any distributions is positive and vanishes if and only if both distributions coincide.

We in the next lemma provide an entropic interpretation of maximum likehlihood estimation as defined in \defref{def:loss}.

\begin{lemma}
    \label{lem:centropyMLE}
    The maximum likelihood estimation \probref{prob:parameterMaxLikelihood} is equivalent to the minimization of cross-entropy and Kullback-Leibler divergence, that is
    \begin{align*}
        \argmin_{\probtensorin} \lossof{\probtensor}
        = \argmin_{\probtensorin} \centropyof{\empdistribution}{\probtensor}
        = \argmin_{\probtensorin} \kldivof{\empdistribution}{\probtensor} \, .
    \end{align*}
\end{lemma}
\begin{proof}
    Comparing the log-likelihood loss in \defref{def:loss} with the cross-entropy in \defref{def:crossEntropy}, we get
    \begin{align*}
        \lossof{\probtensor} = \centropyof{\empdistribution}{\probtensor} \,
    \end{align*}
    which established the equivalence of maximum likelihood estimation and cross-entropy minimization.
    Further, since
    \begin{align*}
        \kldivof{\empdistribution}{\probtensor} = \centropyof{\empdistribution}{\probtensor} - \sentropyof{\empdistribution}
    \end{align*}
    and $\sentropyof{\empdistribution}$ is a constant offset in the objective, maximum likelihood estimation is equivalent to the minimization of the Kullback-Leibler divergence.
\end{proof}

% M-projection
More general than Maximum Likelihood Estimation, we define the moment projection of an arbitrary distribution $\gendistribution$ onto a set $\probtensorset$ of probability distributions as the problem
\begin{align}
    \label{prob:mProjection}\tag{$\probtagtypeinst{\mprojectionsymbol}{\probtensorset,\gendistribution}$}
    \argmin_{\probtensorin} \centropyof{\gendistribution}{\probtensor} \, . % Also give the KL divergence version?
\end{align}
With \lemref{lem:centropyMLE} we have established, that maximum likelihood estimation is the moment projection of the empirical distribution onto a set $\probtensorset$.

% I-projection
We further define the information projection an arbitrary distribution $\gendistribution$ onto a set $\probtensorset$ of probability distributions as the problem
\begin{align}
    \label{prob:iProjection}\tag{$\probtagtypeinst{\iprojectionsymbol}{\probtensorset,\gendistribution}$}
    \argmin_{\probtensorin} \kldivof{\probtensor}{\gendistribution} \, .
\end{align}
The relative entropy is not symmetric, thus the information projection \eqref{prob:iProjection} and the moment projections do not coincide in general.
The differences of both are discussed in Chapter~8 in \cite{koller_probabilistic_2009}.

\begin{example}[Cross entropy with respect to exponential families]
    \label{exa:cEntropyExp}
    If $\secprobtensor$ is a member of an exponential family, we have %with the representation from \lemref{lem:energyCumulantRepresentation}
    \begin{align*}
        \centropyof{\probtensor}{\expdist}
        = \contraction{\probtensor,\lnof{\expdist}}
        = \contraction{\probtensor,\sencsstat,\canparam} - \cumfunctionof{\canparam} + \contraction{\probtensor,\lnof{\basemeasure}} \, .
    \end{align*}
    The last term vanishes, given the convention $0\cdot\lnof{0}=0$, if and only if for any $\shortcatindices$ with $\basemeasureat{\indexedshortcatvariables}=0$ we have $\probat{\indexedshortcatvariables}=0$, and is infinite instead.
    Therefore, the cross entropy between a distribution and a member of an exponential family is finite, if and only if the distribution is representable with respect to the base measure $\basemeasure$ (see \defref{def:representationBaseMeasure}).
    If $\probtensor$ is representable with respect to $\basemeasure$, we can abbreviate the cross-entropy to
    \begin{align*}
        \centropyof{\probtensor}{\expdist}
        = \contraction{\probtensor,\expenergy} -\cumfunctionof{\canparam} \, .
    \end{align*}
\end{example}



\sect{Variational Inference in Exponential Families}

With the maximum likelihood method we have already formulated inference tasks by optimization problem with variations of the probability distribution.
In a broader perspective, we now investigate variational inference problems using the canonical and mean parametrization of exponential families.

\subsect{Forward and Backward Maps}

While defined for arbitrary distributions (see \defref{def:meanPolytope}), we now consider mean parameters to members of exponential families.
First of all, they provide an alternative parameterization to the canonical parameter $\canparam$.
The computation of the mean parameter to a given canonical parameter and vice versa are the central inference problems in exponential families.
We first formalize these inference problems by the forward and backward mapping and then provide in this section further insights into these mappings.

\begin{definition}
    \label{def:meanForwardBackward}
    Let $\sstat$ be a statistic and $\basemeasure$ a base measure and consider the exponential family $\genexpfamily$
    The map
    \begin{align*}
        \forwardmap \defcols \parspace\rightarrow\genmeanset\subset\parspace
        \defspace \forwardmapof{\canparam} = \contractionof{\expdistwith,\sencsstatwith}{\selvariable}
    \end{align*}
    is called the forward map of the exponential family and inverse, that is a map
    \begin{align*}
        \backwardmap \defcols \imageof{\forwardmap} \subset\parspace\rightarrow \parspace
    \end{align*}
    with $\expdistof{(\sstat,\backwardmapof{\forwardmapof{\canparam}},\basemeasure)} = \expdist$ for any $\canparamwithin$, a backward map of the exponential family.
\end{definition}

% Domain of \forwardmap
We notice, that the domain of $\forwardmap$ is always $\parspace$, since the coordinates of $\forwardmapof{\canparam}$ are for any $\canparam\in\parspace$ summations over finitely many products.

%
We already know by \theref{the:meanPolytopeInteriorCharacterization}, that distributions representable by $\basemeasure$ have mean parameters in the interior of $\genmeanset$.
We now state that the elements of the corresponding exponential family $\expfamily$, which are by construction representable by $\basemeasure$, are expressive enough to reproduce the whole interior of $\genmeanset$.

\begin{theorem}
    \label{the:meanPolytopeInterior}
    For any exponential family $\genexpfamily$ the image of the forward mapping is the effective interior (see \defref{def:effectiveInterior}) of $\genmeanset$, that is
    \begin{align*}
        \imageof{\forwardmap} = \sbinteriorof{\genmeanset} \, .
    \end{align*}
\end{theorem}
\begin{proof}
    In case of minimal statistics, we refer for the proof of this statement to Theorem~3.3 in \cite{wainwright_graphical_2008}.
    If $\sstat$ is not minimal, we find a subset $\secsstat$ of its features such that $\secsstat$ is minimal with respect to $\basemeasure$ and there is a matrix $\matrixat{\selvariable,\secselvariable}$ such that for any distribution
    \begin{align*}
        \contractionof{\probat{\shortcatvariables},\sencodingofat{\secsstat}{\shortcatvariables,\secselvariable},\matrixat{\selvariable,\secselvariable}}{\selvariable}
        = \contractionof{\probat{\shortcatvariables},\sencodingofat{\sstat}{\shortcatvariables,\selvariable}}{\selvariable} \, .
    \end{align*}
    This subset $\secsstat$ can be found by iteratively identifying $\vectorat{\selvariable}$ and $\lambda$ such that the condition in \defref{def:minimalStatistics} is violated, and dropping a feature $\sstatcoordinateof{\selindex}$ with $\vectorat{\indexedselvariable}\neq0$.
    At each manipulation step the expressivity of the exponential family stays constant and thus $\expfamilyof{\secsstat,\basemeasure}=\expfamilyof{\sstat,\basemeasure}$.
    The matrix $\matrixat{\selvariable,\secselvariable}$ can be constructed based on the linear dependencies of the dropped features on the remaining.
    The procedure terminates, when there is no pair $\vectorat{\selvariable},\lambda$, which is equal to $\secsstat$ being minimal with respect to $\basemeasure$.
%    \begin{align*}
%        \sencodingofat{\sstat}{\shortcatvariables,\selvariable}
%        = \contractionof{\sencodingofat{\secsstat}{\shortcatvariables,\secselvariable},\matrixat{\selvariable,\secselvariable}}{\shortcatvariables,\selvariable} \, .
%    \end{align*}
    We then have %$\expfamilyof{\secsstat,\basemeasure}=\expfamilyof{\sstat,\basemeasure}$ and
    \begin{align*}
        \meansetof{\sstat,\basemeasure} / \bigcup_{\genfaceset\neq\genmeanset} \genfaceset
        = \{\contractionof{\meanparamat{\secselvariable},\matrixat{\selvariable,\secselvariable}}{\selvariable}\wcols\meanparamat{\secselvariable}\in\sbinteriorof{\meansetof{\secsstat,\basemeasure}}\} \, .
    \end{align*}
    and using that $\secsstat$ is minimal we get
    \begin{align*}
        \imageof{\forwardmap} = \genmeanset / \bigcup_{\genfaceset\neq\genmeanset} \genfaceset \, . & \qedhere
    \end{align*}
\end{proof}

Forward and backward maps in an exponential family $\expfamily$ are the central classes of inference, which transform the description of a member by a canonical parameter into mean parameters and vise versa.
Forward maps calculate to a canonical parameter $\canparamwith$ the corresponding mean parameter $\meanparamwith$.
For any $\canparamwith$ we have a closed form representation of this expectation query by the moment matching condition
\begin{align*}
    \meanparamwith = \contractionof{\sencsstatwith,\expdistwith}{\selvariable} \, .
\end{align*}
The forward map is thus a collection of expectation queries (see \defref{def:queries}) to compute the coordinates of the mean parameter.
The query $\sstatcoordinateof{\selindex}$ asked against $\expdistwith$ computes the coordinate $\meanparamat{\indexedselvariable}$.
% Infeasibility and turn to variational alternatives with selection encodings.
The contraction by the moment matching condition can, however, be infeasible, since it requires the instantiation of the probability distribution, which can be done by basis encodings of the statistic.
In this section, we provide alternative characterization of the forward map and approximations of it, which can be computed based on the selection encoding instead.
Following \cite{wainwright_graphical_2008}, we can characterize the forward mapping to exponential families as a variational problem and provide an alternative characterization to this contraction.


\subsect{Variational Formulation}

We now formulate the forward and backward inference problems in exponential families as convex optimization problems.
To this end we use the cumulative function $\cumfunction$ and its conjugate dual
%The conjugate dual of the cumulative function is
\begin{align*}
    \dualcumfunction(\meanparam) = \max_{\canparam\in\parspace} \contraction{\meanparam,\canparam} - \cumfunctionof{\canparam} \, .
\end{align*}

\begin{lemma}
    \label{lem:gradientCumfunction}
    Let $\sstat$ be a statistic, $\basemeasure$ a basemeasure.
    Let us further choose $\canparamwith\in\parspace$ and let $\meanparamwith$ be the mean parameter to $\expdist$.
    For the gradient of $\cumfunction$ evaluated at $\canparam$ we have
    \begin{align*}
        \gradwrtat{\seccanparamat{\selvariable}}{\canparam} \cumfunctionof{\seccanparam} = \meanparamat{\selvariable}
    \end{align*}
    If the statistic $\sstat$ is minimal with respect to $\basemeasure$, then also
    \begin{align*}
        \gradwrtat{\secmeanparamat{\selvariable}}{\meanparam} \dualcumfunctionof{\secmeanparam}
        = \canparamat{\selvariable} \, .
    \end{align*}
\end{lemma}
\begin{proof}
    We have
    \begin{align*}
        \gradwrtat{\canparam}{\seccanparam} \cumfunctionof{\seccanparam}
        & = \gradwrtat{\canparam}{\seccanparam} \lnof{\contraction{\expof{\contractionof{\sencsstatwith,\canparamwith}{\shortcatvariables}},\basemeasure}} \\
        & = \frac{
            \contractionof{\sencsstatwith,\expof{\contractionof{\sencsstatwith,\canparamwith}{\shortcatvariables}},\basemeasure}{\selvariable}
        }{\contraction{\expof{\contractionof{\sencsstatwith,\canparamwith}{\shortcatvariables}}}}  \\
        & = \contractionof{\expdistwith,\sencsstatwith}{\selvariable} \, .
    \end{align*}
    For the proof of the second claim we refer to Appendix B.2 in \cite{wainwright_graphical_2008}.
\end{proof}

\begin{theorem}[Variational backward map]
    For any $\meanparam\in\interiorof{\meanset}$ choose
    \begin{align*}
        \canparamat{\selvariable} \in \argmax_{\canparam} \contraction{\meanparam,\canparam} - \cumfunctionof{\canparam} \, .
    \end{align*}
    Then $\meanparam$ is the mean parameter to $\expdist$.
\end{theorem}
\begin{proof}
    The maximization over $\canparam$ is an unconstrained concave maximization problem and the optimum is characterized by
    \begin{align*}
        0 = \gradwrtat{\canparam}{\seccanparam}\left(\contraction{\meanparam,\canparam} - \cumfunctionof{\canparam}\right)
    \end{align*}
    which reads
    \begin{align*}
        \meanparamat{\selvariable} = \gradwrtat{\canparam}{\seccanparam[\selvariable]} \cumfunctionof{\canparam} \, .
    \end{align*}
    With \lemref{lem:gradientCumfunction}, the gradient vanishes if and only if $\meanparam$ is the mean parameter to $\expdist$.
\end{proof}

The backward map is closely connected with the conjugate dual of $\cumfunction$.
In particular, while the conjugate dual is defined by the maximization problem
\begin{align*}
    \dualcumfunctionof{\meanparam} = \max_{\canparam\in\parspace} \contraction{\meanparam,\canparam} - \cumfunctionof{\canparam}
\end{align*}
the backward map returns the position of the maximum in $\parspace$.

\begin{lemma}
    \label{lem:dualCumEntropy}
    For any $\canparam$ with mean parameter $\meanparam$ we have
    \begin{align*}
        \dualcumfunction(\meanparam) = - \sentropyof{\expdist} \, .
    \end{align*}
\end{lemma}
\begin{proof}
    We have
    \begin{align*}
        \canparam \in \argmax_{\canparam} \contraction{\meanparam,\canparam} - \cumfunctionof{\canparam}
    \end{align*}
    if and only if
    \begin{align*}
        \meanparamwith = \contractionof{\expdistwith,\sencsstatwith}{\selvariable} \, .
    \end{align*}
    Therefore
    \begin{align*}
        \dualcumfunction(\meanparam)
        &= \contraction{\expdistwith,\sencsstatwith,\canparam} - \cumfunctionof{\canparam} \\
        &= \contraction{\expdistwith, \lnof{\expdistwith}} \\
        &= - \sentropyof{\expdist} \, .  \qedhere
    \end{align*}
\end{proof}

We can use this insight to provide a variational characterization of the forward mapping.

\begin{theorem}[Variational forward mapping]
    Let $\sstat$ be a minimal statistic with respect to $\basemeasure$.
    Given $\canparam$, there is a unique $\meanparam$ with
    \begin{align*}
        \meanparam \in \argmax_{\meanparam\in\meanset} \contraction{\meanparam,\canparam} + \sentropyof{\expdistof{\sstat,\meanparam,\basemeasure}} %-  \dualcumfunction(\meanparam)
    \end{align*}
    and $\meanparam$ is the mean parameter to $\expdistwith$.
    Here, we denote by $\expdistof{\sstat,\meanparam,\basemeasure}$ the member of the exponential family with the mean parameter $\meanparam$.
\end{theorem}
\begin{proof}
    By strong duality, we have $\left(\cumfunction\right)^{**}=\cumfunction$ and
    \begin{align*}
        \cumfunction(\canparam) = \max_{\meanparam\in\meanset} \contraction{\meanparam,\canparam} -  \dualcumfunction(\meanparam) \, .
    \end{align*}
    The statement then follows from the first order condition, where we use the gradient \lemref{lem:gradientCumfunction}, and the characterization of $\dualcumfunction$ by \lemref{lem:dualCumEntropy}. % Do we need to assume minimal statistics?
\end{proof}

\subsect{Maximum Likelihood Problem on Exponential Families}

We now characterize the solution of the Maximum Likelihood Problem for exponential families as hypothesis classes using the backward map of the family.

\begin{lemma}
    \label{lem:minCrossEntropyExponential}
    Let $\gendistributionwith$ be a distribution, $\sstat$ a statistic, and $\basemeasure$ a boolean base measure.
    We build the mean parameter $\genmeanwith=\contractionof{\gendistributionwith,\sencsstatwith}{\selvariable}$ and have the following:
    \begin{itemize}
        \item[(1)] If $\genmeanwith \in \sbinteriorof{\meansetof{\sstat,\basemeasure}}$ then
        \begin{align*}
            \min_{\canparamwithin} \centropyof{\gendistribution}{\probtensorof{\sstat,\canparamwith,\basemeasure}}
            = \sentropyof{\probtensorof{\sstat,\backwardmapwrtof{\sstat,\basemeasure}{\genmeanwith},\basemeasure}} \, .
        \end{align*}
        \item[(2)] If $\genmeanwith \notin \sbinteriorof{\meansetof{\sstat,\basemeasure}}$ and $\genmeanwith\in\closureof{\meansetof{\sstat,\basemeasure}}$ then there is a sequence $\left(\meanparamofat{n}{\selvariable}\right)_{n\in\nn}\subset\hlnmeanset$ converging coordinatewise to $\genmeanwith$ and
        \begin{align*}
            \min_{\canparamwithin} \centropyof{\gendistribution}{\probtensorof{\sstat,\canparamwith,\basemeasure}}
            = \lim_{\meanparamofat{n}{\selvariable}\rightarrow\genmeanwith}
            \sentropyof{\probtensorof{\sstat,\backwardmapwrtof{\sstat,\basemeasure}{\meanparamofat{n}{\selvariable}},\basemeasure}} \, .
        \end{align*}
        \item[(3)] If $\genmeanwith\notin\closureof{\meansetof{\sstat,\basemeasure}}$ then
        \begin{align*}
            \min_{\canparamwithin} \centropyof{\gendistribution}{\probtensorof{\sstat,\canparamwith,\basemeasure}}
            = \infty \, .
        \end{align*}
    \end{itemize}
\end{lemma}
\begin{proof}
    See Theorem~3.4 in \cite{wainwright_graphical_2008}.
\end{proof}

The case $\genmeanwith\notin\genmeanset$ corresponds with the case that $\gendistributionwith$ is not representable with $\basemeasure$.
In more generality, the cross entropy between two distributions is finite, if and only if the support of first in contained in the support of the second.
We will relate this property to logical entailment in \charef{cha:logicalRepresentation}. % ALSO IN logicReasoning?

\subsect{Mode Queries by Annealing}

%% ANNEALING
The mode of a distribution is related to the forward mapping of $\invtemp\cdot\canparam$ in the limit $\invtemp\rightarrow\infty$ of low temperatures.
To sketch this relation, we recall the variational formulation of mode queries by
\begin{align*}
    \convhullof{\sencsstatat{\indexedshortcatvariables,\selvariable} \wcols \shortcatindices\in\cansstatcatindicesargmax}
    = \genmeansetargmax \contraction{\meanparam,\canparam}  \, .
\end{align*}
Further, for any by the inverse temperature $\invtemp\neq0$ annealed canonical parameter $\canparamwith$ (see \secref{sec:simulatedAnnealing}) we have
\begin{align*}
    \genmeansetargmax  \contraction{\meanparam,\invtemp\cdot\canparam}+ \sentropyof{\meanrepprob}
    = \genmeansetargmax  \contraction{\meanparam,\canparam} + \frac{1}{\invtemp} \cdot \sentropyof{\meanrepprob} \, .
\end{align*}
In the annealing limit, that is for large $\invtemp$, the entropy term becomes negligible and the forward mapping tends to the convex hull
\begin{align*}
    \convhullof{\sencsstatat{\indexedshortcatvariables,\selvariable} \wcols \shortcatindices\in\cansstatcatindicesargmax}
    = \genmeansetargmax \contraction{\meanparam,\canparam}  \, .
\end{align*}
For a more detailed discussion of this relation, see Theorem~8.1 in \cite{wainwright_graphical_2008}.


\sect{Maximum Entropy Distributions}

We characterize in this section to any mean parameter the reproducing distribution with maximum entropy, and investigate the tensor network representation of them and their modes.
As we will show, the distributions with maximal entropy are in exponential families with base measure by the minimal face, which contains the mean parameter.

\subsect{Entropy Maximization Problem}\label{sec:maxEntProblem}

The entropy maximization problem with respect to matching expected statistics $\genmean\in\genmeanset$ is the optimization problem
\begin{align}
    \label{prob:maxEntropy}\tag{$\probtagtypeinst{\entropysymbol}{\sstat,\basemeasure,\genmean}$}% Notation clash with cross-entropy!
    \argmax_{\probtensor\in\bmrealprobof{\basemeasure}} \sentropyof{\probtensor} \stspace
    \contractionof{\probtensor,\sencsstat}{\selvariable} = \genmeanat{\selvariable}
\end{align}
where by $\bmrealprobof{\basemeasure}$ we denote all distributions, which are representable with respect to a base measure $\basemeasure$.

% To expectation queries, as an example!
The mean parameters are computed as collections of expectation queries to $\sstatcoordinateof{\selindex}$, which are answered against distributions in $\bmrealprobof{\basemeasure}$.
For any $\selindexin$ we have for the mean parameter $\meanparamwith$ reproduced by a distribution $\probwith$
\begin{align*}
    \meanparamat{\indexedselvariable}
    = \expectationof{\sstatcoordinateof{\selindex}}
    = \contraction{\sencsstatat{\shortcatvariables,\indexedselvariable},\probwith} \, .
\end{align*}

We first show, that any solution has a sufficient statistic $\sstat$ (see \defref{def:sufficientStatistics}) and is therefore in $\realizabledistsof{\sstat,\maxgraph,\basemeasure}$.

\begin{theorem}
    \label{the:maxEntWithSufficientStatistic}
    Any maximum entropy distribution with respect to a moment constraint on $\sstat$ and a base measure $\basemeasure$ has the sufficient statistic $\sstat$.
\end{theorem}
\begin{proof}
    Let $\probwith$ be a feasible distribution for \probref{prob:maxEntropy}, which does not have a sufficient statistic $\sstat$.
    Then we find $\shortcatindices,\tildeshortcatindices\in\facstates$ with $\shortcatindices\neq\tildeshortcatindices$, $\sstatat{\shortcatindices}=\sstatat{\tildeshortcatindices}$, $\basemeasureat{\indexedshortcatvariables}\neq0$, $\basemeasureat{\shortcatvariables=\tildeshortcatindices}$ and $\probat{\indexedshortcatvariables}\neq\probat{\shortcatvariables=\tildeshortcatindices}$.
    We then define a distribution $\secprobat{\shortcatvariables}$ coinciding with $\probwith$ except for the coordinates $\shortcatindices,\tildeshortcatindices$, where we set
    \begin{align*}
        \secprobat{\indexedshortcatvariables} = \secprobat{\shortcatvariables=\tildeshortcatindices} = \frac{\probat{\indexedshortcatvariables}+\probat{\shortcatvariables=\tildeshortcatindices}}{2}
    \end{align*}
    We have by strict convexity of $u\cdot\lnof{u}$ and Jensens inequality
    \begin{align*}
        \sentropyof{\secprobtensor} -  \sentropyof{\probwith}
        &= \probat{\indexedshortcatvariables} \cdot \lnof{\probat{\indexedshortcatvariables}}
        + \probat{\shortcatvariables=\tildeshortcatindices} \cdot \lnof{\probat{\shortcatvariables=\tildeshortcatindices}} \\
        & \quad  - \left(\probat{\indexedshortcatvariables}+\probat{\shortcatvariables=\tildeshortcatindices}\right) \cdot \lnof{\frac{\probat{\indexedshortcatvariables}+\probat{\shortcatvariables=\tildeshortcatindices}}{2}} \\
        & > 0 \, .
    \end{align*}
    Since $\secprobat{\shortcatvariables}$ is also a feasible distribution with larger entropy than $\probwith$, $\probwith$ is not a maximum entropy distribution.
\end{proof}

We in the following provide a more detailed characterization.

\subsect{Characterization based on the Mean Polytope}

% Feasibility
By definition of the mean polytope, \probref{prob:maxEntropy} has a feasible distribution if and only if $\genmeanat{\selvariable}\in\genmeanset$.
If this condition holds, we now characterize the solution of \probref{prob:maxEntropy}.
First of all, we show that the maximum entropy distribution is in the exponential family $\genexpfamily$, when $\meanparam$ does not lie on a proper face of $\genmeanset$.
We then drop this assumption and generalize the statement to exponential families with refined base measures.

\begin{theorem} % TODO: Use the effective interior partition of the mean polytope
    \label{the:maxEntropyInterior}
    If the only face $\genfacesetof{\facecondset}$ of $\genmeanset$ with $\meanparam\in\genfacesetof{\facecondset}$ is $\genmeanset$ itself, then the unique solution of the maximum entropy problem \probref{prob:maxEntropy} is the distribution
    \begin{align*}
        \expdistofat{\sstat,\meanparam,\basemeasure}{\shortcatvariables}\in\expfamilyof{\sstat,\basemeasure}
    \end{align*}
    with $\contractionof{\expdistofat{\sstat,\meanparam,\basemeasure}{\shortcatvariables},\sencsstatwith}{\selvariable}=\meanparamat{\selvariable}$.
\end{theorem}
\begin{proof}
    By \theref{the:meanPolytopeInterior}, since by assumption
    \begin{align*}
        \meanparamwith \in \genmeanset / \bigcup_{\genfaceset\neq\genmeanset} \genfaceset \, ,
    \end{align*}
    there is a canonical parameter $\canparam$ with
    \begin{align*}
        \contractionof{\expdistofat{\sstat,\canparam,\basemeasure}{\shortcatvariables},\sencsstatwith}{\selvariable}=\meanparamat{\selvariable} \, .
    \end{align*}
    For any other feasible distribution $\secprobat{\shortcatvariables}$ we also have $\contractionof{\secprobat{\shortcatvariables},\sencsstatwith}{\selvariable}=\meanparamat{\selvariable}$ and thus
    \begin{align*}
        \centropyof{\secprobtensor}{\expdistof{(\sstat,\canparam,\basemeasure)}}
        &= -\contraction{\secprobtensor,\lnof{\expdistofat{(\sstat,\canparam,\basemeasure)}{\shortcatvariables}}} \\
        &= -\contraction{\secprobtensor,\contractionof{\sencsstatwith,\canparamwith}{\shortcatvariables}} + \cumfunctionof{\canparam} \\
        &= - \contraction{\canparam,\meanparam} + \cumfunctionof{\canparam} \\
        &= \sentropyof{\expdistof{(\sstat,\canparam,\basemeasure)}} \, .
    \end{align*}
    With the Gibbs inequality we have if $\secprobtensor\neq\expdistof{(\sstat,\canparam,\basemeasure)}$
    \begin{align*}
        \sentropyof{\expdistof{(\sstat,\estcanparam,\basemeasure)}} - \sentropyof{\secprobtensor}
        = \centropyof{\secprobtensor}{\expdistof{(\sstat,\estcanparam,\basemeasure)}} - \sentropyof{\secprobtensor} > 0 \, .
    \end{align*}
    Therefore, if $\secprobtensor$ does not coincide with$\expdistof{(\sstat,\estcanparam,\basemeasure)}$, it is not a solution of \probref{prob:maxEntropy}.
\end{proof}

% Interpretation
Let us highlight the fact, that in \probref{prob:maxEntropy} we did not restrict to distributions in an exponential family and only demanded representability with respect to the base measure.
When choosing the trivial base measure, this does not pose a restriction on the distributions.
\theref{the:maxEntropyInterior} states, that when the maximum entropy problem has a solution (i.e. $\genmean\in\genmeanset$), then the solution is in the exponential family to the statistic $\sstat$.

% Generalization
When $\genmean\notin\sbinteriorof{\genmeanset}$, the mean paramater is by \theref{the:meanPolytopeInteriorCharacterization} not reproducible by a member of the exponential family $\expfamilyof{\sstat,\basemeasure}$.
We show that the solution is in this case in a refined exponential family.

\begin{theorem}
    \label{the:maxEntropyFace}
    Let $\genfacesetof{\facecondset}$ be the minimal face of $\genmeanset$ such that $\meanparamat{\selvariable}\in\genfacesetof{\facecondset}$.
    Then, the solution of the maximum entropy problem is the distribution
    \begin{align*}
        \expdistof{\sstat,\meanparam,\secbasemeasure}
    \end{align*}
    where the base measure $\secbasemeasure$ is the refinement of $\basemeasure$ by the face measure $\genfacemeasure$, that is
    \begin{align*}
        \secbasemeasureat{\shortcatvariables} = \contractionof{\basemeasureat{\shortcatvariables},\genfacemeasureat{\shortcatvariables}}{\shortcatvariables} \, .
    \end{align*}
\end{theorem}
\begin{proof}
    By \theref{the:facePolytopeCharacterization} all feasible distributions for the maximum entropy problem have to be representable by the face measure $\genfacemeasure$.
    Since the feasible distributions are further restricted to those representable by $\basemeasure$, they are also representable by the refined base measure $\secbasemeasure$.
    Now, by \theref{the:faceAsRefinedPolytope} the face itself is a polytope $\meansetof{\sstat,\secbasemeasure}$ and the smallest face containing $\meanparam$ is the polytope itself.
    We thus arrive at the claim by applying \theref{the:maxEntropyInterior} on the polytope $\meansetof{\sstat,\secbasemeasure}$.
\end{proof}

% Minimality of the refined base measure
\theref{the:maxEntropyFace} implies, that the by the face measure refined base measure $\secbasemeasure$ is minimal for the maximum entropy problem, in the sense that the solving distribution is positive with respect to it and all feasible distributions have to be representable by it.
This highlights the fact, that the maximum entropy distribution does not vanish beyond those states, which are necessary to vanish to lie on the respective face. %by \theref{the:baseMeasureRefinement}.


\subsect{Tensor Network Representation}

Let us now state a criterion based on face measures for maximum entropy distributions to be in the set of by $\sstat$ and $\graph$ computable distributions $\realizabledistsof{\sstat,\graph,\basemeasure}$.

\begin{theorem}
    \label{the:tnRepresentationMaxEntropy}
    Given $\sstat$ and $\meanparamwith\in\genmeanset$, let $\facecondset$ be the smallest face of $\genmeanset$ such that
    \begin{align*}
        \meanparamat{\selvariable} \in \genfaceset \, .
    \end{align*}
    Then the corresponding maximum entropy distribution is in $\realizabledistsof{\sstat,\graph,\basemeasure}$ if and only if the face measure (see \defref{def:faceMeasure})
    \begin{align*}
        \kcoreofat{\facecondset}{\headvariables}
        = \sum_{\meanparam\in\genfacesetof{\facecondset}\cap\imageof{\sstatencoding}} \onehotmapofat{\meanparam}{\headvariables}
    \end{align*}
    is in $\tnsetof{\graph}$.
\end{theorem}
\begin{proof}
    By \theref{the:maxEntropyFace} the maximum entropy distribution is an element of the exponential family with by the face measure refined base measure $\secbasemeasure$.
    Let $\canparamwith$ be a canonical parameter such that
    \begin{align*}
        \contractionof{\sencsstatwith,\probofat{\sstat,\canparam,\secbasemeasure}{\shortcatvariables}}{\selvariable} = \meanparamwith \, ,
    \end{align*}
    that is $\probofat{\sstat,\canparam,\secbasemeasure}{\shortcatvariables}$ is the maximum entropy distribution.
    We apply \theref{the:faceMeasureCharacterization} to represent the face measure by
    \begin{align*}
        \basemeasureofat{\sstat,\facecondset}{\shortcatvariables} =
        \contractionof{\bencodingofat{\sstat}{\headvariables,\shortcatvariables},\kcoreofat{\facecondset}{\headvariables}}{\shortcatvariables}
    \end{align*}
    Then for the tensor
    \begin{align*}
        \hypercoreat{\headvariables}
        = \contractionof{
            \{\softactlegwith\wcols\selindexin\}
            \cup\{\kcoreofat{\facecondset}{\headvariables}\}}{\headvariables}
    \end{align*}
    we have
    \begin{align*}
        \probofat{\sstat,\canparam,\secbasemeasure}{\shortcatvariables}
        = \normalizationof{
            \bencodingofat{\sstat}{\headvariables,\shortcatvariables}, \hypercoreat{\headvariables}, \basemeasurewith
        }{\shortcatvariables} \, .
    \end{align*}
    Thus, the maximum entropy distribution is in $\realizabledistsof{\sstat,\graph,\basemeasure}$, if $\hypercore$ admits a tensor network decomposition with respect to $\graph$.
    Since the hard activation cores are elementary, this is the case when $\kcoreof{\facecondset}$ admits a tensor network decomposition with respect to $\graph$.
\end{proof}

In the special case of a mean parameter in the interior, \theref{the:tnRepresentationMaxEntropy} implies that the maximum entropy distribution is in $\realizabledistsof{\sstat,\elgraph,\basemeasure}$, as we show next.

\begin{theorem}
    For any statistic $\sstat$, base measure $\basemeasure$ and a mean parameter $\meanparamwith\in\sbinteriorof{\genmeanset}$ in the interior of the mean parameter polytope, the corresponding maximum entropy distribution is in $\realizabledistsof{\sstat,\elgraph,\basemeasure}$.
\end{theorem}
\begin{proof}
    If $\meanparamwith\in\interiorof{\genmeanset}$ then the only face of $\genmeanset$ containing $\meanparamwith$ is $\genmeanset$ itself, that is the face with $\facecondset=\varnothing$.
    In this case we have $\kcoreofat{\varnothing}{\headvariables}=\onesat{\headvariables}$, which is an elementary tensor.
    The claim then follows from \theref{the:tnRepresentationMaxEntropy}, when choosing $\graph=\elgraph$.
\end{proof}

% Representation
In general, we find a $\cpformat$ decomposition to the face measure as sketched in \figref{fig:maxEntropyActcore}.

\begin{example}[Representation by the maximal graph]
    Let us consider the maximal graph $\maxgraph=([\seldim],\{[\seldim]\})$, which has a single hyperedge containing all head variables.
    Since any tensor admits a tensor network decomposition with respect to $\maxgraph$, by \theref{the:tnRepresentationMaxEntropy} all maximum entropy distributions are in $\realizabledistsof{\sstat,\maxgraph}$.
    This reproduces the claim of \theref{the:maxEntWithSufficientStatistic}, that maximum entropy distributions with respect to constraints $\meanparamat{\selvariable}=\contractionof{\probtensor,\sencsstat}{\selvariable}$ therefore always have the sufficient statistic $\sstat$.
\end{example}

\begin{figure}[t]
    \begin{center}
        \begin{tikzpicture}[scale=0.4,thick,xscale=1] % , baseline = -3.5pt

    \begin{scope}[shift={(-15,0)}]
        \draw (-1,-1) rectangle (5,-3);
        \node[anchor=center] (text) at (2,-2) {\small $\probtensor$};
        \draw[->-] (0,-3)--(0,-5) node[midway,left] {\tiny $\catvariableof{0}$};
        \draw[->-] (1.5,-3)--(1.5,-5) node[midway,left] {\tiny $\catvariableof{1}$};
        \node[anchor=center] (text) at (3,-4) {$\cdots$};
        \draw[->-] (4,-3)--(4,-5) node[midway,right] {\tiny $\catvariableof{\atomorder\shortminus1}$};

        \node[anchor=center] (text) at (8,-2) {$= \,\frac{1}{\partitionfunction} \cdot $};
    \end{scope}

    %% Condition cores: Boolean cores selecting faces
    \draw[\concolor] (4,3) to[bend right=20] (2,5);
    \draw[\concolor] (0,3) to[bend left=20] (2,5);
    \draw[fill,\concolor] (2,5) circle (\dotsize);

    \draw[\concolor] (-1,1) rectangle (1,3);
    \node[anchor=center,\concolor] (text) at (0,2) {$\conactcoreof{{0}}$};

    \draw[\concolor] (3,1) rectangle (5,3);
    \node[anchor=center,\concolor] (text) at (4,2) {$\conactcoreof{{\seccatorder\shortminus1}}$};

    \draw[->-] (0,-1)--(0,0);
    \node[right] (text) at (0,0) {\tiny $\headvariableof{0}$};
    \draw[\concolor] (0,0)--(0,1);
    \drawvariabledot{0}{0}
    \node[anchor=center] (text) at (2,0) {$\cdots$};

    \draw[\probcolor] (0,0) -- (-2,0);
    \draw[\probcolor] (-2,1) rectangle (-4,-1);
    \node[anchor=center,\probcolor] (text) at (-3,0) {$\actcoreof{0}$};

    \draw[->-] (4,-1)--(4,0);
    \node[left] (text) at (4,0) {\tiny $\headvariableof{\seccatorder\shortminus1}$};
    \draw[\concolor] (4,0)--(4,1);
    \drawvariabledot{4}{0}

    \draw[\probcolor] (4,0) -- (6,0);
    \draw[\probcolor] (6,1) rectangle (8,-1);
    \node[anchor=center,\probcolor] (text) at (7,0) {$\actcoreof{\seccatorder\shortminus1}$};

    \draw (-1,-1) rectangle (5,-3);
    \node[anchor=center] (text) at (2,-2) {\small $\rencodingof{\sstat}$};
    \draw[-<-] (0,-3)--(0,-5) node[midway,left] {\tiny $\catvariableof{0}$};
    \draw[-<-] (1.5,-3)--(1.5,-5) node[midway,left] {\tiny $\catvariableof{1}$};
    \node[anchor=center] (text) at (3,-4) {$\cdots$};
    \draw[-<-] (4,-3)--(4,-5) node[midway,right] {\tiny $\catvariableof{\atomorder\shortminus1}$};

    \draw (-1,-1) rectangle (5,-3);
    \node[anchor=center] (text) at (2,-2) {\small $\rencodingof{\sstat}$};
    \draw[-<-] (0,-3)--(0,-5) node[midway,left] {\tiny $\catvariableof{0}$};
    \draw[-<-] (1.5,-3)--(1.5,-5) node[midway,left] {\tiny $\catvariableof{1}$};
    \node[anchor=center] (text) at (3,-4) {$\cdots$};
    \draw[-<-] (4,-3)--(4,-5) node[midway,right] {\tiny $\catvariableof{\atomorder\shortminus1}$};

%    \node[anchor=center] (text) at (6,-4.5) {$.$};

%\drawatomcore{3.5}{-8}{$\probtensor$}
%\drawatomindices{3.5}{-12}	
%\draw (5.5,-9)--(5.5,-7) node[midway,right] {\tiny $\catvariableof{\exformula}$};

\end{tikzpicture}
    \end{center}
    \caption{
        Tensor network decomposition of maximum entropy distributions to the constraint $\meanparamat{\selvariable}=\contractionof{\probtensor,\sencsstat}{\selvariable}$.
        Blue: Constraint activation cores $\hardactsymbolof{\selindex}$ in a $\cpformat$ decomposition, representing the face measure to the minimal face, such that $\meanparam\in\genfacesetof{\facecondset}$.
        Red: Probabilistic activation cores $\softactlegwith$ in an elementary decomposition, where each leg core is a scaled exponentials evaluated on the enumerated image $\imageof{\sstatcoordinateof{\selindex}}$.
    }\label{fig:maxEntropyActcore}
\end{figure}



\sect{Modes of Exponential Distributions}

\begin{figure}[t]
    \begin{center}
        \begin{tikzpicture}[scale=0.25]
    % Define points

    \node[below] at (1,7) {${\genmeanset}$};

    \coordinate (A) at (0,0);

    %   \node[below] at (A) {$\meanparamof{1}$};
    %   \draw[fill] (A) circle (\dotsize);

    \coordinate (B) at (12,2.5);
    \path (A) -- (B) coordinate[pos=0.7] (P1);

    %   \node[below] at (P1) {$\meanparamof{2}$};
    %   \draw[fill] (P1) circle (\dotsize);

    \coordinate (P2) at (2,10);
%    \node[below] at (P2) {$\meanparamof{3}$};
%    \draw[fill] (P2) circle (\dotsize);

    \coordinate (C) at (7.5,12);
    \path (B) -- (C) coordinate[pos=0.5] (P4);

    \node[right] at (P4) {$\genfacesetof{\canparam}$};
    \coordinate (D) at (-3,12);
    \coordinate (E) at (-10,5);

    %\node[below] at ($0.5*(A)+0.5*(E)-(0,1.3)$) {$\genmeanset/\interiorof{\genmeanset}$};

    \draw[thick, dashed] (A) -- (B) -- (C) -- (D) -- (E) -- cycle;

    \node[left] at (-12,11) {$\rr^\seldim$};

    \coordinate (Or) at (-9,11);
    \node[below] at (Or) {\corelabelsize $\zerosat{\selvariable}$};
    \drawvariabledot{-9}{11}

    % Face normal
    \draw[thick] (Or) -- ($(Or) + -0.6*(-9.5,-4.5)$) node[midway,below] {\corelabelsize $\canparamwith$};
    \draw[->,dashed] (Or) -- ($(Or) + -1.2*(-9.5,-4.5)$);
    \draw[dashed] (Or) -- ($(Or) + -1.45*(-9.5,-4.5)$);
    \draw[dashed] (B) -- ($(B)!1.57!(C)$);
    \draw[thick] (B) -- (C);

%    \coordinate (int) at ($(Or) + -1.45*(-9.5,-4.5)$);
    % angle
    \draw[thick] ($(Or) + -1.22*(-9.5,-4.5)$) arc[start angle=-154.7, end angle=-64.4, radius=2.5cm];
    \draw[fill] ($(Or) + -1.22*(-9.5,-4.5) + (1.75,0)$) circle (0.08cm);

    \coordinate (M1) at ($(Or) + 0.2*(C) + -0.2*(B)$);
    \coordinate (M2) at ($(Or) + -1.45*(-9.5,-4.5) + 0.2*(C) + -0.2*(B)$);


    \draw[<->] (M1) --(M2) node [midway, left] {\corelabelsize $\frac{1}{\|\canparam\|} \cdot \max_{\meanparam\in\genmeanset}\contraction{\meanparam,\canparam}$};

\end{tikzpicture}


    \end{center}
    \caption{Sketch of a face $\genfacesetof{\canparam}$ to the normal $\canparamwith$.
    The face is characterized by those mean parameters in $\genmeanset$, which maximize the contraction with $\canparamwith$.
    The sketched distance of the origin $\zerosat{\canparam}$ of $\parspace$ to the affine hull of the face is further related to the maximal contraction.
    }\label{fig:meansetSketchFace}
\end{figure}

We now show that modes of the members of the exponential family coincide with face measures.
To this end we introduce face normals and show that the mode of an exponential distribution is the face measure of the unique face, which has the face normal by its canonical parameter.

\subsect{Face Normals}

Let us now investigate, that faces of $\genmeanset$ are the solutions of mode queries, which are linear optimization problems constrained by $\genmeanset$ (see \figref{fig:meansetSketchFace}).
Since such solution sets are intersections of the boundary of $\genmeanset$ with half-spaces, they are faces.
In the next theorem we show, how linear optimization problems are constructed to match a given face.

\begin{lemma}
    For any non-empty face $\genfaceset$ to a subset $\facecondset\subset[n]$ there is a vector $\canparamwith$, which we call a normal of the face, such that
    \begin{align*}
        \genfaceset = \cangenmeansetargmax  \, .
    \end{align*}
    For any collection of positive $\lambda_i$, where $i\in\facecondset$, the vector
    \begin{align*}
        \canparamwith = \sum_{i\in\facecondset} \lambda_i\cdot\normalvecofat{i}{\selvariable}
    \end{align*}
    is a normal for $\genfacesetof{\facecondset}$.
\end{lemma}
\begin{proof}
    The first claim follows trivially from the second.
    To show the second claim, let there be for $i\in\facecondset$ arbitrary positive scalars $\lambda_i$.
    Since the face is non-empty, there is a $\meanparamwith$ with
    \begin{align*}
        \contraction{\meanparamwith,\normalvecofat{i}{\selvariable}}=\normalboundof{i}
    \end{align*}
    for all $i\in\facecondset$.
    Since any $\meanparam\in\genmeanset$ obey
    \begin{align*}
        \contraction{\meanparamwith,\normalvecofat{i}{\selvariable}}\leq \normalboundof{i}
    \end{align*}
    it follows that
    \begin{align*}
        \max_{\meanparam\in\genmeanset} \contraction{\canparamwith,\meanparamwith}
        = \sum_{i\in\facecondset} \lambda_i \cdot \normalboundof{i} \, .
    \end{align*}
    The maximum is attained at a $\meanparamwith$, if and only if the equations $\contraction{\meanparamwith,\normalvecofat{i}{\selvariable}}=\normalboundof{i}$ are satisfied for $i\in\facecondset$.
    This is equal to $\meanparamwith\in\genfacesetof{\facecondset}$.
\end{proof}

As we show next, also a converse statement holds, namely that for any vector $\canparamwith$ we find a face $\genfacesetof{\facecondset}$, such that the $\canparamwith$ is a face normal to that face.

\begin{lemma}
    \label{lem:faceToCanparam}
    For any $\canparamwith$ we find a face $\genfaceset$ such that
    \begin{align*}
        \cangenmeansetargmax = \genfaceset \, .
    \end{align*}
    We denote this face by $\genfacesetof{\canparam} := \genfacesetof{\facecondset}$.
\end{lemma}
\begin{proof} % Not a smooth proof, improve!
    We first notice, that
    \begin{align*}
        & \cangenmeansetargmax \\
        & \quad  = \convhullof{\sencsstatat{\indexedshortcatvariables,\selvariable} \wcols \shortcatindices\in\argmax_{\shortcatindicesin} \contraction{\canparamwith,\sstatat{\shortcatindices}}} \, .
    \end{align*}
    Further, since the contraction with $\canparamwith$ is linear, the set $\cangenmeansetargmax$ is contained in the boundary of the polytope $\genmeanset$.
    We can conclude, that the set is a face, that is we find a subset $\mathcal{I}\subset[n]$ with
    \begin{align*}
        \genfacesetof{\facecondset} = \cangenmeansetargmax \, . & \qedhere
    \end{align*}
\end{proof}

% Notation
%In a slight abuse of notation, we denote the face in \theref{the:modeQueryFaceBM} in this case $\genfacesetof{\canparam} = \genfacesetof{\facecondset}$.

Based on \lemref{lem:faceToCanparam} we can now characterize the solution of mode queries.
To this end, we recall the statistic encoding from \secref{sec:meanPolytopeConvexHull}, which is defined as
\begin{align*}
    \sstatencoding \defcols \stateset \rightarrow \parspace \quad,\quad
    \sstatencodingat{\catindex} = \sencsstatat{\indexedshortcatvariables,\selvariable} \, .
\end{align*}

\begin{theorem}
    \label{the:modeQueryFaceBM}
    For any $\canparamwith$ we have
    \begin{align*}
        \argmax_{\shortcatindicesin} \contraction{\canparamwith,\sencsstatat{\shortcatindices,\selvariable}} =
        \invsstatencodingat{\genfacesetof{\canparam}} \, .
    \end{align*}
    Any $\shortcatindicesin$ is therefore a mode of $\expdistof{\sstat,\canparam,\basemeasure}$, if and only if the face measure of $\genfacesetof{\canparam}$ is supported on $\shortcatindices$.
\end{theorem}
\begin{proof}
    We have
    \begin{align*}
        \max_{\shortcatindicesin} \contraction{\canparamwith,\sencsstatat{\shortcatindices,\selvariable}} =
        \max_{\meanparamwith\in\genmeanset} \contraction{\canparamwith,\meanparamwith}
    \end{align*}
    since $\genmeanset$ is the convex hull of the vectors $\sencsstatat{\indexedshortcatvariables,\selvariable}$.
    By \lemref{lem:faceToCanparam} the maximum is taken at the face $\genfacesetof{\canparam}$.
    We further have, since any face contains vertices, that $\sstatencodingof{\stateset}\cap\genfacesetof{\canparam}\neq\varnothing$.
    The solutions of the mode query are thus the states $\shortcatindices$, which statistics encoding is in the face $\genfaceset$.
    This set is the pre-image $\invsstatencodingat{\genfacesetof{\canparam}}$.

    From \defref{def:faceMeasure} we have that $\shortcatindices\in\invsstatencodingat{\genfacesetof{\canparam}}$ if and only if the face measure of $\genfacesetof{\canparam}$ is supported on $\shortcatindices$.
\end{proof}

% Mode queries
\theref{the:modeQueryFaceBM} provides a geometric perspective for mode queries.
An arbitrary positive tensor $\hypercorewith$ can be understood to be a canonical parameter of the exponential family with statistic $\naivestat$, and base measure $\onesat{\shortcatvariables}$.
For this exponential family, $\bmrealprobof{\ones}$ coincides with the polytope of mean parameters, and is a standard simplex.
Continuing our discussion in \secref{sec:modeQueries}, we have for any $\arbset\subset\facstates$ and $\basemeasure$ being the subset encoding of $\arbset$ that
\begin{align*}
    \max_{\shortcatindices\in\arbset} \hypercoreat{\indexedshortcatvariables} =
    \max_{\meanparamat{\shortcatvariables}\in\bmrealprobof{\basemeasure}} \contraction{\hypercoreat{\shortcatvariables},\meanparamat{\shortcatvariables}} \, .
\end{align*}
The maximum is attained exactly for the mean parameters
\begin{align*}
    \meanparamat{\shortcatvariables}\in
    \convhullof{\onehotmapofat{\shortcatindices}{\shortcatvariables} \wcols \shortcatindices\in\argmax_{\shortcatindices\in\arbset} \hypercoreat{\indexedshortcatvariables}} \, .
\end{align*}
Answering the mode query is thus the characterization of the face of the standard simplex with face normal $\hypercorewith$.

\begin{example}[Mode queries on tensor networks]
% Decompositions of \hypercore
    Let us now consider cases where the queried tensor $\hypercore$ has a tensor network decomposition.
    This is the case for energy tensors to members of exponential families, for which we have a decomposition into selection encodings $\sencsstatwith$ and canonical parameters $\canparamwith$.
    In most generality we assume a decomposition of $\hypercore$ by a tensor network on a hypergraph $\graph=(\nodes,\edges)$, where $[\catorder]\subset\nodes$ as
    \begin{align*}
        \hypercorewith = \contractionof{\extnetasset}{\shortcatvariables} \, .
    \end{align*}
    Let us choose a subset $\secedges\subset\edges$ and
    \begin{align*}
        \max_{\shortcatindices\in\arbset} \hypercoreat{\indexedshortcatvariables}
        = \max_{\shortcatindices\in\arbset} \contraction{\{\onehotmapofat{\shortcatindices}{\shortcatvariables}\} \cup \extnetasset}
%        = \max_{\shortcatindices\in\arbset} \contraction{\onehotmapofat{\shortcatindices}{\shortcatvariables},\hypercoreat{\indexedshortcatvariables}}
    \end{align*}
    We now split the contractions (see \theref{the:splittingContractions}) to contract the cores $\edges/\secedges$ with the one-hot encoding first and keeping $\secnodes=\bigcup_{\edge\in\secedges}\edge$ open.
    With this we get
    \begin{align*}
        & \max_{\shortcatindices\in\arbset} \hypercoreat{\indexedshortcatvariables} \\
        & = \max_{\shortcatindices\in\arbset}
        \contraction{
            \contractionof{\{\onehotmapofat{\shortcatindices}{\shortcatvariables}\} \cup \left\{\hypercoreofat{\edge}{\catvariableof{\edge}}\wcols \edge\in\edges/\secedges\right\}}{\catvariableof{\secnodes}},
            \contractionof{\left\{\hypercoreofat{\edge}{\catvariableof{\edge}}\wcols \edge\in\secedges\right\}}{\catvariableof{\secnodes}}
        } \, .
    \end{align*}
    This optimization problem is the characterization of vectors, which convex hull is the face in the polytope
    \begin{align*}
        \meanset = \left\{\{\contractionof{\onehotmapofat{\shortcatindices}{\shortcatvariables}\} \cup \left\{\hypercoreofat{\edge}{\catvariableof{\edge}}\wcols \edge\in\edges/\secedges\right\}}{\catvariableof{\secnodes}} \wcols \shortcatindices\in\arbset \right\}
    \end{align*}
    with the face normal
    \begin{align*}
        \canparamat{\catvariableof{\secnodes}} =
        \contractionof{\left\{\hypercoreofat{\edge}{\catvariableof{\edge}}\wcols \edge\in\secedges\right\}}{\catvariableof{\secnodes}} \, .
    \end{align*}
\end{example}


\subsect{Cones of Constant Modes}

We have observed, that the modes of maximum entropy distributions are characterized by the corresponding face measures.
Let us now investigate in more detail the sets of maximum entropy distributions, which coincide in their modes.

\begin{definition}
    To each face $\genfaceset$ the normal cone is the set %is the set of canonical parameters, which modes coincide with the face
    \begin{align*}
        \maxcone = \{\canparamwith\wcols\canparamwith\in\parspace\ncond\argmax_{\meanparam\in\genmeanset}\contraction{\canparamwith,\meanparamwith}=\genfaceset\} \, .
    \end{align*}
\end{definition}

Each $\maxcone$ is a convex cone, as we show now.

\begin{lemma}
    For each face $\facecondset$, $\maxcone$ is a convex cone.
\end{lemma}
\begin{proof}
    % Cone
    $\maxcone$ is a cone, since for any $\canparam\in\maxcone$ and $\lambda>0$ we have
    \begin{align*}
        \genfaceset = \argmax_{\meanparam\in\genmeanset} \contraction{\canparam,\meanparam} = \argmax_{\meanparam\in\genmeanset} \contraction{\lambda\cdot\canparam,\meanparam} \, .
    \end{align*}
    % Convex
    $\maxcone$ is convex, since for $\canparam,\seccanparam\in\maxcone$ and $\lambda\in(0,1)$ we have
    \begin{align*}
        \genfaceset = \argmax_{\meanparam\in\genmeanset} \contraction{\canparam,\meanparam} = \argmax_{\meanparam\in\genmeanset} \contraction{\seccanparam,\meanparam} \, .
    \end{align*}
    It follows
    \begin{align*}
        \genfaceset = \argmax_{\meanparam\in\genmeanset} \contraction{\lambda\cdot\canparam+(1-\lambda)\cdot\seccanparam,\meanparam}
    \end{align*}
    and $\lambda\cdot\canparam+(1-\lambda)\cdot\seccanparam \in\maxcone$.
\end{proof}

The faces $\genfaceset$ of the mean parameter polytope build a partially ordered set, which is called the face lattice (see \cite{ziegler_lectures_2013}).
We now investigate a corresponding partial order on the max cones to the faces.

\begin{lemma}
    \label{lem:faceSetsToMaxCones}
    For two non-empty faces $\facecondsetof{0}$ and $\facecondsetof{1}$ we have % Non-empty needed? Can deal with face 1 empty, but with face 0 empty?
    \begin{align*}
        \genfacesetof{\facecondsetof{0}} \subset \genfacesetof{\facecondsetof{1}}
    \end{align*}
    if and only if
    \begin{align*}
        \genmaxconeof{\facecondsetof{1}} \subset \closureof{\genmaxconeof{\facecondsetof{0}}} \, .
    \end{align*}
\end{lemma}
\begin{proof}
    \proofrightsymbol Let us assume $\genfacesetof{\facecondsetof{0}} \subset \genfacesetof{\facecondsetof{1}}$ and let us choose arbitrary $\canparamof{0}\in\genmaxconeof{\facecondsetof{0}}\ncond\canparamof{1}\in\genmaxconeof{\facecondsetof{1}}$.
    It suffices to show, that for all $\lambda\in(0,1)$ we have $\lambda\cdot\canparamof{0}+(1-\lambda)\cdot\canparamof{1}\in\genmaxconeof{\facecondsetof{0}}$, since this implies $\canparamof{1}\in\closureof{\genmaxconeof{\facecondsetof{0}}}$.
    By assumption we have
    \begin{align*}
        \argmax_{\meanparam\in\genmeanset} \contraction{\canparamof{0},\meanparam} \subset  \argmax_{\meanparam\in\genmeanset} \contraction{\canparamof{1},\meanparam}
    \end{align*}
    and thus
    \begin{align*}
        \argmax_{\meanparam\in\genmeanset} \contraction{\lambda\cdot\canparamof{0}+(1-\lambda)\cdot\canparamof{1},\meanparam} = \argmax_{\meanparam\in\genmeanset} \contraction{\canparamof{0},\meanparam}
    \end{align*}
    which implies $\lambda\cdot\canparamof{0}+(1-\lambda)\cdot\canparamof{1}\in\genmaxconeof{\facecondsetof{0}}$.
    \proofleftsymbol Conversely, let us assume $\genmaxconeof{\facecondsetof{1}} \subset \closureof{\genmaxconeof{\facecondsetof{0}}}$.
    We then find a sequence $\left(\canparamof{n}\right)_{n\in\nn}$ with $\canparamof{n}\in\genmaxconeof{\facecondsetof{0}}$ for $n\in\nn$ and which limit $\canparam$ exists in $\genmaxconeof{\facecondsetof{1}}$.
    For an arbitrary $\catindex\in\stateset$ with $\sstatencodingat{\catindex}\in\genfacesetof{\facecondsetof{0}}$ we have
    \begin{align*}
        \contraction{\canparam,\sstatencodingat{\catindex}}
        &= \lim_{n\rightarrow\infty} \contraction{\canparamof{n},\sstatencodingat{\catindex}} \\
        &= \lim_{n\rightarrow\infty} \max_{\seccatindex\in\stateset} \contraction{\canparamof{n},\sstatencodingat{\seccatindex}} \\
        &= \max_{\seccatindex\in\stateset} \contraction{\canparam,\sstatencodingat{\seccatindex}}
    \end{align*}
    and therefore $\sstatencodingat{\catindex}\in\genfacesetof{\facecondsetof{1}}$.
    Note, that we used in the third equation, that $\stateset$ is finite.% and thus avoided to justify these equations based on convexity. -> More precise statement needed !
    We use this property for all elements of the preimage of $\facecondsetof{0}$ and get
    \begin{align*}
        \genfacesetof{\facecondsetof{0}}
        &= \convhullof{\sstatencodingat{\catindex}\wcols\catindex\in\argmax_{\seccatindex\in\stateset}\contraction{\canparamof{1},\sstatencodingat{\seccatindex}}} \\
        &\subset \convhullof{\sstatencodingat{\catindex}\wcols\catindex\in\argmax_{\seccatindex\in\stateset}\contraction{\canparam,\sstatencodingat{\seccatindex}}} \\
        &= \genfacesetof{\facecondsetof{1}} \, . \qedhere
    \end{align*}
\end{proof}

\lemref{lem:faceSetsToMaxCones} suggests that the partial order of faces by inclusion is mimicked by another partial order on the max cones, sketched in \figref{fig:max_cone_sketch}.
We now consider the set of cones with this partial order, and show that this set is homomorphic to the face lattice.

\begin{theorem}
    \label{the:faceSetsToMaxCones}
    The max cone lattice of $\genmeanset$, partially order by the
    \begin{align*}
        \genmaxconeof{\facecondsetof{0}} \prec \genmaxconeof{\facecondsetof{1}} \quad \text{if and only if} \quad
        \genmaxconeof{\facecondsetof{1}} \subset \closureof{\genmaxconeof{\facecondsetof{0}}}
    \end{align*}
    is homomorphic to the face lattice of $\genmeanset$ with the homomorphism
    \begin{align*}
        \psi(\genfaceset) = \genmaxcone \, .
    \end{align*}
\end{theorem}
\begin{proof}
    % Special case of empty faces
    We have to show that for all pairs of faces $\genfacesetof{\facecondsetof{0}},\genfacesetof{\facecondsetof{1}}$
    \begin{align*}
        \psi(\genfacesetof{\facecondsetof{0}}) \prec \psi(\genfacesetof{\facecondsetof{1}}) \quad \Leftrightarrow \quad \genfacesetof{\facecondsetof{0}} \prec \genfacesetof{\facecondsetof{1}} \, .
    \end{align*}
    We show this first for the case, that $\genfacesetof{\facecondsetof{0}}=\varnothing$ or $\genfacesetof{\facecondsetof{1}}=\varnothing$.
    Note, that the empty face is contained in any other face, but contains no non-empty face.
    Conversely, the trivial max cone $\genmaxconeof{\facecondsetof{\varnothing}}$ is for minimal statistics $\{\zerosat{\selvariable}\}$ and contained in the boundary of any other non-empty cone.
    If the statistic is not minimal, the inclusion holds for the equivalence class of canonical parameters.
    Further, since the max cones are a disjoint partition of $\parspace$, $\zerosat{\selvariable}$ is not in any other max cone and thus $\genmaxconeof{\facecondset} \prec \genmaxconeof{\facecondsetof{\varnothing}}$ does not hold for any non-empty $\facecondset$.

    % Non-empty faces
    For all pairs of non-empty faces $\genfacesetof{\facecondsetof{0}},\genfacesetof{\facecondsetof{1}}$, \lemref{lem:faceSetsToMaxCones} ensures that
    \begin{align*}
        \psi(\genfacesetof{\facecondsetof{0}}) \prec \psi(\genfacesetof{\facecondsetof{1}}) \quad \Leftrightarrow \quad \genfacesetof{\facecondsetof{0}} \prec \genfacesetof{\facecondsetof{1}} \, . & \qedhere
    \end{align*}
\end{proof}

As a consequence of \theref{the:faceSetsToMaxCones}, the max cone lattice inherits all properties of the face lattice, for example those shown in Theorem~2.6 in \cite{ziegler_lectures_2013}.

\begin{figure}
    \begin{center}
        \begin{tikzpicture}[scale=0.315]

    %% Meanset
    %\node[anchor=center] at (11.5,7) {$\meansetof{\sstat,\trivbm}$};
%


%    \coordinate (DE) at (0.707, −0.707);
%    \coordinate (EA) at (0.447, 0.894);

    \coordinate (A) at (0,0);
    \coordinate (B) at (12,2.5);
    \coordinate (C) at (7.5,12);
    \coordinate (D) at (-3,12);
    \coordinate (E) at (-10,5);
    \draw[thick] (A) -- (B) -- (C) -- (D) -- (E) -- cycle;

    \draw[fill] (A) circle (0.15cm);
    \draw[fill] (B) circle (0.15cm);
    \draw[fill] (C) circle (0.15cm);
    \draw[fill] (D) circle (0.15cm);
    \draw[fill] (E) circle (0.15cm);

    \node[below] at (A) {$\facecondsetof{40}$};
    \node[right] at (B) {$\facecondsetof{01}$};
    \node[above] at (C) {$\facecondsetof{12}$};
    \node[above] at (D) {$\facecondsetof{23}$};
    \node[left] at (E) {$\facecondsetof{34}$};

    \draw[thick] (A) -- (B) node[midway, below] {$\facecondsetof{0}$};
    \draw[thick] (B) -- (C) node[midway, right] {$\facecondsetof{1}$};
    \draw[thick] (C) -- (D) node[midway, above] {$\facecondsetof{2}$};
    \draw[thick] (D) -- (E) node[midway, left] {$\facecondsetof{3}$};
    \draw[thick] (E) -- (A) node[midway, below] {$\facecondsetof{4}$};


    % Base measure mean
    \coordinate (meanCenter) at (1.5,8);
    \draw[fill] (meanCenter) circle (0.15cm);
    \node[below] at ($(meanCenter)-(0,1)$) {$\meanparamof{\sstat,\zeros,\basemeasure}$};

    % Sketch of mapped maximum cones
    \draw[dashed] (meanCenter) to[bend left=20] ($0.5*(A)+0.5*(B)$);
    \draw[dashed] (meanCenter) to[bend left=10] ($0.5*(B)+0.5*(C)$);
    \draw[dashed] (meanCenter) to[bend right=5] ($0.5*(C)+0.5*(D)$);
    \draw[dashed] (meanCenter) to[bend left=20] ($0.5*(D)+0.5*(E)$);
    \draw[dashed] (meanCenter) to[bend right=10] ($0.5*(E)+0.5*(A)$);

    \coordinate (Or) at (-9,11);
    \draw[fill] (-9,11) circle (0.15cm);
    \node[below] at (Or) {\small $\zerosat{\selvariable}$};

    \coordinate (nAB) at (-0.204,0.9741);
    \coordinate (nBC) at (-0.9037,-0.4281);
    \coordinate (nCD) at (0,-1);
    \coordinate (nDE) at (0.707,-0.707);
    \coordinate (nEA) at (0.447,0.894);

    % Canonical parameter space
    \coordinate (canor) at (-24,6);
    \node[anchor=center] at ($(canor)+(2.5,-0.5)$) {$\zerosat{\selvariable}$};
    \draw[fill] (canor) circle (0.15cm);

    %\draw[dashed] (canor) -- (-6,14);
    \draw[dashed] (canor) -- ($(canor) + -9*(nAB)$);
    \draw[dashed] (canor) -- ($(canor) + -9*(nBC)$);
    \draw[dashed] (canor) -- ($(canor) + -9*(nCD)$);
    \draw[dashed] (canor) -- ($(canor) + -9*(nDE)$);
    \draw[dashed] (canor) -- ($(canor) + -9*(nEA)$);

    \node[anchor=center] at ($(canor) + -9.7*(nAB)$) {$\maxconeof{0}$};
    \node[anchor=center] at ($(canor) + -9.7*(nBC)$) {$\maxconeof{1}$};
    \node[anchor=center] at ($(canor) + -9.7*(nCD)$) {$\maxconeof{2}$};
    \node[anchor=center] at ($(canor) + -9.7*(nDE)$) {$\maxconeof{3}$};
    \node[anchor=center] at ($(canor) + -9.7*(nEA)$) {$\maxconeof{4}$};

    \node[anchor=center] at ($(canor) - 4*(nAB) - 4*(nBC)$) {$\maxconeof{01}$};
    \node[anchor=center] at ($(canor) - 4*(nBC) - 4*(nCD)$) {$\maxconeof{12}$};
    \node[anchor=center] at ($(canor) - 4*(nCD) - 4*(nDE)$) {$\maxconeof{23}$};
    \node[anchor=center] at ($(canor) - 4*(nDE) - 4*(nEA)$) {$\maxconeof{34}$};
    \node[anchor=center] at ($(canor) - 4*(nEA) - 4*(nAB)$) {$\maxconeof{40}$};

    % Forward backward map sketch
    \begin{scope}
        [shift={(2,5)}]
        \node[anchor=center] at (-19,13) {$\simpleparspace$};
        \node[anchor=center] at (0,13) {$\meansetof{\sstat,\basemeasure}\subset\simpleparspace$};
        \draw[->] (-17.5,13.5) to[bend left=20] (-3.5,13.5);
        \draw[->] (-3.5,12.5) to[bend left=20] (-17.5,12.5);
        \node[anchor=center] at (-10.5,16) {$\forwardmap$};
        \node[anchor=center] at (-10.5,10) {$\backwardmap$};
    \end{scope}

\end{tikzpicture}


    \end{center}
    \caption{
        Partition of the canonical parameters in $\parspace$ into convex maximum cones $\maxcone$ to faces $\facecondset$ (left).
        The forward mapping maps the maximum cones into the mean parameter polytope (right).
    }\label{fig:max_cone_sketch}
\end{figure}


\sect{Mean Field Methods}\label{sec:meanField}

Mean field methods are approximation schemes for forward mappings, designed for efficient inference.
To introduce them we turn the maximization over the mean parameter polytope in the variational principle of forwards mappings into a maximization over the reproducing distributions, that is
\begin{align*}
    \max_{\meanparam\in\genmeanset}  \contraction{\meanparam,\canparam} + \sentropyof{\meanrepprob}
    =
    \max_{\probtensor\in\bmrealprobof{\basemeasure}} \contraction{\energytensor,\probtensor} + \sentropyof{\probtensor}
\end{align*}
where
\begin{align*}
    \energytensorwith = \contractionof{\sencsstat,\canparam}{\shortcatvariables} \, .
\end{align*}
Mean field methods now provide lower bounds on this maximization by restricting the distribution optimized over to a tractable subset of distributions.

Let $\probtensorset$ be a subset of $\bmrealprobof{\ones}$, we state the mean field problem as
%Let $\graph$ be any hypergraph, we define the problem
\begin{align}
    \label{prob:meanField}\tag{$\probtagtypeinst{I}{\probtensorset,\probof{\energytensor}}$}
    \argmax_{\probtensor\in\probtensorset} \contraction{\energytensor,\probtensor} + \sentropyof{\probtensor}
\end{align}
%% KL Divergence
We show next, that the mean field problem is an instance of an information projection, as indicated by the notation.

\begin{theorem}
    \label{the:meanFieldIProjection}
    Let there be an energy tensor $\energytensor$ and consider the distribution
    \begin{align*}
        \probofat{\energytensor}{\shortcatvariables}=\normalizationof{\expof{\energytensorwith}}{\shortcatvariables} \, .
    \end{align*}
    For any set $\probtensorset$ of distributions we have
    \begin{align*}
        \argmax_{\probtensorin} \contraction{\energytensor,\probtensor} + \sentropyof{\probtensor}
        = \argmin_{\probtensorin} \kldivof{\probtensor}{\probof{\energytensor}} \, .
    \end{align*}
    \probref{prob:meanField} is thus the information projection of a $\probofat{\energytensor}{\shortcatvariables}$ onto $\probtensorset$.
\end{theorem}
\begin{proof}
%	This follows from the fact, that the objective is the cross-entropy and the position of the maximum is invariant under substracting $\sentropyof{\probtensor}$.
    The cross entropy between a $\probtensorin$ and $\probof{\energytensor}$ is
    \begin{align*}
        \centropyof{\probtensor}{\probof{\energytensor}}
        &= \contraction{\probat{\shortcatvariables},-\lnof{\probofat{\energytensor}{\shortcatvariables}}} \\
        &= \contraction{\probat{\shortcatvariables},-\energytensorwith}
        + \contraction{\probat{\shortcatvariables}} \cdot \lnof{\contraction{\expof{\energytensorwith}}}
    \end{align*}
    Together we have, that
    \begin{align*}
        \kldivof{\probtensor}{\probof{\energytensor}}
        &= \centropyof{\probtensor}{\probof{\energytensor}} - \sentropyof{\probtensor} \\
        &= - \contraction{\probat{\shortcatvariables},\energytensorwith} - \sentropyof{\probtensor} + \lnof{\contraction{\expof{\energytensorwith}}}
    \end{align*}
    Since the last term is constant among $\probtensorin$, it holds that
    \begin{align*}
        \argmax_{\probtensorin} \contraction{\energytensor,\probtensor}+ \sentropyof{\probtensor}
        = \argmax_{\probtensorin} -\kldivof{\probtensor}{\probof{\energytensor}}
        = \argmax_{\probtensorin} \kldivof{\probtensor}{\probof{\energytensor}} \, . & \qedhere
    \end{align*}
\end{proof}


\subsect{Naive Mean Field Method}

%Typically we use the family of independent distributions, also called naive mean field method.
In the naive mean field method, we choose the approximating set $\probtensorset$ by the exponential family of Markov Networks $\mnexpfamilyof{\elgraph}$ on the elementary graph % confusing to do here exponential families: Cannot deal with non-trivial core support!
\begin{align*}
    \elgraph=\Big([\catorder],\{\big\{\catenumerator\}\wcols \catenumeratorin\big\}\Big) \, .
\end{align*}
Markov Networks on this graph are represented by normalized leg cores
\begin{align*}
    \probwith
    = \bigotimes_{\catenumeratorin} \legcoreofat{\catenumerator}{\catvariableof{\catenumerator}}
\end{align*}
\probref{prob:meanField} is for this instance of the form
%The approximation is then  $\legcoreof{\catenumerator}$, that is
\begin{align*}
    \argmax_{\legcoreofat{\catenumerator}{\catvariableof{\catenumerator}} \wcols \contraction{\legcoreof{\catenumerator}}=1\ncond \catenumeratorin}
    \contraction{\{\energytensor\} \cup \{\legcoreof{\catenumerator} \wcols \catenumeratorin\}}
    + \sum_{\catenumeratorin} \sentropyof{\legcoreof{\catenumerator}} \, .
\end{align*}
We approximately solve this problem in \algoref{alg:NMF} by alternation through the leg cores and performing a locally optimal leg core update.
The optimal update equations are derived as the next theorem.

\begin{theorem}[Update equations for the naive mean field approximation]
    %For the \probref{prob:meanField}
    For any $\catenumeratorin$ and leg cores $\{\legcoreofat{\seccatenumerator}{\catvariableof{\seccatenumerator}} \wcols \seccatenumeratorin\ncond\seccatenumerator\neq\catenumerator\}$ the local problem
    \begin{align*}
        \argmax_{\legcoreofat{\catenumerator}{\catvariableof{\catenumerator}} \wcols \contraction{\legcoreof{\catenumerator}}=1} \contraction{\{\energytensor\} \cup \{\legcoreofat{\seccatenumerator}{\catvariableof{\seccatenumerator}}\wcols \seccatenumeratorin\}}
        + \sum_{\catenumeratorin} \sentropyof{\legcoreof{\catenumerator}}
    \end{align*}
    is solved at
    \begin{align*}
        \legcoreofat{\catenumerator}{\catvariableof{\catenumerator}}
        = \normalizationof{ \expof{ \contractionof{ \{\energytensorwith \}\cup
        \{\legcoreofat{\seccatenumerator}{\catvariableof{\seccatenumerator}}\wcols \seccatenumerator\neq\catenumerator\} }{\shortcatvariables} }
        }{\catvariableof{\catenumerator}} \, .
    \end{align*}
\end{theorem}
\begin{proof}
    We have
    \begin{align*}
        \difofwrt{\sentropyof{\legcoreof{\catenumerator}}}{\legcoreof{\catenumerator}}
        =  - \lnof{\legcoreofat{\catenumerator}{\catvariableof{\catenumerator}}}
        + \onesat{\catvariableof{\catenumerator}}
    \end{align*}
    and by multilinearity of tensor contractions
    \begin{align*}
        \difofwrt{\contraction{\{\energytensor\}\cup\{\legcoreof{\seccatenumerator}\wcols \seccatenumeratorin \}}}{\legcoreof{\catenumerator}}
        =  \contractionof{\{\energytensor\}\cup\{\legcoreof{\seccatenumerator}\wcols \seccatenumeratorin \ncond \seccatenumerator\neq\catenumerator \}}{\catvariableof{\catenumerator}} \, .
    \end{align*}
    Combining both, the condition
    \begin{align*}
        0 = \difofwrt{
            \left( \contraction{\{\energytensor\}\cup\{\legcoreof{\seccatenumerator}\wcols \seccatenumeratorin \}} + \sum_{\catenumeratorin} \sentropyof{\legcoreof{\catenumerator}} \right)
        }{\legcoreof{\catenumerator}}
    \end{align*}
    is equal to
    \begin{align*}
        \lnof{\legcoreofat{\catenumerator}{\catvariableof{\catenumerator}}} =
        \onesat{\catvariableof{\catenumerator}} + \contractionof{\{\energytensor\}\cup\{\legcoreof{\seccatenumerator}\wcols \seccatenumeratorin \ncond \seccatenumerator\neq\catenumerator \}}{\catvariableof{\catenumerator}} \, .
    \end{align*}
    Together with the condition $\contraction{\legcoreof{\catenumerator}}=1$ this is satisfied at
    \begin{align*}
        \legcoreofat{\catenumerator}{\catvariableof{\catenumerator}}
        = \normalizationof{ \expof{ \contractionof{ \{\energytensor\}\cup
        \{\legcoreof{\seccatenumerator}\wcols \seccatenumerator\neq\catenumerator\} }{\catvariableof{\catenumerator}} }
        }{\catvariableof{\catenumerator}} \, . & \qedhere
    \end{align*}
\end{proof}

\algoref{alg:NMF} is the alternation of legwise updates until a stopping criterion is met.

\begin{algorithm}[h!]
    \caption{Naive Mean Field Approximation}\label{alg:NMF}
    \begin{algorithmic}
        \Require Energy tensor $\energytensorwith$
        \Ensure Tensor Network $\{\legcoreofat{\catenumerator}{\catvariableof{\catenumerator}}\wcols\catenumeratorin\}$ approximating $\normalizationof{\expof{\energytensorwith}}{\shortcatvariables}$
        \iosepline
        \ForAll{$\catenumeratorin$}
            \State
            \begin{align*}
                \legcoreofat{\catenumerator}{\catvariableof{\catenumerator}}
                \algdefsymbol \normalizationof{\ones}{\catvariableof{\catenumerator}}
            \end{align*}
        \EndFor
        \While{Stopping criterion is not met}
            \ForAll{$\catenumeratorin$}
                \State
                \[ \legcoreofat{\catenumerator}{\catvariableof{\catenumerator}}
                \algdefsymbol \normalizationof{ \expof{ \contractionof{\{\energytensorwith\}\cup
                \{\legcoreofat{\seccatenumerator}{\catvariableof{\seccatenumerator}} \wcols \seccatenumerator\neq\catenumerator\} }{\catvariableof{\catenumerator}} }
                }{\catvariableof{\catenumerator}} \]
            \EndFor
        \EndWhile
        \State \Return $\{\legcoreofat{\catenumerator}{\catvariableof{\catenumerator}}\wcols\catenumeratorin\}$
    \end{algorithmic}
\end{algorithm}


\subsect{Structured Variational Approximation}

%% Structured Variational approximation
%More generically, we restrict the maximum over the mean parameters of efficiently contractable distributions and get a lower bound.
%In this section we use any Markov Network as the approximating family.

We now generalize the naive mean field method towards generic families of Markov Networks.
Let $\graph$ be any hypergraph, the structured variational approximation method is \probref{prob:meanField} with
\begin{align*}
    \probtensorset = \mnexpfamily \, .
\end{align*}
%% IPROJECTION UNDEFINED!
%By \theref{the:meanFieldIProjection} the structured approximation is the information projection \probref{def:iProjection} onto the exponential family of Markov Networks given a hypergraph $\graph$.

We approximate the solution of this problem again by an alternating algorithm, which iteratively updates the cores of the approximating Markov Network.

\begin{theorem}[Update equations for the structured variational approximation]
    \label{the:updateEquationStructuredVariational}
    The Markov Network $\extnet$ with hypercores $\extnetasset$ is a stationary point for structured variational approximation, if for all $\edgein$ we find a $\lambda>0$ with
    \begin{align*}
        \hypercoreofat{\edge}{\edgevariables}
        = \lambda\cdot \expof{
            \frac{
                \contractionof{\{\energytensor\}\cup\{
                \hypercoreof{\secedge}\wcols\secedge\neq\edge
                \}}{\edgevariables}
            }{
                \contractionof{\{
                \hypercoreof{\secedge}\wcols\secedge\neq\edge
                \}}{\edgevariables}
            }
            - \sum_{\thirdedge\neq\edge}
            \frac{
                \contractionof{\{\lnof{\hypercoreof{\thirdedge}}\}\cup\{
                \hypercoreof{\secedge}\wcols\secedge\neq\thirdedge
                \}}{\edgevariables}
            }{
                \contractionof{\{
                \hypercoreof{\secedge}\wcols\secedge\neq\thirdedge
                \}}{\edgevariables}
            }
        } \, .
    \end{align*}
    Here, the quotient denotes the coordinatewise quotient.
\end{theorem}
\begin{proof}%[Proof of \theref{the:updateEquationStructuredVariational}]
    We proof the theorem by the first order condition on the objective
    \begin{align*}
        \objof{\extnet} = \contraction{\energytensor,\extnetdist} + \sentropyof{\extnetdist}
    \end{align*}
    We further use \lemref{lem:difMNExpectation}, which shows a characterization of the derivative of functions dependent on tensors.

    %% Energy Contraction Term
    We have %for $\probtensor\in\mnexpfamily$
    \begin{align*}
        \contraction{\energytensor,\normalizationof{\extnet}{\shortcatvariables}}
        =  \frac{
            \contraction{\{\energytensor\}\cup\extnet}
        }{
            \contraction{\extnet}
        } \, .
    \end{align*}

    %% Entropy Term Decomposition
    Further we have
    \begin{align*}
        \sentropyof{\normalizationof{\extnet}{\shortcatvariables}}
        = \left(\sum_{\secedge\in\edges} \contraction{-\lnof{\hypercoreof{\secedge}},\normalizationof{\extnet}{\shortcatvariables}} \right)
        + \lnof{\contraction{\extnet}}
    \end{align*}

    We define the tensor
    \[ \sechypercore[\catvariableof{\nodes}] = \energytensorat{\catvariableof{\nodes}}
    - \sum_{\secedge\neq\edge} \lnof{\hypercoreofat{\secedge}{\catvariableof{\secedge}}} \otimes \onesat{\catvariableof{\nodes/\secedge}} \]
    and notice, that $\sechypercore$ does not depend on $\hypercoreof{\edge}$.

    The objective has then a representation as
    \begin{align*}
        \objof{\extnet} = \contraction{\sechypercore[\catvariableof{\nodes}], \extnetdist} - \contraction{ \lnof{\hypercoreof{\edge}}, \extnetdist} +  \lnof{\contraction{\extnet}}
    \end{align*}

    Let us now differentiate all terms.
    With \lemref{lem:difMNExpectation} we now get
    \begin{align*}
        \difwrt{\hypercoreofat{\edge}{\seccatvariableof{\edge}}} \contraction{\sechypercore[\catvariableof{\nodes}], \extnetdist}
        & = \contractionof{\sechypercoreat{\nodevariables},
            \identityat{\seccatvariableof{\edge},\edgevariables},
            \frac{\contractionof{\extnet}{\edgevariables}}{\hypercoreofat{\edge}{\edgevariables}},
            \normalizationofwrt{\extnet}{\catvariableof{\nodes/\edge}}{\edgevariables} }{\seccatvariableof{\edge},\nodevariables} \\
        & \quad -  \contraction{\sechypercoreat{\nodevariables},\extnetdist}
        \otimes \contractionof{\frac{\contractionof{\extnet}{\seccatvariableof{\edge}}}{\hypercoreofat{\edge}{\seccatvariableof{\edge}}}
        }{\seccatvariableof{\edge}} \, .
    \end{align*}

    Further we have
    \begin{align*}
        \difwrt{\hypercoreofat{\edge}{\seccatvariableof{\edge}}} \contraction{ \lnof{\hypercoreof{\edge}}, \extnetdist}
        & = \contractionof{\lnof{\hypercoreofat{\edge}{\edgevariables}},
            \identityat{\seccatvariableof{\edge},\edgevariables},
            \frac{\contractionof{\extnet}{\edgevariables}}{\hypercoreofat{\edge}{\edgevariables}},
            \normalizationofwrt{\extnet}{\catvariableof{\nodes/\edge}}{\edgevariables} }{\seccatvariableof{\edge},\nodevariables} \\
        & \quad -  \contraction{\lnof{\hypercoreofat{\edge}{\edgevariables}},\extnetdist}
        \otimes \contractionof{\frac{\contractionof{\extnet}{\seccatvariableof{\edge}}}{\hypercoreofat{\edge}{\seccatvariableof{\edge}}}
        }{\seccatvariableof{\edge}} \\
        & \quad\quad - \contraction{ \frac{1}{\hypercoreofat{\edge}{\edgevariables}}, \extnetdist}
    \end{align*}
    and (see Proof of \ref{lem:difMNprob})
    \begin{align*}
        \difwrt{\hypercoreofat{\edge}{\seccatvariableof{\edge}}} \lnof{\contraction{\extnet}}
        = \frac{\difwrt{\hypercoreofat{\edge}{\seccatvariableof{\edge}}} \contraction{\extnet}}{\contraction{\extnet}}
        = \frac{\contractionof{\extnet}{\seccatvariableof{\edge}}}{\hypercoreofat{\edge}{\seccatvariableof{\edge}}} \, .
    \end{align*}

    Together, the first order condition
    \begin{align*}
        0 = \difwrt{\hypercoreofat{\edge}{\seccatvariableof{\edge}}} \objof{\extnet}
    \end{align*}
    is equal to all $\seccatindexof{\edge}$ satisfying% here drop seccatvariable to catvariable by slicing
    \begin{align*}
        0 & = \frac{\contractionof{\extnet}{\indexedseccatvariableof{\edge}}}{\hypercoreofat{\edge}{\indexedseccatvariableof{\edge}}}
        \Big(
        \contraction{\sechypercoreat{\catvariableof{\nodes/\edge},\catvariableof{\edge}=\seccatindexof{\edge}}, \normalizationofwrt{\extnet}{\catvariableof{\nodes/\edge}}{\catvariableof{\edge}=\seccatindexof{\edge}}} \\
        &\quad \quad - \contraction{\sechypercoreat{\nodevariables}, \extnetdist}  \\
        &\quad \quad - \contraction{\lnof{\hypercoreofat{\edge}{\edgevariables=\seccatindexof{\edge}}}, \normalizationofwrt{\extnet}{\catvariableof{\nodes/\edge}}{\catvariableof{\edge}=\seccatindexof{\edge}}} \\
        &\quad \quad + \contraction{\lnof{\hypercoreofat{\edge}{\edgevariables}}, \extnetdist}
        \Big) \, .
    \end{align*}

    We notice, that by normalization
    \[ \contraction{\lnof{\hypercoreofat{\edge}{\edgevariables=\seccatindexof{\edge}}}, \normalizationofwrt{\extnet}{\catvariableof{\nodes/\edge}}{\catvariableof{\edge}=\seccatindexof{\edge}}} =  \lnof{\hypercoreofat{\edge}{\edgevariables=\seccatindexof{\edge}}} \]
    and that the scalar
    \[ \lambda_1 = \contraction{\sechypercoreat{\nodevariables},\normalizationof{\extnet}{\catvariableof{\nodes}}}
    - \contraction{\lnof{\hypercoreofat{\edge}{\edgevariables}},\normalizationof{\extnet}{\catvariableof{\nodes}}}    \]
    is the constant for all $\seccatindexof{\edge}$.

    The first order condition is therefore equal to the existence of a $\lambda_1\in\rr$ such that for all $\seccatindexof{\edge}$
    \begin{align*}
        \lnof{\hypercoreofat{\edge}{\catvariableof{\edge}=\seccatindexof{\edge}}}
        =    \contraction{\sechypercoreat{\catvariableof{\nodes/\edge},\catvariableof{\edge}=\seccatindexof{\edge}},
            \normalizationofwrt{\extnet}{\catvariableof{\nodes/\edge}}{\catvariableof{\edge}=\seccatindexof{\edge}}} + \lambda_1 \, .
    \end{align*}
    The claim follows when applying the exponential on both sides and with the observation, that
    \begin{align*}
        \contraction{\sechypercoreat{\catvariableof{\nodes/\edge},\catvariableof{\edge}=\seccatindexof{\edge}},
            \normalizationofwrt{\extnet}{\catvariableof{\nodes/\edge}}{\catvariableof{\edge}=\seccatindexof{\edge}}}
        =
        \frac{\contractionof{\{\sechypercore\}\cup\{\hypercoreof{\secedge}\wcols \secedge\neq \edge\}}{\catvariableof{\edge}=\seccatindexof{\edge}} }{
            \contractionof{\{\hypercoreof{\secedge}\wcols \secedge\neq \edge\}}{\catvariableof{\edge}=\seccatindexof{\edge}}
        }
    \end{align*}
    and reparametrization of $\lambda_1$ to
    \begin{align*}
        \lambda = \expof{\lambda_1} \, . & \qedhere
    \end{align*}
\end{proof}

%% MISSING: ALGORITHM?

\sect{Backward Map in Exponential Families}\label{sec:backwardMap}

Let us now continue with the discussion of the backward map, which calculates to a mean parameter $\meanparamwith$ a canonical parameter $\canparamwith$, such that the corresponding member of the exponential family reproduced $\meanparamwith$.

\subsect{Variational Formulation}

We now provide a variational characterization of the backward map.

\begin{theorem}
    \label{the:varBackward}
    Let there be a statistic $\sstat$, which is minimal with respect to a boolean base measure $\basemeasure$.
    The map $\backwardmap: \sbinteriorof{\genmeanset}\rightarrow\parspace$ defined as
    \begin{align*}
        \backwardmapof{\meanparam}
        = \argmax_{\canparam\in\parspace}  \contraction{\meanparam,\canparam} - \cumfunctionof{\canparam} \, .
    \end{align*}
    is a backward mapping.
\end{theorem}
\begin{proof}
    We show the claim can be shown by the first order condition on the objective.
    It holds that
    \begin{align*}
        \difwrt{\canparamat{\selvariable}}  \cumfunctionof{\canparam}
        & = \difwrt{\canparamat{\selvariable}}  \lnof{\contraction{\expof{\contractionof{\sencsstat,\canparam}{\shortcatvariables}}}} \\
        & = \difwrt{\canparamat{\selvariable}} \frac{\contraction{\sencsstat[\selvariable],\expof{\contractionof{\sencsstat,\canparam}{\shortcatvariables}}}}{\contraction{\expof{\contractionof{\sencsstat,\canparam}{\shortcatvariables}}}}   \\
        & = \forwardmapof{\canparam}[\selvariable]
    \end{align*}
    and thus
    \begin{align*}
        \difwrt{\canparamat{\selvariable}} \left( \contraction{\meanparam,\canparam} - \cumfunctionof{\canparam}  \right)
        = \meanparamwith -  \forwardmapof{\canparam}[\selvariable] \, .
    \end{align*}

    The first order condition is therefore
    \[ \meanparamwith =  \forwardmapof{\canparam}[\selvariable] \]
    and any $\canparam$ satisfies this condition exactly when $\canparam=\backwardmapof{\meanparam}$ for a backward map.
    We conclude the proof by noticing, that since $\meanparamwith$ is in the interior $\sbinteriorof{\genmeanset}$ we find by \theref{the:meanPolytopeInterior} a $\canparam$ such that the first order condition is met.
\end{proof}

%We have that $\canparam$ is a solution of the backward problem at $\genmean$, if and only if
%\[ \contractionof{\expdist,\sencsstat}{\selvariable} = \genmeanat{\selvariable} \, . \]
%This contraction equation is called moment matching, since the moment of the empirical distribution is matched by the moment of the fitting distribution.

\subsect{Interpretation as a Moment Projection}

% Backward mapping
Backward maps coincide with the Maximum Likelihood Estimation Problem \eqref{prob:parameterMaxLikelihood}, when we take the hypothesis to be exponential family $\expfamily$. % for a sufficient statistic $\sstat$.
We show this as a more general statement for moment projections onto exponential families.
%We recall, that the Maximum likelihood estimation is the M-projection of the empirical distribution onto the exponential family.
% Cross entropy
%The loss is the cross entropy between a distribution with $\meanparam$ and the distribution $\expdistof{(\sstat,\canparam,\basemeasure)}$.

\begin{theorem}
    Let there be any exponential family, a mean parameter vector $\genmean\in\imageof{\forwardmap}$ and a backward map $\backwardmap$.
    Then $\estcanparam=\backwardmapof{\genmean}$ is the canonical parameter to the solution of the moment projection \probref{prob:mProjection} of any $\gendistribution$ with
    \begin{align*}
        \contractionof{\sencsstat,\gendistribution}{\selvariable}=\genmeanat{\selvariable}
    \end{align*}
    onto the exponential family, if
    \[ \expdistof{(\sstat,\estcanparam,\basemeasure)} \in \argmax_{\probtensor\in\expfamily} \centropyof{\gendistribution}{\probtensor}  \, . \]

\end{theorem}
\begin{proof}
    We exploit the variational characterization of the backward map by \theref{the:varBackward}, and first show that the objective coincides with the cross entropy between the distribution $\gendistribution$ and the respective member of the exponential family.
    For any $\gendistribution$ and $\canparam$ we have with Example~\ref{exa:cEntropyExp}
    \begin{align*}
        \centropyof{\gendistribution}{\expdistof{(\sstat,\canparam,\basemeasure)}}
        = \contraction{\gendistribution,\sencsstat,\canparam} -\cumfunctionof{\canparam} \, .
    \end{align*}
    We use that by assumption $\contractionof{\gendistribution,\sencsstat}{\selvariable}=\genmeanat{\selvariable}$ and thus
    \begin{align*}
        \centropyof{\gendistribution}{\expdistof{(\sstat,\canparam,\basemeasure)}}
        =   \contraction{\genmean,\canparam} -\cumfunctionof{\canparam} \, .
    \end{align*}
    This shows, that the backward map coincides with the moment projection onto $\probtensorset=\expfamily$.
%    Further, if $\meanparam=\datamean$ for a data map $\datamap$, we have that the corresponding empirical distribution $\empdistribution$ satisfies $\contractionof{\sencsstat,\empdistribution}{\selvariable}=\meanparamwith$.
%    The backward map of $\meanparam$ is therefore the M-projection of $\empdistribution$, which is with \lemref{lem:centropyMLE} the maximum likelihood estimator.
\end{proof}

% Maximum likelihood estimation
In particular, if we choose $\meanparamwith=\contractionof{\sencsstatwith,\empdistribution}{\selvariable}$ for an empirical distribution $\empdistribution$, the backward map is a maximum likelihood estimator.
This holds, since by \lemref{lem:centropyMLE} maximum likelihood estimation is a special instance of moment projection, when projecting $\empdistribution$.
In that case, we choose $\meanparamwith=\contractionof{\sencsstatwith,\empdistribution}{\selvariable}$.


%\begin{lemma}
%	Let $\sstat\in\facspace\otimes\parspace$ be a sufficient statistic and $\gendistribution\in\facspace$ a probability distribution.
%	For any member $\expdist\in\expfamily$ we have
%		\[ \centropyof{\gendistribution}{\expdist} = \contraction{\canparam,\genmean} - \cumfunctionof{\canparam} \]
%	where 
%		\[ \genmean = \contractionof{\gendistribution,\sencsstat}{\selvariableof{\sstat}} \,  \]
%	and 
%		\[ \cumfunctionof{\canparam} = \lnof{\contraction{\expof{\expenergy}}} \, . \]
%	The M-projection of $\gendistribution$ onto $\expfamily$ is  $\expdistof{(\sstat,\estcanparam,\basemeasure)}$ for
%		\[ \estcanparam\in \argmax_{\canparam}  \contraction{\canparam,\genmean} - \cumfunctionof{\canparam} \, .  \]
%\end{lemma}
%\begin{proof}
%	By decomposing 
%	\begin{align*}
%		\expdist 	& = \normalizationof{\expof{\contractionof{\sencsstat,\canparam}{\shortcatvariables}}}{\shortcatvariables} \\
%				& = \frac{\expof{\expenergy}}{\contraction{\expof{\expenergy}}}
%	\end{align*}
%	we get
%	\begin{align*}
%		\lnof{\expdist} & = \lnof{\expof{\expenergy}} - \onesat{\shortcatvariables} \cdot \contraction{\expof{\expenergy}} \\
%		& = \expenergy - \cumfunction(\canparam) \cdot \onesat{\shortcatvariables}  \, .
%	\end{align*}
%	If follows that
%	\begin{align*}
%		\centropyof{\gendistribution}{\expdist} 
%		&=  \contraction{\gendistribution,\lnof{\expdist}} \\
%		&=  \contraction{\gendistribution,\expenergy} - \cumfunction(\canparam) \cdot \contraction{\gendistribution}   \\
%		&= \contraction{\canparam, \genmean} - \cumfunction(\canparam) \, .
%	\end{align*}
%\end{proof}


%\begin{theorem}\label{the:maxEntMaxLikeDuality} % In Koller Book, Theorem 20.2
%	If $\genmean\in\imageof{\forwardmap}$, we have that any distribution solving Problem~\ref{prob:maxEntropy} has a representation by $\expdistof{(\sstat,\estcanparam,\basemeasure)}$, 
%	where $\estcanparam=\backwardmapof{\genmean}$ for any backward map of the exponential family. 
%	%where $\estcanparam$ is the Maximum Likelihood Estimate with respect to any $\probtensor$ with $\contractionof{\secprobtensor,\sencsstat}{\selvariable} =\genmean$.
%%
%%	Let $\sstat$ be a map and $\gendistribution$ be any distribution of $\atomstates$ and define
%%		\[ \genmeanat{\selvariable} = \contractionof{\gendistribution,\sencsstat}{\selvariable} \, .  \]
%%	Then the solution of \ref{prob:maxEntropy} coincides with the member $\expdistof{(\sstat,\estcanparam,\basemeasure)}$ of the exponential family $\expfamily$ where
%%		\[ \estcanparam = \backwardmapof{\genmean} \]
%%	for a backward map $\backwardmap$ of $\expfamily$.
%\end{theorem}
%\begin{proof}
%	Since $\genmean\in\imageof{\forwardmap}$, there is a parameter $\estcanparam$ such that 
%		\[ \genmeanat{\selvariable} = \contractionof{\expdistof{(\sstat,\estcanparam,\basemeasure)},\sencsstat}{\selvariable}   \, . \]
%	Let $\secprobtensor$ further be an arbitrary distribution such that
%		\[ \genmeanat{\selvariable} = \contractionof{\secprobtensor,\sencsstat}{\selvariable}  \, . \]
%	We then have
%	\begin{align*}
%		\sentropyof{\expdistof{(\sstat,\estcanparam,\basemeasure)}}
%		= \centropyof{\secprobtensor}{\expdistof{(\sstat,\estcanparam,\basemeasure)}}
%	\end{align*}
%	
%	With the Gibbs inequality we have if $\secprobtensor\neq\expdistof{(\sstat,\estcanparam,\basemeasure)}$
%	\begin{align*}
%		\sentropyof{\expdistof{(\sstat,\estcanparam,\basemeasure)}} - \sentropyof{\secprobtensor}
%		= \centropyof{\secprobtensor}{\expdistof{(\sstat,\estcanparam,\basemeasure)}} - \sentropyof{\secprobtensor} > 0 \, . 
%	\end{align*}	
%	
%	Therefore, if $\secprobtensor$ does not coincide with$\expdistof{(\sstat,\estcanparam,\basemeasure)}$, it is not a solution of Problem~\ref{prob:maxEntropy}.
%	%Classical result based on duality of maximum entropy and maximum likelihood, shown e.g. in Koller Book.
%\end{proof}




\subsect{Approximation by Alternating Algorithms}\label{sec:alternatingBackwardMap}

While the forward map always has a representation in closed form by contraction of the probability tensor, the backward map in general fails to have a closed form representation.
Computation of the Backward map can instead be performed by alternating algorithms, as we show here. % Are these fixpoint iterations?
We alternate through the coordinates of the statistics and adjust $\canparamat{\indexedselvariable}$ to a minimum of the likelihood, i.e. where for any $\selindexin$
\begin{align*}
    0 = \frac{\partial}{\partial \canparamat{\indexedselvariable}} \lossof{\expdist} \, .
\end{align*}

% Moment matching
This condition is equal to the collection of moment matching equations % (see \theref{the:mm})
\begin{align*}
    \contractionof{\expdist,\sencsstat}{\indexedselvariable} = \contractionof{\empdistribution,\sencsstat}{\indexedselvariable} \, .
\end{align*}


\begin{lemma}
    \label{lem:mmContractionEquation}
    For any sufficient statistic $\sstat$ a parameter vector $\canparam$ and an index $\selindexin$ we define
    \begin{align*}
        \hypercoreat{\headvariableof{\selindex}}
        = \contractionof{\{\sstatccwith\}\cup\{\softactsymbolofat{\tilde{\selindex},\canparam}{\headvariableof{\tilde{\selindex}}} \wcols \tilde{\selindex} \in [\seldim] \ncond \tilde{\selindex}\neq\selindex\} \cup \{\basemeasurewith\}}{\headvariableof{\selindex}} \, .
    \end{align*}
    Then the moment matching condition for $\sstatcoordinateof{\selindex}$ relative to $\canparam$ and $\meanparam$ is satisfied for any $\canparamat{\indexedselvariable}$ with
    \begin{align*}
        \contraction{\softactlegwith, \indinttensorofat{\selindex}{\headvariableof{\selindex}}, \hypercoreat{\headvariableof{\selindex}}}
        = \contraction{\softactlegwith, \hypercoreat{\headvariableof{\selindex}}} \cdot \meanparamat{\indexedselvariable} \, .
    \end{align*}
\end{lemma}
\begin{proof}
    We have
    \begin{align*}
        \expdist = \frac{
            \contractionof{\softactlegwith,\hypercoreat{\headvariableof{\selindex}}}{\shortcatvariables}
        }{
            \contraction{\softactlegwith,\hypercoreat{\headvariableof{\selindex}}}
        }
    \end{align*}
    and
    \begin{align*}
        \contraction{\expdist,\sstatcoordinateof{\selindex}}
        = \frac{
            \contractionof{\softactlegwith,\indinttensorofat{\selindex}{\headvariableof{\selindex}}, \hypercoreat{\headvariableof{\selindex}}}{\shortcatvariables}
        }{
            \contraction{\softactlegwith, \hypercoreat{\headvariableof{\selindex}}}
        } \, .
    \end{align*}
    Here we used (see \corref{cor:rhoToNormal})
    \begin{align*}
        \sstatcoordinateof{\selindex}
        = \contractionof{\bencodingofat{\sstatcoordinateof{\selindex}}{\headvariableof{\selindex},\shortcatvariables},\indinttensorofat{\selindex}{\headvariableof{\selindex}}}{\shortcatvariables}
    \end{align*}
    %\[ \sstatcoordinateof{\selindex} = \contractionof{\softactlegwith, \indinttensorofat{\selindex}{\headvariableof{\selindex}}}{\shortcatvariables} \]
    and redundancies of copies of basis encodings.
    It follows that
    \begin{align*}
        \contraction{\expdist,\sstatcoordinateof{\selindex}}
        = \contraction{\empdistribution,\sstatcoordinateof{\selindex}}
    \end{align*}
    is equal to
    \begin{align*}
        \contraction{\softactlegwith, \indinttensorofat{\selindex}{\headvariableof{\selindex}}, \hypercoreat{\headvariableof{\selindex}}}
        = \contraction{\softactlegwith,\hypercoreat{\headvariableof{\selindex}}} \cdot \meanparamat{\indexedselvariable} \, . & \qedhere
    \end{align*}
\end{proof}

% Alternation necessary
The steps have to be alternated until sufficient convergence, since matching the moment to $\selindex$ by modifying $\canparamat{\indexedselvariable}$ will in general change other moments, which will have to be refit.

%Coordinate descent
An alternating optimization is the coordinate descent of the negative likelihood, seen as a function of the coordinates of $\canparam$, see Algorithm~\ref{alg:AMM}.
Since the log likelihood is concave, the algorithm converges to a global minimum.

\begin{algorithm}[h!]
    \caption{Alternating Moment Matching for the Backward Map}\label{alg:AMM}
    \begin{algorithmic}
        \Require Empirical distribution $\empdistribution$, statistic $\sstat$ and base measure $\basemeasure$
        \Ensure Canonical parameter $\canparamwith$, such that $\expdist$ is the (approximative) moment projection of $\empdistribution$ onto $\expfamily$
        \iosepline
        \State Set $\canparamat{\selvariable}=\zerosat{\selvariable}$
        \State Compute $\datameanat{\selvariable}= \contractionof{\empdistribution,\sencsstat}{\selvariable}$
        \While{Convergence criterion not met}
            \ForAll{$\selindexin$}
                \State Compute
                \begin{align*}
                    \hypercoreofat{\selindex}{\headvariableof{\selindex}}
                    \algdefsymbol \contractionof{\{\sstatcc\}\cup\{\softactsymbol^{\tilde{\selindex},\canparamat{\selvariable=\tilde{\selindex}}} \wcols \tilde{\selindex} \in [\seldim] \ncond \tilde{\selindex}\neq\selindex\}}{\headvariableof{\selindex}}
                \end{align*}
                \State Set $\canparamat{\indexedselvariable}$ to a solution of
                \begin{align*}
                    \contraction{\softactlegwith,\indinttensorofat{\selindex}{\headvariableof{\selindex}},\hypercoreof{\selindex}}
                    = \contraction{\softactlegwith,\hypercoreof{\selindex}} \cdot \datameanat{\indexedselvariable} \, .
                \end{align*}
            \EndFor
        \EndWhile
        \State \Return $\canparamat{\selvariable}$
    \end{algorithmic}
\end{algorithm}


% 
In general, if $\imageof{\sstatcoordinateof{\selindex}}$ contains more than two elements, there exists no closed form solutions.
We will investigate the case of binary images, where there are closed form expressions, later in \secref{sec:alternatingParEstMLN}.

%
The computation of $\hypercoreof{\selindex}$ in Algorithm~\ref{alg:AMM} can be intractable and be replaced by an approximative procedure based on message passing schemes.

\subsect{Second Order Methods}

The Hesse matrix of $\cumfunction$ at $\canparam$ is the covariance of the features with respect to $\expdist$, as we show next.

\begin{lemma}
    \label{lem:hesseCumfunction}
    At any $\canparam\in\parspace$ we have
    \begin{align*}
        \gradwrt{\seccanparamat{\secselvariable}}\gradwrtat{\seccanparamat{\selvariable}}{\canparam} \cumfunctionof{\seccanparam}
        =& \contractionof{\expdistwith,\sencsstatat{\shortcatvariables,\selvariable},\sencsstatat{\shortcatvariables,\secselvariable}}{\secselvariable,\selvariable} \\
        &-  \contractionof{\expdistwith,\sencsstatat{\shortcatvariables,\secselvariable}}{\secselvariable} \otimes \contractionof{\expdistwith,\sencsstatat{\shortcatvariables,\selvariable}}{\selvariable} \, .
    \end{align*}
\end{lemma}
\begin{proof}
    By \lemref{lem:gradientCumfunction} we have
    \begin{align*}
        \gradwrtat{\seccanparamat{\selvariable}}{\canparam} \cumfunctionof{\seccanparam}
        = \contractionof{\expdistwith,\sencsstatwith}{\selvariable} \, .
    \end{align*}
    It further holds
    \begin{align*}
        \gradwrtat{\seccanparamat{\secselvariable}}{\canparam} \expdistwith
        = & \contractionof{\expdistwith,\sencsstatat{\shortcatvariables,\secselvariable}}{\shortcatvariables,\secselvariable} \\
        & - \contractionof{\expdistwith,\sencsstatat{\shortcatvariables,\secselvariable}}{\secselvariable} \otimes \expdistwith \, .
    \end{align*}
    Combining both equations, we get the claim.
\end{proof}

With this characterization, we can perform second order optimization algorithms such as the Newton Method, see \algoref{alg:newtonBackward} to solve the backward map.

\begin{algorithm}
    \caption{Newton Method for the Backward Map}\label{alg:newtonBackward}
    \begin{algorithmic}
        \Require Empirical distribution $\empdistribution$, statistic $\sstat$ and base measure $\basemeasure$
        \Ensure Canonical parameter $\canparamwith$, such that $\expdist$ is the (approximative) moment projection of $\empdistribution$ onto $\expfamily$
        \iosepline
        \State Set $\canparamat{\selvariable}=\zerosat{\selvariable}$
        \State Compute $\datameanat{\selvariable}= \contractionof{\empdistribution,\sencsstat}{\selvariable}$
        \While{Convergence criterion not met}
            \State Calculate $\gradwrt{\seccanparamat{\secselvariable}}\gradwrtat{\seccanparamat{\selvariable}}{\canparam} \cumfunctionof{\seccanparam}$ and  $\gradwrtat{\seccanparamat{\selvariable}}{\canparam} \cumfunctionof{\seccanparam}$ as in \lemref{lem:hesseCumfunction}. %and \lemref{lem:cumulativeGradient}.
            \State Solve the linear equation
            \begin{align*}
                \left(\gradwrt{\seccanparamat{\secselvariable}}\gradwrtat{\seccanparamat{\selvariable}}{\canparam} \cumfunctionof{\seccanparam}\right)  \Delta\left[\selvariable\right]
                = \meanparamwith - \gradwrtat{\seccanparamat{\selvariable}}{\canparam} \cumfunctionof{\seccanparam}
            \end{align*}
            \State Update the canonical parameter
            \begin{align*}
                \canparamwith \algdefsymbol \canparamwith - \Delta\left[\selvariable\right]
            \end{align*}
        \EndWhile
        \State \Return $\canparamwith$
    \end{algorithmic}
\end{algorithm}


\sect{Discussion}

%% Forward mapping as gradient of A
%Further in \cite{wainwright_graphical_2008}: Convex Duality.
%Forward mapping coincides with gradient, i.e. $\meanparam = \nabla \cumfunction(\canparam)$.
%
%% Gradient property of the backward mapping
%In \cite{wainwright_graphical_2008}:
%The objective is the conjugate dual $\dualcumfunction$ of $\cumfunction$, and backward mapping has an expression by the gradient, i.e. $\canparam = \nabla \dualcumfunction(\meanparam)$.

% Convex Duality
The forward and backward map have a correspondence with each other in terms of convex duality \cite{rockafellar_convex_1997}.
As such, the cumulant function $\cumfunction$ (see \defref{def:expFamily}) and its conjugate dual $\dualcumfunction$ represent the forward and backward map by their gradient.

% Message Passing
Further approximation schemes arise for the forward and backward map in exponential families arise from loopy message passing algorithms.
For inference on Markov Network families, they are derived based on outer bounds on the mean polytope in terms of the local consistency polytope and approximations on the entropy term by local terms (see Chapter~3 in \cite{wainwright_graphical_2008}).
In particular, different approximations of the polytope and the entropies are made in the Bethe and Kickuchi method.
We will discuss message passing schemes in \charef{cha:messagePassing} with a focus on exact computation of generic contractions by local contractions.

